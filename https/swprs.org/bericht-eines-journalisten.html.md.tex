\protect\hyperlink{content}{Skip to content}

\href{https://swprs.org/}{}

\protect\hyperlink{search-container}{Search}

Search for:

\href{https://swprs.org/}{\includegraphics{https://swprs.files.wordpress.com/2020/05/swiss-policy-research-logo-300.png}}

\href{https://swprs.org/}{Swiss Policy Research}

Geopolitics and Media

Menu

\begin{itemize}
\tightlist
\item
  \href{https://swprs.org}{Start}
\item
  \href{https://swprs.org/srf-propaganda-analyse/}{Studien}

  \begin{itemize}
  \tightlist
  \item
    \href{https://swprs.org/srf-propaganda-analyse/}{SRF / ZDF}
  \item
    \href{https://swprs.org/die-nzz-studie/}{NZZ-Studie}
  \item
    \href{https://swprs.org/der-propaganda-multiplikator/}{Agenturen}
  \item
    \href{https://swprs.org/die-propaganda-matrix/}{Medienmatrix}
  \end{itemize}
\item
  \href{https://swprs.org/medien-navigator/}{Analysen}

  \begin{itemize}
  \tightlist
  \item
    \href{https://swprs.org/medien-navigator/}{Navigator}
  \item
    \href{https://swprs.org/der-propaganda-schluessel/}{Techniken}
  \item
    \href{https://swprs.org/propaganda-in-der-wikipedia/}{Wikipedia}
  \item
    \href{https://swprs.org/logik-imperialer-kriege/}{Kriege}
  \end{itemize}
\item
  \href{https://swprs.org/netzwerk-medien-schweiz/}{Netzwerke}

  \begin{itemize}
  \tightlist
  \item
    \href{https://swprs.org/netzwerk-medien-schweiz/}{Schweiz}
  \item
    \href{https://swprs.org/netzwerk-medien-deutschland/}{Deutschland}
  \item
    \href{https://swprs.org/medien-in-oesterreich/}{Österreich}
  \item
    \href{https://swprs.org/das-american-empire-und-seine-medien/}{USA}
  \end{itemize}
\item
  \href{https://swprs.org/bericht-eines-journalisten/}{Fokus I}

  \begin{itemize}
  \tightlist
  \item
    \href{https://swprs.org/bericht-eines-journalisten/}{Journalistenbericht}
  \item
    \href{https://swprs.org/russische-propaganda/}{Russische Propaganda}
  \item
    \href{https://swprs.org/die-israel-lobby-fakten-und-mythen/}{Die
    »Israel-Lobby«}
  \item
    \href{https://swprs.org/geopolitik-und-paedokriminalitaet/}{Pädokriminalität}
  \end{itemize}
\item
  \href{https://swprs.org/migration-und-medien/}{Fokus II}

  \begin{itemize}
  \tightlist
  \item
    \href{https://swprs.org/covid-19-hinweis-ii/}{Coronavirus}
  \item
    \href{https://swprs.org/die-integrity-initiative/}{Integrity
    Initiative}
  \item
    \href{https://swprs.org/migration-und-medien/}{Migration \& Medien}
  \item
    \href{https://swprs.org/der-fall-magnitsky/}{Magnitsky Act}
  \end{itemize}
\item
  \href{https://swprs.org/kontakt/}{Projekt}

  \begin{itemize}
  \tightlist
  \item
    \href{https://swprs.org/kontakt/}{Kontakt}
  \item
    \href{https://swprs.org/uebersicht/}{Seitenübersicht}
  \item
    \href{https://swprs.org/medienspiegel/}{Medienspiegel}
  \item
    \href{https://swprs.org/donationen/}{Donationen}
  \end{itemize}
\item
  \href{https://swprs.org/contact/}{English}
\end{itemize}

\protect\hyperlink{}{Open Search}

\hypertarget{erfahrungsbericht-eines-journalisten}{%
\section{Erfahrungsbericht eines
Journalisten}\label{erfahrungsbericht-eines-journalisten}}

Wie entsteht der Mainstream in den Medien? Woher kommt die Propaganda?
Im folgenden Beitrag spricht erstmals ein Schweizer Journalist über
seine langjährigen Erfahrungen.

\hypertarget{die-tunnelperspektive-der-mainstream-medien}{%
\subsubsection{Die Tunnelperspektive der
«Mainstream-Medien»:}\label{die-tunnelperspektive-der-mainstream-medien}}

Wie es drinnen wirklich ist

Gastbeitrag eines Schweizer Journalisten

Januar 2018

\hypertarget{das-gestuxe4ndnis}{%
\paragraph{Das Geständnis}\label{das-gestuxe4ndnis}}

Ich bin selbst schuldig, Teil des Systems zu sein oder zumindest das
Spiel mitzumachen. Zwar versuche ich seit fast zwanzig Jahren mit allen
erdenklichen Mitteln den Mainstream um alternative Perspektiven und
unabhängige Stimmen zu bereichern. Sie mögen vereinzelt wahrgenommen
werden, aber am Ende gehen sie im Mainstream-Lärm meist unter.

Der Ausstieg aus den Mainstream-Medien war für mich bisher nur eine
Option für den Fall, dass ich es nicht mehr aushalten sollte. Soweit,
dass ein Einzeljournalist nicht mehr publizieren kann, was er für
richtig hält, wollte ich es nicht kommen lassen. Nicht in einer
Demokratie.

Meine Strategie war immer, drinnen die subtile Gegenstimme zu sein,
immer so weit zu gehen wie möglich. Als Redaktionsmitglied und
Ressortleiter habe ich über Jahre versucht zu verstehen, wie Entscheide,
in diese oder jene Richtung zu publizieren, zu Stande kamen -- und wie
ich sie beeinflussen konnte, ohne als zu konträr oder destruktiv zu
gelten.

Natürlich habe ich riskiert, die rote Linie zu überschreiten und habe
sie auch überschritten und dafür gebüsst. Ich habe den Chefredaktoren
und anderen Ressortleitern gesagt, wenn ich ihre Thesen falsch fand, ich
habe argumentiert, gestritten, resigniert. Für die einen galt ich als
Gewissen der Redaktion, an anderen perlte alles ab und weitere
versuchten mich zu untergraben. Wenn der Alltag unerträglich wurde, half
meistens ein Redaktionswechsel. So arbeitete ich bereits für diverse
Schweizer Tages- und Wochenzeitungen, das Fernsehen, Agenturen und
weitere Medienformate.

Nun möchte ich möglichst anschaulich darstellen, was mich zu meinen
Schlussfolgerungen führte.

\hypertarget{entscheide-ohne-verantwortung}{%
\paragraph{Entscheide ohne
Verantwortung}\label{entscheide-ohne-verantwortung}}

Auf die Frage, wieso die Medien oft voller einseitiger Berichte mit
Tunnelblick sind, gibt es keine einfache Erklärung, aber ein paar
fassbare Gründe. Nach zwei Jahrzehnten auf Redaktionen aller Formate
weiss ich, wie Entscheide von Profilierungsgier, Gruppendynamik und
Willkür geprägt sind. Und warum sachliche Argumente oftmals Floskeln wie
„wir haben exklusiv``, „die anderen Medien haben``, oder „wir müssen
dabei sein`` unterliegen.

Die Entscheidungsprozesse, die zum jeweiligen Fokus und zur Gewichtung
von Berichten führen, erscheinen für einzelne Journalisten vielfach
beliebig und oft ist unklar, wie sie zustande kamen. Doch die Entscheide
basieren auf einer klaren Wertehaltung, die viele Perspektiven
ausblendet, sowie einer Quellenlage, die sehr selektiv ist.
Verantwortlich für die Entscheide und ihre Konsequenzen fühlt sich
sowieso niemand -- der Chefredaktor übernimmt bei Beschwerden oder
Klagen die Verantwortung aufgrund seiner Funktion.

Dieses Fahrlässige, dieses Diffuse und Beliebige habe ich immer als
Wursterei empfunden. Doch die Würste, die herauskamen, waren immer sehr
normiert -- konform mit einer Weltanschauung, wo schon immer klar ist,
wer der Gute und wer der Böse ist. Schliesslich stammen die Quellen,
wenn man sich das genau überlegt, vor allem von der „guten`` Seite.

Die vorherrschenden Denkmuster sind schwer zu erfassen. Man ist zum
Beispiel nicht treu gegenüber den Regierungsmitgliedern, aber weitgehend
gegenüber ihren wichtigsten Interessen. Journalisten spielen gerne auf
den Mann. Mit Personalisierungen lassen sich Leser erreichen. So wurde
der amtierende Schweizer Wirtschaftsminister massiv angegriffen für das
undurchsichtige Steuergeflecht seiner Firma. Seine Wirtschaftspolitik,
zum Beispiel der Freihandel, wird aber nie so genau auseinandergenommen.

Fehlerhafte Berichterstattung hat zudem selten Konsequenzen -- höchstens
ein Korrigendum. Selbst Bundesräte, deren Worte zugespitzt wurden bis
hart an die Grenze des journalistisch Zulässigen, sind nach einer
Schmollphase gerne wieder mit ihren Botschaften in der Publikation
präsent.

Das Grundproblem des Mainstream-Journalismus ist nicht, dass er bewusst
einer Ideologie folgt oder bewusst eine Weltanschauung vertritt, sondern
vielmehr seine Beliebigkeit, sein vorauseilender Gehorsam und die
Selbstzensur, der wichtige Fragen und Quellen entgehen. Die
Tunnel-Perspektive ist selbstauferlegt.

\hypertarget{medienhuxe4user-mehr-oder-weniger-angst}{%
\paragraph{Medienhäuser: Mehr oder weniger
Angst}\label{medienhuxe4user-mehr-oder-weniger-angst}}

Zu vergleichen, wo das Korsett am engsten war, wäre dumm. Die Spielräume
sind abhängig von der Art des Mediums (Tages-, Wochenformat) und auch
von der Position des Journalisten. Dennoch hat jeder Verlag seine
Charakteristika, die das Umfeld der Redaktionen prägen.

Nirgends war das Klima der Angst grösser und expliziter als beim
grössten privaten Medienhaus der Schweiz. Die Angst vor der nächsten
Sparrunde, Umstrukturierung, vor dem unerwarteten Seitenhieb in der
Blattkritik. Immerhin gibt es die interne Blattkritik noch, aber sie hat
oftmals wenig mit Qualitätsgarantie zu tun -- sie dient eher als
Führungsinstrument, um aufzubauen und abzuschiessen. Die Journalisten
sind in, out oder toleriert.

Wer out ist, wird selten entlassen, sondern im Plenum abgekanzelt oder
ignoriert -- aber nie so systematisch, dass es offensichtlich ist für
die Mehrheit. Meistens setzt sich niemand für die Attackierten ein. Die
meisten sind froh, dass sie nicht selbst an der Reihe sind.

Das zweitgrösste private Medienhaus der Schweiz ist menschlich und auch
finanziell grosszügiger. Es lag lange mehr Individualismus drin, schräge
Vögel waren toleriert, die Redaktion lange weniger homogen als anderswo,
was allerdings auch Extreme förderte.

Bei diesem Medienhaus herrschte weniger Existenzangst, aber umso mehr
Beliebigkeit. Jede neue Chefredaktion scheint für eine Mission
ausgewählt worden zu sein, um ``das Volk zu erreichen'' und die Marke
vor dem Niedergang zu retten. Im Ausland- und Politikteil werden die
etablierten Meinungen gerne zugespitzt -- Aufmerksamkeit um jeden Preis.

Auch die Unternehmens-Interessenpolitik ist gnadenlos: In der Wirtschaft
sind die Freunde des Konzernchefs publizistisch penetrant begleitet
worden. Hochgeschrieben werden mussten wegen „oben`` gewisse Mode- und
Uhrenfirmen befreundeter Unternehmer. Auch einige Banken- und
Pharma-Chefs erhielten aus demselben Grund immer wieder einen
prominenten Platz.

In der News-Abteilung des öffentlichen Rundfunks dominiert die Angst der
Leiter, zu „links`` oder angreifbar zu sein. Darum sind die News immer
ein Mitschnitt von allem -- nicht mehrheitsfähige Beiträge kommen nicht
auf den Sender.

Das Eigenständige in den Magazin-Formaten ist manchmal originell, aber
oft irrelevant, und nie konträr. Eine der am tiefsten recherchierenden
Sendungen brachte sogar den syrischen Präsidenten vor die Kamera, aber
dem gängigen Narrativ konträre Fragen wurden dann doch nicht gestellt.

Nirgends wollen die Redaktoren so bestätigt werden und gefallen wie beim
TV. Was beim verbreiteten Typus Narzisst auch nicht ankommt, das ist
Kritik. Es ist immer sehr aufschlussreich von Kameraleuten und Cuttern
zu erfahren, wie sie von Redaktoren behandelt werden. Die
TV-Entscheidungs­träger und viele Redaktoren halten ihren Journalismus
dem der Printredaktionen für weitaus überlegen.

Verlassen kann man ein solches Umfeld immer. Viele hielten es schlicht
nicht mehr aus, arbeiten nun in der PR. Andere schreiben was sie
wirklich denken für eines der leserfinanzierten Online-Magazine. Solange
ich im Mainstream immer wieder in eine eigene Richtung schwimmen kann,
bleibe ich.

\hypertarget{journalismus-zwischen-chaos-und-system}{%
\paragraph{Journalismus zwischen Chaos und
System}\label{journalismus-zwischen-chaos-und-system}}

Teil eines Systems zu sein -- über diese Vorstellung würden viele
Journalisten lachen. Der Alltag auf den Redaktionen ist meistens so
unsystematisch, so beliebig und chaotisch, dass die Vorstellung einer
Systematik geradezu absurd erscheint.

Man kann der breiten Palette der Journalisten und Journalistinnen vieles
vorwerfen. Aber, wenn sie merken würden, dass sie offensichtlich
gesteuert würden oder einem vorgegebenen Narrativ folgen müssten, dann
hätten wohl viele genügend Rückgrat und Know-how, um sich zu
widersetzen.

Dass sie bereits Teil eines bestimmten Narrativs sind und gewisse
Denkmuster verinnerlicht haben, dieses Bewusstsein fehlt jedoch
weitgehend.

Das ist besonders auffällig in der Auslandberichterstattung. Dort
herrscht die US-EU dominierte Sicht- und Erklärweise der Weltereignisse
vor. Ich fragte mich über die Jahre oft, mit wem die Korres­pon­denten
vor Ort sprechen und sich auseinandersetzen.

Nicht die westlich orientierten Eliten, aber die lokalen Intellektuellen
haben gerade in den Ländern Asiens vor dem Hintergrund ihrer eigenen
Grosskulturen ganz andere Lesearten der globalen Entwicklung.

Aber schon die Selektion der Korrespondenten -- sie müssen sich meistens
auf der Redaktion bewiesen haben -- setzt der Offenheit gegenüber
anderen Perspektiven Grenzen.

Das Hauptproblem der Auslandleiter sind nicht Korrespondenten, die zu
„faul`` sind, um ausserhalb des Büros zu recherchieren. Viel
einschneidender ist, dass Auslandressorts zusehends an Bedeutung
verlieren und ihre Mittel reduziert werden. Das bedeutet, dass zwar
Artikel aus den Hotspot-Regionen Platz erhalten. Ein noch so guter
Bericht über ein Land, das gerade nicht im Rampenlicht steht, ist intern
schwierig zu verkaufen.

Die Verantwortlichen müssen gegenüber Korrespondenten, von denen immer
weniger ein Fixum haben, unverbindlich bleiben. Alle Stories, die nicht
„zwingend`` sind wegen aktuellen Ereignissen, sind Manövriermasse. Die
Frage, was „zwingend`` ist, ist ein Thema für sich. Meistens sind
Berichte, die an ein Grossereignis (Anschlag, Wahlen) geknüpft sind,
zwingender als Berichte, die inhaltlich oder journalistisch überragend
sind.

Vor diesem Hintergrund haben es Berichte mit eigenständigen Perspektiven
oder solche, die nicht die „zwingenden Erwartungen`` erfüllen,
schwierig. Korrespondenten haben entsprechend wenig Anreiz, solche
Stories anzupacken.

Auch die Inland/Politik-Ressorts sind voller „zwingender`` Ereignisse
(Wahlen, Parlaments- und Bundesratsentscheide, Naturereignisse).
Allerdings wird der News- und Primeur-Charakter ebenfalls stark
gewichtet. Eine Definition dafür, was „News`` und was ein „Primeur``
(Scoop) ist, habe ich noch nie gesehen. Es scheint dazu einen gewissen
Konsens zu geben. Oder zumindest reagieren die Redaktionsleiter auf eine
bestimmte Art von News sehr positiv. Berichte oder Interviews
beispielsweise, die von Nachrichtenagenturen und der Konkurrenz
nachgezogen, d.h. übernommen, werden, sind per se gut -- weil das
Werbung in eigener Sache ist.

Ein Primeur, der einen „Nachzug`` generiert, kann ein sehr breit und
tief recherchierter Bericht sein, der etwas wirklich Neues, Relevantes
zu Tage fördert. Das ist es aber leider oft nicht. Es gibt viel
einfachere Wege, solche Primeurs zu schaffen. Am einfachsten ist es,
eine neue repräsentative Studie zu verwerten oder selbst eine solche in
Auftrag zu geben. Oft sind Primeurs auch exklusiv gesteckte Geschichten
-- durch PR-Berater oder PR-Abteilungen von Parteien, Politikern,
Unternehmen oder Institutionen. Ob es eine journalistische Leistung ist,
wenn Journalisten oder Chefredaktoren der auflagenstärksten Zeitungen so
etwas „gesteckt`` erhalten, lasse ich gerne offen.

Meistens werden solche Deus-Ex-Machina-News in grossen
Redaktionssitzungen auch nicht präsentiert, geschweige denn zur
Diskussion gestellt. Hinterfragen kann der Einzeljournalist so erhaltene
Stories schon, aber gemäss meiner Erfahrung sind solche „Exklusiv-News``
gesetzt, Fragen werden ignoriert.

Es wird immer noch debattiert in den Redaktionssitzungen. Aber die Dauer
der Diskussionen hat abgenommen. Oft geht es um Detailfragen, seltener
um Grundsätzliches. Die Frage, was ist uns für diese Ausgabe wichtig,
wird nie gestellt. Immerhin gibt es ein Bewusstsein, dass Konzerne ihre
Interessen vertreten, und dass Journalisten keine Konzern-PR machen
sollen. „Das ist eine PR-Story``, diese Beurteilung gibt es immer
wieder.

Doch wo PR anfängt -- diese Grenze ist flexibel. Wenn ein schwer
erreichbarer Konzernchef einem Ressortleiter oder Chefredaktor ein
Interview geben will, dann fühlen sich die Auserwählten geadelt. Auf
führende Finanzblätter gehen die Wirtschaftsleader bzw. ihre Sprecher
direkt zu -- natürlich nur dann, wenn sie etwas platzieren möchten. Bei
anderen Medien waren es über die Jahre vermehrt die Verlagschefs, die
Interviews vermittelten. Bei manchen Verlagen sitzen die Vertreter von
Banken und Konzernen bereits im Verwaltungsrat.

Es ist nicht so, dass sich die für die Interviews auserwählten
Journalisten zensurieren liessen -- sie kämpfen auch für Aussagen, die
die Kommunikationsabteilungen im Nachhinein streichen möchten. Doch es
kommt ihnen gar nicht in den Sinn, gewisse Fragen zu stellen.

Die Steuerung der Journalisten ist so subtil, dass viele den Vorwurf
nicht ernstnehmen könnten, dass sie auf vielen Ebenen manipuliert oder
zumindest eingeschränkt sind.

\hypertarget{woher-kommen-die-stories-und-wieso-werden-sie-publiziert}{%
\paragraph{Woher kommen die Stories und wieso werden sie
publiziert}\label{woher-kommen-die-stories-und-wieso-werden-sie-publiziert}}

Grob geschätzt sind die Themen bei über Dreiviertel der Berichte in
Online- und Tagespublikationen durch aktuelle, äussere
Ereignisse/Mitteilungen initiiert bzw. beeinflusst worden. Die
Nachrichten­agenturen sind weitgehend reaktiv organisiert. Wenn also
eine Mitteilung, Medien­konferenz etc. hereinkommt, dann kommt auch ein
Bericht raus, Eigenständiges hat zweite Priorität.

Bei Wochenpublikationen richten sich die Themen etwas weniger (etwa zu
Zweidrittel) nach aktuellen äusseren Ereignissen -- und wenn, dann mit
dem Ziel, die Begebenheit eigenständig weiterzudrehen.

Kriterien für die Relevanz eines äusseren Ereignisses und
Berichterstattung darüber sind Nähe, Grösse, Aussergewöhnlichkeit und
Folgeeffekte für das eigene Land: Wahlen in den USA, grossen EU-Staaten
oder Iran, Terroranschläge in Frankreich oder Bundesratsentscheide
fallen in diese Kategorie. In den klassischen Relevanzkriterien keinen
oder wenig Platz finden z.B. Terroranschläge mit heimischen Opfern in
Somalia oder Wahlen in Peru.

Da Ausland-News vor allem von amerikanischen, deutschen, englischen und
französischen Agenturen (AP, Reuters, AFP, DPA und Schweizer Partnerin
SDA) in die Redaktionen und an die Korrespondenten fliessen, ist die
Grösse, Nähe etc. aus der Sicht dieser Staaten definiert.

Auch für kleinere Ereignisse gibt es klare Kriterien, oft werden sie
aber nicht klar ausgesprochen. Bei etlichen Tages- und Wochenzeitungen
habe ich zudem erlebt, dass auf grosse Inserenten Rücksicht genommen
wurde, indem man freundliche Texte verfasste und Kritik blockierte.

\hypertarget{abluxe4ufe-auf-der-redaktion}{%
\paragraph{Abläufe auf der
Redaktion}\label{abluxe4ufe-auf-der-redaktion}}

Die vorbereitenden Redaktionssitzungen sind oft nach grossen aktuellen
Metathemen und den Stories aus den Ressorts aufgebaut. Der Vorgang ist
wie auf einem Bazar, wo die Ressortleiter die einzelnen Ideen auf den
Markt werfen. Bevor eine Story gesetzt ist, fragt der Chefredaktor, was
zu den aktuellen Themen vorhanden ist und was Wichtiges vergessen wurde.

Ein solches Thema, „zu dem wir unbedingt etwas haben müssen``, sind
aktuelle Debatten, kürzliche oder erwartete Ereignisse oder eine Story
mit Einschaltquoten, die Konkurrenten angestossen haben. Manchmal ist es
auch eine prominente Person, die noch nichts gesagt hat zu einem Thema
-- die „wir zuerst haben wollen``.

Darüber, was Schwerpunktthemen sein können, herrscht weitgehend Konsens.
Bei der Gewichtung ist Relevanz, Einschaltquote (Promis, Kontroverse,
Ausland ist meistens zu weit weg von den Leuten) ausschlaggebend,
weniger die Ausgewogenheit der Story. Es kommt regelmässig vor, dass
Stories während der Recherche von Konkurrenten „abgetischt`` werden, und
deshalb gekippt werden.

Das Gefühl der Beliebigkeit entsteht, wenn am Morgen (Tageszeitung) oder
Dienstag (Sonntags­publikation) „Die Themen`` zu stehen scheinen, sie
aber bis am Abend oder Freitag vom Tisch sind. Manchmal ist das
berechtigt wegen neuen Entwicklungen, oder wegen Recherchen, die im
Sande verlaufen. Oft wechseln die Prioritäten jedoch, weil der
Chefredaktor plötzlich etwas anderes wichtiger findet. Beispielsweise
eine Story, auf die er selbst gekommen ist. Transparenz über solche
Entscheide oder klare Kriterien gibt es nicht.

Im Sinne der freien Presse sind die einzelnen Journalisten selten in der
Lage, verbindliche Zusagen darüber zu machen, dass erhaltene Infos auch
erscheinen werden. Das ist eigentlich ein Zeichen für die
journalistische Unabhängigkeit. Aber oft ist Unverbindlichkeit auch
notwendig, weil die Führung chaotisch und konzeptlos ist. Die Situation,
dass ein Journalist jemanden kurzfristig aufbietet für ein Interview,
dieses dann aber nicht erscheint, ist normal.

Im Gegenzug ist aber auch der Fall häufig, dass eine Story oder ein
Interview gesetzt ist, egal was. Alles was mit hiesigen IS-Anhängern zu
tun hatte, war die letzten Jahre gefragt -- egal wie lau und ob „more of
the same.`` Die Begründung für eine Story ist oft nicht nachvollziehbar,
wer nachfragt, wird abgewiegelt. Dass sich ein Ressort zusammen für eine
Story stark macht, kommt selten vor. Als Kompromiss lässt sich eine
Story immer auf 60 Zeilen zusammenkürzen.

Die Stories der sogenannten Recherche-Desks, Investigativteams und
Autorenpools müssen selten um Platz oder Zeit kämpfen. Sie sind auch
selten an die Aktualität gebunden. Seit einigen Jahren sind diese Teams
Prestigeprojekte der Verleger. Sie erhalten mehr Mittel und
Datenspezialisten. Dadurch, dass diese Teams zumeist international
vernetzt sind, z.B. mit Lena (Europäische Medienallianz), kommen sie
auch an Recherchen von internationalen Journalisten.

Die Förderung und die internationale Kräftekonzentration von
Investigativ-Journalismus wäre eigentlich ein guter Ansatz zur
Qualitätssteigerung. Leider führen die Produkte auch zu einer
internationalen Gleichschaltung der Stories. Als ob es nicht bereits
ausreichen würde, dass die nationalen Berichte oftmals sehr ähnlich bis
deckungsgleich sind.

Es wäre überdies interessant zu untersuchen, wessen Interessen die
international koordinierten Recherche-Stories dienen, und was die
„Bösen`` dieser Stories gemeinsam haben oder was nicht.

Für den einzelnen Zeitungsjournalisten gilt: Wer eine zündende Idee mit
Rechercheresultat und guten Zitaten hat, bringt seine Story durch.
Mindestens die Hälfte der Redaktoren ist aber zu stark damit
beschäf­tigt, die aktuellen \emph{Musts} oder Abklärungen zu erledigen,
um selbst eine Story entwickeln zu können.

Es gibt verschiedene Strategien: Journalisten, die wenig selbst
entwickeln, aber Ideen/Inputs der Chefs ausführen, können am ehesten mit
prominenter Platzierung rechnen -- Absatzgarantie. Was in der eigenen
Kompetenz der Journalisten ist, und was vorgegeben ist, ist also sehr
individuell.

Deutlich weniger gefragt als bei den Printmedien sind eigene Ideen beim
Fernsehen: Das hat einerseits mit dem Filmaufwand und den vielen
Produktionsschritten zu tun, wo Teamarbeit Sinn macht. In den
TV-Newsabteilungen werden Story-Ideen auch von Inputern eingebracht, die
quasi den Überblick über die News-Lage haben. Es handelt sich dabei um
erfahrene Journalisten, deren Kernkompetenz nicht das TV-Handwerk ist.

Andererseits sind bei der Entwicklung der Story sehr viele Leute
beteiligt (Inputer, Ressortleiter, Produzent und Chefredaktor der
Sendung), die auch noch mitreden und sich quasi im Ergebnis sehen
wollen.

Die Entwicklung eines TV-Beitrags ist ein langer Prozess, bei dem der
Redaktor seine Idee verfeinern kann, oder sie aber verwässert wird. Beim
TV gibt es mehr handwerkliche Gründe als anderswo, wieso sich erste
Story-Ideen nicht umsetzen lassen (z.B. visuell unattraktiv). Der eigene
Gestaltungsspielraum der TV-Journalisten ist deutlich geringer als bei
den Print-Journalisten.

\hypertarget{deutungsmuster}{%
\paragraph{Deutungsmuster}\label{deutungsmuster}}

Die Einordnung von Ereignissen und Entwicklungen geschieht in den
Massenmedien nicht mit einem verbindlichen Deutungsmuster -- weil es
ohne verbindliche Haltung (siehe oben) auch kein Deutungs­muster geben
kann.

Nichtsdestotrotz herrschen aufgrund der Quellenlage bestimmte
Denkstrassen vor, von denen der einzelne zwar abweichen kann, aber nur
ab und zu, nie systematisch. Bei internationalen Konflikten wie in der
Ukraine oder Syrien dominiert die Sicht der USA. Bei Terroranschlägen
dasselbe.

\hypertarget{woher-propaganda-kommt}{%
\paragraph{Woher Propaganda kommt}\label{woher-propaganda-kommt}}

Es besteht keinerlei Bereitschaft/Anlass, die bestehenden Feindbilder zu
hinterfragen. Aufgrund der Verarbeitung der Informationsflut, vor allem
aus dem Ausland (Agenturen), bleibt keine Zeit für Grundsätzliches. Für
Auslandressortleiter ist das Briefing und Updating der Korrespondenten
teilweise sehr aufwändig. Ein langjähriger US-Korrespondent z.B.
brauchte stundenlanges Zureden, damit er das Büro verliess, um in der
Realität zu recherchieren.

Während der Ukrainekrise waren zwar Interviews und Artikel aus
russischer Sicht erlaubt, sie gingen aber in der Konsensmeinung unter.
Und sowieso glaubt man die Sicht der anderen Seite immer schon zu
kennen, da muss man nicht noch tiefer gehen. Der Propaganda wird nur die
„böse`` Seite bezichtigt.

Zur Gleichschaltung bei den Quellen und Perspektiven tragen auch die
neuen internationalen Netzwerke bei. So druckte das grösste Schweizer
Medienhaus unter anderem Auslandberichte der \emph{Welt/Welt am Sonntag}
ab, heute werden vornehmlich Berichte aus der \emph{Süddeutschen} in der
Schweiz zweitverwertet. Geliefert wurde z.B. eine Reportage darüber, wie
die Putin-Jugend in Camps nationalistisch gedrillt wird -- das bedient
alle Klischees von Russlands Propaganda-Krieg.

\hypertarget{uxfcber-legitime-quellen-herrscht-weitgehend-konsens}{%
\paragraph{Über legitime Quellen herrscht weitgehend
Konsens}\label{uxfcber-legitime-quellen-herrscht-weitgehend-konsens}}

Neben den handelsüblichen Quellen, den Agenturen und nationalen
Konkurrenzmedien/-portalen, gehen Journalisten Auskunftsquellen auch
direkt an. Es zählt: Wer bringt News und Klicks (Prominenz, knackige
Aussagen). Bei den Experten: Wer macht nicht zu viel Aufwand, wer war
nicht schon überall zum jeweils aktuellen Thema. Beim TV geht es auch
darum, wer kameratauglich ist und etwas sagt, das wir brauchen können.

Die Wahl weniger bekannter Quellen muss begründet werden. Öfters
basieren Stories auf anonymen Quellen. Diese werden aber im Normalfall
sauber geprüft, bei heiklen Aussagen gibt es Doppelchecks.

Bei den sozialen Medien ist Twitter als Quelle sehr verbreitet.
Allerdings ist der Twitter-Kreis, dem einzelne Journalisten in
Mainstreammedien folgen, ziemlich eng -- man folgt den anderen in der
Branche und den Twitter-Kanälen von Mainstreammedien. Zum Teil sind die
Gefolgten deckungsgleich -- Journalisten mit originellen Twitter-Quellen
sind rar. Twitter wird auch zum Werben für die eigenen Publikationen
genutzt. Man folgt, wer einem folgt -- de facto ist das ein
geschlossener Kreis.

Der öffentliche Rundfunk ist sehr akribisch mit der Zeit, die einzelnen
Partei-Quellen gewährt wird, besonders vor Abstimmungen. Die Zeitungen
versuchen ebenfalls alle Seiten zu Wort kommen zu lassen. Am Ende
erhalten aber nur jene Politiker Platz, die etwas Spannendes oder
Kontroverses zu sagen haben.

Die eigenen Korrespondenten-Netzwerke der Zeitungen sind die letzten
Jahre massiv geschrumpft -- eine versiegende Quelle. Die
Korrespondenten-Artikel gelten oft als Manövriermasse, wenn es ums
Sparen geht. «Wir können auch die Agentur nehmen\ldots{}», heisst es
dann. Korrespondenten, die Publikationen oder Aufträge der Redaktion
hinterfragen, sind rar -- wer sägt schon am dünnen Ast, auf dem er
sitzt.

In der Auslands- bzw. Wirtschaftsberichterstattung werden
\emph{Spiegel-Online, Die Welt, Die Süddeutsche,} die \emph{Financial
Times,} der \emph{Economist,} der \emph{Guardian,} die \emph{New York
Times,} und das \emph{Wall Street Journal} konsultiert. Systematisch
bearbeitet werden diese internationalen Medien in erster Linie von
Chefredaktoren und Ressortleitern. Da geht es nicht nur um das
Informieren, sondern auch um das Inspirieren bzw. Kopieren.

Dass eigene Themen oder spannende Thesen, die die \emph{New York Times}
publiziert hat, zwei bis drei Wochen später 1:1 in einer Planungssitzung
vorgeschlagen werden, kommt oft vor. Chefredaktoren zeigen sich meistens
begeistert und machen nicht den Anschein, dass sie das schon gelesen
haben. Wer darauf hinweist, dass das eine Kopie ist, macht sich
entsprechend unbeliebt.

Ob das eine bewusste Orientierung an solchen Leitmedien ist, ist
fraglich. Es ist wohl eher eine unbewusste Orientierung verbunden mit
mangelndem Ehrgeiz und Zeit, etwas Eigenes auf die Beine zu stellen.
Aber es ist wahrscheinlicher, dass sich die leitenden Redaktoren zu
wenig über Auslandmedien informieren, als dass sie sich zu stark davon
beeinflussen liessen.

Um die Jahrtausendwende waren Redaktionen ein Paradies mit
verschiedensten Publikationen, in denen man rumschmökern konnte.
Inzwischen bestellen die Redaktionen immer mehr Publikationen ab -- die
oftmals nicht mit einem Online-Zugang ersetzt werden.

Um manche Quellen müssen sich Journalisten nicht selber bemühen, sie
kommen von selbst. Bei wirtschaftlichen Ereignissen kommt es öfters vor,
dass PR-Agenturen der jeweiligen Akteure die Medien füttern. Ihre
eigenen Interessen vertreten auch Verwaltungsräte, die off-the-record
Informationen an Journalisten weitergeben. Beliebte Infoquellen sind
auch manche Politiker. Mit deren ``exklusiven'' Informationen über
geplante politische Vorstösse lassen sich ganze Sonntags­zeitungen
füllen.

Für internationale Berichte kommen auch Nichtregierungsorganisationen
(NGO) vor Ort zu Wort, allerdings oft nur mit Ergänzungen von Beispielen
und Schicksalen. Die politischen Ansichten der NGO werden meist nicht
transportiert. Wenn ein Auslandbericht etwas tiefer geht, dann wird oft
personifiziert -- Schicksale lassen sich relativ apolitisch abhandeln.

\hypertarget{welche-quellen-tabu-sind}{%
\paragraph{Welche Quellen tabu sind}\label{welche-quellen-tabu-sind}}

Eine rote Linie bei den Quellen gibt es nicht explizit. Vielmehr ist
Selbstzensur üblich, indem man nicht die Quellen fragt, die total gegen
den Strom schwimmen. Oft sind Quellen auch nur „die üblichen
Verdächtigen`` -- man klappert diejenigen ab, die man schon immer
abgeklappert hat. Daher gleichen sich die Berichte zwischen den Medien
auch sehr. Eine Art rote Linie gibt es für den Platz, den man einräumt.
Den meisten Platz erhalten nicht die Systemkritiker, sondern jene, die
die Allgemeinmeinung nicht zu fest angreifen.

\hypertarget{klare-hierarchien-trotz-kumpelhaftigkeit}{%
\paragraph{Klare Hierarchien trotz
Kumpelhaftigkeit}\label{klare-hierarchien-trotz-kumpelhaftigkeit}}

Die Hierarchien spielen auf manchen Redaktionen eine sehr starke Rolle.
Auch wenn sich die Vorgesetzten sehr offen geben: Entscheide von
Ressortleitern, geschweige denn von Chefredaktoren werden kaum
hinterfragt. In allen Medien machen Blattmacher, Art Directors und
Ressortleiter mitunter die meiste Vorarbeit, können konträre Ansichten
aber nicht gegen den Chefredaktor und seine ein bis zwei loyalen
Mitarbeiter (z.B. Stellvertreter) durchsetzen -- sofern sie nicht
sowieso nur den Chefredaktor bestätigen. Oft setzen die unteren
Kaderstufen die Wünsche ihrer Chefs härter durch als ihre Vorgesetzten
es selbst würden (vorauseilender Gehorsam, absolute Loyalität in der
Hierarchie). Ein echtes internes Korrektiv ist inzwischen die Ausnahme.

In die Mainstreammedien geht man nicht, um viel Geld zu verdienen. Die
Löhne sind vergleichsweise tief. Nur in wenigen Kaderposten sind sie
lukrativ. Wer sich die sich stets verschlechternden Arbeitsbedingungen
(schrumpfende Lohnnebenleistungen, immer längere Arbeitszeiten) antut,
macht dies am ehesten aus Überzeugung oder aber aus Profilierungslust.

Journalist kann sich jeder nennen, und so manche Leute denken, das
Metier könnten sie auch. Tatsache ist aber, dass die Qualität der
Ausbildung durch neue Studiengänge in den letzten Jahren deutlich
zugenommen hat. Die heutigen Journalismus-Lehrgänge sind sehr umfassend:
von Recht, Ethik, Multimedia, Genres, bis zu Stil etc. wird sehr vieles
vermittelt.

Jungjournalisten müssen sich meistens abverdienen mit mühsamen Umfragen
und Zudienen, bevor sie etwas Eigenes machen können. Immer öfters
steigen Praktikanten nicht mehr mit Feuer und Flamme in den Journalismus
ein. Weil viele von ihnen Journalismus zusammen mit PR studiert haben,
ist es typisch, dass sie offen lassen, ob sie sich mit ihren
Kommunikationsfähigkeiten im Journalismus oder in der PR
weiterentwickeln möchten.

Die Selektion von Jungen ist leistungsorientiert, aber auch immer
abhängig von der Zusammensetzung der Teams. Wer ein Journalismus-Studium
und Journalismus-Praktika ausweist, hat gute Chancen. In der Praxis wird
schnell klar, wen man ``brauchen'' kann. Wer einmal im Journalismus
``gelandet ist'', braucht ein gutes Netzwerk und einen guten Ruf. Die
Leistung ist nicht mehr so wichtig, mehr zählt, ``wer passt''.
Chefredaktoren stellen viele ihrer Ex-Kollegen an -- der Kuchen der
erfahrenen Journalisten ist relativ klein.

Kaderpositionen gehen nicht unbedingt an die besten Schreiber, sondern
an jene, die zwar gut sind, aber wenig polarisieren. Wer den
Ressortleitern und Chefredaktoren inhaltlich zu stark widerspricht,
fördert seine Aufstiegschancen nicht. Wer ``gefällt'' hat es einfacher
aufzusteigen. Chefredaktoren machen in der Blattkritik und in Sitzungen
etwas „verpackt`` klar, was sie mochten und was nicht.

\hypertarget{endzeitstimmung}{%
\paragraph{Endzeitstimmung}\label{endzeitstimmung}}

Der Arbeitsdruck und die Angst vor Jobverlust auf den Redaktionen sind
überragend. Es handelt sich um eine Branche, in der die meisten davon
ausgehen, dass sie nur noch schrumpft und weniger befriedigender zum
Arbeiten wird. Die Vorgabe, mehr zu sparen, ist bei den führenden
Schweizer Medienhäusern seit zwei Jahrzehnten omnipräsent. Manche
Redaktionen sind bereits sehr ausgedünnt. Zehn- bis zwölfstündige
Arbeitstage gelten als normal. Die Überzeit aufschreiben können
Journalisten in den Printmedien selten.

Das Sparen hat Folgen: Selbst Chefredaktoren und Ressortleiter, die
Mehrwert liefern wollen, müssen sich mit dem reinen Abfüllen von Seiten
mit Agenturen oder second class Stories zufrieden geben. Erfahrene
Journalisten versuchen first class stories mit möglichst wenig Aufwand
vorzuschlagen. Man weiss, was zieht -- wieso vom Bewährten abrücken?

Die meisten seilen sich bevor sie 50 werden in die PR ab. Das Angebot
von gutbezahlten PR-Jobs ist für erfahrene Journalisten gross. Der
Spruch: «jetzt gehe ich dann in die PR, weil das was ich hier mache ist
nicht viel anders, einfach schlechter bezahlt», ist nicht unüblich.

Wer die Überzeugung hat, mit Journalismus etwas bewirken zu können, ist
sehr jung, oder wird ausgelacht von der Mehrheit.

Vordergründig herrscht in den Mainstreammedien kompletter
Meinungspluralismus -- abgesehen von Einzelfällen tun sich Journalisten
kaum mit festen Überzeugungen hervor: Man ist für alles offen, suspekt
ist, wer eine Haltung hat und konsequent aus dieser Haltung berichtet.
Die Journalisten mit einer konsequenten Weltanschauung sind rar. So wie
man persönlich bei niemandem anecken will, zeigt man auch journalistisch
keine Kanten, womit man polarisieren könnte.

Hinter den Kulissen verstehen die Verlage ihre öffentlichen Medien immer
noch als «vierte Gewalt» im Staat. Aber sie wissen wohl selbst, dass sie
diese Schritt um Schritt entmachten -- mit ihren Sparstrategien, dem
Abbau der Vielfalt etwa durch vereinheitlichte Mantelredaktionen und
ihrer Orientierungslosigkeit.

Es ist nicht so, dass es vor 20 bis 30 Jahren auf den Redaktionsstuben
keine Selbstzensur, Gleich­schaltung, vorauseilenden Gehorsam und
Tunnelblick gegeben hätte. Doch die Rahmen­bedingungen in den heutigen
„Redaktions­fabriken`` der Mainstream-Medien fördern geradezu den
ideologielosen, opportunistischen, Klick-orientierten Journalismus, dem
die wichtigen Fragen entgleiten.

Es ist leider ein Journalismus, der durch die Röhre guckt. Eine
geförderte, aber dennoch freiwillige Beschränkung der Perspektive
bedroht die Unabhängigkeit der vierten Gewalt mehr denn je.

***

Der Autor ist Schweizer Journalist. Er ist nicht Mitglied der
Forschungsgruppe SPR.

\textbf{Copyright-Hinweis:} Bei Interesse am Beitrag bitte auf diese
Seite verlinken. Einzelne Auszüge und Zitate dürfen übernommen werden
(keine Volltext-Kopie).

\begin{center}\rule{0.5\linewidth}{\linethickness}\end{center}

Beitrag teilen auf:
\href{https://twitter.com/intent/tweet?url=https://swprs.org/bericht-eines-journalisten/}{Twitter}
/
\href{https://www.facebook.com/share.php?u=https://swprs.org/bericht-eines-journalisten/}{Facebook}\\
Publiziert: Januar 2018

~

\hypertarget{swiss-policy-research}{%
\subsubsection{Swiss Policy Research}\label{swiss-policy-research}}

\begin{itemize}
\tightlist
\item
  \href{https://swprs.org/kontakt/}{Kontakt}
\item
  \href{https://swprs.org/uebersicht/}{Übersicht}
\item
  \href{https://swprs.org/donationen/}{Donationen}
\item
  \href{https://swprs.org/disclaimer/}{Disclaimer}
\end{itemize}

\hypertarget{english}{%
\subsubsection{English}\label{english}}

\begin{itemize}
\tightlist
\item
  \href{https://swprs.org/contact/}{About Us / Contact}
\item
  \href{https://swprs.org/media-navigator/}{The Media Navigator}
\item
  \href{https://swprs.org/the-american-empire-and-its-media/}{The CFR
  and the Media}
\item
  \href{https://swprs.org/donations/}{Donations}
\end{itemize}

\hypertarget{follow-by-email}{%
\subsubsection{Follow by email}\label{follow-by-email}}

Follow

\href{https://wordpress.com/?ref=footer_custom_com}{WordPress.com}.

\protect\hyperlink{}{Up ↑}

\includegraphics{https://pixel.wp.com/b.gif?v=noscript}
