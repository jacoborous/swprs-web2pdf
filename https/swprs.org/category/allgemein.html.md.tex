\protect\hyperlink{content}{Skip to content}

\href{https://swprs.org/}{}

\protect\hyperlink{search-container}{Search}

Search for:

\href{https://swprs.org/}{\includegraphics{https://swprs.files.wordpress.com/2020/05/swiss-policy-research-logo-300.png}}

\href{https://swprs.org/}{Swiss Policy Research}

Geopolitics and Media

Menu

\begin{itemize}
\tightlist
\item
  \href{https://swprs.org}{Start}
\item
  \href{https://swprs.org/srf-propaganda-analyse/}{Studien}

  \begin{itemize}
  \tightlist
  \item
    \href{https://swprs.org/srf-propaganda-analyse/}{SRF / ZDF}
  \item
    \href{https://swprs.org/die-nzz-studie/}{NZZ-Studie}
  \item
    \href{https://swprs.org/der-propaganda-multiplikator/}{Agenturen}
  \item
    \href{https://swprs.org/die-propaganda-matrix/}{Medienmatrix}
  \end{itemize}
\item
  \href{https://swprs.org/medien-navigator/}{Analysen}

  \begin{itemize}
  \tightlist
  \item
    \href{https://swprs.org/medien-navigator/}{Navigator}
  \item
    \href{https://swprs.org/der-propaganda-schluessel/}{Techniken}
  \item
    \href{https://swprs.org/propaganda-in-der-wikipedia/}{Wikipedia}
  \item
    \href{https://swprs.org/logik-imperialer-kriege/}{Kriege}
  \end{itemize}
\item
  \href{https://swprs.org/netzwerk-medien-schweiz/}{Netzwerke}

  \begin{itemize}
  \tightlist
  \item
    \href{https://swprs.org/netzwerk-medien-schweiz/}{Schweiz}
  \item
    \href{https://swprs.org/netzwerk-medien-deutschland/}{Deutschland}
  \item
    \href{https://swprs.org/medien-in-oesterreich/}{Österreich}
  \item
    \href{https://swprs.org/das-american-empire-und-seine-medien/}{USA}
  \end{itemize}
\item
  \href{https://swprs.org/bericht-eines-journalisten/}{Fokus I}

  \begin{itemize}
  \tightlist
  \item
    \href{https://swprs.org/bericht-eines-journalisten/}{Journalistenbericht}
  \item
    \href{https://swprs.org/russische-propaganda/}{Russische Propaganda}
  \item
    \href{https://swprs.org/die-israel-lobby-fakten-und-mythen/}{Die
    »Israel-Lobby«}
  \item
    \href{https://swprs.org/geopolitik-und-paedokriminalitaet/}{Pädokriminalität}
  \end{itemize}
\item
  \href{https://swprs.org/migration-und-medien/}{Fokus II}

  \begin{itemize}
  \tightlist
  \item
    \href{https://swprs.org/covid-19-hinweis-ii/}{Coronavirus}
  \item
    \href{https://swprs.org/die-integrity-initiative/}{Integrity
    Initiative}
  \item
    \href{https://swprs.org/migration-und-medien/}{Migration \& Medien}
  \item
    \href{https://swprs.org/der-fall-magnitsky/}{Magnitsky Act}
  \end{itemize}
\item
  \href{https://swprs.org/kontakt/}{Projekt}

  \begin{itemize}
  \tightlist
  \item
    \href{https://swprs.org/kontakt/}{Kontakt}
  \item
    \href{https://swprs.org/uebersicht/}{Seitenübersicht}
  \item
    \href{https://swprs.org/medienspiegel/}{Medienspiegel}
  \item
    \href{https://swprs.org/donationen/}{Donationen}
  \end{itemize}
\item
  \href{https://swprs.org/contact/}{English}
\end{itemize}

\protect\hyperlink{}{Open Search}

\hypertarget{category-allgemein}{%
\section{Category: Allgemein}\label{category-allgemein}}

\hypertarget{covid-19-august-update}{%
\section{\texorpdfstring{\href{https://swprs.org/2020/07/30/covid-19-august-update/}{Covid-19
August Update}}{Covid-19 August Update}}\label{covid-19-august-update}}

SPR has published its
\href{https://swprs.org/a-swiss-doctor-on-covid-19/}{Covid-19 August
update}. The overview has been slightly updated and some new diagrams
have been added.

This is our final monthly update. The next update and review will follow
later this year.

Low-prevalence countries and regions -- such as Germany with 1.3\% IgG
antibody prevalence -- are likely to face a renewed increase in
infections and disease and should urgently consider an
\href{https://swprs.org/on-the-treatment-of-covid-19/}{early treatment
strategy} for their high-risk and high-exposure groups.

A new page on the \href{https://swprs.org/face-masks-evidence/}{evidence
regarding cloth face masks} has also been added.

\textbf{August update}:
\href{https://swprs.org/a-swiss-doctor-on-covid-19/}{English →} /
\href{https://swprs.org/covid-19-hinweis-ii/}{Deutsch →}

\begin{center}\rule{0.5\linewidth}{\linethickness}\end{center}

\href{https://swprs.org/2020/07/30/covid-19-august-update/}{**30. July
2020}

\hypertarget{mediennutzung-sieben-tipps}{%
\section{\texorpdfstring{\href{https://swprs.org/2020/07/21/mediennutzung-sieben-tipps/}{Mediennutzung:
Sieben
Tipps}}{Mediennutzung: Sieben Tipps}}\label{mediennutzung-sieben-tipps}}

Sieben Tipps für den cleveren Umgang mit Medien.

\href{https://swprs.org/mediennutzung-sieben-tipps/}{Weiterlesen →}

\begin{center}\rule{0.5\linewidth}{\linethickness}\end{center}

\href{https://swprs.org/2020/07/21/mediennutzung-sieben-tipps/}{**21.
July 2020}

\hypertarget{willkommen}{%
\section{\texorpdfstring{\href{https://swprs.org/2018/10/09/willkommen/}{Willkommen}}{Willkommen}}\label{willkommen}}

Swiss Policy Research (SPR) ist ein Forschungs- und Informationsprojekt
zu geopolitischer Propaganda in Schweizer und internationalen Medien. Im
Folgenden finden Sie einige unserer bekanntesten Arbeiten auf einen
Blick:

\begin{itemize}
\tightlist
\item
  Die Studien zur \href{https://swprs.org/die-nzz-studie/}{Neuen Zürcher
  Zeitung}, dem
  \href{https://swprs.org/srf-propaganda-analyse/}{Schweizer Radio und
  Fernsehen}, und den drei globalen
  \href{https://swprs.org/der-propaganda-multiplikator/}{Nachrichtenagenturen}
\item
  Den \href{https://swprs.org/medien-navigator/}{Medien-Navigator} zur
  geopolitischen Ausrichtung deutschsprachiger Medien
\item
  Den
  \href{https://swprs.org/der-propaganda-schluessel/}{Propaganda-Schlüssel}
  zu den wich­tig­sten medialen Manipulations­techniken
\item
  Die Infografiken zu medialen Netzwerken in der
  \href{https://swprs.org/netzwerk-medien-schweiz/}{Schweiz},
  \href{https://swprs.org/netzwerk-medien-deutschland/}{Deutschland},
  und den
  \href{https://swprs.org/das-american-empire-und-seine-medien/}{USA}
\item
  Die Analyse zur
  \href{https://swprs.org/propaganda-in-der-wikipedia/}{Wikipedia-Manipulation}
\item
  Die Analysen zur
  \href{https://swprs.org/logik-imperialer-kriege/}{Logik imperialer
  Kriege} und der medialen
  \href{https://swprs.org/die-propaganda-matrix/}{Informationsmatrix}
\item
  Den exklusiven
  \href{https://swprs.org/bericht-eines-journalisten/}{Journalistenbericht}
\item
  Die Analyse zu
  \href{https://swprs.org/russische-propaganda/}{russischer Propaganda}
\end{itemize}

Bei Fragen oder Rückmeldungen können Sie die Forschungsgruppe gerne
\href{https://swprs.org/kontakt/}{kontaktieren}.

\begin{center}\rule{0.5\linewidth}{\linethickness}\end{center}

\href{https://swprs.org/2018/10/09/willkommen/}{**9. October 2018}

\hypertarget{facts-about-covid-19}{%
\section{\texorpdfstring{\href{https://swprs.org/2018/10/01/covid-19-hinweis-ii/}{Facts
about Covid-19}}{Facts about Covid-19}}\label{facts-about-covid-19}}

Fully referenced facts about Covid-19, provided by experts in the field,
to help our readers make a realistic risk assessment. (Regulary
updated)*\\
*

~\href{https://swprs.org/a-swiss-doctor-on-covid-19/}{English →} /
\href{https://swprs.org/covid-19-hinweis-ii/}{Deutsch →}

\begin{center}\rule{0.5\linewidth}{\linethickness}\end{center}

\href{https://swprs.org/2018/10/01/covid-19-hinweis-ii/}{**1. October
2018}

\hypertarget{zensur-und-selbstzensur}{%
\section{\texorpdfstring{\href{https://swprs.org/2017/03/01/zensur-in-schweizer-medien/}{Zensur
und
Selbstzensur}}{Zensur und Selbstzensur}}\label{zensur-und-selbstzensur}}

\href{https://swprs.org/2017/03/01/zensur-in-schweizer-medien/}{\includegraphics{https://swprs.files.wordpress.com/2016/07/zensur-selbstzensur-c.png?w=306}}

Zensur und Selbst­zensur bei geo­po­li­tischen Kon­f‌lik­ten sind in der
Schweiz keines­wegs un­be­kannt, wie ein Blick in die Ge­schichte zeigt.

Um das Land keinen un­nöti­gen Ri­si­ken aus­zu­setzen,
\href{http://www.amazon.de/Selbstzensur-schweizerische-Pressepolitik-Zweiten-Weltkrieg/dp/3719304566}{muss­ten}
sich Medien und
\href{https://www.chronos-verlag.ch/node/20528}{Buch­ver­lage} wäh­rend
des

\begin{enumerate}
\def\labelenumi{\arabic{enumi}.}
\tightlist
\item
  und 2. Welt­kriegs und während des
  \href{http://www.swissinfo.ch/ger/das-ende-eines-nationalen-maenner-netzwerks/4205194}{Kal­ten
  Kriegs} an einen po­li­tisch definierten Mei­nungs­korri­dor halten,
  der sich an den welt­wei­ten Kräfte­ver­hält­nissen orientierte.
\end{enumerate}

Durch die Ereignisse von 1990 und 2001 nahm der
\href{https://www.youtube.com/watch?v=a4eGtXFDFJA}{Druck} auf
Drittstaaten und ihre Medien wei­ter zu: »Entweder mit uns, oder gegen
uns.«

Aufgrund der Medien­kon­zen­tration werden in­zwi­schen zudem
\href{https://swprs.files.wordpress.com/2018/03/broschur_jahrbuch_foeg_deutsch_2015.pdf\#page=13}{über
90\%} des Schwei­zer Mark­tes von nur noch fünf Medien­häusern bedient:
Tamedia, Ringier, NZZ Medien und AZ Medien, sowie der SRG (siehe
\href{https://swprs.org/netzwerk-medien-schweiz/}{Info­grafik}).

Eine echte \href{https://swprs.org/medien-navigator/}{Medienvielfalt}
entstand mithin erst durch das Internet -- obschon auch hier bereits
diverse
\href{https://www.heise.de/tp/features/Facebook-Fake-News-und-die-Privatisierung-der-Zensur-3599878.html}{Zensurversuche}
zu beobachten sind.

\begin{center}\rule{0.5\linewidth}{\linethickness}\end{center}

\href{https://swprs.org/2017/03/01/zensur-in-schweizer-medien/}{**1.
March 2017}

\hypertarget{die-partnerschaft-mit-der-nato}{%
\section{\texorpdfstring{\href{https://swprs.org/2017/03/01/schweizer-medien-nato/}{Die
Partnerschaft mit
der~NATO}}{Die Partnerschaft mit der~NATO}}\label{die-partnerschaft-mit-der-nato}}

\href{https://swprs.org/2017/03/01/schweizer-medien-nato/}{\includegraphics{https://swprs.files.wordpress.com/2016/07/nato-logo-3s.png?w=305}}

Die Schweiz ist nicht Mit­glied in der NATO, trat jedoch 1996 der
\emph{\href{http://www.pfp.admin.ch/}{»NATO Partner­ship for Peace«}}
und 1997 dem
\emph{\href{http://www.nato.int/docu/review/2007/issue2/german/art5.html}{Euro-Atlan­tischen
Par­tner­schafts­rat}} bei -- je­weils ohne Volks­ab­stimmung.

Seit­dem kommt das Schweizer Militär im Zuge von NATO-Inter­­ven­­tionen
zum \href{https://www.peace-support.ch/de/}{Einsatz}, so im Kosovo, in
Bosnien und in Afgha­ni­stan (ISAF). Auch der Schweizer
Nach­richten­dienst (NDB) wird inzwischen von einem
\href{https://www.admin.ch/gov/de/start/dokumentation/medienmitteilungen.msg-id-70400.html}{General}
geführt, der durch die NATO ausgebildet wurde.

Würden Schweizer Medien trotz NATO-Part­ner­schaft allzu kritisch über
Interventionen der US-Allianz berichten, so könnte dies als
\href{https://swprs.org/russische-propaganda/}{»feind­li­che
Pro­pa­gan­da«} ge­wer­tet werden -- was po­li­tisch und ökonomisch
wenig opportun wäre.

Auf diese Weise ergibt sich eine weitgehend
\href{https://swprs.org/medien-navigator/}{NATO-kon­forme} Darstellung
von geopolitischen Kon­flik­ten, so in Jugoslawien, Afgha­ni­stan, Irak,
Li­by­en, Syrien, Jemen oder der Ukraine.

An wirtschaftlichen Sanktionen muss sich die Schweiz auf Wunsch der USA
schon seit 1951
\href{https://de.wikipedia.org/wiki/Hotz-Linder-Agreement}{be­tei­li­gen}.
Jour­na­listen, die diese Ver­letzung der Neu­tra­lität damals
kri­ti­sierten,
\href{https://web.archive.org/web/20141206061445/http://buchundnetz.com/online-buch/schnueffelstaat-schweiz-ob/iii-modernisieren-oder-abschaffen/staatsschutz-je-nach-wetterlage/}{er­hielten}
15 Mo­nate Gefäng­nis wegen Landes­verrats.

\begin{center}\rule{0.5\linewidth}{\linethickness}\end{center}

\href{https://swprs.org/2017/03/01/schweizer-medien-nato/}{**1. March
2017}

\hypertarget{die-nzz-und-das-new-american-century}{%
\section{\texorpdfstring{\href{https://swprs.org/2017/03/01/nzz-new-american-century/}{Die
NZZ und das New
American~Century}}{Die NZZ und das New American~Century}}\label{die-nzz-und-das-new-american-century}}

\href{https://swprs.org/2017/03/01/nzz-new-american-century/}{\includegraphics{https://swprs.files.wordpress.com/2016/03/pnac.png?w=600}}

2010 schrieb der heutige *NZZ-*Chef­re­dakteur und vor­ma­lige
*NZZ-*Aus­lands­chef Eric Gujer ein
\href{https://www.amazon.com/Safety-Liberty-Islamist-Terrorism-Counterterrorism/dp/084474333X}{Buch}
über den \emph{War on Terror} zu­sammen mit
\href{https://en.wikipedia.org/wiki/Gary_Schmitt}{Gary J. Schmitt}, dem
ehe­ma­li­gen Dir­ektor des \emph{Project for the New American Century
(\href{https://en.wikipedia.org/wiki/Project_for_the_New_American_Century}{PNAC}).}

PNAC wurde 1997 von füh­ren­den Neo­kon­ser­va­ti­ven ge­gründet,
darunter \href{https://de.wikipedia.org/wiki/Dick_Cheney}{Dick Cheney}.
Die Gruppe for­derte die weltweite Prädominanz der USA und
anti­zi­pierte be­reits im Jahre 2000 in ei­nem
\href{https://web.archive.org/web/20130817122719/http://www.newamericancentury.org/RebuildingAmericasDefenses.pdf}{Stra­tegie­­papier}
ein »neues Pearl Harbor« als Legi­ti­ma­tion für die globale
US-Offensive.*\\
*

*NZZ-*Chef­redakteur und
\href{http://bazonline.ch/schweiz/Ein-Atlantiker-an-der-Spitze/story/18216373}{»Atlan­ti­ker«}
Gujer pf‌legte zudem \href{https://www.taz.de/!430263/}{Kon­takte} zu
mehreren
\href{https://web.archive.org/web/20150515195718/http://www.schweizamsonntag.ch/ressort/medien/nzz-chefredaktor_gujer_und_der_geheimdienst/}{Ge­heim­dien­s­ten}
-- Edward Snow­den ist für ihn denn auch kein \emph{Whistle­blower,}
sondern ein
\href{http://www.nzz.ch/schweiz/bern-ist-nicht-bagdad-1.18122326}{»Ver­rä­ter«}.

Vermag die \emph{NZZ} den­noch ob­jek­tiv über das Welt­ge­sche­hen zu
berichten? Diese Frage wur­de im Rah­men der
\emph{\href{https://swprs.org/die-nzz-studie/}{NZZ-Studie}} untersucht.

\begin{center}\rule{0.5\linewidth}{\linethickness}\end{center}

\href{https://swprs.org/2017/03/01/nzz-new-american-century/}{**1. March
2017}

\hypertarget{srf-die-propaganda-analyse}{%
\section{\texorpdfstring{\href{https://swprs.org/2017/03/01/srf-propaganda-analyse/}{SRF:
Die
Propaganda-Analyse}}{SRF: Die Propaganda-Analyse}}\label{srf-die-propaganda-analyse}}

\href{https://swprs.org/2017/03/01/srf-propaganda-analyse/}{\includegraphics{https://swprs.files.wordpress.com/2016/10/srf-analyse-s.png?w=500}}

Das Schweizer Radio und Fern­se­hen (SRF) leistet mit seinen
Nach­rich­ten- und In­for­ma­tions­sen­dungen einen wich­tigen Bei­trag
zur öffent­lichen Meinungs­bildung in der Schweiz. Doch wie objektiv und
kritisch be­rich­tet das SRF über geo­po­li­tische The­men?

Um dies zu über­prü­fen, wurde erst­mals eine sys­te­ma­tische Ana­lyse
der SRF-​Be­richt­er­stat­tung zu einem geo­po­li­tisch relevanten
Ereig­nis durch­ge­führt.

Die Resul­tate sind alar­mie­rend: In allen unter­such­ten Bei­trä­gen
des SRF wurden Pro­pa­ganda- und Mani­pu­la­tions­tech­niken auf
re­dak­tio­nel­ler, sprach­licher und audio­vi­su­el­ler Ebene
fest­ge­stellt.

\href{https://swprs.org/srf-propaganda-analyse/}{Zur SRF
Propaganda-Analyse →}

\begin{center}\rule{0.5\linewidth}{\linethickness}\end{center}

\href{https://swprs.org/2017/03/01/srf-propaganda-analyse/}{**1. March
2017}

\hypertarget{das-gewuxfcnschte-narrativ}{%
\section{\texorpdfstring{\href{https://swprs.org/2017/03/01/das-gewuenschte-narrativ/}{Das
gewünschte
Narrativ}}{Das gewünschte Narrativ}}\label{das-gewuxfcnschte-narrativ}}

\href{https://swprs.org/2017/03/01/das-gewuenschte-narrativ/}{\includegraphics{https://swprs.files.wordpress.com/2016/02/medien-narrativ1.png?w=400}}

Bei geopolitischen Konflikten bestehen oftmals vordefinierte mediale
Narrative. Was geschieht, wenn ein Schweizer Jour­na­list davon abweicht
und über die »falschen« Themen be­richtet?

Heute kaum noch vorstellbar, doch mitten im
\href{https://de.wikipedia.org/wiki/Bosnienkrieg}{Bosnien­krieg}
(1992-95) veröffentlichte der damalige Aus­lands­chef der
\emph{Welt­woche} einen Artikel zu Kriegs­lügen in west­lichen Medien.

Daraufhin geschah Folgendes:

\href{https://swprs.org/das-gewuenschte-narrativ\#weiterlesen}{Weiterlesen
→}

\begin{center}\rule{0.5\linewidth}{\linethickness}\end{center}

\href{https://swprs.org/2017/03/01/das-gewuenschte-narrativ/}{**1. March
2017}

\hypertarget{die-konferenz}{%
\section{\texorpdfstring{\href{https://swprs.org/2017/03/01/schweizer-medien-bilderberg-konferenz/}{Die
Konferenz}}{Die Konferenz}}\label{die-konferenz}}

\href{https://swprs.org/2017/03/01/schweizer-medien-bilderberg-konferenz/}{\includegraphics{https://swprs.files.wordpress.com/2016/02/bilderberg_2011.png?w=440}}

Die großen Schweizer Medien­­häuser sind in geo­poli­tische
\href{https://swprs.org/netzwerk-medien-schweiz/}{Netz­werke}
ein­ge­bun­den: So nehmen die wichtigsten Schweizer Verleger und
Chef­redakteure im Turnus an der jähr­lichen
\href{http://www.bilderbergmeetings.org/}{Bilderberg-Konferenz} teil, wo
sie im privaten Rahmen auf die trans­atlan­tische Elite aus
Wirt­schaf‌t, Politik und Militär treffen.

Teilnehmer seit 1991 (siehe
\href{https://swprs.org/netzwerk-medien-schweiz/}{Info­grafik}):

\includegraphics{https://swprs.files.wordpress.com/2017/03/teilnehmer-bilderberg-ch-1.png?w=736}

Auch der journa­lis­tische Nach­wuchs wird ge­för­dert: Sowohl der
\href{http://www.americanswiss.org/news/arthur-honegger-spotlight/}{*10vor10-*​Mode­ra­tor}
des SRF wie auch der
\href{http://www.americanswiss.org/news/niklaus-nuspliger-spotlight/}{*NZZ-*Korres­pon­dent}
für die EU \& NATO wurden von der
\href{http://www.americanswiss.org/}{\emph{Ameri­can Swiss
Foun­da­tion}} zu »Young Leaders« ernannt -- und neh­men in dieser Rolle
an
\href{http://www.americanswiss.org/ambassador-barras-hosts-dinner-for-young-leaders-1/}{exklu­siven
Dinners} mit hoch­rang­igen US-Ver­tre­tern teil.

\emph{Foto:} Bilder­berg-Meeting
\href{https://www.theguardian.com/world/gallery/2011/jun/15/bilderberg-in-pictures}{2011}
in St. Moritz.

\begin{center}\rule{0.5\linewidth}{\linethickness}\end{center}

\href{https://swprs.org/2017/03/01/schweizer-medien-bilderberg-konferenz/}{**1.
March 2017}

\hypertarget{warum-der-tagi-nichts-verpasst}{%
\section{\texorpdfstring{\href{https://swprs.org/2017/03/01/warum-der-tagesanzeiger-nichts-verpasst/}{Warum
der Tagi
nichts~verpasst}}{Warum der Tagi nichts~verpasst}}\label{warum-der-tagi-nichts-verpasst}}

\href{https://swprs.org/2017/03/01/warum-der-tagesanzeiger-nichts-verpasst/}{\includegraphics{https://swprs.files.wordpress.com/2016/06/lena-logo2.png?w=440}}

Ob Ukraine, Syrien oder Chi­na: Der \emph{Zürcher Tages-Anzeiger}
schreibt viele seiner Aus­lands­berichte nicht mehr selbst, sondern
bezieht sie im Rah­men einer
\href{https://www.tagesanzeiger.ch/schweiz/standard/In-eigener-Sache/story/24648194}{»umfassenden
Ko­ope­ra­tion«} von der \emph{Süd­deut­schen Zeitung.}

Deren Außen­politik­chef
\href{https://swprs.org/netzwerk-medien-deutschland/}{zählt} indes zu
den bekanntesten Trans­at­lan­tikern Deutsch­lands -- und ent­spre­chend
le­sen sich die Arti­kel im \emph{Tagi.} Aus dem ara­bi­schen Raum
\href{https://web.archive.org/web/20170606085220/http://www.icfj.org/sites/default/files/Kr\%C3\%BCger.pdf}{berichtet}
z.B. ein Absol­vent des ameri­ka­ni­schen
\emph{Arthur-F.-Burns-Fellowship}, aus Mos­kau ein
\href{https://spiegelkabinett-blog.blogspot.com/2016/09/julian-hans-von-der-suddeutschen.html}{Ab­gänger}
der \emph{Henri-Nannen-Schule}.

Über den Onlinedienst
\emph{\href{https://de.wikipedia.org/wiki/Newsnet}{Newsnet}} werden
Aus­lands­be­rich­te des \emph{Tagi} zudem an andere Schwei­zer
Zei­tungen wei­ter­ge­reicht. Auf diese Weise
\href{http://www.tagesanzeiger.ch/ausland/europa/Den-Ausloeser-zum-Krieg-habe-ich-gedrueckt/story/16330278}{er­scheinen}
Beiträge der \emph{Süd­deutschen Zeitung} via \emph{Tages­-Anzeiger} und
\emph{Newsnet} zu­sätz­lich im \emph{Berner Bund} und der \emph{Basler
Zeitung.}

Seit 2015 ist der \emph{Tages­-Anzeiger} über­dies Teil der
\emph{\href{https://de.wikipedia.org/wiki/Leading_European_Newspaper_Alliance}{Leading
European News­paper Alliance} (LENA).} Zweck des Ver­bunds ist die
``Ent­wick­lung und der Aus­tausch re­dak­tio­neller In­hal­te'' mit
anderen LENA- Mit­glie­dern wie \emph{Le Fi­g­aro}, \emph{Die Welt},
\emph{El País} oder \emph{La Re­pub­blica}.

Alle diese Zei­tungen sind in das
\href{https://swprs.files.wordpress.com/2016/07/bilderberg_teilnehmer_1954-2014.pdf}{Bilder­berg-Netz­werk}
ein­ge­bun­den -- kann es da noch über­raschen, dass auch
LENA- Repor­ta­gen zumeist auf
\href{http://www.tagesanzeiger.ch/ausland/europa/Wer-sagt-was-er-denkt-nobrriskiert-allesnobr/story/17225010}{trans­at­lan­tischer
Linie} sind?

\begin{center}\rule{0.5\linewidth}{\linethickness}\end{center}

\href{https://swprs.org/2017/03/01/warum-der-tagesanzeiger-nichts-verpasst/}{**1.
March 2017}

\hypertarget{eine-bruxfccke-uxfcber-den-atlantik}{%
\section{\texorpdfstring{\href{https://swprs.org/2017/03/01/eine-bruecke-ueber-den-atlantik/}{Eine
Brücke über
den~Atlantik}}{Eine Brücke über den~Atlantik}}\label{eine-bruxfccke-uxfcber-den-atlantik}}

\href{https://swprs.org/2017/03/01/eine-bruecke-ueber-den-atlantik/}{\includegraphics{https://swprs.files.wordpress.com/2016/07/atlantikbruecke-logo.png?w=600}}

Der deutsche Medien­gigant
\href{https://de.wikipedia.org/wiki/Axel_Springer_SE}{Axel Springer}
(\emph{BILD}, \emph{Die Welt,} etc.) gewinnt auch in der Schweiz
zu­neh­mend an Einfluss. Bereits 1999 wurde die \emph{Handels­zeitung}
\href{https://de.wikipedia.org/wiki/Handelszeitung}{über­nommen}, 2007
\href{https://de.wikipedia.org/wiki/Jean_Frey_AG}{folgten} u.a. die
\emph{Bilanz} und der \emph{Beobachter}. 2014 wurde eine umfang­reiche
\href{http://www.blick.ch/news/wirtschaft/medien-ringier-und-axel-springer-gruenden-gemeinschaftsunternehmen-in-der-schweiz-id3357037.html}{Koope­ration}
mit *Blick-*Verleger Ringier publik, und 2015 der
\href{http://www.persoenlich.com/marketing/die-werbeallianz-prasentiert-sich-zum-ersten-mal-der-branche}{Ein­stieg}
in die Werbe­allianz \emph{Admeira} mit Ringier, Swiss­com und der SRG.

Als angel­säch­sische
\href{https://de.wikipedia.org/wiki/Lizenzzeitung}{Li­zenz­grün­dung}
von 1946 ist Axel Springer -- wie die meisten deutschen Leit­medien --
bis heute tief in den US-Macht­­struk­turen ver­wur­zelt. So war
Konzern­­chef Mathias Döpfner Mit­glied im
\href{https://www.cfr.org/global-board-advisors}{Bei­rat} des \emph{U.S.
Council on Foreign Relations (CFR),} und der lang­jährige *BILD-*​Chef
Kai Diek­­mann ist Vor­stands­mitglied der ame­ri­ka­treuen
\emph{\href{https://de.wikipedia.org/wiki/Atlantik-Br\%C3\%BCcke}{Atlantik-Brücke},}
in der viele der bekanntesten Medien­leute Deutsch­lands ver­ei­nigt
sind (siehe
\href{https://swprs.org/netzwerk-medien-deutschland/}{Infografik}).

Springer-Journa­listen sind zudem
\href{https://bildblog.de/89290/axel-springer-gibt-sich-neue-alte-grundsaetze/}{ver­­trag­­lich
ver­pfli­ch­tet}, das \emph{»trans­at­lantische Bündnis«} bzw. die
\emph{»Soli­da­rität mit den USA«} zu unter­stützen. So auch der heutige
\href{https://de.wikipedia.org/wiki/Roger_K\%C3\%B6ppel}{Ver­leger} der
\emph{Welt­woche}, der zuvor Chef­re­dakteur von Springers \emph{Welt}
war und den Irak­krieg noch 2004 mit diesen Worten ver­tei­digte:

\emph{``Die UNO schützt die Welt­ordnung nicht, zu deren Hüterin sie
sich irr­tüm­licher­weise erklärt. Sie ist im Gegen­teil das Derivat
eines Friedens, den ameri­ka­nische Truppen sichern'',} weshalb
\emph{``Europa auf die USA als hege­mon­ialer Hüter der west­lichen
`Welt­gewalt­ordnung' nicht ver­zichten kann''.}

\begin{center}\rule{0.5\linewidth}{\linethickness}\end{center}

\href{https://swprs.org/2017/03/01/eine-bruecke-ueber-den-atlantik/}{**1.
March 2017}

\hypertarget{der-medien-navigator}{%
\section{\texorpdfstring{\href{https://swprs.org/2017/03/01/der-medien-navigator/}{Der
Medien-Navigator}}{Der Medien-Navigator}}\label{der-medien-navigator}}

Wie sind deutsch­spra­chige Medien geo­po­li­tisch und the­matisch
posi­tio­niert? Für den Medien-Navi­gator wurden erst­mals 80
verschiedene Publi­ka­tionen unter­sucht.

\href{https://swprs.org/medien-navigator/}{\includegraphics{https://swprs.files.wordpress.com/2020/02/medien-navigator-2020-s.png?w=736}}

\href{https://swprs.org/medien-navigator/}{Zum Medien-Navigator →}

\begin{center}\rule{0.5\linewidth}{\linethickness}\end{center}

\href{https://swprs.org/2017/03/01/der-medien-navigator/}{**1. March
2017}

\hypertarget{das-transatlantik-netzwerk}{%
\section{\texorpdfstring{\href{https://swprs.org/2017/03/01/das-netzwerk/}{Das
Transatlantik-Netzwerk}}{Das Transatlantik-Netzwerk}}\label{das-transatlantik-netzwerk}}

Wie sind Schweizer Medien in trans­at­lantische Netz­werke
ein­ge­bunden? Welche Personen, Organi­sa­tionen und Kon­fe­ren­zen sind
von Bedeutung? Unsere Info­grafik gibt Auskunft.

\href{https://swprs.org/netzwerk-medien-schweiz}{\includegraphics{https://swprs.files.wordpress.com/2019/10/medien-netzwerk-schweiz-hdz-s.png?w=736}}

\href{https://swprs.org/netzwerk-medien-schweiz}{Zur Infografik →}

\begin{center}\rule{0.5\linewidth}{\linethickness}\end{center}

\href{https://swprs.org/2017/03/01/das-netzwerk/}{**1. March 2017}

\hypertarget{die-nzz-studie}{%
\section{\texorpdfstring{\href{https://swprs.org/2017/03/01/die-nzz-studie/}{Die
NZZ-Studie}}{Die NZZ-Studie}}\label{die-nzz-studie}}

\href{https://swprs.org/2017/03/01/die-nzz-studie/}{\includegraphics{https://swprs.files.wordpress.com/2017/03/nzz-propaganda-gesamt-small.png?w=500}}

Die \emph{Neue Zürcher Zeitung} ist das Flagg­schiff unter den Schweizer
Tages­zei­tungen. Doch wie objektiv und kritisch berichtet die
\emph{NZZ} über geo­politische Konf‌likte?

Um dies zu be­ant­worten, wurde die Bericht­erstattung der \emph{NZZ}
zur Ukraine-Krise und zum Syrien­krieg während je eines Monats
unter­sucht. Die Ergebnisse sind eindeutig.

\href{https://swprs.org/die-nzz-studie/}{Zur NZZ-Studie →}

\begin{center}\rule{0.5\linewidth}{\linethickness}\end{center}

\href{https://swprs.org/2017/03/01/die-nzz-studie/}{**1. March 2017}

\hypertarget{abschied-von-usa-kritikern}{%
\section{\texorpdfstring{\href{https://swprs.org/2017/03/01/abschied-von-usa-kritikern/}{Abschied
von
USA-Kritikern}}{Abschied von USA-Kritikern}}\label{abschied-von-usa-kritikern}}

\href{https://swprs.org/2017/03/01/abschied-von-usa-kritikern/}{\includegraphics{https://swprs.files.wordpress.com/2016/07/hummler-nzz.png?w=348}}

Februar 2012: Der US-kritische Banquier Konrad Hummler muss nach nur
einem Jahr im Amt als \emph{NZZ}-Präsi­dent
\href{https://www.tagesanzeiger.ch/wirtschaft/unternehmen-und-konjunktur/Konrad-Hummler-gibt-NZZVRPraesidium-ab/story/25627682}{zurück­treten}.
Seine Privatbank \emph{Wegelin} wurde von den USA
\href{https://www.tagesanzeiger.ch/wirtschaft/unternehmen-und-konjunktur/USA-erhoehen-mit-WegelinKlage-den-Druck/story/25658973}{an­ge­klagt}und
zerschlagen.

Für Schweizer Banken wie Journalisten war dies ein kaum über­hör­ba­rer
\href{http://www.nzz.ch/ein-weckruf-fuer-die-schweiz-1.14608280}{»Weck­ruf«}
(O-Ton \emph{NZZ}). Denn Hummler hatte nicht nur bank­politisch, sondern
auch publi­zistisch die »rote Linie« über­schritten, als er sich in
einem viel­beachteten An­la­ge­kom­men­tar mit dras­ti­schen Worten zur
Po­li­tik der USA äußerte und einen
\href{https://swprs.files.wordpress.com/2016/03/usa_wegelin_kommentar_2009.pdf}{»Abschied
von Amerika«} forderte.

Viele Schwei­zer Banken und Konzerne -- und da­mit Ar­beits­plätze,
Steuer- und Werbe­ein­nah­men -- hän­gen vom
\href{http://www.finews.ch/news/finanzplatz/21359-us-steuerstreit-kategorie-2-kategorie-1-lombard-odier-julius-b\%C3\%A4r-department-of-justice}{Good­will}
der USA ab. Ob Schweizer Ver­le­ger und Chef­re­dak­teure diesen aufs
Spiel setzen wollen?

\begin{center}\rule{0.5\linewidth}{\linethickness}\end{center}

\href{https://swprs.org/2017/03/01/abschied-von-usa-kritikern/}{**1.
March 2017}

\hypertarget{der-propaganda-multiplikator}{%
\section{\texorpdfstring{\href{https://swprs.org/2017/03/01/propaganda-multiplikator/}{Der
Propaganda-Multiplikator}}{Der Propaganda-Multiplikator}}\label{der-propaganda-multiplikator}}

\href{https://swprs.org/2017/03/01/propaganda-multiplikator/}{\includegraphics{https://swprs.files.wordpress.com/2016/12/propaganda-multiplikator-300.png?w=346}}

Es ist einer der wichtigsten Aspekte unseres Medien­systems -- und
dennoch in der Öf‌fent­lich­keit nahezu unbekannt: Der größte Teil der
inter­na­tio­nalen Nach­rich­ten in all unseren Medien stammt von nur
drei glo­balen Nach­rich­ten­agen­turen aus~New York, London und Paris.

Die Schlüssel­rolle dieser Agen­turen be­wirkt, dass west­liche Medien
zu­meist über die glei­chen The­men be­richten und dabei sogar oftmals
dieselben For­mu­lie­rungen ver­wenden.

Zu­dem nutzen Re­gie­rungen, Mi­li­tärs und Ge­heim­dienste die
glo­balen Agen­turen als Mul­ti­pli­kator für die welt­weite
Ver­brei­tung ihrer Bot­schaf‌ten. Die trans­at­lan­tische Ver­netzung
der eta­blier­ten Medien ge­währ­leis­tet da­bei, dass die ge­wün­schte
Sicht­weise kaum hin­ter­fragt wird.

Eine Unter­suchung der Syrien-Bericht­er­stat­tung von je drei
füh­ren­den Tages­zei­tungen aus Deutsch­land, Öster­reich und der
Schweiz illus­triert diese Ef‌fekte deutlich.

\href{https://swprs.org/der-propaganda-multiplikator/}{Zur
Multiplikator-Studie →}

\begin{center}\rule{0.5\linewidth}{\linethickness}\end{center}

\href{https://swprs.org/2017/03/01/propaganda-multiplikator/}{**1. March
2017}

\hypertarget{das-gewuxfcnschte-narrativ-ii}{%
\section{\texorpdfstring{\href{https://swprs.org/2017/03/01/das-gewuenschte-narrativ-ii/}{Das
gewünschte
Narrativ~II}}{Das gewünschte Narrativ~II}}\label{das-gewuxfcnschte-narrativ-ii}}

\href{https://swprs.org/2017/03/01/das-gewuenschte-narrativ-ii/}{\includegraphics{https://swprs.files.wordpress.com/2017/03/zeitungen-schweiz.png?w=450}}

Im Dezember 2015 publi­zierte das News­portal \emph{Watson} (AZ Medien)
einen \href{https://www.watson.ch/!148360008}{Artikel} des lang­jährigen
Tages­schau-Kor­res­pon­denten Hel­mut Sche­­ben zum Syrien­krieg.
Scheben stellte den Krieg in einen geo­po­li­tischen Kontext und
kri­ti­sierte die westliche Be­richt­er­stattung als einseitig und
ma­ni­pu­la­tiv.

Der Artikel un­ter­schied sich deutlich von anderen Aus­lands­bei­trägen
auf \emph{Watson}, die meist vom deutsch-transatlantischen
\href{https://www.watson.ch/Corporate/articles/502582965-Spiegel-Online-und-watson-machen-gemeinsame-Sache}{Content
Partner} \emph{Spiegel Online} geliefert werden.

Keine zwei Tage später veröffentlichte \emph{Watson} jedoch einen
aufgebrachten \href{https://www.watson.ch/!491379853}{Rückruf}, in dem
sich das Portal vom Artikel distanzierte und Helmut Scheben wüst
beschimpfte: Man sei auf einen ``Putin-Troll'' herein­ge­fallen, der
wo­möglich in der ``russischen Propaganda-Maschinerie'' mit­wirke. Auch
Leser, die sich positiv zum ur­sprüng­lichen Artikel geäußert hatten,
wurden als »Trolle« verun­glimpft.

Wer oder was hat wohl hinter den Kulissen zu dieser selt­samen Reak­tion
geführt? Jeden­falls wurde den hiesigen Journa­listen damit einmal mehr
in Er­in­nerung gerufen: Wer sich in der Schweiz nicht an das
ge­wünschte Nar­ra­tiv hält, ris­kiert Ruf und Karriere.

\begin{center}\rule{0.5\linewidth}{\linethickness}\end{center}

\href{https://swprs.org/2017/03/01/das-gewuenschte-narrativ-ii/}{**1.
March 2017}

\hypertarget{propaganda-im-staatsauftrag}{%
\section{\texorpdfstring{\href{https://swprs.org/2017/03/01/propaganda-im-staatsauftrag/}{Propaganda
im
Staatsauftrag?}}{Propaganda im Staatsauftrag?}}\label{propaganda-im-staatsauftrag}}

\href{https://swprs.org/2017/03/01/propaganda-im-staatsauftrag/}{\includegraphics{https://swprs.files.wordpress.com/2016/02/srf-syrien11.png?w=600}}

Von den öf‌fentlichen Rund­funk­an­stalten er­war­tet das Pu­bli­kum
eine aus­ge­wogene Bericht­er­stattung. Doch of‌t ist ge­rade dort der
politische Druck be­sonders hoch, sich an das
\href{https://swprs.org/das-gewuenschte-narrativ/}{trans­at­lan­tische
Narra­tiv} zu halten.

So haben Mitarbeiter der \emph{ARD} gemäß internen Memos Weisung, bei
geo­po­li­tischen Kon­f‌lik­ten
\emph{\href{https://www.heise.de/tp/features/Ukraine-Konflikt-ARD-Programmbeirat-bestaetigt-Publikumskritik-3367400.html}{»west­liche
Posi­tionen zu ver­tei­di­gen«}}, ver­trau­liche
\href{https://www.heise.de/tp/features/Die-vertraulichen-Sprachregelungen-der-ARD-3758887.html}{Sprach­­re­­ge­lungen}
zu be­fol­gen und aus­­schließ­­lich
\href{https://www.oxmoxhh.de/magazin/story-interview/oxmox-exklusiv-interview-mit-volker-braeutigam-friedhelm-klinkhammer/}{konforme
Quellen} zu ver­wen­den.

Beim \emph{ZDF} machte der ehe­ma­lige Chef­re­dakteur publik, dass
Bei­träge zu US-Kriegen poli­tisch
\href{https://www.youtube.com/watch?v=i2423aDq_hE}{be­ein­f‌‌lusst}
werden. Nahost-Kor­res­pon­dent Ulrich Tilgner be­klagte
re­dak­tio­nelle Ein­grif‌fe aufgrund von
\href{http://www.berliner-zeitung.de/korrespondent-ulrich-tilgner-sucht-mehr-distanz-zum-zdf--ich-fuehle-mich-eingeschraenkt--15870684}{»Bünd­nis­rück­sich­ten«},
und der vormalige Leiter des *ZDF-*Studios Bonn be­stä­tig­te
\href{https://propagandaschau.wordpress.com/2016/01/30/wolfgang-herles-es-gibt-in-den-oeffentlich-rechtlichen-anweisungen-von-oben/}{»An­wei­sungen
von oben«} und eine
\href{http://www.rolandtichy.de/daili-es-sentials/meinungsfreiheit-anordnung-zur-anpassung/}{»frei­willige
Gleich­schal­tung«} der Jour­na­lis­ten.

Auch das \emph{SRF} verwendet diverse
\href{https://swprs.org/srf-propaganda-analyse/}{Mani­pu­lations­tech­niken}
zugunsten der Konflikt­partei USA \& NATO und
\href{http://www.srf.ch/sendungen/srfglobal/propagandagruesse-aus-moskau-2}{thematisiert}
Propaganda stets nur auf der Gegenseite. Selbst vor dem Einsatz
sub­tiler
\href{http://www.srf.ch/play/tv/10vor10/video/warum-assad-bleibt?id=a6d267c9-52b3-470b-868e-95bb919a0b96}{Grusel­musik}
in den Nach­rich­ten schreckt das \emph{SRF} nicht zurück, um Gegner der
US-Allianz zu dämo­ni­sieren.

Programmbe­schwer­den sind indes chan­cen­los, denn: Beiträge zu
inter­na­tio­nalen Kon­flik­ten müssten
\emph{\href{https://swprs.org/srf-ombudsstelle-im-faktencheck/}{``weder
neutral noch ausgewogen''}} sein, und \emph{``die­je­ni­gen, die dem SRF
vor­wer­fen, ein­sei­tig der US- und Nato-Pro­pa­gan­da zu er­lie­gen,
be­trei­ben ihrer­seits das Ge­schäf‌t der russischen Pro­pa­ganda''} --
so die erstaun­liche
\href{https://swprs.org/srf-ombudsstelle-im-faktencheck/}{Ar­gu­men­ta­tion}
der Om­buds­stelle.*\\
*

\begin{center}\rule{0.5\linewidth}{\linethickness}\end{center}

\href{https://swprs.org/2017/03/01/propaganda-im-staatsauftrag/}{**1.
March 2017}

\hypertarget{der-korrespondent}{%
\section{\texorpdfstring{\href{https://swprs.org/2017/03/01/der-korrespondent/}{Der
Korrespondent}}{Der Korrespondent}}\label{der-korrespondent}}

\href{https://swprs.org/2017/03/01/der-korrespondent/}{\includegraphics{https://swprs.files.wordpress.com/2017/03/srf-gsteiger-nato.jpg?w=600}}

Wie wird man Kor­res­pon­dent beim \emph{Schwei­zer Radio und
Fern­sehen}? Fredy Gsteiger muss es wissen: Er ist
\href{http://www.persoenlich.com/medien/fredy-gsteiger-neu-in-der-radio-chefredaktion-232921}{stv.
Chef­redakteur}, Auslands­chef und
\href{http://www.srf.ch/radio-srf-1/radio-srf-1/fredy-gsteiger-unser-mann-in-der-uno}{diplo­ma­tischer
Korres­pon­dent} des \emph{Schwei­zer Radios SRF}. In dieser Funktion
be­richtet er etwa über die UNO, NATO und EU -- und damit z.B. auch über
\href{http://www.srf.ch/news/international/dieser-eu-rueckzieher-ist-peinlich}{Russ­land-Sanktionen}
und die Genfer
\href{http://www.srf.ch/news/international/assad-kommt-mit-giftgaseinsaetzen-vorlaeufig-davon}{Syrien-Ver­hand­lungen}.

Gsteiger begann seine journa­lis­tische Lauf­bahn Ende der 80er Jahre
als Nahost-Redakteur bei der
\href{https://swprs.org/netzwerk-medien-deutschland/}{deutsch-trans­atlan­tischen}
Wochen­zeitung \emph{Die Zeit}. Die Schwei­zer Neutra­lität war für ihn
schon vor dem Ersten Irak­krieg 1991 ein
\emph{\href{http://www.zeit.de/1990/44/ein-konzept-von-gestern}{»Konzept
von gestern«},} wirt­schaft­liche Neutralität ohnehin
\emph{\href{http://www.zeit.de/1990/44/ein-konzept-von-gestern}{»gänz­lich
über­holt«}.} Von 1997 bis 2001 war Gsteiger dann Chef­redakteur bei der
\emph{Welt­woche}. Unter seiner Leitung trat das Blatt »\emph{für den
Bei­tritt der Schweiz zur NATO«} ein, wie er in seinem
\href{https://web.archive.org/web/20040722094101/http://www.weltwoche.ch/artikel/?AssetID=400\&CategoryID=60}{Abschieds­artikel}
schrieb.

Damit kam Gsteiger 2002 zum Schweizer Radio. Be­schwer­den über eine
ein­sei­tige Be­richt­er­stattung wurden von der Ombuds­stelle mehr­fach
\href{https://www.srgd.ch/de/aktuelles/news/2016/09/28/sendung-info-3-auf-radio-srf-3-uber-waffenruhe-syrien-beanstandet/}{abge­lehnt}.
Und so
\href{http://www.swissinfo.ch/ger/kooperation_die-nato-umwirbt-die-schweiz/42225918}{be­tont}
Gsteiger auch heute noch die »\emph{vielen
Koope­ra­tions­möglich­keiten«} mit der NATO;
\href{http://www.srf.ch/news/international/dieser-eu-rueckzieher-ist-peinlich}{be­dauert},
dass die Russ­land-Sanktionen nicht ver­schärft werden; und
\href{http://www.srf.ch/news/international/assad-kommt-mit-giftgaseinsaetzen-vorlaeufig-davon}{weiß
genau}, wer in Syrien der Böse­wicht ist.

\emph{Update:} 2019 erhält das SRF eine neue
\href{https://www.srgd.ch/de/aktuelles/news/2017/12/05/luzia-tschirky-wird-neue-russland-korrespondentin/}{Russland-Korrespondentin}
-- die zuvor beim amerikanischen
\href{https://de.wikipedia.org/wiki/Radio_Free_Europe}{\emph{Radio Free
Europe}} arbeitete. (\emph{Foto oben:} Gsteiger 2014 auf einer
Jour­na­­listen-​Tour der US NATO-Mission.)

\begin{center}\rule{0.5\linewidth}{\linethickness}\end{center}

\href{https://swprs.org/2017/03/01/der-korrespondent/}{**1. March 2017}

\hypertarget{der-kriegsreporter}{%
\section{\texorpdfstring{\href{https://swprs.org/2017/03/01/der-kriegsreporter/}{Der
Kriegsreporter}}{Der Kriegsreporter}}\label{der-kriegsreporter}}

\href{https://swprs.org/2017/03/01/der-kriegsreporter/}{\includegraphics{https://swprs.files.wordpress.com/2016/11/pelda-syrien.jpg?w=600}}

Wie wird man in den Schweizer Medien zum »Nahost-Experten«? Kurt Pelda
muss es wissen: Von der \emph{Welt­woche} bis zum \emph{Schwei­zer
Fern­se­hen} ist er der Mann, der die Ereig­nisse in Sy­ri­en und Irak
für das Publi­kum
\href{http://www.srf.ch/news/international/assad-ist-nur-noch-an-der-macht-weil-er-so-brutal-ist}{»ein­ord­nen«}
darf.

Pelda
\href{https://www.youtube.com/watch?v=dtV25eIECKY}{be­glei­tete}schon in
den 80er Jahren als junger Journa­list die Mudschahedin im von den USA
\href{https://www.voltairenet.org/article165889.html}{lancier­ten} Krieg
gegen die afgha­nische Regie­rung, die mit Moskau verbün­det war. Nach
Sta­tionen bei der \emph{Financial Times} und der \emph{NZZ} bereist er
heute als freier Journa­list erneut Kriegs­ge­biete -- wie damals meist
nur
\href{https://tageswoche.ch/politik/ein-basler-im-syrischen-kampfgebiet/}{auf
Seiten} der US-unter­stützten Milizen.

Ist diese Ein­seitig­keit ein Pro­blem? Nicht für Pelda, denn er sei
schließ­lich -- so erklärte er in einem
\href{https://www.tageswoche.ch/de/2014_36/international/667493/}{Interview}
-- ein »Mei­nungs­jour­na­list« und »kein objek­ti­ver Be­obach­ter«,
wes­wegen Neutra­li­tät für ihn »keine Option« ist; viel­mehr gehe es
ihm um »gute Ge­schich­ten«, für die die Medien zu zahlen be­reit sind.
Wer in diesen Ge­schich­ten die Guten sind -- und wer
\href{http://www.srf.ch/news/international/assad-ist-nur-noch-an-der-macht-weil-er-so-brutal-ist}{die
Bösen} -- dürf‌te dabei niemanden über­raschen.

Mit diesem Ansatz wurde Pelda 2014 zum
\href{http://www.srf.ch/news/panorama/kurt-pelda-ist-journalist-des-jahres}{»Jour­na­list
des Jahres«} gekürt. Andere Nahost-Ken­ner, denen Objek­ti­vi­tät und
Neutra­lität wich­ti­ger sind als eine »gute Ge­schichte«, kommen in
Schwei­zer Medien indes
\href{https://swprs.org/das-gewuenschte-narrativ-ii/}{kaum noch} zu
Wort. Statt »ein­ge­ordnet« wurde hier -- aus­sor­tiert.

\emph{Foto oben:} Pelda in Syrien.
\emph{(\href{https://tageswoche.ch/politik/ein-basler-im-syrischen-kampfgebiet/}{TW})}

\begin{center}\rule{0.5\linewidth}{\linethickness}\end{center}

\href{https://swprs.org/2017/03/01/der-kriegsreporter/}{**1. March 2017}

\hypertarget{iduxe9e-suisse}{%
\section{\texorpdfstring{\href{https://swprs.org/2017/03/01/srg-idee-suisse/}{Idée
suisse}}{Idée suisse}}\label{iduxe9e-suisse}}

\href{https://swprs.org/2017/03/01/srg-idee-suisse/}{\includegraphics{https://swprs.files.wordpress.com/2016/07/srg-logo-1.png?w=350}}

Groß war der Auf­schrei in den Schweizer Medien, als Polen 2016 ein
\href{http://www.nzz.ch/international/europa/wie-medien-zu-nationalen-kulturinstituten-werden-1.18670792}{neues
Medien­ge­setz} erließ, welches die Er­nennung von Di­rek­toren des
öffent­lichen Rundfunks der Regierung übertrug. Doch wie un­ab­hängig
sind die öffentlichen Medien in der Schweiz?

Die Realität ist er­nüch­ternd: Obschon die \emph{Schwei­ze­rische
Radio- und Fern­seh­ge­sell­schaft (SRG)} gerne betont, dass sie als
\href{https://web.archive.org/web/20190412225655/https://www.srginsider.ch/service-public/2013/10/30/warum-ist-der-ausdruck-staatsfernsehen-oder-oeffentlich-rechtlicher-sender-falsch/}{privater
Verein} orga­ni­siert ist,
\href{https://www.srgd.ch/de/aktuelles/news/2016/11/04/srg-konzession-weiterhin-den-handen-des-bundesrats/}{definiert}
der Bundesrat nicht nur die Sendekonzession, sondern
\href{http://www.srgssr.ch/de/srg/organe/verwaltungsrat/}{ernennt} auch
meh­rere Ver­wal­tungs­rats­mit­glieder sowie
\href{https://www.ubi.admin.ch/}{alle} Mit­glieder der obersten
Pro­gramm­auf­sicht UBI.

Selbst der SRG-Präsi­dent wurde bis 2012 offiziell von der
Landesregierung
\href{https://web.archive.org/web/20150919041519/http://www.srgssr.ch/fileadmin/pdfs/Vereinsgeschichte_SRG.pdf}{be­stimmt}.
Seit­her kommt ein un­durch­sich­tiges Pro­ce­dere zum Ein­satz, bei dem
das Minis­terium vorab über die Kan­di­daten
\href{http://www.tagesanzeiger.ch/schweiz/standard/Neuer-SRGPraesident-verzweifelt-gesucht/story/18371394}{»infor­miert«}
wird. Dabei wurde das An­for­de­rungs­profil sowohl bei der
\href{http://www.aargauerzeitung.ch/schweiz/srg-extrawurst-fuer-roger-de-weck-8808607}{Wahl
des General­di­rektors 2010} wie auch bei der
\href{http://www.nzz.ch/nzzas/nzz-am-sonntag/favorit-fuer-das-srg-praesidium-leuthard-will-cvp-freund-an-srg-spitze-ld.90097}{Wahl
des Prä­si­denten 2016} noch während des Ver­fahrens ange­passt -- und
in beiden Fällen letzt­lich ein
\href{http://www.aargauerzeitung.ch/schweiz/war-roger-de-weck-der-lieblingskandidat-von-moritz-leuenberger-8833796}{»Wunsch­kan­di­dat«}
des am­tie­renden Medien­mi­nis­ters
\href{http://www.nzz.ch/nzzas/nzz-am-sonntag/favorit-fuer-das-srg-praesidium-leuthard-will-cvp-freund-an-srg-spitze-ld.90097}{gewählt}.

\emph{Update:} Auch die neue SRF-Direktorin wurde 2018 in einem
erstaunlich
\href{http://www.kleinreport.ch/news/geheimloge-srg-intransparenz-bei-der-besetzung-der-srf-direktion-91015/}{intransparenten}
Ver­fah­ren bestimmt.

\begin{center}\rule{0.5\linewidth}{\linethickness}\end{center}

\href{https://swprs.org/2017/03/01/srg-idee-suisse/}{**1. March 2017}

\hypertarget{die-srf-rundschau-hinterfragt}{%
\section{\texorpdfstring{\href{https://swprs.org/2017/03/01/srf-rundschau/}{Die
SRF-Rundschau
hinterfragt}}{Die SRF-Rundschau hinterfragt}}\label{die-srf-rundschau-hinterfragt}}

\href{https://swprs.org/2017/03/01/srf-rundschau/}{\includegraphics{https://swprs.files.wordpress.com/2018/07/rundschau.png?w=500}}

Die
\emph{\href{https://de.wikipedia.org/wiki/Rundschau_(SRF)}{Rundschau}}
ist das bekannteste Polit­ma­ga­zin des \emph{Schweizer Fernsehens.} Sie
\href{https://www.srf.ch/sendungen/rundschau/50-jahre-rundschau-die-jubilaeumssendung}{möchte}
»die Mächtigen hinterfragen« -- doch geht's um Geo­po­litik, so steht
sie meist selbst auf deren Seite.

Während die
\href{https://www.srf.ch/sendungen/rundschau/subventionierte-piloten-vaeter-am-limit-bombenhoelle-aleppo}{»Bombenhölle«}
Aleppo »fällt«, wird Mossul »befreit« -- von einem Familien­vater, der
gerne »US-Popmusik«
\href{https://www.srf.ch/sendungen/rundschau/buben-beschneidung-michel-bollag-pkb-west-mossul}{hört}.

Vom »Giftgasangriff« bei Ghouta
\href{https://www.srf.ch/sendungen/rundschau/kriminaltouristen-verhuetungsmittel-j-bitzer-giftgaseinsatz}{berichtet}
der »Augen­zeuge« einer »Hilfs­organisation« -- wer diese
\href{http://www.uossm.org/who_we_are}{finanziert}, verrät die
\emph{Rundschau} nicht.

Beim
\href{https://www.srf.ch/sendungen/rundschau/gehorsam-und-ehelos-klamauk-statt-kompromiss-vergessener-krieg}{»Vergessenen
Krieg«} im Jemen werden prompt die saudischen Luftangriffe »vergessen«
-- und deren westliche
\href{https://www.strategic-culture.org/news/2018/06/18/western-media-whitewash-yemen-genocide.html}{Unterstützung}
ebenso.

Putin indes hege
\href{https://www.srf.ch/sendungen/rundschau/gianni-infantino-fatma-samoura-iv-kosovaren-zittern-vor-russen}{»Expansionsgelüste«}
und füh­re einen
\href{https://www.srf.ch/sendungen/rundschau/putins-informationskrieg-milliarden-jongleur-bastos-camorra}{»Informationskrieg«},
seine Angriffe auf den Westen seien bereits »mehrfach be­legt« -- doch
statt Fakten
\href{https://www.srf.ch/sendungen/rundschau/putins-informationskrieg-milliarden-jongleur-bastos-camorra}{folgen}
finstere Sound­effekte.

Schon der Gründer der \emph{Rundschau} und spätere Leiter der
\emph{Tagesschau} nahm an der Konferenz der trans­atlantischen Elite
\href{https://wikileaks.org/plusd/cables/1978ZURICH00660_d.html}{teil}
-- ob man dort lernt, die Mächtigen zu hinterfragen?

\begin{center}\rule{0.5\linewidth}{\linethickness}\end{center}

\href{https://swprs.org/2017/03/01/srf-rundschau/}{**1. March 2017}

\hypertarget{der-schweizer-presserat}{%
\section{\texorpdfstring{\href{https://swprs.org/2017/03/01/der-schweizer-presserat/}{Der
Schweizer
Presserat}}{Der Schweizer Presserat}}\label{der-schweizer-presserat}}

\href{https://swprs.org/2017/03/01/der-schweizer-presserat/}{\includegraphics{https://swprs.files.wordpress.com/2016/07/presserat-logo.png?w=200}}

Der \href{https://presserat.ch/}{Schweizer Presse­rat} nimmt
Be­schwer­den zu Me­dien­be­rich­ten ent­ge­gen und prüft, ob die
Beiträge seinen
\href{https://presserat.ch/journalistenkodex/richtlinien/}{Richt­linien}
ent­spre­chen.

Aller­dings
\href{https://presserat.ch/der-presserat/presseratsmitglieder/}{besteht}
das Gre­mium selbst aus 15 Jour­na­listen und nur sechs
\emph{Pub­li­kums­ver­tre­tern} -- und auch diese werden von einem
\href{https://presserat.ch/der-presserat/stiftungsratsmitglieder/}{Stif‌­tungs­rat}
er­nannt, der gänz­lich von Medien­orga­ni­sa­tionen
\href{https://presserat.ch/der-presserat/geschaeftsreglement/}{kon­trol­liert}
wird.

Das Resultat ist naheliegend. Im Som­mer 2014 wurde etwa eine
Be­schwerde gegen die
\href{https://swprs.org/die-nzz-studie/}{no­to­risch ein­sei­tige}
Ukraine-Bericht­er­stattung der \emph{NZZ} ein­ge­legt. Ganze zwei Jahre
später kam der Presse­rat zu seinem
\href{https://presserat.ch/complaints/wahrheitspflicht-kommentarfreiheit-unterschlagen-wichtiger-informationen-entstellen-von-tatsachen/}{Verdikt}:
Die Rich­tig­keit der *NZZ- *Dar­stel­lung stehe \emph{»außer Frage«},
denn auf \emph{»amt­liche Ver­laut­ba­rungen und Agen­tur­mel­dungen«}
sei \emph{»Verlass«,} während russische Quel­len weder glaub­haf‌t noch
erforderlich wären; Kom­men­tare müss­ten nicht auf Fak­ten ba­sie­ren,
Ge­gen­mei­nungen ein­zu­holen sei \emph{»un­üb­lich«,} und an den
Aus­füh­rungen der \emph{NZZ} zu \emph{»Kreml- Trollen«} sei
\emph{»nicht zu zwei­feln«}. Be­schwerde ab­ge­lehnt.

Pikant: Einige der be­ur­teil­ten Ar­tikel stam­mten von einem
\href{http://www.nzz.ch/international/europa/beschwerde-beim-presserat-kritik-an-nzz-abgewiesen-ld.104814}{*NZZ-*Redak­teur},
der selbst im Stif‌­tungs­rat des Gremiums sitzt -- und inzwischen wurde
der damalige *NZZ-*Chef gar zu dessen
\href{http://www.nzz.ch/schweiz/medien-selbstregulierung-markus-spillmann-wird-praesident-des-presserats-ld.135619}{Prä­si­denten}
ernannt. Beim Presse­rat nennt man dies
\href{https://de.wikipedia.org/wiki/Schweizer_Presserat}{»Selbst­re­gu­lierung«\ldots{}}

\begin{center}\rule{0.5\linewidth}{\linethickness}\end{center}

\href{https://swprs.org/2017/03/01/der-schweizer-presserat/}{**1. March
2017}

\hypertarget{die-vertrauensfrage}{%
\section{\texorpdfstring{\href{https://swprs.org/2017/03/01/schweizer-medien-vertrauen/}{Die
Vertrauensfrage}}{Die Vertrauensfrage}}\label{die-vertrauensfrage}}

\href{https://swprs.org/2017/03/01/schweizer-medien-vertrauen/}{\includegraphics{https://swprs.files.wordpress.com/2017/03/foeg-jahrbuch_logo.png?w=500}}

Das Forschungs­institut für Öf­fent­lich­­keit und Gesell­schaft der
Uni­ver­sität Zürich publi­ziert all­jähr­lich das »Jahr­buch Qualität
der Medien«. 2016 ver­mel­dete das In­sti­tut, das Ver­trau­en in die
Schwei­zer Me­dien sei
\href{http://www.foeg.uzh.ch/dam/jcr:7234c6d3-1f09-4d36-b6ab-f14e659d046e/Medienmitteilung_JB_2016_dt.pdf}{»weiter­hin
hoch«} -- so das Er­geb­nis eines Länder­ver­gleichs in
Zu­sam­men­ar­beit mit dem \emph{\emph{Reu­ters Insti­tute.}}

Doch wie hoch ist das Vertrauen in die Schweizer Medien nun wirklich?
Dazu findet man in der Mit­tei­lung des Instituts keine An­ga­ben. Und
auch die
\href{http://www.tagesanzeiger.ch/schweiz/standard/Diese-Menschen-sind-anfaellig-fuer-Populisten/story/23804017}{Zei­tungs­be­richte}
zur Studie er­wäh­nen diese wich­tige Kenn­zahl
\href{http://www.nzz.ch/schweiz/analyse-zum-medienvertrauen-oeffentliche-medien-staerken-auch-die-privaten-ld.128965}{nicht}.
Aus gutem Grund -- denn die Resultate sind er­schüt­ternd.

Demnach
\href{http://media.digitalnewsreport.org/wp-content/uploads/2018/11/Digital-News-Report-2016.pdf\#page=60}{halten}
nur noch 50\% der Schwei­zer Be­völ­ke­rung die Nach­rich­ten für
glaub­würdig. Das Ver­trauen in die Medien­unter­nehmen und in die
Jour­na­listen liegt mit 39\% bzw. 35\% sogar noch tiefer. Mit anderen
Worten: Rund zwei Drittel der Schweizer Be­völ­ke­rung ver­traut den
ei­ge­nen Jour­na­listen nicht mehr*.*

Dennoch glaubt das For­schungs­in­sti­tut -- das u.a. vom Bundes­amt für
Kom­mu­ni­ka­tion finanziert wird -- die Nutzung tra­di­tio­neller und
ins­b. öffent­licher Medien würde das Ver­trauen ins Medien­system
\href{http://www.foeg.uzh.ch/dam/jcr:7234c6d3-1f09-4d36-b6ab-f14e659d046e/Medienmitteilung_JB_2016_dt.pdf}{»för­dern«}.
Die Da­ten zei­gen je­doch nur, dass regel­mäßige Kon­su­menten die­ser
Me­dien we­ni­ger kri­tisch sind -- und ihre An­zahl immer ge­ringer
wird.

\emph{Update:} 2017
\href{http://www.digitalnewsreport.org/survey/2017/switzerland-2017/}{sank}
das Medienvertrauen auf 46\%. Die Werte bzgl. Journalisten und
Unter­neh­men wurden nicht mehr erhoben. Gemäß FÖG war das Vertrauen
\href{http://www.foeg.uzh.ch/dam/jcr:0d0e5a10-27be-4e97-b264-b2cf7de96bbd/Broschur_Jahrbuch_foeg_deutsch_2017_ohne_Sperrvermerk.pdf}{»weiterhin
hoch«}.

\begin{center}\rule{0.5\linewidth}{\linethickness}\end{center}

\href{https://swprs.org/2017/03/01/schweizer-medien-vertrauen/}{**1.
March 2017}

\hypertarget{die-angst-vor-den-lesern}{%
\section{\texorpdfstring{\href{https://swprs.org/2017/03/01/leserkommentare/}{Die
Angst vor
den~Lesern}}{Die Angst vor den~Lesern}}\label{die-angst-vor-den-lesern}}

\href{https://swprs.org/2017/03/01/leserkommentare/}{\includegraphics{https://swprs.files.wordpress.com/2016/07/leserkommentare.png?w=600}}

Weil Propaganda von kritischen Lesern immer öfter und schneller entlarvt
wird, sind viele Medien dazu über­ge­gangen, die Kommentar­funktion auf
ihren Inter­net­­seiten stark zu zensieren oder ganz zu
\href{https://www.heise.de/tp/features/Konzentriertes-Gejammer-NZZ-schliesst-Kommentarspalte-3618957.html}{deaktivieren}.
Zuletzt griff selbst die vermeintlich liberale \emph{NZZ} zu dieser
\href{https://www.heise.de/tp/features/Konzentriertes-Gejammer-NZZ-schliesst-Kommentarspalte-3618957.html}{Maßnahme}.

Schließlich versuchten die ertappten Medien, die kri­ti­schen Leser als
Trolle
\href{https://www.nzz.ch/international/putins-internetpiraten-1.18324628}{dar­zu­stellen},
die womöglich von aus­län­dischen Re­gie­rungen fürs Kom­men­tieren
bezahlt würden. Be­lege da­für blie­ben aus, und inhaltlich wurde auf
die Leser­kritik ohnehin nicht ein­ge­gangen.

Doch nicht nur von den Medien, auch im Online-Lexikon \emph{Wikipedia}
werden die Leser an der freien Meinungs­bil­dung
\href{https://swprs.org/propaganda-in-der-wikipedia/}{gehindert}: Hier
sorgt eine kleine Gruppe anonymer »Adminis­tra­toren« dafür, dass bei
geo­po­li­tisch brisanten Themen ab­wei­chende Positionen gelöscht,
Autoren gesperrt und kritische Forscher diffamiert werden (siehe
\href{https://swprs.org/propaganda-in-der-wikipedia/}{Vertiefungsstudie}).

\begin{center}\rule{0.5\linewidth}{\linethickness}\end{center}

\href{https://swprs.org/2017/03/01/leserkommentare/}{**1. March 2017}

\hypertarget{der-chefredakteur-und-die-cia}{%
\section{\texorpdfstring{\href{https://swprs.org/2017/03/01/chefredakteur-cia/}{Der
Chefredakteur und
die~CIA}}{Der Chefredakteur und die~CIA}}\label{der-chefredakteur-und-die-cia}}

\href{https://swprs.org/2017/03/01/chefredakteur-cia/}{\includegraphics{https://swprs.files.wordpress.com/2016/05/cia-media.png?w=450}}

Die klandestine Zu­sam­men­arbeit west­licher Geheim­dienste mit Medien,
Think Tanks und NGOs ist seit langem
\href{http://carlbernstein.com/magazine_cia_and_media.php}{bekannt} und
vielfach
\href{http://www.amazon.de/Geheimdienst-Politik-Medien-Meinungsmache-Zeitgeschichte/dp/3897068796}{doku­men­tiert}.

Wie eng und um­fas­send bisweilen selbst füh­ren­de deutsch­spra­chige
Jour­na­listen mit den Ge­heim­diens­ten kooperieren, dies zeigt
bei­spiel­haft der Fall von Otto Schul­meister.

Schul­meister war lang­jäh­riger Chef­re­dak­teur der
\href{https://de.wikipedia.org/wiki/Die_Presse}{\emph{Presse}}, einer
der größ­ten und tra­di­tions­reich­sten Tages­­zeitungen Öster­reichs.
2009 wurde sein ehemaliges CIA-Dossier publik -- mit bemerkenswerten
Einzel­heiten zur ver­deckten Kol­la­bo­ration:

\href{https://swprs.org/der-chefredakteur-und-die-cia\#weiterlesen}{Weiterlesen
→}

\begin{center}\rule{0.5\linewidth}{\linethickness}\end{center}

\href{https://swprs.org/2017/03/01/chefredakteur-cia/}{**1. March 2017}

\hypertarget{die-grenzen-der-pressefreiheit}{%
\section{\texorpdfstring{\href{https://swprs.org/2017/03/01/die-grenzen-der-pressefreiheit/}{Die
Grenzen
der~Pressefreiheit}}{Die Grenzen der~Pressefreiheit}}\label{die-grenzen-der-pressefreiheit}}

\href{https://swprs.org/2017/03/01/die-grenzen-der-pressefreiheit/}{\includegraphics{https://swprs.files.wordpress.com/2017/12/reporter_ohne_grenzen_logo_s.png?w=530}}

Der \href{http://pressclub.ch/?lang=en}{Schweizer Presseclub} in Genf
genießt einen ausgezeichneten Ruf: Seit seiner Gründung hat er über
zweitausend Anlässe mit illustren Rednern von Fidel Castro bis Henry
Kissinger und von Jean Ziegler bis Klaus Schwab organisiert.

Doch für Ende November 2017 war ein
\href{http://pressclub.ch/they-dont-care-about-us-white-helmets-true-agenda/?lang=en}{Vortrag}
angekündigt, der sich kritisch mit den in west­li­chen Medien populären
\href{https://www.hintergrund.de/globales/kriege/weisse-helme-ohne-weisse-westen/}{Syrischen
Weiß­helmen} befassen wollte. Daraufhin geschah Folgendes:

\href{https://swprs.org/die-grenzen-der-pressefreiheit/}{Weiterlesen →}

\begin{center}\rule{0.5\linewidth}{\linethickness}\end{center}

\href{https://swprs.org/2017/03/01/die-grenzen-der-pressefreiheit/}{**1.
March 2017}

\hypertarget{anschlag-auf-die-forschungsfreiheit}{%
\section{\texorpdfstring{\href{https://swprs.org/2017/03/01/anschlag-auf-die-forschungsfreiheit/}{Anschlag
auf die
Forschungsfreiheit}}{Anschlag auf die Forschungsfreiheit}}\label{anschlag-auf-die-forschungsfreiheit}}

\href{https://swprs.org/2017/03/01/anschlag-auf-die-forschungsfreiheit/}{\includegraphics{https://swprs.files.wordpress.com/2018/11/ganser.png?w=500}}

So ergeht es US-kritischen Forschern in der Schweiz: Der Historiker Dr.
Daniele Ganser geriet 2006 nach einer öffentlichen Inter­vention der
amerika­nischen Bot­schaf­terin unter Druck und musste seine Forschung
an der ETH Zürich schließlich aufgeben.

Ganser forschte zu ver­deckter Kriegs­führung und
\href{http://ofv.ch/sachbuch/detail/natogeheimarmeen-in-europa/3193/}{ins­ze­nier­tem
Terror} durch die NATO im Kalten Krieg sowie zu den An­schlägen vom 11.
September 2001 (s.
\href{http://archiv.ethlife.ethz.ch/articles/9.11.html}{Artikel im
ETH-Magazin}).

\href{https://swprs.org/anschlag-auf-die-forschungsfreiheit\#weiterlesen}{Weiterlesen
→}

\begin{center}\rule{0.5\linewidth}{\linethickness}\end{center}

\href{https://swprs.org/2017/03/01/anschlag-auf-die-forschungsfreiheit/}{**1.
March 2017}

\hypertarget{das-american-empire-und-seine-medien}{%
\section{\texorpdfstring{\href{https://swprs.org/2017/03/01/netzwerk-medien-usa/}{Das
American Empire und
seine~Medien}}{Das American Empire und seine~Medien}}\label{das-american-empire-und-seine-medien}}

Top-Journalisten und Führungskräfte nahezu aller bekannten US-Medien
sind in das Netz­werk des einflussreichen \emph{Council on Foreign
Relations (CFR)} eingebunden.

Im folgenden Beitrag wird dieses Netzwerk erstmals grafisch
dar­ge­stellt.

\href{https://swprs.org/das-american-empire-und-seine-medien/}{\includegraphics{https://swprs.files.wordpress.com/2017/08/cfr-media-network-hdv-spr-s.png?w=736}}

\href{https://swprs.org/das-american-empire-und-seine-medien/}{Zur
Analyse →}

\begin{center}\rule{0.5\linewidth}{\linethickness}\end{center}

\href{https://swprs.org/2017/03/01/netzwerk-medien-usa/}{**1. March
2017}

\hypertarget{bericht-eines-journalisten}{%
\section{\texorpdfstring{\href{https://swprs.org/2017/03/01/bericht-eines-journalisten/}{Bericht
eines
Journalisten}}{Bericht eines Journalisten}}\label{bericht-eines-journalisten}}

\href{https://swprs.org/2017/03/01/bericht-eines-journalisten/}{\includegraphics{https://swprs.files.wordpress.com/2018/01/mainstreammedia.png?w=600}}

Wie entsteht der Mainstream in den Medien? Woher kommt die Propaganda?
Im folgenden Beitrag spricht erstmals ein Schweizer Top-Journalist über
seine langjährigen Erfahrungen.

\href{https://swprs.org/bericht-eines-journalisten/}{Zum Beitrag →}

\begin{center}\rule{0.5\linewidth}{\linethickness}\end{center}

\href{https://swprs.org/2017/03/01/bericht-eines-journalisten/}{**1.
March 2017}

\hypertarget{migration-und-medien}{%
\section{\texorpdfstring{\href{https://swprs.org/2017/03/01/migration-und-medien/}{Migration
und Medien}}{Migration und Medien}}\label{migration-und-medien}}

\href{https://swprs.org/2017/03/01/migration-und-medien/}{\includegraphics{https://swprs.files.wordpress.com/2017/03/migration-grafik.png?w=450}}

Worum geht es bei der Migration nach Europa, und warum wird sie von den
etablierten Medien zumeist
\href{https://www.otto-brenner-stiftung.de/wissenschaftsportal/informationsseiten-zu-studien/studien-2017/die-fluechtlingskrise-in-den-medien/}{begrüßt},
während ihre
\href{https://www.heise.de/tp/features/Massenwanderungen-haben-sowohl-in-den-Herkunftslaendern-als-auch-den-Ziellaendern-der-Migranten-4205760.html?seite=all}{Ursachen
und Folgen} kaum kritisch hinterfragt werden?

\href{https://swprs.org/migration-und-medien/}{Weiterlesen →}

\begin{center}\rule{0.5\linewidth}{\linethickness}\end{center}

\href{https://swprs.org/2017/03/01/migration-und-medien/}{**1. March
2017}

\hypertarget{propaganda-in-der-wikipedia}{%
\section{\texorpdfstring{\href{https://swprs.org/2017/03/01/propaganda-in-der-wikipedia/}{Propaganda
in
der~Wikipedia}}{Propaganda in der~Wikipedia}}\label{propaganda-in-der-wikipedia}}

Die Online-Enzyklopädie Wikipedia ist ein integraler Bestandteil des
transatlantischen Medien- und Informationssystems. In der folgenden
Analyse werden zentrale Aspekte ihrer Organisationsstruktur,
Funktionsweise und Manipulation dargestellt.

\href{https://swprs.org/propaganda-in-der-wikipedia/}{\includegraphics{https://swprs.files.wordpress.com/2019/03/wikipedia-2019-s.png?w=736}}

\href{https://swprs.org/propaganda-in-der-wikipedia/}{Zur Analyse →}

\begin{center}\rule{0.5\linewidth}{\linethickness}\end{center}

\href{https://swprs.org/2017/03/01/propaganda-in-der-wikipedia/}{**1.
March 2017}

\hypertarget{medienaufsicht-im-faktencheck}{%
\section{\texorpdfstring{\href{https://swprs.org/2017/03/01/medienaufsicht-faktencheck/}{Medienaufsicht
im
Faktencheck}}{Medienaufsicht im Faktencheck}}\label{medienaufsicht-im-faktencheck}}

\href{https://swprs.org/2017/03/01/medienaufsicht-faktencheck/}{\includegraphics{https://swprs.files.wordpress.com/2017/03/srf-ombudsstelle.png?w=600}}

Die Ombudsstelle des \emph{SRF} ist die erste Anlaufstelle für
Programm­be­schwerden des Publi­kums. Doch wie un­vor­ein­ge­nommen und
objektiv behandelt sie Beschwerden zu geo­po­li­tischen Themen?

Um dies zu über­prüfen, wurden während eines halben Jahres alle
Schluss­be­richte zum Syrien­kon­flikt einem Fakten­check unter­zogen.
Die Resul­tate sind bedenk­lich.

\href{https://swprs.org/srf-ombudsstelle-im-faktencheck/}{Zum
Faktencheck~→}

\begin{center}\rule{0.5\linewidth}{\linethickness}\end{center}

\href{https://swprs.org/2017/03/01/medienaufsicht-faktencheck/}{**1.
March 2017}

\hypertarget{der-absturz-swissair-111}{%
\section{\texorpdfstring{\href{https://swprs.org/2017/03/01/der-absturz-swissair-111/}{Der
Absturz:
Swissair~111}}{Der Absturz: Swissair~111}}\label{der-absturz-swissair-111}}

\href{https://swprs.org/2017/03/01/der-absturz-swissair-111/}{\includegraphics{https://swprs.files.wordpress.com/2018/12/swissair-111.png?w=500}}

Es ist die größte Katastrophe der Schweizer Luft­fahrt­geschichte: Am 2.
September 1998 stürzte der
\href{https://en.wikipedia.org/wiki/Swissair_Flight_111}{Swissair-Flug
111} von New York nach Genf mit 229 Menschen an Bord beim kanadischen
Halifax in den Atlantik. Die Ab­sturz­ursache wur­de bis heute nicht
überzeugend aufgeklärt. Doch für Schweizer Medien ist der Fall ein Tabu.

\href{https://swprs.org/der-absturz-swissair-111/}{Zum Beitrag →}

\begin{center}\rule{0.5\linewidth}{\linethickness}\end{center}

\href{https://swprs.org/2017/03/01/der-absturz-swissair-111/}{**1. March
2017}

\hypertarget{der-atlantic-council}{%
\section{\texorpdfstring{\href{https://swprs.org/2017/03/01/der-atlantic-council/}{Der
Atlantic Council}}{Der Atlantic Council}}\label{der-atlantic-council}}

\href{https://swprs.org/2017/03/01/der-atlantic-council/}{\includegraphics{https://swprs.files.wordpress.com/2018/11/atlantic-council.png?w=532}}

Der \emph{Atlantic Council} ist
\href{https://www.rubikon.news/artikel/facebook-als-waffe}{bekannt} für
sein En­ga­ge­ment gegen NATO-kritische »Des­in­for­ma­tion«, seine
Kooperation mit Facebook, die zur Lö­schung diverser Seiten führte,
sowie seine Ein­wir­kungen auf die eu­ro­pä­ische Außen­politik. Doch
wer ist der *Atlantic Council?\\
*

\href{https://swprs.org/atlantic-council/}{Weiterlesen →}

\begin{center}\rule{0.5\linewidth}{\linethickness}\end{center}

\href{https://swprs.org/2017/03/01/der-atlantic-council/}{**1. March
2017}

\hypertarget{russische-propaganda}{%
\section{\texorpdfstring{\href{https://swprs.org/2017/03/01/russische-propaganda/}{Russische
Propaganda}}{Russische Propaganda}}\label{russische-propaganda}}

\href{https://swprs.org/2017/03/01/russische-propaganda/}{\includegraphics{https://swprs.files.wordpress.com/2018/11/kreml.png?w=495}}

Wie funktioniert russische Propaganda, und was macht sie so
wirkungs­voll?

\href{https://swprs.org/russische-propaganda/}{Zum Beitrag →}

\begin{center}\rule{0.5\linewidth}{\linethickness}\end{center}

\href{https://swprs.org/2017/03/01/russische-propaganda/}{**1. March
2017}

\hypertarget{die-integrity-initiative}{%
\section{\texorpdfstring{\href{https://swprs.org/2017/03/01/die-integrity-initiative/}{Die
»Integrity
Initiative«}}{Die »Integrity Initiative«}}\label{die-integrity-initiative}}

\href{https://swprs.org/2017/03/01/die-integrity-initiative/}{\includegraphics{https://swprs.files.wordpress.com/2018/12/ii-logo-e1549798726940.png?w=350}}

Es ist die wohl größte Geheimdienstenthüllung seit Edward Snowden. Doch
von den etablierten Medien wurde sie nahezu vollständig ignoriert. Ein
Überblick.

\href{https://swprs.org/die-integrity-initiative/}{Zum Beitrag →}

\begin{center}\rule{0.5\linewidth}{\linethickness}\end{center}

\href{https://swprs.org/2017/03/01/die-integrity-initiative/}{**1. March
2017}

\hypertarget{die-republik-und-das-imperium}{%
\section{\texorpdfstring{\href{https://swprs.org/2017/03/01/die-republik-und-das-imperium/}{Die
Republik und
das~Imperium}}{Die Republik und das~Imperium}}\label{die-republik-und-das-imperium}}

\href{https://swprs.org/2017/03/01/die-republik-und-das-imperium/}{\includegraphics{https://swprs.files.wordpress.com/2018/11/republik.png?w=480}}

Das Online-Magazin \emph{Republik} startete 2018 mit dem
\href{https://www.persoenlich.com/medien/weltrekord-fur-journalistisches-crowdfunding-gebrochen}{erfolgreichsten}
Medien-Crowdfunding aller Zeiten. »Journalismus ist ein Kind der
Aufklä­rung. Seine Aufgabe ist die Kritik der Macht.«, proklamierte das
\href{https://www.republik.ch/manifest}{Manifest} verheißungsvoll. Doch
wie sieht es damit in der Realität aus?

\href{https://swprs.org/die-republik-und-das-imperium/}{Zum Beitrag →}

\begin{center}\rule{0.5\linewidth}{\linethickness}\end{center}

\href{https://swprs.org/2017/03/01/die-republik-und-das-imperium/}{**1.
March 2017}

\hypertarget{die-woz-und-die-weltpolitik}{%
\section{\texorpdfstring{\href{https://swprs.org/2017/03/01/die-woz-und-die-weltpolitik/}{Die
WOZ und
die~Weltpolitik}}{Die WOZ und die~Weltpolitik}}\label{die-woz-und-die-weltpolitik}}

\href{https://swprs.org/2017/03/01/die-woz-und-die-weltpolitik/}{\includegraphics{https://swprs.files.wordpress.com/2017/03/woz-logo-n.png?w=522}}

»Linksalternativ« und doch NATO-konform? Die WOZ zeigt wie's geht: In
Syrien etwa hätten ein paar
\href{https://www.woz.ch/1203/syrien/assad-geht-das-licht-aus}{Graffiti­sprayer}
eine
\href{https://www.woz.ch/1616/syriens-zukunft/assads-spiel-mit-dem-westen}{marxis­tisch}
ange­hauchte
\href{https://www.woz.ch/1511/kommentar-von-francois-moore/die-revolution-in-syrien-ist-am-ende}{»Revo­lution«}
junger
\href{https://www.woz.ch/1606/syrien/mithilfe-dieser-verdammten-russen-wird-dieser-bastard-noch-ueberleben}{Idealisten}
und \href{https://www.woz.ch/1235/syrien/kaempfen-und-beten}{frommer
Gottes­krieger} ausgelöst, während
\href{https://www.woz.ch/1324/syrien/ein-land-zersplittert-immer-mehr}{»das
Regime«} einen Krieg vom Zaun
\href{https://www.woz.ch/1321/syrien-und-der-westen/assad-kann-nur-gewinnen}{brach}
und mit »Fass­bomben« Kranken­häuser
\href{https://www.woz.ch/1416/syrien/fassbomben-gottes-wille-und-demokratie}{bombar­dierte},
sodass selbst eine NATO-Inter­vention
\href{https://www.woz.ch/1335/syrien/intervention-als-kleineres-uebel}{»das
kleinere Übel«} sei.

NATO-Kritiker Ganser hingegen
\href{https://www.woz.ch/1703/wahrheit-und-verschwoerung/das-ganser-phaenomen}{biete}
eine »Plattform für rechte Ver­schwö­rungs­theo­retiker«, und Wiki­leaks
-- an der Nieder­lage Clintons mitschuldig --
\href{https://www.woz.ch/1711/cia-dokumente/die-alternativen-fakten-von-wikileaks}{produziere}
»alter­native Fakten« für die »Neurechten«. Auch vor »alter­na­tiven
Medien« wird
\href{https://www.woz.ch/1743/qualitaet-der-medien/unterinformiert-und-ausgeliefert}{gewarnt}:
Diese »bedienen unverblümt Ver­schwörungs­theorien oder ver­breiten
rechte Propaganda«.

Wer die Global­isierung unvor­sichtig kriti­siert, sei womöglich ein
verkappter
\href{https://www.woz.ch/1708/wirtschaftlicher-protektionismus/die-voelkische-kritik-an-der-globalisierung}{»Rechts­nationa­list«},
und bei der Wachs­tums­politik des IWF dürfe man »nicht zu dogma­tisch
sein«, denn es
\href{https://www.woz.ch/1742/weltwirtschaft/die-hueterin-des-kapitalismus}{gelte},
»den Kapita­lismus vor der Rechten zu retten«. Selbst die
\href{https://www.woz.ch/1414/schweizerische-aussenpolitik/opportunistische-neutralitaet}{Schweizer
Neutra­lität} ist irgendwie »rechts«.

Medien­historisch erinnert die WOZ damit ein wenig an jene
\href{https://www.youtube.com/watch?v=3QAgCFjNXJE}{CIA-finanzierten
Publika­tionen}, die während des Kalten Krieges die potentiell kritische
Linke auf US-Kurs zu bringen versuchten. Und offenbar wird
geo­poli­tische Konfor­mität auch heute noch honoriert: Etwa mit
\href{https://swprs.files.wordpress.com/2017/10/amnesty-international-werbung.png}{ganz­seitigen
Farb­inseraten} von \emph{Amnesty Inter­national}, die in der WOZ den
Sturz von Washingtons Feinden
\href{https://consortiumnews.com/2012/06/18/amnestys-shilling-for-us-wars/}{bewerben}.

\begin{center}\rule{0.5\linewidth}{\linethickness}\end{center}

\href{https://swprs.org/2017/03/01/die-woz-und-die-weltpolitik/}{**1.
March 2017}

\hypertarget{was-ist-medienqualituxe4t}{%
\section{\texorpdfstring{\href{https://swprs.org/2017/03/01/medienqualitaet/}{Was
ist
Medienqualität?}}{Was ist Medienqualität?}}\label{was-ist-medienqualituxe4t}}

\href{https://swprs.org/2017/03/01/medienqualitaet/}{\includegraphics{https://swprs.files.wordpress.com/2018/09/mqr_logo.png?w=300}}

2018 wurde die zweite Ausgabe des \emph{Schweizer
Medien­qualitäts­rankings}
\href{http://medienqualitaet-schweiz.ch/files/3115/3578/3114/MQR-18_Hauptbefunde.pdf}{vorgestellt}.
Zuoberst fanden sich erneut die \emph{NZZ} sowie einige
*SRF-*Nach­rich­ten­for­mate. Stehen diese Resultate im Widerspruch zu
unseren \href{https://swprs.org/die-nzz-studie/}{Untersuchungen}, wonach
gerade jene Medien eine besonders hohe Propaganda-Intensität
\href{https://swprs.org/srf-propaganda-analyse/}{aufweisen}?

Keineswegs, denn das Qualitätsranking basiert auf rein formalen
\href{http://www.medienqualitaet-schweiz.ch/index.php/qualitaetsrating/}{Kriterien}
wie Relevanz, Aktualität und Professionalität -- woraus sich im
Endeffekt eine weitgehend triviale Sortierung der Medien von
boulevardesk bis bildungs­bürger­lich ergibt. Wer über den Syrienkrieg
statt über Superstars berichtet und dazu noch den Experten vom
NATO-Thinktank befragt, der schwingt im Ranking schon oben aus.

Das Qualitätsranking ist gut gemeint, die Autoren sorgen sich um den
ökonomisch bedingten Niedergang der klassischen Medien und die
\href{https://www.nzz.ch/feuilleton/medien/was-die-medien-fuer-die-schweizer-demokratie-leisten-ld.1416854}{Auswirkungen}
auf das Staatswesen. Doch für den kritischen Leser genügt ein solch
formaler Ansatz längst nicht mehr -- denn gefragt ist wahrhaftiger
Journalismus, und nicht bloß
\href{https://swprs.org/der-propaganda-schluessel/}{Manipulation} auf
hohem Niveau.

\begin{center}\rule{0.5\linewidth}{\linethickness}\end{center}

\href{https://swprs.org/2017/03/01/medienqualitaet/}{**1. March 2017}

\hypertarget{die-propaganda-matrix}{%
\section{\texorpdfstring{\href{https://swprs.org/2017/03/01/propaganda-matrix/}{Die
Propaganda-Matrix}}{Die Propaganda-Matrix}}\label{die-propaganda-matrix}}

\href{https://swprs.org/2017/03/01/propaganda-matrix/}{\includegraphics{https://swprs.files.wordpress.com/2017/03/propaganda-matrix-fs.png?w=449}}

Ob Russland, Syrien oder Donald Trump: Um die geopolitische
Bericht­erstattung westlicher Medien zu verstehen, muss man die
Schlüssel­rolle des amerikanischen \emph{Council on Foreign Relations
(CFR)} kennen.

In der folgenden Studie wird erstmals dargestellt, wie der CFR einen in
sich weitgehend geschlossenen, trans­atlantischen
Informations­­kreislauf schuf, in dem nahezu alle relevanten Quellen und
Bezugs­punkte von Mitgliedern des Councils und seiner
Partner­­organisationen kontrolliert werden.

Auf diese Weise entstand eine historisch einzigartige
Informations­­matrix, die klassischer Regierungs­propaganda autoritärer
Staaten deutlich überlegen ist, indes durch den Erfolg unabhängiger
Medien zunehmend an Wirksamkeit verliert.

\href{https://swprs.org/die-propaganda-matrix}{Zur Studie →}

\begin{center}\rule{0.5\linewidth}{\linethickness}\end{center}

\href{https://swprs.org/2017/03/01/propaganda-matrix/}{**1. March 2017}

\hypertarget{filmtipp-the-magnitsky-act}{%
\section{\texorpdfstring{\href{https://swprs.org/2017/03/01/magnitsky/}{Filmtipp:
The
Magnitsky~Act}}{Filmtipp: The Magnitsky~Act}}\label{filmtipp-the-magnitsky-act}}

\href{https://swprs.org/2017/03/01/magnitsky/}{\includegraphics{https://swprs.files.wordpress.com/2019/11/the-magnitsky-act-movie.jpeg?w=600}}

Von \emph{ARTE} bei einem russischen Star­re­gisseur in Auftrag gegeben,
doch niemals ausgestrahlt: der politisch höchst brisante
Doku­men­tar­film \emph{The Magnitsky Act -- Behind the Scenes.}

\href{https://swprs.org/der-fall-magnitsky/}{Zum Beitrag und Film →}

\begin{center}\rule{0.5\linewidth}{\linethickness}\end{center}

\href{https://swprs.org/2017/03/01/magnitsky/}{**1. March 2017}

\hypertarget{israel-lobby-fakten--mythen}{%
\section{\texorpdfstring{\href{https://swprs.org/2017/03/01/die-israel-lobby-fakten-und-mythen/}{»Israel-Lobby«:
Fakten
\&~Mythen}}{»Israel-Lobby«: Fakten \&~Mythen}}\label{israel-lobby-fakten--mythen}}

\href{https://swprs.org/2017/03/01/die-israel-lobby-fakten-und-mythen/}{\includegraphics{https://swprs.files.wordpress.com/2015/03/israel-lobby-e1549800362648.png?w=300}}

Analysen und Dokumentationen zur Rolle der »Israel-Lobby« in Politik und
Medien.

\href{https://swprs.org/die-israel-lobby-fakten-und-mythen/}{Zum Beitrag
→}

\begin{center}\rule{0.5\linewidth}{\linethickness}\end{center}

\href{https://swprs.org/2017/03/01/die-israel-lobby-fakten-und-mythen/}{**1.
March 2017}

\hypertarget{geopolitik-und-puxe4dokriminalituxe4t}{%
\section{\texorpdfstring{\href{https://swprs.org/2017/03/01/geopolitik-paedokriminalitaet/}{Geopolitik
und
Pädokriminalität}}{Geopolitik und Pädokriminalität}}\label{geopolitik-und-puxe4dokriminalituxe4t}}

\href{https://swprs.org/2017/03/01/geopolitik-paedokriminalitaet/}{\includegraphics{https://swprs.files.wordpress.com/2015/01/dutroux2-e1549800704925.jpg?w=450}}

Der folgende Beitrag bietet eine Übersicht zum Themen­komplex der
organisierten Pädo­krimi­nalität im Kontext von Geopolitik,
Eliten­ma­na­ge­ment, und westlichem Medien­system.

\href{https://swprs.org/geopolitik-und-paedokriminalitaet/}{Zum Beitrag
→}

\begin{center}\rule{0.5\linewidth}{\linethickness}\end{center}

\href{https://swprs.org/2017/03/01/geopolitik-paedokriminalitaet/}{**1.
March 2017}

\hypertarget{trump-medien-geopolitik}{%
\section{\texorpdfstring{\href{https://swprs.org/2017/03/01/trump-medien-geopolitik/}{Trump,
Medien,
Geopolitik}}{Trump, Medien, Geopolitik}}\label{trump-medien-geopolitik}}

\href{https://swprs.org/2017/03/01/trump-medien-geopolitik/}{\includegraphics{https://swprs.files.wordpress.com/2017/08/trump-media-geopolitics.jpg?w=600}}

Die folgende Analyse widmet sich der Frage, wie sich die auf­fallend
negative Bericht­er­stattung der tra­di­tio­nellen west­lichen Medien
über die Trump-Prä­si­dent­schaft schlüssig er­klären lässt.

Dabei zeigt sich, dass keine der übli­chen Er­klä­rungen -- die
angeb­liche In­kom­petenz Trumps, eine an­geb­liche »Links­las­tig­keit«
der Medien, Ein­schalt­quoten oder Par­ti­ku­lar­inte­ressen
ein­fluss­reicher Lobbys -- stich­haltig ist.

Vielmehr dürfte die negative Bericht­erstattung auf geostra­tegische
Aspekte und die (bedrohte) Rolle des \emph{Council on Foreign Relations}
als oberstes geopolitisches Gremium der Vereinigten Staaten
zurück­­zu­führen sein.

\href{https://swprs.org/trump-medien-geopolitik/}{Zur Analyse →}

\begin{center}\rule{0.5\linewidth}{\linethickness}\end{center}

\href{https://swprs.org/2017/03/01/trump-medien-geopolitik/}{**1. March
2017}

\hypertarget{die-logik-imperialer-kriege}{%
\section{\texorpdfstring{\href{https://swprs.org/2017/03/01/die-logik-imperialer-kriege/}{Die
Logik
imperialer~Kriege}}{Die Logik imperialer~Kriege}}\label{die-logik-imperialer-kriege}}

Wie lassen sich die amerikanischen Kriege der letzten Jahrzehnte
rational erklären?

Die folgende Analyse zeigt anhand des Modells der Professoren David
Sylvan und Stephen Majeski, dass diese Kriege auf einer eigenen, genuin
imperialen Handlungslogik basieren.

Eine besondere Rolle kommt dabei dem traditionellen Mediensystem zu.

\href{https://swprs.org/logik-imperialer-kriege/}{\includegraphics{https://swprs.files.wordpress.com/2018/05/logik-imperialer-kriege-spr-s.png?w=736}}

\href{https://swprs.org/logik-imperialer-kriege/}{Zur Analyse →}

\href{https://swprs.org/us-foreign-policy/}{Zur englischen Version →}

\begin{center}\rule{0.5\linewidth}{\linethickness}\end{center}

\href{https://swprs.org/2017/03/01/die-logik-imperialer-kriege/}{**1.
March 2017}

\hypertarget{medien-in-deutschland}{%
\section{\texorpdfstring{\href{https://swprs.org/2017/03/01/medien-in-deutschland/}{Medien
in Deutschland}}{Medien in Deutschland}}\label{medien-in-deutschland}}

Deutsche Medien und Journalisten sind aus historischen Gründen besonders
eng in trans­at­lan­tische Netz­werke eingebunden. Die folgende
Info­grafik gibt einen Über­blick über die wich­tigsten Akteure und
Ver­bindungen.

Auch die Schweiz ist von diesem Netzwerk tangiert: etwa durch hiesige
Marktanteile, diverse Kooperationen, sowie Interventionen bei
nicht-konformer Berichterstattung.

\href{https://swprs.org/netzwerk-medien-deutschland/}{\includegraphics{https://swprs.files.wordpress.com/2017/08/netzwerk-medien-deutschland-spr-mts.png?w=736}\\
Zur Infografik~→}

\begin{center}\rule{0.5\linewidth}{\linethickness}\end{center}

\href{https://swprs.org/2017/03/01/medien-in-deutschland/}{**1. March
2017}

\hypertarget{leserbriefe}{%
\section{\texorpdfstring{\href{https://swprs.org/2017/03/01/leserbriefe/}{Leserbriefe}}{Leserbriefe}}\label{leserbriefe}}

\href{https://swprs.org/2017/03/01/leserbriefe/}{\includegraphics{https://swprs.files.wordpress.com/2016/10/comments3.png?w=250}}

\emph{»Ich habe 5 Jahre auf einer renom­mier­ten Schweizer
Nachrichten­redaktion ge­ar­bei­tet. Es ist das erste Mal, dass ich eine
so umfassende Arbeit zum Thema sehe. Konnte noch nicht alles lesen, aber
das, was ich gelesen habe, deckt sich mit meiner Erfahrung und
Wahr­nehmung.«}

\emph{»Ich hatte selbst in den 80er Jahren bei der NZZ gearbeitet.
Damals eine Auszeichnung. Heute leider so wie von Ihnen beschrieben.«}

\emph{»Ganz herzliche Gratulation zu eurer sehr infor­ma­tiven Seite. So
etwas hat für die Schweiz noch gefehlt. Freue mich schon auf weitere
Beiträge!«}

\href{https://swprs.org/leserbriefe/}{Zu den Leserbriefen →}

\begin{center}\rule{0.5\linewidth}{\linethickness}\end{center}

\href{https://swprs.org/2017/03/01/leserbriefe/}{**1. March 2017}

\hypertarget{medienspiegel}{%
\section{\texorpdfstring{\href{https://swprs.org/2017/03/01/medienspiegel/}{Medienspiegel}}{Medienspiegel}}\label{medienspiegel}}

\href{https://swprs.org/2017/03/01/medienspiegel/}{\includegraphics{https://swprs.files.wordpress.com/2017/03/newspaper2.png?w=500}}

Als eines der bekanntesten medien­kritischen Forschungs­projekte werden
unsere Arbeiten zunehmend auch von tradi­tio­nellen und neuen Medien
rezipiert.

Im Folgenden findet sich ein fort­laufend aktuali­sierter Medien­spiegel
mit den wich­tig­sten Beiträgen und Übersetzungen.

\href{https://swprs.org/medienspiegel/}{Zum Medienspiegel →}

\begin{center}\rule{0.5\linewidth}{\linethickness}\end{center}

\href{https://swprs.org/2017/03/01/medienspiegel/}{**1. March 2017}

\hypertarget{das-forschungsprojekt}{%
\section{\texorpdfstring{\href{https://swprs.org/2017/03/01/das-forschungsprojekt/}{Das
Forschungsprojekt}}{Das Forschungsprojekt}}\label{das-forschungsprojekt}}

\href{https://swprs.org/2017/03/01/das-forschungsprojekt/}{\includegraphics{https://swprs.files.wordpress.com/2017/03/spr_logo_post.png?w=364}}

Swiss Policy Research (SPR) ist ein Forschungs- und
Infor­ma­tions­projekt zu geo­po­li­tischer Pro­pa­ganda in Schweizer
Medien.

Sämtliche Studien und Bei­träge wurden von einer po­li­tisch und
pu­bli­zis­tisch un­ab­hän­gigen For­schungs­gruppe ohne Beauf­tra­gung
oder Fremd­finan­zierung er­­stellt.

Das Forschungsprojekt wurde 2016 lanciert und zählt inzwischen zu den
bekanntesten Publi­ka­tionen auf diesem Gebiet.

Hier können Sie uns \href{https://swprs.org/kontakt/}{kon­­tak­tieren}.

\begin{center}\rule{0.5\linewidth}{\linethickness}\end{center}

\href{https://swprs.org/2017/03/01/das-forschungsprojekt/}{**1. March
2017}

\hypertarget{swiss-policy-research}{%
\subsubsection{Swiss Policy Research}\label{swiss-policy-research}}

\begin{itemize}
\tightlist
\item
  \href{https://swprs.org/kontakt/}{Kontakt}
\item
  \href{https://swprs.org/uebersicht/}{Übersicht}
\item
  \href{https://swprs.org/donationen/}{Donationen}
\item
  \href{https://swprs.org/disclaimer/}{Disclaimer}
\end{itemize}

\hypertarget{english}{%
\subsubsection{English}\label{english}}

\begin{itemize}
\tightlist
\item
  \href{https://swprs.org/contact/}{About Us / Contact}
\item
  \href{https://swprs.org/media-navigator/}{The Media Navigator}
\item
  \href{https://swprs.org/the-american-empire-and-its-media/}{The CFR
  and the Media}
\item
  \href{https://swprs.org/donations/}{Donations}
\end{itemize}

\hypertarget{follow-by-email}{%
\subsubsection{Follow by email}\label{follow-by-email}}

Follow

\href{https://wordpress.com/?ref=footer_custom_com}{WordPress.com}.

\protect\hyperlink{}{Up ↑}

Post to

\protect\hyperlink{}{Cancel}

\includegraphics{https://pixel.wp.com/b.gif?v=noscript}
