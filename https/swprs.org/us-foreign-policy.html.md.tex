\protect\hyperlink{content}{Skip to content}

\href{https://swprs.org/}{}

\protect\hyperlink{search-container}{Search}

Search for:

\href{https://swprs.org/}{\includegraphics{https://swprs.files.wordpress.com/2020/05/swiss-policy-research-logo-300.png}}

\href{https://swprs.org/}{Swiss Policy Research}

Geopolitics and Media

Menu

\begin{itemize}
\tightlist
\item
  \href{https://swprs.org}{Start}
\item
  \href{https://swprs.org/srf-propaganda-analyse/}{Studien}

  \begin{itemize}
  \tightlist
  \item
    \href{https://swprs.org/srf-propaganda-analyse/}{SRF / ZDF}
  \item
    \href{https://swprs.org/die-nzz-studie/}{NZZ-Studie}
  \item
    \href{https://swprs.org/der-propaganda-multiplikator/}{Agenturen}
  \item
    \href{https://swprs.org/die-propaganda-matrix/}{Medienmatrix}
  \end{itemize}
\item
  \href{https://swprs.org/medien-navigator/}{Analysen}

  \begin{itemize}
  \tightlist
  \item
    \href{https://swprs.org/medien-navigator/}{Navigator}
  \item
    \href{https://swprs.org/der-propaganda-schluessel/}{Techniken}
  \item
    \href{https://swprs.org/propaganda-in-der-wikipedia/}{Wikipedia}
  \item
    \href{https://swprs.org/logik-imperialer-kriege/}{Kriege}
  \end{itemize}
\item
  \href{https://swprs.org/netzwerk-medien-schweiz/}{Netzwerke}

  \begin{itemize}
  \tightlist
  \item
    \href{https://swprs.org/netzwerk-medien-schweiz/}{Schweiz}
  \item
    \href{https://swprs.org/netzwerk-medien-deutschland/}{Deutschland}
  \item
    \href{https://swprs.org/medien-in-oesterreich/}{Österreich}
  \item
    \href{https://swprs.org/das-american-empire-und-seine-medien/}{USA}
  \end{itemize}
\item
  \href{https://swprs.org/bericht-eines-journalisten/}{Fokus I}

  \begin{itemize}
  \tightlist
  \item
    \href{https://swprs.org/bericht-eines-journalisten/}{Journalistenbericht}
  \item
    \href{https://swprs.org/russische-propaganda/}{Russische Propaganda}
  \item
    \href{https://swprs.org/die-israel-lobby-fakten-und-mythen/}{Die
    »Israel-Lobby«}
  \item
    \href{https://swprs.org/geopolitik-und-paedokriminalitaet/}{Pädokriminalität}
  \end{itemize}
\item
  \href{https://swprs.org/migration-und-medien/}{Fokus II}

  \begin{itemize}
  \tightlist
  \item
    \href{https://swprs.org/covid-19-hinweis-ii/}{Coronavirus}
  \item
    \href{https://swprs.org/die-integrity-initiative/}{Integrity
    Initiative}
  \item
    \href{https://swprs.org/migration-und-medien/}{Migration \& Medien}
  \item
    \href{https://swprs.org/der-fall-magnitsky/}{Magnitsky Act}
  \end{itemize}
\item
  \href{https://swprs.org/kontakt/}{Projekt}

  \begin{itemize}
  \tightlist
  \item
    \href{https://swprs.org/kontakt/}{Kontakt}
  \item
    \href{https://swprs.org/uebersicht/}{Seitenübersicht}
  \item
    \href{https://swprs.org/medienspiegel/}{Medienspiegel}
  \item
    \href{https://swprs.org/donationen/}{Donationen}
  \end{itemize}
\item
  \href{https://swprs.org/contact/}{English}
\end{itemize}

\protect\hyperlink{}{Open Search}

\hypertarget{the-logic-of-us-foreign-policy}{%
\section{The Logic of US
Foreign~Policy}\label{the-logic-of-us-foreign-policy}}

\textbf{Published}: May 2018; Updated: December 2019*\\
\emph{\textbf{Languages}:
\href{https://swprs.org/logik-imperialer-kriege/}{German},
\href{https://swprs.files.wordpress.com/2019/12/logic-of-us-foreign-policy-spanish.pdf}{Spanish},
\href{https://swprs.org/the-logic-of-us-foreign-policy-arabic/}{Arabic},
\href{https://swprs.org/us-foreign-policy-hebrew/}{Hebrew},
\href{https://swprs.files.wordpress.com/2019/12/logic-of-us-foreign-policy-persian.pdf}{Persian}}\\
*

How can US foreign policy be explained in a systematic and rational way?
The following chart -- based on a model developed by political science
professors David Sylvan and Stephen Majeski -- reveals the longstanding
imperial logic behind US diplomatic and military interventions around
the globe.

\href{https://swprs.files.wordpress.com/2018/05/us-foreign-policy-hd.png}{\includegraphics{https://swprs.files.wordpress.com/2018/05/us-foreign-policy-hd.png?w=736}}~\href{https://swprs.files.wordpress.com/2018/05/us-foreign-policy-hd.png}{Click
to enlarge 🔎}

Due to its economic and military supremacy, the Unites States has been
assuming the role of a modern \textbf{empire} since the Second World War
and especially since 1990. This status implies a very specific, and a
genuinely imperial, logic of action for its foreign policy (see figure
above).

The central distinction (\textbf{\#1} in the figure above) from the
perspective of an empire is that between \textbf{client and non-client
states}. The concept of the client state dates back to the time of the
Roman Empire and denotes states that are basically self-governing but
nevertheless align their foreign and security policy and their
succession of government with the empire.

In the case of \textbf{existing client states} (left side of the
diagram), the empire has to decide between routine administration
(\textbf{B} -- e.g. Switzerland and Austria), military or non-military
(e.g. economic) support (\textbf{D to I} -- e.g. Colombia and Pakistan),
or an attempt to replace unacceptable client governments
demo­cra­tically or militarily (\textbf{A} -- e.g. Greece 1967, Chile
1973). In certain cases, a client government can no longer hold on to
power despite imperial support and must be dropped or the client state
has to be abandoned altogether (\textbf{C, F, G} -- e.g. South Vietnam
1975 or Iran 1979).

In the case of \textbf{non-client states} (right side of diagram), the
situation is quite different. If a region newly comes into the empire's
sphere of influence, the empire will first attempt to acquire its
members peacefully as client states (\textbf{J}). This was the case,
e.g., in Eastern Europe and the Baltic States after 1990.

\href{https://swprs.files.wordpress.com/2018/05/nato_map_final.png}{\includegraphics{https://swprs.files.wordpress.com/2018/05/nato_map_final.png?w=650\&h=416}}

The eastern expansion of NATO
(\href{https://www.cfr.org/backgrounder/north-atlantic-treaty-organization-nato}{CFR/NATO})*\\
*

If, on the other hand, a state refuses to become a client state, it
sooner or later becomes an \textbf{enemy state}. This is because, solely
through its independence and autonomy, it calls into question the
empire's claim to hegemony and thus is threatening its internal and
external stability, as an empire that can no longer assert its hegemony
will fall apart. In this way, most empires slide into an almost
unavoidable compulsion to expand, which in the end will affect even
fundamentally peaceful states.

In case of enemy states, the empire must first decide whether
\textbf{military action} is promising or not (\textbf{\#11}). If not,
the empire will possibly start negotiations and, depending on the
chances of success, either end the enemy status (\textbf{K}) or impose
\textbf{sanctions} or strive for a (civilian) \textbf{regime change}
(\textbf{L}).

Typical examples of this situation are currently Iran, North Korea,
Russia and increasingly China. It is no coincidence that most of these
states possess or are striving for nuclear weapons, because only in this
way can the decisive switch \textbf{\#11} be shifted permanently from
military to non-military scenarios. Another important factor is the
availability of essential raw materials such as oil and gas, as without
them it will not be possible to maintain independence in the long term.

The primary concern with \textbf{raw materials}, therefore, is not that
the empire wants to own them directly -- after all, even enemy states
such as the former USSR, modern Russia, Iran, Libya or Venezuela have
always sold their raw materials to the West -- but that raw materials
give enemy states inde­pen­dence and influence, which in itself
constitutes a threat from an imperial and hegemonic perspective.

If, on the other hand, the empire judges a military action as promising,
the next question is whether or not the enemy state or its government
has \textbf{international legitimacy} (\textbf{\#13}). If yes, a
\emph{covert} hostile intervention has to be prepared, if no, an
\emph{overt} hostile intervention is possible. In many cases, their
autocratic form of government can be used to deny enemy states
international legitimacy.\\
\href{https://swprs.files.wordpress.com/2018/05/nato-partnerships.png}{\includegraphics{https://swprs.files.wordpress.com/2018/05/nato-partnerships.png?w=650\&h=459}}

Libya and Syria/Lebanon were the last Mediterranean countries that were
not members of NATO's Mediterranean Partnership (red) and instead wanted
to pursue their own regional policy.
(\href{https://www.nato.int/cps/ua/natohq/topics_81136.htm}{NATO})*\\
*

\textbf{Covert hostile interventions} include, in particular, a coup
d'état (\textbf{M} -- e.g. Iran 1953, Egypt 1956) and covert support for
rebels (\textbf{N} -- e.g. Afghanistan 1979ff) or exile groups
(\textbf{O} -- e.g. Cuba 1961ff). These are, in general, classic
intelligence operations.

In the case of \textbf{overt hostile interventions}, the first step is
to determine if an enemy state is already engaged in a conflict, if
local insurgents are present, and if own ground troops are required.
Possible results include asymmetric (air) attacks (\textbf{Q} -- e.g.
Serbia 1999), support for rebels (\textbf{R} -- e.g. Syria 2011ff), a
targeted invasion (\textbf{S} -- e.g. Iraq 2003), or a full-scale war
(\textbf{P} -- e.g. Germany 1941-45, Korea 1950-51).

The imperial logic is fundamentally \textbf{independent of the
respective US government}. However, different governments may come to
different conclusions regarding the prospects of success of military
action (\textbf{\#11}) and diplomatic negotiations (\textbf{\#12}), the
advantages of overt versus covert operations (\textbf{\#13}), the
acceptance of existing client regimes (\textbf{\#2}), and domestic
political support for military intervention (\textbf{E}).

The logic described above also implies the key geopolitical functions of
\href{https://swprs.org/the-american-empire-and-its-media/}{imperially
oriented media}, viz. the delegitimization of enemy states or their
governments (\textbf{\#13}), the support of overt and the con­ceal­ment
of covert hostile operations (\textbf{\#14} to \textbf{\#18}), the
justification of sanctions and regime changes (\textbf{L}), as well as
assisting imperial leadership (\textbf{B}) and the deposition of
unwanted client regimes (\textbf{A}).

It is important to note, however, that the rapidly growing range of
\href{https://swprs.org/media-navigator/}{independent media outlets}
available on the Internet makes a coherent media portrayal of imperial
interventions increasingly difficult. This is a rather new development
whose effects on imperial policy are not yet fully predictable.

Retired U.S. General Wesley Clark about the ``seven countries in five
years'' strategy\\
Source:
\href{https://www.democracynow.org/2007/3/2/gen_wesley_clark_weighs_presidential_bid}{Democracy
Now, 2007}

\hypertarget{traditional-explanations}{%
\paragraph{Traditional Explanations}\label{traditional-explanations}}

The Logic of US Foreign Policy by Sylvan and Majeski offers a consistent
explanation for American interventions of the past several decades. The
usual explanations -- by proponents as well as opponents of these wars
-- are, however, mostly to be seen as pretexts, rationalizations or at
best partial aspects, as the following overview shows.

\begin{enumerate}
\def\labelenumi{\arabic{enumi}.}
\tightlist
\item
  \textbf{Defending democracy and human rights}: This classical
  justification is not very convincing, since democratic governments
  have been overthrown (\textbf{A, M, N}), autocrats have been supported
  (\textbf{E} and \textbf{I}), human rights and international law have
  been violated or violations tolerated by the US.
\item
  \textbf{Combating terrorism}: Paramilitary groups -- including
  Islamist organizations -- have been used
  \href{https://en.wikipedia.org/wiki/Operation_Cyclone}{for decades} by
  the US to eliminate opposing regimes (\textbf{N} and \textbf{R}).
\item
  \textbf{Specific threats or aggressions against the US}: In
  retrospect, most of these scenarios turned out to be incorrect or
  made-up (\textbf{\#13}; e.g. Tonkin, incubator and WMD claims).
\item
  \textbf{Raw materials (especially oil and gas)}: Even enemy states
  generally want to sell their raw materials to the West, but are
  prevented from doing so by means of sanctions or war. This is because
  from an imperial point of view, their independence and influence is
  seen as a threat.

  \begin{enumerate}
  \def\labelenumii{\arabic{enumii}.}
  \tightlist
  \item
    \textbf{Was the Iraq war about oil?} Hardly. Already prior to 2003,
    Iraq had supplied its oil mainly to the West; the Iraqi oil sector
    was
    \href{https://theconversation.com/iraq-what-happened-to-the-oil-after-the-war-62188}{not
    privatized} after the war, and production licences were
    \href{https://www.reuters.com/article/us-iraq-oil-contracts/oil-companies-temper-iraqs-dreams-of-production-expansion-idUSKCN1GQ1ID}{also
    issued} to corporations in France, Russia and China (which opposed
    the war).
  \item
    \textbf{Was the Syrian war about natural gas pipelines?} No (see
    \href{https://truthout.org/articles/the-war-against-the-assad-regime-is-not-a-pipeline-war/}{here}
    and
    \href{https://www.middleeasteye.net/big-story/pipelineistan-conspiracy-war-syria-has-never-been-about-gas}{here}).
    The plans for regime change and war against geopolitically
    independent Syria had existed
    \href{https://www.globalresearch.ca/syrian-regime-change-a-70-year-project/5636433}{for
    decades} and were to be implemented during the so-called ``Arab
    Spring''. (See also
    a~\href{https://www.youtube.com/watch?v=flaqLAp0Yp4\&t=1674}{comment}
    by the Syrian president).
  \item
    \textbf{Was the Afghanistan war about a natural gas pipeline?}
    \href{https://slate.com/culture/2001/12/is-the-afghan-war-about-an-oil-pipeline.html}{No}.
    The Taliban were and are interested in the
    \href{https://en.wikipedia.org/wiki/Turkmenistan\%E2\%80\%93Afghanistan\%E2\%80\%93Pakistan\%E2\%80\%93India_Pipeline}{TAPI
    pipeline}, but didn't accept US political and military demands.
  \item
    \textbf{Was the Libya war about oil reserves?} No. Libya was already
    one of Europe's most important suppliers of oil under Gaddafi, and
    security of supply has
    \href{http://www.businessinsider.com/r-how-unstable-is-libyas-oil-production-2018-3}{declined}
    significantly since the war. Libya, however, pursued an independent
    and comprehensive
    \href{https://globalresearch.ca/libya-a-war-on-africa/26474}{Africa
    policy} -- financed by its oil wealth -- which collided with the
    plans of the US and France.
  \item
    \textbf{Was the Iranian regime change in 1953 about the
    nationalization of oil?} No. The US tried to mediate in the
    British-Iranian oil dispute and urged the British to compromise.
    Only when Iranian Prime Minister Mossadegh cooperated with the
    Communist Tudeh Party and opened the country to the Soviet Union did
    the CIA intervene. Iranian oil, however,
    \href{https://www.nationalreview.com/2015/07/what-really-happened-shahs-iran/}{remained
    nationalized} even after the coup.
  \item
    \textbf{What was the 2019 Venezuela coup attempt about?} See
    \href{https://swprs.org/venezuela-its-not-about-the-oil/}{Venezuela:
    It's Not About Oil}.
  \item
    \textbf{Could renewable energies solve the raw materials problem?}
    Hardly, because renewable energies, storage technologies and
    high-tech electronics require
    \href{https://en.wikipedia.org/wiki/Rare-earth_element}{rare-earth
    metals}, 97\% of which are currently produced by China, and conflict
    minerals such as \href{https://en.wikipedia.org/wiki/Coltan}{coltan}
    from the Congo.
  \end{enumerate}
\item
  \textbf{The} \textbf{``Petro-Dollar''}: The petro-dollar thesis was
  developed in the course of the Iraq war. However, the significance of
  the US dollar
  \href{https://foreignpolicy.com/2009/10/07/debunking-the-dumping-the-dollar-conspiracy/}{does
  not derive from oil}, but from US economic power. While many states
  naturally prefer the stable dollar for their raw material exports,
  enemy states often have to switch to other currencies in order to
  circumvent sanctions (\textbf{L}, e.g. Iran).
\item
  \textbf{Capitalism}: In 1917 Lenin
  \href{https://en.wikipedia.org/wiki/Imperialism,_the_Highest_Stage_of_Capitalism}{described}
  ``imperialism as the highest stage of capitalism,'' since capitalist
  states would have to conquer markets for their overproduction.
  However, even enemy states want to trade with the West, but are
  prevented from doing so by sanctions or war. Moreover, pre-capitalist
  states like Rome and Spain and even anti-capitalist states like the
  Soviet Union had already waged imperial wars.
\item
  \textbf{National debt}: The national debt is also no reason for US
  wars, as the US is
  \href{https://www.investopedia.com/articles/investing/081415/understanding-how-federal-reserve-creates-money.asp}{creating}
  its own money by using the Fed. Moreover, wars themselves contribute
  immensely to national expenses.
\item
  \textbf{Arms industry}: In 1961 US President Eisenhower warned of the
  increasing influence of the
  \href{https://en.wikipedia.org/wiki/Military\%E2\%80\%93industrial_complex}{``military-industrial
  complex''}. The latter is certainly one of the main profiteers of
  wars, but this applies as well to countries such as Russia, China,
  Sweden and Switzerland. Moreover, US wars are not arbitrary, but
  follow a certain logic; after all, even the Roman Empire did not
  conduct its wars merely to produce as many weapons as possible.
\item
  \textbf{The ``Israel Lobby''}: This aspect was emphasized in the
  \href{https://en.wikipedia.org/wiki/The_Israel_Lobby_and_U.S._Foreign_Policy}{book}
  of the same name by Professors Walt and Mearsheimer. The Israeli
  government and pro-Israeli organizations such as AIPAC
  \href{https://fpif.org/dont_blame_the_iraq_debacle_on_the_israel_lobby/}{lobbied}
  for the 2003 Iraq War and a war against Iran. As a hegemonic power,
  however, the US must intervene from East Asia to Central Africa and
  South America, and even the wars in the Middle East follow a
  superordinate logic. (More:
  \href{https://swprs.org/the-israel-lobby-facts-and-myths/}{The
  ``Israel Lobby'': Facts and Myths})
\item
  \textbf{Neoconservatives}: Another hypothesis proposes that US wars
  are driven by the so-called
  \href{https://en.wikipedia.org/wiki/Neoconservatism}{neo­con­ser­vatives}.
  This idea is disconfirmed, for instance, by the numerous wars
  initiated or continued by the liberal Clinton and Obama
  administrations (Yugoslavia, Somalia, Syria, Yemen, etc.)
\end{enumerate}

\emph{»We've got about five or ten years to clean up those old Soviet
client regimes\\
-- Syria, Iran, Iraq -- before the next great superpower comes on to
challenge us.«}\\
Pentagon policy chief Paul Wolfowitz to General Wesley Clark in 1991
(\href{https://youtu.be/TY2DKzastu8?t=3m6s}{FORA})

\hypertarget{literature}{%
\paragraph{Literature}\label{literature}}

Sylvan, David \& Majeski, Stephen
(\href{http://www.us-foreign-policy-perspective.org/}{2009}):
\href{http://www.us-foreign-policy-perspective.org/}{U.S. Foreign Policy
in Perspective: Clients, Enemies and Empire}. Routledge, London.

Blum, William
(\href{https://www.zedbooks.net/shop/book/killing-hope/}{2014}): US
Military and CIA Interventions Since World War II -- Updated Edition.
ZED Books, London.

Brzezinski, Zbigniew
(\href{https://archive.org/details/TheGrandChessboardAmericanPrimacyAndItsGeostrategicImperatives1997ZbigniewBrzezinski}{1998}):
The Grand Chessboard: American Primacy And Its Geostrategic Imperatives.
Basic Books, New York.

Haass, Richard (\href{https://www.cfr.org/book/world-disarray}{2017}): A
World in Disarray: American Foreign Policy and the Crisis of the Old
Order. Penguin Press, London.

Kagan, Robert
(\href{http://carnegieendowment.org/1998/06/01/benevolent-empire-pub-275}{1998}):
The Benevolent Empire. Foreign Policy Magazine.

Kissinger, Henry
(\href{https://www.penguinrandomhouse.com/books/316669/world-order-by-henry-kissinger/9780143127710}{2015}):
World Order. Penguin Books, London.

\hypertarget{see-also}{%
\paragraph{See also}\label{see-also}}

\begin{itemize}
\tightlist
\item
  \href{https://swprs.org/rwanda-what-did-really-happen-in-1994/}{Rwanda:
  What Did Really Happen in 1994?}
\item
  \href{https://swprs.org/propaganda-in-the-war-on-yugoslavia/}{Propaganda
  in the War on Yugoslavia}
\item
  \href{https://swprs.org/the-american-empire-and-its-media/}{The
  American Empire and its Media}
\end{itemize}

\begin{center}\rule{0.5\linewidth}{\linethickness}\end{center}

Share this on:
\href{https://twitter.com/intent/tweet?url=https://swprs.org/us-foreign-policy/}{Twitter}
/
\href{https://www.facebook.com/share.php?u=https://swprs.org/us-foreign-policy/}{Facebook}\\
Published: May 2018; Updated: December 2019

\hypertarget{swiss-policy-research}{%
\subsubsection{Swiss Policy Research}\label{swiss-policy-research}}

\begin{itemize}
\tightlist
\item
  \href{https://swprs.org/kontakt/}{Kontakt}
\item
  \href{https://swprs.org/uebersicht/}{Übersicht}
\item
  \href{https://swprs.org/donationen/}{Donationen}
\item
  \href{https://swprs.org/disclaimer/}{Disclaimer}
\end{itemize}

\hypertarget{english}{%
\subsubsection{English}\label{english}}

\begin{itemize}
\tightlist
\item
  \href{https://swprs.org/contact/}{About Us / Contact}
\item
  \href{https://swprs.org/media-navigator/}{The Media Navigator}
\item
  \href{https://swprs.org/the-american-empire-and-its-media/}{The CFR
  and the Media}
\item
  \href{https://swprs.org/donations/}{Donations}
\end{itemize}

\hypertarget{follow-by-email}{%
\subsubsection{Follow by email}\label{follow-by-email}}

Follow

\href{https://wordpress.com/?ref=footer_custom_com}{WordPress.com}.

\protect\hyperlink{}{Up ↑}

Post to

\protect\hyperlink{}{Cancel}

\includegraphics{https://pixel.wp.com/b.gif?v=noscript}
