\protect\hyperlink{content}{Skip to content}

\href{https://swprs.org/}{}

\protect\hyperlink{search-container}{Search}

Search for:

\href{https://swprs.org/}{\includegraphics{https://swprs.files.wordpress.com/2020/05/swiss-policy-research-logo-300.png}}

\href{https://swprs.org/}{Swiss Policy Research}

Geopolitics and Media

Menu

\begin{itemize}
\tightlist
\item
  \href{https://swprs.org}{Start}
\item
  \href{https://swprs.org/srf-propaganda-analyse/}{Studien}

  \begin{itemize}
  \tightlist
  \item
    \href{https://swprs.org/srf-propaganda-analyse/}{SRF / ZDF}
  \item
    \href{https://swprs.org/die-nzz-studie/}{NZZ-Studie}
  \item
    \href{https://swprs.org/der-propaganda-multiplikator/}{Agenturen}
  \item
    \href{https://swprs.org/die-propaganda-matrix/}{Medienmatrix}
  \end{itemize}
\item
  \href{https://swprs.org/medien-navigator/}{Analysen}

  \begin{itemize}
  \tightlist
  \item
    \href{https://swprs.org/medien-navigator/}{Navigator}
  \item
    \href{https://swprs.org/der-propaganda-schluessel/}{Techniken}
  \item
    \href{https://swprs.org/propaganda-in-der-wikipedia/}{Wikipedia}
  \item
    \href{https://swprs.org/logik-imperialer-kriege/}{Kriege}
  \end{itemize}
\item
  \href{https://swprs.org/netzwerk-medien-schweiz/}{Netzwerke}

  \begin{itemize}
  \tightlist
  \item
    \href{https://swprs.org/netzwerk-medien-schweiz/}{Schweiz}
  \item
    \href{https://swprs.org/netzwerk-medien-deutschland/}{Deutschland}
  \item
    \href{https://swprs.org/medien-in-oesterreich/}{Österreich}
  \item
    \href{https://swprs.org/das-american-empire-und-seine-medien/}{USA}
  \end{itemize}
\item
  \href{https://swprs.org/bericht-eines-journalisten/}{Fokus I}

  \begin{itemize}
  \tightlist
  \item
    \href{https://swprs.org/bericht-eines-journalisten/}{Journalistenbericht}
  \item
    \href{https://swprs.org/russische-propaganda/}{Russische Propaganda}
  \item
    \href{https://swprs.org/die-israel-lobby-fakten-und-mythen/}{Die
    »Israel-Lobby«}
  \item
    \href{https://swprs.org/geopolitik-und-paedokriminalitaet/}{Pädokriminalität}
  \end{itemize}
\item
  \href{https://swprs.org/migration-und-medien/}{Fokus II}

  \begin{itemize}
  \tightlist
  \item
    \href{https://swprs.org/covid-19-hinweis-ii/}{Coronavirus}
  \item
    \href{https://swprs.org/die-integrity-initiative/}{Integrity
    Initiative}
  \item
    \href{https://swprs.org/migration-und-medien/}{Migration \& Medien}
  \item
    \href{https://swprs.org/der-fall-magnitsky/}{Magnitsky Act}
  \end{itemize}
\item
  \href{https://swprs.org/kontakt/}{Projekt}

  \begin{itemize}
  \tightlist
  \item
    \href{https://swprs.org/kontakt/}{Kontakt}
  \item
    \href{https://swprs.org/uebersicht/}{Seitenübersicht}
  \item
    \href{https://swprs.org/medienspiegel/}{Medienspiegel}
  \item
    \href{https://swprs.org/donationen/}{Donationen}
  \end{itemize}
\item
  \href{https://swprs.org/contact/}{English}
\end{itemize}

\protect\hyperlink{}{Open Search}

\hypertarget{der-propaganda-multiplikator}{%
\section{Der
Propaganda-Multiplikator}\label{der-propaganda-multiplikator}}

Sprachen: \href{https://swprs.org/the-propaganda-multiplier/}{EN},
\href{https://www.bibliotecapleyades.net/sociopolitica2/sociopol_mediacontrol225.htm}{ES},
\href{https://swprs.org/le-multiplicateur-de-propagande/}{FR},
\href{https://www.bibliotecapleyades.net/sociopolitica2/sociopol_mediacontrol228.htm}{IT},
\href{https://swprs.files.wordpress.com/2019/12/propaganda-multiplier-dutch.pdf}{NL},
\href{https://midtifleisen.wordpress.com/2018/01/04/en-titt-pa-nyhetsbyraenes-rolle/}{NO},
\href{https://wolnemedia.net/powielacze-propagandy/}{PL},
\href{https://revistaopera.com.br/2019/04/23/a-propagacao-hegemonica-como-as-agencias-globais-e-a-midia-ocidental-cobrem-a-geopolitica-parte-1/}{PT},
\href{https://csa.pnzgu.ru/infopswars/ipw1}{RU}

Es ist einer der wichtigsten Aspekte unseres Mediensystems -- und
dennoch in der Öffentlichkeit nahezu unbekannt: Der größte Teil der
internationalen Nachrichten in all unseren Medien stammt von nur drei
globalen Nachrichtenagenturen aus New York, London und Paris.

Die Schlüsselrolle dieser Agenturen hat zur Folge, dass unsere Medien
zumeist über dieselben Themen berichten und dabei sogar oftmals
dieselben Formulierungen verwenden. Zudem nutzen Regierungen, Militärs
und Geheimdienste die globalen Agenturen als Multiplikator für die
weltweite Verbreitung ihrer Botschaften. Die transatlantische Vernetzung
der etablierten Medien gewährleistet dabei, dass die gewünschte
Sichtweise kaum hinterfragt wird.

Eine Untersuchung der Syrien-Berichterstattung von je drei führenden
Tageszeitungen aus Deutschland, Österreich und der Schweiz illustriert
diese Effekte deutlich: 78\% aller Artikel basieren ganz oder teilweise
auf Agenturmeldungen, jedoch 0\% auf investigativer Recherche. Zudem
sind 82\% aller Kommentare und Interviews USA/NATO-freundlich, während
Propaganda ausschließlich auf der Gegenseite verortet wird.

\href{https://swprs.files.wordpress.com/2017/12/der-propaganda-multiplikator-2016-mt.pdf}{Studie
als PDF herunterladen}

\includegraphics{https://swprs.files.wordpress.com/2019/02/propaganda-multiplikator.png?w=512\&h=588}

(Hinweis: Bei Interesse an der Studie bitte auf diese Seite verlinken.
Obige Zusammen­fassung und einzelne Auszüge können übernommen werden.
Keine Volltext-Kopie.)

\begin{center}\rule{0.5\linewidth}{\linethickness}\end{center}

\hypertarget{der-propaganda-multiplikator-1}{%
\subsection{Der
Propaganda-Multiplikator:}\label{der-propaganda-multiplikator-1}}

\hypertarget{wie-globale-nachrichtenagenturen-und}{%
\subsection{Wie globale Nachrichtenagenturen
und}\label{wie-globale-nachrichtenagenturen-und}}

westliche Medien über Geopolitik berichten

Eine Studie von \href{https://swprs.org/}{Swiss Propaganda Research}

Juni 2016

»Man muss sich deshalb immer fragen: Weshalb kommt jetzt\\
gerade diese Information in dieser Form auf mich zu?\\
Letztlich stecken immer Macht­fragen da­hinter.«
(\href{http://www.nzz.ch/wer-lustvoll-schreibt-der-schreibt-auch-gut-1.11329756}{*})\\
Dr. Konrad Hummler, ehemaliger NZZ-Präsident

Inhaltsübersicht

\begin{enumerate}
\def\labelenumi{\arabic{enumi}.}
\tightlist
\item
  \protect\hyperlink{k1}{Teil 1: Der Propaganda-Multiplikator}
\item
  \protect\hyperlink{k2}{Teil 2: Fallstudie zur
  Syrien-Berichterstattung}
\item
  \protect\hyperlink{k3}{Anmerkungen und Literatur}
\end{enumerate}

\hypertarget{einleitung-etwas-eigenartiges}{%
\paragraph{Einleitung: »Etwas
Eigenartiges«}\label{einleitung-etwas-eigenartiges}}

»Woher weiß die Zeitung, was sie weiß?« Die Antwort auf diese Frage
dürfte manchen Zeitungsleser überraschen: »In der Hauptsache bezieht sie
ihr Wissen von Nachrichtenagenturen. Die nahezu anonym arbeitenden
Nachrichtenagenturen sind gewissermaßen der Schlüssel zu den
Ge­scheh­nissen in der Welt. Wer also sind die Nachrichtenagenturen, wie
arbeiten sie und wer finanziert diese Unternehmen? All dies sollte man
wissen, um sich ein Bild machen zu können, ob man auch wirklich über die
Ereignisse in Ost und West zutreffend unterrichtet wird.« (Höhne 1977,
S. 11)

Ein Schweizer Medienforscher gibt deshalb zu bedenken: »Die
Nachrichtenagenturen sind die «AktualiTäter», sind die wichtigsten
Stofflieferanten der Massenmedien. Kein tages­aktuelles Medium kommt
ohne sie aus. () So beeinflussen die Nachrichtenagenturen unser Bild von
der Welt; wir erfahren vor allem das, was sie ausgewählt haben.« (Blum
1995, S. 9)

Angesichts ihrer essentiellen Bedeutung erstaunt es umso mehr, dass
diese Agenturen in der Öffentlichkeit kaum bekannt sind: »Einem Großteil
der Gesellschaft ist nicht klar, dass es Nachrichtenagenturen überhaupt
gibt \ldots{} Dabei nehmen sie tatsächlich eine enorm wichtige Rolle auf
dem Medienmarkt ein. Doch trotz dieser großen Bedeutung wurde ihnen in
der Vergangenheit nur wenig Aufmerksamkeit geschenkt.« (Schulten-Jaspers
2013, S. 13)

Selbst der Vorsitzende einer Nachrichtenagentur wunderte sich: »Es ist
etwas Eigenartiges um die Nachrichtenagenturen. Sie sind in der großen
Öffentlichkeit wenig bekannt. Im Gegensatz beispielsweise zu einer
Zeitung vollzieht sich ihre Tätigkeit nicht so stark im
Scheinwerfer­licht, obwohl sie doch immer an der Quelle der Nachricht zu
finden sind.« (Segbers 2007, S. 9)

\hypertarget{das-unsichtbare-nervenzentrum-des-mediensystems}{%
\paragraph{»Das unsichtbare Nervenzentrum des
Mediensystems«}\label{das-unsichtbare-nervenzentrum-des-mediensystems}}

Wer also sind diese Agenturen, die »immer an der Quelle der Nachricht«
zu finden sind? Globale Agenturen gibt es inzwischen nur noch drei:

\begin{enumerate}
\def\labelenumi{\arabic{enumi}.}
\tightlist
\item
  Die amerikanische \textbf{Associated Press}
  (\href{https://de.wikipedia.org/wiki/Associated_Press}{AP}) mit
  weltweit über 4000 Mitarbeitern. Die AP gehört US-Medienunternehmen
  und hat ihre Hauptredaktion in New York. AP-Nachrichten werden von
  rund 12 000 Medien genutzt und erreichen dadurch täglich
  \href{http://www.ap.org/company/overview}{mehr als die Hälfte der
  Welt­bevölkerung}.
\item
  Die quasi-staatliche französische \textbf{Agence France-Presse}
  (\href{https://de.wikipedia.org/wiki/Agence_France-Presse}{AFP}) mit
  Sitz in Paris und ebenfalls rund 4000 Mitarbeitern. Die AFP versendet
  pro Tag über 3000 Meldungen und 2500 Fotos an Medien in aller Welt.
\item
  Die britische \textbf{Reuters} in London, die privatwirtschaftlich
  organisiert ist und etwas über 3000 Mitarbeiter beschäftigt. Reuters
  wurde 2008 vom kanadischen Medienunternehmer Thomson -- einer der 25
  reichsten Menschen der Welt -- gekauft und zu
  \href{https://de.wikipedia.org/wiki/Reuters}{Thomson-Reuters} mit Sitz
  in New York fusioniert.
\end{enumerate}

Daneben gibt es noch diverse kleinere, nationale Nachrichtenagenturen.
In den deutsch­sprachigen Ländern sind dies insbesondere:

\begin{itemize}
\tightlist
\item
  Die \textbf{Deutsche Presse-Agentur}
  (\href{https://de.wikipedia.org/wiki/Deutsche_Presse-Agentur}{DPA}),
  die als semi-globale Agentur über rund 1000 journalistische
  Mitarbeiter in circa hundert Ländern verfügt. Die DPA ist im Besitz
  von deutschen Medienverlagen und Rundfunkanstalten und hat ihre
  Hauptredaktion seit 2010 im
  \href{http://www.spiegel.de/kultur/gesellschaft/konflikt-um-springer-naehe-tagesspiegel-kuendigt-der-dpa-a-659850.html}{Axel-Springer-Haus}
  in Berlin.
\item
  Die \textbf{Austria Presse Agentur}
  (\href{https://de.wikipedia.org/wiki/Austria_Presse_Agentur}{APA}) mit
  rund 165 Redakteuren. Die APA ist im Besitz von österreichischen
  Tageszeitungen und des ORF.
\item
  Die \textbf{Schweizerische Depeschenagentur}
  (\href{https://de.wikipedia.org/wiki/Schweizerische_Depeschenagentur}{SDA})
  mit rund 150 Mitarbeitern, die im Besitz von Schweizer Medien­verlagen
  ist, darunter die Tamedia und NZZ-Gruppe sowie die SRG.
\end{itemize}

Die SDA und APA verfügen über kein eigenes Korrespondentennetz im
Ausland. Stattdessen
\href{https://web.archive.org/web/20170425121815/http://www.sda.ch/de/dienste/unternehmen-ohne-verlag/internationale-kooperationen}{kooperieren}
sie mit der DPA und den globalen Agenturen, um Zugang zu den
internationalen Nachrichten zu erhalten und damit ihre nationalen Medien
über das Weltgeschehen zu informieren. Die DPA
\href{https://www.presseportal.de/pm/8218/2363699}{kooperiert}
ihrerseits eng mit der amerikanischen AP und besitzt die Lizenz zur
Vermarktung der AP-Dienste im deutsch­sprachigen Raum.

\includegraphics{https://swprs.files.wordpress.com/2017/01/logos_agenturen.png?w=600\&h=259}

\emph{Die Logos der drei Weltagenturen Reuters, AFP und AP, sowie der
drei nationalen Agenturen in Österreich (APA), Deutschland (DPA) und der
Schweiz (SDA).}

Wolfang Vyslozil, der ehemalige Geschäftsführer der APA, beschrieb die
Schlüsselrolle der Nachrichtenagenturen 2005 mit diesen Worten:
»Nachrichtenagenturen stehen selten im Blickpunkt des öffentlichen
Interesses. Dennoch sind sie eine der einflussreichsten und gleichzeitig
eine der am wenigsten bekannten Mediengattungen. Sie sind
Schlüssel­institutionen mit substanzieller Bedeutung für jedes
Mediensystem. Sie sind das unsichtbare Nervenzentrum, das alle Teile
dieses Systems verbindet.« (Segbers 2007, S.10)

\hypertarget{kleines-kuxfcrzel-grouxdfe-wirkung}{%
\paragraph{Kleines Kürzel, große
Wirkung}\label{kleines-kuxfcrzel-grouxdfe-wirkung}}

Es gibt jedoch einen einfachen Grund, warum die globalen Agenturen trotz
ihrer Bedeutung dem brei­ten Publikum so gut wie unbekannt sind, denn:
\emph{»Radio und Fernsehen nennen ihre Quellen in der Regel nicht, und
die Quellenangaben in Zeitschriften entziffern nur Spezialisten.« (Blum
1995, S. 9)}

Das Motiv für diese Zurückhaltung dürfte indes klar sein: Unsere Medien
sind nicht besonders stolz darauf, dass sie viele ihrer Beiträge in
Wirklichkeit gar nicht selbst recherchiert haben, sondern aus den immer
gleichen Quellen beziehen.

Die folgende Abbildung zeigt einige Beispiele zur
Quellen­kenn­zeich­nung in bekannten deutsch­sprachigen Zeitungen. Neben
den Agentur­kürzeln finden sich die Kürzel von Redakteuren, die den
jeweiligen Agenturbericht bearbeitet haben.

\href{https://swprs.files.wordpress.com/2019/02/agenturen-quellen.png}{\includegraphics{https://swprs.files.wordpress.com/2019/02/agenturen-quellen.png?w=736\&h=784}}

\emph{Kürzel der Nachrichtenagenturen in den Quellenangaben von
Zeitungsartikeln}

Hin und wieder verwenden die Zeitungen aber auch Agenturmaterial, ohne
dies klar zu kennzeichnen. Das \emph{Forschungsinstitut für
Öffentlichkeit und Gesellschaft} der Universität Zürich kam in einer
\href{http://www.foeg.uzh.ch/de/jahrbuch.html}{Studie} von 2011 unter
anderem zu folgendem Ergebnis:

»Agenturbeiträge werden integral verwertet, ohne sie zu kennzeichnen,
oder sie werden partiell umgeschrieben, um sie als redaktionelle
Eigenleistung erscheinen zu lassen. Zudem herrscht eine Praxis vor,
Agentur­meldungen mit wenig Aufwand «aufzupeppen»; hierzu werden etwa
Visualisierungstechniken eingesetzt: Ungezeichnete Agentur­meldungen
werden mit Bildern und Grafiken angereichert und als umfangreiche
Berichte dargeboten.« (FOEG 2011)

Dabei spielen die Agenturen nicht nur in der Presse eine heraus­ragende
Rolle, sondern ebenso im privaten und öffentlichen Rundfunk. Dies
\href{http://www.heise.de/tp/artikel/47/47821/3.html}{bestätigt} Volker
Bräutigam, der zehn Jahre für die Tagesschau der \emph{ARD} gearbeitet
hat und die Dominanz der Agenturen kritisch sieht:

»Ein grundsätzliches Problem liegt darin, dass (die
Nach­rich­ten­redaktion) ARD-aktuell ihre Informationen hauptsächlich
aus drei Quellen bezieht: den Nachrichtenagenturen DPA/AP, Reuters und
AFP: Eine deutsche, eine US-amerikanische, eine britische und eine
französische. () Der ein Nachrichten­thema bearbeitende Redakteur kann
gerade noch einige wenige für wesentlich erachtete Text­passagen auf dem
Schirm auswählen, sie neu zusammen­­stellen und mit ein paar Schnörkeln
zusammen­kleben.«

Auch das \emph{Schweizer Radio und Fernsehen (SRF)} richtet sich
weitgehend nach den Meldungen der Agenturen. Auf die Frage von
Zuschauern, weshalb über einen Friedens­marsch in der Ukraine nicht
berichtet werde, gab die Redaktion zur
\href{http://www.srf.ch/sendungen/hallosrf/warum-berichtet-srf-nicht-ueber-den-friedensmarsch-in-der-ukraine}{Antwort}:
\emph{``Bis heute haben wir von den unabhängigen Agenturen Reuters, AP
und AFP, von denen wir Bildmaterial erhalten, keine einzige Meldung und
auch kein Videomaterial von diesem Marsch erhalten. () Falls wir jedoch
Bilder vom Eintreffen des Marsches in Kiew erhalten sollten, werden wir
darüber berichten.''}

Tatsächlich stammen nicht nur die Texte, sondern auch die Bilder, Ton-
und Videoaufnahmen, denen man in unseren Medien Tag für Tag begegnet,
zumeist von denselben Agenturen. Was für das uneingeweihte Publikum wie
ein Beitrag der lokalen Zeitung, des bevorzugten Radiosenders oder der
vertrauten Tagesschau aussieht, sind in Wirklichkeit (übersetzte)
Meldungen aus New York, London, Paris und Berlin.

Manche Medien gingen sogar noch einen Schritt weiter und haben ihre
Auslands­redaktion mangels Ressourcen inzwischen komplett an eine
Agentur
\href{https://web.archive.org/web/20130402032805/http://www.sda.ch/fileadmin/user_upload/domain1/pdf_s/medienmitteilung.pdf}{ausgelagert}.
Auch auf vielen Online-News­portalen werden vorwiegend Agenturberichte
publiziert (vgl. Paterson 2007; Johnston 2011; MacGregor 2013).

Im Endeffekt entsteht durch diese Abhängigkeit von den globalen
Agenturen eine frappierende Gleichartigkeit in der internationalen
Berichterstattung: Von Wien bis Washington berichten unsere Medien
oftmals über dieselben Themen und verwenden dabei sogar vielfach
dieselben Formulierungen -- ein Phänomen, das man sonst eher mit
»gelenkten Medien« in autoritären Staaten in Verbindung bringen würde.

Die folgende Abbildung zeigt einige Beispiele aus deutsch­sprachigen und
internationalen Publikationen. Wie man sieht, schleicht sich trotz der
angestrebten Objektivität der Agenturen mitunter eine leichte
(geo-)politische Färbung ein.

\href{https://swprs.files.wordpress.com/2016/06/schlagzeilen-agenturen-r.png}{\includegraphics{https://swprs.files.wordpress.com/2016/06/schlagzeilen-agenturen-r.png?w=736}}

\emph{»Putin droht«, »Iran provoziert«, »NATO besorgt«,
»Assad-Hochburg«: Inhaltliche und sprachliche Ähnlichkeit der
geopolitischen Bericht­erstattung aufgrund von Meldungen der globalen
Agenturen.}

\hypertarget{die-rolle-der-korrespondenten}{%
\paragraph{Die Rolle der
Korrespondenten}\label{die-rolle-der-korrespondenten}}

Ein Großteil unserer Medien besitzt keine eigenen
Auslands­korrespondenten und hat folglich keine andere Wahl, als sich
für die Auslandsnachrichten vollständig auf die globalen Agenturen zu
verlassen. Doch wie sieht es bei den großen Tageszeitungen und
TV-Stationen aus, die über eigene internationale Korrespondenten
verfügen? Im deutschsprachigen Raum sind dies beispielsweise eine
\emph{NZZ, FAZ, Süddeutsche Zeitung, Welt} und die öffentlichen
Rundfunkanstalten.

Zunächst sind die Größen­verhältnisse im Auge zu behalten: Während die
globalen Agenturen weltweit über mehrere Tausend Mitarbeiter verfügen,
unterhält selbst eine für ihre internationale Berichterstattung bekannte
\emph{NZZ} nur gerade 35 Auslands-Korrespondenten (inklusive den
Wirtschafts-Korrespondenten). In riesigen Ländern wie China oder Indien
ist lediglich ein Korrespondent stationiert, ganz Südamerika wird von
nur zwei Journalisten abgedeckt, während im noch größeren Afrika gar
niemand fest vor Ort ist.

Auch in Kriegsgebiete wagen sich die Korrespondenten nur selten vor.
Über den Syrienkrieg berichten die Journalisten der deutsch­sprachigen
Medien beispielsweise aus Städten wie Istanbul, Beirut, Kairo oder gar
aus Zypern. Zudem fehlt vielen Journalisten die Sprachkenntnis, um
Menschen und Medien vor Ort zu verstehen.

Wie erfahren die Korrespondenten unter solchen Umständen, was die
``Nachrichten'' in ihrer Weltregion sind? Die Antwort lautet zur
Hauptsache einmal mehr: von den globalen Agenturen. Der niederländische
Nahost-Korrespondent Joris Luyendijk hat die Arbeitsweise von
Korrespondenten und ihre Abhängigkeit von den Weltagenturen in seinem
Buch
``\href{https://www.klett-cotta.de/buch/Tropen-Sachbuch/Von_Bildern_und_Luegen_in_Zeiten_des_Krieges/48944}{Von
Bildern und Lügen in Zeiten des Krieges: Aus dem Leben eines
Kriegsberichterstatters}'' eindrücklich beschrieben:

»Ich hatte mir einen Korrespondenten immer als eine Art
Echtzeit-Historiker vorgestellt. Wenn irgendwo etwas Wichtiges geschah,
zog er los, ging der Sache auf den Grund und berichtete darüber. Aber
ich zog nicht los, um irgendeiner Sache auf den Grund zu gehen. Das
hatten andere längst erledigt. Ich zog nur los, um mich als Moderator an
einen Original­schauplatz hinzustellen und die Informationen aufzusagen.
()

Die Redaktion in Holland meldete telefonisch, dass irgendwo etwas los
war. Per Fax oder Mail kamen dann Agenturberichte, die ich im Radio mit
meinen eigenen Worten nacherzählte und für die Zeitung zu einem Artikel
zusammenschrieb. Es war den Redaktionen stets wichtiger, dass ich vor
Ort erreichbar war, als dass ich im Bilde war. Die Presseagenturen
lieferten genug Informationen, um sich durch jede Krise durchzuboxen. ()

Und so kommt es, dass man beim Blättern in Zeitungen und beim Zappen
durch die Fernsehnachrichten häufig immer wieder den gleichen Bildern
und Geschichten begegnet.

Unsere Männer und Frauen in London, Paris, Berlin und Washington -- alle
fanden, dass oft die falschen Themen die Nachrichten beherrschten und
dass wir uns allzu sehr und allzu sklavisch nach den Vorgaben der
Presseagenturen richteten. ()

Die Vorstellung vom Korrespondenten ist, dass er »die Story« hat, aber
in Wirklichkeit sind die Nachrichten wie ein Fließband in der
Brotfabrik. Am hinteren Ende stehen die Korrespondenten, und wir tun hin
und wieder so, als hätten wir die Brötchen selber gebacken, dabei haben
wir sie nur eingetütet. ()

Später wollte ein Freund einmal wissen, wie ich während der ganzen
Interviews mit dem Sender immer ohne zu zögern die richtige Antwort auf
alle Fragen wusste. Als ich ihm schrieb, dass die Fragen genau wie in
den Fernsehnachrichten vorher abgesprochen werden, bekam ich eine E-Mail
voller Verwünschungen zurück, denn meinem Freund dämmerte, was ich
bereits früher erkennen musste: Jahrzehntelang war er in den Nachrichten
auf Schauspieler herein­ge­fallen.« (Luyendjik 2015, K. I.1, I.4, II.5)

Mit anderen Worten: Der typische Korrespondent kann im Allgemeinen keine
eigenständige Recherche betreiben, sondern bearbeitet und verstärkt vor
allem diejenigen Themen, die von den Nachrichten­agenturen ohnehin
vorgegeben werden -- der berüchtigte »Mainstream-Effekt«.

Hinzu kommt, dass sich die Medien im deutschsprachigen Raum ihre wenigen
Auslands-Korrespondenten aus Kostengründen sogar noch teilen müssen: So
greifen die deutsche \emph{Welt} und die österreichische \emph{Presse}
bisweilen auf dieselben Korrespondenten zurück, während die
\emph{Süddeutsche Zeitung} und der Zürcher \emph{Tagesanzeiger} ihre
Korrespondentennetze bereits weitgehend zusammen­gelegt haben. Innerhalb
der einzelnen Medienkonzerne werden die Auslandsberichte dann oft noch
von mehreren Publikationen verwertet -- all dies trägt nicht gerade zur
medialen Vielfalt bei.

\hypertarget{was-die-agentur-nicht-meldet-findet-nicht-statt}{%
\paragraph{»Was die Agentur nicht meldet, findet nicht
statt«}\label{was-die-agentur-nicht-meldet-findet-nicht-statt}}

Die zentrale Rolle der Nachrichtenagenturen erklärt ferner, warum bei
geopolitischen Konflikten die meisten Medien dieselben Quellen
verwenden. Im Syrienkrieg hat es insbesondere die \emph{``Syrische
Beobachtungs­stelle für Menschenrechte''}, eine
\href{http://www.sueddeutsche.de/politik/syrische-beobachtungsstelle-fuer-menschenrechte-ominoese-protokollanten-des-todes-1.1522443}{zweifelhafte
Ein-Mann-Organisation} in London, zu einiger Bekanntheit gebracht. Es
ist nun nicht etwa so, dass sich die Medien einzeln und direkt bei
dieser ``Beobachtungs­stelle'' erkundigen, denn tatsächlich ist ihr
Betreiber selbst für Journalisten oft schwer zu erreichen.

Vielmehr liefert die ``Beobachtungsstelle'' ihre Meldungen an die
globalen Agenturen, die diese sodann an tausende Medien weiterleiten,
welche damit hunderte Millionen von Lesern und Zuschauern weltweit
``informieren''. Warum die Agenturen ihre Informationen ausgerechnet bei
dieser seltsamen ``Beobachtungs­stelle'' beziehen -- und wer diese
wirklich gegründet und finanziert hat -- das ist eine andere Frage, die
jedoch selten gestellt wird.

Der ehemalige leitende DPA-Redakteur Manfred Steffens gibt in seinem
Buch ``Das Geschäft mit der Nachricht'' deshalb zu bedenken:

»Eine Nachricht wird () nicht dadurch richtiger, dass man für sie eine
Quelle angeben kann. Es ist deshalb durchaus fragwürdig, einer Nachricht
deshalb mehr Vertrauen zu schenken, weil eine Quelle zitiert wird. ()
Hinter dem Schutzschild, den so eine ``Quelle'' für eine Nachricht
bedeutet, ist mancher dann durchaus geneigt, auch recht abenteuerliche
Dinge in alle Welt zu verbreiten, selbst wenn er selber berechtigte
Zweifel an ihrer Richtigkeit hegt; die Verantwortung, zu­min­dest
moralisch, kann jederzeit der zitierten Quelle angelastet werden.«
(Steffens 1969, S. 106)

Die Abhängigkeit von den globalen Agenturen ist zudem ein wesentlicher
Grund, warum die mediale Bericht­erstattung zu geopolitischen Konflikten
oftmals oberflächlich und sprunghaft wirkt, während historische
Zusammenhänge und Hintergründe bruchstückhaft bleiben oder gänzlich
fehlen. Denn: \emph{``Nachrichten-Agenturen empfangen ihre Impulse fast
ausschließlich aus dem Tagesgeschehen und sind ihrer Natur nach deshalb
ahistorisch. Würdigenden Rückblicken wird dementsprechend nur ungern
mehr Raum} \emph{gegeben} \emph{als unerlässlich.''} (Steffens 1969:32)
Je mehr sich unsere Medien auf die Agenturen verlassen, desto
episodenhafter gerät mithin ihre eigene Bericht­erstattung (vgl. FÖG
2011).

Schließlich erklärt die Dominanz der globalen Agenturen, warum gewisse
geopolitische Themen und Ereignisse -- die oft nicht so gut ins
atlantische Narrativ passen oder zu ``unwichtig'' sind -- in unseren
Medien überhaupt nicht erwähnt werden: Wenn die Agenturen nicht darüber
berichten, dann erfahren auch die meisten westlichen Medien nichts
davon.

Als beispielsweise die syrische Armee Mitte November 2015 die
Luftwaffenbasis
\emph{\href{https://en.wikipedia.org/wiki/Kuweires_offensive_\%28September\%E2\%80\%93November_2015\%29}{Kuweires}}
nach über zweijähriger Belagerung durch die IS-Miliz zurückeroberte --
einer ihrer ersten strategischen Erfolge seit Eingreifen der russischen
Luftwaffe -- erschien darüber in unseren Medien keinerlei Mitteilung.
Auch der bereits erwähnte
\href{http://www.srf.ch/sendungen/hallosrf/warum-berichtet-srf-nicht-ueber-den-friedensmarsch-in-der-ukraine}{Friedensmarsch
in der Ukraine} schaffte es nicht in das \emph{Schweizer Fernsehen},
weil ``von den Agenturen keine einzige Meldung'' vorlag. Wie anlässlich
des 50. Jubiläums der DPA einmal pointiert angemerkt wurde: »Was die
Agentur nicht meldet, findet nicht statt.« (Wilke 2000, S. 1)

Mitunter werden westliche Darstellungen von Ereignissen auch in Form von
\href{http://www.reuters.com/article/us-mideast-crisis-syria-idlib-toll-idUSKCN0YM0GO}{prominenten
Schlagzeilen} gemeldet, Gegen­darstellungen und Dementis jedoch nicht
erwähnt, unauffällig am Ende einer Meldung platziert oder erst
nachträglich und diskret ergänzt. Dadurch entsteht in den meisten
geopolitischen Konflikten bereits auf Ebene der globalen Agenturen eine
tendenziell einseitige Perspektive, die sich zwangsläufig auf unsere
Medien überträgt.

\hypertarget{fragwuxfcrdige-nachrichten-einschleusen}{%
\paragraph{»Fragwürdige Nachrichten
einschleusen«}\label{fragwuxfcrdige-nachrichten-einschleusen}}

Während einige Themen in unseren Medien gar nicht auftauchen, erscheinen
andere Themen sehr wohl -- obwohl sie es eigentlich nicht sollten:
\emph{»So berichten denn die Massenmedien vielfach gar nicht über die
Wirklichkeit, sondern über eine konstruierte oder inszenierte
Wirklichkeit. () Verschiedene Studien kamen zum Schluss, dass die
Massenmedien überwiegend durch die PR-Aktivitäten der Akteure
determiniert seien und dass bei den Medien­schaffenden die passive,
rezeptive Haltung überwiege und nicht die aktiv-recherchierende.«} (Blum
1995, S. 16)

Tatsächlich ist es aufgrund der eher geringen journalistischen
Eigenleistung unserer Medien und ihrer hohen Abhängigkeit von einigen
wenigen Nachrichtenagenturen für interessierte Kreise ein Leichtes,
Propaganda und Desinformation in einem vermeintlich seriösen Format an
ein weltweites Publikum zu verbreiten. Auch DPA-Redakteur Steffens
warnte vor dieser Gefahr:

»Der kritische Sinn () wird um so mehr eingeschläfert, je angesehener
die Nachrichten-Agentur oder die Zeitung ist, die eine Nachricht bringt.
Derjenige, der eine fragwürdige Nachricht in die Weltpresse einschleusen
will, braucht also nur zu versuchen, seine Nachricht bei einer halbwegs
seriösen Agentur unterzubringen, um sicher zu sein, dass sie dann wenig
später auch bei den anderen auftaucht. Manchmal geschieht es so, dass
eine Falschmeldung von Agentur zu Agentur weiter­gereicht und dabei
immer glaubwürdiger wird.« (Steffens 1969, S. 234)

Zu den aktivsten Akteuren im ``Einschleusen'' von fragwürdigen
geopolitischen Nachrichten gehören dabei die Militärs und
Verteidigungsministerien: Im Jahre 2009 machte der damalige Chef der
amerikanischen Nachrichtenagentur AP, Tom Curley, publik, dass
beispielsweise das Pentagon über mehr als
\href{http://www.tagesanzeiger.ch/ausland/amerika/27000-PRBerater-polieren-Image-der-USA/story/20404513}{27'000
PR-Spezialisten} verfügt, die mit einem Budget von fast fünf Milliarden
Dollar pro Jahr Medien- und Öffentlichkeitsarbeit betreiben. Zudem
hätten hohe US-Generäle gedroht, dass man die AP und ihn persönlich
»ruinieren« werde, falls die Reporter allzu kritisch über das US-Militär
berichten sollten. (Siehe auch: Becker 2015)

Trotz -- oder wegen? -- solcher Drohungen publizieren unsere Medien
regelmäßig Schlagzeilen wie die folgende:

\includegraphics{https://swprs.files.wordpress.com/2017/08/titel-marschflugkc3b6rper.png?w=736}

In diesem
\href{http://www.tagesanzeiger.ch/ausland/asien-und-ozeanien/Vier-russische-Marschflugkoerper-im-Iran-eingeschlagen/story/19373226}{Artikel}
vom Oktober 2015 zum Syrienkrieg geht es nicht etwa darum, dass
tatsächlich vier russische Marschflugkörper im Iran eingeschlagen sind,
obschon der Titel dies nahelegt. Vielmehr handelt es sich um eine
Meldung europäischer Nachrichtenagenturen, wonach zwei namentlich nicht
genannte ``Gewährsleute'' aus ``US-Verteidigungskreisen'' dies gegenüber
der amerikanischen Agentur AP behauptet haben. Belege gibt es keine --
die Geschichte könnte deshalb genauso gut erfunden und Teil einer
Propaganda-Operation gewesen sein. Dennoch werden Berichte wie dieser
von nahezu allen etablierten Medien veröffentlicht -- eigene Recherchen
oder kritische Rückfragen scheinen tabu zu sein (Beispiele zu obiger
Meldung:
\emph{\href{http://www.nzz.ch/international/naher-osten-und-nordafrika/russische-raketen-offenbar-statt-in-syrien-im-iran-eingeschlagen-1.18626791}{NZZ},
\href{http://www.faz.net/aktuell/politik/ausland/naher-osten/auf-dem-weg-nach-syrien-sollen-russische-raketen-in-iran-eingeschlagen-sein-13846834.html}{FAZ},
\href{http://www.welt.de/politik/ausland/article147399821/Russische-Raketen-schlagen-offenbar-im-Iran-ein.html}{Welt},
\href{http://www.spiegel.de/politik/ausland/russische-raketen-laut-usa-in-iran-eingeschlagen-a-1056938.html}{Spiegel},
\href{http://www.focus.de/politik/ausland/us-offizielle-berichten-vier-russische-marschflugkoerper-im-iran-eingeschlagen_id_5001154.html}{Focus}},
\href{http://www.tagesanzeiger.ch/ausland/asien-und-ozeanien/Vier-russische-Marschflugkoerper-im-Iran-eingeschlagen/story/19373226}{\emph{Tagesanzeiger}},
\emph{\href{http://kurier.at/politik/ausland/russische-marschflugkoerper-im-iran-eingeschlagen/157.417.677}{Kurier}},
\href{http://diepresse.com/home/politik/aussenpolitik/4839232/USA_Russische-Marschflugkorper-im-Iran-eingeschlagen-}{\emph{Die
Presse}}).

Ulrich Tilgner, der langjährige Nahost-Korrespondent des \emph{ZDF} und
\emph{Schweizer Fernsehens}, warnte bereits 2003 im Rückblick auf den
Irakkrieg vor medialen Täuschungsmanövern durch die Militärs:

„Mit Hilfe der Medien bestimmen die Militärs die öffentliche Wahrnehmung
und nutzen sie für ihre Planungen. Sie schaffen es, Erwartungen zu
wecken und Szenarien und Täuschungen zu verbreiten. In dieser neuen Art
von Krieg erfüllen die PR-Strategen der US-Administration eine ähnliche
Funktion wie sonst die Bomberpiloten. Die Spezial-Abteilungen für
Öffentlichkeits­­arbeit im Pentagon und in den Geheim­diensten sind zu
Kombattanten im Informationskrieg geworden. () Dabei nutzen die
amerikanischen Militärs die mangelnde Transparenz der Bericht­erstattung
in den Medien gezielt für ihre Täuschungsmanöver. Die von ihnen
gestreuten Informationen, die von Zeitungen und Rundfunk aufgenommen und
verbreitet werden, können Leser, Zuhörer oder Zuschauer unmöglich bis
zur Quelle zurück­verfolgen. Somit gelingt es ihnen nicht, die
ursprüngliche Absicht der Militärs zu erkennen.`` (Tilgner 2003, S. 132)

Was dem US-Militär bekannt ist, das dürfte auch dem US-Geheimdienst
nicht fremd sein. In einer
\href{https://swprs.org/video-the-cia-and-the-media/}{bemerkenswerten
Reportage} des britischen \emph{Channel 4} sprachen ehemalige
Mitarbeiter der CIA und der Agentur Reuters offen über die systematische
Verbreitung von Propaganda und Desinformation in der Berichterstattung
zu geopolitischen Konflikten:

Der ehemalige CIA-Offizier und Whistleblower
\emph{\href{https://en.wikipedia.org/wiki/John_Stockwell}{John
Stockwell}} sagte zu seiner Arbeit im Angola-Krieg: »Das grundsätzliche
Ziel war, es wie eine gegnerische Aggression in Angola aussehen zu
lassen. In diesem Sinne schrieben wir irgendwelche Geschichten und
brachten sie in den Medien unter. () Ein Drittel meines Teams in dieser
Mission waren PR-Experten, deren Aufgabe es war, Nachrichten zu erfinden
und sie in der Presse zu platzieren. () Die Redakteure in den meisten
westlichen Zeitungen sind nicht allzu skeptisch bei Meldungen, die den
allgemeinen Ansichten und Vorurteilen entsprechen. () Einige unserer
Geschichten liefen während Wochen. Aber es war alles erfunden.«

\emph{\href{https://en.wikipedia.org/wiki/Fred_Bridgland}{Fred
Bridgland}} blickte auf seine Tätigkeit als Kriegs­korrespondent bei der
Agentur Reuters zurück: »Wir basierten unsere Berichte auf offiziellen
Mitteilungen. Erst Jahre später erfuhr ich, dass in der US-Botschaft ein
Desinformations-Experte der CIA saß und diese Mitteilungen erfand, die
überhaupt keinen Bezug zur Realität hatten. () Aber ehrlich gesagt, egal
was die Agenturen publizieren, es wird von den Redaktionen sowieso
aufgenommen.«

Und der ehemalige CIA-Analyst
\href{https://en.wikipedia.org/wiki/David_MacMichael}{\emph{David
MacMichael}} beschrieb seine Arbeit im Contra-Krieg in Nicaragua mit
diesen Worten: »Man sagte, unser Geheimdienstwissen über Nicaragua sei
so gut, dass wir sogar das Spülen einer Toilette registrieren können.
Ich hatte aber eher das Gefühl, dass die Geschichten, die wir der Presse
zuspielten, direkt aus der Toilette kamen.« (Hird 1985)

Natürlich verfügen die Geheimdienste auch über zahlreiche
\href{https://swprs.org/der-chefredakteur-und-die-cia/}{direkte
Kontakte} in unseren Medien, denen bei Bedarf ``Informationen''
zugespielt werden können. Doch ohne die zentrale Rolle der globalen
Agenturen wäre die weltweit synchronisierte Verbreitung von Propaganda
und Desinformation unmöglich so effizient realisierbar.

Durch diesen medialen \emph{Propaganda-Multiplikator} erreichen die
zweifelhaften Botschaften der PR-Experten von Regierungen, Militärs und
Geheimdiensten mehr oder weniger ungeprüft und ungefiltert die breite
Öffentlichkeit. Die Journalisten beziehen sich dabei auf die Agenturen,
und die Agenturen berufen sich auf ihre Quellen. Zwar wird oft versucht,
mit Ausdrücken wie ``offenbar'', ``angeblich'' und dergleichen auf
Unsicherheiten hinzuweisen (und sich selbst abzusichern) -- doch da ist
das Gerücht längst in die Welt gesetzt und entfaltet seine Wirkung.

\href{https://swprs.files.wordpress.com/2019/02/propaganda-multiplikator.png}{\includegraphics{https://swprs.files.wordpress.com/2019/02/propaganda-multiplikator.png?w=580\&h=666}}

\emph{Der Propaganda-Multiplikator: Regierungen, Geheimdienste und
Militärs verbreiten ihre Bot­schaften über die globalen Agenturen und
die angeschlossenen Medien an das weltweite Publikum.}

\hypertarget{wie-die-new-york-times-berichtete}{%
\paragraph{Wie die »New York Times«
berichtete\ldots{}}\label{wie-die-new-york-times-berichtete}}

Neben den globalen Nachrichtenagenturen gibt es aber noch eine weitere
Quelle, die oft genutzt wird, um über geopolitische Konflikte zu
berichten: Es sind dies die großen und bekannten Medien in England und
den USA.

Eine \emph{New York Times} oder \emph{BBC} verfügen beispielsweise über
bis zu 100 Auslands­korrespondenten und weitere externe Mitarbeiter.
Nahost­korrespondent Luyendijck gibt jedoch zu bedenken:

``Die westlichen Nachrichtenredaktionen, also auch ich, orientierten
sich an der Nachrichten­auswahl angesehener Medien wie \emph{CNN}, der
\emph{BBC} und der \emph{New York Times}. Sie gingen davon aus, dass
deren Korrespondenten einen Überblick über die arabische Welt hätten.
Wie sich aber herausstellte, konnten viele von ihnen nicht einmal
Arabisch, zumindest nicht genug, um ein Gespräch zu führen oder das
Fernsehen zu verstehen. Das galt für viele Topleute bei \emph{CNN}, der
\emph{BBC}, dem \emph{Independent}, \emph{The Guardian}, \emph{The New
Yorker} und der \emph{New York Times}.'' (Luyendijck 2015, K. I.3)

Hinzu kommt, dass die Quellen dieser Medien oftmals nicht leicht zu
überprüfen sind („Militärkreise``, „anonyme Regierungsbeamte``,
„Geheimdienstmitarbeiter`` und dergleichen) und deshalb ebenfalls für
die Streuung von Propaganda genutzt werden können. Auf jeden Fall aber
führt die verbreitete Orientierung an den angelsächsischen Publikationen
zu einer weiteren Angleichung in der geopolitischen Berichterstattung
unserer Medien.

Die folgende Abbildung zeigt einige Beispiele solcher Zitierungen anhand
der Syrien-Bericht­erstattung des \emph{Zürcher Tagesanzeigers} --
immerhin die größte Tageszeitung der Schweiz und wie erwähnt ein Partner
der \emph{Süddeutschen Zeitung}. Die Artikel stammen alle aus den ersten
Oktobertagen 2015, als Russland direkt in den Syrienkrieg eingriff:

\href{https://swprs.files.wordpress.com/2017/01/us-medien.png}{\includegraphics{https://swprs.files.wordpress.com/2017/01/us-medien.png?w=600\&h=616}}

\emph{Häufiges Zitieren von britischen und amerikanischen Medien, am
Beispiel der Syrien­berichterstattung des Zürcher Tagesanzeigers von
Anfang Oktober 2015.}

\hypertarget{das-gewuxfcnschte-narrativ}{%
\paragraph{Das gewünschte Narrativ}\label{das-gewuxfcnschte-narrativ}}

Doch warum versuchen Journalisten in unseren Medien nicht einfach,
selbstständig zu recherchieren und unabhängig von den globalen Agenturen
und den angel­sächsischen Medien zu berichten? Nahost-Korrespondent
Luyendijk beschreibt seine diesbezüglichen Erfahrungen:

»Jetzt könnte man fragen: Warum sucht er (der Journalist) sich dann
nicht vernünftige Quellen? Das habe ich ja probiert, aber immer wenn ich
an den Presseagenturen, den großen angel­sächsischen Medien und den
\emph{talking heads} (westliche Gesprächspartner und NGOs in arabischen
Ländern) vorbei eine Reportage machen wollte, ging das daneben. ()
Offenbar konnte ich als Korrespondent ganz verschiedene Geschichten über
ein und dieselbe Situation erzählen. Doch die Medien konnten nur eine
davon bringen, und oft genug war das genau diejenige Geschichte, die das
bereits vorherrschende Bild bestätigte.« (Luyendijk 2015, K. I.3)

Der Medienforscher Noam Chomsky hat diesen Effekt in seinem Aufsatz
``\href{https://chomsky.info/199710__/}{Was die Mainstream-Medien zum
Mainstream macht''} wie folgt beschrieben: »Wenn du die offizielle Linie
verlässt, wenn du abweichende Berichte produzierst, dann wirst du das
bald zu spüren bekommen. () Es gibt viele Möglichkeiten, wie man dich
rasch wieder auf Linie bekommt. Wenn du die Vorgaben nicht beachtest,
dann wirst du deine Stelle nicht lange behalten. Dieses System
funktioniert ziemlich gut, und es widerspiegelt die etablierten
Machtstrukturen.« (Chomsky 1997)

Auch in deutschsprachigen Medien sind solche Fälle dokumentiert. Als
beispielsweise ein \emph{ARD}-Korrespondent im Libanesischen Bürgerkrieg
über die mit eigenen Augen beobachtete materielle Unterstützung einer
Miliz durch ein Nachbarland berichtete, kontaktierte der damalige
Intendant der \emph{ARD} ob des politisch inopportunen Beitrags sogleich
den Chefredakteur, welcher dem betroffenen Journalisten umgehend einen
Wechsel nach Südafrika nahelegte. (Mükke 2014, S.33)

Ebenso
\href{http://www.berliner-zeitung.de/korrespondent-ulrich-tilgner-sucht-mehr-distanz-zum-zdf--ich-fuehle-mich-eingeschraenkt--15870684}{verließ}
der langjährige Nahost- und Afghanistan-Korrespondent Ulrich Tilgner das
\emph{ZDF}, weil er aufgrund von »Bündnisrücksichten« und »Eingriffen in
seine Arbeit« nicht mehr frei berichten
\href{https://web.archive.org/web/20080211184752/http://www.neue-oz.de/information/noz_print/interviews/tilgner.html}{konnte}:
»{[}Früher{]} wollte man wissen: Was haben die Leute vor Ort zu sagen?
Heute werden Beiträge nur zu oft in den Redaktionen zusammengebaut und
der Sendeablauf wird designed.« (siehe auch: Tilgner 2003)

Auch die langjährige Nahost-Korrespondentin Karin Leukefeld
\href{https://www.watson.ch/International/Kommentar/148360008-Spielball-der-M\%C3\%A4chte--Weshalb-der-Syrien-Konflikt-in-erster-Linie-ein-Stellvertreterkrieg-ist}{machte
die Erfahrung}, dass ihre Reportagen aus Syrien von deutschen
Redaktionen plötzlich nicht mehr angenommen wurden. Man beschied ihr,
sie müsse sich an die »einschlägigen Agenturmeldungen« halten.

In der Schweiz traf es unter anderem den ehemaligen Auslandschef der
\emph{Weltwoche}, der den ``Fehler'' beging, mitten im Bosnienkrieg über
nachweisliche Kriegslügen der westlichen Allianz zu berichten: Er
erhielt ein Schreibverbot und sah sich mit seiner möglichen Absetzung
\href{https://swprs.org/das-gewuenschte-narrativ}{konfrontiert}. Zuletzt
``erwischte'' es den ehemaligen \emph{Tagesschau}-Korrespondenten Helmut
Scheben, der auf dem Newsportal \emph{Watson} -- ein Partner von
\emph{Spiegel Online} -- die westliche Berichterstattung zu Syrien als
einseitig und manipulativ kritisierte. Keine zwei Tage später wurde er
in einem aufgebrachten Rückruf von der Redaktion als »Putin-Troll«
\href{https://swprs.org/das-gewuenschte-narrativ-ii/}{beschimpft}.

Trotz solcher Beispiele sind einige der
\href{https://www.youtube.com/watch?v=xmPCwjrJjRc}{führenden
Journalisten} weiterhin der Ansicht, ihnen würde niemand vorschreiben,
was sie zu sagen haben. Wie passt dies zusammen? Medienforscher Chomsky
klärt den scheinbaren Widerspruch
\href{https://chomsky.info/199710__/}{auf}:

»Der entscheidende Punkt ist folgender: Diese Journalisten wären ihren
Job längst los, wenn sie nicht schon lange bewiesen hätten, dass ihnen
niemand sagen muss, was sie zu schreiben haben -- weil sie ohnehin das
``Richtige'' schreiben werden. Wenn sie zu Beginn ihrer Karriere die
``falschen'' Storys verfolgt hätten, wären sie gar nicht erst in die
Position gekommen, in der sie jetzt ``alles sagen können, was sie
wollen''. () Mit anderen Worten: Diese Journalisten durchliefen bereits
einen Sozialisierungsprozess.« (Chomsky 1997)

Im Endeffekt führt dieser »Sozialisierungsprozess« zu einem
Journalismus, der über geopolitische Konflikte (und einige andere
Themen) im Allgemeinen nicht mehr unabhängig recherchiert und kritisch
berichtet, sondern mittels
\href{https://swprs.org/die-nzz-studie/}{geeigneter Leitartikel,
Kommentare und Interviewpartner} das gewünschte Narrativ zu festigen
versucht. (Siehe auch: Gritsch 2010)

\hypertarget{fazit-das-erste-gesetz-des-journalismus}{%
\paragraph{Fazit: Das »Erste Gesetz des
Journalismus«*}\label{fazit-das-erste-gesetz-des-journalismus}}

\begin{itemize}
\item
\end{itemize}

Der ehemalige AP-Journalist Herbert Altschull nannte es das \emph{Erste
Gesetz des Journalismus}: »In allen Pressesystemen sind die
Nachrichtenmedien Instru­mente derer, die die politische und
wirtschaftliche Macht ausüben. Zeitungen, Zeitschriften, Radio- und
Fernsehsender handeln also nicht unabhängig, obwohl sie die Möglichkeit
unabhängiger Machtausübung besitzen.« (Altschull 1984/1995, S. 298)

Insofern ist es folgerichtig, wenn unsere etablierten Medien -- die ja
überwiegend durch Werbung finanziert oder aber quasi-staatlich
organisiert sind -- die geopolitischen Interessen der
Transatlantik-Allianz vertreten. Denn sowohl die werbetreibenden Banken
und Konzerne wie auch die Staaten selbst sind \emph{nolens volens} auf
die transatlantische Wirtschafts- und Sicherheitsarchitektur
amerikanischer Prägung angewiesen.

Zudem sind unsere führenden Medien bzw. deren Schlüsselpersonen -- ganz
im Sinne von Chomskys »Sozialisierung« -- oftmals selbst in die
Netzwerke der trans­atlantischen Elite eingebunden. Die folgende
Abbildung illustriert dies am Beispiel der
\href{https://de.wikipedia.org/wiki/Atlantik-Br\%C3\%BCcke}{Atlantik-Brücke}
und der jährlichen
\href{https://de.wikipedia.org/wiki/Bilderberg-Konferenz}{Bilderberg-Konferenz}
-- zwei der wichtigsten derartigen Institutionen (siehe auch: Krüger
2013).

\href{https://swprs.files.wordpress.com/2017/01/medien-transatlantik-netzwerke.png}{\includegraphics{https://swprs.files.wordpress.com/2017/01/medien-transatlantik-netzwerke.png?w=736\&h=433}}

*Einige der führenden deutsch­sprachigen Medien, von denen
Schlüssel­personen (Herausgeber, Geschäftsführer, Chefredakteure,
Journalisten etc.) in die transatlantischen Netzwerke der
Atlantik-Brücke oder der Bilderberg-Konferenz eingebunden sind oder
waren (\href{https://swprs.org/netzwerk-medien-deutschland/}{mehr
dazu}).\\
*

Die meisten etablierten Publikationen sind mithin der Sparte
\href{https://swprs.org/netzwerk-medien-deutschland/}{»Transatlantik-Medien«}
zuzurechnen. Vielleicht gerade deswegen ging dieser wichtige Aspekt
bisweilen vergessen und es entstand beim Publikum der Eindruck einer
vermeintlichen Vielfalt, die es jedoch insbesondere im tagesaktuellen
Bereich eigentlich nie gab: Der theoretischen Medienfreiheit standen
hier allzu hohe praktische Eintrittshürden entgegen (Senderkonzessionen,
limitierte Frequenz- und Programmplätze, Anforderungen an Finanzierung
und technische Infrastruktur, beschränkte Verkaufskanäle, Abhängigkeit
von Werbung und Agenturen etc.).

Erst durch das Internet ist Altschulls \emph{Erstes Gesetz} ein Stück
weit durchbrochen worden. In den letzten Jahren konnte so ein qualitativ
hochwertiger, leserfinanzierter Journalismus
\href{https://swprs.org/medien-navigator/}{entstehen}, der die
herkömmlichen Medien in Bezug auf kritische Bericht­erstattung und
Ausleuchtung von Hintergründen teils deutlich übertrifft. Einige dieser
»alternativen« Publikationen erreichen inzwischen allein im
deutsch­sprachigen Raum über 100'000 Leser und Zuschauer, was zeigt,
dass die »Masse« für die Qualität eines Mediums keineswegs ein Problem
sein muss -- im Gegenteil.

Andererseits gelang es den etablierten Medien bislang, auch im Internet
die große Mehrzahl der Besucher auf sich zu vereinen (siehe Statistiken
für die
\href{http://www.itmagazine.ch/Artikel/61130/Die_meistbesuchten_Schweizer_Websites_Top-5_unveraendert.html}{Schweiz},
\href{http://meedia.de/2016/02/09/ivw-news-top-50-januar-bringt-rekorde-fuer-focus-welt-faz-und-viele-andere/}{Deutschland,}
und
\href{http://www.oewa.at/basic/online-angebote?date=20160401}{Österreich}).
Dies hängt wiederum eng mit den Nachrichtenagenturen zusammen, deren
stets aktuelle Meldungen das Rückgrat der meisten Newsportale bilden.

Die kommenden Jahre werden es zeigen: Wird die »politische und
wirtschaftliche Macht« gemäß Altschulls »Gesetz« die Kontrolle über die
Nachrichten behalten -- oder werden »unkontrollierte« Nachrichten das
politische und ökonomische Machtgefüge verändern?

\begin{center}\rule{0.5\linewidth}{\linethickness}\end{center}

\hypertarget{fallstudie-syrien-berichterstattung}{%
\subsubsection{Fallstudie:
Syrien-Berichterstattung*}\label{fallstudie-syrien-berichterstattung}}

\begin{itemize}
\item
\end{itemize}

Im Rahmen einer Fallstudie wurde die Syrien-Berichterstattung von je
drei führenden Tageszeitungen aus Deutschland, Österreich und der
Schweiz auf Vielseitigkeit und journalistische Eigenleistung hin
untersucht. Ausgewählt wurden hierfür die folgenden Zeitungstitel:

\begin{itemize}
\tightlist
\item
  \textbf{Für Deutschland:} Die \emph{Welt}, die \emph{Süddeutsche
  Zeitung} (SZ), und die \emph{Frankfurter Allgemeine Zeitung} (FAZ)
\item
  \textbf{Für die Schweiz:} Die \emph{Neue Zürcher Zeitung} (NZZ), der
  Zürcher \emph{Tagesanzeiger} (TA), und die \emph{Basler Zeitung} (BaZ)
\item
  \textbf{Für Österreich:} Der \emph{Standard}, der \emph{Kurier}, und
  die \emph{Presse}
\end{itemize}

Als Untersuchungs­zeitraum wurde der 1. bis 15. Oktober 2015 definiert,
d.h. die ersten beiden Wochen nach dem direkten Eingreifen Russlands in
den Konflikt. Berücksichtigt wurde die gesamte Print- und
Online-Berichterstattung der genannten Zeitungen. Nicht berücksichtigt
wurden allfällige Sonntags­ausgaben, da nicht alle untersuchten
Zeitungen über eine solche verfügen. Insgesamt entsprachen 381
Zeitungs­artikel den genannten Kriterien.

In einem ersten Schritt wurden die Artikel anhand ihrer Eigenschaften in
folgende Gruppen eingeteilt:

\begin{enumerate}
\def\labelenumi{\arabic{enumi}.}
\tightlist
\item
  \textbf{Agenturen}: Meldungen und Berichte von Nachrichtenagenturen
  (mit Agenturkürzel)
\item
  \textbf{Berichte/Agenturen}: Einfache Berichte (mit Autorennamen), die
  ganz oder teilweise auf Meldungen von Agenturen basieren
\item
  \textbf{Hintergrund}: Redaktionelle Hintergrundberichte und Analysen
\item
  \textbf{Meinungen/Kommentare}: Meinungsbeiträge und Gastkommentare
\item
  \textbf{Interviews}: Interviews mit Experten, Politikern etc.
\item
  \textbf{Investigativ}: Investigative Recherchen, die neue
  Informationen oder Zusammenhänge aufdecken
\end{enumerate}

Die folgende Abbildung 1 zeigt die Zusammensetzung der Artikel für die
neun untersuchten Zeitungen insgesamt. Wie ersichtlich bestanden 55\%
der Artikel aus Meldungen und Berichten von Nachrichtenagenturen; 23\%
aus redaktionellen Berichten auf Basis von Agenturmaterial; 9\% aus
Hintergrundberichten; 10\% aus Meinungen und Gastkommentaren; 2\% aus
Interviews; und 0\% aus investigativen Recherchen.

\href{https://swprs.files.wordpress.com/2017/07/artikel-gesamt.png}{\includegraphics{https://swprs.files.wordpress.com/2017/07/artikel-gesamt.png?w=736\&h=447}}

\emph{Abbildung 1: Artikelarten und journalistische Eigenleistung
(insgesamt; n=381)}

Die reinen Agenturtexte -- von der kurzen Meldung bis hin zum
ausführlichen Bericht -- befanden sich dabei mehrheitlich auf den
Internet­seiten der Tageszeitungen: Auf diesen ist einerseits der
Aktualitätsdruck höher als in der gedruckten Ausgabe, andererseits
bestehen keine Platzbeschränkungen. Die meisten übrigen Artikelarten
fanden sich sowohl in der Online- wie in der gedruckten Ausgabe; einige
exklusive Interviews und Hintergrundberichte fanden sich nur in den
gedruckten Ausgaben. Sämtliche Artikel wurden für die Untersuchung nur
einmal erfasst.

Die folgende Abbildung 2 zeigt dieselbe Klassifikation aufgeschlüsselt
pro Zeitung. Die meisten Zeitungen publizierten im Beobachtungszeitraum
(zwei Wochen) zwischen 40 und 50 Artikel zum Syrienkonflikt (Print und
Online). Einzig bei der deutschen \emph{Welt} waren es mehr (58), bei
der \emph{Basler Zeitung} und dem österreichischen \emph{Kurier}
hingegen deutlich weniger (29 bzw. 33).

Der Anteil der Agenturmeldungen liegt je nach Zeitung bei knapp 50\%
(\emph{Welt, Süddeutsche, NZZ, Basler Zeitung}), knapp 60\% (\emph{FAZ,
Tagesanzeiger}), und 60 bis 70\% (\emph{Presse, Standard, Kurier}).
Zusammen mit den agenturbasierten Berichten liegt der Anteil bei den
meisten Zeitungen zwischen circa 70\% (\emph{Basler Zeitung}) und 80\%,
bei der österreichischen \emph{Presse} bei 88\%. Diese Anteile decken
sich mit früheren medien­wissen­schaftlichen Studien (siehe z.B. Blum
1995; Johnston 2011; MacGregor 2013; Paterson 2007).

Bei den Hintergrundberichten lagen die Schweizer Zeitungen vorne (fünf
bis sechs Stück), gefolgt von der \emph{Welt}, der \emph{Süddeutschen}
und dem \emph{Standard} (je vier) sowie den übrigen Zeitungen (ein bis
drei). Die Hintergrundberichte und Analysen widmeten sich insbesondere
der Situation und Entwicklung im Nahen Osten, sowie den Motiven und
Interessen einzelner Akteure (z.B. Russland, Türkei, »Islamischer
Staat«).

Am meisten Kommentare waren indessen bei den deutschen Zeitungen zu
beobachten (je sieben Kommentare), gefolgt vom Standard (fünf), der
\emph{NZZ} und dem \emph{Tagesanzeiger} (je vier). Die \emph{Basler
Zeitung} brachte im Beobachtungszeitraum keinen Kommentar, dafür zwei
Interviews. Weitere Interviews führten der \emph{Standard} (drei) sowie
der \emph{Kurier} und die \emph{Presse} (je eines). Investigative
Recherchen konnten hingegen bei keiner der Zeitungen festgestellt
werden.

Insbesondere bei den drei deutschen Zeitungen wurde zudem eine aus
journalistischer Sicht problematische Vermischung von Kommentaren und
Berichten festgestellt, d.h. Berichte enthielten starke
Meinungs­äußerungen, obwohl sie nicht als Kommentar gekennzeichnet
waren. Die Erfassung für die vorliegende Studie basierte aber in jedem
Fall auf der Artikel-Kennzeichnung durch die Zeitung.

\href{https://swprs.files.wordpress.com/2017/07/artikel-zeitung.png}{\includegraphics{https://swprs.files.wordpress.com/2017/07/artikel-zeitung.png?w=736\&h=373}}

\emph{Abbildung 2: Artikelarten pro Zeitung}

Die folgende Abbildung 3 zeigt die Aufschlüsselung der Agenturmeldungen
(anhand der Agenturkürzel) auf die einzelnen Nachrichtenagenturen,
insgesamt und pro Land. Die 211 Agentur­meldungen trugen insgesamt 277
Agenturkürzel (eine Meldung kann aus Material von mehr als einer Agentur
bestehen). Gesamthaft stammten 24\% der Agenturmeldungen von der AFP; je
rund 20\% von der DPA, APA und Reuters; 9\% von der SDA; 6\% von der AP;
und 11\% waren unbekannt (keine Kennzeichnung oder pauschal
``Agenturen'').

In Deutschland teilen sich die DPA, AFP und Reuters je etwa einen
Drittel der Meldungen. In der Schweiz führen die SDA und die AFP, und in
Österreich die APA und Reuters.

Tatsächlich dürften die Anteile der globalen Agenturen AFP, AP und
Reuters noch höher liegen, da die schweizerische SDA und die
österreichische APA ihre internationalen Meldungen hauptsächlich von den
globalen Agenturen beziehen und die deutsche DPA ihrerseits eng mit der
amerikanischen AP kooperiert.

Anzumerken ist noch, dass die globalen Agenturen aus historischen
Gründen unterschiedlich stark in den verschiedenen Weltregionen
vertreten sind. Bei Ereignissen in Asien, der Ukraine oder in Afrika
wird der Anteil der einzelnen Agenturen deshalb ein anderer sein als bei
Ereignissen im Nahen Osten.

\href{https://swprs.files.wordpress.com/2017/07/anteil-agenturen.png}{\includegraphics{https://swprs.files.wordpress.com/2017/07/anteil-agenturen.png?w=736\&h=443}}

\emph{Abbildung 3: Anteil der Nachrichtenagenturen, insgesamt (n=277)
und pro Land.}

In einem nächsten Schritt wurde anhand der zentralen Aussagen die
Ausrichtung von redaktionellen Meinungsbeiträgen (28), Gastkommentaren
(10) und Interviewpartnern (7) bewertet (insgesamt 45 Artikel). Wie
Abbildung 4 zeigt, fielen 82\% der Beiträge grundsätzlich
USA/NATO-freundlich aus, 16\% waren neutral oder ausgewogen, und 2\%
waren überwiegend USA/NATO-kritisch.

Zu den Gastkommentatoren und Interviewpartnern zählten unter anderem ein
ehemaliger NATO-Generalsekretär (``Staaten der Region sollen Truppen
stellen''), der Präsident des amerikanischen \emph{Council on Foreign
Relations} (``Syrien und Europas Krise''), der Direktor einer
US-Menschenrechtsorganisation (``Assad kann nicht bleiben''), die
Büro-Leiterin eines NATO-affinen \emph{Think Tanks} (``Die Spieler von
Damaskus''), ein ehemaliger \emph{Senior Transatlantic Fellow} beim
\emph{German Marshall Fund} (``Putin und Syrien: Der Diplomatie eine
Gasse''), ein ehemaliger deutscher Außenminister und Befürworter des
Kosovo-Krieges (``Europa darf seine Grundwerte nicht opfern''), sowie
ein \emph{Senior Associate} am \emph{Carnegie Moscow Center} (``Putins
krummer Weg nach Damaskus''). Hinzu kamen einige Akademiker an
westlichen Universitäten, russische (zumeist Kreml-kritische)
Intellektuelle, sowie der nach Europa geflüchtete Direktor des
Archäologischen Museums von Aleppo (``Die Zerstörung der Antike wird
weitergehen'').

Beim einzigen überwiegend USA/NATO-kritischen Beitrag handelte es sich
um einen redaktionellen Meinungsbeitrag im österreichischen
\emph{Standard} vom 2. Oktober 2015 mit dem Titel: \emph{``Die Strategie
des Regime Change ist gescheitert. Eine Unterscheidung in ``gute'' und
``schlechte'' Terrorgruppen in Syrien macht die westliche Politik
unglaubwürdig.''}

\href{https://swprs.files.wordpress.com/2017/07/kommentare-interviews-gesamt.png}{\includegraphics{https://swprs.files.wordpress.com/2017/07/kommentare-interviews-gesamt.png?w=736\&h=443}}

\emph{Abbildung 4: Grundsätzliche Ausrichtung der Meinungs­beiträge,
Gastkommentare und Interview­partner (insgesamt; n=45).}

Die folgende Abbildung zeigt die Ausrichtung der Meinungsbeiträge,
Gastkommentare und Interviewpartner wiederum aufgeschlüsselt auf die
einzelnen Zeitungen. Wie ersichtlich brachten die \emph{Welt}, die
\emph{Süddeutsche Zeitung}, die \emph{NZZ}, der Zürcher
\emph{Tagesanzeiger} und der österreichische \emph{Kurier}
ausschließlich USA/NATO-freundliche Meinungs- und Gastbeiträge, die
\emph{FAZ} mit Ausnahme eines neutralen/ ausgewogenen Beitrags
ebenfalls. Der \emph{Standard} brachte vier USA/NATO-freundliche, drei
ausgewogene/ neutrale, sowie den bereits genannten USA/NATO-kritischen
Meinungsbeitrag. Die \emph{Presse} brachte als einzige der untersuchten
Zeitungen überwiegend neutrale/ ausgewogene Meinungs- und Gastbeiträge.
Die \emph{Basler Zeitung} brachte je einen USA/NATO-freundlichen und
einen ausgewogenen Beitrag. Kurz nach dem Beobachtungszeitraum (am 16.
Oktober 2015) erschien in der \emph{Basler Zeitung} zudem ein Interview
mit dem Präsidenten des russischen Parlaments. Dieses hätte selbstredend
als ein USA/NATO-kritischer Beitrag gezählt.

\href{https://swprs.files.wordpress.com/2017/01/kommentare-interviews-zeitung.png}{\includegraphics{https://swprs.files.wordpress.com/2017/01/kommentare-interviews-zeitung.png?w=736\&h=395}}

\emph{Abbildung 5: Grundsätzliche Ausrichtung der Meinungsbeiträge,
Gastkommentare und Interviewpartner (insgesamt).}

In einer weiteren Analyse wurde mittels einer Volltext-Stichwortsuche
nach ``Propaganda'' (und Wortkombinationen damit) untersucht, in welchen
Fällen die untersuchten Zeitungen selbst Propaganda bei einer der beiden
geopolitischen Konfliktparteien USA/NATO oder Russland identifizierten
(nicht berücksichtigt wurde die Konfliktpartei ``IS/ISIS''). Insgesamt
wurden zwanzig solcher Fälle ermittelt. Abbildung 6 zeigt das Ergebnis:
Demnach wurde in 85\% der Fälle die Propaganda auf Seiten der
Konfliktpartei Russland identifiziert, in 15\% war die Verortung neutral
oder unbestimmt, und in 0\% der Fälle wurde Propaganda auf Seiten der
Konfliktpartei USA/NATO verortet.

Anzumerken ist, dass es in ca. der Hälfte der Fälle (neun) die
\emph{NZZ} war, die von russischer Propaganda sprach
(\emph{``Kreml-Propaganda'', ``Moskauer Propagadamaschine'',
``Propagandamärchen'', ``russischer Propagandaapparat'' etc.}), gefolgt
von der \emph{FAZ} (drei), der \emph{Welt} und der \emph{Süddeutschen}
(je zwei) und dem \emph{Kurier} (einmal). Die anderen Zeitungen sprachen
nicht von Propaganda oder nur in einem neutralen Kontext (oder im
Zusammenhang mit dem IS).

\href{https://swprs.files.wordpress.com/2017/01/verortung-propaganda.png}{\includegraphics{https://swprs.files.wordpress.com/2017/01/verortung-propaganda.png?w=736\&h=443}}

\emph{Abbildung 6: Verortung von Propaganda durch die untersuchten
Zeitungen (insgesamt; n=20).}

\hypertarget{fazit}{%
\paragraph{Fazit}\label{fazit}}

In dieser Fallstudie wurde am Beispiel des Syrienkriegs die
geopolitische Berichterstattung von je drei führenden Tageszeitungen aus
Deutschland, Österreich und der Schweiz auf Vielfältigkeit und
journalistische Eigenleistung hin untersucht.

Die Resultate bestätigen die hohe Abhängigkeit der geopolitischen
Berichterstattung von den globalen Nachrichtenagenturen (63 bis 90\%;
ohne Kommentare und Interviews) bei gleichzeitigem Fehlen von eigener
investigativer Recherche, sowie die ein­seitige Kommentierung der
Ereig­nisse zugunsten der Konfliktpartei USA/NATO (82\% positiv vs. 2\%
kritisch), deren Botschaften von den Zeitungen zudem nicht auf
allfällige Propaganda hin überprüft werden.

\hypertarget{studie-als-pdf-herunterladen}{%
\paragraph{\texorpdfstring{\href{https://swprs.files.wordpress.com/2017/12/der-propaganda-multiplikator-2016-mt.pdf}{Studie
als PDF
herunterladen}}{Studie als PDF herunterladen}}\label{studie-als-pdf-herunterladen}}

\begin{center}\rule{0.5\linewidth}{\linethickness}\end{center}

\hypertarget{literatur}{%
\subsubsection{Literatur}\label{literatur}}

Altschull, Herbert J. (1984/1995): Agents of power. The media and public
policy. \emph{Longman,} New York.

Becker, Jörg (2015): Medien im Krieg -- Krieg in den Medien.
\emph{Springer Verlag für Sozialwissenschaften,} Wiesbaden.

Blum, Roger et al. (Hrsg.) (1995): Die AktualiTäter.
Nachrichtenagenturen in der Schweiz. \emph{Verlag Paul Haupt,} Bern.

Chomsky, Noam (1997): What Makes Mainstream Media Mainstream. \emph{Z
Magazine,} MA. (\href{https://chomsky.info/199710__/}{PDF})

Forschungsinstitut für Öffentlichkeit und Gesellschaft der Universität
Zürich (FOEG) (2011): Jahrbuch Qualität der Medien, Ausgabe 2011.
\emph{Schwabe,} Basel.
(\href{http://www.foeg.uzh.ch/de/jahrbuch.html}{PDF})

Gritsch, Kurt (2010): Inszenierung eines gerechten Krieges?
Intellektuelle, Medien und der ``Kosovo-Krieg'' 1999. \emph{Georg Olms
Verlag,} Hildesheim.

Hird, Christopher (1985): Standard Techniques. \emph{Diverse Reports,
Channel 4 TV.} 30. Oktober 1985.
(\href{https://swprs.org/video-the-cia-and-the-media/}{Link})

Höhne, Hansjoachim (1977): Report über Nachrichtenagenturen. Band 1: Die
Situation auf den Nachrichtenmärkten der Welt. Band 2: Die Geschichte
der Nachricht und ihrer Verbreiter. \emph{Nomos Verlagsgesellschaft,}
Baden-Baden.

Johnston, Jane \& Forde, Susan (2011): The Silent Partner: News Agencies
and 21st Century News. \emph{International Journal of Communication 5
(2011),} p. 195--214.
(\href{https://ijoc.org/index.php/ijoc/article/view/928}{PDF})

Krüger, Uwe (2013): Meinungsmacht. Der Einfluss von Eliten auf
Leitmedien und Alpha-Journalisten -- eine kritische Netzwerkanalyse.
\emph{Herbert von Halem Verlag,} Köln.

Luyendijk, Joris (2015): Von Bildern und Lügen in Zeiten des Krieges:
Aus dem Leben eines Kriegsberichterstatters -- Aktualisierte Neuausgabe.
\emph{Tropen,} Stuttgart.

MacGregor, Phil (2013): International News Agencies. Global eyes that
never blink. In: Fowler-Watt/Allan (ed.): Journalism: New Challenges.
\emph{Centre for Journalism \& Communication Research,} Bournemouth
University.
(\href{https://microsites.bournemouth.ac.uk/cjcr/files/2013/10/JNC-2013-Chapter-3-MacGregor.pdf}{PDF})

Mükke, Lutz (2014): Korrespondenten im Kalten Krieg. Zwischen Propaganda
und Selbstbehauptung. \emph{Herbert von Halem Verlag,} Köln.

Paterson, Chris (2007): International news on the internet. \emph{The
International Journal of Communication Ethics.} Vol 4, No 1/2 2007.
(\href{http://www.communicationethics.net/journal/v4n1-2/v4n1-2_12.pdf}{PDF})

Queval, Jean (1945): Première page, Cinquième colonne. \emph{Arthème
Fayard,} Paris.

Schulten-Jaspers, Yasmin (2013): Zukunft der Nachrichtenagenturen.
Situation, Entwicklung, Prognosen. \emph{Nomos,} Baden-Baden.

Segbers, Michael (2007): Die Ware Nachricht. Wie Nachrichtenagenturen
ticken. \emph{UVK,} Konstanz.

Steffens, Manfred {[}Ziegler, Stefan{]} (1969): Das Geschäft mit der
Nachricht. Agenturen, Redaktionen, Journalisten. \emph{Hoffmann und
Campe}, Hamburg.

Tilgner, Ulrich (2003): Der inszenierte Krieg -- Täuschung und Wahrheit
beim Sturz Saddam Husseins. \emph{Rowohlt}, Reinbek.

Wilke, Jürgen (Hrsg.) (2000): Von der Agentur zur Redaktion.
\emph{Böhlau}, Köln.

\begin{center}\rule{0.5\linewidth}{\linethickness}\end{center}

\hypertarget{swiss-policy-research}{%
\subsubsection{Swiss Policy Research}\label{swiss-policy-research}}

\begin{itemize}
\tightlist
\item
  \href{https://swprs.org/kontakt/}{Kontakt}
\item
  \href{https://swprs.org/uebersicht/}{Übersicht}
\item
  \href{https://swprs.org/donationen/}{Donationen}
\item
  \href{https://swprs.org/disclaimer/}{Disclaimer}
\end{itemize}

\hypertarget{english}{%
\subsubsection{English}\label{english}}

\begin{itemize}
\tightlist
\item
  \href{https://swprs.org/contact/}{About Us / Contact}
\item
  \href{https://swprs.org/media-navigator/}{The Media Navigator}
\item
  \href{https://swprs.org/the-american-empire-and-its-media/}{The CFR
  and the Media}
\item
  \href{https://swprs.org/donations/}{Donations}
\end{itemize}

\hypertarget{follow-by-email}{%
\subsubsection{Follow by email}\label{follow-by-email}}

Follow

\href{https://wordpress.com/?ref=footer_custom_com}{WordPress.com}.

\protect\hyperlink{}{Up ↑}

Post to

\protect\hyperlink{}{Cancel}

\includegraphics{https://pixel.wp.com/b.gif?v=noscript}
