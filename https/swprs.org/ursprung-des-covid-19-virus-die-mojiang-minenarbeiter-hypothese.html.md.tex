\protect\hyperlink{content}{Skip to content}

\href{https://swprs.org/}{}

\protect\hyperlink{search-container}{Search}

Search for:

\href{https://swprs.org/}{\includegraphics{https://swprs.files.wordpress.com/2020/05/swiss-policy-research-logo-300.png}}

\href{https://swprs.org/}{Swiss Policy Research}

Geopolitics and Media

Menu

\begin{itemize}
\tightlist
\item
  \href{https://swprs.org}{Start}
\item
  \href{https://swprs.org/srf-propaganda-analyse/}{Studien}

  \begin{itemize}
  \tightlist
  \item
    \href{https://swprs.org/srf-propaganda-analyse/}{SRF / ZDF}
  \item
    \href{https://swprs.org/die-nzz-studie/}{NZZ-Studie}
  \item
    \href{https://swprs.org/der-propaganda-multiplikator/}{Agenturen}
  \item
    \href{https://swprs.org/die-propaganda-matrix/}{Medienmatrix}
  \end{itemize}
\item
  \href{https://swprs.org/medien-navigator/}{Analysen}

  \begin{itemize}
  \tightlist
  \item
    \href{https://swprs.org/medien-navigator/}{Navigator}
  \item
    \href{https://swprs.org/der-propaganda-schluessel/}{Techniken}
  \item
    \href{https://swprs.org/propaganda-in-der-wikipedia/}{Wikipedia}
  \item
    \href{https://swprs.org/logik-imperialer-kriege/}{Kriege}
  \end{itemize}
\item
  \href{https://swprs.org/netzwerk-medien-schweiz/}{Netzwerke}

  \begin{itemize}
  \tightlist
  \item
    \href{https://swprs.org/netzwerk-medien-schweiz/}{Schweiz}
  \item
    \href{https://swprs.org/netzwerk-medien-deutschland/}{Deutschland}
  \item
    \href{https://swprs.org/medien-in-oesterreich/}{Österreich}
  \item
    \href{https://swprs.org/das-american-empire-und-seine-medien/}{USA}
  \end{itemize}
\item
  \href{https://swprs.org/bericht-eines-journalisten/}{Fokus I}

  \begin{itemize}
  \tightlist
  \item
    \href{https://swprs.org/bericht-eines-journalisten/}{Journalistenbericht}
  \item
    \href{https://swprs.org/russische-propaganda/}{Russische Propaganda}
  \item
    \href{https://swprs.org/die-israel-lobby-fakten-und-mythen/}{Die
    »Israel-Lobby«}
  \item
    \href{https://swprs.org/geopolitik-und-paedokriminalitaet/}{Pädokriminalität}
  \end{itemize}
\item
  \href{https://swprs.org/migration-und-medien/}{Fokus II}

  \begin{itemize}
  \tightlist
  \item
    \href{https://swprs.org/covid-19-hinweis-ii/}{Coronavirus}
  \item
    \href{https://swprs.org/die-integrity-initiative/}{Integrity
    Initiative}
  \item
    \href{https://swprs.org/migration-und-medien/}{Migration \& Medien}
  \item
    \href{https://swprs.org/der-fall-magnitsky/}{Magnitsky Act}
  \end{itemize}
\item
  \href{https://swprs.org/kontakt/}{Projekt}

  \begin{itemize}
  \tightlist
  \item
    \href{https://swprs.org/kontakt/}{Kontakt}
  \item
    \href{https://swprs.org/uebersicht/}{Seitenübersicht}
  \item
    \href{https://swprs.org/medienspiegel/}{Medienspiegel}
  \item
    \href{https://swprs.org/donationen/}{Donationen}
  \end{itemize}
\item
  \href{https://swprs.org/contact/}{English}
\end{itemize}

\protect\hyperlink{}{Open Search}

\hypertarget{ursprung-des-covid-19-virus-die-mojiang-minenarbeiter-hypothese}{%
\section{Ursprung des Covid-19-Virus: Die
Mojiang-Minenarbeiter-Hypothese}\label{ursprung-des-covid-19-virus-die-mojiang-minenarbeiter-hypothese}}

\textbf{Publiziert}: 16. Juli 2020; \textbf{Aktualisiert}: 2. August
2020\\
\textbf{Sprachen}:
\href{https://swprs.org/ursprung-des-covid-19-virus-die-mojiang-minenarbeiter-hypothese/}{DE},
\href{https://swprs.org/covid-19-virus-origin-the-mojiang-miners-passage-hypothesis/}{EN};
\textbf{Teilen auf}:
\href{https://twitter.com/intent/tweet?url=https://swprs.org/ursprung-des-covid-19-virus-die-mojiang-minenarbeiter-hypothese/}{Twitter}
/
\href{https://www.facebook.com/share.php?u=https://swprs.org/ursprung-des-covid-19-virus-die-mojiang-minenarbeiter-hypothese/}{Facebook}

Der Virologe Jonathan Latham und die Genetikerin Allison Wilson haben
\href{https://www.independentsciencenews.org/commentaries/a-proposed-origin-for-sars-cov-2-and-the-covid-19-pandemic/}{eine
neue Hypothese} für den Ursprung des SARS-CoV-2-Virus und der
Covid-19-Pandemie vorgeschlagen.

Seit Februar ist bekannt, dass ein Fledermausvirus namens RaTG13, das
2013 vom Wuhan Institute of Virology (WIV) gefunden wurde, der engste
bekannte Verwandte von SARS-CoV-2 ist.

Seit Mai ist (für einige) bekannt, dass RaTG13, das zuvor BtCoV/4991
genannt wurde, 2013 in Fledermauskot in einem Minenschacht in der Nähe
von Mojiang im Südwesten Chinas
\href{https://www.thetimes.co.uk/article/seven-year-covid-trail-revealed-l5vxt7jqp}{gefunden
wurde}, nachdem sechs Bergleute an einer Covid19-ähnlichen
Lungenentzündung erkrankten und drei von ihnen schließlich daran
starben.

Die WIV selbst hat diesen Zusammenhang jedoch nicht offengelegt. Die
berühmte WIV-Fledermaus-Forscherin Shi Zhengli behauptete in einem
Interview im März 2020 sogar irre­füh­render­weise, die Krankheit der
Bergleute sei ``durch einen Pilz'' ausgelöst worden.

Es ist auch bekannt, dass RaTG13 trotz seiner 96\%igen Ähnlichkeit nicht
der direkte Vorfahre von SARS-CoV-2 sein kann, da natürliche Mutationen
in der Umwelt mindestens mehrere Jahrzehnte gebraucht hätten. Es wurde
aber rechnerisch nachgewiesen, dass RaTG13 bereits zu einem gewissen
Grad menschliche Lungenzellen infizieren kann.

Es ist
\href{https://www.documentcloud.org/documents/6981198-Analysis-of-Six-Patients-With-Unknown-Viruses.html}{zudem
bekannt}, dass die erkrankten Bergleute aus Mojiang, die eine schwere
Covid19-ähnliche Lungenentzündung aufwiesen, \emph{bis zu vier Monate
lang} im Krankenhaus behandelt wurden, bevor sie 2012 entweder entlassen
wurden oder starben.

Latham und Wilson schlagen nun vor, dass die Bergleute von Mojiang
während ihres bis zu zweiwöchigen Aufenthalts im Minenschacht, in dem
sie Aerosole aus Fledermauskot einatmeten, ursprünglich mit RaTG13
und/oder ähnlichen Coronaviren infiziert wurden. RaTG13 entwickelte sich
dann durch Mutationen und/oder Rekombinationen \emph{in den Lungen der
Bergleute} während ihres bis zu viermonatigen Krankenhausaufenthaltes zu
SARS-CoV-2.

Somit könnten die infizierten Lungen der Bergleute als ``menschlicher
Inkubator'' gedient haben, der es RaTG13 ermöglichte, sich sowohl an den
menschlichen ACE2-Zellrezeptor als auch an das menschliche Immunsystem
in nur vier Monaten statt in mehreren Jahrzehnten anzupassen, wie dies
in einer natürlichen (tierischen) Umgebung zu erwarten gewesen wäre.

Das Wuhan-Institut für Virologie erhielt 2012/2013 Gewebe- und
Blutproben von den überlebenden oder toten Bergleuten, die
möglicherweise bereits das, was heute als SARS-CoV-2 bekannt ist,
enthalten haben. Die WIV-Virologen haben dann möglicherweise weitere
fünf Jahre bis zur Fertigstellung ihres BSL-4-(Hochsicherheits-)Labors
im Jahr 2018 gewartet, bevor sie mit der Forschung über das heute als
SARS-CoV-2 bekannte Virus begannen.

SARS-CoV-2 könnte dann, wahrscheinlich im Herbst 2019, aus dem
BSL-4-Labor des WIV entwichen sein, vielleicht durch einen infizierten
Laboranten, wodurch eine inzwischen weltweite Coronavirus-Pandemie
ausgelöst wurde.

Dies ist, kurz gesagt, die
Latham-Wilson\href{https://www.independentsciencenews.org/commentaries/a-proposed-origin-for-sars-cov-2-and-the-covid-19-pandemic/}{Mojiang
Miners Passage (MMP)} Hypothese. Sie kann die meisten oder alle der
ungewöhnlichen Eigenschaften von SARS-CoV-2 erklären, einschließlich
seiner sehr starken Bindung an menschliche ACE2-Zellrezeptoren und
seiner sehr niedrigen Mutationsrate, sogar ohne dass man von einer
Funktionsforschung (d.h. Genmanipulationen) ausgehen muss -- obschon die
Hypothese das auch nicht ausschließt.

Interessanterweise zeigen Archiveinträge, dass der Ursprung von RaTG13
in einer chinesischen Datenbank im Juli 2020 ohne Anmerkung von
``Lungenflüßigkeit'' (der Minenarbeiter) auf ``Fledermauskot''
\href{https://twitter.com/TheSeeker268/status/1286327367019839490}{geändert
wurde}. Zudem wurde vom WIV
\href{https://www.thetimes.co.uk/article/seven-year-covid-trail-revealed-l5vxt7jqp}{behauptet},
die RaTG13-Probe habe sich bei Analysen Anfang 2020 ``desintegriert''
und sei nicht mehr verfügbar (und überprüfbar).

\textbf{Lesen Sie den vollständigen Artikel von Latham and Wilson:}

\href{https://www.independentsciencenews.org/commentaries/a-proposed-origin-for-sars-cov-2-and-the-covid-19-pandemic/}{A
Proposed Origin for SARS-CoV-2 and the COVID-19 Pandemic} (ISN, July 15,
2020)

\textbf{Siehe auch:}

\begin{itemize}
\tightlist
\item
  \href{https://www.thetimes.co.uk/article/seven-year-covid-trail-revealed-l5vxt7jqp}{Seven
  year coronavirus trail from mine deaths to a Wuhan lab} (London Times,
  7/4/20)
\item
  \href{https://www.newsweek.com/dr-fauci-backed-controversial-wuhan-lab-millions-us-dollars-risky-coronavirus-research-1500741}{Dr.
  Fauci Backed Controversial Wuhan Lab with U.S. Dollars for Risky
  Coronavirus Research} (Newsweek, 4/28/20)
\item
  \href{https://www.scientificamerican.com/article/how-chinas-bat-woman-hunted-down-viruses-from-sars-to-the-new-coronavirus1/}{How
  China's `Bat Woman' Hunted Down Viruses from SARS to the New
  Coronavirus} (Scientific American, 3/11/20)
\item
  Ein früher Artikel von 2014:
  \href{https://www.sciencemag.org/news/2014/03/new-killer-virus-china}{A
  New Killer Virus in China?} (Science Magazine, 03/20/14)
\item
  \href{https://armswatch.com/project-g-2101-pentagon-biolab-discovered-mers-and-sars-like-coronaviruses-in-bats/}{Pentagon
  biolab discovered MERS and SARS-like coronaviruses in bats} (Arms
  Watch, 4/30/20)
\end{itemize}

\begin{center}\rule{0.5\linewidth}{\linethickness}\end{center}

\textbf{Teilen auf}:
\href{https://twitter.com/intent/tweet?url=https://swprs.org/ursprung-des-covid-19-virus-die-mojiang-minenarbeiter-hypothese/}{Twitter}
/
\href{https://www.facebook.com/share.php?u=https://swprs.org/ursprung-des-covid-19-virus-die-mojiang-minenarbeiter-hypothese/}{Facebook}

\hypertarget{swiss-policy-research}{%
\subsubsection{Swiss Policy Research}\label{swiss-policy-research}}

\begin{itemize}
\tightlist
\item
  \href{https://swprs.org/kontakt/}{Kontakt}
\item
  \href{https://swprs.org/uebersicht/}{Übersicht}
\item
  \href{https://swprs.org/donationen/}{Donationen}
\item
  \href{https://swprs.org/disclaimer/}{Disclaimer}
\end{itemize}

\hypertarget{english}{%
\subsubsection{English}\label{english}}

\begin{itemize}
\tightlist
\item
  \href{https://swprs.org/contact/}{About Us / Contact}
\item
  \href{https://swprs.org/media-navigator/}{The Media Navigator}
\item
  \href{https://swprs.org/the-american-empire-and-its-media/}{The CFR
  and the Media}
\item
  \href{https://swprs.org/donations/}{Donations}
\end{itemize}

\hypertarget{follow-by-email}{%
\subsubsection{Follow by email}\label{follow-by-email}}

Follow

\href{https://wordpress.com/?ref=footer_custom_com}{WordPress.com}.

\protect\hyperlink{}{Up ↑}

\includegraphics{https://pixel.wp.com/b.gif?v=noscript}
