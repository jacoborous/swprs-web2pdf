\protect\hyperlink{content}{Skip to content}

\href{https://swprs.org/}{}

\protect\hyperlink{search-container}{Search}

Search for:

\href{https://swprs.org/}{\includegraphics{https://swprs.files.wordpress.com/2020/05/swiss-policy-research-logo-300.png}}

\href{https://swprs.org/}{Swiss Policy Research}

Geopolitics and Media

Menu

\begin{itemize}
\tightlist
\item
  \href{https://swprs.org}{Start}
\item
  \href{https://swprs.org/srf-propaganda-analyse/}{Studien}

  \begin{itemize}
  \tightlist
  \item
    \href{https://swprs.org/srf-propaganda-analyse/}{SRF / ZDF}
  \item
    \href{https://swprs.org/die-nzz-studie/}{NZZ-Studie}
  \item
    \href{https://swprs.org/der-propaganda-multiplikator/}{Agenturen}
  \item
    \href{https://swprs.org/die-propaganda-matrix/}{Medienmatrix}
  \end{itemize}
\item
  \href{https://swprs.org/medien-navigator/}{Analysen}

  \begin{itemize}
  \tightlist
  \item
    \href{https://swprs.org/medien-navigator/}{Navigator}
  \item
    \href{https://swprs.org/der-propaganda-schluessel/}{Techniken}
  \item
    \href{https://swprs.org/propaganda-in-der-wikipedia/}{Wikipedia}
  \item
    \href{https://swprs.org/logik-imperialer-kriege/}{Kriege}
  \end{itemize}
\item
  \href{https://swprs.org/netzwerk-medien-schweiz/}{Netzwerke}

  \begin{itemize}
  \tightlist
  \item
    \href{https://swprs.org/netzwerk-medien-schweiz/}{Schweiz}
  \item
    \href{https://swprs.org/netzwerk-medien-deutschland/}{Deutschland}
  \item
    \href{https://swprs.org/medien-in-oesterreich/}{Österreich}
  \item
    \href{https://swprs.org/das-american-empire-und-seine-medien/}{USA}
  \end{itemize}
\item
  \href{https://swprs.org/bericht-eines-journalisten/}{Fokus I}

  \begin{itemize}
  \tightlist
  \item
    \href{https://swprs.org/bericht-eines-journalisten/}{Journalistenbericht}
  \item
    \href{https://swprs.org/russische-propaganda/}{Russische Propaganda}
  \item
    \href{https://swprs.org/die-israel-lobby-fakten-und-mythen/}{Die
    »Israel-Lobby«}
  \item
    \href{https://swprs.org/geopolitik-und-paedokriminalitaet/}{Pädokriminalität}
  \end{itemize}
\item
  \href{https://swprs.org/migration-und-medien/}{Fokus II}

  \begin{itemize}
  \tightlist
  \item
    \href{https://swprs.org/covid-19-hinweis-ii/}{Coronavirus}
  \item
    \href{https://swprs.org/die-integrity-initiative/}{Integrity
    Initiative}
  \item
    \href{https://swprs.org/migration-und-medien/}{Migration \& Medien}
  \item
    \href{https://swprs.org/der-fall-magnitsky/}{Magnitsky Act}
  \end{itemize}
\item
  \href{https://swprs.org/kontakt/}{Projekt}

  \begin{itemize}
  \tightlist
  \item
    \href{https://swprs.org/kontakt/}{Kontakt}
  \item
    \href{https://swprs.org/uebersicht/}{Seitenübersicht}
  \item
    \href{https://swprs.org/medienspiegel/}{Medienspiegel}
  \item
    \href{https://swprs.org/donationen/}{Donationen}
  \end{itemize}
\item
  \href{https://swprs.org/contact/}{English}
\end{itemize}

\protect\hyperlink{}{Open Search}

\hypertarget{fatti-su-covid-19-archivio}{%
\section{Fatti su
Covid-19~(Archivio)}\label{fatti-su-covid-19-archivio}}

\hypertarget{alla-pagina-principale-fatti-su-covid-19}{%
\paragraph{\texorpdfstring{\href{https://swprs.org/un-medico-svizzero-su-covid-19/}{Alla
pagina principale: Fatti su
Covid-19}}{Alla pagina principale: Fatti su Covid-19}}\label{alla-pagina-principale-fatti-su-covid-19}}

\hypertarget{3-aprile-2020}{%
\paragraph{3 aprile 2020}\label{3-aprile-2020}}

\textbf{Austria}: Anche qui le ``morti per corona virus'' sono
apparentemente definite ``molto liberalmente'', come
\href{https://www.heute.at/s/osterreich-bei-corona-todesstatistik-sehr-liberal-48665863}{riferisce
la stampa}: ``Si contano come morti per corona anche le persone che sono
state infettate dal virus ma che sono morte per qualcos'altro? Sì,
dicono Rudi Anschober e Bernhard Benka, membri della Corona Task Force
del Ministero della Salute. ``Al momento c'è una regola chiara: Morire
con il virus corona o per il virus corona'', spiega Benka. Tutti questi
casi contano per le statistiche. Non fa alcuna differenza per quanto
riguarda le cause della morte del paziente. In altre parole, anche un
uomo di 90 anni che muore con una frattura del collo del femore e si
infetta con il Covid19 nelle ore precedenti alla sua morte viene
conteggiato come morte per corona virus. Per fare solo un esempio''.

\textbf{Germania}: L'Istituto Robert Koch tedesco sta ora consigliando
di non eseguire autopsie su persone decedute positive ai test, poiché il
rischio di infezione da goccioline da parte di aerosol è presumibilmente
\href{https://www.youtube.com/watch?v=gSn_YaOYYcY}{troppo alto}.
Tuttavia, in molti casi questo significa che la vera causa della morte
non può più essere determinata.

Uno specialista in
patologia\href{https://www.youtube.com/watch?v=gSn_YaOYYcY}{commenta}
così (lettera stampata in video): ``Un mascalzone che pensa male! Finora
era ovvio che i patologi effettuassero autopsie con adeguate precauzioni
di sicurezza anche in caso di malattie infettive come l'HIV/AIDS,
l'epatite, la tubercolosi, le malattie PRION, ecc. E' abbastanza
notevole che in una malattia che sta uccidendo migliaia di pazienti in
tutto il mondo e che sta portando l'economia di interi paesi a un
virtuale blocco, sono disponibili solo pochissimi risultati di autopsie
(sei pazienti provenienti dalla Cina). Dal punto di vista sia della
polizia epidemica che della comunità scientifica, dovrebbe esserci un
interesse pubblico particolarmente elevato per i risultati
dell'autopsia. Tuttavia, è vero il contrario. Avete paura di scoprire le
vere cause della morte del defunto testato positivamente? È possibile
che il numero di morti da corona virus si sciolga come neve al sole
primaverile?''

\textbf{Italia}: i professionisti sanitari russi hanno notato
\href{https://de.sputniknews.com/panorama/20200402326767475-fachpersonal-todesfaelle-lombardei-zeitung/}{``morti
strane''} nelle case di cura in Lombardia: ``Nella città di Gromo, ad
esempio, secondo quanto riportato dai giornali, sono stati registrati
diversi casi in cui persone presumibilmente infettate dal coronavirus si
sono semplicemente addormentate e non si sono più svegliate. Fino ad
allora non era stato osservato alcun sintomo grave della malattia nel
defunto. ()Come il direttore della casa di cura ha poi chiarito in
un'intervista a RIA Novosti, non è chiaro se i defunti siano stati
effettivamente infettati dal coronavirus, perché nessuno nella casa di
cura era stato testato per questo. () Nelle case di riposo, dove
lavorano squadre mediche e infermieristiche russe, i corridoi, le camere
da letto e le sale da pranzo sono disinfettate''.

Casi simili sono già stati
\href{https://web.archive.org/web/20200330082928/https:/www.sueddeutsche.de/panorama/coronavirus-news-deutschland-wolfsburg-laschet-1.4828033}{segnalati}
dalla Germania: I pazienti infermieristici senza sintomi di malattia
muoiono improvvisamente nell'attuale situazione eccezionale e sono
quindi considerati ``morti per corona''. Anche qui sorge la domanda
seria: chi muore per il virus e chi per misure talvolta estreme?

\textbf{Personale infermieristico}: La Süddeutsche Zeitung
\href{https://www.sueddeutsche.de/politik/coronavirus-pflegekraefte-ausland-1.4866124}{riferisce}:
``In tutta Europa, la pandemia minaccia la cura degli anziani a casa
perché il personale infermieristico non può più visitarli -- o ha
lasciato il rispettivo paese in fretta e furia per tornare a casa.

\textbf{Inoltre}: il professore di medicina di Stanford, il dottor Jay
Bhattacharya, ha rilasciato
\href{https://www.youtube.com/watch?v=-UO3Wd5urg0}{un'intervista di
mezz'ora} in cui mette in discussione la ``saggezza convenzionale'' di
Covid19. Ha affermato che le misure adottate finora si basavano su dati
molto incerti e in parte discutibili.

\hypertarget{2-aprile-2020-i}{%
\paragraph{2 aprile 2020 (I)}\label{2-aprile-2020-i}}

\hypertarget{germania}{%
\subparagraph{\texorpdfstring{\textbf{Germania}}{Germania}}\label{germania}}

Secondo
\href{https://influenza.rki.de/Wochenberichte/2019_2020/2020-13.pdf}{l'ultimo
rapporto sull'influenza} dell'istituto tedesco Robert Koch, il numero di
malattie respiratorie acute è recentemente ``calato drasticamente a
livello nazionale''. I valori sono ``scesi bruscamente in tutte le fasce
d'età''.

Al 20 marzo (12a settimana), il numero totale di casi di ricovero con
malattie respiratorie acute era diminuito significativamente. Nella
fascia d'età dagli 80 anni in su, il numero di malati si era addirittura
quasi dimezzato rispetto alla settimana precedente.

Nei 73 ospedali esaminati, il 7\% di tutti i casi con malattie
respiratorie ha ricevuto una diagnosi COVID-19. Nella fascia d'età 35-59
anni era il 16\% e nella fascia d'età 60-79 anni era il 13\% che ha
ricevuto una diagnosi COVID-19.

Queste cifre corrispondono a quelle di altri paesi, così come la
diffusione sostanzialmente tipica dei coronavirus (dal 5 al 15\%).

\href{https://swprs.files.wordpress.com/2020/04/rki-ili-kw13.png}{}

\includegraphics{https://swprs.files.wordpress.com/2020/04/rki-ili-kw13.png?w=279\&h=171}

Chřipková onemocnění (RKI, 13.kal. týden)

\href{https://swprs.files.wordpress.com/2020/04/rki-sari-kw12.png}{}

\includegraphics{https://swprs.files.wordpress.com/2020/04/rki-sari-kw12.png?w=449\&h=171}

Akutní onemocnění dýchacích cest v nemocnicích

Malattie simili all'influenza in generale e malattie respiratorie acute
negli ospedali (Istituto Robert Koch, KW13 e KW12)

~

Un
\href{https://www.zeit.de/wissen/2020-04/krankenhaeuser-kapazitaeten-coronavirus-patienten-deutschland/seite-2}{articolo
su DIE ZEIT} tratta la questione dei pazienti in terapia intensiva in
Germania:

``Attualmente i politici, gli esperti e molti cittadini osservano con
preoccupazione il numero esponenzialmente crescente di persone che
vengono infettate ogni giorno.Tuttavia, questo non è l'indicatore
decisivo per valutare quanto grave sia e colpirà la crisi del corona
virus in Germania. Perché è falsificato soprattutto dal numero di test,
che sono in aumento da settimane.

Per misurare l'onere che grava sul sistema sanitario, è particolarmente
importante sapere il numero di persone che sono così gravemente malate
da dover essere ventilate. Finché ci sono abbastanza posti in terapia
intensiva, molti di loro possono essere salvati. Solo quando questi
letti scarseggiano, si rischierà una situazione come quella italiana.~

Il registro DIVI mostra ora che la situazione nei reparti di terapia
intensiva tedeschi è stata finora tranquilla. ``Siamo ancora in una zona
confortevole'', dice Grabenhenrich. Il numero di pazienti gravemente
malati non è in forte aumento come quello dei pazienti infetti, ma anche
se così fosse, sarebbe comunque possibile fornire un gran numero di
letti per la terapia intensiva con un'ottima attrezzatura.''

\hypertarget{svizzera}{%
\subparagraph{\texorpdfstring{\textbf{Svizzera}}{Svizzera}}\label{svizzera}}

L'Ufficio federale della sanità pubblica (UFSP)
\href{https://www.bag.admin.ch/bag/it/home/krankheiten/ausbrueche-epidemien-pandemien/aktuelle-ausbrueche-epidemien/novel-cov/situation-schweiz-und-international.html}{riferisce}
che finora sono stati effettuati circa 139 330 test Covid19 , di cui il
risultato è stato positivo al 15\% (PDF). Questo numero corrisponde
anche al tipico valore del virus corona conosciuto da altri Paesi e, per
quanto si può vedere, non sembra aumentare nemmeno in Svizzera.

Solo il numero di test spesso citato dai media aumenta in modo
esponenziale, ma non il numero di ``infetti'', malati o addirittura
morti.

Il 31 marzo è stata tuttavia pubblicata una nuova
\href{https://www.bfs.admin.ch/bfs/it/home/statistiche/salute/stato-salute/mortalita-cause-morte.html}{statistica
settimanale sulla mortalità} che prevede per la prima volta un aumento
dei decessi complessivi nella fascia d'età superiore ai 65 anni in
Svizzera per la 12a settimana di calendario (fino al 22 marzo) (cfr.
grafico seguente). In particolare, Il totale dovrebbe aumentare di circa
200 morti alla settimana.

Questo aumento è ``un'espressione dell'attuale pandemia''. Qui sorge il
seguente problema: fino al 22 marzo in Svizzera ci sono stati
complessivamente
\href{https://it.wikipedia.org/wiki/Pandemia_di_COVID-19_del_2020_in_Svizzera}{106
decessi positivi ai test}. Un aumento di 200 morti alla settimana
significherebbe che gran parte della mortalità aggiuntiva non è causata
dal virus, ma dalle ``contromisure''.

Un'altra spiegazione potrebbe essere che i circa 200 decessi positivi al
test della settimana successiva
(\href{https://it.wikipedia.org/wiki/Pandemia_di_COVID-19_del_2020_in_Svizzera}{la
13a settimana}) sono già stati inclusi. Ciò significherebbe che tutti le
morti comprovate sono considerate come ulteriori decessi. Tuttavia, in
considerazione dell'età e del profilo della malattia, nonché
\href{https://swprs.org/rki-relativiert-corona-todesfaelle/}{dell'esperienza
internazionale}, questo sarebbe un presupposto molto dubbioso.

Infatti, il rapporto afferma: ``Queste stime iniziali sono ancora molto
incerte, per cui non è possibile pubblicare cifre esatte''.

Se si scopre che una gran parte dei decessi positivi ai test (età media:
83 anni) non sono ulteriori decessi, la mortalità complessiva non
aumenterebbe, oppure aumenterebbe soprattutto a causa delle misure
drastiche, come alcuni esperti
\href{https://swprs.org/offener-brief-von-professor-sucharit-bhakdi-an-bundeskanzlerin-dr-angela-merkel/}{temono}.

\includegraphics{https://swprs.files.wordpress.com/2020/04/bfs-mortaliaet-22-03.png?w=600\&h=400}

~

Il Tages-Anzeiger svizzero ha
\href{https://interaktiv.tagesanzeiger.ch/2020/uebersterblichkeit-wegen-coronavirus/}{presentato}
l'attuale mortalità totale rispetto agli anni precedenti (vedi grafico
sotto). Ciò dimostra che, anche se in realtà è aumentato, l'attuale
tasso della percentuale dei decessi è ancora al di sotto dei più forti
inverni influenzali degli ultimi anni.

\includegraphics{https://swprs.files.wordpress.com/2020/04/mortalitc3a4t-schweiz.png?w=720\&h=339}

\hypertarget{usa}{%
\subparagraph{USA}\label{usa}}

Un biofisico svizzero
\href{https://swprs.org/rate-of-positive-covid19-tests/}{ha
visualizzato} il fatto che negli USA (come nel resto del mondo) non è il
numero di persone ``infette'' ad aumentare in modo esponenziale, ma il
numero delle valutazioni. Le cifre dei test positivi rispetto al numero
di test rimangono costanti o aumentano solo lentamente, il che in linea
di principio parla~contro~un'epidemia virale esponenziale.

\includegraphics{https://swprs.files.wordpress.com/2020/04/ud-data-2-fs.png?w=736}

\hypertarget{ulteriori-informazioni}{%
\subparagraph{\texorpdfstring{\textbf{Ulteriori
informazioni}}{Ulteriori informazioni}}\label{ulteriori-informazioni}}

\begin{itemize}
\tightlist
\item
  I kit di prova del virus destinati alla Gran Bretagna
  \href{https://www.telegraph.co.uk/news/2020/03/30/uks-attempt-ramp-coronavirus-testing-hindered-key-components/}{hanno
  dovuto essere richiamati} perché contenevano già componenti del corona
  virus.
\item
  Lo studio dell'Imperial College britannico, che prevedeva centinaia di
  migliaia di morti aggiuntive ma non è mai stato pubblicato su una
  rivista o recensito, si basava su
  \href{https://judithcurry.com/2020/04/01/imperial-college-uk-covid-19-numbers-dont-seem-to-add-up/}{presupposti
  in gran parte irrealistici}, come è stato ora dimostrato.
\item
  La BBC ha chiesto: \href{https://www.bbc.com/news/health-51979654}{``I
  decessi sono causati dal coronavirus?'',} e ha risposto: ``Potrebbe
  essere una causa importante, un fattore aggiuntivo, o semplicemente
  essere solo presente''. Per esempio, un uomo di 18 anni è stato
  segnalato come ``la più giovane vittima del virus'' perché un test il
  giorno prima della sua morte era positivo. In seguito, però,
  l'ospedale ha riferito che il giovane era morto a causa di una grave
  malattia preesistente.
\item
  L'autorità sanitaria europea ECDC ha pubblicato
  \href{https://www.ecdc.europa.eu/sites/default/files/documents/COVID-19-safe-handling-of-bodies-or-persons-dying-from-COVID19.pdf}{linee
  guida molto severe} per il trattamento dei cadaveri positivi ai test o
  ``presunti positivi ai test''. In considerazione dei tassi di
  mortalità molto bassi finora registrati, tali linee guida appaiono
  discutibili dal punto di vista medico; tuttavia, esse aumentano
  notevolmente l'onere per i servizi sanitari e funebri e, allo stesso
  tempo, hanno un elevato impatto mediatico.
\item
  Bayerischer Rundfunk ha pubblicato un
  \href{https://www.br.de/nachrichten/wissen/bhakdis-brief-an-die-kanzlerin-was-ist-dran-an-seinen-fragen,RutYDhd}{commento
  critico} sulla lettera aperta del professore Sucharit Bhakdi alla
  cancelliera tedesca Angela Merkel.
\item
  Il documentario di ARTE
  \href{https://www.youtube.com/watch?v=1--c2SBYlMY}{``Profiteure der
  Angst''} (Profittatori della paura) del 2009 mostra come l'OMS,
  finanziata principalmente da privati, ha trasformato una leggera
  ondata di influenza (la cosiddetta ``influenza suina'') in una
  pandemia globale e successivamente ha venduto vaccini ai governi per
  diversi miliardi di dollari, alcuni dei quali pericolosi. Alcuni dei
  protagonisti di allora sono ancora una volta rappresentati
  \href{https://www.nature.com/articles/news.2009.424}{in modo
  prominente} nella situazione attuale.
\item
  L'ex giudice della Corte Suprema britannica, Jonathan Sumption, ha
  dichiarato in
  \href{https://www.spectator.co.uk/article/former-supreme-court-justice-this-is-what-a-police-state-is-like-}{un'intervista
  della BBC} sulle misure britanniche: ``Questo è l'aspetto di uno stato
  di polizia.''
\end{itemize}

\hypertarget{2-aprile-2020-ii}{%
\paragraph{2 aprile 2020 (II)}\label{2-aprile-2020-ii}}

\begin{itemize}
\tightlist
\item
  Già nel 2018 il Guardian britannico aveva sottolineato:
  \href{https://www.theguardian.com/society/2018/dec/09/steep-rise-lung-related-illness-hospitals-nhs}{``L'inquinamento
  e l'influenza portano ad un forte aumento delle malattie polmonari''}.
  La carenza di specialisti si aggiunge alle preoccupazioni che
  l'aumento delle malattie respiratorie sta mettendo sotto pressione i
  Pronto Soccorso.
\item
  Nel frattempo, anche i
  \href{https://pflege-prisma.de/2020/03/31/sterbezahlen-in-pflegeheimen/}{rappresentanti
  delle case di cura} si lamentano delle misure restrittive e
  dell'inopportuna copertura mediatica di Covid19: ``Anche prima del
  virus, nei mesi invernali accadeva spesso che molti residenti
  morissero in un tempo relativamente breve senza che le troupe
  televisive stessero in piedi fuori dalla porta e che le persone
  vestite con tute protettive si mostrassero eroicamente esposte al
  rischio di infezione.
\item
  I dati della città di Treviso (vicino a Venezia) mostrano che,
  nonostante i 108 decessi positivi ai test di fine marzo, la mortalità
  complessiva negli ospedali comunali
  \href{https://swprs.files.wordpress.com/2020/04/reppublica-treviso.jpg}{è
  rimasta pressoché invariata rispetto agli anni precedenti.} Questo è
  un'ulteriore indicazione del fatto che l'aumento temporaneo della
  mortalità in alcuni luoghi è più probabile che sia dovuto a fattori di
  terzi, come il panico e il collasso, piuttosto che al solo corona
  virus.
\item
  Il professor Martin Haditsch, specialista in microbiologia, virologia
  ed epidemiologia delle malattie infettive,
  \href{https://www.youtube.com/watch?v=PtzHH8DhgZM}{è molto critico nei
  confronti delle misure Covid19}. Questi sono ``completamente
  infondati'' e calpesterebbero ``il buon senso e i principi etici''.
\item
  Il professor John Oxford della Queen Mary University di Londra, uno
  dei più importanti virologi e specialisti dell'influenza del mondo,
  \href{https://novuscomms.com/2020/03/31/a-view-from-the-hvivo-open-orphan-orph-laboratory-professor-john-oxford/}{giunge
  alla seguente conclusione riguardo al Covid19}: ``Personalmente, direi
  che il consiglio migliore è quello di passare meno tempo a guardare
  notizie televisive sensazionali e non molto buone. Personalmente
  ritengo che questa epidemia di Covid sia una grave epidemia di
  influenza invernale. In questo caso l'anno scorso abbiamo avuto 8000
  decessi nei gruppi a rischio, vale a dire oltre il 65\% delle persone
  con malattie cardiache, ecc. Non credo che l'attuale Covid supererà
  questo numero. Siamo colpiti da un'epidemia mediatica''!
\end{itemize}

\hypertarget{1-aprile-2020}{%
\paragraph{1 aprile 2020}\label{1-aprile-2020}}

\hypertarget{la-situazione-in-italia}{%
\subparagraph{\texorpdfstring{\textbf{La situazione in
Italia}}{La situazione in Italia}}\label{la-situazione-in-italia}}

I medici italiani hanno riferito di aver già
\href{https://www.scmp.com/news/china/society/article/3076334/coronavirus-strange-pneumonia-seen-lombardy-november-leading}{osservato}
una grave polmonite nel nord Italia alla fine dello scorso anno.
Tuttavia, le analisi genetiche mostrano ora che il ``virus Covid19''
apparentemente è comparso in Italia solo a gennaio. ``La grave polmonite
diagnosticata in Italia a novembre e dicembre dev'essere dovuta quindi a
un altro agente patogeno'',
\href{https://www.nzz.ch/wissenschaft/coronavirus-der-stammbaum-verraet-woher-es-kommt-ld.1548271}{dice
la NZZ}. Questo solleva ancora una volta la questione del ruolo del
virus Covid19 sulla situazione italiana e del ruolo di altri fattori.

Il 30 marzo è stata qui richiamata l'attenzione sull'elenco dei medici
italiani morti ``durante la crisi del Corona virus'', molti dei quali
erano in realtà già da tempo in pensione, avendo circa 90 anni e non
avevano nulla a che fare direttamente con la crisi in corso. Oggi,
\href{https://portale.fnomceo.it/elenco-dei-medici-caduti-nel-corso-dellepidemia-di-covid-19/}{tutti
gli anni di nascita sono stati rimossi dall'elenco dei decessi} ( si
veda l'ultima versione
dell'\href{https://web.archive.org/web/20200328152430/https://portale.fnomceo.it/elenco-dei-medici-caduti-nel-corso-dellepidemia-di-covid-19/}{archivio}).
Una procedura strana.

Riceviamo anche il seguente messaggio da un osservatore in Italia, che
aggiunge ulteriori aspetti alla drammatica situazione italiana, che
probabilmente andrà ben oltre un virus:

``Nelle ultime settimane, la maggior parte dei badanti dell'Europa
dell'Est che hanno lavorato 24 ore al giorno, 7 giorni su 7 per
assistere le persone bisognose di cure, hanno lasciato l'Italia in
fretta e furia. Ciò è dovuto non da ultimo al panico, ai coprifuoco e
alle chiusure delle frontiere minacciate dai ``governi di emergenza''.
Di conseguenza, gli anziani bisognosi di assistenza e i disabili, alcuni
senza parenti, sono stati lasciati indifesi da chi si prendeva cura di
loro.

Molte di queste persone abbandonate, dopo qualche giorno,~ sono poi
finite disidratate negli ospedali già da tempo sovraccaricati. Purtroppo
negli ospedali mancava ormai il personale che doveva badare ai bambini
rinchiusi nei loro appartamenti perché le scuole e gli asili erano stati
chiusi. Questo ha poi contribuito al caos e al completo collasso
dell'assistenza ai disabili e agli anziani, soprattutto in quelle zone
dove sono state ordinate ``misure'' ancora più restrittive.

L'emergenza infermieristica, causata dal panico, ha portato
temporaneamente a molti decessi tra le persone bisognose di cure e
sempre più tra i pazienti più giovani negli ospedali. Questi morti sono
poi serviti a creare ancora più panico tra i responsabili e i media, che
hanno riportato, ad esempio, ``altri 475 morti'', ``I morti vengono
portati via dagli ospedali dall'esercito'', accompagnati da foto di bare
e camion dell'esercito allineati.

Tuttavia, questo è stato il risultato del timore dei direttori delle
pompe funebri per il ``virus killer'', che hanno quindi rifiutato i loro
servizi. Inoltre, da un lato ci sono stati troppi morti in una sola
volta e dall'altro il governo ha approvato una legge secondo la quale i
cadaveri portatori del coronavirus dovevano essere cremati. In Italia,
fino a quella data, erano state effettuate poche cremazioni. Quindi
c'erano solo pochi piccoli crematori, che molto rapidamente raggiunsero
i loro limiti. Per questo motivo il defunto doveva essere sistemato in
diverse chiese.

In linea di principio, questo sviluppo è stato lo stesso in tutti i
paesi. Tuttavia, la qualità del sistema sanitario ha una notevole
influenza sugli effetti. Pertanto, ci sono meno problemi in Germania,
Austria o Svizzera che in Italia, Spagna o Stati Uniti. Tuttavia, come
si evince dalle cifre ufficiali, non vi è un aumento significativo del
tasso di mortalità. Solo una piccola ``montagna'' che è nata da questa
tragedia''.

\includegraphics{https://swprs.files.wordpress.com/2020/03/covid-iss-stat-bloomberg.png?w=550\&h=301}

\hypertarget{cliniche-negli-usa-in-germania-e-in-svizzera}{%
\subparagraph{\texorpdfstring{\textbf{Cliniche negli USA, in Germania e
in
Svizzera}}{Cliniche negli USA, in Germania e in Svizzera}}\label{cliniche-negli-usa-in-germania-e-in-svizzera}}

\begin{itemize}
\tightlist
\item
  L'emittente televisiva statunitense CBS
  \href{https://nypost.com/2020/04/01/cbs-admits-to-using-footage-from-italy-in-report-about-nyc/}{ha
  utilizzato delle riprese} effettuate in un reparto di terapia
  intensiva italiano, usandole come se fossero state riprese in un
  ospedale Newyorchese senza però identificarlo.\\
\item
  Contrariamente a quanto riportato dai media, anche il registro dei
  reparti di terapia intensiva tedeschi
  \href{https://www.divi.de/register/intensivregister}{non mostra un
  aumento dell'occupazione}. I giornalisti hanno visitato i
  \href{https://www.in-opr.de/2020/03/28/coronakrise-und-ruppiner-kliniken-was-stimmt-hier-nicht/}{centri
  di ricovero Covid19 completamente abbandonati} nelle cliniche di
  Berlino. Un dipendente di una clinica di Monaco di Baviera ha spiegato
  che ``aspettavano da settimane l'ondata'', ma che non c'è stato
  ``nessun aumento del numero di pazienti''. Ha detto che le
  dichiarazioni dei politici non corrispondevano alla loro esperienza, e
  che il ``mito del virus killer'' non poteva ``essere confermato''.
\item
  Anche nelle cliniche svizzere non è stato osservato finora alcun
  aumento dell'occupazione. Un visitatore dell'ospedale cantonale di
  Lucerna riferisce che c'è ``meno attività rispetto ai tempi normali''.
  Interi piani sono stati chiusi a Covid19 , ma il personale ``sta
  ancora aspettando i pazienti''. Anche gli ospedali di Berna, Basilea,
  Zugo e Zurigo sono stati ``ripuliti''. Anche in Ticino i reparti di
  terapia intensiva
  \href{https://www.nzz.ch/schweiz/tessin-verlegt-erste-corona-patienten-in-deutschschweizer-spitaeler-ld.1549417}{non
  funzionano a pieno regime}, ma alcuni pazienti vengono ora trasferiti
  nei reparti vuoti svizzero-tedeschi. Da un punto di vista puramente
  medico, tutto ciò ha poco senso.
\end{itemize}

\hypertarget{altri-messaggi-medici}{%
\subparagraph{\texorpdfstring{\textbf{Altri messaggi
medici}}{Altri messaggi medici}}\label{altri-messaggi-medici}}

\begin{itemize}
\tightlist
\item
  L'infettivologo e direttore del Centro Medico Universitario di
  Amburgo-Eppendorf, il dottor Ansgar Lohse,
  \href{https://www.mopo.de/hamburg/uke-infektiologe-fordert-es-muessen-sich-mehr-menschen-mit-corona-infizieren-36483636}{chiede
  una rapida abolizione del coprifuoco e dei divieti di contatto}. Più
  persone dovrebbero essere infettate dal corona virus. Kitas e le
  scuole dovrebbero essere riaperte il più presto possibile in modo che
  i bambini e i loro genitori possano diventare immuni attraverso
  l'infezione con il corona virus. La continuazione delle severe misure
  porterebbe a una crisi economica, che costerebbe anche delle vite
  umane, ha detto il medico.
\item
  In Spagna, il
  \href{https://www.heise.de/tp/features/Das-ist-keine-Krise-sondern-eine-Katastrophe-4694104.html}{15\%
  dei positivi} ai test sono medici e infermieri. Anche se la maggior
  parte di loro rimangono per lo più asintomatiche, devono andare in
  quarantena, causando il collasso del sistema sanitario spagnolo.
\item
  Il dottor John Lee, professore emerito di patologia, in un articolo
  nel britannico The Spectator, si occupa della
  \href{https://www.spectator.co.uk/article/how-to-understand-and-report-figures-for-covid-19-deaths-}{definizione
  e della comunicazione altamente fuorviante} delle ``morti per
  corona''.
\item
  Gli
  \href{https://swprs.files.wordpress.com/2020/04/die-lage-in-norwegen.pdf}{ultimi
  dati provenienti dalla Norvegia,} valutati da un tossicologo
  ambientale con un dottorato di ricerca, mostrano anche che il tasso di
  positività ai test non sta aumentando -- come ci si aspetterebbe in
  caso di epidemia -- ma oscilla tra il 2 e il 10\%, il che è normale
  per i corona virus. L'età media dei deceduti positivi al test è di 84
  anni, le cause del decesso non sono riportate pubblicamente, non c'è
  un eccesso di mortalità.
\item
  La Svezia, che finora se l'è cavata senza misure radicali e non ha
  segnalato un aumento della mortalità (simile a paesi asiatici come il
  Giappone o la Corea del Sud), è notevolmente
  \href{https://www.theguardian.com/world/2020/mar/30/catastrophe-sweden-coronavirus-stoicism-lockdown-europe}{sotto
  pressione da parte dei media internazionali} per cambiare la sua
  strategia.
\item
  I dati dello Stato di New York mostrano che il tasso di
  ospedalizzazione degli individui positivi ai test potrebbe essere
  \href{https://www.nytimes.com/2020/03/27/nyregion/new-rochelle-coronavirus.html}{più
  di venti volte inferiore} a quello inizialmente previsto.
\item
  Un articolo sul
  \href{https://www.doccheck.com/de/detail/articles/26271-covid-19-beatmung-und-dann}{portale
  specialistico DocCheck} affronta il problema della ventilazione dei
  pazienti positivi al test. Nei pazienti positivi al test, la semplice
  ventilazione attraverso una maschera è ufficialmente sconsigliata, tra
  l'altro per evitare che il coronavirus si diffonda attraverso gli
  aerosol. Pertanto, i pazienti in terapia intensiva positivi ai test
  vengono spesso intubati direttamente. Tuttavia, l'intubazione ha
  scarse possibilità di successo e spesso porta a ulteriori danni ai
  polmoni (il cosiddetto danno polmonare indotto dal ventilatore). Come
  per i farmaci, ci si chiede se un trattamento più delicato dei
  pazienti non sarebbe più sensato dal punto di vista medico.
\end{itemize}

\hypertarget{rapporti-sugli-sviluppi-politici}{%
\subparagraph{\texorpdfstring{\textbf{Rapporti sugli sviluppi
politici}}{Rapporti sugli sviluppi politici}}\label{rapporti-sugli-sviluppi-politici}}

\begin{itemize}
\tightlist
\item
  Un ministro di Stato tedesco ha
  \href{https://de.nachrichten.yahoo.com/strobl-b\%C3\%BCrger-verst\%C3\%B6\%C3\%9Fe-gegen-corona-regeln-polizei-melden-095746341.html}{invitato}
  la popolazione a ``essere vigile e a denunciare alla polizia le
  violazioni delle regole per il contenimento dell'epidemia di corona''.
  ``\href{https://www.br.de/nachrichten/bayern/buerger-melden-eifrig-verstoesse-gegen-corona-regeln,RuGXp1h}{Segnalati
  con entusiasmo}'' da cittadini vigili sono, per esempio, ``la
  formazione di gruppi proibiti, i bambini nei parchi giochi, le feste''
  e gli escursionisti. 
\item
  Gli esperti tedeschi di diritto costituzionale lanciano l'allarme a
  causa di
  ``\href{https://www.focus.de/politik/deutschland/corona-regelungen-der-regierung-medizin-darf-nicht-gefaehrlicher-sein-als-die-krankheit_id_11827625.html}{gravi
  violazioni dei diritti fondamentali}``. L'esperto di diritto
  costituzionale Hans Michael Heinig avverte che ``lo Stato
  costituzionale democratico potrebbe trasformarsi nel più breve tempo
  possibile in uno Stato d'igiene fascista-isterico''. Il professor
  Christoph Möllers dell'Università Humboldt di Berlino spiega che la
  legge sulla protezione dalle infezioni ``non può servire come base per
  restrizioni così ampie dei diritti di libertà dei cittadini''. Secondo
  l'ex presidente della Corte costituzionale federale tedesca, Hans
  Jürgen Papier, ``le misure d'emergenza non giustificano la sospensione
  delle libertà civili a favore di uno Stato autoritario e di
  sorveglianza''.
\item
  In diversi paesi sono state lanciate petizioni online per porre fine
  al coprifuoco e ad altre violazioni dei diritti fondamentali. Allo
  stesso tempo, i contributi video critici, anche da parte dei medici,
  vengono sempre più spesso cancellati. A Berlino, un evento registrato
  sui diritti fondamentali, nel corso del quale è stata distribuita la
  costituzione tedesca, è stato
  \href{https://www.heise.de/tp/features/Wenn-Demonstranten-zu-Gefaehrdern-erklaert-werden-4692869.html}{revocato
  dalla polizia.}
\end{itemize}

\hypertarget{31-marzo-2020-i}{%
\paragraph{31 marzo 2020 (I)}\label{31-marzo-2020-i}}

Il Dr. Richard Capek e altri ricercatori
\href{https://coronadaten.wordpress.com/}{hanno già dimostrato} che il
numero di individui positivi ai test in relazione al numero di test
eseguiti rimane costante in tutti i paesi studiati, il che
parla~contro~una diffusione esponenziale (``epidemia'') del virus e
indica semplicemente un aumento esponenziale del numero di test.

A seconda del paese, la percentuale di individui positivi ai test è
compresa tra il 5 e il 15\% circa, il che corrisponde alla diffusione
abituale dei virus corona. È interessante notare che questi~valori
numerici costanti~non vengono comunicati attivamente
(\href{https://multipolar-magazin.de/artikel/coronavirus-irrefuhrung-fallzahlen}{o
addirittura rimossi}) dalle autorità e dai media. Invece, le curve
esponenziali ma irrilevanti e fuorvianti vengono mostrate senza
contesto.

Naturalmente ciò non corrisponde agli standard medici professionali,
come mostra anche il tradizionale
\href{https://influenza.rki.de/Saisonberichte/2017.pdf}{rapporto
sull'influenza} dell'Istituto Robert Koch tedesco (pag. 130, vedi
grafico sotto). Qui, oltre al numero di rilevamenti (a destra), viene
mostrato il~numero di campioni~(a sinistra, barre grigie) e
la~percentuale positiva~(a sinistra, curva blu).

Ciò dimostra che durante una stagione influenzale il tasso positivo sale
da 0 a 10\% fino all'80\% dei campioni e scende al valore normale dopo
poche settimane.~In confronto, i test Covid19 mostrano un tasso positivo
costante nell'intervallo normale (vedi sotto).

\includegraphics{https://swprs.files.wordpress.com/2020/03/rki-influenza-report-2017.png?w=650\&h=530}

Tasso positivo costante di Covid19 sull'esempio degli USA (Dr. Richard
Capek). Ciò vale analogamente per tutti gli altri paesi per i quali sono
attualmente disponibili dati sul numero di campioni.

\includegraphics{https://swprs.files.wordpress.com/2020/03/infizierte-pro-test2603.jpg?w=600\&h=325}

\hypertarget{31-marzo-2020-ii}{%
\paragraph{31 marzo 2020 (II)}\label{31-marzo-2020-ii}}

\begin{itemize}
\tightlist
\item
  Una
  \href{https://off-guardian.org/2020/03/30/covid19-yet-to-impact-europes-overall-mortality/}{rappresentazione
  grafica dei dati di monitoraggio europei} mostra in modo
  impressionante che la mortalità complessiva in tutta Europa,
  indipendentemente dalle misure adottate, è entro il 25 marzo nella
  fascia normale o inferiore, e spesso significativamente al di sotto
  dei livelli degli anni precedenti. Solo in Italia (65+) il tasso di
  mortalità complessivo è aumentato di recente (probabilmente per
  diversi motivi), ma è rimasto al di sotto dei precedenti inverni
  influenzali.
\item
  Il presidente dell'istituto tedesco Robert Koch ha confermato in
  un'altra conferenza stampa che le malattie precedenti e la causa reale
  della morte
  \href{https://swprs.org/rki-relativiert-corona-todesfaelle/}{non
  giocano un ruolo} nella definizione delle cosiddette ``morti per
  corona'' (vedi video qui sotto). Da un punto di vista medico, tale
  definizione è chiaramente fuorviante. Ha l'ovvio e noto effetto di
  spaventare la politica e la società.
\item
  In Italia la situazione
  \href{https://www.tagesspiegel.de/politik/die-verlangsamung-ist-da-in-italien-zeichnet-sich-die-wende-in-der-coronakrise-ab/25698124.html}{comincia
  a calmarsi}. Per quanto si può vedere finora, i tassi di mortalità
  temporaneamente aumentati (65+) sono stati effetti molto locali,
  spesso accompagnati dal panico e da un'interruzione dell'assistenza
  sanitaria. Un politico del Nord Italia chiede, ad esempio, ``come mai
  i pazienti Covid di Brescia vengono trasportati anche in Germania,
  mentre nel vicino Veneto, a Verona, due terzi dei letti di terapia
  intensiva sono vuoti''.
\item
  In un
  \href{https://onlinelibrary.wiley.com/doi/abs/10.1111/eci.13222}{articolo
  pubblicato} sull'European Journal of Clinical Investigation, il
  professore di medicina di Stanford John C. Ioannidis critica i ``danni
  causati da un eccesso di informazioni e da misure non basate su
  evidenze scientifiche''. Anche le riviste specializzate avevano
  pubblicato affermazioni dubbie all'inizio.
\item
  Uno studio cinese pubblicato sul Chinese Journal of Epidemiology
  all'inizio di marzo, che ha dimostrato l'inaffidabilità dei test del
  virus Covid (circa il 50\% di risultati falsi positivi in pazienti
  asintomatici), è stato successivamente ritirato. L'autore principale
  dello studio, dopo tutto preside di una scuola di medicina, non ha
  voluto dare il motivo del ritiro e ha parlato di una
  ``\href{https://choice.npr.org/index.html?origin=https://www.npr.org/sections/health-shots/2020/03/26/822084429/in-defense-of-coronavirus-testing-strategy-administration-cited-retracted-study}{questione
  delicata}``. Indipendentemente da questo studio, tuttavia, la
  suscettibilità agli errori dei cosiddetti test del virus PCR è nota da
  tempo: nel 2006, ad esempio, è stata ``provata'' in una casa di cura
  canadese un'infezione di massa da virus corona della SARS, che in
  seguito
  \href{https://www.ncbi.nlm.nih.gov/pmc/articles/PMC2095096/}{si è
  rivelata} essere un comune virus corona del raffreddore (che può
  essere fatale anche per i gruppi a rischio).
\item
  Gli autori della rete tedesca di gestione del rischio RiskNET parlano
  in
  \href{https://www.risknet.de/themen/risknews/covid-19-und-der-blindflug/}{un'analisi}
  di Covid19 di un ``volo alla cieca'', nonché di ``insufficiente
  competenza ed etica dei dati''.~ Al posto di un numero sempre maggiore
  di test e misure è necessario un campione rappresentativo. Il ``senso
  e il rapporto'' delle misure adottate devono essere messi in
  discussione in modo critico.
\item
  L'intervista spagnola con il virologo argentino-francese di fama
  internazionale Pablo Goldschmidt è stata
  \href{https://www.rubikon.news/artikel/der-corona-totalitarismus}{tradotta
  in tedesco.} Goldschmidt considera le misure adottate come
  controproducenti dal punto di vista medico e osserva che ora bisogna
  ``leggere Hannah Arendt'' per capire ``le origini del totalitarismo di
  allora''.
\item
  Il primo ministro ungherese Viktor Orban, come gli altri primi
  ministri e presidenti, ha in gran parte
  \href{https://www.krone.at/2127086}{svincolato} il parlamento
  ungherese da una ``legge d'emergenza'' e ora può governare
  essenzialmente per decreto.
\end{itemize}

\hypertarget{30-marzo-2020-i}{%
\paragraph{30 marzo 2020 (I)}\label{30-marzo-2020-i}}

\begin{itemize}
\tightlist
\item
  Il professor Sucharit Bhakdi ha nel frattempo pubblicato un
  \href{https://www.youtube.com/watch?v=LsExPrHCHbw\&feature=emb_title}{video}
  (tedesco/inglese) in cui spiega la sua
  \href{https://swprs.org/offener-brief-von-professor-sucharit-bhakdi-an-bundeskanzlerin-dr-angela-merkel/}{lettera
  aperta} alla cancelliera tedesca Dr. Angela Merkel.
\item
  In Germania, alcune cliniche non sono più in grado di accettare
  pazienti -- non perché ci siano troppi pazienti o troppo pochi letti,
  ma perché il
  \href{https://web.archive.org/web/20200330082928/https:/www.sueddeutsche.de/panorama/coronavirus-news-deutschland-wolfsburg-laschet-1.4828033}{personale
  infermieristico è risultato positivo}, anche se nella maggior parte
  dei casi non mostrano quasi nessun sintomo. Qui diventa chiaro ancora
  una volta come e perché il sistema sanitario sia paralizzato.
\item
  In una casa di riposo tedesca per anziani e di cura per persone
  gravemente dementi, sono
  \href{https://web.archive.org/web/20200330082928/https:/www.sueddeutsche.de/panorama/coronavirus-news-deutschland-wolfsburg-laschet-1.4828033}{morti}
  15 soggetti positivi al test: ``Sorprendentemente molti individui sono
  deceduti senza mostrare sintomi di corona. Un medico tedesco scrive:
  ``Dal mio punto di vista medico, ci sono prove che alcune di queste
  persone potrebbero essere morte a causa delle regole. Le persone
  affette da demenza senile sono sottoposte a forte stress quando le
  cose decisive cambiano nella loro vita quotidiana: Isolamento, nessun
  contatto fisico, possibilmente infermiere incappucciate''.
  Ciononostante, anche questi deceduti vengono conteggiati come ``morti
  corona'' nelle statistiche tedesche e internazionali. In relazione
  alla ``crisi del corona virus'', ora è anche possibile morire di una
  malattia senza averne nemmeno i sintomi. 
\item
  Secondo un
  \href{https://twitter.com/sneatio/status/1244157986832101376}{farmacologo},
  l'Inselspital svizzero di Berna ha costretto il personale a prendere
  un congedo, ha interrotto delle terapie e ha rinviato gli interventi
  chirurgici per paura di Covid19.
\item
  Il professor Gérard Krause, capo del Dipartimento di Epidemiologia del
  Centro tedesco Helmholtz per la ricerca sulle infezioni, avverte il
  canale televisivo tedesco ZDF che le misure anti-corona
  \href{https://www.zdf.de/nachrichten/politik/coronavirus-epidemiologe-folgen-helmholtz-100.html}{``potrebbero
  portare a più morti del virus stesso''.}
\item
  Diversi media hanno riferito che più di 40 medici in Italia sono già
  morti ``durante la crisi della corona'', come i soldati in guerra.
  Tuttavia, uno sguardo alla
  \href{https://web.archive.org/web/20200328152430/https://portale.fnomceo.it/elenco-dei-medici-caduti-nel-corso-dellepidemia-di-covid-19/}{lista
  corrispondente} mostra che la maggior parte dei deceduti sono medici
  da tempo in pensione di tutti i tipi, compresi psichiatri e pediatri
  di 90 anni, la maggior parte dei quali potrebbe essere morta per cause
  naturali.
\item
  Secondo
  \href{https://www.buzzfeed.com/albertonardelli/coronavirus-testing-iceland}{un'ampia
  indagine} in Islanda, il 50\% di tutti i positivi al test ha mostrato
  ``nessun sintomo'', mentre l'altro 50\% ha mostrato per lo più
  ``sintomi molto moderati e simili a raffreddore''. Secondo i dati
  islandesi, il tasso di mortalità di Covid19 è nell'intervallo per
  mille, cioè nell'intervallo influenzale o inferiore. Dei
  \href{https://www.government.is/news/article/?newsid=c65cf658-6eb6-11ea-9462-005056bc4d74}{due
  decessi} positivi al test, uno era ``un turista con sintomi
  insoliti''. \href{https://www.covid.is/data}{(Ulteriori dati
  islandesi)}
\item
  Il giornalista del Daily Mail britannico Peter Hitchens
  \href{https://hitchensblog.mailonsunday.co.uk/2020/03/theres-powerful-evidence-this-great-panic-is-foolish-yet-our-freedom-is-still-broken-and-our-economy.html}{scrive}:
  ``Ci sono chiare prove che questo grande panico è stupido. Ma le
  nostre libertà sono ancora limitate e la nostra economia viene
  distrutta''. Hitchens fa notare che in alcune parti del Regno Unito, i
  droni della polizia
  \href{https://www.youtube.com/watch?v=fHNxDzLsPeg}{controllano e
  segnalano} passeggiate ``non essenziali'' di persone in natura. In
  alcuni casi, i
  \href{https://www.youtube.com/watch?v=D4GEZjUTkqc}{droni della
  polizia} invitano le persone a tornare a casa ``per salvare vite
  umane''. Nota: Nemmeno George Orwell aveva pensato così lontano.
\item
  I servizi segreti italiani
  \href{https://www.focus.de/panorama/welt/sorge-vor-sozialen-unruhen-supermaerkte-gepluendert-apotheken-ueberfallen-italiens-geheimdienst-warnt-vor-aufstaenden_id_11826664.html}{mettono
  in guardia} dai disordini sociali e dalle rivolte. I supermercati sono
  già stati saccheggiati e le farmacie sono già state rapinate.
\end{itemize}

\hypertarget{30-marzo-2020-ii}{%
\paragraph{30 marzo 2020 (II)}\label{30-marzo-2020-ii}}

In diversi Paesi, in relazione a Covid19 è sempre più evidente che ``il
trattamento potrebbe essere peggiore della malattia''.

Da un lato, c'è il rischio di cosiddette
\href{https://it.wikipedia.org/wiki/Infezioni_correlate_all\%27assistenza}{infezioni
nosocomiali}, cioè infezioni che il paziente, che può essere solo
lievemente malato, contrae per primo in ospedale. In Europa sono attesi
2,5 milioni e mezzo di infezioni nosocomiali e 50.000 decessi correlati
ogni anno. Anche nei reparti di terapia intensiva tedeschi, circa il
15\% dei pazienti soffre di infezioni nosocomiali, compresa la polmonite
da respirazione artificiale. Un problema particolare è anche il
crescente numero di batteri resistenti agli antibiotici negli ospedali.

Un ulteriore aspetto è rappresentato dai metodi di trattamento
certamente ben intenzionati, ma a volte molto aggressivi, che sono
sempre più utilizzati nei pazienti Covid19 . Questi includono in
particolare la somministrazione di steroidi, antibiotici e farmaci
antivirali (o una combinazione di questi). Già nel trattamento dei
pazienti affetti da SARS-1, è stato dimostrato che l'esito con tale
trattamento era
\href{https://www.sciencedaily.com/releases/2020/02/200206110703.htm}{spesso
peggiore e più fatale} che senza tale trattamento.

\hypertarget{29-marzo-2020}{%
\paragraph{29 marzo 2020}\label{29-marzo-2020}}

\begin{itemize}
\tightlist
\item
  Il dottor Sucharit Bhakdi, professore emerito di microbiologia medica
  a Magonza, Germania, ha scritto una
  \href{https://swprs.org/offener-brief-von-professor-sucharit-bhakdi-an-bundeskanzlerin-dr-angela-merkel/}{lettera
  aperta alla cancelliera tedesca Angela Merkel} ~giovedì 26 marzo 2020,
  chiedendo una rivalutazione urgente della risposta del Covid19 e
  ponendo al cancelliere cinque domande cruciali.
  \href{https://swprs.org/open-letter-from-professor-sucharit-bhakdi-to-german-chancellor-dr-angela-merkel/}{(traduzione
  in inglese)}
\item
  Gli
  \href{https://multipolar-magazin.de/artikel/coronavirus-irrefuhrung-fallzahlen}{ultimi
  dati dell'Istituto Robert Koch} mostrano che l'aumento delle persone
  positive ai test è proporzionale all'aumento del numero di test, cioè
  in termini percentuali rimane all'incirca lo stesso. Ciò potrebbe
  indicare che l'aumento del numero di casi è principalmente il
  risultato di un aumento del numero di test, piuttosto che di
  un'epidemia in corso.
\item
  La microbiologa milanese Maria Rita Gismondo
  \href{https://www.secoloditalia.it/2020/03/coronavirus-la-gismondo-ammonisce-duramente-basta-snocciolare-numeri-sui-positivi-sono-dati-falsati/}{chiede
  al governo italiano} di smettere di comunicare il numero giornaliero
  di ``positivi a corona'' in quanto questi dati sono ``falsi'' e
  mettono inutilmente in panico la popolazione. Il numero di test
  positivi dipende molto dal tipo e dal numero di test e non dice nulla
  sullo stato di salute.
\item
  Il Dr. John Ioannidis, professore di medicina ed epidemiologia a
  Stanford, ha rilasciato
  \href{https://www.youtube.com/watch?v=d6MZy-2fcBw}{un'intervista di
  un'ora} sulla mancanza di dati sugli interventi di Covid19.
\item
  Il virologo argentino Pablo Goldschmidt, che vive in Francia,
  considera la reazione politica al Covid19 ``completamente esagerata''
  e mette in guardia contro le
  \href{https://www.infobae.com/coronavirus/2020/03/28/para-un-prestigioso-cientifico-argentino-el-coronavirus-no-merece-que-el-planeta-este-en-un-estado-de-parate-total/}{``misure
  totalitarie''.} In Francia, il movimento delle persone è già
  parzialmente monitorato dai droni. 
\item
  Il pubblicista italiano Fulvio Grimaldi, nato nel 1934, spiega che i
  provvedimenti statali attualmente in vigore in Italia sono
  \href{https://www.youtube.com/watch?v=O3BuNp01vpc}{``peggiori di
  quelli del fascismo}``. Il Parlamento e la società erano stati
  completamente depotenziati.
\end{itemize}

\hypertarget{28-marzo-2020}{%
\paragraph{28 marzo 2020}\label{28-marzo-2020}}

\begin{itemize}
\tightlist
\item
  Un
  \href{https://news.yahoo.com/oxford-study-suggests-millions-people-221100162.html}{nuovo
  studio dell'Università di Oxford~}conclude che il Covid19 è
  probabilmente presente nel Regno Unito dal gennaio 2020 e che metà
  della popolazione è ora infetta e quindi immunizzata, con la maggior
  parte delle persone che non presentano sintomi o mostrano sintomi
  molto lievi. Ciò significherebbe che solo una persona su mille
  dovrebbe essere ricoverata in ospedale per Covid19, una cifra
  relativamente bassa.
  \href{https://www.medrxiv.org/content/10.1101/2020.03.24.20042291v1}{(Studio)}
\item
  I media britannici
  \href{https://www.bbc.com/news/uk-england-beds-bucks-herts-52041709}{hanno
  riferito} di una donna di 21 anni morta di Covid19 senza alcuna
  malattia precedente. Tuttavia, da allora
  \href{https://archive.is/20200329015127/https:/www.theguardian.com/world/2020/mar/27/chloe-middleton-death-21-year-old-not-recorded-nhs-covid-19-related}{si
  è saputo} che la donna non è risultata positiva al test Covid19 ed è
  morta per un altro motivo, forse addirittura per suicidio. La voce
  Covid19 era nata ``perché aveva una leggera tosse''.
\item
  Lo scienziato dei media tedesco professor Otfried Jarren critica il
  fatto che molti mezzi di comunicazione di massa sono impegnati in
  \href{https://www.deutschlandfunk.de/covid-19-scharfe-kritik-an-ard-und-zdf-wegen.2849.de.html?drn:news_id=1114517}{un
  giornalismo acritico}, che mette in scena minacce e potere esecutivo.
  Non c'è quasi nessuna differenziazione e un vero e proprio dibattito
  tra esperti.
\end{itemize}

\hypertarget{27-marzo-2020-i}{%
\paragraph{27 marzo 2020 (I)}\label{27-marzo-2020-i}}

\textbf{Italia}:~secondo
gli~\href{http://www.salute.gov.it/portale/caldo/SISMG_sintesi_ULTIMO.pdf}{ultimi
dati~}pubblicati dal Ministero della Salute il 14 marzo scorso, la
mortalità complessiva è ora significativamente più alta in tutte le
fasce d'età oltre i 65 anni, dopo essere stata minore alla media a causa
dell'inverno mite. Fino al 14 marzo, la mortalità complessiva era ancora
inferiore alla stagione influenzale del 2016/2017, ma potrebbe averla
già superata. La maggior parte di questo eccesso di mortalità proviene
attualmente dal nord Italia. Tuttavia, non è ancora chiaro quale sia il
ruolo di Covid19 in tutto questo e quale possa essere il ruolo di
fattori come il panico, il collasso sistemico e l'isolamento stesso.

\includegraphics{https://swprs.files.wordpress.com/2020/03/italia-mortalita-marzo-14.png?w=600\&h=343}

\textbf{Francia}:~in Francia, secondo
gli~\href{https://www.santepubliquefrance.fr/maladies-et-traumatismes/maladies-et-infections-respiratoires/infection-a-coronavirus/documents/bulletin-national/covid-19-point-epidemiologique-du-24-mars-2020}{ultimi
dati}a livello nazionale, la mortalità complessiva rimane entro i valori
normali dopo una stagione influenzale mite. Tuttavia, in alcuni
dipartimenti, in particolare nel nord-est della Francia, la mortalità
complessiva nella fascia d'età superiore ai 65 anni è già nettamente
aumentata in relazione a Covid19 (cfr. figura).

\includegraphics{https://swprs.files.wordpress.com/2020/03/france-mortality.png?w=650\&h=400}

La Francia fornisce
anche~\href{https://www.santepubliquefrance.fr/maladies-et-traumatismes/maladies-et-infections-respiratoires/infection-a-coronavirus/documents/bulletin-national/covid-19-point-epidemiologique-du-24-mars-2020}{informazioni
dettagliate}sulla distribuzione per età e sulle condizioni preesistenti
dei pazienti in terapia intensiva positivi ai test e dei pazienti
deceduti (vedi figura sotto):

\begin{itemize}
\tightlist
\item
  L'età media dei defunti è di 81,2 anni.
\item
  Il 78\% dei deceduti aveva più di 75 anni; il 93\% aveva più di 65
  anni.
\item
  Il 2,4\% dei deceduti aveva meno di 65 anni e non era affetto da
  alcuna (nota) malattia precedente.
\item
  L'età media dei pazienti in terapia intensiva è di 65 anni.
\item
  Il 26\% dei pazienti in terapia intensiva ha più di 75 anni; il 67\%
  ha malattie precedenti.
\item
  Il 17\% dei pazienti in terapia intensiva ha meno di 65 anni e non ha
  malattie precedenti.
\end{itemize}

Le autorità francesi aggiungono che ``la quota dell'epidemia (Covid-19)
nella mortalità globale resta da determinare''.

\includegraphics{https://swprs.files.wordpress.com/2020/03/france-age-distribution-march-24.png?w=736}

\textbf{USA}:~Il ricercatore Stephen McIntyre
ha~\href{https://twitter.com/ClimateAudit/status/1243019315462516736}{valutato}i
dati ufficiali sui decessi per polmonite negli USA. Questi sono
tipicamente tra i 3000 e i 5500 decessi alla settimana e quindi
chiaramente al di sopra delle cifre attuali di Covid19. I decessi totali
negli Stati Uniti sono tra i 50.000 e i 60.000 alla settimana. (Nota:
nel grafico sottostante, gli ultimi dati di marzo 2020 non sono ancora
completamente aggiornati, quindi la curva è in pendenza).

\includegraphics{https://swprs.files.wordpress.com/2020/03/us-pneumonia-deaths.png?w=400\&h=360}

\textbf{Gran Bretagna:}

\begin{itemize}
\tightlist
\item
  Neil Ferguson dell'Imperial College di Londra
  \href{https://www.newscientist.com/article/2238578-uk-has-enough-intensive-care-units-for-coronavirus-expert-predicts/}{presuppone
  ora~}che il Regno Unito abbia una capacità sufficiente nelle unità di
  terapia intensiva per trattare i pazienti affetti da Covid19.
\item
  John Lee, professore emerito di
  patologia,\href{https://www.spectator.co.uk/article/The-evidence-on-Covid-19-is-not-as-clear-as-we-think}{sostiene~}che
  il modo particolare in cui vengono registrati i casi di Covid-19 porta
  a sopravvalutare il rischio rappresentato da Covid19 rispetto ai
  normali casi di influenza e di raffreddore.\\
\end{itemize}

\textbf{Altri:}

\begin{itemize}
\tightlist
\item
  Uno
  \href{https://medium.com/@nigam/higher-co-infection-rates-in-covid19-b24965088333}{studio
  preliminare~}condotto da ricercatori dell'Università di Stanford ha
  dimostrato che il 20-25\% dei pazienti positivi al Covid19 è risultato
  inoltre positivo ad altri virus dell'influenza o del raffreddore.
\item
  Il numero di richieste di assicurazione contro la disoccupazione negli
  Stati Uniti è salito a un livello record di
  \href{https://www.businessinsider.com/us-weekly-jobless-claims-record-coronavirus-unemployment-insurance-labor-recession-2020-3}{oltre~tre
  milioni.} In questo contesto, si prevede anche un
  forte~\href{https://twitter.com/KoenSwinkels/status/1243066532390977544}{aumento
  dei suicidi.}
\item
  Il primo paziente positivo al test in Germania è ora guarito. Secondo
  la sua stessa dichiarazione, l'uomo di 33 anni aveva sperimentato la
  malattia
  ``\href{https://www.br.de/nachrichten/bayern/coronavirus-patient-nummer-1-wie-ich-die-quarantaene-erlebte,Rrm4Ul8}{non
  così grave come l'influenza}``.
\item
  I media
  spagnoli~\href{https://elpais.com/sociedad/2020-03-25/los-test-rapidos-de-coronavirus-comprados-en-china-no-funcionan.html}{riferiscono}~che
  i test anticorpali rapidi per il Covid19 hanno una sensibilità solo
  del 30\%, anche se dovrebbe essere almeno dell'80\%.
\item
  Uno~\href{https://ehjournal.biomedcentral.com/articles/10.1186/1476-069X-2-15}{studio
  condotto in Cina~}nel 2003 ha concluso che la probabilità di morire di
  SARS è superiore dell'84\% nelle persone esposte a un moderato
  inquinamento dell'aria rispetto ai pazienti provenienti da regioni con
  aria pulita. Il rischio è addirittura superiore del 200\% tra le
  persone provenienti da aree con aria fortemente inquinata.
\item
  La rete tedesca per la medicina basata sulle evidenze scientifiche
  (EbM)~\href{https://www.ebm-netzwerk.de/en/publications/covid-19}{critica
  il lavoro dei media}sul Covid19: ``La copertura mediatica non tiene in
  alcun modo conto dei criteri di comunicazione del rischio basata sulle
  evidenze scientifiche che noi chiediamo. () La presentazione di dati
  grezzi senza riferimento ad altre cause di morte porta ad una
  sopravvalutazione del rischio''.
\end{itemize}

\hypertarget{27-marzo-2020-ii}{%
\paragraph{27 marzo 2020 (II)}\label{27-marzo-2020-ii}}

\begin{itemize}
\tightlist
\item
  Il ricercatore tedesco Dr. Richard Capek
  \href{https://coronadaten.wordpress.com/}{sostiene in
  un'analisi}quantitativa~che l'''epidemia di corona'' è in realtà una
  ``epidemia di test''. Capek dimostra che il numero di test è aumentato
  in modo esponenziale, ma la percentuale di positivi ai test è rimasta
  stabile e la mortalità è diminuita, il che è in contrasto con una
  diffusione esponenziale del virus stesso.
\item
  Il professore di virologia Dr. Carten Scheller dell'Università di
  Würzburg~\href{https://www.youtube.com/watch?v=w-uub0urNfw}{spiega in
  un podcast}che il Covid19 è abbastanza paragonabile all'influenza e
  finora ha portato anche a meno morti. Il professor Scheller sospetta
  che le curve esponenziali spesso presentate dai media abbiano più a
  che fare con il numero crescente di test che con un'insolita
  diffusione del virus stesso. Per paesi come la Germania, non Italia,
  ma il Giappone o la Corea del Sud dovrebbero fungere da modello.
  Nonostante i milioni di turisti cinesi e le minime restrizioni
  sociali, questi paesi non hanno ancora vissuto una crisi di Covid19.
  Una delle ragioni potrebbe essere l'uso di mascherine per la bocca:
  Ciò difficilmente proteggerebbe dall'infezione, ma limiterebbe la
  diffusione del virus da parte delle persone infette.
\item
  Gli~\href{https://www.ecodibergamo.it/stories/bergamo-citta/a-bergamo-decessi-4-volte-oltre-la-medialeco-lancia-unindagine-nei-comuni_1346651_11/}{ultimi
  dati di Bergamo}mostrano che la mortalità totale è quasi quadruplicata
  nel marzo 2020, passando da 200 a 300 persone al mese a circa 900
  persone. Non è ancora chiaro quale parte di questo sia dovuta al
  Covid19 e quale parte sia dovuta ad altri fattori specifici locali
  (vedi sopra).
\item
  I due professori di medicina di Stanford, il dottor Eran Bendavid e il
  dottor Jay Bhattacharya, spiegano in
  un~\href{https://web.archive.org/web/20200325103650/https://www.wsj.com/articles/is-the-coronavirus-as-deadly-as-they-say-11585088464}{contributo}
  che la letalità di Covid19 è sovrastimata di diversi ordini di
  grandezza e probabilmente anche in Italia è solo dello 0,01\% a
  0,06\%, quindi inferiore a quella dell'influenza. La ragione di questa
  sopravvalutazione è il numero molto sottostimato di persone già
  infette (senza sintomi). A titolo di esempio, viene menzionata la
  comunità italiana di Vo completamente testata, che ha mostrato
  dal\href{https://www.repubblica.it/salute/medicina-e-ricerca/2020/03/16/news/coronavirus_studio_il_50-75_dei_casi_a_vo_sono_asintomatici_e_molto_contagiosi-251474302/}{50
  al 75\% di persone positive al test senza sintomi}.
\item
  Il dottor Gerald Gaß, presidente dell'Associazione ospedaliera
  tedesca, ha spiegato
  in~\href{https://www.handelsblatt.com/politik/deutschland/coronakrise-deutsche-krankenhausgesellschaft-wir-sind-besser-vorbereitet-als-italien/25651268.html}{un'intervista
  all'Handelsblatt}che ``la situazione estrema in Italia è dovuta
  principalmente alle bassissime capacità di terapia intensiva''.
\item
  Il Dr. Wolfgang Wodarg, uno
  dei~\href{https://www.youtube.com/watch?v=p_AyuhbnPOI}{primi
  critici}della presentazione di Covid19, è stato
  provvisoriamente\href{https://www.transparency.de/aktuelles/detail/article/in-eigener-sache-vorstand-beschliesst-ruhen-der-mitgliedschaft-von-wolfgang-wodarg-1/}{escluso~}dal
  consiglio di amministrazione di Trasparency Germania, dove ha diretto
  il gruppo di lavoro sulla salute. Wodarg era già stato
  severamente~attaccatodai media per le sue critiche.
\item
  L'informatore della NSA Edward
  Snowden~\href{https://www.cnet.com/news/snowden-warns-government-surveillance-amid-covid-19-could-be-long-lasting/}{avverte}che
  i governi stanno usando la situazione attuale per espandere lo stato
  di sorveglianza e limitare i diritti fondamentali. Le misure di
  controllo attualmente in vigore non verrebbero smantellate dopo la
  crisi.
\end{itemize}

\includegraphics{https://swprs.files.wordpress.com/2020/03/anzahl-infizierte-und-tests-2603.jpg?w=600\&h=339}

\hypertarget{26-marzo-2020-i}{%
\paragraph{26 marzo 2020 (I)}\label{26-marzo-2020-i}}

\begin{itemize}
\tightlist
\item
  \textbf{USA}:\href{https://healthweather.us/}{gli ultimi dati degli
  USA}del 25 marzo mostrano un numero decrescente di malattie
  simil-influenzali in tutto il paese, la cui frequenza è ormai ben al
  di sotto della media pluriennale. Le misure governative possono essere
  escluse come motivo, in quanto sono in vigore da meno di una
  settimana.
\end{itemize}

\href{https://swprs.org/covid-19-hinweis-ii/us-influenza-trend/}{}

\includegraphics{https://swprs.files.wordpress.com/2020/03/us-influenza-trend.png?w=404\&h=242}

US Influenza Trend (March 25, 2020)

\href{https://swprs.org/covid-19-hinweis-ii/us-illness-levels/}{}

\includegraphics{https://swprs.files.wordpress.com/2020/03/us-illness-levels.png?w=324\&h=242}

US Influenza Trend (March 25, 2020)

USA: malattia simil-influenzale in diminuzione (25 marzo 2020, KINSA)

\begin{itemize}
\tightlist
\item
  \textbf{Germania}:\href{https://influenza.rki.de/Wochenberichte/2019_2020/2020-12.pdf}{l'ultimo
  rapporto sull'influenza}dell'Istituto tedesco Robert Koch del 24 marzo
  documenta una ``diminuzione a livello nazionale dell'attività delle
  malattie respiratorie acute'': Il numero di malattie simil-influenzali
  e il numero di degenze ospedaliere da esse causate è inferiore al
  livello degli anni precedenti e continua a diminuire. L'RKI ha
  continuato: ``L'aumento del numero di visite mediche () non può essere
  attualmente spiegato né dai virus dell'influenza che circolano nella
  popolazione né dalla SARS-CoV-2''.
\end{itemize}

\href{https://swprs.org/covid-19-hinweis-ii/rki-atemwegserkrankungen-20-2-2020/}{}

\includegraphics{https://swprs.files.wordpress.com/2020/03/rki-atemwegserkrankungen-20-2-2020.png?w=327\&h=202}

Deutschland: Atemwegserkrankungen 2019/2020 ggü. Vorjahren

\href{https://swprs.org/covid-19-hinweis-ii/rki-kliniken-belegung/}{}

\includegraphics{https://swprs.files.wordpress.com/2020/03/rki-kliniken-belegung.png?w=401\&h=202}

Deutschland: Krankenhausaufenthalte durch Atemwegserkrankungen nach
Altersgruppen

Germania: Malattie simil-influenzali in diminuzione (20 marzo 2020, RKI)

\begin{itemize}
\tightlist
\item
  \textbf{Italia}:~il noto virologo italiano Giulio
  Tarro~\href{https://www.cybermednews.eu/index.php/it/health/70871-interview-to-the-virologist-giulio-tarro-the-death-rate-of-covid-19-is-less-than-1-as-confirmed-by-the-national-institute-of-allergy-and-infectious-diseases}{sostiene}~che
  il tasso di mortalità di Covid19 è inferiore all'1\% anche in Italia
  ed è quindi paragonabile all'influenza. I valori più alti nascono solo
  perché non si fa distinzione tra i decessi con e da Covid19 e perché
  il numero di persone infette (prive di sintomi) è molto sottostimato.
\item
  \textbf{Regno Unito}:~Gli autori dello studio dell'Imperial College
  britannico, che hanno previsto fino a 500.000 morti, stanno di nuovo
  riducendo le loro previsioni. Dopo aver
  già~\href{https://www.bbc.com/news/health-51979654}{ammesso~}che una
  gran parte dei decessi positivi ai test fa parte della normale
  mortalità, ora affermano che il picco della
  malattia\href{https://www.thetimes.co.uk/article/nhs-now-likely-to-cope-with-coronavirus-says-key-scientist-rn5m6nggk}{viene
  raggiunto in due o tre settimane}.
\item
  \textbf{Regno Unito}:~il British
  Guardian~\href{https://www.theguardian.com/society/2019/feb/20/britons-urged-to-get-flu-vaccine-as-critical-cases-rise-above-2000}{ha
  riferito nel febbraio 2019}che nel Regno Unito ci sono già stati più
  di 2180 ricoveri per influenza in terapia intensiva durante la
  stagione di influenza effettivamente debole 2018/2019.
\item
  \textbf{Svizzera}:~In Svizzera, la mortalità in eccesso dovuta a
  Covid19 sembra essere ancora pari a zero. L'ultima
  ``vittima''\href{https://www.nau.ch/ort/basel/drei-weitere-covid-19-todesfalle-in-basel-stadt-65684099}{presentata}dai
  media è una donna di 100 anni. Ciononostante, il governo svizzero
  continua a inasprire le misure restrittive.
\end{itemize}

\hypertarget{26-marzo-2020-ii}{%
\paragraph{26 marzo 2020 (II)}\label{26-marzo-2020-ii}}

\begin{itemize}
\tightlist
\item
  \textbf{Svezia}:~la Svezia ha finora perseguito la strategia più
  liberale nel trattare con il Covid19 , basata
  su~\href{https://www.zeit.de/politik/ausland/2020-03/coronavirus-schweden-stockholm-oeffentliches-leben/komplettansicht}{due
  principi}:~I gruppi a rischio sono protetti e le persone con sintomi
  influenzali rimangono a casa. ``Se si seguono queste due regole, non
  c'è bisogno di ulteriori misure, il cui effetto è comunque solo
  marginale'', ha detto il capo epidemiologo Anders Tegnell. La vita
  sociale ed economica continuerà normalmente. La grande corsa agli
  ospedali non si è finora concretizzata.
\item
  Dott.ssa Jessica Hamed, esperta tedesca di diritto penale e
  costituzionale,
  \href{https://www.fr.de/politik/coronakrise-deutschland-sind-kontaktsperren-ausgangsbeschraenkungen-rechtswidrig-13611821.html}{sostiene~}che
  misure come il coprifuoco generale e il divieto di contatto sono una
  massiccia e sproporzionata violazione dei diritti fondamentali della
  libertà e sono quindi presumibilmente ``tutte illegali''.
\item
  L'ultimo rapporto di monitoraggio~europeo sulla mortalità complessiva
  del 26 marzo continua a mostrare valori normali o inferiori alla media
  in tutti i paesi e in tutte le fasce d'età, ma ora con una
  \href{https://www.euromomo.eu/outputs/zscore_country65.html}{sola~eccezione}:
  nella fascia d'età superiore ai 65 anni in Italia si prevede un
  aumento della mortalità complessiva attualmente in aumento (cosiddetto
  z-score ritardato), che tuttavia è ancora inferiore ai valori delle
  ondate influenzali del 2016/2017 e 2017/2018.
\end{itemize}

\hypertarget{25-marzo-2020}{%
\paragraph{25 marzo 2020}\label{25-marzo-2020}}

\begin{itemize}
\tightlist
\item
  L'immunologo e tossicologo tedesco, il professor Stefan Hockertz,
  \href{https://www.youtube.com/watch?v=7wfb-B0BWmo}{spiega} in
  un'intervista che il Covid19 non è più pericoloso dell'influenza, ma
  che viene osservato solo molto più da vicino. Più pericolosi del virus
  sono la paura e il panico causati dai media e la ``reazione
  autoritaria'' di molti governi. Il professor Hockertz sottolinea
  inoltre che molte delle presunte ``morti per corona'' sono in realtà
  morti a causa di altre malattie e sono risultate positive ai virus
  corona. Hockertz sospetta che fino a dieci volte più persone di quanto
  riferito hanno già avuto il Covid19 quasi senza accorgersene.
\item
  Il virologo e biochimico argentino Pablo Goldschmidt
  \href{https://www.clarin.com/buena-vida/coronavirus-panico-injustificado-dice-virologo-argentino-francia_0_yVcmJ4RM.html}{spiega}
  che il Covid19 non è più pericoloso di un brutto raffreddore o
  dell'influenza. È anche possibile che l'agente patogeno Covid19
  circolasse negli anni precedenti, ma non sia stato ancora scoperto
  perché nessuno lo cercava. Il dottor Goldschmidt parla di un ``terrore
  globale'' generato dai media e dalla politica. Ogni anno, solo negli
  Stati Uniti, tre milioni di neonati e 50.000 adulti in tutto il mondo
  muoiono di polmonite.
\item
  Il professor Martin Exner, direttore dell'Istituto di Igiene
  dell'Università di Bonn,
  \href{https://www.youtube.com/watch?v=9mI9trSm3PY}{spiega in
  un'intervista }al canale phoenix il motivo per cui il personale
  sanitario è attualmente sotto pressione, anche se finora in Germania
  il numero di pazienti non è aumentato quasi per niente: Da un lato, i
  medici e gli infermieri risultati positivi devono essere messi in
  quarantena e sono spesso difficili da sostituire; dall'altro, gli
  infermieri dei paesi vicini, che forniscono una parte importante
  dell'assistenza, non possono attualmente entrare nel paese a causa
  della chiusura delle frontiere.
\item
  Il professor Julian Nida-Rümelin, ex ministro di Stato tedesco della
  cultura e professore di etica,
  \href{https://www.zdf.de/nachrichten/zdf-morgenmagazin/julian-nida-ruemelin-zur-corona-krise-100.html}{fa
  notare }che il Covid19 non comporta alcun rischio per la popolazione
  in buona salute e che misure estreme come il coprifuoco non sono
  quindi giustificate.
\item
  Il professor John Ioannidis di Stanford ha
  \href{https://www.statnews.com/2020/03/17/a-fiasco-in-the-making-as-the-coronavirus-pandemic-takes-hold-we-are-making-decisions-without-reliable-data/}{dimostrato},
  utilizzando i dati della nave da crociera Diamond Princess, che la
  letalità corretta per età di Covid19 è compresa tra lo 0,025\% e lo
  0,625\%, cioè nell'intervallo di un raffreddore o di un'influenza
  grave. Uno
  \href{https://www.niid.go.jp/niid/en/2019-ncov-e/9407-covid-dp-fe-01.html}{studio
  giapponese} mostra inoltre che, nonostante l'età media elevata, il
  48\% di tutti i passeggeri che hanno riscontrato il test positivo è
  rimasto completamente privo di sintomi; anche nella fascia di età
  80-89 anni il 48\% è rimasto privo di sintomi, mentre nella fascia di
  età 70-79 anni il 60\% non ha mostrato alcun sintomo. Ciò solleva la
  questione se le malattie precedenti non siano un fattore più
  importante del virus stesso. Il caso dell'Italia mostra che
  \href{https://www.bloomberg.com/news/articles/2020-03-18/99-of-those-who-died-from-virus-had-other-illness-italy-says}{il
  99\% dei deceduti positivi al test }aveva una o più condizioni
  preesistenti, e anche con queste solo il
  \href{https://web.archive.org/web/20200324214448/https://www.telegraph.co.uk/global-health/science-and-disease/have-many-coronavirus-patients-died-italy/}{12\%
  dei certificati di morte} denominava il Covid19 come fattore causale.
\end{itemize}

\hypertarget{24-marzo-2020}{%
\paragraph{24 Marzo 2020}\label{24-marzo-2020}}

\begin{itemize}
\tightlist
\item
  Il Regno Unito ha eliminato il Covid19 dalla lista ufficiale delle
  malattie infettive ad alta consistenza (HCID), affermando che i tassi
  di mortalità sono
  \href{https://www.gov.uk/guidance/high-consequence-infectious-diseases-hcid\#status-of-covid-19}{``complessivamente
  bassi''}.
\item
  Il direttore dell'Istituto nazionale tedesco per la salute (RKI)
  \href{https://swprs.org/rki-relativiert-corona-todesfaelle/}{ha
  ammesso} di annoverare tutti i decessi positivi ai test,
  indipendentemente dalla reale causa del decesso, tra i ``decessi da
  coronavirus''. L'età media dei deceduti è di 82 anni, la maggior parte
  con gravi presupposti. Come nella maggior parte degli altri Paesi, la
  mortalità in eccesso dovuta a Covid19 è probabilmente prossima allo
  zero in Germania.
\item
  I letti nei reparti di terapia intensiva svizzeri riservati ai
  pazienti di Covid19 sono ancora
  \href{https://www.aargauerzeitung.ch/aargau/kanton-aargau/erst-3-von-100-aargauer-betten-der-intensivstationen-sind-belegt-so-ruesten-sich-die-spitaeler-auf-die-epidemie-137332716}{``per
  lo più vuoti''}.
\item
  La professoressa tedesca Karin Moelling, ex cattedra di virologia
  medica dell'Università di Zurigo, ha dichiarato
  \href{https://www.radioeins.de/programm/sendungen/die_profis/archivierte_sendungen/beitraege/corona-virus-kein-killervirus.html}{in
  un'intervista} che Covid19 è ``nessun virus killer'' e che ``il panico
  deve finire''.
\end{itemize}

\hypertarget{23-marzo-2020-i}{%
\paragraph{23 marzo 2020 (I)}\label{23-marzo-2020-i}}

\begin{itemize}
\tightlist
\item
  Un nuovo studio francese intitolato
  \href{https://www.sciencedirect.com/science/article/abs/pii/S0924857920300972}{SARS-CoV-2:
  ansia contro dati}conclude che ``il problema causato dalla SARS-CoV-2
  è probabilmente sopravvalutato'' perché ``la mortalità della
  SARS-CoV-2 non è significativamente diversa dai normali corona virus
  (virus del raffreddore) studiati in un ospedale in Francia.
\item
  Uno~\href{https://www.ijidonline.com/article/S1201-9712(19)30328-5/fulltext}{studio
  italiano dell'agosto 2019}ha rilevato che negli ultimi anni in Italia
  ci sono stati tra le 7.000 e le 25.000 morti per influenza all'anno.
  Il dato è superiore a quello di altri paesi europei a causa della
  popolazione anziana dell'Italia, ed è molto più alto di qualsiasi
  altra cosa precedentemente associata al Covid-19.
\item
  In una
  nuova~\href{https://www.who.int/news-room/q-a-detail/q-a-similarities-and-differences-covid-19-and-influenza}{scheda
  informativa}, l'OMS scrive che, secondo le attuali conoscenze, il
  Covid-19 si sta diffondendo più~lentamente~dell'influenza (di circa il
  50\%) e che la trasmissione pre-sintomatica del Covid-19 è molto
  inferiore a quella dell'influenza.
\item
  Un primario italiano ha segnalato
  ``\href{https://www.scmp.com/news/china/society/article/3076334/coronavirus-strange-pneumonia-seen-lombardy-november-leading}{strani
  casi di polmonite}'' in Lombardia già nel novembre 2019, il che
  solleva nuovamente la questione se il nuovo virus (che non è apparso
  ufficialmente in Italia fino al febbraio 2020) sia responsabile di
  questo, o altri fattori come il
  \href{https://www.thelocal.it/20170131/our-lungs-are-breaking-smog-levels-way-above-safe-limits-in-northern-italy}{forte
  inquinamento dell'aria} nel nord Italia.
\item
  Il ricercatore danese Peter Gøtzsche, fondatore della celebre Cochrane
  Collaboration, scrive che il corona virus è una
  ``\href{https://www.deadlymedicines.dk/corona-an-epidemic-of-mass-panic/}{epidemia
  di panico}'' e che ``la logica è stata una delle prime vittime''.
\end{itemize}

\hypertarget{23-marzo-2020-ii}{%
\paragraph{23 marzo 2020 (II)}\label{23-marzo-2020-ii}}

\begin{itemize}
\tightlist
\item
  Secondo l'ex ministro della salute israeliano, il professor Yoram
  Lass, il nuovo virus corona è meno pericoloso dell'influenza e il
  coprifuoco
  \href{https://en.globes.co.il/en/article-lockdown-lunacy-1001322696}{ucciderebbe
  più persone del virus}. ``I numeri non giustificano il panico'', ha
  detto Lass. È noto che ``l'Italia ha un'enorme morbilità dovuta a
  malattie respiratorie, più di tre volte superiore a quella del resto
  d'Europa''.
\item
  Secondo Pietro Vernazza, specialista svizzero in malattie infettive,
  le misure
  ordinate~\href{https://www.tagblatt.ch/leben/ostschweizer-infektiologe-pietro-vernazza-die-zahlen-zu-den-jungen-corona-virus-erkrankten-sind-irrefuehrend-ld.1206440}{non
  sono scientificamente giustificate}e devono essere riconsiderate.
  Secondo Vernazza, i test di massa non hanno senso, perché fino al 90\%
  della popolazione rimarrà privo di sintomi, mentre il coprifuoco e la
  chiusura delle scuole sono addirittura ``controproducenti''. Vernazza
  raccomanda di proteggere solo i gruppi a rischio e di revocare le
  restrizioni.
\item
  Anche il presidente dell'Associazione Medica Internazionale, Frank
  Ulrich Montgomery, considera il coprifuoco, come in
  Italia,~\href{https://www.general-anzeiger-bonn.de/news/politik/deutschland/interview-mit-weltaerztepraesident-montgomery-ueber-corona-pandemie-ist-chaos_aid-49609561}{``irragionevole''
  e ``controproducente''}.
\item
  Svizzera: Nonostante l'eccitazione dei media, la mortalità in eccesso
  rimane a zero o quasi: gli ultimi due
  \href{https://www.bluewin.ch/de/newsregional/zuerich/1068-bestatigte-corona-falle-und-funf-todesfalle-im-kanton-zurich-371873.html}{``vittime}''
  positivi al test sono stati un 96enne in cure palliative e un 97enne
  con diverse patologie preesistenti.
\item
  L'ultimo rapporto statistico dell'ISS sull'Italia è ora disponibile
  \href{https://www.epicentro.iss.it/coronavirus/bollettino/Report-COVID-2019_20_marzo_eng.pdf}{anche
  in inglese}.
\end{itemize}

\hypertarget{22-marzo-2020-i}{%
\paragraph{22 marzo 2020 (I)}\label{22-marzo-2020-i}}

\textbf{Per quanto riguarda la situazione in Italia}: la maggior parte
dei media riporta erroneamente che l'Italia ha fino a 800 morti al
giorno a causa del corona virus. In realtà, il Presidente della
Protezione Civile italiana sottolinea che le morti sono ``causate con il
corona virus e non dal corona virus'' (minuto 03:30
della~\href{https://youtu.be/0M4kbPDHGR0?t=210}{conferenza stampa}). In
altre parole, queste persone sono morte mentre risultavano positive.

Come~\href{https://www.statnews.com/2020/03/17/a-fiasco-in-the-making-as-the-coronavirus-pandemic-takes-hold-we-are-making-decisions-without-reliable-data/}{hanno
sottolineato}i professori Ioannidis e Bhakdi, paesi come la Corea del
Sud e il Giappone, che~non hanno introdotto alcuna misura di divieto,
hanno registrato un tasso di mortalità in eccesso quasi pari a zero in
relazione al Covid-19, mentre la nave da crociera Diamond Princess aveva
un tasso di mortalità previsto nell'intervallo 1/1000, cioè a livello o
al di sotto del livello dell´influenza stagionale o di un raffreddore
grave.

Gli attuali tassi di mortalità di test-positivi in Italia sono ancora
inferiori al 50\% del normale tasso di mortalità totale giornaliera in
Italia, che si aggira intorno alle 1800 morti al giorno. È quindi
possibile, forse anche probabile, che una gran parte
della~normale~mortalità giornaliera sia ora semplicemente conteggiata
come ``Covid19'' (in quanto risulta positiva). Questo è il punto
sottolineato dal Presidente della Protezione Civile italiana.

Tuttavia, è ormai noto che nelle regioni del nord Italia, cioè quelle
che devono far fronte
ai~\href{https://en.wikipedia.org/wiki/2020_Italy_coronavirus_lockdown}{coprifuoco
più severi}, è aumentato in modo significativo il tasso di mortalità
giornaliera. È anche noto che in Lombardia il 90\% dei decessi per test
positivi non avviene in terapia intensiva, ma per lo
più~\href{https://www.tgcom24.mediaset.it/cronaca/coronavirus-in-lombardia-9-morti-su-10-mai-giunti-in-terapia-intensiva_16362350-202002a.shtml}{a
domicilio}. E più del 99\% di questi decessi presenta gravi condizioni
di salute (ad esempio problemi cardiaci, problemi respiratori, o
cancro).

Il professor Sucharit Bhakdi
ha~\href{https://www.youtube.com/watch?v=JBB9bA-gXL4}{descritto}~le
misure di blocco come ``inutili'', ``autodistruttive'' e ``suicidio
collettivo''. Ciò solleva la questione estremamente preoccupante della
misura in cui l'aumento della mortalità di queste persone anziane,
isolate, altamente stressate e con molteplici condizioni preesistenti
possa essere stato causato dalle settimane di isolamento ancora in atto.

Sarebbe quindi uno di quei casi in cui il trattamento è peggiore della
malattia. (Vedere l'aggiornamento qui sotto: Solo il 12\% dei
certificati di morte indica il coronavirus come causa)\\

\includegraphics{https://swprs.files.wordpress.com/2020/03/borrelli2.jpg?w=550\&h=309}

\hypertarget{22-marzo-2020-ii}{%
\paragraph{22 marzo 2020 (II)}\label{22-marzo-2020-ii}}

\begin{itemize}
\tightlist
\item
  In Svizzera ci sono stati finora 56 decessi positivi ai test. Tutti
  erano
  ``\href{https://www.nzz.ch/schweiz/coronavirus-in-der-schweiz-die-neusten-entwicklungen-ld.1542664\#subtitle-wie-viele-infizierte-und-todesf-lle-gibt-es-second}{pazienti
  a rischio}''~a causa dell'età e/o di malattie precedenti. Non ci sono
  ancora informazioni sull'esatta causa del decesso, cioè se del virus o
  solo con il virus.
\item
  Il governo svizzero ha sostenuto che la situazione nella Svizzera
  meridionale (proprio accanto all'Italia) fosse ``drammatica'', ma i
  medici locali
  hanno~\href{https://www.nzz.ch/schweiz/punkto-intensivbetten-sind-wir-im-tessin-besser-ausgeruestet-als-der-rest-der-schweiz-ld.1547728}{smentito}:
  la situazione non è affatto drammatica.
\item
  Secondo
  quanto~\href{https://www.blick.ch/news/schweiz/nicht-nur-beatmungsgeraete-werden-knapp-im-kampf-gegen-corona-es-droht-ein-engpass-beim-sauerstoff-id15808185.html}{riportato
  dalla stampa}, c'è la minaccia di una carenza di bombole di ossigeno.
  Il motivo, tuttavia, non è stato l'attuale aumento della domanda, ma
  l'accaparramento per paura della scarsità.
\item
  In molti paesi c'è già
  una~\href{https://www.washingtonpost.com/health/covid-19-hits-doctors-nurses-emts-threatening-health-system/2020/03/17/f21147e8-67aa-11ea-b313-df458622c2cc_story.html}{crescente
  carenza}~di medici e infermieri. Il motivo principale è che gli
  specialisti che sono stati trovati positivi al test devono essere
  posti in quarantena, anche se nella maggior parte dei casi non
  sviluppano alcun sintomo o solo sintomi lievi.
\end{itemize}

\hypertarget{22-marzo-2020-iii}{%
\paragraph{22 marzo 2020 (III)}\label{22-marzo-2020-iii}}

\begin{itemize}
\tightlist
\item
  Un modello dell'Imperial College di Londra ha previsto tra i 250.000 e
  i 500.000 decessi nel Regno Unito ``da'' Covid-19, ma gli autori dello
  studio hanno ora
  \href{https://www.bbc.com/news/health-51979654}{ammesso}~che molti di
  questi decessi non sono aggiuntivi ma fanno parte del normale tasso di
  mortalità annuale, che nel Regno Unito è di 600.000 persone all'anno.
\item
  Il dottor David Katz, direttore fondatore del Yale University
  Prevention Research Center, chiede
  nella~\href{https://www.nytimes.com/2020/03/20/opinion/coronavirus-pandemic-social-distancing.html}{New
  York Times}: ``La nostra lotta contro il corona virus è peggiore della
  malattia? Ci sono modi più mirati per sconfiggere la pandemia''.
\item
  Secondo il professore italiano Walter Ricciardi, ``solo il 12\% dei
  certificati di
  morte~\href{https://web.archive.org/web/20200324214448/https://www.telegraph.co.uk/global-health/science-and-disease/have-many-coronavirus-patients-died-italy/}{danno}
  il corona virus come motivo'', mentre le cronache pubbliche riportano
  che ``tutti i decessi in ospedale con corona virus sono conteggiati
  come morti per corona virus''. Per ottenere i decessi effettivamente
  provocati dal corona virus, quindi, le cifre di morte italiane citate
  dai media devono essere ridotte di almeno 1/8. Questo dà un massimo di
  alcune decine di morti al giorno, rispetto a una normale mortalità
  totale di 1800 al giorno e fino a 20.000 morti per influenza all'anno.
\end{itemize}

\hypertarget{21-marzo-2020-i}{%
\paragraph{21 marzo 2020 (I)}\label{21-marzo-2020-i}}

\begin{itemize}
\tightlist
\item
  La Spagna ha finora riportato solo tre decessi positivi al test
  \href{https://www.20minutos.es/noticia/4193883/0/media-edad-coronavirus-espana/}{al
  di sotto dei 65 anni}(su un totale di circa 1000). Le loro precedenti
  malattie e la causa reale della loro morte non sono ancora note.
\item
  L'Italia
  \href{https://www.msn.com/en-au/news/coronavirus/italy-coronavirus-deaths-surge-by-627-in-a-day-lifting-total-death-toll-to-4032/ar-BB11tDnS}{ha
  riportato} 627 decessi positivi al test in un solo giorno il 20 marzo.
  Il tasso di mortalità normale in Italia è di circa 1800 morti al
  giorno. Dal 21 febbraio, l'Italia ha registrato un totale di circa
  4000 decessi positivi ai test. Nello stesso periodo, l'Italia ha avuto
  una mortalità naturale totale di circa 50.000 morti. Non è ancora
  chiaro di quanto la mortalità complessiva sia aumentata o sia
  semplicemente diventata positiva ai test. L'Italia e l'Europa hanno
  anche avuto una stagione influenzale molto mite nel 2019/2020, che ha
  risparmiato molte persone altrimenti vulnerabili.
\item
  Secondo
  quanto~\href{https://www.tgcom24.mediaset.it/cronaca/coronavirus-in-lombardia-9-morti-su-10-mai-giunti-in-terapia-intensiva_16362350-202002a.shtml}{riportato
  dai media italiani}, circa il 90\% dei decessi positivi ai test nella
  regione Lombardia sono avvenuti in casa o nel reparto comune,
  piuttosto che in terapia intensiva. Le cause della morte e il
  possibile ruolo delle misure di quarantena non sono ancora chiare.
  Solo 260 dei 2168 decessi positivi al test si sono verificati in
  terapia intensiva.
\item
  Bloomberg\href{https://www.bloomberg.com/news/articles/2020-03-18/99-of-those-who-died-from-virus-had-other-illness-italy-says}{riferisce~}che
  il 99\% dei decessi italiani sono dovuti ad altre malattie.
\end{itemize}

\includegraphics{https://swprs.files.wordpress.com/2020/03/covid-iss-stat-bloomberg.png?w=550\&h=301}

\hypertarget{21-marzo-2020-ii}{%
\paragraph{21 marzo 2020 (II)}\label{21-marzo-2020-ii}}

\begin{itemize}
\tightlist
\item
  Il Japan Times chiede:
  \href{https://www.japantimes.co.jp/news/2020/03/20/national/coronavirus-explosion-expected-japan/}{il
  Giappone si aspettava un'esplosione del virus corona. Dov'è?}~Anche se
  il Giappone è stato uno dei primi paesi ad avere risultati positivi ai
  test e non ha introdotto un ``blocco'', finora è uno dei paesi meno
  colpiti. Non c'è stato un aumento della polmonite e non c'è stato un
  aumento dei ricoveri ospedalieri.
\item
  I ricercatori
  italiani~\href{https://www.heise.de/tp/features/Feinstaubpartikel-als-Viren-Vehikel-4687454.html}{sostengono~}che
  l'inquinamento atmosferico estremo del nord Italia -- il più alto in
  Europa -- potrebbe avere un ruolo causale nell'attuale aumento locale
  della polmonite, simile al precedente aumento di Wuhan, in Cina (vedi
  sopra)
\item
  In una~\href{https://www.youtube.com/watch?v=JBB9bA-gXL4}{recente
  intervista}, il professor Sucharit Bhakdi, uno dei più citati esperti
  nel campo della microbiologia medica, spiega che è ``sbagliato'' e
  ``pericolosamente fuorviante'' dare la colpa dei decessi al nuovo
  virus corona, poiché le condizioni preesistenti e l'inquinamento
  atmosferico giocano un ruolo molto importante nelle città cinesi e del
  nord Italia. Il professor Bhakdi descrive le misure attualmente
  discusse o decise come ``grottesche'', ``insensate'',
  ``autodistruttive'' e come ``suicidio collettivo'', che accorceranno
  l'aspettativa di vita degli anziani e che non dovrebbero essere
  accettate dalla società.
\end{itemize}

\hypertarget{20-marzo-2020}{%
\paragraph{20 marzo 2020}\label{20-marzo-2020}}

\begin{itemize}
\tightlist
\item
  Secondo l'ultimo \href{https://www.euromomo.eu/index.html}{rapporto di
  monitoraggio europeo}, la mortalità totale in tutti i paesi (Italia
  compresa) e in tutte le fasce d'età è stata finora nei valori medi o
  inferiori.
\item
  Secondo
  gli~\href{https://de.wikipedia.org/wiki/COVID-19-Pandemie_in_Deutschland\#Todesf\%C3\%A4lle_in_den_Medien}{ultimi
  dati}della Germania, l'età media dei decessi positivi ai test è di
  circa 83 anni, la maggior parte dei quali con patologie croniche
  preesistenti.
\item
  Uno~\href{https://www.ncbi.nlm.nih.gov/pmc/articles/PMC2095096/}{studio
  canadese del 2006}, condotto dal professor John Ioannidis di Stanford,
  mostra il caso di una casa di cura dove anche i comuni corona virus
  (virus del raffreddore) possono causare una mortalità fino al 6\% nei
  gruppi a rischio, e che i kit di test del virus inizialmente
  indicavano falsamente un'infezione da corona virus della SARS.
\end{itemize}

\hypertarget{19-marzo-2020-i}{%
\paragraph{19 marzo 2020 (I)}\label{19-marzo-2020-i}}

L'ISS italiana ha pubblicato
un~\href{https://www.epicentro.iss.it/coronavirus/bollettino/Report-COVID-2019_17_marzo-v2.pdf}{nuovo
rapporto}~sulle persone decedute positive al test:

\begin{itemize}
\tightlist
\item
  ~L'età media è di 80,5 anni (79,5 per gli uomini, 83,7 per le donne).
\item
  Il 10\% dei deceduti aveva più di 90 anni; il 90\% aveva più di 70
  anni.
\item
  Al massimo lo 0,8\% dei deceduti non aveva precedenti malattie
  croniche.
\item
  Circa il 75\% dei deceduti aveva due o più malattie preesistenti,
  circa il 50\% aveva tre o più patologie già esistenti, tra cui in
  particolare malattie cardiache, diabete e cancro.
\item
  Cinque dei defunti avevano un´età dai 31 ai 39 anni, tutti con gravi
  condizioni precedenti.
\item
  L'Istituto Superiore di Sanità lascia ancora aperto ciò di cui i
  pazienti esaminati sono morti e parla in generale di ``deceduti
  Covid19-positivi''.
\end{itemize}

\hypertarget{19-marzo-2020-ii}{%
\paragraph{19 marzo 2020 (II)}\label{19-marzo-2020-ii}}

\begin{itemize}
\tightlist
\item
  Un~\href{https://milano.corriere.it/notizie/cronaca/18_gennaio_10/milano-terapie-intensive-collasso-l-influenza-gia-48-malati-gravi-molte-operazioni-rinviate-c9dc43a6-f5d1-11e7-9b06-fe054c3be5b2.shtml}{articolo
  del~Corriere della Sera}~descrive che i reparti di terapia intensiva
  italiani sono già crollati sotto la marcata ondata di influenza nel
  2017/2018, rinviando le operazioni e richiamando gli infermieri dalle
  ferie.
\item
  Il virologo tedesco Hendrik
  Streeck~\href{https://www.faz.net/aktuell/gesellschaft/gesundheit/coronavirus/virologe-hendrik-streeck-ueber-corona-neue-symptome-entdeckt-16681450.html?printPagedArticle=true\#pageIndex_2}{sospetta
  in un'intervista}~che Covid19 non aumenterà il tasso di mortalità
  complessivo in Germania, che normalmente è di circa 2500 persone al
  giorno. Streeck cita il caso di un uomo di 78 anni con patologie
  preesistenti morto per insufficienza cardiaca, risultato positivo al
  test Covid19 e quindi inserito nelle statistiche dei decessi di
  Covid19.
\item
  Secondo il professor John P.A. Ioannidis di Stanford, non esiste
  una~\href{https://www.statnews.com/2020/03/17/a-fiasco-in-the-making-as-the-coronavirus-pandemic-takes-hold-we-are-making-decisions-without-reliable-data/}{base
  di dati medici sufficiente} per le misure attualmente decise. Il nuovo
  coronavirus non è probabilmente più pericoloso di alcuni dei comuni
  coronavirus, anche nelle persone anziane.
\end{itemize}

\hypertarget{18-marzo-2020}{%
\paragraph{18 marzo 2020}\label{18-marzo-2020}}

\begin{itemize}
\tightlist
\item
  Un~\href{https://www.medrxiv.org/content/10.1101/2020.02.12.20022434v2}{nuovo
  studio epidemiologico}~(stampa preliminare) conclude che il tasso di
  mortalità di Covid19 anche nella città cinese di Wuhan è stato solo
  dello 0,04\% a 0,12\% e quindi piuttosto inferiore a quello
  dell'influenza stagionale, il cui tasso di mortalità è di circa lo
  0,1\%. Come motivo per la mortalità apparentemente sovrastimata di
  Covid19, i ricercatori ipotizzano che a Wuhan in origine siano stati
  registrati solo pochi casi, poiché la malattia era probabilmente
  asintomatica o lieve in molte persone.
\item
  I ricercatori
  cinesi~\href{https://www.eurasiareview.com/01022020-polluted-air-could-be-an-important-cause-of-wuhan-pneumonia-oped/}{sostengono~}che
  lo smog invernale estremo nella città di Wuhan potrebbe aver avuto un
  ruolo causale nello scoppio della polmonite. Nell'estate del 2019, a
  Wuhan erano già in
  corso~\href{https://www.cnn.com/2019/07/10/asia/china-wuhan-pollution-problems-intl-hnk/index.html}{proteste
  pubbliche}~a causa della scarsa qualità dell'aria.
\item
  Nuove immagini satellitari mostrano come il Nord Italia
  abbia~\href{https://twitter.com/esa/status/1238480433047916545}{i più
  alti livelli di inquinamento atmosferico}~in Europa e come questo
  inquinamento sia stato notevolmente ridotto dalla quarantena.
\item
  Un produttore del kit di prova Covid19 dichiara che deve essere
  utilizzato solo
  per~\href{https://www.creative-diagnostics.com/sars-cov-2-coronavirus-multiplex-rt-qpcr-kit-277854-457.htm}{scopi
  di ricerca}~e non per applicazioni diagnostiche, in quanto non è stato
  ancora convalidato clinicamente.
\end{itemize}

\includegraphics{https://swprs.files.wordpress.com/2020/03/covid-testkit.png?w=550\&h=149}

\hypertarget{17-marzo-2020-i}{%
\paragraph{17 marzo 2020 (I)}\label{17-marzo-2020-i}}

\begin{itemize}
\tightlist
\item
  Alcuni pronto soccorso svizzeri sono già sovraccarichi semplicemente a
  causa del gran numero di persone
  che~\href{https://insideparadeplatz.ch/2020/03/16/notfall-stationen-bereits-seit-tagen-am-anschlag/}{vogliono
  essere testate}. Ciò indica un'ulteriore componente psicologica e
  logistica della situazione attuale.
\item
  Il profilo della mortalità rimane sconcertante dal punto di vista
  virologico, poiché, a differenza dei virus influenzali, i bambini
  vengono risparmiati e gli uomini più anziani ne sono colpiti circa il
  doppio rispetto alle donne più anziane. D'altra parte, questo profilo
  corrisponde
  alla~\href{http://www.gbe-bund.de/gbe10/abrechnung.prc_abr_test_logon?p_uid=gast\&p_aid=0\&p_knoten=FID\&p_sprache=D\&p_suchstring=820}{mortalità
  naturale}, che è vicina allo zero nei bambini e quasi il doppio negli
  uomini di 75 anni rispetto alle donne della stessa età.
\item
  I giovani deceduti positivi al test erano ancora in gran parte o
  addirittura esclusivamente persone con le più gravi condizioni
  preesistenti. Ad esempio, un allenatore di calcio spagnolo di 21 anni
  è morto con un test positivo. Tuttavia, i
  medici\href{https://sports.yahoo.com/spanish-football-coach-francisco-garcia-163153573.html}{hanno
  diagnosticato} una leucemia non riconosciuta, le cui complicazioni
  tipiche includono una polmonite grave.
\item
  Il fattore decisivo per valutare il pericolo della malattia non è
  quindi il numero di persone positive al test e di deceduti, che viene
  spesso citato dai media, ma il numero di coloro che effettivamente e
  inaspettatamente si ammalano o muoiono di~polmonite~(la cosiddetta
  mortalità in eccesso). Questo valore è~molto basso~nella maggior parte
  dei paesi.
\end{itemize}

\hypertarget{17-marzo-2020-ii}{%
\paragraph{17 marzo 2020 (II)}\label{17-marzo-2020-ii}}

\begin{itemize}
\tightlist
\item
  Il professore italiano di immunologia Sergio Romagnani dell'Università
  di Firenze giunge alla conclusione, in uno studio su 3000 persone, che
  dal 50 al 75\% delle persone positive al test, di tutte le fasce
  d'età,
  rimane~\href{https://www.repubblica.it/salute/medicina-e-ricerca/2020/03/16/news/coronavirus_studio_il_50-75_dei_casi_a_vo_sono_asintomatici_e_molto_contagiosi-251474302/}{completamente
  privo di sintomi~}-- molto più di quanto si pensasse.
\item
  Il tasso di occupazione dei reparti di terapia intensiva del Nord
  Italia nei mesi invernali è tipicamente già
  \href{https://jamanetwork.com/journals/jama/fullarticle/2763188}{dall}'85
  al 90\%. Alcuni o molti di questi pazienti all'interno della struttura
  ospedaliera potrebbero già essere positivi al test. Tuttavia, non ci
  sono cifre ufficiali sul numero di ulteriori casi di polmonite
  inaspettata.
\item
  Un medico dell'ospedale della città spagnola di Malaga
  \href{https://twitter.com/NeurologaenSAS/status/1239498772570308609}{scrive
  su Twitter}che attualmente le persone hanno più probabilità di morire
  per panico e collasso sistemico che per il virus. L'ospedale è invaso
  da persone con raffreddori, influenza e forse Covid19 e le procedure
  sono crollate.
\end{itemize}

\hypertarget{14-marzo-2020}{%
\paragraph{14 marzo 2020}\label{14-marzo-2020}}

\href{https://www.epicentro.iss.it/coronavirus/sars-cov-2-decessi-italia}{Secondo
l'Istituto Superiore di Sanità}, l'età media dei deceduti positivi al
test in Italia è attualmente di circa 81 anni. Il 10\% dei deceduti ha
più di 90 anni. Il 90\% dei deceduti ha più di 70 anni.

 L'80\% dei deceduti aveva due o più patologie croniche preesistenti. Il
50\% dei deceduti aveva tre o più malattie croniche precedenti. Le
condizioni croniche preesistenti includono problemi cardiovascolari,
diabete, problemi respiratori e cancro.

Meno dell'1\% dei deceduti erano persone sane, cioè persone senza
precedenti malattie croniche. Solo il 30\% circa dei deceduti erano
donne.

L'Istituto Superiore di Sanità
\href{https://youtu.be/0M4kbPDHGR0?t=210}{distingue} anche tra chi è
morto per il coronavirus e chi è morto con il coronavirus. In molti casi
non è ancora chiaro se le persone sono morte per il virus o per le loro
condizioni croniche preesistenti o per una combinazione di entrambi.

Nel caso dei due italiani deceduti sotto i 40 anni (entrambi di 39
anni), si trattava di un malato di cancro e un diabetico con altre
complicanze. Anche qui, la causa esatta della morte non è ancora chiara
(cioè se è stata causata dal virus o dalle malattie precedenti).

Il sovraffollamento degli ospedali è dovuto all'afflusso generale di
pazienti e all'aumento del numero di pazienti che necessitano di cure
speciali o intensive. In particolare, l'attenzione si concentra sulla
stabilizzazione della funzione respiratoria e, nei casi più gravi, sulle
terapie antivirali.

(\textbf{Aggiornamento}:~l'Istituto Superiore di Sanità ha ora
pubblicato
un~\href{https://www.epicentro.iss.it/coronavirus/bollettino/Report-COVID-2019_17_marzo-v2.pdf}{rapporto
statistico}~sui pazienti positivi ai test e sui deceduti, che conferma i
dati di cui sopra).

\textbf{Inoltre, devono essere presi in considerazione anche i seguenti
aspetti:}

Il Nord Italia ha una delle popolazioni più anziane
e~\href{https://twitter.com/esa/status/1238480433047916545}{la più
scarsa qualità dell'aria d'Europa}, il che in passato ha già portato ad
\href{https://www.thelocal.it/20170131/our-lungs-are-breaking-smog-levels-way-above-safe-limits-in-northern-italy}{un
aumento} delle malattie respiratorie e dei decessi. Questo dovrebbe
essere visto come un ulteriore fattore di rischio.

 La Corea del Sud, ad esempio, ha avuto un andamento molto più blando
rispetto all'Italia e ha già superato il picco dell'epidemia. In Corea
del Sud, finora sono stati registrati solo circa 70 decessi con un test
positivo. Come in Italia, sono stati colpiti soprattutto i pazienti a
rischio.

I circa dodici decessi svizzeri positivi ai test finora effettuati erano
anche pazienti a rischio con patologie preesistenti e un'età media di 80
anni. La loro causa precisa di morte, ossia se da virus o da patologie
precedenti, non è ancora nota.

Inoltre, gli studi hanno dimostrato che i kit di test del virus
utilizzati in tutto il mondo potrebbero dare un
\href{https://www.ncbi.nlm.nih.gov/pmc/articles/PMC2095096/}{risultato
falso positivo} in alcuni casi, cioè le persone in questi casi non si
sarebbero ammalate con il nuovo coronavirus, ma forse con uno dei
precedenti coronavirus, che fanno parte dell'epidemia annuale (e
attuale) di raffreddore e influenza.\\

Per valutare la pericolosità della malattia non è quindi determinante il
numero dei test positivi e dei decessi spesso menzionati, ma il numero
di persone che si ammalano o muoiono (inaspetta­ta­mente) di polmonite
(la cosiddetta mortalità in eccesso).

Per la popolazione generale sana in età scolastica e lavorativa è da
aspettarsi un decorso da lieve a moderato di Covid-19, secondo tutti gli
antecedenti risultati. Gli anziani e le persone con malattie croniche
preesistenti dovrebbero essere particolarmente protetti. Le strutture
sanitarie devono essere preparate in modo ottimale.

\hypertarget{alla-pagina-principale-fatti-su-covid-19-1}{%
\paragraph{\texorpdfstring{\href{https://swprs.org/un-medico-svizzero-su-covid-19/}{Alla
pagina principale: Fatti su
Covid-19}}{Alla pagina principale: Fatti su Covid-19}}\label{alla-pagina-principale-fatti-su-covid-19-1}}

\begin{center}\rule{0.5\linewidth}{\linethickness}\end{center}

\hypertarget{swiss-policy-research}{%
\subsubsection{Swiss Policy Research}\label{swiss-policy-research}}

\begin{itemize}
\tightlist
\item
  \href{https://swprs.org/kontakt/}{Kontakt}
\item
  \href{https://swprs.org/uebersicht/}{Übersicht}
\item
  \href{https://swprs.org/donationen/}{Donationen}
\item
  \href{https://swprs.org/disclaimer/}{Disclaimer}
\end{itemize}

\hypertarget{english}{%
\subsubsection{English}\label{english}}

\begin{itemize}
\tightlist
\item
  \href{https://swprs.org/contact/}{About Us / Contact}
\item
  \href{https://swprs.org/media-navigator/}{The Media Navigator}
\item
  \href{https://swprs.org/the-american-empire-and-its-media/}{The CFR
  and the Media}
\item
  \href{https://swprs.org/donations/}{Donations}
\end{itemize}

\hypertarget{follow-by-email}{%
\subsubsection{Follow by email}\label{follow-by-email}}

Follow

\href{https://wordpress.com/?ref=footer_custom_com}{WordPress.com}.

\protect\hyperlink{}{Up ↑}

Post to

\protect\hyperlink{}{Cancel}

\includegraphics{https://pixel.wp.com/b.gif?v=noscript}
