\protect\hyperlink{content}{Skip to content}

\href{https://swprs.org/}{}

\protect\hyperlink{search-container}{Search}

Search for:

\href{https://swprs.org/}{\includegraphics{https://swprs.files.wordpress.com/2020/05/swiss-policy-research-logo-300.png}}

\href{https://swprs.org/}{Swiss Policy Research}

Geopolitics and Media

Menu

\begin{itemize}
\tightlist
\item
  \href{https://swprs.org}{Start}
\item
  \href{https://swprs.org/srf-propaganda-analyse/}{Studien}

  \begin{itemize}
  \tightlist
  \item
    \href{https://swprs.org/srf-propaganda-analyse/}{SRF / ZDF}
  \item
    \href{https://swprs.org/die-nzz-studie/}{NZZ-Studie}
  \item
    \href{https://swprs.org/der-propaganda-multiplikator/}{Agenturen}
  \item
    \href{https://swprs.org/die-propaganda-matrix/}{Medienmatrix}
  \end{itemize}
\item
  \href{https://swprs.org/medien-navigator/}{Analysen}

  \begin{itemize}
  \tightlist
  \item
    \href{https://swprs.org/medien-navigator/}{Navigator}
  \item
    \href{https://swprs.org/der-propaganda-schluessel/}{Techniken}
  \item
    \href{https://swprs.org/propaganda-in-der-wikipedia/}{Wikipedia}
  \item
    \href{https://swprs.org/logik-imperialer-kriege/}{Kriege}
  \end{itemize}
\item
  \href{https://swprs.org/netzwerk-medien-schweiz/}{Netzwerke}

  \begin{itemize}
  \tightlist
  \item
    \href{https://swprs.org/netzwerk-medien-schweiz/}{Schweiz}
  \item
    \href{https://swprs.org/netzwerk-medien-deutschland/}{Deutschland}
  \item
    \href{https://swprs.org/medien-in-oesterreich/}{Österreich}
  \item
    \href{https://swprs.org/das-american-empire-und-seine-medien/}{USA}
  \end{itemize}
\item
  \href{https://swprs.org/bericht-eines-journalisten/}{Fokus I}

  \begin{itemize}
  \tightlist
  \item
    \href{https://swprs.org/bericht-eines-journalisten/}{Journalistenbericht}
  \item
    \href{https://swprs.org/russische-propaganda/}{Russische Propaganda}
  \item
    \href{https://swprs.org/die-israel-lobby-fakten-und-mythen/}{Die
    »Israel-Lobby«}
  \item
    \href{https://swprs.org/geopolitik-und-paedokriminalitaet/}{Pädokriminalität}
  \end{itemize}
\item
  \href{https://swprs.org/migration-und-medien/}{Fokus II}

  \begin{itemize}
  \tightlist
  \item
    \href{https://swprs.org/covid-19-hinweis-ii/}{Coronavirus}
  \item
    \href{https://swprs.org/die-integrity-initiative/}{Integrity
    Initiative}
  \item
    \href{https://swprs.org/migration-und-medien/}{Migration \& Medien}
  \item
    \href{https://swprs.org/der-fall-magnitsky/}{Magnitsky Act}
  \end{itemize}
\item
  \href{https://swprs.org/kontakt/}{Projekt}

  \begin{itemize}
  \tightlist
  \item
    \href{https://swprs.org/kontakt/}{Kontakt}
  \item
    \href{https://swprs.org/uebersicht/}{Seitenübersicht}
  \item
    \href{https://swprs.org/medienspiegel/}{Medienspiegel}
  \item
    \href{https://swprs.org/donationen/}{Donationen}
  \end{itemize}
\item
  \href{https://swprs.org/contact/}{English}
\end{itemize}

\protect\hyperlink{}{Open Search}

\hypertarget{covid-19--lettre-ouverte-du-professeur-sucharit-bhakdi-uxe0-la-chanceliuxe8re-allemande-dre-angela-merkel}{%
\section{Covid-19 : Lettre ouverte du Professeur Sucharit Bhakdi à la
Chancelière allemande Dre.
Angela~Merkel}\label{covid-19--lettre-ouverte-du-professeur-sucharit-bhakdi-uxe0-la-chanceliuxe8re-allemande-dre-angela-merkel}}

\includegraphics{https://swprs.files.wordpress.com/2020/03/bakhdi-letter-header.png?w=736\&h=297}

\textbf{Languages}:
\href{https://swprs.org/offener-brief-von-professor-sucharit-bhakdi-an-bundeskanzlerin-dr-angela-merkel/}{DE},
\href{https://swprs.org/open-letter-from-professor-sucharit-bhakdi-to-german-chancellor-dr-angela-merkel/}{EN},
\href{https://swprs.org/professor-sucharit-bhakdi-avalik-kiri-angela-merkelile/}{EE},
\href{http://piensachile.com/2020/03/carta-abierta-a-angela-merkel/}{ES},
\href{https://swprs.org/covid-19-lettre-ouverte-du-professeur-sucharit-bhakdi-a-la-chanceliere-allemande-dre-angela-merkel/}{FR},
\href{https://swprs.org/professor-bhakdi-open-letter-greek/}{GR},
\href{https://yanivhamo.com/open-letter-from-professor-sucharit-bhakdi-to-german-chancellor-dr-angela-merkel-hebrew/}{HE},
\href{https://swprs.org/lettera-aperta-del-professor-sucharit-bhakdi-al-cancelliere-tedesco-dr-angela-merkel/}{IT},
\href{https://swprs.org/open-brief-van-professor-sucharit-bhakdi-aan-de-duitse-bondskanselier-dr-angela-merkel/}{NL},
\href{https://swprs.org/carta-aberta-do-professor-sucharit-bhakdi-a-chanceler-alema-dra-angela-merkel/}{PT},
\href{https://swprs.org/\%d0\%be\%d1\%82\%d0\%ba\%d1\%80\%d1\%8b\%d1\%82\%d0\%be\%d0\%b5-\%d0\%bf\%d0\%b8\%d1\%81\%d1\%8c\%d0\%bc\%d0\%be-\%d0\%bf\%d1\%80\%d0\%be\%d1\%84\%d0\%b5\%d1\%81\%d1\%81\%d0\%be\%d1\%80\%d0\%b0-\%d1\%81\%d1\%83\%d1\%87\%d0\%b0\%d1\%80\%d0\%b8\%d1\%82\%d0\%b0/}{RU},
\href{https://alatyr.sk/open-letter-from-professor_sk.htm}{SK},
\href{https://swprs.org/prof-dr-sucharit-bhakdiden-basbakan-dr-angela-merkele-acik-mektup/}{TR}\\
\textbf{Partagez sur}:
\href{https://twitter.com/intent/tweet?url=https://swprs.org/covid-19-lettre-ouverte-du-professeur-sucharit-bhakdi-a-la-chanceliere-allemande-dre-angela-merkel/}{Twitter}
/
\href{https://www.facebook.com/share.php?u=https://swprs.org/covid-19-lettre-ouverte-du-professeur-sucharit-bhakdi-a-la-chanceliere-allemande-dre-angela-merkel/}{Facebook}

Lettre ouverte du Professeur Sucharit Bhakdi à la Chancelière allemande
Dre. Angela Merkel. Article originel :
\href{https://swprs.org/offener-brief-von-professor-sucharit-bhakdi-an-bundeskanzlerin-dr-angela-merkel/}{Open
Letter from Professor Sucharit Bhakdi to German Chancellor Dr. Angela
Merkel}

\hypertarget{lettre-ouverte}{%
\subsubsection{Lettre ouverte}\label{lettre-ouverte}}

Chère Chancelière,

En tant qu'émérite de l'Université Johannes-Gutenberg de Mayence et
directeur de longue date de l'Institut de microbiologie médicale, je me
sens obligé de remettre en question de manière critique les restrictions
de grande envergure que nous nous imposons actuellement à nous-mêmes
dans la vie publique afin de réduire la propagation du virus COVID-19.

Il n'est expressément pas dans mon intention de minimiser les dangers du
virus ou de diffuser un message politique. J'estime toutefois qu'il est
de mon devoir d'apporter une contribution scientifique à la mise en
perspective des données et des faits actuels -- et, en outre, de poser
des questions qui risquent de se perdre dans le débat houleux.

La raison de mon inquiétude réside avant tout dans les conséquences
socio-économiques réellement imprévisibles des mesures d'endiguement
drastiques qui sont actuellement appliquées dans de grandes parties de
l'Europe et qui sont également déjà pratiquées à grande échelle en
Allemagne.

Mon souhait est de discuter de manière critique -- et avec la prévoyance
nécessaire -- des avantages et des inconvénients de la restriction de la
vie publique et des effets à long terme qui en résultent.

À cette fin, je suis confronté à cinq questions qui n'ont pas reçu de
réponse suffisante jusqu'à présent, mais qui sont indispensables pour
une analyse équilibrée.

Je voudrais vous demander de vous exprimer rapidement et, en même temps,
d'inviter le gouvernement fédéral à élaborer des stratégies qui
protègent efficacement les groupes à risque sans restreindre la vie
publique de manière générale et à semer les graines d'une polarisation
de la société encore plus intense que celle qui existe déjà.

Avec le plus grand respect,\textbf{Prof. em. Dr. med. Sucharit Bhakdi}

\hypertarget{1-statistiques}{%
\subparagraph{\texorpdfstring{\textbf{1.
Statistiques}}{1. Statistiques}}\label{1-statistiques}}

En infectiologie -- fondée par Robert Koch lui-même -- une distinction
traditionnelle est faite entre l'infection et la maladie. Une maladie
nécessite une manifestation clinique. Par conséquent, seuls les patients
présentant des symptômes tels que la fièvre ou la toux doivent être
inclus dans les statistiques en tant que nouveaux cas.

En d'autres termes, une nouvelle infection -- telle que mesurée par le
test COVID-19 -- ne signifie pas nécessairement que nous avons affaire à
un patient nouvellement malade qui a besoin d'un lit d'hôpital.
Toutefois, on suppose actuellement que cinq pour cent de toutes les
personnes infectées deviennent gravement malades et ont besoin d'une
ventilation. Les projections basées sur cette estimation suggèrent que
le système de santé pourrait être surchargé.

\textbf{Ma question} : Les projections ont-elles fait une distinction
entre les personnes infectées ne présentant pas de symptômes et les
patients réellement malades, c'est-à-dire les personnes qui développent
des symptômes ?

\hypertarget{2-dangerosituxe9}{%
\subparagraph{\texorpdfstring{\textbf{2.
Dangerosité}}{2. Dangerosité}}\label{2-dangerosituxe9}}

Un certain nombre de coronavirus circulent depuis longtemps -- largement
inaperçus des médias. S'il s'avérait que le virus COVID-19 ne présente
pas un potentiel de risque nettement plus élevé que les coronavirus déjà
en circulation, toutes les contre-mesures deviendraient évidemment
inutiles.

Le Journal international des agents antimicrobiens, reconnu au niveau
international, publiera bientôt un article qui traitera précisément de
cette question. Les résultats préliminaires de l'étude sont déjà
visibles aujourd'hui et permettent de conclure que le nouveau virus
n'est PAS différent des coronavirus traditionnels en termes de
dangerosité. Les auteurs expriment cette conclusion dans le titre de
leur article ``SARS-CoV-2 : peur contre données''. {[}3{]}

\textbf{Ma question} : Comment la charge de travail actuelle des unités
de soins intensifs avec des patients ayant reçu un diagnostic de
COVID-19 se compare-t-elle à celle d'autres infections à coronavirus, et
dans quelle mesure ces données seront-elles prises en compte dans les
décisions futures du gouvernement fédéral ? En outre : L'étude ci-dessus
a-t-elle été prise en compte dans la planification jusqu'à présent ?~
Ici aussi, bien sûr, ``diagnostiqué'' signifie que le virus joue un rôle
décisif dans l'état de santé du patient, et non que les maladies
antérieures jouent un rôle plus important.

\hypertarget{3-diffusion}{%
\subparagraph{\texorpdfstring{\textbf{3.
Diffusion}}{3. Diffusion}}\label{3-diffusion}}

Selon un rapport du Süddeutsche Zeitung, même l'Institut Robert Koch,
très cité, ne sait pas exactement quelle quantité est testée pour le
COVID-19. Il est cependant un fait que l'on a récemment observé en
Allemagne une augmentation rapide du nombre de cas, à mesure que le
volume des tests augmente. {[}4{]}

Il est donc raisonnable de penser que le virus s'est déjà propagé de
manière inaperçue dans la population saine. Cela aurait deux
conséquences : premièrement, cela signifierait que le taux de mortalité
officiel -- le 26 mars 2020, par exemple, il y avait 206 décès dus à
environ 37 300 infections, soit 0,55 \% {[}5{]} -- est trop élevé ; et
deuxièmement, cela signifierait qu'il ne serait guère possible
d'empêcher le virus de se propager dans la population saine.

\textbf{Ma question est la suivante} : Y a-t-il déjà eu un échantillon
aléatoire de la population générale en bonne santé pour valider la
propagation réelle du virus, ou est-ce prévu dans un avenir proche ?

\hypertarget{4-mortalituxe9}{%
\subparagraph{\texorpdfstring{\textbf{4.
Mortalité}}{4. Mortalité}}\label{4-mortalituxe9}}

La crainte d'une augmentation du taux de mortalité en Allemagne
(actuellement 0,55 \%) fait actuellement l'objet d'une attention
médiatique particulièrement intense. De nombreuses personnes craignent
qu'il n'augmente comme en Italie (10 \%) et en Espagne (7 \%) si des
mesures ne sont pas prises à temps.

En même temps, l'erreur est commise dans le monde entier de signaler les
décès liés au virus dès qu'il est établi que le virus était présent au
moment du décès -- indépendamment d'autres facteurs. Cela viole un
principe de base de l'infectiologie : un diagnostic ne peut être établi
que lorsqu'il est certain qu'un agent a joué un rôle important dans la
maladie ou le décès. L'Association des sociétés médicales scientifiques
d'Allemagne écrit expressément dans ses directives : ``Outre la cause du
décès, il faut indiquer une chaîne causale, la maladie sous-jacente
correspondante occupant la troisième place sur le certificat de décès.
Parfois, il faut également indiquer les chaînes causales à quatre
liens''. {[}6{]}

À l'heure actuelle, il n'existe aucune information officielle sur la
question de savoir si, au moins rétrospectivement, des analyses plus
critiques des dossiers médicaux ont été entreprises pour déterminer
combien de décès ont été effectivement causés par le virus.

\textbf{Ma question est la suivante} : L'Allemagne a-t-elle simplement
suivi cette tendance d'un soupçon général de COVID-19 ? Et : a-t-elle
l'intention de poursuivre cette catégorisation sans esprit critique
comme dans d'autres pays ? Comment, dès lors, faire la distinction entre
les véritables décès liés au coronavirus et la présence accidentelle du
virus au moment du décès ?

\hypertarget{5-comparabilituxe9}{%
\subparagraph{\texorpdfstring{\textbf{5.
Comparabilité}}{5. Comparabilité}}\label{5-comparabilituxe9}}

La situation épouvantable en Italie est utilisée à plusieurs reprises
comme scénario de référence. Cependant, le véritable rôle du virus dans
ce pays est totalement incertain pour de nombreuses raisons -- non
seulement parce que les points 3 et 4 ci-dessus s'appliquent également
ici, mais aussi parce qu'il existe des facteurs externes exceptionnels
qui rendent ces régions particulièrement vulnérables.

L'un de ces facteurs est l'augmentation de la pollution atmosphérique
dans le nord de l'Italie. Selon les estimations de l'OMS, cette
situation, même sans le virus, a entraîné plus de 8 000 décès
supplémentaires par an en 2006 dans les 13 plus grandes villes d'Italie
seulement. La situation n'a pas beaucoup changé depuis lors{[}7{]}.
Enfin, il a également été démontré que la pollution atmosphérique
augmente considérablement le risque de maladies pulmonaires virales chez
les personnes très jeunes et âgées. {[}9{]}

En outre, 27,4 \% de la population particulièrement vulnérable de ce
pays vit avec des jeunes, et en Espagne, ce pourcentage atteint 33,5 \%.
En Allemagne, ce chiffre n'est que de 7 \% {[}10{]}. De plus, selon le
professeur Reinhard Busse, chef du département de gestion des soins de
santé à l'Université technique de Berlin, l'Allemagne est nettement
mieux équipée que l'Italie en termes d'unités de soins intensifs -- par
un facteur d'environ 2,5 {[}11{]}.

\textbf{Ma question :} Quels efforts sont faits pour sensibiliser la
population à ces différences élémentaires et pour lui faire comprendre
que des scénarios comme ceux de l'Italie ou de l'Espagne ne sont pas
réalistes ici ?

\hypertarget{references}{%
\subparagraph{\texorpdfstring{\textbf{References:}}{References:}}\label{references}}

{[}1{]} Fachwörterbuch Infektionsschutz und Infektionsepidemiologie.
\href{https://www.rki.de/DE/Content/Service/Publikationen/Fachwoerterbuch_Infektionsschutz.html}{Fachwörter
-- Definitionen -- Interpretationen}. Robert Koch-Institut, Berlin 2015.
(abgerufen am 26.3.2020)

{[}2{]} Killerby et al., Human Coronavirus Circulation in the United
States 2014--2017. J Clin Virol. 2018, 101, 52-56

{[}3{]} Roussel et al. SARS-CoV-2: Fear Versus Data. Int. J. Antimicrob.
Agents 2020, 105947

{[}4{]} Charisius, H.
\href{https://www.sueddeutsche.de/gesundheit/covid-19-coronavirus-testverfahren-1.4855487}{Covid-19:
Wie gut testet Deutschland?} Süddeutsche Zeitung. (abgerufen am
27.3.2020)

{[}5{]} Johns Hopkins University,
\href{https://coronavirus.jhu.edu/map.html}{Coronavirus Resource
Center}. 2020. (abgerufen am 26.3.2020)

{[}6{]} S1-Leitlinie 054-001,
\href{https://www.awmf.org/uploads/tx_szleitlinien/054-002l_S1_Regeln-zur-Durchfuehrung-der-aerztlichen-Leichenschau_2018-02_01.pdf}{Regeln
zur Durchführung der ärztlichen Leichenschau}. AWMF Online (abgerufen am
26.3.2020)

{[}7{]} Martuzzi et al. Health Impact of PM10 and Ozone in 13 Italian
Cities. World Health Organization Regional Office for Europe. WHOLIS
number E88700 2006

{[}8{]} European Environment Agency,
\href{https://www.eea.europa.eu/themes/air/country-fact-sheets/2019-country-fact-sheets}{Air
Pollution Country Fact Sheets 2019}, (abgerufen am 26.3.2020)

{[}9{]} Croft et al. The Association between Respiratory Infection and
Air Pollution in the Setting of Air Quality Policy and Economic Change.
Ann. Am. Thorac. Soc. 2019, 16, 321--330.

{[}10{]} United Nations, Department of Economic and Social Affairs,
Population Division. Living Arrange­ments of Older Persons: A Report on
an Expanded International Dataset (ST/ESA/SER.A/407). 2017

{[}11{]} Deutsches Ärzteblatt,
\href{https://www.aerzteblatt.de/nachrichten/111029/Ueberlastung-deutscher-Krankenhaeuser-durch-COVID-19-laut-Experten-unwahrscheinlich}{Überlastung
deutscher Krankenhäuser durch COVID-19 laut Experten unwahrscheinlich},
(abgerufen am 26.3.2020)

\hypertarget{retour-uxe0-larticle-principal-a-swiss-doctor-on-covid-19}{%
\paragraph{\texorpdfstring{Retour à l'article principal:
\href{https://swprs.org/coronavirus-un-medecin-suisse-parle/}{A Swiss
Doctor on
Covid-19}}{Retour à l'article principal: A Swiss Doctor on Covid-19}}\label{retour-uxe0-larticle-principal-a-swiss-doctor-on-covid-19}}

\begin{center}\rule{0.5\linewidth}{\linethickness}\end{center}

\hypertarget{swiss-policy-research}{%
\subsubsection{Swiss Policy Research}\label{swiss-policy-research}}

\begin{itemize}
\tightlist
\item
  \href{https://swprs.org/kontakt/}{Kontakt}
\item
  \href{https://swprs.org/uebersicht/}{Übersicht}
\item
  \href{https://swprs.org/donationen/}{Donationen}
\item
  \href{https://swprs.org/disclaimer/}{Disclaimer}
\end{itemize}

\hypertarget{english}{%
\subsubsection{English}\label{english}}

\begin{itemize}
\tightlist
\item
  \href{https://swprs.org/contact/}{About Us / Contact}
\item
  \href{https://swprs.org/media-navigator/}{The Media Navigator}
\item
  \href{https://swprs.org/the-american-empire-and-its-media/}{The CFR
  and the Media}
\item
  \href{https://swprs.org/donations/}{Donations}
\end{itemize}

\hypertarget{follow-by-email}{%
\subsubsection{Follow by email}\label{follow-by-email}}

Follow

\href{https://wordpress.com/?ref=footer_custom_com}{WordPress.com}.

\protect\hyperlink{}{Up ↑}

Post to

\protect\hyperlink{}{Cancel}

\includegraphics{https://pixel.wp.com/b.gif?v=noscript}
