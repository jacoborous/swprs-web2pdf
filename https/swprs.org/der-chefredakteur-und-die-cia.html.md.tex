\protect\hyperlink{content}{Skip to content}

\href{https://swprs.org/}{}

\protect\hyperlink{search-container}{Search}

Search for:

\href{https://swprs.org/}{\includegraphics{https://swprs.files.wordpress.com/2020/05/swiss-policy-research-logo-300.png}}

\href{https://swprs.org/}{Swiss Policy Research}

Geopolitics and Media

Menu

\begin{itemize}
\tightlist
\item
  \href{https://swprs.org}{Start}
\item
  \href{https://swprs.org/srf-propaganda-analyse/}{Studien}

  \begin{itemize}
  \tightlist
  \item
    \href{https://swprs.org/srf-propaganda-analyse/}{SRF / ZDF}
  \item
    \href{https://swprs.org/die-nzz-studie/}{NZZ-Studie}
  \item
    \href{https://swprs.org/der-propaganda-multiplikator/}{Agenturen}
  \item
    \href{https://swprs.org/die-propaganda-matrix/}{Medienmatrix}
  \end{itemize}
\item
  \href{https://swprs.org/medien-navigator/}{Analysen}

  \begin{itemize}
  \tightlist
  \item
    \href{https://swprs.org/medien-navigator/}{Navigator}
  \item
    \href{https://swprs.org/der-propaganda-schluessel/}{Techniken}
  \item
    \href{https://swprs.org/propaganda-in-der-wikipedia/}{Wikipedia}
  \item
    \href{https://swprs.org/logik-imperialer-kriege/}{Kriege}
  \end{itemize}
\item
  \href{https://swprs.org/netzwerk-medien-schweiz/}{Netzwerke}

  \begin{itemize}
  \tightlist
  \item
    \href{https://swprs.org/netzwerk-medien-schweiz/}{Schweiz}
  \item
    \href{https://swprs.org/netzwerk-medien-deutschland/}{Deutschland}
  \item
    \href{https://swprs.org/medien-in-oesterreich/}{Österreich}
  \item
    \href{https://swprs.org/das-american-empire-und-seine-medien/}{USA}
  \end{itemize}
\item
  \href{https://swprs.org/bericht-eines-journalisten/}{Fokus I}

  \begin{itemize}
  \tightlist
  \item
    \href{https://swprs.org/bericht-eines-journalisten/}{Journalistenbericht}
  \item
    \href{https://swprs.org/russische-propaganda/}{Russische Propaganda}
  \item
    \href{https://swprs.org/die-israel-lobby-fakten-und-mythen/}{Die
    »Israel-Lobby«}
  \item
    \href{https://swprs.org/geopolitik-und-paedokriminalitaet/}{Pädokriminalität}
  \end{itemize}
\item
  \href{https://swprs.org/migration-und-medien/}{Fokus II}

  \begin{itemize}
  \tightlist
  \item
    \href{https://swprs.org/covid-19-hinweis-ii/}{Coronavirus}
  \item
    \href{https://swprs.org/die-integrity-initiative/}{Integrity
    Initiative}
  \item
    \href{https://swprs.org/migration-und-medien/}{Migration \& Medien}
  \item
    \href{https://swprs.org/der-fall-magnitsky/}{Magnitsky Act}
  \end{itemize}
\item
  \href{https://swprs.org/kontakt/}{Projekt}

  \begin{itemize}
  \tightlist
  \item
    \href{https://swprs.org/kontakt/}{Kontakt}
  \item
    \href{https://swprs.org/uebersicht/}{Seitenübersicht}
  \item
    \href{https://swprs.org/medienspiegel/}{Medienspiegel}
  \item
    \href{https://swprs.org/donationen/}{Donationen}
  \end{itemize}
\item
  \href{https://swprs.org/contact/}{English}
\end{itemize}

\protect\hyperlink{}{Open Search}

\hypertarget{der-chefredakteur-und-die-cia}{%
\section{Der Chefredakteur und
die~CIA}\label{der-chefredakteur-und-die-cia}}

Weitere Sprachen:
\href{https://swprs.org/the-editor-in-chief-and-the-cia/}{English}

Die klandestine Zu­sam­men­arbeit zwischen west­lichen Geheim­diensten
und Medien ist seit langem
\href{http://carlbernstein.com/magazine_cia_and_media.php}{bekannt} und
vielfach
\href{http://www.amazon.de/Geheimdienst-Politik-Medien-Meinungsmache-Zeitgeschichte/dp/3897068796}{dokumentiert}.
Wie eng und um­fas­send bisweilen selbst füh­ren­de
deutsch­­­spra­­chige Jour­na­listen mit den Diens­ten kooperieren, dies
zeigt bei­spiel­haft der Fall von
\href{https://de.wikipedia.org/wiki/Otto_Schulmeister}{Otto
Schul­meister}.

Schul­meister war von 1961 bis 1989 Chef­re­dak­teur und Herausgeber der
\href{https://de.wikipedia.org/wiki/Die_Presse}{\emph{Presse}}, einer
der größ­ten und tra­di­tions­reich­sten Tages­­zeitungen Öster­reichs.
2009 wurde sein ehemaliges CIA-Dossier publik -- mit bemerkenswerten
Einzel­heiten zur ver­deckten Kol­la­bo­ration.

Das österreichische Nachrichten­magazin \emph{Profil}
\href{https://www.profil.at/home/ex-presse-chef-dienste-cia-otto-schulmeister-geheimdienst-239634}{berichtete}
wie folgt (Auszüge):

``Das Dossier zeugt von einer proble­ma­tischen, im Grunde ver­botenen
Be­zie­hung eines Jour­­na­listen zur CIA, der glaubte, eine Mission zu
erfüllen: Schulmeister (Deckname GRCAMERA) gestaltete Leit­artikel
argumentativ nach den Wünschen der CIA, unterdrückte Geschichten, wenn
sie dem US-Standpunkt schadeten, drängte seine Redakteure zur
Kontakt­aufnahme mit den in Wien statio­nier­ten Vertretern der
US-Regierung -- der sogenannten Herrenrunde -- und gab Informationen aus
Hinter­­grund­­ge­sprächen mit österreichischen Politikern und
Botschaftern des Ostblocks preis.

Begonnen hatte die Anwerbung mit den üblichen Erkundigungen über Dritte
und der Suche nach biografischen Schwach­stellen, die bei Gelegen­heit
auch gegen den Betroffenen eingesetzt werden konnten.

Von da an gingen bei Bedarf Anweisungen der CIA, wie diese oder jene
politische Situation einzuschätzen sei, direkt in das Büro des
Chef­redakteurs. Wenn es eilte und keine Zeit für ein persönliches
Treffen war, wurden die Unterlagen per Boten zuhanden Schulmeister
zugestellt.

Am 29. Oktober 1962 berichtete die CIA: »Material ausgehändigt. Es
erschien ein Leitartikel nach unseren Anweisungen.«

Am 28. Dezember 1962 hieß es: »Der Herrenabend hat sich ausgezahlt. Die
politische Linie der \emph{Presse} könnte von unserem Standpunkt aus
kaum besser sein. () Wir können Artikel unter­bringen. Nach Anweisung
der Zentrale wurde dies betreffend Kuba-Krise gewünscht. () Ich traf
GRCAMERA am selben Tag und übergab ihm Material, das die Zentrale
veröffentlicht sehen will. Die Geschichte erschien in der
Sonntags­ausgabe in Form eines von ihm gezeichneten Leitartikels auf der
ersten Seite. () GRCAMERA hat um Unterstützung ­gebeten, damit sein Sohn
ein Quäker­stipendium bekommt, () es sieht nicht so aus, als würde er
genommen, doch sind auch das nützliche Mittel, die Verbindung zu
festigen.«

Am 17. Dezember 1962 berichtete die CIA, man sei beunruhigt über einen
Artikel, den die \emph{Presse}-Korrespondentin in Washington über die
geheime US-Unterstützung für das österreichische Bundes­heer
veröffentlichen will. »Die Geschichte ist richtig, () sollte aber nicht
zum jetzigen Zeitpunkt an die Öffentlichkeit kommen, um einen Aufruhr
über Österreichs Neutralität zu vermeiden. () Ich überzeugte GRCAMERA,
dass dies nur den Sowjets nützen würde. GRCAMERA stimmte zu, die
Geschichte nicht zu drucken () GRCAMERA ist erfreut, dass er als erster
Journalist vom neuen US-Botschafter in Wien empfangen wird. () Für ihn
ist es ein Scoop.«

Am 19. September 1963 wollte die CIA die Bericht­erstattung über eine
peinliche Spionage­affäre im Innen­ministerium unterbinden: »GRCAMERA ()
versprach, nichts über diesen Fall zu veröffent­lichen.«

Am 3. April 1964 **** wieder Lob für die inhaltliche Ausrichtung der
Zeitung: »lässt kaum zu wünschen übrig«. Oft sei Schulmeister sogar den
Anweisungen der Zentrale voraus. Nur die USA-Korrespondentin der
\emph{Presse} verursache kleinere Irritationen. »Das bedeutet nicht,
dass Schulmeister unser Agent ist. () Doch wir können ihn führen, gerade
so, als wäre er unser Agent ().«

Am 19. Januar 1965 **** wurde Schulmeister ­Material über die
Kongo-Krise übergeben. »GRCAMERA sagte, er müsse nicht überzeugt werden
von den amerikanischen Interessen im Kongo, doch () habe die \emph{New
York Times} erst kürzlich einen Bericht der kongolesischen Regierung
über ein Massa­ker an den dortigen Rebellen gebracht, der auch in
Österreich aufgegriffen wurde. Das bringe seine Zeitung in die
unangenehme Situation, dass er Geschichten veröffentliche, die konträr
zur Version der \emph{Times} liegen. () GRCAMERA meinte, wir sollten die
\emph{New York Times} auf Linie bringen.«

Im Laufe des Jahres 1965 konnte die CIA einen weiteren Konfidenten in
der \emph{Presse} gewinnen, von dem Schulmeister allerdings nicht weiß,
wie im Bericht vom 12. Oktober 1965 zu lesen ist.

In den darauf­folgenden Jahren gingen zahlreiche CIA-Unterlagen über den
Krieg in Vietnam und andere Brennpunkte der US-Politik über den
Schreibtisch des \emph{Presse}-Chefredakteurs. Schulmeister bat auch von
sich aus des Öfteren um entsprechende Analysen. 1968 wurde Schulmeister
im Rahmen eines \emph{Red Carpet}-Programms in die USA eingeladen.

Als sich Anfang der siebziger Jahre eine Entspannungs­politik zwischen
den Blöcken abzeichnete, begann es in der Beziehung Schulmeisters zur
CIA zu kriseln. () Schulmeister verhalte sich wie ein »flüchtender
Vogel«. Die CIA hatte zu diesem Zeitpunkt schon einen neuen, angeblich
»weniger ausweichenden« Konfidenten ins Auge gefasst, der in den Akten
IDENTITY genannt wird. Nach der Beschreibung -- Innen­politik­redakteur,
Studium in den USA, Wochenend­haus in Nieder­österreich -- hatten sie es
auf den zukünftigen \emph{Presse}-Chefredakteur Thomas Chorherr
abgesehen.

Thomas Chorherr gegenüber \emph{profil:} »Ich hatte wohl häufig mit
Sekretären der US-Botschaft zu tun, dass dabei die CIA im Spiel war,
wusste ich nicht. Ich hatte auch keinen Verdacht, und ich habe ein
reines Gewissen.« Er könne nicht glauben, dass Schulmeister mit der CIA
Kontakt gehalten hatte. »Das hätte ich merken müssen«, meint Chorherr.

Nach offizieller Darstellung der \emph{Presse}-Homepage machte
Schulmeister die \emph{Presse} in den sechziger Jahren zu einem
»Reservat unabhängigen Denkens«.''

In den Schulmeister-Akten fanden sich zudem Hinweise auf CIA-Konfidenten
in anderen bekannten Medien Österreichs, so im \emph{ORF}, dem
\emph{Kurier} und~ den \emph{Salz­burger Nach­richten}. Der frühere
ORF-General­intendant Gerd Bacher sagte, er habe nie ein
Kooperationsangebot der CIA erhalten, fügte indes
\href{https://www.diepresse.com/471594/otto-schulmeister-in-den-akten-der-cia}{hinzu}:
»Wenn man mich gefragt hätte, wäre ich dabei gewesen.«

\hypertarget{siehe-auch}{%
\paragraph{Siehe auch}\label{siehe-auch}}

\begin{itemize}
\tightlist
\item
  \href{https://swprs.org/video-the-cia-and-the-media/}{The CIA and the
  Media (Doku)}
\item
  \href{https://swprs.org/die-integrity-initiative/}{Die
  Integrity-Initiative}
\item
  \href{https://swprs.org/medien-in-oesterreich/}{Medien in Österreich}
\end{itemize}

\begin{center}\rule{0.5\linewidth}{\linethickness}\end{center}

Publiziert: Mai 2016

\hypertarget{swiss-policy-research}{%
\subsubsection{Swiss Policy Research}\label{swiss-policy-research}}

\begin{itemize}
\tightlist
\item
  \href{https://swprs.org/kontakt/}{Kontakt}
\item
  \href{https://swprs.org/uebersicht/}{Übersicht}
\item
  \href{https://swprs.org/donationen/}{Donationen}
\item
  \href{https://swprs.org/disclaimer/}{Disclaimer}
\end{itemize}

\hypertarget{english}{%
\subsubsection{English}\label{english}}

\begin{itemize}
\tightlist
\item
  \href{https://swprs.org/contact/}{About Us / Contact}
\item
  \href{https://swprs.org/media-navigator/}{The Media Navigator}
\item
  \href{https://swprs.org/the-american-empire-and-its-media/}{The CFR
  and the Media}
\item
  \href{https://swprs.org/donations/}{Donations}
\end{itemize}

\hypertarget{follow-by-email}{%
\subsubsection{Follow by email}\label{follow-by-email}}

Follow

\href{https://wordpress.com/?ref=footer_custom_com}{WordPress.com}.

\protect\hyperlink{}{Up ↑}

\includegraphics{https://pixel.wp.com/b.gif?v=noscript}
