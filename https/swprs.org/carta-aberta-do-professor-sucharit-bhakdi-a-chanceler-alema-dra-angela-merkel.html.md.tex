\protect\hyperlink{content}{Skip to content}

\href{https://swprs.org/}{}

\protect\hyperlink{search-container}{Search}

Search for:

\href{https://swprs.org/}{\includegraphics{https://swprs.files.wordpress.com/2020/05/swiss-policy-research-logo-300.png}}

\href{https://swprs.org/}{Swiss Policy Research}

Geopolitics and Media

Menu

\begin{itemize}
\tightlist
\item
  \href{https://swprs.org}{Start}
\item
  \href{https://swprs.org/srf-propaganda-analyse/}{Studien}

  \begin{itemize}
  \tightlist
  \item
    \href{https://swprs.org/srf-propaganda-analyse/}{SRF / ZDF}
  \item
    \href{https://swprs.org/die-nzz-studie/}{NZZ-Studie}
  \item
    \href{https://swprs.org/der-propaganda-multiplikator/}{Agenturen}
  \item
    \href{https://swprs.org/die-propaganda-matrix/}{Medienmatrix}
  \end{itemize}
\item
  \href{https://swprs.org/medien-navigator/}{Analysen}

  \begin{itemize}
  \tightlist
  \item
    \href{https://swprs.org/medien-navigator/}{Navigator}
  \item
    \href{https://swprs.org/der-propaganda-schluessel/}{Techniken}
  \item
    \href{https://swprs.org/propaganda-in-der-wikipedia/}{Wikipedia}
  \item
    \href{https://swprs.org/logik-imperialer-kriege/}{Kriege}
  \end{itemize}
\item
  \href{https://swprs.org/netzwerk-medien-schweiz/}{Netzwerke}

  \begin{itemize}
  \tightlist
  \item
    \href{https://swprs.org/netzwerk-medien-schweiz/}{Schweiz}
  \item
    \href{https://swprs.org/netzwerk-medien-deutschland/}{Deutschland}
  \item
    \href{https://swprs.org/medien-in-oesterreich/}{Österreich}
  \item
    \href{https://swprs.org/das-american-empire-und-seine-medien/}{USA}
  \end{itemize}
\item
  \href{https://swprs.org/bericht-eines-journalisten/}{Fokus I}

  \begin{itemize}
  \tightlist
  \item
    \href{https://swprs.org/bericht-eines-journalisten/}{Journalistenbericht}
  \item
    \href{https://swprs.org/russische-propaganda/}{Russische Propaganda}
  \item
    \href{https://swprs.org/die-israel-lobby-fakten-und-mythen/}{Die
    »Israel-Lobby«}
  \item
    \href{https://swprs.org/geopolitik-und-paedokriminalitaet/}{Pädokriminalität}
  \end{itemize}
\item
  \href{https://swprs.org/migration-und-medien/}{Fokus II}

  \begin{itemize}
  \tightlist
  \item
    \href{https://swprs.org/covid-19-hinweis-ii/}{Coronavirus}
  \item
    \href{https://swprs.org/die-integrity-initiative/}{Integrity
    Initiative}
  \item
    \href{https://swprs.org/migration-und-medien/}{Migration \& Medien}
  \item
    \href{https://swprs.org/der-fall-magnitsky/}{Magnitsky Act}
  \end{itemize}
\item
  \href{https://swprs.org/kontakt/}{Projekt}

  \begin{itemize}
  \tightlist
  \item
    \href{https://swprs.org/kontakt/}{Kontakt}
  \item
    \href{https://swprs.org/uebersicht/}{Seitenübersicht}
  \item
    \href{https://swprs.org/medienspiegel/}{Medienspiegel}
  \item
    \href{https://swprs.org/donationen/}{Donationen}
  \end{itemize}
\item
  \href{https://swprs.org/contact/}{English}
\end{itemize}

\protect\hyperlink{}{Open Search}

\hypertarget{carta-aberta-do-professor-sucharit-bhakdi-uxe0-chanceler-alemuxe3-dra-angela-merkel}{%
\section{Carta aberta do professor Sucharit Bhakdi à chanceler alemã
Dra.
Angela~Merkel}\label{carta-aberta-do-professor-sucharit-bhakdi-uxe0-chanceler-alemuxe3-dra-angela-merkel}}

\includegraphics{https://swprs.files.wordpress.com/2020/03/bakhdi-letter-header.png?w=736\&h=297}

\textbf{Línguas}:
\href{https://swprs.org/offener-brief-von-professor-sucharit-bhakdi-an-bundeskanzlerin-dr-angela-merkel/}{DE},
\href{https://swprs.org/open-letter-from-professor-sucharit-bhakdi-to-german-chancellor-dr-angela-merkel/}{EN};
\href{https://swprs.org/professor-sucharit-bhakdi-avalik-kiri-angela-merkelile/}{EE},
\href{http://piensachile.com/2020/03/carta-abierta-a-angela-merkel/}{ES},
\href{https://swprs.org/covid-19-lettre-ouverte-du-professeur-sucharit-bhakdi-a-la-chanceliere-allemande-dre-angela-merkel/}{FR},
\href{https://yanivhamo.com/open-letter-from-professor-sucharit-bhakdi-to-german-chancellor-dr-angela-merkel-hebrew/}{HE},
\href{https://swprs.org/lettera-aperta-del-professor-sucharit-bhakdi-al-cancelliere-tedesco-dr-angela-merkel/}{IT},
\href{https://swprs.org/open-brief-van-professor-sucharit-bhakdi-aan-de-duitse-bondskanselier-dr-angela-merkel/}{NL},
\href{https://swprs.org/carta-aberta-do-professor-sucharit-bhakdi-a-chanceler-alema-dra-angela-merkel/}{PT},
\href{https://swprs.org/\%d0\%be\%d1\%82\%d0\%ba\%d1\%80\%d1\%8b\%d1\%82\%d0\%be\%d0\%b5-\%d0\%bf\%d0\%b8\%d1\%81\%d1\%8c\%d0\%bc\%d0\%be-\%d0\%bf\%d1\%80\%d0\%be\%d1\%84\%d0\%b5\%d1\%81\%d1\%81\%d0\%be\%d1\%80\%d0\%b0-\%d1\%81\%d1\%83\%d1\%87\%d0\%b0\%d1\%80\%d0\%b8\%d1\%82\%d0\%b0/}{RU},
\href{https://alatyr.sk/open-letter-from-professor_sk.htm}{SK},
\href{https://swprs.org/prof-dr-sucharit-bhakdiden-basbakan-dr-angela-merkele-acik-mektup/}{TR}

Uma carta aberta do Dr. Sucharit Bhakdi, Professor Emérito de
Microbiologia Médica da Universidade Johannes Gutenberg Mainz, à
Chanceler alemã Dra. Angela Merkel. O Professor Bhakdi apela a uma
reavaliação urgente da resposta ao Covid-19 e faz cinco perguntas
cruciais à Chanceler. A carta tem a data de 26 de Março. Trata-se de uma
tradução inoficial; ver a carta original em alemão
\href{https://swprs.org/offener-brief-von-professor-sucharit-bhakdi-an-bundeskanzlerin-dr-angela-merkel/}{como
um PDF}.

\hypertarget{carta-aberta}{%
\subsubsection{Carta aberta}\label{carta-aberta}}

Caro Chanceler,

Como Emérito da Universidade Johannes-Gutenberg-Universidade de Mainz e
durante muitos anos director do Instituto de Microbiologia e Higiene
Médica de Mainz, sinto-me obrigado a questionar criticamente as
restrições de grande alcance à vida pública que estamos actualmente a
assumir para reduzir a propagação do vírus COVID-19.

Não é expressamente minha intenção minimizar os perigos do vírus ou
difundir uma mensagem política. No entanto, sinto que é meu dever dar um
contributo científico para perspectivar a actual situação dos dados,
para colocar em perspectiva os factos que conhecemos até agora -- e
também para fazer perguntas que correm o risco de se perderem no
acalorado debate.

A razão da minha preocupação reside sobretudo nas consequências
socioeconómicas verdadeiramente imprevisíveis das medidas drásticas de
contenção que estão actualmente a ser aplicadas em grandes partes da
Europa e que também já estão a ser praticadas em grande escala na
Alemanha.

O meu desejo é discutir criticamente -- e com a necessária previsão --
as vantagens e desvantagens de restringir a vida pública e os
consequentes efeitos a longo prazo.

Para tal, sou confrontado com cinco questões que ainda não obtiveram
resposta adequada, mas que são indispensáveis para uma análise
equilibrada.

Gostaria de vos pedir que comentassem rapidamente e, ao mesmo tempo,
apelassem ao Governo Federal para que desenvolva estratégias que
protejam eficazmente os grupos de risco sem restringir a vida pública em
geral e semear as sementes para uma polarização ainda mais intensa da
sociedade do que já está a acontecer.

Com o maior respeito,

\textbf{Prof. em. Dr. med. Sucharit Bhakdi}

\hypertarget{1-statistik}{%
\subparagraph{\texorpdfstring{\textbf{1.
Statistik}}{1. Statistik}}\label{1-statistik}}

Na infectologia -- fundada pelo próprio Robert Koch -- é feita uma
distinção tradicional entre infecção e doença. Uma doença requer uma
manifestação clínica. 1{]} Por conseguinte, apenas os doentes com
sintomas como febre ou tosse devem ser incluídos nas estatísticas como
novos casos.

Por outras palavras, uma nova infecção -- tal como medida pelo teste
COVID-19 -- significa -- não necessariamente que estejamos a lidar com
um doente recentemente doente que precisa de uma cama de hospital.
Contudo, presume-se actualmente que 5\% de todas as pessoas infectadas
ficam gravemente doentes e necessitam de ventilação. As projecções
baseadas nesta estimativa sugerem que o sistema de saúde poderá estar
sobrecarregado.

\textbf{A minha pergunta}: as projecções fizeram a distinção entre
pessoas infectadas sem sintomas e doentes reais, ou seja, pessoas que
desenvolvem sintomas?

\hypertarget{2-perigo}{%
\subparagraph{\texorpdfstring{\textbf{2.
Perigo}}{2. Perigo}}\label{2-perigo}}

Há muito tempo que circula um certo número de coronavírus -- em grande
parte despercebido pelos meios de comunicação social. 2{]} Se se
verificar que o vírus COVID-19 não é considerado como tendo um potencial
de risco significativamente mais elevado do que os vírus corona já em
circulação, todas as contramedidas seriam obviamente desnecessárias.

O internacionalmente reconhecido ``International Journal of
Antimicrobial Agents'' publicará em breve um artigo que aborda
exactamente esta questão. Os resultados preliminares do estudo já hoje
podem ser vistos e levam a concluir que o novo vírus NÃO é diferente dos
vírus corona tradicionais em termos de perigo. É isto que os autores
afirmam no título do seu artigo ``SARS-CoV-2: Medo versus Dados''.
{[}3{]}

\textbf{A minha pergunta}: Como se compara a actual utilização de
unidades de cuidados intensivos com doentes diagnosticados com COVID-19
com outras infecções por vírus corona, e em que medida estes dados serão
tidos em conta na futura tomada de decisões do governo federal? Além
disso: o estudo acima referido foi tido em conta no planeamento até à
data? Evidentemente, o mesmo deve aplicar-se aqui: Diagnosticado
significa que o vírus também desempenha um papel decisivo no estado de
doença do doente e não que as doenças anteriores desempenham um papel
mais importante.

\hypertarget{3-repartiuxe7uxe3o}{%
\subparagraph{\texorpdfstring{\textbf{3.
Repartição}}{3. Repartição}}\label{3-repartiuxe7uxe3o}}

Segundo um relatório do Süddeutsche Zeitung , nem mesmo o muito citado
Instituto Robert Koch sabe exactamente quanto é testado para o COVID-19.
É um facto, porém, que se verificou recentemente um rápido aumento do
número de casos na Alemanha, à medida que o volume de testes aumenta.
4{]} A suspeita é portanto evidente de que o vírus já passou
despercebido na população saudável. Isto teria duas consequências: Em
primeiro lugar, isso significaria que a taxa de mortalidade oficial --
em 26.03.2020, por exemplo, havia 206 mortes de cerca de 37.300
infecções, ou 0,55 por cento {[}5{]} -- é demasiado elevado; e, em
segundo lugar, tornaria quase impossível evitar a propagação na
população saudável.

\textbf{A minha pergunta é}: já foram efectuados testes aleatórios à
população em geral saudável para validar a verdadeira propagação do
vírus ou está isso previsto para um futuro próximo?

\hypertarget{4-mortalidade}{%
\subparagraph{\texorpdfstring{\textbf{4.
Mortalidade}}{4. Mortalidade}}\label{4-mortalidade}}

O receio de um aumento da taxa de mortalidade na Alemanha (actualmente
de 0,55\%) é actualmente objecto de uma atenção mediática
particularmente intensa. Muitas pessoas receiam que possa disparar como
em Itália (10\%) e em Espanha (7\%) se não forem tomadas medidas a
tempo.

Ao mesmo tempo, está a ser cometido o erro, a nível mundial, de
comunicar as mortes relacionadas com o vírus assim que se verifica que
este estava presente no momento da morte -- independentemente de outros
factores. Isto viola um princípio básico da infectologia: só quando se
tiver estabelecido que um agente desempenhou um papel significativo na
doença ou na morte é que se pode fazer um diagnóstico. Nas suas
directrizes, a Associação das Sociedades Médicas Científicas da Alemanha
declara explicitamente: ``Para além da causa de morte, deve ser indicada
uma cadeia causal, com a doença subjacente correspondente em terceiro
lugar na certidão de óbito. Ocasionalmente, devem também ser indicadas
as cadeias causais de quatro elos''. {[}6{]}

Actualmente, não existe informação oficial sobre se, pelo menos
retrospectivamente, foram efectuadas análises mais críticas dos registos
médicos para determinar quantas mortes foram efectivamente devidas ao
vírus.

\textbf{A minha pergunta}: será que a Alemanha seguiu simplesmente a
tendência para a suspeita geral da COVID-19? E: tenciona continuar esta
categorização sem qualquer crítica, como noutros países? Como
distinguir, então, entre as verdadeiras mortes relacionadas com a Coroa
e a presença acidental do vírus no momento da morte?

\hypertarget{5-comparabilidade}{%
\subparagraph{\texorpdfstring{\textbf{5.
Comparabilidade}}{5. Comparabilidade}}\label{5-comparabilidade}}

A situação terrível em Itália é repetidamente utilizada como cenário de
referência. No entanto, o verdadeiro papel do vírus naquele país é
completamente obscuro por muitas razões -- não só porque os pontos 3 e 4
também se aplicam aqui, mas também porque existem factores externos
excepcionais que tornam estas regiões particularmente vulneráveis.

Uma delas é o aumento da poluição atmosférica no Norte de Itália.
Segundo estimativas da OMS, esta situação, mesmo sem o vírus, provocou
mais de 8 000 mortes adicionais por ano em 2006, só em Itália, nas 13
maiores cidades. 7{]} A situação não se alterou significativamente desde
então. 8{]} Por último, foi também demonstrado que a poluição
atmosférica aumenta consideravelmente o risco de doenças pulmonares
virais em pessoas muito jovens e idosas. {[}9{]}

Além disso, 27,4\% da população particularmente vulnerável deste país
vive com jovens, em Espanha até 33,5\%. Na Alemanha, este valor é apenas
de 7\% {[}10{]}.

Além disso, segundo o Prof. Dr. Reinhard Busse, chefe do Departamento de
Gestão dos Cuidados de Saúde da TU Berlim, a Alemanha está
significativamente melhor equipada do que a Itália em termos de unidades
de cuidados intensivos -- por um factor de cerca de 2,5 {[}11{]}.

\textbf{A minha pergunta}: que esforços estão a ser feitos para
sensibilizar a população para estas diferenças elementares e para fazer
compreender que cenários como os de Itália ou de Espanha não são
realistas neste domínio?

\hypertarget{referuxeancias}{%
\subparagraph{\texorpdfstring{\textbf{Referências}:}{Referências:}}\label{referuxeancias}}

{[}1{]} Fachwörterbuch Infektionsschutz und Infektionsepidemiologie.
\href{https://www.rki.de/DE/Content/Service/Publikationen/Fachwoerterbuch_Infektionsschutz.html}{Fachwörter
-- Definitionen -- Interpretationen}. Robert Koch-Institut, Berlin 2015.
(abgerufen am 26.3.2020)

{[}2{]} Killerby et al., Human Coronavirus Circulation in the United
States 2014--2017. J Clin Virol. 2018, 101, 52-56

{[}3{]} Roussel et al. SARS-CoV-2: Fear Versus Data. Int. J. Antimicrob.
Agents 2020, 105947

{[}4{]} Charisius, H.
\href{https://www.sueddeutsche.de/gesundheit/covid-19-coronavirus-testverfahren-1.4855487}{Covid-19:
Wie gut testet Deutschland?} Süddeutsche Zeitung. (abgerufen am
27.3.2020)

{[}5{]} Johns Hopkins University,
\href{https://coronavirus.jhu.edu/map.html}{Coronavirus Resource
Center}. 2020. (abgerufen am 26.3.2020)

{[}6{]} S1-Leitlinie 054-001,
\href{https://www.awmf.org/uploads/tx_szleitlinien/054-002l_S1_Regeln-zur-Durchfuehrung-der-aerztlichen-Leichenschau_2018-02_01.pdf}{Regeln
zur Durchführung der ärztlichen Leichenschau}. AWMF Online (abgerufen am
26.3.2020)

{[}7{]} Martuzzi et al. Health Impact of PM10 and Ozone in 13 Italian
Cities. World Health Organization Regional Office for Europe. WHOLIS
number E88700 2006

{[}8{]} European Environment Agency,
\href{https://www.eea.europa.eu/themes/air/country-fact-sheets/2019-country-fact-sheets}{Air
Pollution Country Fact Sheets 2019}, (abgerufen am 26.3.2020)

{[}9{]} Croft et al. The Association between Respiratory Infection and
Air Pollution in the Setting of Air Quality Policy and Economic Change.
Ann. Am. Thorac. Soc. 2019, 16, 321--330.

{[}10{]} United Nations, Department of Economic and Social Affairs,
Population Division. Living Arrange­ments of Older Persons: A Report on
an Expanded International Dataset (ST/ESA/SER.A/407). 2017

{[}11{]} Deutsches Ärzteblatt,
\href{https://www.aerzteblatt.de/nachrichten/111029/Ueberlastung-deutscher-Krankenhaeuser-durch-COVID-19-laut-Experten-unwahrscheinlich}{Überlastung
deutscher Krankenhäuser durch COVID-19 laut Experten unwahrscheinlich},
(abgerufen am 26.3.2020)

\begin{center}\rule{0.5\linewidth}{\linethickness}\end{center}

\textbf{O Professor Sucharit Bhakdi explicando a sua Carta Aberta:}

\begin{center}\rule{0.5\linewidth}{\linethickness}\end{center}

Partilhe esta carta em:
\href{https://twitter.com/intent/tweet?url=https://swprs.org/carta-aberta-do-professor-sucharit-bhakdi-a-chanceler-alema-dra-angela-merkel/}{Twitter}
/
\href{https://www.facebook.com/share.php?u=https://swprs.org/carta-aberta-do-professor-sucharit-bhakdi-a-chanceler-alema-dra-angela-merkel/}{Facebook}

Back to main article:
\href{https://swprs.org/a-swiss-doctor-on-covid-19/}{A Swiss Doctor on
Covid-19}

\hypertarget{swiss-policy-research}{%
\subsubsection{Swiss Policy Research}\label{swiss-policy-research}}

\begin{itemize}
\tightlist
\item
  \href{https://swprs.org/kontakt/}{Kontakt}
\item
  \href{https://swprs.org/uebersicht/}{Übersicht}
\item
  \href{https://swprs.org/donationen/}{Donationen}
\item
  \href{https://swprs.org/disclaimer/}{Disclaimer}
\end{itemize}

\hypertarget{english}{%
\subsubsection{English}\label{english}}

\begin{itemize}
\tightlist
\item
  \href{https://swprs.org/contact/}{About Us / Contact}
\item
  \href{https://swprs.org/media-navigator/}{The Media Navigator}
\item
  \href{https://swprs.org/the-american-empire-and-its-media/}{The CFR
  and the Media}
\item
  \href{https://swprs.org/donations/}{Donations}
\end{itemize}

\hypertarget{follow-by-email}{%
\subsubsection{Follow by email}\label{follow-by-email}}

Follow

\href{https://wordpress.com/?ref=footer_custom_com}{WordPress.com}.

\protect\hyperlink{}{Up ↑}

Post to

\protect\hyperlink{}{Cancel}

\includegraphics{https://pixel.wp.com/b.gif?v=noscript}
