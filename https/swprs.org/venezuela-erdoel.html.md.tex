\protect\hyperlink{content}{Skip to content}

\href{https://swprs.org/}{}

\protect\hyperlink{search-container}{Search}

Search for:

\href{https://swprs.org/}{\includegraphics{https://swprs.files.wordpress.com/2020/05/swiss-policy-research-logo-300.png}}

\href{https://swprs.org/}{Swiss Policy Research}

Geopolitics and Media

Menu

\begin{itemize}
\tightlist
\item
  \href{https://swprs.org}{Start}
\item
  \href{https://swprs.org/srf-propaganda-analyse/}{Studien}

  \begin{itemize}
  \tightlist
  \item
    \href{https://swprs.org/srf-propaganda-analyse/}{SRF / ZDF}
  \item
    \href{https://swprs.org/die-nzz-studie/}{NZZ-Studie}
  \item
    \href{https://swprs.org/der-propaganda-multiplikator/}{Agenturen}
  \item
    \href{https://swprs.org/die-propaganda-matrix/}{Medienmatrix}
  \end{itemize}
\item
  \href{https://swprs.org/medien-navigator/}{Analysen}

  \begin{itemize}
  \tightlist
  \item
    \href{https://swprs.org/medien-navigator/}{Navigator}
  \item
    \href{https://swprs.org/der-propaganda-schluessel/}{Techniken}
  \item
    \href{https://swprs.org/propaganda-in-der-wikipedia/}{Wikipedia}
  \item
    \href{https://swprs.org/logik-imperialer-kriege/}{Kriege}
  \end{itemize}
\item
  \href{https://swprs.org/netzwerk-medien-schweiz/}{Netzwerke}

  \begin{itemize}
  \tightlist
  \item
    \href{https://swprs.org/netzwerk-medien-schweiz/}{Schweiz}
  \item
    \href{https://swprs.org/netzwerk-medien-deutschland/}{Deutschland}
  \item
    \href{https://swprs.org/medien-in-oesterreich/}{Österreich}
  \item
    \href{https://swprs.org/das-american-empire-und-seine-medien/}{USA}
  \end{itemize}
\item
  \href{https://swprs.org/bericht-eines-journalisten/}{Fokus I}

  \begin{itemize}
  \tightlist
  \item
    \href{https://swprs.org/bericht-eines-journalisten/}{Journalistenbericht}
  \item
    \href{https://swprs.org/russische-propaganda/}{Russische Propaganda}
  \item
    \href{https://swprs.org/die-israel-lobby-fakten-und-mythen/}{Die
    »Israel-Lobby«}
  \item
    \href{https://swprs.org/geopolitik-und-paedokriminalitaet/}{Pädokriminalität}
  \end{itemize}
\item
  \href{https://swprs.org/migration-und-medien/}{Fokus II}

  \begin{itemize}
  \tightlist
  \item
    \href{https://swprs.org/covid-19-hinweis-ii/}{Coronavirus}
  \item
    \href{https://swprs.org/die-integrity-initiative/}{Integrity
    Initiative}
  \item
    \href{https://swprs.org/migration-und-medien/}{Migration \& Medien}
  \item
    \href{https://swprs.org/der-fall-magnitsky/}{Magnitsky Act}
  \end{itemize}
\item
  \href{https://swprs.org/kontakt/}{Projekt}

  \begin{itemize}
  \tightlist
  \item
    \href{https://swprs.org/kontakt/}{Kontakt}
  \item
    \href{https://swprs.org/uebersicht/}{Seitenübersicht}
  \item
    \href{https://swprs.org/medienspiegel/}{Medienspiegel}
  \item
    \href{https://swprs.org/donationen/}{Donationen}
  \end{itemize}
\item
  \href{https://swprs.org/contact/}{English}
\end{itemize}

\protect\hyperlink{}{Open Search}

\hypertarget{venezuela-es-geht-nicht-ums-erduxf6l}{%
\section{Venezuela: Es geht nicht
ums~Erdöl}\label{venezuela-es-geht-nicht-ums-erduxf6l}}

\includegraphics{https://swprs.files.wordpress.com/2019/02/venezuela.jpg?w=600\&h=302}

Februar 2019
(\href{https://swprs.org/venezuela-its-not-about-the-oil/}{English})

\textbf{Auch im erdölreichsten Land der Welt geht es nicht ums Erdöl.}

Im Beitrag \href{https://swprs.org/logik-imperialer-kriege/}{»Die Logik
imperialer Kriege«} wurde dargelegt, dass die US-Kriege der letzten
Jahrzehnte entgegen einer weitverbreiteten Annahme keine »Erdölkriege«
waren.

Jugoslawien, Afghanistan und Jemen (die kein Erdöl besitzen) ohnehin
nicht, aber auch Irak, Libyen und Syrien nicht: Bei Syrien beruhte die
Pipeline-These auf einer einzigen
\href{https://truthout.org/articles/the-war-against-the-assad-regime-is-not-a-pipeline-war/}{Falschmeldung},
das irakische Erdöl wurde
\href{https://theconversation.com/iraq-what-happened-to-the-oil-after-the-war-62188}{nie
privatisiert}, und die libysche Produktion brach
\href{http://www.businessinsider.com/r-how-unstable-is-libyas-oil-production-2018-3}{zusammen}.

Das Erdöl-Narrativ taucht nun auch im Falle Venezuelas wieder auf, denn
Venezuela verfügt über die größten konventionellen Erdölreserven der
Welt. Doch auch hier geht es nicht ums Erdöl.

Man sieht dies schon daran, dass die USA längst der größte Abnehmer
venezuelanischen Erdöls
\href{https://www.eia.gov/todayinenergy/detail.php?id=9651}{sind}. Und
wenn es den USA um die »Kontrolle« des Erdöls ginge, warum ist dann
ausgerechnet Venezuela (zusammen mit Bolivien) das letzte von den USA
geopolitisch unabhängige Land Südamerikas?

Die geopolitische Logik funktioniert eben gerade umgekehrt: Durch das
Erdöl ist Venezuela unabhängig, und diese Unabhängigkeit ist aus
amerikanischer Sicht ein Problem. Einerseits ist Venezuela ein
politisches, ökonomisches und militärisches
\href{https://www.heise.de/tp/features/Schlagabtausch-ueber-russische-Atombomber-in-Venezuela-4249160.html}{Einfallstor}
für Russland und China, andererseits unterstützt Venezuela weitere
»Feindstaaten« wie insbesondere Kuba und Nicaragua.

Durch einen Regimewechsel könnte dieses ganze russisch-chinesische
Latino-Netzwerk auf einen Schlag neutralisiert werden, im Idealfall ohne
einen Schuss abzugeben -- ein geostrategischer Schachzug zur erneuten
Durchsetzung der
\href{https://de.wikipedia.org/wiki/Monroe-Doktrin}{Monroe-Doktrin} von
1823.

\includegraphics{https://swprs.files.wordpress.com/2019/02/bolton-maduro-map.jpg?w=450\&h=263}

Als »Beleg« für das Erdöl-Narrativ wird hauptsächlich ein
\href{https://www.youtube.com/watch?v=8av-cPP1uPE}{FOX-Interview} mit
Sicherheitsberater John Bolton angeführt, in dem es um die Situation
Venezuelas insgesamt geht. Doch zum Erdöl sagte Bolton lediglich, der
»sozialistische Diktator« Maduro habe die Erdölindustrie verfallen
lassen, und der »neue Präsident« werde zusammen mit den USA wieder
investieren. Eine triviale Aussage.

(Nebenbei: Sowohl Bolton als auch die FOX-Moderatorin sind
\href{https://swprs.org/das-american-empire-und-seine-medien/}{Mitglieder}
des \emph{Council on Foreign Relations}, ebenso der neue
US-Sondergesandte für Venezuela, Elliott Abrams. Die positiven
Reaktionen der »Trump-kritischen« Medien und Klientelstaaten überraschen
daher \href{https://swprs.org/trump-medien-geopolitik/}{nicht}.)

Das ölreiche Venezuela kann weder den Weltmarktpreis bestimmen noch sein
Erdöl einem Land vorenthalten; es ist im Gegenteil auf die Exporterlöse
angewiesen. Aus demselben Grund belieferte die UdSSR im Kalten Krieg
Westeuropa ohne Unterbruch und ohne jede Drohung.

Der einzige Moment, in dem die militärische Kontrolle von Erdöl relevant
wird, ist bei einem aktiven Krieg. Nun ist aber offensichtlich, dass
Venezuela in einem Dritten Weltkrieg aufgrund seiner Lage kein Erdöl
nach China (oder sonstwohin) liefern wird, ganz egal, wie der Präsident
heißt. China wird sein Erdöl in einem solchen Fall aus Russland und
Kasachstan beziehen müssen.

(Nebenbei: Chinas größter
\href{https://oilprice.com/Energy/Crude-Oil/The-Battle-For-Chinas-Oil-Market.html}{Erdöllieferant}
ist derzeit das US-Protektorat Saudi-Arabien.)

Ebenso irreführend ist die
\href{https://foreignpolicy.com/2009/10/07/debunking-the-dumping-the-dollar-conspiracy/}{Petrodollar-Theorie}:
Länder wie Irak, Iran, Libyen, Russland oder Venezuela werden nicht zu
Feindstaaten, weil sie auf den US-Dollar verzichten, sondern sie
(müssen) verzichten, weil sie Feindstaaten sind -- und z.B.
US-Sanktionen umgehen möchten.
(\href{https://www.nachdenkseiten.de/?p=44020}{Mehr dazu})

Das Erdöl-Narrativ wird indes gerade auch von russischen Medien gerne
\href{https://www.rt.com/usa/449982-john-bolton-oil-venezuela/}{verbreitet}
-- offenbar spricht man lieber über gierige Amerikaner als von
russischen Einflusszonen in Lateinamerika.

\hypertarget{siehe-auch}{%
\paragraph{Siehe auch}\label{siehe-auch}}

\begin{itemize}
\tightlist
\item
  \href{https://swprs.org/logik-imperialer-kriege/}{Die Logik imperialer
  Kriege}
\item
  \href{https://swprs.org/das-american-empire-und-seine-medien/}{Das
  American Empire und seine Medien}
\item
  \href{https://swprs.org/russische-propaganda/}{Russische Propaganda}
\end{itemize}

\begin{center}\rule{0.5\linewidth}{\linethickness}\end{center}

Beitrag teilen auf:
\href{https://twitter.com/intent/tweet?url=https://swprs.org/venezuela-erdoel/}{Twitter}
/
\href{https://www.facebook.com/share.php?u=https://swprs.org/venezuela-erdoel/}{Facebook}\\
Publiziert: Februar 2019

\hypertarget{swiss-policy-research}{%
\subsubsection{Swiss Policy Research}\label{swiss-policy-research}}

\begin{itemize}
\tightlist
\item
  \href{https://swprs.org/kontakt/}{Kontakt}
\item
  \href{https://swprs.org/uebersicht/}{Übersicht}
\item
  \href{https://swprs.org/donationen/}{Donationen}
\item
  \href{https://swprs.org/disclaimer/}{Disclaimer}
\end{itemize}

\hypertarget{english}{%
\subsubsection{English}\label{english}}

\begin{itemize}
\tightlist
\item
  \href{https://swprs.org/contact/}{About Us / Contact}
\item
  \href{https://swprs.org/media-navigator/}{The Media Navigator}
\item
  \href{https://swprs.org/the-american-empire-and-its-media/}{The CFR
  and the Media}
\item
  \href{https://swprs.org/donations/}{Donations}
\end{itemize}

\hypertarget{follow-by-email}{%
\subsubsection{Follow by email}\label{follow-by-email}}

Follow

\href{https://wordpress.com/?ref=footer_custom_com}{WordPress.com}.

\protect\hyperlink{}{Up ↑}

Post to

\protect\hyperlink{}{Cancel}

\includegraphics{https://pixel.wp.com/b.gif?v=noscript}
