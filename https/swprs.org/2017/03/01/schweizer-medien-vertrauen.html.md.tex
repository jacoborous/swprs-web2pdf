\protect\hyperlink{content}{Skip to content}

\href{https://swprs.org/}{}

\protect\hyperlink{search-container}{Search}

Search for:

\href{https://swprs.org/}{\includegraphics{https://swprs.files.wordpress.com/2020/05/swiss-policy-research-logo-300.png}}

\href{https://swprs.org/}{Swiss Policy Research}

Geopolitics and Media

Menu

\begin{itemize}
\tightlist
\item
  \href{https://swprs.org}{Start}
\item
  \href{https://swprs.org/srf-propaganda-analyse/}{Studien}

  \begin{itemize}
  \tightlist
  \item
    \href{https://swprs.org/srf-propaganda-analyse/}{SRF / ZDF}
  \item
    \href{https://swprs.org/die-nzz-studie/}{NZZ-Studie}
  \item
    \href{https://swprs.org/der-propaganda-multiplikator/}{Agenturen}
  \item
    \href{https://swprs.org/die-propaganda-matrix/}{Medienmatrix}
  \end{itemize}
\item
  \href{https://swprs.org/medien-navigator/}{Analysen}

  \begin{itemize}
  \tightlist
  \item
    \href{https://swprs.org/medien-navigator/}{Navigator}
  \item
    \href{https://swprs.org/der-propaganda-schluessel/}{Techniken}
  \item
    \href{https://swprs.org/propaganda-in-der-wikipedia/}{Wikipedia}
  \item
    \href{https://swprs.org/logik-imperialer-kriege/}{Kriege}
  \end{itemize}
\item
  \href{https://swprs.org/netzwerk-medien-schweiz/}{Netzwerke}

  \begin{itemize}
  \tightlist
  \item
    \href{https://swprs.org/netzwerk-medien-schweiz/}{Schweiz}
  \item
    \href{https://swprs.org/netzwerk-medien-deutschland/}{Deutschland}
  \item
    \href{https://swprs.org/medien-in-oesterreich/}{Österreich}
  \item
    \href{https://swprs.org/das-american-empire-und-seine-medien/}{USA}
  \end{itemize}
\item
  \href{https://swprs.org/bericht-eines-journalisten/}{Fokus I}

  \begin{itemize}
  \tightlist
  \item
    \href{https://swprs.org/bericht-eines-journalisten/}{Journalistenbericht}
  \item
    \href{https://swprs.org/russische-propaganda/}{Russische Propaganda}
  \item
    \href{https://swprs.org/die-israel-lobby-fakten-und-mythen/}{Die
    »Israel-Lobby«}
  \item
    \href{https://swprs.org/geopolitik-und-paedokriminalitaet/}{Pädokriminalität}
  \end{itemize}
\item
  \href{https://swprs.org/migration-und-medien/}{Fokus II}

  \begin{itemize}
  \tightlist
  \item
    \href{https://swprs.org/covid-19-hinweis-ii/}{Coronavirus}
  \item
    \href{https://swprs.org/die-integrity-initiative/}{Integrity
    Initiative}
  \item
    \href{https://swprs.org/migration-und-medien/}{Migration \& Medien}
  \item
    \href{https://swprs.org/der-fall-magnitsky/}{Magnitsky Act}
  \end{itemize}
\item
  \href{https://swprs.org/kontakt/}{Projekt}

  \begin{itemize}
  \tightlist
  \item
    \href{https://swprs.org/kontakt/}{Kontakt}
  \item
    \href{https://swprs.org/uebersicht/}{Seitenübersicht}
  \item
    \href{https://swprs.org/medienspiegel/}{Medienspiegel}
  \item
    \href{https://swprs.org/donationen/}{Donationen}
  \end{itemize}
\item
  \href{https://swprs.org/contact/}{English}
\end{itemize}

\protect\hyperlink{}{Open Search}

\hypertarget{die-vertrauensfrage}{%
\section{\texorpdfstring{\href{https://swprs.org/2017/03/01/schweizer-medien-vertrauen/}{Die
Vertrauensfrage}}{Die Vertrauensfrage}}\label{die-vertrauensfrage}}

\includegraphics{https://swprs.files.wordpress.com/2017/03/foeg-jahrbuch_logo.png?w=500}

Das Forschungs­institut für Öf­fent­lich­­keit und Gesell­schaft der
Uni­ver­sität Zürich publi­ziert all­jähr­lich das »Jahr­buch Qualität
der Medien«. 2016 ver­mel­dete das In­sti­tut, das Ver­trau­en in die
Schwei­zer Me­dien sei
\href{http://www.foeg.uzh.ch/dam/jcr:7234c6d3-1f09-4d36-b6ab-f14e659d046e/Medienmitteilung_JB_2016_dt.pdf}{»weiter­hin
hoch«} -- so das Er­geb­nis eines Länder­ver­gleichs in
Zu­sam­men­ar­beit mit dem \emph{\emph{Reu­ters Insti­tute.}}

Doch wie hoch ist das Vertrauen in die Schweizer Medien nun wirklich?
Dazu findet man in der Mit­tei­lung des Instituts keine An­ga­ben. Und
auch die
\href{http://www.tagesanzeiger.ch/schweiz/standard/Diese-Menschen-sind-anfaellig-fuer-Populisten/story/23804017}{Zei­tungs­be­richte}
zur Studie er­wäh­nen diese wich­tige Kenn­zahl
\href{http://www.nzz.ch/schweiz/analyse-zum-medienvertrauen-oeffentliche-medien-staerken-auch-die-privaten-ld.128965}{nicht}.
Aus gutem Grund -- denn die Resultate sind er­schüt­ternd.

Demnach
\href{http://media.digitalnewsreport.org/wp-content/uploads/2018/11/Digital-News-Report-2016.pdf\#page=60}{halten}
nur noch 50\% der Schwei­zer Be­völ­ke­rung die Nach­rich­ten für
glaub­würdig. Das Ver­trauen in die Medien­unter­nehmen und in die
Jour­na­listen liegt mit 39\% bzw. 35\% sogar noch tiefer. Mit anderen
Worten: Rund zwei Drittel der Schweizer Be­völ­ke­rung ver­traut den
ei­ge­nen Jour­na­listen nicht mehr*.*

Dennoch glaubt das For­schungs­in­sti­tut -- das u.a. vom Bundes­amt für
Kom­mu­ni­ka­tion finanziert wird -- die Nutzung tra­di­tio­neller und
ins­b. öffent­licher Medien würde das Ver­trauen ins Medien­system
\href{http://www.foeg.uzh.ch/dam/jcr:7234c6d3-1f09-4d36-b6ab-f14e659d046e/Medienmitteilung_JB_2016_dt.pdf}{»för­dern«}.
Die Da­ten zei­gen je­doch nur, dass regel­mäßige Kon­su­menten die­ser
Me­dien we­ni­ger kri­tisch sind -- und ihre An­zahl immer ge­ringer
wird.

\emph{Update:} 2017
\href{http://www.digitalnewsreport.org/survey/2017/switzerland-2017/}{sank}
das Medienvertrauen auf 46\%. Die Werte bzgl. Journalisten und
Unter­neh­men wurden nicht mehr erhoben. Gemäß FÖG war das Vertrauen
\href{http://www.foeg.uzh.ch/dam/jcr:0d0e5a10-27be-4e97-b264-b2cf7de96bbd/Broschur_Jahrbuch_foeg_deutsch_2017_ohne_Sperrvermerk.pdf}{»weiterhin
hoch«}.

\begin{center}\rule{0.5\linewidth}{\linethickness}\end{center}

** 1. March 2017

**\href{https://swprs.org/category/allgemein/}{Allgemein}

**\href{https://swprs.org/tag/foeg/}{Foeg},
\href{https://swprs.org/tag/forschungsinstitut-fuer-oeffentlichkeit-und-gesellschaft/}{Forschungsinstitut
für Öffentlichkeit und Gesellschaft},
\href{https://swprs.org/tag/jahrbuch/}{Jahrbuch},
\href{https://swprs.org/tag/qualitaet-der-medien/}{Qualität der Medien},
\href{https://swprs.org/tag/schweizer-medien/}{Schweizer Medien},
\href{https://swprs.org/tag/universitaet-zuerich/}{Universität Zürich}

\hypertarget{post-navigation}{%
\subsection{Post navigation}\label{post-navigation}}

\href{https://swprs.org/2017/03/01/leserkommentare/}{Die Angst vor
den~Lesern}

\href{https://swprs.org/2017/03/01/der-schweizer-presserat/}{Der
Schweizer Presserat}

Comments are closed.

\hypertarget{swiss-policy-research}{%
\subsubsection{Swiss Policy Research}\label{swiss-policy-research}}

\begin{itemize}
\tightlist
\item
  \href{https://swprs.org/kontakt/}{Kontakt}
\item
  \href{https://swprs.org/uebersicht/}{Übersicht}
\item
  \href{https://swprs.org/donationen/}{Donationen}
\item
  \href{https://swprs.org/disclaimer/}{Disclaimer}
\end{itemize}

\hypertarget{english}{%
\subsubsection{English}\label{english}}

\begin{itemize}
\tightlist
\item
  \href{https://swprs.org/contact/}{About Us / Contact}
\item
  \href{https://swprs.org/media-navigator/}{The Media Navigator}
\item
  \href{https://swprs.org/the-american-empire-and-its-media/}{The CFR
  and the Media}
\item
  \href{https://swprs.org/donations/}{Donations}
\end{itemize}

\hypertarget{follow-by-email}{%
\subsubsection{Follow by email}\label{follow-by-email}}

Follow

\href{https://wordpress.com/?ref=footer_custom_com}{WordPress.com}.

\protect\hyperlink{}{Up ↑}

\includegraphics{https://pixel.wp.com/b.gif?v=noscript}
