\protect\hyperlink{content}{Skip to content}

\href{https://swprs.org/}{}

\protect\hyperlink{search-container}{Search}

Search for:

\href{https://swprs.org/}{\includegraphics{https://swprs.files.wordpress.com/2020/05/swiss-policy-research-logo-300.png}}

\href{https://swprs.org/}{Swiss Policy Research}

Geopolitics and Media

Menu

\begin{itemize}
\tightlist
\item
  \href{https://swprs.org}{Start}
\item
  \href{https://swprs.org/srf-propaganda-analyse/}{Studien}

  \begin{itemize}
  \tightlist
  \item
    \href{https://swprs.org/srf-propaganda-analyse/}{SRF / ZDF}
  \item
    \href{https://swprs.org/die-nzz-studie/}{NZZ-Studie}
  \item
    \href{https://swprs.org/der-propaganda-multiplikator/}{Agenturen}
  \item
    \href{https://swprs.org/die-propaganda-matrix/}{Medienmatrix}
  \end{itemize}
\item
  \href{https://swprs.org/medien-navigator/}{Analysen}

  \begin{itemize}
  \tightlist
  \item
    \href{https://swprs.org/medien-navigator/}{Navigator}
  \item
    \href{https://swprs.org/der-propaganda-schluessel/}{Techniken}
  \item
    \href{https://swprs.org/propaganda-in-der-wikipedia/}{Wikipedia}
  \item
    \href{https://swprs.org/logik-imperialer-kriege/}{Kriege}
  \end{itemize}
\item
  \href{https://swprs.org/netzwerk-medien-schweiz/}{Netzwerke}

  \begin{itemize}
  \tightlist
  \item
    \href{https://swprs.org/netzwerk-medien-schweiz/}{Schweiz}
  \item
    \href{https://swprs.org/netzwerk-medien-deutschland/}{Deutschland}
  \item
    \href{https://swprs.org/medien-in-oesterreich/}{Österreich}
  \item
    \href{https://swprs.org/das-american-empire-und-seine-medien/}{USA}
  \end{itemize}
\item
  \href{https://swprs.org/bericht-eines-journalisten/}{Fokus I}

  \begin{itemize}
  \tightlist
  \item
    \href{https://swprs.org/bericht-eines-journalisten/}{Journalistenbericht}
  \item
    \href{https://swprs.org/russische-propaganda/}{Russische Propaganda}
  \item
    \href{https://swprs.org/die-israel-lobby-fakten-und-mythen/}{Die
    »Israel-Lobby«}
  \item
    \href{https://swprs.org/geopolitik-und-paedokriminalitaet/}{Pädokriminalität}
  \end{itemize}
\item
  \href{https://swprs.org/migration-und-medien/}{Fokus II}

  \begin{itemize}
  \tightlist
  \item
    \href{https://swprs.org/covid-19-hinweis-ii/}{Coronavirus}
  \item
    \href{https://swprs.org/die-integrity-initiative/}{Integrity
    Initiative}
  \item
    \href{https://swprs.org/migration-und-medien/}{Migration \& Medien}
  \item
    \href{https://swprs.org/der-fall-magnitsky/}{Magnitsky Act}
  \end{itemize}
\item
  \href{https://swprs.org/kontakt/}{Projekt}

  \begin{itemize}
  \tightlist
  \item
    \href{https://swprs.org/kontakt/}{Kontakt}
  \item
    \href{https://swprs.org/uebersicht/}{Seitenübersicht}
  \item
    \href{https://swprs.org/medienspiegel/}{Medienspiegel}
  \item
    \href{https://swprs.org/donationen/}{Donationen}
  \end{itemize}
\item
  \href{https://swprs.org/contact/}{English}
\end{itemize}

\protect\hyperlink{}{Open Search}

\hypertarget{das-gewuxfcnschte-narrativ-ii}{%
\section{Das gewünschte
Narrativ~II}\label{das-gewuxfcnschte-narrativ-ii}}

\href{https://swprs.files.wordpress.com/2017/03/zeitungen-schweiz.png}{\includegraphics{https://swprs.files.wordpress.com/2017/03/zeitungen-schweiz.png?w=736}}

Im Dezember 2015 publi­zierte das News­portal \emph{Watson} (AZ Medien)
einen \href{https://www.watson.ch/!148360008}{Artikel}~des lang­jährigen
\emph{Tages­schau}-Korrespon­denten Helmut Scheben zum Syrien­krieg.
Scheben stellte den Krieg in einen geopolitischen Kontext und
kritisierte die westliche Be­richt­er­stattung als einseitig und
manipulativ.

Der Artikel un­ter­schied sich deutlich von anderen Aus­lands­bei­trägen
auf \emph{Watson}, die meist vom deutsch-transatlantischen
\href{https://www.watson.ch/Corporate/articles/502582965-Spiegel-Online-und-watson-machen-gemeinsame-Sache}{Ko­ope­ra­tions­par­tner}
\emph{Spiegel Online} geliefert werden.

Keine zwei Tage später ver­öffent­lichte \emph{Watson} jedoch einen
aufgebrachten \href{https://www.watson.ch/!491379853}{Rückruf}, in dem
sich das Portal vom Artikel distanzierte und Helmut Scheben wüst
beschimpfte: Man sei auf einen ``Putin-Troll'' herein­ge­fallen, der
wo­möglich in der ``russischen Propaganda-Maschinerie'' mit­wirke. Auch
Leser, die sich positiv zum ur­sprüng­lichen Artikel geäußert hatten,
wurden als Trolle verun­glimpft.

Wer oder was hat wohl hinter den Kulissen zu dieser selt­samen Reak­tion
geführt? Jeden­falls wurde den hiesigen Journa­listen damit einmal mehr
in Er­in­nerung gerufen: Wer sich in der Schweiz nicht an das
ge­wünschte Nar­ra­tiv hält, ris­kiert Ruf und Karriere.

\hypertarget{weitere-themen}{%
\paragraph{Weitere Themen}\label{weitere-themen}}

\begin{itemize}
\tightlist
\item
  \href{https://swprs.org/das-gewuenschte-narrativ/}{Das gewünschte
  Narrativ I}
\item
  \href{https://swprs.org/bericht-eines-journalisten/}{Bericht eines
  Journalisten}
\item
  \href{https://swprs.org/medien-navigator/}{Der Medien-Navigator}
\end{itemize}

\begin{center}\rule{0.5\linewidth}{\linethickness}\end{center}

Publiziert: Mai 2016

\hypertarget{swiss-policy-research}{%
\subsubsection{Swiss Policy Research}\label{swiss-policy-research}}

\begin{itemize}
\tightlist
\item
  \href{https://swprs.org/kontakt/}{Kontakt}
\item
  \href{https://swprs.org/uebersicht/}{Übersicht}
\item
  \href{https://swprs.org/donationen/}{Donationen}
\item
  \href{https://swprs.org/disclaimer/}{Disclaimer}
\end{itemize}

\hypertarget{english}{%
\subsubsection{English}\label{english}}

\begin{itemize}
\tightlist
\item
  \href{https://swprs.org/contact/}{About Us / Contact}
\item
  \href{https://swprs.org/media-navigator/}{The Media Navigator}
\item
  \href{https://swprs.org/the-american-empire-and-its-media/}{The CFR
  and the Media}
\item
  \href{https://swprs.org/donations/}{Donations}
\end{itemize}

\hypertarget{follow-by-email}{%
\subsubsection{Follow by email}\label{follow-by-email}}

Follow

\href{https://wordpress.com/?ref=footer_custom_com}{WordPress.com}.

\protect\hyperlink{}{Up ↑}

Post to

\protect\hyperlink{}{Cancel}

\includegraphics{https://pixel.wp.com/b.gif?v=noscript}
