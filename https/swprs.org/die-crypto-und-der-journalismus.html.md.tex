\protect\hyperlink{content}{Skip to content}

\href{https://swprs.org/}{}

\protect\hyperlink{search-container}{Search}

Search for:

\href{https://swprs.org/}{\includegraphics{https://swprs.files.wordpress.com/2020/05/swiss-policy-research-logo-300.png}}

\href{https://swprs.org/}{Swiss Policy Research}

Geopolitics and Media

Menu

\begin{itemize}
\tightlist
\item
  \href{https://swprs.org}{Start}
\item
  \href{https://swprs.org/srf-propaganda-analyse/}{Studien}

  \begin{itemize}
  \tightlist
  \item
    \href{https://swprs.org/srf-propaganda-analyse/}{SRF / ZDF}
  \item
    \href{https://swprs.org/die-nzz-studie/}{NZZ-Studie}
  \item
    \href{https://swprs.org/der-propaganda-multiplikator/}{Agenturen}
  \item
    \href{https://swprs.org/die-propaganda-matrix/}{Medienmatrix}
  \end{itemize}
\item
  \href{https://swprs.org/medien-navigator/}{Analysen}

  \begin{itemize}
  \tightlist
  \item
    \href{https://swprs.org/medien-navigator/}{Navigator}
  \item
    \href{https://swprs.org/der-propaganda-schluessel/}{Techniken}
  \item
    \href{https://swprs.org/propaganda-in-der-wikipedia/}{Wikipedia}
  \item
    \href{https://swprs.org/logik-imperialer-kriege/}{Kriege}
  \end{itemize}
\item
  \href{https://swprs.org/netzwerk-medien-schweiz/}{Netzwerke}

  \begin{itemize}
  \tightlist
  \item
    \href{https://swprs.org/netzwerk-medien-schweiz/}{Schweiz}
  \item
    \href{https://swprs.org/netzwerk-medien-deutschland/}{Deutschland}
  \item
    \href{https://swprs.org/medien-in-oesterreich/}{Österreich}
  \item
    \href{https://swprs.org/das-american-empire-und-seine-medien/}{USA}
  \end{itemize}
\item
  \href{https://swprs.org/bericht-eines-journalisten/}{Fokus I}

  \begin{itemize}
  \tightlist
  \item
    \href{https://swprs.org/bericht-eines-journalisten/}{Journalistenbericht}
  \item
    \href{https://swprs.org/russische-propaganda/}{Russische Propaganda}
  \item
    \href{https://swprs.org/die-israel-lobby-fakten-und-mythen/}{Die
    »Israel-Lobby«}
  \item
    \href{https://swprs.org/geopolitik-und-paedokriminalitaet/}{Pädokriminalität}
  \end{itemize}
\item
  \href{https://swprs.org/migration-und-medien/}{Fokus II}

  \begin{itemize}
  \tightlist
  \item
    \href{https://swprs.org/covid-19-hinweis-ii/}{Coronavirus}
  \item
    \href{https://swprs.org/die-integrity-initiative/}{Integrity
    Initiative}
  \item
    \href{https://swprs.org/migration-und-medien/}{Migration \& Medien}
  \item
    \href{https://swprs.org/der-fall-magnitsky/}{Magnitsky Act}
  \end{itemize}
\item
  \href{https://swprs.org/kontakt/}{Projekt}

  \begin{itemize}
  \tightlist
  \item
    \href{https://swprs.org/kontakt/}{Kontakt}
  \item
    \href{https://swprs.org/uebersicht/}{Seitenübersicht}
  \item
    \href{https://swprs.org/medienspiegel/}{Medienspiegel}
  \item
    \href{https://swprs.org/donationen/}{Donationen}
  \end{itemize}
\item
  \href{https://swprs.org/contact/}{English}
\end{itemize}

\protect\hyperlink{}{Open Search}

\hypertarget{die-crypto-ag-und-der-journalismus}{%
\section{Die Crypto AG und
der~Journalismus}\label{die-crypto-ag-und-der-journalismus}}

Zu den bemerkens­wertesten Aspekten der
\href{https://www.tagesanzeiger.ch/schweiz/standard/riesige-spionageoperation-der-cia-lief-ueber-die-schweiz/story/20957930}{Crypto-Affäre}
gehören gewisse Parallelen zum modernen Journalismus.

Denn die meisten Crypto-Mitarbeiter wussten tatsächlich nicht, dass sie
Teil einer globalen Geheim­dienst­operation waren. Manche wollten es
auch nicht wissen. Andere ahnten es zwar, aber schwiegen, aus Angst vor
Arbeits­platz­verlust -- oder Schlimmerem.

Nur an der Spitze wussten einige wenige Bescheid, und wurden ziemlich
reich damit.

Manipulierte Komponenten wurden -- ähnlich
\href{https://swprs.org/der-propaganda-multiplikator/}{Agenturmeldungen}
-- vorgefertigt angeliefert und durften nicht mehr verändert werden.
Einige arglose Mitarbeiter wollten die Algorithmen dennoch
selbst­ständig verbessern und mussten, subtil, davon abgehalten werden.

In einem Fall warnte die NSA sogar vor der Einstellung einer
Ingenieurin: diese sei
\href{https://www.washingtonpost.com/graphics/2020/world/national-security/cia-crypto-encryption-machines-espionage/img/pdfs/brightengineer.jpg}{»zu
intelligent«} und würde den Betrug durch­schauen. Was sie dann auch tat.

Doch gerade weil die meisten Mitarbeiter den Betrug nicht durchschauten
wirkte die Firma seriös und war die Operation während Jahrzehnten so
erfolgreich.

\hypertarget{crypto-leaks-cia-pressemitteilung-statt-investigativjournalismus}{%
\subparagraph{\texorpdfstring{\textbf{»Crypto-Leaks«:
CIA-Pressemitteilung statt
Investigativ­journalismus.}}{»Crypto-Leaks«: CIA-Pressemitteilung statt Investigativ­journalismus.}}\label{crypto-leaks-cia-pressemitteilung-statt-investigativjournalismus}}

Die Fakten zur Schweizer Crypto AG -- die von CIA und BND kontrolliert
wurde und Hinter­türen in ihre weltweit führenden
Ver­schlüs­se­lungs­­produkte einbaute -- sind schon seit 25 Jahren
bekannt, siehe etwa
\href{https://magazin.spiegel.de/EpubDelivery/spiegel/pdf/9088423}{Spiegel
1996}.

Der aktuelle
\href{https://www.tagesanzeiger.ch/schweiz/standard/riesige-spionageoperation-der-cia-lief-ueber-die-schweiz/story/20957930}{Wirbel}
beruht, wie so oft, nicht auf Investigativ­journalismus, sondern auf
einem »zugespielten« CIA-Bericht.

Von wem »zugespielt«? Von der CIA selbst na­tür­lich. Warum? Weil die
Operation inzwi­schen abgeschlossen ist, alle Spuren verwischt sind, und
nun kontrolliert publiziert werden kann.

Kenner sehen denn auch auf den ersten Blick, dass die »Crypto-Leaks« mit
Desinformation durchsetzt sind. Etwa wenn berichtet wird, dank Crypto
habe der »libysche Anschlag« auf die Berliner Disco La Belle von 1986
»aufgeklärt« werden können.

Tatsächlich ist seit über 20 Jahren bekannt, dass La Belle eine
israelische und ameri­ka­nische
\href{https://www.wsws.org/de/articles/1998/08/bell-a28.html}{Geheim­dienst­operation}
war, mit dem Ziel, Libyen bombardieren zu können. Die entschlüsselten
»libyschen Funksprüche« waren eine israelische
\href{http://ariwatch.com/OurAlly/Libya.htm}{Fälschung}.

»Crypto-Leaks«? Mit besten Grüßen von der CIA.

\begin{center}\rule{0.5\linewidth}{\linethickness}\end{center}

Feburar 2020

\hypertarget{swiss-policy-research}{%
\subsubsection{Swiss Policy Research}\label{swiss-policy-research}}

\begin{itemize}
\tightlist
\item
  \href{https://swprs.org/kontakt/}{Kontakt}
\item
  \href{https://swprs.org/uebersicht/}{Übersicht}
\item
  \href{https://swprs.org/donationen/}{Donationen}
\item
  \href{https://swprs.org/disclaimer/}{Disclaimer}
\end{itemize}

\hypertarget{english}{%
\subsubsection{English}\label{english}}

\begin{itemize}
\tightlist
\item
  \href{https://swprs.org/contact/}{About Us / Contact}
\item
  \href{https://swprs.org/media-navigator/}{The Media Navigator}
\item
  \href{https://swprs.org/the-american-empire-and-its-media/}{The CFR
  and the Media}
\item
  \href{https://swprs.org/donations/}{Donations}
\end{itemize}

\hypertarget{follow-by-email}{%
\subsubsection{Follow by email}\label{follow-by-email}}

Follow

\href{https://wordpress.com/?ref=footer_custom_com}{WordPress.com}.

\protect\hyperlink{}{Up ↑}

\includegraphics{https://pixel.wp.com/b.gif?v=noscript}
