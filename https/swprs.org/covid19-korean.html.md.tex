\protect\hyperlink{content}{Skip to content}

\href{https://swprs.org/}{}

\protect\hyperlink{search-container}{Search}

Search for:

\href{https://swprs.org/}{\includegraphics{https://swprs.files.wordpress.com/2020/05/swiss-policy-research-logo-300.png}}

\href{https://swprs.org/}{Swiss Policy Research}

Geopolitics and Media

Menu

\begin{itemize}
\tightlist
\item
  \href{https://swprs.org}{Start}
\item
  \href{https://swprs.org/srf-propaganda-analyse/}{Studien}

  \begin{itemize}
  \tightlist
  \item
    \href{https://swprs.org/srf-propaganda-analyse/}{SRF / ZDF}
  \item
    \href{https://swprs.org/die-nzz-studie/}{NZZ-Studie}
  \item
    \href{https://swprs.org/der-propaganda-multiplikator/}{Agenturen}
  \item
    \href{https://swprs.org/die-propaganda-matrix/}{Medienmatrix}
  \end{itemize}
\item
  \href{https://swprs.org/medien-navigator/}{Analysen}

  \begin{itemize}
  \tightlist
  \item
    \href{https://swprs.org/medien-navigator/}{Navigator}
  \item
    \href{https://swprs.org/der-propaganda-schluessel/}{Techniken}
  \item
    \href{https://swprs.org/propaganda-in-der-wikipedia/}{Wikipedia}
  \item
    \href{https://swprs.org/logik-imperialer-kriege/}{Kriege}
  \end{itemize}
\item
  \href{https://swprs.org/netzwerk-medien-schweiz/}{Netzwerke}

  \begin{itemize}
  \tightlist
  \item
    \href{https://swprs.org/netzwerk-medien-schweiz/}{Schweiz}
  \item
    \href{https://swprs.org/netzwerk-medien-deutschland/}{Deutschland}
  \item
    \href{https://swprs.org/medien-in-oesterreich/}{Österreich}
  \item
    \href{https://swprs.org/das-american-empire-und-seine-medien/}{USA}
  \end{itemize}
\item
  \href{https://swprs.org/bericht-eines-journalisten/}{Fokus I}

  \begin{itemize}
  \tightlist
  \item
    \href{https://swprs.org/bericht-eines-journalisten/}{Journalistenbericht}
  \item
    \href{https://swprs.org/russische-propaganda/}{Russische Propaganda}
  \item
    \href{https://swprs.org/die-israel-lobby-fakten-und-mythen/}{Die
    »Israel-Lobby«}
  \item
    \href{https://swprs.org/geopolitik-und-paedokriminalitaet/}{Pädokriminalität}
  \end{itemize}
\item
  \href{https://swprs.org/migration-und-medien/}{Fokus II}

  \begin{itemize}
  \tightlist
  \item
    \href{https://swprs.org/covid-19-hinweis-ii/}{Coronavirus}
  \item
    \href{https://swprs.org/die-integrity-initiative/}{Integrity
    Initiative}
  \item
    \href{https://swprs.org/migration-und-medien/}{Migration \& Medien}
  \item
    \href{https://swprs.org/der-fall-magnitsky/}{Magnitsky Act}
  \end{itemize}
\item
  \href{https://swprs.org/kontakt/}{Projekt}

  \begin{itemize}
  \tightlist
  \item
    \href{https://swprs.org/kontakt/}{Kontakt}
  \item
    \href{https://swprs.org/uebersicht/}{Seitenübersicht}
  \item
    \href{https://swprs.org/medienspiegel/}{Medienspiegel}
  \item
    \href{https://swprs.org/donationen/}{Donationen}
  \end{itemize}
\item
  \href{https://swprs.org/contact/}{English}
\end{itemize}

\protect\hyperlink{}{Open Search}

\hypertarget{uxcf54uxb85cuxb09819-uxd329uxd2b8}{%
\section{코로나19 팩트}\label{uxcf54uxb85cuxb09819-uxd329uxd2b8}}

\textbf{업데이트}: 2020 년 5 월 18 일 ; \textbf{쉐어}:
\href{https://twitter.com/intent/tweet?url=https://swprs.org/covid19-korean/}{Twitter}
/
\href{https://www.facebook.com/share.php?u=https://swprs.org/covid19-korean/}{Facebook}\\
\textbf{언어}: \href{https://swprs.org/fakta-o-covid-19/}{CZ},
\href{https://swprs.org/covid-19-hinweis-ii/}{DE},
\href{https://swprs.org/a-swiss-doctor-on-covid-19/}{EN},
\href{https://swprs.org/hechos-sobre-covid-19/}{ES},
\href{https://swprs.org/faktoja-covid-19sta/}{FI},
\href{https://swprs.org/coronavirus-un-medecin-suisse-parle/}{FR},
\href{https://swprs.org/facts-about-covid19-greek/}{GR},
\href{https://swprs.org/covid-19-cinjenice/}{HBS},
\href{https://yanivhamo.com/facts-about-covid-19-hebrew/}{HE},
\href{https://swprs.org/egy-svajci-orvos-a-covid-19-rol/}{HU},
\href{https://swprs.org/un-medico-svizzero-su-covid-19/}{IT},
\href{https://swprs.org/covid19-facts-japanese/}{JP},
\href{https://swprs.org/covid19-korean/}{KO},
\href{https://www.globalinfo.nl/Achtergrond/een-kritische-kijk-op-het-coronabeleid-transparantie-in-tijden-van-crisis}{NL},
\href{https://midtifleisen.wordpress.com/2020/03/14/en-sveitsisk-lege-om-covid-19/}{NO},
\href{https://swprs.org/szwajcarski-lekarz-o-covid-19/}{PL},
\href{https://swprs.org/fatos-sobre-covid-19/}{PT},
\href{https://swprs.org/informatii-despre-covid-19/}{RO},
\href{https://swprs.org/\%d0\%bd\%d0\%b0-\%d0\%ba\%d0\%be\%d0\%b2\%d0\%b8\%d0\%b4-19/}{RU},
\href{https://swprs.org/fakta-om-covid-19/}{SE},
\href{http://www.ninamvseeno.org/pregled-clanka.aspx?naslov=pomembne-informacije-o-novem-koronavirusu-covid-19\&id=148}{SI},
\href{https://alatyr.sk/covid-19_swiss_propaganda_research.htm}{SK},
\href{https://swprs.org/isvicreli-bir-doktordan-kovid-19-uezerine/}{TR}

분야별 전문가들이 제공하는, 검증과 확인이 가능한 코로나19에 관한
객관적인 정보들입니다. 독자들로 하여금 현실적인 위험 평가를 내릴 수
있도록 돕는 것이 목적입니다.

\textbf{„페스트 싸울 수 있는 유일한 방법은 정직뿐이다.`` 알베르 카뮈,
페스트 (1947)}

\hypertarget{uxac1cuxc694}{%
\paragraph{개요}\label{uxac1cuxc694}}

\begin{enumerate}
\def\labelenumi{\arabic{enumi}.}
\tightlist
\item
  연구가 잘 된 국가와 지역의 데이터를 종합해 볼 때, 코로나19의 치명률은
  \href{https://swprs.org/studies-on-covid-19-lethality/}{0.1\% 에서
  0.36\% 사이}에 머물고 있으며, 이는
  심각한~\href{https://www.ebm-netzwerk.de/en/publications/covid-19}{인플루엔자}~(독감)와
  비슷한 수준으로 최초
  WHO~\href{https://www.businessinsider.com/coronavirus-death-rate-by-age-countries-2020-3}{예상}의
  20분의 1 수준에 그치고 있다.
\item
  글로벌 ``핫스팟''으로 분류되는 위험지역에서도 학생과 근로자 연령대
  사망률은
  \href{https://www.medrxiv.org/content/10.1101/2020.04.05.20054361v1}{일일
  자동차 사고} 위험과 비슷한 수준이다. 감염자 중 대다수가 경미한 증상
  또는 무증상이라는 것을 감안하지 못해 초기에 위험이 부풀려졌다. 
\item
  양성 확진자들 중 최대 80\%까지
  \href{https://www.bmj.com/content/369/bmj.m1375}{무증상}이다. 70대
  (70\textasciitilde{}79세) 노인 연령대에서도
  \href{https://www.niid.go.jp/niid/en/2019-ncov-e/9407-covid-dp-fe-01.html}{약
  60\%}가 무증상이며, 감염자의 95\%
  이상이\href{https://swprs.org/studies-on-covid-19-lethality/\#hospitalizations}{경미한
  증상}에 그친다.
\item
  전체 인구의 3분의 1 정도가 과거에 있었던 (일반 감기 바이러스 등과
  같은) 코로나바이러스와의 접촉으로인해 이미 코로나19에
  대한\href{https://www.medrxiv.org/content/10.1101/2020.04.17.20061440v1}{사전
  면역}을 획득했을 가능성이 있다.
\item
  \href{https://www.epicentro.iss.it/coronavirus/sars-cov-2-decessi-italia}{이탈리아}를
  포함한 대부분의 국가에서 사망자들의 중간 연령은 80세 이상이며, 심각한
  기저질환이 없었던 건강했던 환자들의 비율은
  \href{https://www.bloomberg.com/news/articles/2020-03-18/99-of-those-who-died-from-virus-had-other-illness-italy-says}{약
  1\%}에 불과하다. 따라서 사망자의 연령 및 위험 프로파일은
  \href{https://www.vienna.at/analyse-zeigt-covid-19-opferkurve-entspricht-normaler-mortalitaet/6581246}{평상시
  사망률}과 일치한다.
\item
  대부분의 국가에서 평년 대비 추가 사망자의 50\%에서
  70\%는~\href{https://ltccovid.org/2020/04/12/mortality-associated-with-covid-19-outbreaks-in-care-homes-early-international-evidence/}{양로병원}에서
  발생했으며 국가 전체 봉쇄령의 효과가 없음을 시사한다. 한 발 더 나아가,
  이들의 사망원인이 코로나19로 인한 사망인지, 공포와 외로움으로 인한
  \href{https://www.nytimes.com/2020/04/16/world/canada/montreal-nursing-homes-coronavirus.html}{극도의
  스트레스}인지 명확하게 밝혀지지 않은
  \href{https://www.hsj.co.uk/commissioning/thousands-of-extra-deaths-outside-hospital-not-attributed-to-covid-19/7027459.article}{상황}이다.
\item
  평년 대비 추가 사망자의 최대 50\%까지는
  \href{https://www.thetimes.co.uk/edition/news/coronavirus-record-weekly-death-toll-as-fearful-patients-avoid-hospitals-bm73s2tw3}{코로나19가
  원인이 아닌},
  \href{https://www.telegraph.co.uk/global-health/science-and-disease/two-new-waves-deaths-break-nhs-new-analysis-warns/}{봉쇄령으로
  인한 공포와 두려움}에 의한 사망일 수 있다. 예를 들어 환자들이 병원
  방문을 꺼려 심장마비와 뇌졸중의 치료가 60\%까지
  \href{https://www.nytimes.com/2020/04/06/well/live/coronavirus-doctors-hospitals-emergency-care-heart-attack-stroke.html}{감소}했다.
\item
  많은 경우 ``코로나19 사망자''의 사망원인이 코로나바이러스에 의한
  사망인지, 코로나바이러스와 함께 (기저질환으로 인한) 사망인지, 또는
  진단키트 검사 없이 코로나바이러스일 것으로
  \href{https://www.youtube.com/watch?v=V0lIWZpiRU0}{``가정''}한 것인지
  에 대한
  구분이~\href{https://spectator.us/understand-report-figures-covid-deaths/}{명확하지
  않다.} 대부분의 공식집계는 이들 차이를
  \href{https://www.hsj.co.uk/coronavirus/systematic-reviews-to-discover-true-cause-of-outbreak-deaths/7027491.article}{반영하지
  않고있다}.
\item
  젊고 건강한 사람도 코로나19로 사망할 수 있다는 다수 언론의 보도는
  면밀한 조사 결과 사실이 아닌 것으로 판명되었다. 젊은 사망자의 대다수는
  코로나19로
  인해\href{https://www.dailymail.co.uk/news/article-8193487/Coroner-refuses-rule-COVID-19-cause-death-six-week-old-Connecticut-baby.html}{사망한
  것이 아니며}, 대부분의 경우 (모르고 있었던 백혈병과 같은)
  \href{https://sports.yahoo.com/spanish-football-coach-francisco-garcia-163153573.html}{심각한
  다른 질환}이 있었던 것으로 확인되었다.
  또는~\href{https://www.tagesanzeiger.ch/bund-muss-in-seiner-todesfallstatistik-fehler-korrigieren-584308129723}{109세를
  9세로 }연령을 잘 못 표기한 경우도 있었다.
\item
  미국의 정상적인 일 평균 사망자는 약
  \href{https://www.cdc.gov/mmwr/volumes/68/wr/mm6826a5.htm}{8,000명}이며,
  독일은 약 2,600명, 이탈리아는 약 1,800명 정도가 매일 사망한다.
  겨울동안 인플루엔자 독감 사망자는 미국에서
  \href{https://www.statnews.com/2018/09/26/cdc-us-flu-deaths-winter/}{최대
  80,000명}, 독일과
  이탈리아는~\href{https://www.sciencedirect.com/science/article/pii/S1201971219303285}{각각
  25,000}명까지 발생한다. 다 수의 국가에서 코로나19 사망자 수는 심각한
  독감 시즌 \href{https://www.euromomo.eu/graphs-and-maps/}{이하 수준에
  머무르고 있다}.
\item
  특정 지역의 사망률이 급격하게 증가한
  원인으로는\href{https://www.theguardian.com/environment/2020/apr/20/air-pollution-may-be-key-contributor-to-covid-19-deaths-study?utm_medium}{극심한
  공해},
  \href{https://www.ansa.it/english/news/science_tecnology/2019/11/19/italy-top-in-eu-in-antibiotic-resistance_369e0123-0107-445e-8c17-f11932c9d27c.html}{미생물
  감염}, ~\href{https://swprs.org/covid-19-a-report-from-italy/}{노인층
  감염병 질환자 의료 지원의 붕궤}, 집단 공포와 봉쇄령 등의 추가적인 위험
  요소들이 영향을 끼쳤다. 까다롭게 신설된 사망자의 신변처리
  절차~\href{https://www.ecdc.europa.eu/sites/default/files/documents/COVID-19-safe-handling-of-bodies-or-persons-dying-from-COVID19.pdf}{규정}으로
  인해 장례식과 화장터에 병목현상이 초래되었다. 
\item
  이탈리아와 스페인, 영국과 미국 일부 지역에서 독감으로 인해 병원에
  과부하가 걸리는 것은
  \href{https://off-guardian.org/2020/04/02/coronavirus-fact-check-1-flu-doesnt-overwhelm-our-hospitals/}{흔히
  발생하는 현상}이다. 현재, 최대 15\%에 달하는 의사와 간호사들이, 증상이
  없는데도 불구하고,
  \href{https://www.reuters.com/article/us-health-coronavirus-spain-morgue-idUSKBN21B1PP}{자가격리
  조치}를 취하고 있다. 
\item
  기하급수적으로 증가하는 ``코로나19 확진자''
  그래프는\href{https://fivethirtyeight.com/features/coronavirus-case-counts-are-meaningless/}{오해의
  소지}가 있다. 진단 검사가 기하급수적으로 증가하는 것이 확진자 증가의
  원인이기 때문이다. 대부분의 국가에서 진단 검사 수 대비 확진자 수
  비율은5\% 에서 25\% 사이에 머무르거나 느린 폭의 증가에 그친. 대부분의
  국가들에서
  이미~\href{https://www.dailymail.co.uk/news/article-8235979/UKs-coronavirus-crisis-peaked-lockdown-Expert-argues-draconian-measures-unnecessary.html}{봉쇄령
  실시 이전}에 감염 확산이 정점을 찍었다. 
\item
  \href{https://www.japantimes.co.jp/news/2020/03/20/national/coronavirus-explosion-expected-japan/}{일본},~\href{https://www.businessinsider.com/south-korea-coronavirus-testing-death-rate-2020-3?op=1}{대한민국},~\href{https://www.youtube.com/watch?v=bfN2JWifLCY}{스웨덴}과같이
  자택대피령 과 사람간 접촉금지령을 실시하지 않은 국가들이라고 해서 다른
  국가들에 비해 대규모 피해가
  \href{https://www.washingtontimes.com/news/2020/apr/15/sweden-coronavirus-rates-easing-despite-loose-rule/}{발생하지는
  않았다}.~ WHO에서도 스웨덴의
  사례를~\href{https://nypost.com/2020/04/29/who-lauds-sweden-as-model-for-resisting-coronavirus-lockdown/}{칭찬}했으며,
  이로서 스웨덴은 봉쇄령을 실시한 다른 국가들에 비해 더 강한 면역력을
  갖출 수 있게 되었다.
\item
  산소마스크가 부족하다는
  공포는~\href{https://apnews.com/8ccd325c2be9bf454c2128dcb7bd616d}{사실이
  아니다}. 저명한 폐질환 전문의들의 주장에 따르면, 초기 침습식
  인공호흡기 치료는 바이러스 감염에대한
  \href{https://www.dailymail.co.uk/news/article-8262351/Nurse-New-York-claims-city-killing-COVID-19-patients-putting-ventilators.html}{우려에서
  비롯되었으나}, 많은 경우
  \href{https://www.medscape.com/viewarticle/928156}{역효과}를나타낼 수
  있으며 추가적인 폐 손상을 유발하기도 한다. 
\item
  최초 가정과 달리, 다 수의 연구들이 코로나바이러스가 비말 (예: 공기중
  떠 다니는 입자) 또는
  \href{https://www.telegraph.co.uk/news/2020/04/02/no-proof-coronavirus-can-spread-shopping-says-leading-german/}{도포감염}~(예:
  손잡이, 스마트폰, 미용실)을 통해
  전파된다는~\href{https://www.who.int/news-room/commentaries/detail/modes-of-transmission-of-virus-causing-covid-19-implications-for-ipc-precaution-recommendations}{증거가
  없다}고 밝히고 있다.~
\item
  건강한 무증상자의 경우 마스크 착용의 실효성을 증명하는
  \href{https://www.researchgate.net/publication/340570735_Masks_Don't_Work_A_review_of_science_relevant_to_COVID-19_social_policy}{과학적인
  근거는 없다}. 오히려 반대로, 마스크를 착용하는 것이 정상적인 호흡을
  방해해 마스크 사용자들이
  \href{https://de.sputniknews.com/interviews/20200425326953541-corona-gefahr-virologe/}{``세균을
  옮기는 매개체''}가 될 수 있다고 전문가들은 경고한다. ~저명한 의사들도
  마스크 착용은 ``언론의 과장''이자
  \href{https://www.aerztezeitung.de/Politik/Montgomery-haelt-Maskenpflicht-fuer-falsch-408844.html}{``코미디''}에
  불과하다고 일침을 놓았다.
\item
  유럽과 미국의 대다수 병원들은 코로나19가 확산이 정점을 찍을 무렵
  \href{https://www.hsj.co.uk/acute-care/nhs-hospitals-have-four-times-more-empty-beds-than-normal/7027392.article}{심각하게
  환자가 감소}하거나 병동이 거의 비어있었으며
  의료진들의~\href{https://www.usatoday.com/story/news/health/2020/04/02/coronavirus-pandemic-jobs-us-health-care-workers-furloughed-laid-off/5102320002/}{근무시간을
  축소}한 바 있다. 장기이식이나 암검진과 같은 다수의 수술과 진료들이
  \href{https://www.sfchronicle.com/bayarea/article/Stanford-hospital-system-to-cut-pay-20-furlough-15227591.php}{취소되었다}.
\item
  다수의 언론매체들이 사진과 영상 등을 조작해 병원들의 상황을 보다
  자극적으로 보도하려고 시도하다
  \href{https://nypost.com/2020/04/01/cbs-admits-to-using-footage-from-italy-in-report-about-nyc/}{적발}되었다.
  다수
  언론사의~\href{https://onlinelibrary.wiley.com/doi/full/10.1111/eci.13222}{프로페셔널하지
  못한 보도행태}로 인해 대중들의 공포와 불안감만 확산되었다. 
\item
  전세계적으로 사용되는 바이러스 진단 키트의
  \href{https://www.ncbi.nlm.nih.gov/pubmed/32219885}{에러 발생률}이
  높아 위양성과 위음성 결과 빈도가 높다. 한 발 더 나아가, 현재 상용되고
  있는 바이러스 진단 키트는 시간이 촉박하다는 이유로
  \href{https://www.youtube.com/watch?v=p_AyuhbnPOI}{임상적인 검증을
  거치지 않은 상태}이다.
\item
  바이러스학, 면역학, 전염병 역학 분야의 세계적인 석학과
  \href{https://off-guardian.org/2020/03/24/12-experts-questioning-the-coronavirus-panic/}{전문가들}은
  지금의 조치들이
  \href{https://off-guardian.org/2020/03/28/10-more-experts-criticising-the-coronavirus-panic/}{역효과}를
  나타낼 수 있다고 경고하며, 고위험군을 보호하되 일반 대중들의 경우는
  \href{https://off-guardian.org/2020/04/08/watch-perspectives-on-the-pandemic-2/}{자연면역}을
  빠르게 증가하는 조치를 취할 것을 조언했다. 어린이들에게서 위험성은
  \href{https://www.thelancet.com/journals/lanchi/article/PIIS2352-4642(20)30095-X/fulltext}{제로에
  가까우며}~학교를 폐쇄하는 것은 의학적으로 검증된 바 없다.
\item
  일부 의사들은 코로나바이러스 백신은
  \href{https://www.youtube.com/watch?v=vrL9QKGQrWk}{불필요}하며심지어
  \href{https://www.nature.com/articles/d41586-020-00751-9}{위험할 수
  있다}고 경고했다. 실제로, 2009년
  \href{https://www.forbes.com/2010/02/05/world-health-organization-swine-flu-pandemic-opinions-contributors-michael-fumento.html\#658c006c48e8}{신종플루
  (돼지독감)} 백신의 경우
  \href{https://www.ibtimes.co.uk/brain-damaged-uk-victims-swine-flu-vaccine-get-60-million-compensation-1438572}{심각한
  신경장애}를 유발해 수 백만 달러에 달하는 소송으로 이어진 바 있다.
\item
  전 세계 각국에서는 강도높은 봉쇄 조치로 인해 실업, 우울증, 가정폭력
  등으로 고통받는 피해자들이
  \href{https://www.reuters.com/article/us-health-coronavirus-usa-layoffs/us-weekly-jobless-claims-seen-at-record-high-again-idUSKBN21K0FX}{폭발적으로
  증가}하고 있다. 다수의 전문가들은 바이러스로 인한 사망자보다
  봉쇄조치로 인한
  \href{https://www.nytimes.com/2020/03/20/opinion/coronavirus-pandemic-social-distancing.html}{사망자가
  더 많을 것}으로 관측하고 있다. UN의 보고에 따르면, 전
  세계적으로~\href{https://www.theguardian.com/global-development/2020/apr/21/coronavirus-pandemic-will-cause-famine-of-biblical-proportions}{수
  백만 명}이 절대 빈곤층으로 전락할 전망이다.
\item
  미국 국가안전보장국 NSA의 내부고발자 에드워드 스노든 (Edward
  Snowden)은 전 세계적으로 대규모의 영구적인 감시체계를 구축하는데
  코로나바이러스 위기가 악용되고 있다고
  \href{https://www.youtube.com/watch?v=-pcQFTzck_c}{경고}했다. 저명한
  바이러스 학자 파블로 골드슈미트 (Pablo Goldschmidt)는 ``글로벌 미디어
  테러''와 ``전체주의적 접근''에 대해
  \href{https://www.rubikon.news/artikel/der-corona-totalitarismus}{경고}했고,
  영국의 권위있는 바이러스학자 존 옥스포드 (John Oxford) 교수는 ``미디어
  감염''을
  \href{https://novuscomms.com/2020/03/31/a-view-from-the-hvivo-open-orphan-orph-laboratory-professor-john-oxford/}{경고}한
  바 있다.~ 
\item
  500명 이상의 과학자들이 현재 논란이 되고 있는 ``접촉(이동)경로''
  어플을 통한 ``전례없는 감시통제 사회''에
  대해~\href{https://www.esat.kuleuven.be/cosic/sites/contact-tracing-joint-statement/}{경고했다}.
  일부 국가에서는 이미
  \href{https://www.jewishpress.com/news/the-courts/state-to-high-court-even-more-shin-bet-involvement-in-fighting-the-coronavirus/2020/04/14/}{국가정보원}에
  의해 ``접촉경로''를 감시하고 있으며,
  \href{https://off-guardian.org/2020/04/25/50-headlines-darker-more-of-the-new-normal/}{드론을
  이용한 감시}를 통해 공권력의 남용이 심각해지는 상황이다. 
\end{enumerate}

\hypertarget{uxcd94uxcc9c}{%
\subparagraph{\texorpdfstring{\textbf{추천}:}{추천:}}\label{uxcd94uxcc9c}}

\begin{itemize}
\tightlist
\item
  \href{https://swprs.org/open-letter-from-professor-sucharit-bhakdi-to-german-chancellor-dr-angela-merkel/}{수카리
  박디 (Sucharit Bhakdi) 교수의 공개서한}
\item
  \href{https://www.euromomo.eu/}{유럽연합 사망률 모니터링 통계
  (EuroMomo)}
\item
  \href{https://swprs.org/corona-media-propaganda/}{코로나,
  언론,}\href{https://swprs.org/corona-media-propaganda/}{그리고
  프로파간다}
\end{itemize}

\begin{center}\rule{0.5\linewidth}{\linethickness}\end{center}

\hypertarget{uxc5c5uxb370uxc774uxd2b8uxb294-uxb2e4uxc74cuxacfc-uxac19uxc2b5uxb2c8uxb2e4-uxc601uxc5b4uxb85c-uxb354}{%
\paragraph{\texorpdfstring{업데이트는 다음과 같습니다.
\href{https://swprs.org/a-swiss-doctor-on-covid-19/}{영어로
더}.}{업데이트는 다음과 같습니다. 영어로 더.}}\label{uxc5c5uxb370uxc774uxd2b8uxb294-uxb2e4uxc74cuxacfc-uxac19uxc2b5uxb2c8uxb2e4-uxc601uxc5b4uxb85c-uxb354}}

\hypertarget{2020uxb144-4uxc6d4-21uxc77c}{%
\paragraph{2020년 4월 21일}\label{2020uxb144-4uxc6d4-21uxc77c}}

\hypertarget{uxba54uxb514uxceec-uxc5c5uxb370uxc774uxd2b8}{%
\subparagraph{\texorpdfstring{\textbf{메디컬
업데이트}}{메디컬 업데이트}}\label{uxba54uxb514uxceec-uxc5c5uxb370uxc774uxd2b8}}

\begin{itemize}
\tightlist
\item
  ~캘리포니아 산타 클라라 카운티 인구를 대상으로 한
  \href{https://www.medrxiv.org/content/10.1101/2020.04.14.20062463v1}{스탠포드
  의과대학의 혈청학 연구 결과} 기존에 알려진 것 보다 50배에서 85배 많은
  인구가 코로나바이러스 항체를 보유하고 있는 것으로 확인됐다. 이는
  코로나19의 치명률이 0.12\%에서 0.2\% 수준 (심각한 독감 수준)으로 낮을
  수 있다는 것을 의미한다. 연구를 주도한 존 론니디스 교수는
  \href{https://www.youtube.com/watch?v=jGUgrEfSgaU}{영상 인터뷰}를 통해
  연구 내용을 자세하게 설명하였다. 
\item
  \href{http://publichealth.lacounty.gov/phcommon/public/media/mediapubhpdetail.cfm?prid=2328}{로스앤젤레스
  카운티의 혈청학 연구}에서도 기존 예상보다 28배에서 55배 많은 인구가
  (별다른 증상 없이) 코로나19에 감염 된 것으로 확인되었다. 그만큼 질병의
  위험성이 낮아짐을 의미한다.
\item
  핀란드 헬싱키 대학의 역학 교수 미코 파우니오는 다양한 국가들의
  연구결과를 토대로 코로나19의 치명률 (IFR) 은 0.1\% 이하에 그친다고
  \href{https://lockdownsceptics.org/wp-content/uploads/2020/04/How-the-World-got-Fooled-by-COVID-ed-2c.pdf}{발표}했다.
  파이니오 교수에 따르면 치명률이 높아 보이는 이유는 이탈리아나 스페인과
  같은 다가구 지역과 뉴욕과 같은 인구밀집 지역에서 바이러스의 전파속도가
  너무 빠르기 때문이라는 것. 그는 정부의 `락다운(봉쇄령)' 조치가 효과를
  보기에는 너무 늦게 발동되었다고 주장했다. (* 대한민국은
  사회적거리두기만 실시하고 사회봉쇄령을 실시한 적이 없음) 
\item
  옥스포드 대학의 근거중심의학센터 (CEBM)는 코로나19의 치명률이 0.1\%
  에서 0.36\%에 불과하다고
  \href{https://www.cebm.net/covid-19/global-covid-19-case-fatality-rates/}{발표}했다.
  70세 이상 중증 증상이 없는 경우 치명률은 1\% 이하이며, 80세 이상
  연령대의 치명률은 3\%에서 15\% 사이이다. 코로나바이러스가 원인이 되어
  사망한 경우와 코로나바이러스와 함께 사망한 경우를 구분지어야 한다고
  강조했다. 
\item
  옥스포드 대학 CEBM의 센터장을 맡고 있는 칼 헤니건 교수는
  \href{https://news.yahoo.com/lockdown-damage-outweighs-coronavirus-warning-121940675.html}{언론과의
  인터뷰}를 통해 사회봉쇄령 (lockdown)으로 인해 발생하는 피해가
  바이러스로 인한 피해보다 더 크다고 경고했다. 대부분의 국가에서 이미
  사회봉쇄령을 실시하기 이전에 감염 확산이 정점을 찍었다는 사실을
  강조했다.
\item
  독감과 대조적으로 어린이들의 사망률은 0에 가깝다. 이탈리아 북부지역의
  사망자가 증가한 것과 관련해, 유럽에서 이탈리아가
  \href{https://www.ansa.it/english/news/science_tecnology/2019/11/19/italy-top-in-eu-in-antibiotic-resistance_369e0123-0107-445e-8c17-f11932c9d27c.html}{항생제
  내성이 가장 높은 것}도 하나의 원인일 것으로 추측되고 있다. 이탈리아
  보건당국이 발표한 데이터에 따르면 사망자의 80\%가 항생제 치료를 받아
  박테리아 감염이 존재했음을 알 수 있다.
\item
  보스톤 인근 첼시 지역에서는,
  \href{https://archive.is/20200418222442/https:/www.bostonglobe.com/2020/04/17/business/nearly-third-200-blood-samples-taken-chelsea-show-exposure-coronavirus/}{200명의
  헌혈자 가운데 3분의 1이} 코로나19 병원체에 대한 항체를 보유하고 있는
  것으로 나타났다. 이 중 절반은 지난 겨울 감기 증상이 있었다고 밝혔다.
  보스톤 인근 노숙사 보호소에서도 3분의 1이 양성 확진 판정을 받았으나
  \href{https://www.wsbtv.com/news/trending/coronavirus-cdc-reviewing-stunning-universal-testing-results-boston-homeless-shelter/ZADQ45HCAZEVJAZA3OTCUR7M6M/}{단
  한 명도 증상을 보인이는 없었다}. 
\item
  프랑스
  \href{https://www.ouest-france.fr/sante/virus/coronavirus/coronavirus-au-moins-940-marins-positifs-sur-le-charles-de-gaulle-et-son-escorte-6810816}{항공모함
  승조원 1,081명이 코로나19 양성 확진}을 받았다. 이 중 약 50\%는 증상이
  없으며, 약 50\%는 경미한 증상을 나타냈다. 24명이 입원 치료를 받았으며,
  1명은 중환자실에 입원한 상태다. 중환자실에 입원한 환자의 기저질환
  여부는 알려지지 않은 상태다.
\item
  캐나다의
  \href{https://www.nytimes.com/2020/04/16/world/canada/montreal-nursing-homes-coronavirus.html}{양로병원에서
  31명의 환자가 사망}했다. 모든 간호사들이 코로나 바이러스 감염을
  두려워하여 병원에서 대피한 것이 원인. 보건당국이 병원에 남겨진
  노인들을 발견했을 때, 대부분의 생존자들은 심각한 영양실조와 탈수
  증세를 보였다.
  \href{https://swprs.org/covid-19-a-report-from-italy/}{북부
  이탈리아에서도 이와 유사}한 비극이 보고된 바 있다. 
\item
  스코틀랜드의 양로병원에서 진료하는 한 의사는 ``양로병원들에 대한
  정부의 정책이 무었인가? 지금과 같은 방식은 상황을 더욱 악화시키고
  있다.''고
  \href{https://drmalcolmkendrick.org/2020/04/17/care-homes-and-covid19/}{질타}했다.
  ~~
\item
  스코틀랜드는
  \href{https://www.heraldscotland.com/news/18377095.coronavirus-scotland-half-icu-beds-empty/}{중환자실의
  절반이 비어있다}고 보고했다. 보건당국 관계자는 신규 입원 환자가 더이상
  증가하지 않는다고 밝혔다. 
\item
  영국 런던의 나이팅게일 병원의 임시 진료소
  \href{https://www.hsj.co.uk/service-design/exclusive-nightingale-largely-empty-as-icus-handle-surge/7027398.article}{대부분이
  공실}이다. 런던시는 코로나19 환자 치료를 위해 중환자실을 두 배로
  늘렸는데 아직까지는 응급환자들 진료를 잘 소화하고 있다고 밝혔다. ~~
\item
  4월 14일 보고에 따르면 스위스의 병원들도 중환자실을 포함하여 대부분의
  병실에
  \href{https://swprs.files.wordpress.com/2020/04/intensivbettenbelegung-schweiz-2020-04-14.png}{환자가
  없는 것}으로 보고 되었다. 확진자 ( 평균 연령 84세) 대부분이 일반
  병원이 아닌 양로병원일 것이라는 가정을 뒷받침한다. ~~
\item
  이탈리아 베르가못 보건소의 응급실은45일만에 처음으로, 이번 주 초부터
  \href{https://orf.at/stories/3162642/}{완전 비어있으며}, 다시 코로나19
  환자보다 더 많은 수의 타 질환 환자들이 치료를 받기 시작했다고 전했다. 
\item
  의학 전문지 랜싯 Lancet 에 실린 보고서에 따르면 코로나 바이러스 확산을
  막기 위해 학교를 폐쇄하는 것은
  \href{https://www.thelancet.com/journals/lanchi/article/PIIS2352-4642(20)30095-X/fulltext}{실효성이
  없거나 미미한 효과만 있을 뿐}이라는 결론에 도달했다. 프랑스에서 코로나
  바이러스에 감염된 9세 소년이 172명과 접촉했으나
  \href{https://www.n-tv.de/panorama/172-Kontaktpersonen-von-Corona-verschont-article21727469.html}{단
  한 명도 감염되지 않았다}. 이는 (인플루엔자 독감과 달리) 코로나
  바이러스 감염은 어린이를 통해 전파되지 않는 다는 기존의 연구결과들을
  확인하는 사례다. 
\item
  \href{https://www.rubikon.news/artikel/120-expertenstimmen-zu-corona}{코로나19에
  관한 120명 전문가들의 견해}. 현재의 코로나19 대응 방식에 관한 세계적인
  과학자, 의사, 변호사 및 기타 분야 전문가들의 비판. (독일어) 
\item
  \href{https://off-guardian.org/2020/03/24/12-experts-questioning-the-coronavirus-panic/}{코로나19
  패닉 현상에 질문을 던지는 12명의 전문가}.
\item
  \href{https://off-guardian.org/2020/03/28/10-more-experts-criticising-the-coronavirus-panic/}{코로나19
  패닉 현상을 비판하는 추가 10명의 전문가.}
\item
  \href{https://off-guardian.org/2020/04/17/8-more-experts-questioning-the-coronavirus-panic/}{코로나19
  패닉에 질문을 던지는 추가 8명의 전문가}.\\
\end{itemize}

\hypertarget{who-uxd32cuxb370uxbbf9-uxb4f1uxae09-uxbd84uxb958}{%
\subparagraph{\texorpdfstring{\textbf{WHO 팬데믹 등급
분류}}{WHO 팬데믹 등급 분류}}\label{who-uxd32cuxb370uxbbf9-uxb4f1uxae09-uxbd84uxb958}}

2007년 미국 보건당국은 펜데믹 인플루엔자를
\href{https://www.forbes.com/2010/02/05/world-health-organization-swine-flu-pandemic-opinions-contributors-michael-fumento.html\#1806e16948e8}{다섯
등급}으로 분류하고 각각의 등급에 따른 대응책을 규정했다 (원래 펜데믹이란
용어는 인플루엔자 독감에 사용되는 용어). 각 등급은 치명률 (CFR)에 따라
1등급 (0.1\% 이하) 에서 5등급 (2\% 이상)까지 분류되었으며 이 기준에
따르면 코로나19 팬데믹은 2등급 (0.1\%\textasciitilde{}0.5\%)에 해당한다.
2등급의 경우 ``질환자에 한하여 자발적 격리''를 권고하고 있다.

하지만 2009년 세계보건기구 WHO는
\href{https://www.forbes.com/2010/02/05/world-health-organization-swine-flu-pandemic-opinions-contributors-michael-fumento.html\#1806e16948e8}{펜데믹의
등급 분류를 삭제}해 버렸다. 이 때부터 전 세계의 인플루엔자 유행에 대해
자의적으로 펜데믹이 선언될 수 있게 되었고, 마침 공교롭게도 WHO는
2009/2010년 신종플루 (swine flu)에 대해 펜데믹을 선언한다. 결론적으로
신종플루는 매우 온화한 독감이었으나 이에따른 공포감의 확산으로 180억
달러어치의 백신이 판매되었다. 이 후 WHO는
\href{https://healthcare-in-europe.com/en/news/european-parliament-to-investigate-who-pandemic-scandal.html}{펜데믹
스캔들}에 대해 EU의 조사를 받고
\href{https://www.bmj.com/content/340/bmj.c3033}{의학저널 BMJ}에도
이러한 구조적 문제를 지적하는 논문이 여러 편 실렸다. ~

WHO는 여전히 재정의 상당 부분을 제약회사로부터 지원받고 있다.

신종플루와 관련 한 WHO의 미심쩍인 역할을 파헤친 다큐멘터리 필름 Trust
WHO (WHO 누구를 믿는가)는 VIEMO에서 시청이 가능했는데
\href{https://www.youtube.com/watch?v=VjQGyqVN5RM}{최근 삭제
조치되었다}.

\hypertarget{uxc2a4uxc704uxc2a4-uxcd1duxc0acuxb9dduxc790-uxc218uxb294-uxc815uxc0c1-uxbc94uxc704-uxc774uxb0b4}{%
\subparagraph{\texorpdfstring{\textbf{스위스: 총사망자 수는 정상 범위
이내}}{스위스: 총사망자 수는 정상 범위 이내}}\label{uxc2a4uxc704uxc2a4-uxcd1duxc0acuxb9dduxc790-uxc218uxb294-uxc815uxc0c1-uxbc94uxc704-uxc774uxb0b4}}

스위스의 1분기 (4월5일까지 집계) 총사망자 수는
\href{https://swprs.files.wordpress.com/2020/04/ch-sterblichkeit-kumuliert-q1-2020.pdf}{정상
범위 이내}에 머물고 있으며 정상 범위 상한선보다 1,500명 적은 것으로
나타났다. 4월 중순까지 집계한 결과 총사망자 수는 독감이 극심했던
2015년에 비해 2,000명 적은 것으로 집계되었다. (아래 표)

\href{https://swprs.files.wordpress.com/2020/04/schweiz-todesfaelle-2010-2020.png}{\includegraphics{https://swprs.files.wordpress.com/2020/04/schweiz-todesfaelle-2010-2020.png?w=700\&h=339}}

2010-2020 누적 총사망자 비교 (BFS)

\hypertarget{uxc2a4uxc6e8uxb374--uxbd09uxc1c4uxc870uxce58-uxc5c6uxc774-uxd655uxc0b0-uxac10uxc18cuxc138}{%
\subparagraph{\texorpdfstring{\textbf{스웨덴: ~봉쇄조치 없이 확산
감소세}}{스웨덴: ~봉쇄조치 없이 확산 감소세}}\label{uxc2a4uxc6e8uxb374--uxbd09uxc1c4uxc870uxce58-uxc5c6uxc774-uxd655uxc0b0-uxac10uxc18cuxc138}}

스웨덴의 최근 확진자와 사망자 추이를 볼 때 전염이 감소되는 것을 확인할
수 있다.
\href{https://www.washingtontimes.com/news/2020/apr/15/sweden-coronavirus-rates-easing-despite-loose-rule/}{스웨덴
보건당국에 따르면} 다른 나라들과 마찬가지로 스웨덴에서도 사망자의
대부분은 의료환경이 열악한 양로병원에 입원중인 노인들이 대부분이었다.

다른 국가들과 달리 스웨덴 국민들은 코로나19 바이러스에 대한 보다 나은
면역력을 획득하여 오는 겨울에 있을 수 있는 ``2차 웨이브''에 더 잘 견딜
가능성이 크다.

\href{https://swprs.files.wordpress.com/2020/04/sweden-deaths-day-2.png}{\includegraphics{https://swprs.files.wordpress.com/2020/04/sweden-deaths-day-2.png?w=736\&h=293}}

스웨덴 양성 판정 사망자
(\href{https://en.wikipedia.org/wiki/2020_coronavirus_pandemic_in_Sweden\#Charts7be6f9f87457ed9aa}{FOHM/wikipeida})

\hypertarget{uxcf54uxb85cuxb09819uxc640-uxc0b0uxc18cuxd638uxd761uxae30}{%
\subparagraph{\texorpdfstring{\textbf{코로나19와
산소호흡기}}{코로나19와 산소호흡기}}\label{uxcf54uxb85cuxb09819uxc640-uxc0b0uxc18cuxd638uxd761uxae30}}

유럽과 미국의 일선 의사들과 전문가들은 코로나19 환자 치료에 있어 침습적
인공호흡기 (기도삽관) 치료를 하지 말라고 경고했다. 코로나19 환자증
급성중증호흡기증후군 (ARDS)이 아닌 (세포내) 산소 결핍으이 의심되는데
이는 바이러스 또는 면역반응에 의해 발생하는 것이 의심된다고 밝혔다.

\begin{itemize}
\tightlist
\item
  AP:
  \href{https://apnews.com/8ccd325c2be9bf454c2128dcb7bd616d}{바이러스
  감염 환자에게 산소호흡기 사용을 꺼리는 의사들} 
\item
  영상: \href{https://www.youtube.com/watch?v=QPlEUAVjxV8}{코로나19
  치료에 있어서 조기 기도삽관에 관한 논의}
\item
  영상: \href{https://www.youtube.com/watch?v=NmRlvX3VrAQ}{뉴욕 중환자실
  진료 의사, 코로나 19 환자 저산소혈증 가능성 시사} 
\item
  논문:
  \href{https://link.springer.com/article/10.1007/s00134-020-06033-2}{코로나19
  폐렴: 표현형에 따른 다른 치료 접근}
\item
  빌트지(독일어):
  \href{https://www.welt.de/vermischtes/article207221877/Corona-Pandemie-Sterberate-bei-Beatmungspatienten-gibt-Raetsel-auf.html}{호흡기질환
  환자 사망률의 의문점} 
\end{itemize}

\hypertarget{2020uxb144-4uxc6d4-16uxc77c}{%
\paragraph{2020년 4월 16일}\label{2020uxb144-4uxc6d4-16uxc77c}}

\begin{itemize}
\tightlist
\item
  덴마크에서는 `락다운(봉쇄령)'에 대한
  \href{https://jyllands-posten.dk/debat/breve/ECE12074246/vi-skulle-aldrig-have-trykket-paa-stopknappen/}{회의론이
  대두되며,~ 학교들의 개학}을 시작하고 있다.
\item
  ~지금까지 스웨덴과 영국은
  \href{http://www.theblogmire.com/a-comparison-of-lockdown-uk-with-non-lockdown-sweden/}{인구대비
  대등}한 확진자와 사망자를 보이고 있다. 
\item
  ~에일 대학의 데이빗 캐츠 교수는 일찍부터 `봉쇄령`의
  \href{https://www.youtube.com/watch?v=VK0Wtjh3HVA}{부작용을 경고}한 바
  있다. 
\item
  ~이탈리아의 작은 도시 로비아에서 2,000명을 대상으로 검사한 결과,
  22\%에서 양성 또는 항체가 검출되었으며 대부분이 무증상 또는 경미한
  증상만을 나타냈다. 알려진 것 보다
  \href{https://www.tgcom24.mediaset.it/cronaca/a-robbio-pv-il-22-ha-o-ha-avuto-il-coronavirus-ok-del-sindaco-ai-test-per-tutti_17285128-202002a.shtml}{10배
  이상 많은 사람들이 감염}되었다는 뜻이며, 또한 감염이 반드시 질환으로
  이어지지는 않는다는 것을 의미한다. 이에따라 치명률도 낮아지게 된다. 
\item
  ~스위스의 최고 감염병학 권위자 피에트로 베르나짜 박사는 ``바이러스와
  동행하는 전략``을
  \href{https://infekt.ch/2020/04/exitstrategie-lockdown/}{권고}했다.
  개인의 면역과 건강이 가장 확실한 방어이며, 면역력이 강한 개인들이
  많아야 노약자들도 보호를 받는다고 설명했다. 백신을 기다리기 보다는,
  건강한 개인들이 많은 것이 가장 확실한 방역이라는 설명이다.
\item
  ~독일의 미생물학자이자 감염병 역학 권위자인 알렉산더 케쿨레 교수는
  영국 텔레그래프와의
  \href{https://www.telegraph.co.uk/news/2020/04/11/german-scientist-predicted-european-epidemic-calls-end-lockdown/}{인터뷰}에서
  봉쇄령을 해제해야 한다고 설명했다. 봉쇄령이 바이러스보다 더 많은
  피해를 초래한다는 설명이다. 
\item
  ~뉴잉글랜드저널오브메디슨 NEJM에 발표된
  \href{https://www.nejm.org/doi/full/10.1056/NEJMc2009316}{보고}에
  따르면 코로나19 양성 판정을 받은 임산부의 88\%가 무증상인 것으로
  나타났다. 높은 숫자인 것처럼 보이지만, 이 전에 보고된 중국과
  아이스랜드의 데이터도 이와 유사하다.
\item
  ~영국 병원들의 전체 병상 중 40\%가 공실인 것으로
  \href{https://www.hsj.co.uk/acute-care/nhs-hospitals-have-four-times-more-empty-beds-than-normal/7027392.article}{보고}되고
  있다. 평소의 4배에 해당한다. 타 질환 환자들의 진료나 수술 등이 연기된
  것이 원인이라는 분석이다. 평상시 중환자실의 공실율은 22\% 정도이다.
\item
  ~뉴욕에 설치된~ 임시 군 병원이 아직까지는 대부분 비어 있는 것으로
  \href{https://nypost.com/2020/04/09/usns-comfort-and-javits-center-mostly-empty-amid-coronavirus/}{확인}되었다.
\item
  ~뉴욕 병원의 입원율 예측이 최대 7배로
  \href{https://nypost.com/2020/04/09/usns-comfort-and-javits-center-mostly-empty-amid-coronavirus/}{지나치게
  과장}되었다는 지적이 나왔다.
\item
  ~미국에서 이 번주
  \href{https://finance.yahoo.com/news/coronavirus-covid-weekly-initial-jobless-claims-april-11-192401571.html}{524만
  명이 추가 실업수당을 신청}하면서 누적 신청 건수가 2,200만 명을
  돌파했다. 실업자, 자살자, 우울증, 가정폭력 피해자의 숫자가 심각한
  수준으로 증가하고있어, `현실`적인 피해자가 코로나19 `예상`피해자를
  압도한다는 의사들과 통계학자들의 분석이다. 
\end{itemize}

\hypertarget{uxc601uxc5b4uxb85c-uxb354-uxb9ceuxc740-uxc5c5uxb370uxc774uxd2b8}{%
\paragraph{\texorpdfstring{\href{https://swprs.org/a-swiss-doctor-on-covid-19/}{영어로
더 많은
업데이트.}}{영어로 더 많은 업데이트.}}\label{uxc601uxc5b4uxb85c-uxb354-uxb9ceuxc740-uxc5c5uxb370uxc774uxd2b8}}

\begin{center}\rule{0.5\linewidth}{\linethickness}\end{center}

쉐어:
\href{https://twitter.com/intent/tweet?url=https://swprs.org/covid19-korean/}{Twitter}
/
\href{https://www.facebook.com/share.php?u=https://swprs.org/covid19-korean/}{Facebook}

\hypertarget{swiss-policy-research}{%
\subsubsection{Swiss Policy Research}\label{swiss-policy-research}}

\begin{itemize}
\tightlist
\item
  \href{https://swprs.org/kontakt/}{Kontakt}
\item
  \href{https://swprs.org/uebersicht/}{Übersicht}
\item
  \href{https://swprs.org/donationen/}{Donationen}
\item
  \href{https://swprs.org/disclaimer/}{Disclaimer}
\end{itemize}

\hypertarget{english}{%
\subsubsection{English}\label{english}}

\begin{itemize}
\tightlist
\item
  \href{https://swprs.org/contact/}{About Us / Contact}
\item
  \href{https://swprs.org/media-navigator/}{The Media Navigator}
\item
  \href{https://swprs.org/the-american-empire-and-its-media/}{The CFR
  and the Media}
\item
  \href{https://swprs.org/donations/}{Donations}
\end{itemize}

\hypertarget{follow-by-email}{%
\subsubsection{Follow by email}\label{follow-by-email}}

Follow

\href{https://wordpress.com/?ref=footer_custom_com}{WordPress.com}.

\protect\hyperlink{}{Up ↑}

Post to

\protect\hyperlink{}{Cancel}

\includegraphics{https://pixel.wp.com/b.gif?v=noscript}
