\protect\hyperlink{content}{Skip to content}

\href{https://swprs.org/}{}

\protect\hyperlink{search-container}{Search}

Search for:

\href{https://swprs.org/}{\includegraphics{https://swprs.files.wordpress.com/2020/05/swiss-policy-research-logo-300.png}}

\href{https://swprs.org/}{Swiss Policy Research}

Geopolitics and Media

Menu

\begin{itemize}
\tightlist
\item
  \href{https://swprs.org}{Start}
\item
  \href{https://swprs.org/srf-propaganda-analyse/}{Studien}

  \begin{itemize}
  \tightlist
  \item
    \href{https://swprs.org/srf-propaganda-analyse/}{SRF / ZDF}
  \item
    \href{https://swprs.org/die-nzz-studie/}{NZZ-Studie}
  \item
    \href{https://swprs.org/der-propaganda-multiplikator/}{Agenturen}
  \item
    \href{https://swprs.org/die-propaganda-matrix/}{Medienmatrix}
  \end{itemize}
\item
  \href{https://swprs.org/medien-navigator/}{Analysen}

  \begin{itemize}
  \tightlist
  \item
    \href{https://swprs.org/medien-navigator/}{Navigator}
  \item
    \href{https://swprs.org/der-propaganda-schluessel/}{Techniken}
  \item
    \href{https://swprs.org/propaganda-in-der-wikipedia/}{Wikipedia}
  \item
    \href{https://swprs.org/logik-imperialer-kriege/}{Kriege}
  \end{itemize}
\item
  \href{https://swprs.org/netzwerk-medien-schweiz/}{Netzwerke}

  \begin{itemize}
  \tightlist
  \item
    \href{https://swprs.org/netzwerk-medien-schweiz/}{Schweiz}
  \item
    \href{https://swprs.org/netzwerk-medien-deutschland/}{Deutschland}
  \item
    \href{https://swprs.org/medien-in-oesterreich/}{Österreich}
  \item
    \href{https://swprs.org/das-american-empire-und-seine-medien/}{USA}
  \end{itemize}
\item
  \href{https://swprs.org/bericht-eines-journalisten/}{Fokus I}

  \begin{itemize}
  \tightlist
  \item
    \href{https://swprs.org/bericht-eines-journalisten/}{Journalistenbericht}
  \item
    \href{https://swprs.org/russische-propaganda/}{Russische Propaganda}
  \item
    \href{https://swprs.org/die-israel-lobby-fakten-und-mythen/}{Die
    »Israel-Lobby«}
  \item
    \href{https://swprs.org/geopolitik-und-paedokriminalitaet/}{Pädokriminalität}
  \end{itemize}
\item
  \href{https://swprs.org/migration-und-medien/}{Fokus II}

  \begin{itemize}
  \tightlist
  \item
    \href{https://swprs.org/covid-19-hinweis-ii/}{Coronavirus}
  \item
    \href{https://swprs.org/die-integrity-initiative/}{Integrity
    Initiative}
  \item
    \href{https://swprs.org/migration-und-medien/}{Migration \& Medien}
  \item
    \href{https://swprs.org/der-fall-magnitsky/}{Magnitsky Act}
  \end{itemize}
\item
  \href{https://swprs.org/kontakt/}{Projekt}

  \begin{itemize}
  \tightlist
  \item
    \href{https://swprs.org/kontakt/}{Kontakt}
  \item
    \href{https://swprs.org/uebersicht/}{Seitenübersicht}
  \item
    \href{https://swprs.org/medienspiegel/}{Medienspiegel}
  \item
    \href{https://swprs.org/donationen/}{Donationen}
  \end{itemize}
\item
  \href{https://swprs.org/contact/}{English}
\end{itemize}

\protect\hyperlink{}{Open Search}

\hypertarget{covid19-hakkux131nda-bilgiler-arux15fiv}{%
\section{Covid19 hakkında
bilgiler~(Arşiv)}\label{covid19-hakkux131nda-bilgiler-arux15fiv}}

\hypertarget{ana-madde-covid19-hakkux131nda-geruxe7ekler}{%
\paragraph{\texorpdfstring{\href{https://swprs.org/isvicreli-bir-doktordan-kovid-19-uezerine/}{Ana
madde: Covid19 hakkında
gerçekler}}{Ana madde: Covid19 hakkında gerçekler}}\label{ana-madde-covid19-hakkux131nda-geruxe7ekler}}

\hypertarget{25-nisan-2020}{%
\paragraph{25 Nisan 2020}\label{25-nisan-2020}}

\hypertarget{tux131bbi-guxfcncellemeler}{%
\subparagraph{\texorpdfstring{\textbf{Tıbbi
güncellemeler}}{Tıbbi güncellemeler}}\label{tux131bbi-guxfcncellemeler}}

\begin{itemize}
\tightlist
\item
  Berlin'deki Charité Kliniği'nde ünlü Alman virolog Christian
  Drosten'nın selefi olan Profesör Detlef Krüger,
  \href{https://de.sputniknews.com/interviews/20200425326953541-corona-gefahr-virologe/}{kendisiyle
  bu yakınlarda yapılan bir söyleşide}, Kovid-19'un ``bir çok açıdan
  grip ile karşılaştırılabilir olduğu''nu ve ``bazı değişik tür grip
  virüslerinden daha tehlikeli olmadığı''nı anlatıyor. Profesör Krüger,
  ``politikacılarca keşfedilen ağız ve burun koruma işini'', ``eylem
  severlik'' ve potansiyel ``mikrop-yayıcılık'' diye kabul ediyor. Aynı
  zamanda da alınan önlemlerin yol açtığı ``kitlesel ikincil tahribat''a
  karşı uyarıda bulunuyor.
\item
  İsveç ve Avrupa eski baş epidemiyoloğu Profesör Johan Giesecke,
  Avusturya'da yayınlanan Addendum dergisine verdiği
  \href{https://www.addendum.org/coronavirus/interview-johan-giesecke/}{samimi
  söyleşide}, bazı insanların hiç belirti göstermemesi, bazılarınınsa
  çok hafif belirtiler yaşaması nedeniyle, salgının \%75 ile \%90'ının
  ``görünmez'' olduğunu söylüyor. İşte bu yüzden tecrit ``anlamsız'' ve
  topluma zararlı olacaktır. İsveç'in stratejisinin temelinde
  ``insanların aptal olmadığı'' kabulü yatıyordu. Profesör Giesecke,
  influenza'dakine benzer, yani \%0,1 ile 0,2 arasında bir ölüm oranı
  bekliyor; İtalya ve New York'un virüse çok hazırlıksız yakalandığını
  ve kendi risk gruplarını korumadıklarını ileri sürüyor.
\item
  \href{https://www.epicentro.iss.it/coronavirus/bollettino/Bollettino-sorveglianza-integrata-COVID-19_16-aprile-2020.pdf\#page=13}{İtalya'dan
  gelen son rakamlara} göre (s.12-13), testleri pozitif çıkan yaklaşık
  17.000 doktor ve hemşireden 60'ının öldüğünü gösteriyor. Bu da
  Kovid-19 ölümcüllüğünün, 50 yaşının altındakiler için \%0,1'den daha
  az; 50-60 yaş aralığındakiler için \%0,27; 60-70 yaş aralığındakiler
  için \%1,4; 70-80 yaş aralığındakiler içinse \%12,6 olduğu sonucunu
  veriyor. Bu ölümler korona virüsüyle olup, illa korona virüsünden
  olmadığından dolayı ve \%80'e varan oranda insan belirti göstermediği
  ve bazılarına da test yapılmamış olduğu için, bu rakamlar bile yüksek
  olabilir. Yine de bu değerler toplamda, örneğin Güney Kore'dekilerle
  uyumlu olup, genel nüfus için influenza aralığındadır.
\item
  İtalyan Sivil Savunma Kurumu başkanının
  \href{https://www.theguardian.com/world/2020/apr/16/italian-police-broaden-care-home-coronavirus-milan}{Nisan
  ayı ortasında yaptığı açıklamaya göre,} Lombardiya'daki bakım
  evlerinde 1800'den fazla insan ölmüştür ve birçok vakada ölüm nedeni
  hala açık değildir. Lombardiya'nın bazı bölgelerinde bulunan huzur
  evleri ve bakım evlerindeki bakımın, bunun sonucunda da tüm sağlık
  sisteminin, kısmen
  \href{https://swprs.org/covid-19-a-report-from-italy/}{virüs korkusu
  ve tecrit} nedeniyle, çökmüş olduğu zaten biliniyordu.
\item
  \href{https://covid-19.sciensano.be/sites/default/files/Covid19/Meest\%20recente\%20update.pdf}{Belçika'dan
  gelen son rakamlar}, orada da tüm fazladan ölümlerin \%50'den biraz
  fazlasının, genel tecritten yarar görmeyen bakım evlerinde olduğunu
  gösteriyor. Bu ölümlerin \%6'sının Kovid-19'dan olduğu ``teyit
  edilmiştir'', \%94'ünde ise Kovid-19 ``kuşkusu'' vardır. Testleri
  pozitif çıkmış insanların (bakım evi çalışanları ve sakinleri)
  yaklaşık \%70'i hiç belirti göstermemiştir.
\item
  İngiltere'de yayınlanan Guardian gazetesinde, Kovid-19 ölümlerinde
  hava kirliliğinin ``ana etken'' olabileceğine işaret eden
  \href{https://www.theguardian.com/environment/2020/apr/20/air-pollution-may-be-key-contributor-to-covid-19-deaths-study?utm_medium}{yeni
  araştırmalardan alıntılar yapılıyor}. Örneğin, dört ülkedeki ölümlerin
  \%80'i (Lombardiya ve Madrid dahil) en fazla hava kirliliği olan
  yerlerdeydi.
\item
  Kaliforniyalı Dr. Dan Erickson,
  \href{https://www.turnto23.com/news/coronavirus/video-interview-with-dr-dan-erickson-and-dr-artin-massihi-taken-down-from-youtube}{çok
  izlenen basın açıklamasında} Kovid-19 konusundaki gözlemlerini
  anlattı. Kaliforniya ve diğer eyaletlerdeki hastaneler ve yoğun bakım
  üniteleri şu ana kadar büyük ölçüde boş kalmıştır. Dr. Erickson,
  ABD'denin birçok eyaletinde doktorların hemfikir olmasalar da Kovid-19
  yazılı ölüm belgeleri vermeye ``zorlandıklarını'' belirtiyor. Dr.
  Erickson, sağlıklı insanların değil yalnızca hastaların karantinada
  tutulmasını tavsiye ediyor, bunun aksinin beden ve ruh sağlığı
  üzerinde olumsuz etkilere yol açabileceğine dikkat çekiyor. Alkolizm,
  depresyon, intihar, çocuk ve eşlerin kötü muameleye maruz kalması gibi
  ``ikincil etkiler''de önemli ölçüde artış gözlenmeye başlamış
  durumdadır. Dr. Erickson, çeşitli ülkelerden gelen rakamlara
  dayanarak, Kovid-19'un ölümcüllüğünün, influenza gibi yaklaşık \%0,1
  olduğunu hesaplamaktadır. Dr Erickson'a göre, yüz maskeleri gündelik
  hayatta değil, hastanedeki gibi akut durumlarda mantıklıdır.
\item
  Almanya'da yayınlanan DIE ZEIT gazetesi,
  \href{https://www.zeit.de/2020/18/kliniken-coronavirus-intensivbetten-patienten-behandlung-notaufnahme}{Almanya'daki
  hastanelerindeki boş yatak oranlarına} odaklanmaktadır. Bazı
  bölümlerde \%70 gibi yüksek oranlara rastlanıyor. Akut olarak hayati
  olmayan kanser tetkikleri ve organ nakilleri, şu ana kadar ortada
  olmayan Kovid-19 hastalarına yer açmak amacıyla iptal edilmektedir.
\item
  İngiltere'de yapılan yeni bir analiz, sağlık sistemini
  kullanamadıkları veya kullanmak istemedikleri için şu anda
  \href{https://www.telegraph.co.uk/global-health/science-and-disease/two-new-waves-deaths-break-nhs-new-analysis-warns/}{haftada
  yaklaşık 2000 kişinin} Kovid-19 olmaksızın evlerinde öldüğü sonucuna
  varıyor. Bunlar en başta, kronik hastalıkları olan insanların yanısıra
  kalp krizi ve inme geçiren acil servis hastalarıdır.
\item
  Avusturya'da araştırmacılar, Mart ayında ülkede Kovid-19'dan
  ölenlerden daha fazla insanın, tedavisi yapılmayan kalp krizlerinden
  öldüğü
  \href{https://academic.oup.com/eurheartj/advance-article/doi/10.1093/eurheartj/ehaa314/5820829}{sonucuna
  varmıştır}.
\item
  Almanya'da, toplu taşıma araçlarında ve alışveriş yerlerinde maske
  zorunluluğu getirilmişti. Dünya Tıp Birliği başkanı Frank Montgomery,
  bunu ``yanlış'', bu amaçla kullanılan atkı veya kumaşları ise
  ``gülünç'' bulduğunu belirterek
  \href{https://www.aerztezeitung.de/Politik/Montgomery-haelt-Maskenpflicht-fuer-falsch-408844.html}{eleştirmişti}.
  Gerçekten de araştırmalar, İsviçreli enfeksiyolog Dr. Vernazza'nın bir
  \href{https://infekt.ch/2020/04/atemschutzmasken-fuer-alle-medienhype-oder-unverzichtbar/}{``medya
  aldatmacası''} diye söz etmesine neden olan, gündelik hayatta maske
  kullanımının, sağlıklı ve belirti göstermeyen insanlara neden
  ölçülebilir bir yarar getirmediğini gösteriyor. Bunu eleştiren
  başkaları da
  \href{https://multipolar-magazin.de/artikel/maskenpflicht-gesellschaftliches-klima}{``kamusal
  alanda görünür halde olan zoraki bir itaat''} simgesi olduğundan söz
  ediyor.
\item
  Dünya Sağlık Örgütü'nün 2019 yılında yayınlanan bir çalışmasında,
  ``sosyal mesafe'', seyahat kısıtları ve tecritler gibi önlemlerin
  etkisine ilişkin
  \href{https://www.heise.de/tp/features/COVID-19-WHO-Studie-findet-kaum-Belege-fuer-die-Wirksamkeit-von-Eindaemmungsmassnahmen-4706446.html}{ya
  hiç kanıt bulunmadığını veya çok az kanıt olduğu} ortaya konmuştu.
  (\href{https://www.who.int/influenza/publications/public_health_measures/publication/en/}{Orijinal
  çalışma})
\item
  Almanya'daki bir laboratuvar tarafından
  \href{http://www.labor-augsburg-mvz.de/de/aktuelles/coronavirus}{Nisan
  ayı başında belirtildiği gibi}, Dünya Sağlık Örgütü'nün tavsiyelerine
  uygun olarak, Kovid-19 virüsü spesifik hedef dizisi negatif olup,
  yalnızca daha genel korona virüs hedef dizisi pozitif bile olsa,
  Kovid-19 virüs testleri artık pozitif kabul ediliyor. Halbuki, bu
  durum diğer korona virüslerinin (soğuk algınlığı virüslerinin) de
  hatalı pozitif test sonuçları tetiklemesine yol açabiliyor. Bu
  laboratuvar aynı zamanda, Kovid-19 antikorlarının çoğunlukla
  belirtilerin ortaya çıkışından yalnızca iki üç hafta sonra
  belirlenebildiğini de açıkladı. Bu durum, Kovid-19'a zaten bağışıklık
  geliştirmiş insanların gerçek sayısının olduğundan düşük tahmin
  edilmemesi için hesaba katılmalıdır.
\item
  Hem
  \href{https://www.20min.ch/schweiz/news/story/-rzte-und-Politiker-fordern-Corona-Impfzwang-20853917}{İsviçre'de}
  hem de
  \href{https://www.faz.net/agenturmeldungen/dpa/soeder-waere-fuer-deutschlandweite-impfpflicht-gegen-corona-16738369.html}{Almanya'da},
  bazı politikacılar ``koronaya karşı zorunlu aşılama'' çağrısında
  bulunmuştu. Halbuki, 2009/2010 yılındaki ``domuz gribi'' denilen
  hastalığa karşı yapılan aşılar, örneğin, özellikle de çocuklarda bazen
  \href{https://www.ibtimes.co.uk/brain-damaged-uk-victims-swine-flu-vaccine-get-60-million-compensation-1438572}{şiddetli
  olabilen nörolojik tahribata} ve milyonluk tazminatlar ödenmesine yol
  açmıştır.
\item
  Profesör Christopher Kuhbandner,
  \href{https://www.heise.de/tp/features/Von-der-fehlenden-wissenschaftlichen-Begruendung-der-Corona-Massnahmen-4709563.html?seite=all}{korona
  önlemlerine ilişkin bilimsel gerekçe eksikliği} üzerine şunları
  söylemiştir: ``Yeni enfeksiyonlara ilişkin rakamlar, korona virüsünün
  gerçek yayılımının çok çarpıcı bir biçimde abartılmasına neden oluyor.
  Yeni enfeksiyonlarda gözlenen hızlı artış, neredeyse tümüyle, test
  sayısının zaman içinde hızla arttığı gerçeğinden kaynaklanıyor (bkz.
  alttaki şekil). Yani, en azından bildirilen rakamlara göre, korona
  virüsünün eksponansiyel (giderek hızlanan) yayılımı aslında hiçbir
  zaman olmadı. Bildirilen yeni enfeksiyon sayıları, yeni enfeksiyon
  sayısının Mart ayı başından veya ortasından bu yana azaldığı gerçeğini
  gizliyor.''
\end{itemize}

\includegraphics{https://swprs.files.wordpress.com/2020/04/zunahme-infektionen-tests-tag.png?w=550\&h=404}

\hypertarget{isveuxe7-medya-ile-geruxe7ek}{%
\subparagraph{\texorpdfstring{\textbf{İsveç: Medya ile
gerçek}}{İsveç: Medya ile gerçek}}\label{isveuxe7-medya-ile-geruxe7ek}}

Çoğu medya kuruluşu dimdik yükselen bir eğri gösterdiği için, İsveç'te
ölümlerin azalması bazı okurları şaşırtmıştı. Bunun nedeni nedir?
İsveçli yetkililer, çok daha anlamlı olan, ölüm tarihi itibariyle
gerçekleşen gündelik ölüm sayılarını yayınlarken, çoğu medya kuruluşu
haber tarihi itibariyle gerçekleşmiş kümülatif ölüm rakamlarını veriyor.

İsveçli yetkililer, yeni haber yapılan tüm ölüm vakalarının son 24
saatte olmadığını her zaman vurguluyor, ama çoğu medya kuruluşu bunu yok
sayıyor (bkz. aşağıdaki grafik). Tüm ülkelerde olduğu gibi İsveç'teki en
son rakamlar da bir biçimde artabilecek olsa bile, bu geneldeki azalma
eğilimini değiştirmez.

Buna ek olarak, bu rakamlar illa korona virüsünden olan ölümleri değil,
korona virüslü ölümleri temsil ediyor. İsveç, Avrupa'daki
\href{https://link.springer.com/article/10.1007/s00134-012-2627-8}{en
düşük yoğun bakım kapasitelerinden birine} sahip olduğu halde, genel
halk üzerindeki etki en alt düzeyde kalırken, ölümlerin \%50'si
hassasiyeti yüksek bakım evlerinde gerçekleşmiş olup, İsveç'teki
ortalama ölüm yaşı da 80'in üzerindedir.

Halbuki, İsveç hükumetine de ``korona'' nedeniyle,
\href{https://www.tagesschau.de/faktenfinder/ausland/corona-kursaenderung-schweden-103.html}{yeni
acil durum yetkileri} verilmiştir ve daha sonraki temas izleme
programlarına hala katılabilir.

\includegraphics{https://swprs.files.wordpress.com/2020/04/sweden-corona-media-vs-reality.png?w=736\&h=338}

\hypertarget{buxfcyuxfck-britanyadaki-durum}{%
\subparagraph{\texorpdfstring{\textbf{Büyük Britanya'daki
durum}}{Büyük Britanya'daki durum}}\label{buxfcyuxfck-britanyadaki-durum}}

İngiltere'de ölümler son haftalarda keskin bir artış göstermiş olsa da,
halen son elli yılın \href{http://inproportion2.talkigy.com/}{güçlü grip
mevsimleri} aralığındadır (bkz. aşağıdaki grafik). İngiltere'de de
fazladan ölümlerin
\href{https://ltccovid.org/2020/04/12/mortality-associated-with-covid-19-outbreaks-in-care-homes-early-international-evidence/}{\%50'ye
varan} bölümü genel tecritten yarar görmeyen bakım evlerinde olmuştur.

Dahası, fazladan ölümlerin
\href{https://www.thetimes.co.uk/edition/news/coronavirus-record-weekly-death-toll-as-fearful-patients-avoid-hospitals-bm73s2tw3}{\%50'ye
varan} bölümü Kovid-19-dışı ölümler olup,
\href{https://www.ft.com/content/67e6a4ee-3d05-43bc-ba03-e239799fa6ab}{\%25'e
varan oranda} evlerde gerçekleşmiş olduğu söyleniyor. Bu nedenle, genel
tecritin toplumun tamamı için aslında yararlı mı zararlı mı olduğu belli
değildir.

İngiltere'de yayınlanan Spectator gazetesinin editörü, hükumet
kuruluşlarının, tecritin uzun vadede, 150.000 fazladan ölüm yol
açacağını beklediğini
\href{https://www.telegraph.co.uk/politics/2020/04/09/boris-worried-lockdown-has-gone-far-can-end/}{iddia
ediyor}. Bu rakam Kovid-19'un yol açması beklenen ölümlerden önemli
ölçüde fazladır. En son olarak, 17 yaşındaki bir öğrenci-şarkıcının
tecrit nedeniyle yaşamına son verdiği
\href{https://sports.yahoo.com/coronavirus-bethany-palmer-teenager-death-suicide-152707750.html}{duyuldu}.

İngiltere'de, (İsveç dahil) çoğu diğer ülkenin aksine,
\href{https://www.euromomo.eu/}{15 -- 64 yaş aralığındaki
i}\href{https://www.euromomo.eu/}{nsanlarda} bile, ölüm sayısının önemli
ölçüde yükselmiş olması dikkat çekicidir. Bunun nedeni, önceden var olan
ve sık rastlanan kalp-damar hastalıkları veya
\href{https://www.telegraph.co.uk/global-health/science-and-disease/two-new-waves-deaths-break-nhs-new-analysis-warns/}{tecritin
etkileri} olabilir.

\href{http://inproportion2.talkigy.com/}{InProportion adlı proje}
kapsamında, İngiltere'de şu andaki ölüm sayılarını önceki grip
salgınları ve başka ölüm nedenleriyle karşılaştıran çok sayıda yeni
grafik yayınlanmıştır. İngiltere'nin durumunu ve önlemleri eleştirel bir
biçimde ele alan diğer websiteleri,
\href{https://lockdownsceptics.org/}{Lockdown Skeptics} ve
\href{https://www.ukcolumn.org/}{UK Column}'dur.

\includegraphics{https://swprs.files.wordpress.com/2020/04/inproportion2_chart5.png?w=736\&h=363}

\hypertarget{isviuxe7re-fazladan-uxf6luxfcmler-guxfcuxe7luxfc-grip-dalgalarux131nux131nkinin-epey-altux131nda}{%
\subparagraph{\texorpdfstring{\textbf{İsviçre: Fazladan ölümler güçlü
grip dalgalarınınkinin epey
altında}}{İsviçre: Fazladan ölümler güçlü grip dalgalarınınkinin epey altında}}\label{isviuxe7re-fazladan-uxf6luxfcmler-guxfcuxe7luxfc-grip-dalgalarux131nux131nkinin-epey-altux131nda}}

\begin{itemize}
\tightlist
\item
  Cenevre Üniversitesi tarafından yapılan ilk serolojik araştırmada,
  Cenevre kantonunda önceden düşünüldüğünden en az altı kat daha fazla
  insanın Kovid-19'a maruz kaldığı
  \href{https://www.hug-ge.ch/medias/communique-presse/seroprevalence-covid-19-premiere-estimation}{sonucuna
  varılmıştır}. Bu ise, resmi kaynaklar ölümcüllük oranını hala \%5'e
  kadar, diye veriyor olsa da Kovid-19'un ölümcüllüğünün İsviçre'de de
  \%1'in altına düştüğü anlamına geliyor.
\item
  En şiddetli etkilenen Ticino kantonunda bile, fazladan ölümlerin
  \href{https://www.bluewin.ch/de/news/schweiz/sp-chef-levrat-will-die-reichen-schropfen-383977.html}{neredeyse
  yarısı} genel tecritten yarar görmeyen bakım evlerinde olmuştur.
\item
  İsviçre'de, 1,85 milyon kişi veya tüm çalışanların üçte birinden
  fazlası,
  \href{https://www.bluewin.ch/de/news/schweiz/sp-chef-levrat-will-die-reichen-schropfen-383977.html}{kısaltılmış
  mesai yapmak üzere şimdiden kaydedildi}. Mart'tan Haziran'a kadar olan
  dönem için bunun ekonomik maliyetinin 32 milyar İsviçre Frangı olacağı
  tahmin ediliyor.
\item
  Infosperber:
  \href{https://www.infosperber.ch/Artikel/Medien/Corona-NZZ-deckt-das-Nachplappern-anderer-Medien-auf}{Korona:
  Medyanın papağanlığı}. ``Büyük medya kuruluşları, Kovid-19 rakamları
  için kuşkulu verileri dayanak aldıkları gerçeğini gizliyor.''
\item
  Ktipp:
  \href{https://www.ktipp.ch/artikel/artikeldetail/bund-fast-alle-zahlen-ohne-gewaehr/}{İsviçreli
  yetkililer: Rakamların neredeyse hiçbirinin `garantisi
  yok'}\href{https://www.ktipp.ch/artikel/artikeldetail/bund-fast-alle-zahlen-ohne-gewaehr/}{.}
  ``Bu yılın ilk 14 haftasında, son beş yıla göre daha az 65 yaşının
  altında insan öldü. 65 yaşının üzerindekiler arasında da rakam görece
  düşüktü.''
\end{itemize}

Aşağıdaki grafik, 2020 yılının ilk dört ayındaki toplam ölüm sayısının
normal aralıkta, Nisan ayı ortasında ise hala 2015 yılı grip
dalgasındakinin yaklaşık 2000 kişi daha altında olduğunu gösteriyor.
Ölümlerin \%50'si tecritten yarar görmeyen
\href{https://www.nzz.ch/zuerich/coronavirus-zuerich-aendert-nun-das-testregime-in-heimenauch-viele-aeltere-covid-19-infizierte-entwickeln-keine-symptome-zuerich-aendert-nun-das-testregime-in-heimen-ld.1552089}{bakım
evlerinde} olmuştur.

Toplamda, hastaneler ve yoğun bakım üniteleri
\href{https://swprs.files.wordpress.com/2020/04/intensivbettenbelegung-schweiz-2020-04-14.png}{büyük
ölçüde kapasitelerinin altında çalışırken} ve birçok ameliyat iptal
edilmiş durumdayken, fazladan ölümlerin yaklaşık \% 75'i
\href{https://www.tagesspiegel.de/wissen/woran-sterben-corona-patienten-wirklich-ein-schweizer-forscher-macht-hoffnung-im-kampf-gegen-covid-19/25750666.html}{evlerde}
gerçekleşmiştir. Bu nedenle, İsviçre'de de ``tecrit''in kurtardığından
daha fazla cana malolmuş olma olasılığına ilişkin çok ciddi bir soruyu
ortaya çıkıyor.

\includegraphics{https://swprs.files.wordpress.com/2020/04/schweiz-todesfaelle-2010-2020.png?w=736\&h=357}

\hypertarget{politik-guxfcncellemeler}{%
\subparagraph{\texorpdfstring{\textbf{Politik
güncellemeler}}{Politik güncellemeler}}\label{politik-guxfcncellemeler}}

\begin{itemize}
\tightlist
\item
  Video: Avustralya'nın Queensland eyaletinde, gece görüş donanımı olan
  bir polis helikopteri, bir evin çatısında gece bira içen ve böylece
  ``Korona düzenlemeleri''ni ihlal eden
  \href{https://twitter.com/Independent/status/1252911273597120513}{üç
  genç adamın izini sürerek yakaladı}. Gençlere, megafonla binanın
  ``polis tarafından sarıldığı'' ve çıkışa gelmek zorunda oldukları
  bildirildi. Herbirine 1000'er Avustralya doları
  \href{https://www.dailystar.co.uk/news/world-news/police-helicopter-uses-night-vision-21899640}{ceza
  yazıldı}.
\item
  İsrail'de, ulusal anti-terör gizli servisi Shin Bet, polisle işbirliği
  yaparak, Mart ayının ortasından beri Kovid-19 bağlamında, temas izleme
  ve ev hapsi emri için, halkın cep telefonlarını izlemekle
  \href{https://www.jewishpress.com/news/the-courts/state-to-high-court-even-more-shin-bet-involvement-in-fighting-the-coronavirus/2020/04/14/}{görevlendirilmiştir}.
  Bu önlemler başlangıçta Parlamento'nun onayı olmaksızın alınmış olup
  en az Nisan sonuna kadar yürürlükte kalacaktır.
\item
  OffGuardian:
  P\href{https://off-guardian.org/2020/04/23/the-seven-step-path-from-pandemic-to-totalitarianism/}{andemiden
  Totaliterliğ}\href{https://off-guardian.org/2020/04/23/the-seven-step-path-from-pandemic-to-totalitarianism/}{e
  giden yolun 7 Adımı}
\item
  UK Column:
  \href{https://www.ukcolumn.org/article/who-controls-british-government-response-covid19-part-one}{İngiliz
  Hükümeti'nin Kovid-19'a verdiği yanıtı kim kontrol ediyor?}
\item
  İsviçre polisinin özel bir birimi tarafından gözaltına alınıp, bir
  psikiyatri kliniğine gönderilmiş olan, korona önlemlerini eleştiren
  İsviçreli doktor (bkz. 15 Nisan tarihli güncelleme), bu arada
  salıverildi. Weltwoche dergisinde yayınlanan bir haber, doktorun
  gerçek dışı nedenlerle gözaltına alındığını
  \href{https://uncut-news.ch/wp-content/uploads/2020/04/Wer-l\%C3\%B6ste-den-Fehlalarm-aus.pdf}{açıklıyor}:
  Akrabalarına ve yetkililere yönelik bir tehdidi de yoktur, dolu bir
  silah taşıması da söz konusu değildir. İşte bu yüzden, bunun siyasi
  motivasyonla yapılmış bir operasyon olma olasılığı vardır.
\item
  Medya denetim kurumu, Mart ayında korona uygulamalarını eleştiren
  doktorlarla söyleşi yapan Münih'teki bir yerel radyo istasyonuna,
  şikayetler nedeniyle ``bu tür sorunlu yayınların gelecekte yapılmaması
  gerektiğini''
  \href{https://norberthaering.de/medienversagen/radiomuenchen-blm-meinungsvielfalt/}{bildirdi}.
\item
  Bir Alman uzman avukata ait olan
  \href{https://kollateral.news/}{kollateral.news} websitesinde,
  ``tecrit yüzünden çekilen eziyet'' ve Almanya'daki hastanelerin gerçek
  durumuna ilişkin haberler toplanıyor.
\item
  Almanya'da pratisyen hekimler, ``korona krizinin daha sorumlu bir
  biçimde ele alınması'' için
  \href{https://aerzteinnenvorort.de/der-appell}{politika ve bilim
  dünyasına yönelik bir çağrı yayınladılar.}
\item
  Hem
  \href{https://www.sn.at/panorama/oesterreich/arzt-droht-berufsverbot-wegen-kritik-an-corona-massnahmen-86594140}{Avusturya'da}
  hem de
  \href{https://magyarhang.org/belfold/2020/04/16/etikai-vizsgalat-indul-az-orvos-ellen-aki-szerint-nincs-jarvany-es-az-idosek-csak-a-felelemtol-halnak-meg/}{Macaristan'da},
  korona önlemlerini eleştiren doktorlar meslekten men edilme tehdidi
  altında.
\item
  Nijerya'da, resmi rakamlara göre, şu ana kadar virüsten ölenlerden
  daha fazla insan,
  \href{https://www.bbc.com/news/world-africa-52317196}{korona sokağa
  çıkma yasağını uygulayan polis tarafından} öldürüldü.
\end{itemize}

\hypertarget{21-nisan-2020}{%
\paragraph{21 Nisan 2020}\label{21-nisan-2020}}

\hypertarget{tux131bbi-guxfcncellemeler-1}{%
\subparagraph{\texorpdfstring{\textbf{Tıbbi
güncellemeler}}{Tıbbi güncellemeler}}\label{tux131bbi-guxfcncellemeler-1}}

\begin{itemize}
\tightlist
\item
  Stanford Üniversitesi tıp profesörü John Ioannidis,
  \href{https://www.youtube.com/watch?v=cwPqmLoZA4s}{kendisiyle yapılan
  yeni bir 1 saatlik söyleşide} Kovid-19 ile ilgili birkaç yeni
  çalışmanın sonuçlarını anlatıyor. Profesör Ioannidis'e göre,
  Kovid-19'un ölümcüllüğü ``mevsimsel grip aralığında''dır. 65 yaşının
  altındaki sağlıklı insanlar için ölüm riski ``tümüyle ihmal
  edilebilir'' düzeydeyken, hastalığın dünyada ``en yoğun görüldüğü
  yerler''deki 65 yaşının altındaki insanlar için ölüm riski, işe gidip
  gelirkenki bir otomobil yolculuğundakine eşdeğerdir. Yalnızca New York
  kentinde yaşayan 65 yaşının altındaki insanlar için ölüm riski ise bir
  uzun mesafe kamyon sürücüsününkine eşdeğerdir.
\item
  Oxford Üniversitesi'ndeki Kanıta-Dayalı Tıp Merkezi Müdürü Profesör
  Carl Heneghan,
  \href{https://news.yahoo.com/lockdown-damage-outweighs-coronavirus-warning-121940675.html}{yeni
  bir makalede} tecritin yol açtığı tahribatın virüsün yol açtığından
  daha fazla olabileceği uyarısında bulunuyor. Profesör Heneghan,
  salgının birçok ülkede tecritten önce zaten en tepe noktaya ulaşmış
  durumda olduğunu ileri sürüyor.
\item
  Los Angeles Bölgesi'ndeki
  \href{http://publichealth.lacounty.gov/phcommon/public/media/mediapubhpdetail.cfm?prid=2328}{yeni
  bir serolojik çalışmada} daha önce varsayılandan 28 ile 55 kat daha
  fazla insanın (önemli belirtiler göstermeksizin) Kovid-19'a
  yakalandığı bulunmuştur ki bu da hastalığın tehlike derecesini görece
  olarak düşürmektedir.
\item
  Boston yakınlarındaki Chelsea kentinde,
  \href{https://archive.is/20200418222442/https://www.bostonglobe.com/2020/04/17/business/nearly-third-200-blood-samples-taken-chelsea-show-exposure-coronavirus/}{kan
  veren 200 kişinin yaklaşık üçte biri} Kovid-19 patojenine karşı
  antikorlara sahipti. Bu insanların yarısı, geçen ay içinde bir soğuk
  algınlığı belirtisi yaşadığını belirtmişti. Boston yakınlarında
  bulunan bir barınaktaki evsizlerin üçte birinden az fazlasının
  testleri pozitif çıkmış, ama
  \href{https://www.wsbtv.com/news/trending/coronavirus-cdc-reviewing-stunning-universal-testing-results-boston-homeless-shelter/ZADQ45HCAZEVJAZA3OTCUR7M6M/}{hiçkimse
  herhangi bir belirti göstermemiştir.}
\item
  İskoçya, (stoklanmış) yoğun bakım yataklarının yarısının
  \href{https://www.heraldscotland.com/news/18377095.coronavirus-scotland-half-icu-beds-empty/}{boş
  kaldığını} bildiriyor. Yetkililere göre, yeni hasta kabulündeki artış,
  bitmeye yüz tutmuş durumdadır.
\item
  Bergamo'daki belediye hastanesinin acil servisi 45 gündür ilk kez bu
  hafta başında \href{https://orf.at/stories/3162642/}{tamamen boştu}.
  Şu sırada, yeniden tedavi edilmeye başlanan farklı hastalıklara sahip
  insanların sayısı ``Kovid-19 hastaları''ndan daha fazladır.
\item
  Lancet tıp dergisindeki bir raporda, korona virüslerini kontrol
  altında tutmak için okulların kapatılmasının
  \href{https://www.thelancet.com/journals/lanchi/article/PIIS2352-4642(20)30095-X/fulltext}{ya
  hiçbir etkisinin olmadığı ya da yalnızca minimal düzeyde etkili
  olduğu} sonucuna varılmıştır.
\item
  Korona enfeksiyonu olan dokuz yaşındaki bir fransız çocuk, 172 kişiyle
  temasta bulunduğu halde,
  \href{https://www.n-tv.de/panorama/172-Kontaktpersonen-von-Corona-verschont-article21727469.html}{hiçbirine
  enfeksiyon bulaşmamıştır}. Bu da hastalığın (gribin aksine) çocuklar
  tarafından bulaştırılmadığı ya da hemen hemen hiç bulaştırılmadığına
  ilişkin daha önceki sonuçları doğruluyor.
\item
  Alman emeritus mikrobiyoloji profesörü Sucharit Bhakdi ile Kovid-19
  konusunda
  \href{https://kenfm.de/kenfm-am-set-gespraech-mit-prof-dr-sucharit-bhakdi-zu-covid-19/}{yeni
  bir 1 saatlik söyleşi yapılmıştır}. Profesör Bhakdi, Kovid-19 salgını
  sırasında çoğu medya kuruluşunun ``tümüyle sorumsuz'' davrandığını
  ileri sürüyor.
\item
  Alman Bakım Etiği Girişimi, ziyaretlerin topyekun yasaklanmasını ve
  bakım evindeki hastalar için acı verici yoğun bakım tedavisini
  \href{http://pflegeethik-initiative.de/2020/04/15/corona-krise-falsche-prioritaeten-gesetzt-und-ethische-prinzipien-verletzt/}{eleştiriyor}:
  ``Korona'dan da önce, Alman bakım evlerinde her gün 900 kişi hastaneye
  kaldırılmaksızın ölüyordu. Gerçekten, bu hastalar için palyatif
  (yatıştırıcı) tedavi uygulanması veya hiç tedavi uygulanmaması daha
  uygun olacaktır. (\ldots{}) Şu ana kadar Korona ile ilgili bütün
  bildiklerimiz ışığında, enfeksiyon korunumunu temel yurttaş
  haklarından üstün tutmayı sürdürmemiz için tek bir ikna edici neden
  yoktur. İnsanlık dışı ziyaret yasaklarını kaldırın!''
\item
  İsviçre'deki St. Gallen kantonunda yaşayan en yaşlı kadın geçen hafta
  109 yaşında öldü. Bu kadın 1918 ``İspanyol gribi'' salgınından sağ
  çıkmıştı, koronaya yakalanmış değildi ve ``yaşına göre çok iyi bir
  durumdaydı''. Buna karşın, ``korona izolasyonu'' onu
  \href{https://swprs.files.wordpress.com/2020/04/tagblatt-109.jpg}{``çok
  fazla etkilemişti''}: ``Ailesinin günlük ziyaretlerinden yoksun sönüp
  gitti.''
\item
  İsviçreli kardiyolog Dr. Nils Kucher'e göre, şu anda İsviçre'deki
  bütün fazladan ölümlerin yaklaşık \%75'i hastanede değil,
  \href{https://www.tagesspiegel.de/wissen/woran-sterben-corona-patienten-wirklich-ein-schweizer-forscher-macht-hoffnung-im-kampf-gegen-covid-19/25750666.html}{evlerde}
  oluyor. Bu ise İsviçre hastanelerinin ve yoğun bakım ünitelerinin
  \href{https://swprs.files.wordpress.com/2020/04/intensivbettenbelegung-schweiz-2020-04-14.png}{büyük
  ölçüde boş kalma} nedenini kesinlikle açıklıyor. Bütün fazladan
  ölümlerin yaklaşık \%50'sinin
  \href{https://www.nzz.ch/zuerich/coronavirus-zuerich-aendert-nun-das-testregime-in-heimenauch-viele-aeltere-covid-19-infizierte-entwickeln-keine-symptome-zuerich-aendert-nun-das-testregime-in-heimen-ld.1552089}{bakım
  evlerinde} olduğu da biliniyor. Dr. Kucher, bu insanların bazılarının
  ani akciğer embolisinden öldüğünden kuşkulanıyor. Bu anlaşılır bir
  durumdur. Yine de bu fazladan ölümlerde ``tecrit''in nasıl bir rol
  oynadığı sorusu ortaya çıkıyor.
\item
  İtaliyan sağlık kurumu ISS, çoğunlukla favizm denilen az rastlanan bir
  genetik metabolik özellik taşıyan ve Akdeniz bölgesinden gelen,
  Kovid-19 hastalarının, ölüme yol açabileceğinden dolayı, klorokin gibi
  sıtma ilaçları ile tedavi edilmemeleri
  \href{https://www.iss.it/en/rapporti-covid-19/-/asset_publisher/btw1J82wtYzH/content/id/5334891}{uyarısında
  bulunuyor}. Bu, hatalı veya aşırı sert ilaçların hastalığı daha
  kötüleştirebildiği konusunda
  \href{https://www.sciencedaily.com/releases/2020/02/200206110703.htm}{yeni
  bir göstergedir}.
\item
  Rubicon:
  \href{https://www.rubikon.news/artikel/120-expertenstimmen-zu-corona}{Korona
  üzerine 120 uzman görüşü}. Dünya çapında, kıdemli biliminsanları,
  doktorlar, avukatlar ve diğer uzmanlar, korona virüsünün ele alınışını
  eleştiriyor. (Almanca)
\end{itemize}

\hypertarget{pandeminin-sux131mux131flandux131rux131lmasux131}{%
\subparagraph{\texorpdfstring{\textbf{Pandeminin
sımıflandırılması}}{Pandeminin sımıflandırılması}}\label{pandeminin-sux131mux131flandux131rux131lmasux131}}

2007 yılında, Amerikalı sağlık yetkilileri, pandemik influenza ve buna
karşı önlemlere ilişkin
\href{https://www.cidrap.umn.edu/news-perspective/2007/02/hhs-ties-pandemic-mitigation-advice-severity}{beş-aşamalı
bir sınıflandırma} tanımladılar. 1. kategoriden (\textless{}\%0,1) 5.
kategoriye (\textgreater{}\%2) uzanan bu beş aşama, pandeminin gözlenen
ölümcüllüğünü (CFR) temel almaktadır. Bu anahtara göre, şu andaki korona
pandemisi muhtemelen 2. kategori (\%0,1'den \%0,5'e) diye
sınıflandırılacaktır. Bu kategori için zamanında esas önlem olarak,
sadece ``hasta bireylerin gönüllü izolasyonu'' öngörülmüştü.

Halbuki, 2009 yılında Dünya Sağlık Örgütü WHO,
\href{https://www.forbes.com/2010/02/05/world-health-organization-swine-flu-pandemic-opinions-contributors-michael-fumento.html\#5ae32fb848e8}{hastalığın
şiddeti kıstasını kendi pandemi tanımından çıkarttı}. Yaklaşık 18 milyar
dolarlık aşı satışı yapılmış olan çok hafif 2009/2010 ``domuz gribi''nde
olduğu gibi, o zamandan beri, ilkesel olarak her küresel influenza
dalgası bir pandemi olarak ilan edilebiliyor..

Dünya Sağlık Örgütü'nün ``domuz gribi'' bağlamındaki kuşkulu rolünü ele
alan TrustWHO (``Trust who?'' ``WHO'ya mı güvenelim?'') adlı belgesel
\href{https://www.youtube.com/watch?v=VjQGyqVN5RM}{VIMEO tarafından bu
yakınlarda silinmiştir}.

\hypertarget{isviuxe7reli-baux15fhekim-pietro-vernazza-basit-uxf6nlemler-yeterlidir}{%
\subparagraph{\texorpdfstring{\textbf{İsviçreli Başhekim Pietro
Vernazza: Basit önlemler
yeterlidir}}{İsviçreli Başhekim Pietro Vernazza: Basit önlemler yeterlidir}}\label{isviuxe7reli-baux15fhekim-pietro-vernazza-basit-uxf6nlemler-yeterlidir}}

İsviçreli enfeksiyonoloji başhekimi Pietro Vernazza,
\href{https://infekt.ch/2020/04/sind-wir-tatsaechlich-im-blindflug/}{konu
ile ilgili sunduğu en son katkıda}, Kovid-19 salgınının ``tecrit''
başlatılmadan da önce zaten kontrol altında olduğunu göstermek için,
Almanya'daki Robert Koch Enstitüsü ve Zürih'teki ETH Üniversitesi'nin
aldığı sonuçları kullanmaktadır. ``Sonuçlar çarpıcı: Her iki çalışma da
belli başlı etkinliklerden vazgeçilmesi ve hijyen kurallarının
uygulamaya sokulması gibi basit önlemlerin çok etkili olduğunu
gösteriyor. Halk bu tavsiyeleri iyi bir biçimde uygulayabilecek haldedir
ve bu önlemler salgını neredeyse durma noktasına getirebilir. Her iki
durumda da, sağlık sistemimizi, hastaneleri aşırı yük altına sokmayacak
biçimde, korumak için bu önlemler yeterlidir.''

\includegraphics{https://swprs.files.wordpress.com/2020/04/ch-reproduktionszahl-eth-infekt.png?w=650\&h=379}

\hypertarget{isviuxe7re-kuxfcmuxfclatif-toplam-uxf6luxfcm-sayux131sux131-normal-aralux131kta}{%
\subparagraph{\texorpdfstring{\textbf{İsviçre: Kümülatif toplam ölüm
sayısı normal
aralıkta}}{İsviçre: Kümülatif toplam ölüm sayısı normal aralıkta}}\label{isviuxe7re-kuxfcmuxfclatif-toplam-uxf6luxfcm-sayux131sux131-normal-aralux131kta}}

İsviçre'de, ilk dört ayın (5 Nisan'a kadar olan) kümülatif toplam ölüm
sayısı,
\href{https://swprs.files.wordpress.com/2020/04/ch-sterblichkeit-kumuliert-q1-2020.pdf}{beklenen
ortalama değerde} olup, beklenen üst değerin en az 1500 altındaydı.
Dahası, Nisan ayının ortasına gelindiğinde, toplam ölüm sayısı, 2015
yılının şiddetli grip mevsimine ait karşılaştırma değerinin hala en az
2000 altındaydı (bkz. aşağıdaki şekil).

\includegraphics{https://swprs.files.wordpress.com/2020/04/schweiz-todesfaelle-2010-2020.png?w=700\&h=339}

\hypertarget{isveuxe7-salgux131n-tecrit-olmadan-da-bitiyor}{%
\subparagraph{\texorpdfstring{\textbf{İsveç: Salgın tecrit olmadan da
bitiyor}}{İsveç: Salgın tecrit olmadan da bitiyor}}\label{isveuxe7-salgux131n-tecrit-olmadan-da-bitiyor}}

Hasta ve ölüm sayılarına ilişkin en son rakamlar, İsveç'te salgının sona
ermekte olduğunu gösteriyor. Baş epidemiyolog, çoğu başka ülkede olduğu
gibi İsveç'te de fazladan ölümlerin en başta yeterince iyi korunmayan
bakımevlerinde olduğunu
\href{https://www.washingtontimes.com/news/2020/apr/15/sweden-coronavirus-rates-easing-despite-loose-rule/}{açıkladı}.

Başka ülkelerle karşılaştırıldığında İsveç halkı, gelecek kış gelmesi
muhtemel bir ``ikinci dalga''dan kendisini daha iyi koruyabilecek daha
yüksek bir Kovid-19 bağışıklığından artık yararlanabilir durumda.

2020 yılı sonunda, Kovid-19'un İsveç'in toplam ölüm sayısında görünür
olmayacağı varsayılabilir. İsveç örneği ``tecritler''in, toplumsal ve
ekonomik açıdan yıkıcı olduğu gibi, tıbbi olarak da gereksiz ya da hatta
ters etkili olduğunu gösteriyor.

\includegraphics{https://swprs.files.wordpress.com/2020/04/sweden-deaths-day-2.png?w=736\&h=293}

\hypertarget{anekdotlar-kanux131tlara-karux15fux131}{%
\subparagraph{\texorpdfstring{\textbf{Anekdotlar kanıtlara
karşı}}{Anekdotlar kanıtlara karşı}}\label{anekdotlar-kanux131tlara-karux15fux131}}

Bilimsel kanıt eksikliği karşısında, bazı medya kuruluşları, halkı korku
içinde tutmak için, tüyler ürpertici anekdotlara gittikçe daha fazla yer
veriyor. Bunun tipik bir örneği, çoğunlukla daha sonra Kovid-19'dan
\href{https://www.dailymail.co.uk/news/article-8193487/Coroner-refuses-rule-COVID-19-cause-death-six-week-old-Connecticut-baby.html}{ölmemiş
olduğu} ortaya çıkan, sözde koronadan ölmüş veya
\href{https://sports.yahoo.com/spanish-football-coach-francisco-garcia-163153573.html}{ciddi
biçimde hastalanmış} ``sağlıklı çocuklar''la ilgilidir.

Akciğere inmiş bir Kovid-19 hastalığından altı hafta sonra, hala düşük
performans gösteren ve tıbbi görüntülemeleri hala dikkat çekici durumda
olan
\href{https://www.rainews.it/tgr/tagesschau/articoli/2020/04/tag-Coronavirus-Lungeschaden-Forschung-Uniklinik-Innsbruck-6708e11e-28dc-4843-a760-e7f926ace61c.html}{bazı
dalgıçlar}, Avusturya medyası tarafından yakın geçmişte haber
yapılmıştı. Haberin bir bölümünde ``geri dönüşsüz tahribat'', bir
sonraki bölümünde ise bunun ``belirsiz ve spekülatif'' olduğu
anlatılıyor. Dalgıçların ciddi bir zatürre geçirdikten sonra genellikle
\href{https://www.deeperblue.com/pulmonary-considerations-in-diving/}{6
ile 12 aylık bir izin} almak zorunda olduğundan ise söz edilmiyor.

Haberlerde çoğunlukla, geçici koku veya tat alma kaybı gibi nörolojik
etkiler olduğu da anlatılıyor. Burada da bunun genelde soğuk algınlığı
ve grip virüslerinin
\href{https://www.ncbi.nlm.nih.gov/pubmed/25294743}{iyi bilinen bir
etkisi} olduğu ve Kovid-19'un
\href{https://www.ncbi.nlm.nih.gov/pubmed/23948436}{bu açıdan oldukça
hafif etki yaptığı} anlatılmıyor.

Başka haberlerde, etkilenen hastaların birçoğunun zaten çok yaşlı olup,
önceden mevcut şiddetli
\href{https://www.epicentro.iss.it/coronavirus/sars-cov-2-decessi-italia}{kronik
sağlık sorunları} olduğundan söz edilmeden, hastalığın böbrekler,
karaciğer veya beyin gibi çeşitli organlar üzerindeki olası etkileri
vurgulanmaktadır.

\hypertarget{politik-guxfcncellemeler-1}{%
\subparagraph{\texorpdfstring{\textbf{Politik
güncellemeler}}{Politik güncellemeler}}\label{politik-guxfcncellemeler-1}}

\begin{itemize}
\tightlist
\item
  WOZ:
  \href{https://www.woz.ch/2016/grundrechte/wenn-die-angst-regiert}{Korku
  hükmettiğinde}. ``Drone'lar, app'lar ve gösteri yasakları: Korona
  krizinin ardından temel haklar erozyona uğratılıyor. Eğer dikkat
  etmezsek, tecrit sonrasında da öyle kalacaklar -- fakat bu aşırı hal
  aynı zamanda umutlanmak için nedenler de sunuyor.''
\item
  Multipolar:
  \href{https://multipolar-magazin.de/artikel/die-massnahmen-wirken}{Gündem
  ne?} ''Hükümet kendini övüyor, sebat üzerine sloganlar atıyor, aynı
  zamanda da virüsün yayılımı ve tehlikesinin güvenilir ölçümünü
  sağlayacak temel verilerin toplanmasını yavaşlatıyor. Yetkililer
  aksine, toplu nabız ölçme ve temas izleme için yeni ``korona app''ları
  gibi kuşkulu araçları yaygınlaştırmakta hızlı ve kararlı hareket
  ediyor.
\item
  Viyana'daki amme hukuku ve hukuk teknolojisi uzmanı Profesör Christian
  Piska, şöyle diyor: ``Avusturya değişti. Birçok kişi bunu öylece kabul
  etmiş görünüyorsa da çok çok değişti. Ekonomi büyüse de büyümese de
  adım adım ve birden bire, diktatoryal rejimlere mükemmelen eşlik
  edecek polis devleti koşulları ile, temel haklarımıza ve insan
  haklarına yönelik şiddetli kısıtlamalar ile yaşıyor olduk. (\ldots{})
  Bir kez açıldı mı
  \href{https://kurier.at/meinung/das-juristische-totschlagargument-vom-menschenleben/400814570}{bir
  daha asla kapanmayabilecek olan} Pandora'nın kutusudur bu.``
\item
  26 ülkeden 300'ü aşkın biliminsanı, verilerin korunumunu ayaklar
  altına alan korona app'larıyla
  \href{https://www.golem.de/news/corona-app-300-wissenschaftler-warnen-vor-zentraler-datenspeicherung-2004-147973.html}{``toplumun
  daha önce görülmemiş bir biçimde gözetlenmesi''}ne karşı uyarıda
  bulunuyor. Birçok biliminsanı ve üniversite, saydamlık eksikliği
  yüzünden, Avrupa temas izleme projesi PEPP-PT'den zaten çekilmiştir.
  Son zamanlarda, Arap ülkelerinde kitlelerin gözetlenmesi için
  sistemler kurmuş olan İsviçre firması AGT'nin bu projede yer aldığı
  ortaya çıkmıştır.
\item
  İsrail'de, yaklaşık 5000 kişi (birbirlerine 2m mesafede durarak),
  Netanyahu hükümetinin
  \href{https://edition.cnn.com/2020/04/20/middleeast/israel-protest-social-distancing-intl/index.html}{aldığı
  önlemlere karşı gösteri yaptı}: ``Korona vakalarının eksponansiyel
  (giderek hızlanan) artışından söz ediyorlar, ama eksponansiyel olarak
  artan tek şey, ülkemizi ve demokrasimizi koruma taraftarı olan
  insanların sayısıdır.''
\item
  Madrid'de çalışan İrlandalı gazeteci Jason O'Toole,
  \href{https://www.rt.com/op-ed/486350-spain-tough-rules-covid-19-lockdown/}{İspanya'daki
  durumu şöyle anlatıyor}: ``Ordunun İspanya sokaklarında görünür hale
  gelmesiyle, yaşanan hali adı konmamış bir sıkıyönetim olarak tarif
  etmemek zor. İspanyol polisinin CCTV (çev. Notu: Kapalı Devre
  Televizyon) kullanarak veya tepelerinde drone'lar uçurarak herkesi
  izlemesiyle, burada George Orwell'in Büyük Birader'i capcanlı ayakta.
  İlk dört haftada, şaşırtıcı bir rakam olan 650.000 kişi ceza aldı ve
  5568 kişi gözaltına alındı. (\ldots{}) Bir polisin, görünüşe göre
  yalnızca elinde ekmekle evine gitmekte olan zihinsel engelli bir genci
  tutuklamak için ağır güç kullandığını gösteren video klibi izlediğimde
  şok geçirdim.''
\item
  OffGuardian:
  \href{https://off-guardian.org/2020/04/18/the-disturbing-developments-in-uk-policing/}{İngiltere'de
  zabıta hizmetindeki rahatsız edici gelişmeler}.
\item
  Amerikalı araştırmacı gazeteci Whitney Webb yeni makalesinde
  \href{https://www.thelastamericanvagabond.com/top-news/techno-tyranny-how-us-national-security-state-using-coronavirus-fulfill-orwellian-vision/}{``ABD
  ulusal Güvenlik Devleti Orwell'ci bir vizyonu gerçekleştirmek üzere
  Korona Virüsünü Nasıl Kullanıyor''} : ``Geçen yıl, bir hükumet
  komisyonu, yapay zeka alanındaki Amerikan hegemonyasının garanti
  altına alınması için, ABD'nin herhangi bir ülkenin sahip olduğunun çok
  daha ötesindeki bir Yapay Zeka temelli kitle gözetleme sistemini
  benimsemesi çağrısında bulundu. Şimdi, bunun uygulanmasını önlediğini
  belirttikleri ``engeller''in çoğu, korona virüsü kriziyle savaş
  kisvesi altında hızla kaldırılıyor.
\item
  Whitney Webb
  \href{https://www.thelastamericanvagabond.com/top-news/all-roads-lead-dark-winter/}{daha
  önceki bir makalesinde}, Johns Hopkins Üniversitesi'ndeki ``Sağlık
  Güvenliği Merkezi''nin, daha önceki pandemi ve biyosilah
  simülasyonlarındaki rolünün ve ABD güvenlik aygıtıyla yakın
  bağlantılarının yanısıra, şu andaki pandemi yönetiminde oynadığı
  merkezi rolü de ele almıştı.
\item
  Küresel gözetleme ve denetim araçlarının yaygınlaştırılması için bir
  pandemi fikrinin kullanılması yeni değildir. Amerikan Rockefeller
  Vakfı, 2010 gibi erken bir tarihte, şu andaki gelişmelerin etkileyici
  bir kesinlikte beklendiği,
  \href{https://swprs.files.wordpress.com/2020/04/rockefeller-foundation-scenarios-2010.pdf}{gelecekteki
  teknolojik ve toplumsal gelişmeleri konu alan çalışma belgesinde} bir
  ``uygun adım yürüme senaryosu'' tarif etmişti. (sayfa: 18).
\item
  \href{https://childrenshealthdefense.org/news/the-truth-about-fauci-featuring-dr-judy-mikovits/}{``Fauci
  hakkındaki gerçek''}: Amerikalı virolog Dr. Judy Mikovits kendisiyle
  yapılan yeni bir söyleşide, şu anda ABD hükumetinin Kovid-19
  önlemlerinin biçimlendirilmesinde büyük bir rol oynamakta olan Dr.
  Anthony Fauci ile ilgili deneyimlerini anlatıyor.
\item
  Yardım kuruluşları, Kovid-19'un kendisinden ölenlerden ``çok daha
  fazla insan''ın
  \href{https://www.welt.de/wirtschaft/article207092745/Corona-Pandemie-Rezession-beschert-der-Welt-die-noch-groessere-Katastrophe.html}{önlemlerin
  ekonomik sonuçlarından öleceği} uyarısında bulunuyor. Şu anda
  tahminler 35 ile 65 milyon insanın mutlak yoksulluk çekeceği ve
  bunların çoğunun da açlık tehdidi altında olduğunu gösteriyor.
\item
  2020 yılında Almanya'da, 2008/2009 finans krizi sonrasındakinin iki
  katından daha fazla olan, 2,35 milyon insanın
  \href{https://www.boeckler.de/pdf/p_wsi_pb_38_2020.pdf}{kısaltılmış
  mesai yaparak çalışacağı} tahmin ediliyor.
\end{itemize}

\includegraphics{https://swprs.files.wordpress.com/2020/04/kurzarbeit-de-corona.png?w=650\&h=461}

\hypertarget{18-nisan-2020}{%
\paragraph{18 Nisan 2020}\label{18-nisan-2020}}

\hypertarget{tux131bbi-guxfcncellemeler-2}{%
\subparagraph{\texorpdfstring{\textbf{Tıbbi
Güncellemeler}}{Tıbbi Güncellemeler}}\label{tux131bbi-guxfcncellemeler-2}}

\begin{itemize}
\tightlist
\item
  Kaliforniya'nın Santa Clara Bölgesi'nde,
  \href{https://www.medrxiv.org/content/10.1101/2020.04.14.20062463v1}{Stanford
  Üniversitesi tarafından yapılan yeni bir serolojik çalışmada,} önceden
  düşünülenden 50 ile 85 kat daha fazla insanda antikorlar bulunmuş,
  \%0,12 ile \%0,2 veya daha da düşük (yani şiddetli bir influenza
  aralığında) bir Kovid-19 ölümcüllüğü sonucu ortaya çıkmıştır. Profesör
  John Ioannidis \href{https://www.youtube.com/watch?v=jGUgrEfSgaU}{yeni
  bir videoda} bu çalışmayı anlatıyor.
\item
  Oxford Üniversitesi'ndeki Kanıta Dayalı Tıp Merkezi'nin (CEBM) yaptığı
  yeni bir analizde, Kovid-19'un ölümcüllüğünün (IFR -- çev. Notu:
  enfeksiyona yakalananların ölüm oranı) \%0,1 ile \%0,36 arasında (yani
  şiddetli bir influenza aralığında) olduğu
  \href{https://www.cebm.net/covid-19/global-covid-19-case-fatality-rates/}{ileri
  sürülüyor}. Önceden ciddi hiçbir sağlık sorunu bulunmayan 70 yaşının
  üzerindeki insanlarda ölüm oranının \%1'den düşük olacağı tahmin
  ediliyor. 80 yaşının üzerindeki insanlarda, ölüm oranı, ölümlerin bu
  hastalık ile ya da bu hastalık yüzünden olduğuna bağlı olarak, \%3 ile
  \%15 arasındadır. Gribin aksine, çocuk ölümleri sıfıra yakındır.
  Araştırma grubu, Kuzey İtalya'daki yüksek ölüm oranları konusunda,
  İtalya'nın Avrupa'da görülen
  \href{https://www.ansa.it/english/news/science_tecnology/2019/11/19/italy-top-in-eu-in-antibiotic-resistance_369e0123-0107-445e-8c17-f11932c9d27c.html}{en
  yüksek antibiyotik direncine} sahip olduğuna dikkat çekiyor. Gerçekten
  de İtalyan yetkililerden gelen veriler, ölenlerin \%80'inin,
  bakteriyel süperenfeksiyonları olduğuna işaret eden antibiyotik
  tedavisi altında olduğunu gösteriyor.
\item
  Helsinki Üniversitesi'nden Finlandiyalı epidemiyoloji profesörü Mikko
  Paunio, birçok uluslararası çalışmaları
  \href{https://lockdownsceptics.org/wp-content/uploads/2020/04/How-the-World-got-Fooled-by-COVID-ed-2c.pdf}{bilimsel
  bir makalede} değerlendirmiş ve Kovid-19 ölümcüllüğünün (IFR) \%0,1
  veya daha az (yani mevsimsel bir influenza civarında) olduğu sonucuna
  varmıştır. Paunio'ya göre, daha yüksek bir ölümcüllük olduğu izlenimi,
  özellikle İtalya ve İspanya'da birden fazla kuşağın bir arada yaşadığı
  evlerde ve New York gibi şehirlerde virüs çok daha hızlı yayıldığından
  dolayı ortaya çıkmıştı. ``Tecrit'' önlemleri çok geç alındı ve etkili
  olmadı.
\item
  İngiltere: Londra'daki geçici Nightingale hastanesi, Paskalya hafta
  sonunda tedavi gören yalnızca 19 hasta ile
  \href{https://www.hsj.co.uk/service-design/exclusive-nightingale-largely-empty-as-icus-handle-surge/7027398.article}{büyük
  ölçüde boş kalmıştır}. Londra'nın mevcut hastaneleri Yoğun Bakım
  kapasitelerini iki kat artırmış olup, şimdilik hasta akınıyla başa
  çıkabiliyor.
\item
  Kanada'da, ``neredeyse tüm bakım evi çalışanları korona virüsünün
  yayılmasından korkarak telaşla orayı terkettikten sonra, 31 kişi
  \href{https://www.nytimes.com/2020/04/16/world/canada/montreal-nursing-homes-coronavirus.html}{bir
  bakım evinde ölmüştür}. Sağlık yetkilileri, Montreal yakınlarında
  Dorval'de bulunan bakım evindeki insanlara günler sonra ulaşmıştı.
  Hayatta kalanların çoğu aç, susuz ve cansız durumdaydı.'' Buna benzer
  trajediler daha önce zaten, panik çıkmasının ve tecrit önlemleri ilan
  edilmesinin ardından, Doğu Avrupalı hemşirelerin telaş içinde ülkeyi
  terk ettikleri
  \href{https://swprs.org/covid-19-a-report-from-italy/}{kuzey İtalya'da
  da olmuştu}.
\item
  Aynı zamanda bakım evlerine de hizmet veren bir İskoçyalı bir doktor
  şöyle
  \href{https://drmalcolmkendrick.org/2020/04/17/care-homes-and-covid19/}{yazmıştı}r:
  ``Bakım evleri için hükümet stratejisi neydi? Şu ana kadar yapılanlar
  durumu çok çok kötüleştirdi.''
\item
  İsviçre'de, Kovid-19'a rağmen, 2020'nin ilk dört ayındaki (5 Nisan'a
  kadar olan) toplam ölüm sayıları
  \href{https://swprs.files.wordpress.com/2020/04/ch-mortalitaet-q1-2020.pdf}{orta
  normal aralığında} idi. Bunun bir nedeni, şimdi Kovid-19 tarafından
  kısmen ``dengelenen'' hafif geçen kışa bağlı hafif geçen grip mevsimi
  olabilir.
\item
  14 Nisan tarihli bir habere göre, İsviçre'deki hastaneler ve hatta
  yoğun bakım üniteleri hala
  \href{https://swprs.files.wordpress.com/2020/04/intensivbettenbelegung-schweiz-2020-04-14.png}{çok
  düşük kapasiteyle çalışmaya} devam ediyor. Bu yine İsviçre'de (yaş
  ortalaması 84 olan) pozitif testli ölümlerin gerçekte tam olarak nerde
  ve nasıl oluştuğu sorusunu ortaya getiriyor.
\item
  Alman Hastaneler Birliği Başkanı
  \href{https://www.bz-berlin.de/deutschland/kliniken-verband-schlaegt-alarm-wegen-corona-regeln}{alarm
  verdi}: Almanya'da planlanmış ameliyatların \%50'sinden fazlası iptal
  olmuştur ve ``biriken ameliyat sıraları'' ise binlere ulaşmıştır. Buna
  ek olarak, korona korkusundan artık hastanelere gitmeye cesaret
  edemedikleri için, kalp krizi ve inme yaşayan hastalar, \%30 ile 40
  daha az tedavi görüyor. Ülke çapında, 150.000 boş hastane yatağı ve
  10.000 boş yoğun bakım yatağı mevcuttur. Berlin'de yalnızca 68 yoğun
  bakım yatağı korona hastaları tarafından kullanılmaktadır ve 1000
  yataklı acil kliniği şu anda kullanım dışıdır.
\item
  Alman yetkililerden gelen yeni veriler, Almanya'da da Kovid-19'un
  üreme hızının tecritten önce kritik değer olan 1'in altına zaten
  düşmüş olduğunu gösteriyor. Bu nedenle genel hijyen önlemleri
  eksponansiyel (hızlanarak artan) yayılımı önlemeye yeterliydi. Zürih
  ETH Üniversitesi tarafından da bu durum İsviçre için zaten ortaya
  konmuştu.
\item
  Bir fransız uçak gemisinde,
  \href{https://www.ouest-france.fr/sante/virus/coronavirus/coronavirus-au-moins-940-marins-positifs-sur-le-charles-de-gaulle-et-son-escorte-6810816}{1081
  askerin testleri pozitif çıkmıştır}. Şu ana kadar \%50 kadarı hiç
  belirti göstermemiş, \%50 kadarı ise hafif belirtiler göstermiştir. 24
  asker hastaneye yatırılmıştır, bunlardan biri de yoğun bakımdadır
  (önceki hastalıkları bilinmiyor).
\item
  Önde gelen Alman virolog Christian Drosten, normal soğuk algınlığı
  korona virüsleriyle temas yoluyla yeni korona virüsüne karşı
  \href{https://www.watson.de/!324026684}{arkaplan bağışıklığı} denilen
  bağışıklığı etkili biçimde geliştirmiş olabileceğini düşünüyor.
\item
  Testleri pozitif çıkmış çok sayıda ölümü incelemiş olan Hamburglu adli
  tıp doktoru Klaus Püschel,
  \href{https://www.abendblatt.de/hamburg/article228908865/hamburg-corona-virus-uke-infektion-covid-19-pueschel-coronavirus-krise-patienten-impfstoff-immunitaet-krankenhaeuser-kontaktverbot-kliniken-infektionsrate-krankheit-pandemie-test-lungenkrankheit-sars-cov-epidemie-sars-cov-2.html}{yeni
  bir makalede şunları açıklamıştır}: ``Rakamlar korona korkusunu haklı
  çıkarmıyor''. Püschel'in bulguları şöyledir: ``Korona görece zararsız
  bir viral hastalıktır. Korona'nın normal bir enfeksiyon olduğu
  gerçeğini ele almalı ve onunla karantinasız yaşamayı öğrenmeliyiz.''
  İncelediği ölümlerin hepsinde önceden var olan o kadar ciddi
  hastalıklar vardı ki ``kulağınıza acımasız gelse de hepsi nasıl olsa
  bu yılın içinde öleceklerdi''. Püschel sözlerine şunları eklemiştir:
  ``Virologların zamanı geçti. Korona krizinde neyin yapılmasının doğru
  olacağını artık başkalarına, örneğin, yoğun bakım doktorlarına
  sormalıyız.''
\item
  \href{https://emedicine.medscape.com/article/227820-overview}{Medscape'te
  yer alan bir inceleme}, korona virüslerinin yol açtığı adi soğuk
  algınlığı enfeksiyonunun normalde -- tecrit olsun olmasın -- Nisan
  sonunda düşüşe geçtiğini gösteriyor.
\item
  İsviçre'de yayınlanan Infosperber dergisi şöyle yazıyor:
  \href{https://www.infosperber.ch/Artikel/Gesundheit/Weniger-Corona-Falle-Einfach-weniger-testen}{``Daha
  az sayıda korona vakası mı istiyorsunuz? Yalnızca daha az test
  yapın!''} Bildirilen gündelik ``yeni vaka'' sayısı, salgının durumuna
  ilişkin pek az bilgi verir. Araştırmacılara göre, testleri pozitif
  çıkmış ölümlerin kümülatif eğrisi ile korkuyu tetiklemenin
  pervasızlıktı.
\item
  OffGuardian:
  \href{https://off-guardian.org/2020/04/17/8-more-experts-questioning-the-coronavirus-panic/}{Sekiz
  uzman daha korona virüsü paniğini sorguluyor}.
\item
  Video: \href{https://www.youtube.com/watch?v=bfN2JWifLCY}{Tecritler
  neden hatalı politikalardır -- İsveçli uzman Profesör Johan Giesecke}.
  İsveçli epidemiyoloji profesörü Johan Giesecke, bunun ``hafif bir
  hastalıktan yaratılmış bir tsunami'' olduğunu söylüyor ve tecritlerin
  ters etki yarattığını düşünüyor. Giesecke'ye göre, en önemli şey risk
  grupları için, özellikle de bakım evleri için, etkin koruma
  sağlamaktır.
\end{itemize}

\hypertarget{kovid-19-ile-suni-solunum}{%
\subparagraph{\texorpdfstring{\textbf{Kovid-19 ile Suni
Solunum}}{Kovid-19 ile Suni Solunum}}\label{kovid-19-ile-suni-solunum}}

Avrupa ve ABD'deki diğer uzmanlar da kritik durumda olan Kovid-19
hastalarının suni solunumla tedavisi konusunda kendi görüşlerini
bildirmiş, sert bir yöntem olan suni solunumun (entübasyon) yapılmaması
konusunda kuvvetle tavsiyede bulunmuştur. Kovid-19 hastaları, akut solum
iflasından (ARDS) değil, virüsün kendisi veya ona karşı bağışıklık
tepkisi ile tetiklenen, oksijen difüzyonu sorununa bağlı bir oksijen
yetmezliğinden muzdariptir.

\begin{itemize}
\tightlist
\item
  AP: \href{https://apnews.com/8ccd325c2be9bf454c2128dcb7bd616d}{Bazı
  doktorlar virüs hastaları için suni solunum cihazı kullanımından
  uzaklaşıyor}
\item
  Video: \href{https://www.youtube.com/watch?v=QPlEUAVjxV8}{Kovid-19:
  Erken Entübasyon Tavsiyesi üzerine Eleştirel Tartışma}
\item
  Video: \href{https://www.youtube.com/watch?v=NmRlvX3VrAQ}{New York'lu
  yoğun bakım doktoru olası bir difüzyon hipoksemisi olarak Kovid-19'u
  anlatıyor}
\item
  Dergi:
  \href{https://link.springer.com/article/10.1007/s00134-020-06033-2}{COVID-19
  zatürresi: farklı fenotipler için farklı solunum tedavileri mi?}
\item
  (Almanca) Die WELT:
  \href{https://www.welt.de/vermischtes/article207221877/Corona-Pandemie-Sterberate-bei-Beatmungspatienten-gibt-Raetsel-auf.html}{Sterberate
  bei Beatmungspatienten gibt Rätsel auf}
\end{itemize}

\hypertarget{politik-guxfcncellemeler-2}{%
\subparagraph{\texorpdfstring{\textbf{Politik
Güncellemeler}}{Politik Güncellemeler}}\label{politik-guxfcncellemeler-2}}

\begin{itemize}
\tightlist
\item
  Video: \href{https://www.youtube.com/watch?v=ZphK_CMUbKg}{Dünyada
  korona tecritleri sırasında polis şiddeti ve izleme faaliyeti}.
\item
  ABD'nin birçok eyaletinde
  \href{https://news.yahoo.com/protests-draw-thousands-over-state-024328374.html}{tecrit
  önlemlerine karşı protesto ve gösteriler olmuştur}.
\item
  Alman ekonomist Norbert Haering
  \href{https://norberthaering.de/}{birkaç makalesinde}, korona
  krizinin'' yolculuk, ödemeler, temas izleme ve biyometri (çev. notu:
  bireylerin fiziksel ve davranışsal özelliklerinin güvenlik amaçlı
  kaydedilmesi) alanlarında bir süredir planlanan, dünya çapındaki
  izleme araçlarının yerleştirilmesi işi için nasıl kullanıldığını
  açıklıyor.
\item
  İtalyan felsefeci Giorgio Agamben
  \href{https://www.nzz.ch/feuilleton/coronavirus-giorgio-agamben-zum-zusammenbruch-der-demokratie-ld.1551896}{Korona
  önlemlerine ilişkin şöyle demektedir}: ``Şu anda bir ülke, daha
  doğrusu bir kültür çöküyor ve kimsenin umurunda değil. Uygarlık
  iddiasındaki ülkelerde gözlerimizin önünde neler oluyor?''
\item
  İtalyan avukatlar hükümetin korona önlemlerine karşı
  \href{https://www.tvprato.it/2020/04/la-camera-civile-degli-avvocati-pratesi-chiede-lannullamento-del-dpcm-del-10-aprile-e-illegittimo/}{bir
  suç duyurusunda bulundu}.
\item
  Alman ekonomi profesörü Stefan Homburg, DIE WELT gazetesinde
  yayınlanan
  \href{https://www.msn.com/de-de/nachrichten/coronavirus/warum-deutschlands-lockdown-falsch-ist-\%E2\%80\%93-und-schweden-vieles-besser-macht/ar-BB12E6km}{``Almanya'nın
  uyguladığı tecrit neden yanlış -- ve İsveç neden çok daha iyi
  durumda''} başlıklı makalesinde şöyle demektedir: ``Özetle, İsveç,
  Güney Kore ya da Tayvan gibi ülkeler tecrit uygulamayarak bilgece
  davrandılar. O ülkelerdeki virologlar, sürekli yön değiştirip halkın
  ve politikacıların dengesini bozmak yerine, onlara kriz sırasında
  istikrarlı bir biçimde rehberlik ettiler. Korona virüsü, insanların
  temel haklarına ve işlerine zarar vermeksizin başarıyla kontrol altına
  alındı. Almanya bu politikayı kendine örnek almalıdır.``
\item
  Bir İsviçre yurttaşı tecritin hemen kaldırılması için Federal İdare
  Mahkemesi'ne ve Federal Konsey'e
  \href{https://faktenb-covid-19-massnahmen.jimdofree.com/}{acil bir
  başvuru yaptı}.
\item
  Video: \href{https://www.youtube.com/watch?v=eU6IdglI-wc}{``İsviçreli
  doktorlar susturuldu, Federal Konsey bölünmüş durumda.''}
  \href{https://www.insidecorona.ch/}{InsideCorona.ch} websitesinin
  kurucusu Dr. Stephan Rietiker ile yapılan bir söyleşi.
\item
  Video: \href{https://www.youtube.com/watch?v=SO2JMkKtq40}{``İsviçre
  Hükümeti hapse atılmayı hakkediyor. Bir polemik.''}
\end{itemize}

\hypertarget{16-nisan-2020}{%
\paragraph{16 Nisan 2020}\label{16-nisan-2020}}

\begin{itemize}
\tightlist
\item
  \emph{London Times} gazetesi şu an İngiltere'de yaşanan fazladan
  ölümlerin \%50'sinin korona virüsünden değil, ama tecrit, genel panik
  ve kısmi sosyal çöküşün etkilerinden kaynaklanabileceğini,
  \href{https://archive.is/2eKCW}{bildiriyor}. Bu ölüm sayısı
  \href{https://www.ons.gov.uk/peoplepopulationandcommunity/birthsdeathsandmarriages/deaths/bulletins/deathsregisteredweeklyinenglandandwalesprovisional/weekending3april2020}{haftada}
  3000'i bulmaktadır. İngiltere'deki korona tanımı (korona virüsünden
  değil de) korona virüslü ölümleri ve ``kuşkulu vakaları'' da içerdiği
  için, aslında bu rakam daha da yüksek olabilir. Buna ek olarak,
  ``korona ölümleri'' `nin yaklaşık \%50'si, genel bir tecritte hiç de
  daha iyi korunmayan, bakım evlerini de
  \href{https://ltccovid.org/2020/04/12/mortality-associated-with-covid-19-outbreaks-in-care-homes-early-international-evidence/}{içermektedir}.
\item
  Danimarka'da tecrit
  \href{https://jyllands-posten.dk/debat/breve/ECE12074246/vi-skulle-aldrig-have-trykket-paa-stopknappen/}{şu
  an pişmanlık uyandırmaktadır}. Aarhus Üniversitesi Hastanesi'nden
  Profesör Jens Otto Lunde Jørgensen'e göre: ``Durdurma düğmesine asla
  basmamalıydık. Danimarka sağlık sistemi durumu kontrol altına almıştı.
  Tam tecrit aşırı bir adımdı.'' Danimarka şu sıra okullarını yeniden
  açmaktadır.
\item
  Tecritin olumsuz sonuçları konusunda ilk uyarıda bulunanlardan biri
  olan Yale Üniversitesi profesörlerinden David Katz, şu anki duruma
  ilişkin \href{https://www.youtube.com/watch?v=VK0Wtjh3HVA}{bir saat
  süren ayrıntılı bir söyleşide bulunmuştur}.
\item
  Alman virolog Hendrik Streeck, süpermarketler, restoranlar veya
  kuaförlerde şu ana kadar hiçbir ``smear enfeksiyonu`` bulunmadığını
  \href{https://today.rtl.lu/news/science-and-environment/a/1498185.html}{açıklamaktadır}.
\item
  İtalya'nın Lombardiya bölgesindeki Robbio belediyesinden gelen yeni
  antikor verileri, hiçbir belirti yaşamadıkları veya hastalığı çok
  hafif belirtilerle geçirdikleri için, başlangıçta düşünülenden
  yaklaşık
  \href{https://www.tgcom24.mediaset.it/cronaca/a-robbio-pv-il-22-ha-o-ha-avuto-il-coronavirus-ok-del-sindaco-ai-test-per-tutti_17285128-202002a.shtml}{on
  kat daha fazla insanın} korona virüsü kaptığını gösteriyor.
  Gerçekleşen bağışıklık geliştirme oranı \%22'dir.
\item
  İsviçre'nin Zürih kantonundan gelen yeni veriler,
  \href{https://www.srf.ch/news/regional/zuerich-schaffhausen/corona-uebersicht-zh-und-sh-schlieren-als-zentrum-der-forschung-im-kampf-gegen-corona}{shows
  that} Kovid-19'a bağlı ölümlerin yaklaşık \%50'sinin huzur evlerinde
  veya bakım evlerinde gerçekleştiğini gösteriyor. Yine de oralarda bile
  testleri pozitif çıkan insanların yaklaşık \%40'ı hiçbir belirti
  göstermemiştir. İsviçre'de testleri pozitif çıkıp ölenlerin yaş
  ortalaması şu an 84'tür.
\item
  İsviçre enfeksiyonoloji baş hekimi Pietro Vernazza,
  \href{https://infekt.ch/2020/04/exitstrategie-lockdown/}{``virüsle
  yaşama'' stratejisi} konusunda yaptığı yorumda, başka şeylerin
  yanısıra, riskli insanların bireysel olarak en iyi biçimde korunmasını
  da öneriyor ve genel nüfusun bağışıklık geliştirmesinin riskli
  insanlar için de bir koruma olduğunu, söylüyor.
\item
  İngiltere'deki bir websitesi olan
  \href{https://lockdownsceptics.org/}{Lockdown Skeptics}, Kovid-19'a
  ile ilgili olarak alınan önlemleri ve genel medya haberlerini
  eleştiriyor.
\item
  Avusturyalı sivil toplum kuruluşu
  \href{https://www.initiative-corona.info/}{``Kanıta-dayalı korona
  bilgilendirme girişimi''} yeni korona virüsüne ilişkin çalışmalara ve
  analizlere bir genel bakış sunmaktadır.
\item
  Almanca Belgesel:
  ``\href{https://www.youtube.com/watch?v=dYlia_fQOLk}{Die WHO -- Im
  Griff der
  Lobbyisten}\href{https://www.youtube.com/watch?v=dYlia_fQOLk}{``} (WHO
  (Dünya Sağlık Örgütü) -- Lobilerin Kıskacında) (ARTE televizyonu,
  2017)
\end{itemize}

\hypertarget{15-nisan-2020}{%
\paragraph{15 Nisan 2020}\label{15-nisan-2020}}

\hypertarget{tux131bbi-guxfcncellemeler-3}{%
\subparagraph{\texorpdfstring{\textbf{Tıbbi
Güncellemeler}}{Tıbbi Güncellemeler}}\label{tux131bbi-guxfcncellemeler-3}}

\begin{itemize}
\tightlist
\item
  Almanya'nın önde gelen mikrobiyologlarından ve epidemiyologlarından
  Profesör Alexander Kekulé, İngiltere'nin Telegraph gazetesine verdiği
  demeçte, virüsten daha fazla zarara yol açtığı için,
  \href{https://www.telegraph.co.uk/news/2020/04/11/german-scientist-predicted-european-epidemic-calls-end-lockdown/}{tecrite
  son verilmesi çağrısında bulunuyor}. 50 yaşının altındaki insanlarda,
  bu virüs nedeniyle şiddetli hastalık ve ölümler görülmesi ``çok çok
  uzak bir olasılıktır''. Risk altındaki gruplar korunurken, nüfusun
  geneli hızla bağışıklık geliştirmelidir. Profesör Kekulé, hazır olması
  en az altı ile oniki ay alacak bir aşı için beklemenin de mümkün
  olmadığını, ama bu virüsle yaşamanın bir yolunun bulunması gerektiğini
  söylemiştir.
\item
  Almanya'daki Kanıta-Dayalı Tıp Ağı'na göre, Robert Koch Enstitüsü,
  2017/2018'deki gibi şiddetli bir mevsimsel influenzanın (grip)
  ölümcüllüğünün 0,4\% ile 0,5\% arasında olduğunu
  \href{https://www.ebm-netzwerk.de/en/publications/covid-19}{hesaplamıştı},
  bu da daha önce varsayılanın 0,1\%'i bile değildi. Bunun anlamı,
  Kovid-19 daha hızlı yayılabiliyor olsa bile, ölümcüllüğünün güçlü bir
  mevsimsel gribinkinden daha da düşük olabileceğidir.
\item
  Lüksemburg'da yayınlanan Tageblatt gazetesinin bir
  \href{https://swprs.files.wordpress.com/2020/04/volksblatt_schweden_corona_20200414_18.pdf}{haberine
  göre}, ``İsveç'in Kovid-19'a ilişkin gevşek stratejisi işe yarıyor
  gibi görünüyor''. En alt seviyede önlem alındığı halde, durum görünüşe
  göre, ``şu anda net bir biçimde sakinleşiyor''. Stockholm yakınlarında
  kurulmuş olan dev sahra hastanesi talep yokluğundan kapalı duruyor.
  Yoğun bakım ünitelerindeki hasta sayısı düşük bir düzeyde sabitlenmiş,
  hatta hafifçe düşüşe geçmiştir. Karolinska Klinik'te çalışan kıdemli
  bir doktor şöyle açıklamıştır: ``Bütün Stockholm hastanelerinde yoğun
  bakım ünitelerinde birçok boş yer var. Hastalık eğrisinin düzleşme
  evresine yaklaşıyoruz.'' Şu ana kadar İsveç'te Kovid-19'lu 900 ölüm
  olmuştur.
\item
  Birleşik Krallık (tecrit uygulanan) ile İsveç (tecrit uygulanmayan)
  arasında doğrudan bir karşılaştırma, iki ülkenin nüfuslarına oranla
  vaka sayıları ve ölümler açısından
  \href{http://www.theblogmire.com/a-comparison-of-lockdown-uk-with-non-lockdown-sweden/}{neredeyse
  benzer} olduklarını gösteriyor.
\item
  New England Journal of Medicine dergisine yazılmış bir mektupta,
  hamile kadınlar üzerinde yapılmış bir çalışmanın, testleri pozitif
  çıkan kadınların \%88'inin
  \href{https://www.nejm.org/doi/full/10.1056/NEJMc2009316}{hastalık
  belirtisi olmadığını gösterdiği} bildirilmektedir. Bu rakam oldukça
  yüksektir, ama daha önce Çin ve İzlanda'dan gelen raporlarla
  uyumludur.
\item
  Tel Aviv Üniversitesi Epidemiyoloji Araştırma Laboratuvarı müdürü
  Profesör Dan Yamin,
  \href{https://www.ynet.co.il/articles/0,7340,L-5714371,00.html}{kendisiyle
  yapılan bir söyleşide şunları anlatıyor}: Yeni korona virüsü, toplumun
  geniş bir kesimi için ``neredeyse hiç tehlikeli değildir'' ve hızla
  doğal bağışıklık geliştirilmesi hedeflenmelidir. Para, tecrit yüzünden
  oluşan zararları ödemektense, bir kliniğin genişletilmesi için
  harcansa daha iyi olur.
\item
  İsrail Ulusal Araştırma Konseyi başkanı Profesör Isaac Ben-Israel, şu
  anki bulgulara göre, korona salgınının çoğu ülkede
  \href{http://www.israelnationalnews.com/News/News.aspx/278658}{yaklaşık
  8 hafta sonra}, alınan önlemlerden bağımsız olarak bittiğini ileri
  sürmektedir. Bu nedenle de ``tecrit'' `e acilen son verilmesini
  tavsiye etmektedir.
\item
  İngiliz istatistik profesörü David Spiegelhalter, Kovid-19'dan ölüm
  riskinin kabaca
  \href{https://medium.com/wintoncentre/how-much-normal-risk-does-covid-represent-4539118e1196}{normal
  ölüm sayılarına} denk geldiğini ve yalnızca yaklaşık 70 ile 80 yaş
  grubu arasında görünür şekilde arttığını göstermiştir (bkz. bağlantısı
  verilen makalenin sonundaki grafik).
\item
  Zürih Üniversitesi Viroloji Enstitüsü emeritus müdürü olan ve
  \href{https://www.rubikon.news/artikel/die-stimme-der-vernunft}{abartılı
  önlemleri ilk eleştirenlerden} Profesör Karin Mölling,
  \href{https://www.youtube.com/watch?v=4rl2sqLcDoQ}{kendisiyle yapılan
  yeni bir söyleşide} hava kirliliği ve nüfus yoğunluğu gibi yerele özel
  etkenlerin rolünü vurgulamaktadır.
\item
  İngiliz gazetesi Guardian, Çin'in kentlerinde 2015 yılında aşırı hava
  kirliliğinden günde 4000 kişinin öldüğüne
  \href{https://www.theguardian.com/world/2015/aug/14/air-pollution-in-china-is-killing-4000-people-every-day-a-new-study-finds}{işaret
  etmiştir}. Bu rakam Çin'in şu ana kadar bildirdiği toplam Kovid-19
  ölümlerinden daha fazladır.
\item
  Alman virolog Hendrik Streeck, yaptığı pilot çalışmaya yönelik
  eleştirilere karşı
  \href{https://www.tagesspiegel.de/wissen/virologe-streeck-zur-coronavirus-studie-die-veroeffentlichung-zu-heinsberg-war-nicht-leichtfertig/25735672.html}{kendisini
  savunmuştur}. Streeck, şiddetli bir mevsimsel gribe denk düşen \%0,37
  (vakalara göre) ölümcüllük ve \%0,06 (nüfusa göre) ölüm oranı
  bulmuştur.
\item
  Avusturya'daki iç hastalıkları uzmanları,
  \href{https://wien.orf.at/stories/3044064}{``ikincil hasarlar''
  konusunda uyarıda bulunuyor}: Korona virüsü yüzünden, kontrol ve
  ameliyat tarihleri ertelenmiştir, daha az sayıda hasta kalp krizi
  belirtileriyle hastanelere gelmektedir.
\item
  İsviçreli bir biyofizikçi Mart ayının başından bu yana, İsviçre'de
  pozitif çıkan Kovid-19 testlerinin artış hızını ilk kez
  \href{https://swprs.org/rate-of-positive-covid19-tests/}{grafik olarak
  göstermiştir}. Sonuç, pozitiflerin oranının \%10 ile 25 arasında gidip
  geldiğini ve ``tecrit'' `in önemli hiçbir etki yaratmadığını
  gösteriyor (bkz. aşağıdaki grafik). İlginç olan, İsviçreli
  yetkililerin ve medyanın bu grafiği hiç göstermemiş olmasıdır.
\item
  İsviçreli bir araştırmacı Federal Halk Sağlığı Ofisi'nin en son
  Kovid-19 raporunun analizini yapmış ve yine
  \href{https://covid-19-fakten.blogspot.com/2020/04/der-bag-situationsbericht-vom-1442020.html}{eleştirel
  bir değerlendirme} sunmuştur: ``Bu durum raporu, politikacılara ve
  yeterli kararlar almaya uygun değil, gayet belirsiz ve eksik,
  bilgilendirici değeri yok.''
\item
  İsviçreli enfeksiyonoloji başhekimi Dr. Pietro Vernazza, yeni bir
  makalede
  \href{https://infekt.ch/2020/04/hinterlaesst-coronavirus-eine-immunitaet/}{şu
  açıklamada bulunmuştur}: Kovid-19'a karşı sözde bağışıklık
  oluşturamama meselesi, ``daha yakından incelendiğinde sorun
  oluşturmayan'', ``nadir görülen bireysel vakalar, hatta sadece önemsiz
  şüphelerdir'', fakat bazı medya kuruluşlarınca ``abartılmış ve
  alelacele şok haber diye servis edilmiştir''.
\item
  Fransa'da, korona virüsü korkusu veya birine korona virüsü bulaştırmış
  olma korkusuna bağlı olarak gittikçe artan sayıda
  \href{https://www.midilibre.fr/2020/04/09/coronavirus-ces-suicides-de-malades-ou-de-personnes-tenaillees-par-langoisse,8839373.php}{intihar
  bildirilmektedir}.
\item
  Yeni Fransızca site \href{https://covidinfos.net/}{Covid Infos},
  Kovid-19'u ve medya haberlerini eleştirel bir incelemeye tabi tutuyor.
\end{itemize}

\includegraphics{https://swprs.files.wordpress.com/2020/04/fs-ch-pos-rate.png?w=600\&h=383}

\hypertarget{abd-ve-ingiltere}{%
\subparagraph{\texorpdfstring{\textbf{ABD ve
İngiltere}}{ABD ve İngiltere}}\label{abd-ve-ingiltere}}

\begin{itemize}
\tightlist
\item
  ABD'nin savaş gemisi Theodore Roosevelt'teki 600 denizciye yapılan
  Kovid-19 testleri pozitif çıkmış, olaydan bu yana Kovid-19 yüzünden ya
  da Kovid-19 taşıyıcısı olarak
  \href{https://www.theguardian.com/world/2020/apr/14/sailor-dies-from-covid-19-and-600-test-positive-after-outbreak-on-uss-theodore-roosevelt-guam}{ilk
  denizci ölmüştür}. Savaş gemisi, 65 yaşının altındaki sağlıklı genel
  nüfus üzerindeki etkiler açısından çok önemli bir ``vaka çalışması''
  olacaktır.
\item
  İngiliz patoloji emeritus profesörü Dr. John Lee,
  \href{https://www.spectator.co.uk/article/to-understand-covid-we-need-evidence-scepticism-and-vigorous-debate}{``büyük
  yanlışlardan sakınmak''} için canlı ve kanıta dayalı bir tartışma
  gerektiğini ileri sürmekte, hükümetlerin ve medyanın kullandığı
  rakamların çoğunun güvenilir olmadığını söylemektedir.
\item
  İngiltere'de, hastane yataklarının \%40'ı
  \href{https://www.hsj.co.uk/acute-care/nhs-hospitals-have-four-times-more-empty-beds-than-normal/7027392.article}{şu
  anda boş olup}, bu oran her zamankinden dört kat daha fazladır. Bunun
  nedeni genel hasta kabulündeki keskin düşüştür. Kapasitesi artırılmış
  olan yoğun bakım yatakları, ortalama \%78 oranında doludur. Buna ek
  olarak, hemşirelerin \%10'u karantinadadır.
\item
  New York yakınlarındaki ABD ordusuna ait geçici korona hastaneleri şu
  ana kadar
  \href{https://nypost.com/2020/04/09/usns-comfort-and-javits-center-mostly-empty-amid-coronavirus/}{``büyük
  ölçüde boş kalmıştır''}. New York'ta hastaneye yatırılma oranı
  \href{https://www.nytimes.com/2020/04/10/nyregion/new-york-coronavirus-hospitals.html}{yedi
  kat} fazla tahmin edilmişti.
\item
  ABD'de yapılan bir çalışmada, yeni korona virüsünün başlangıçta
  varsayılandan çok daha fazla yayıldığı, fakat çoğu insanda ya hiç
  belirtiye yol açmadığı ya da çok hafif belirtilere neden olduğu, bu
  nedenle de ölümcüllük oranının kabaca mevsimsel gribe eşit olan, 0,1\%
  kadar düşük olabildiği
  \href{https://www.medrxiv.org/content/10.1101/2020.04.01.20050542v1}{sonucuna
  varılmıştır}. Halbuki, hastalığın daha kolay bulaşması yüzünden,
  örneğin New York'taki vakalar,
  \href{https://archive.is/7w2XE}{normalden daha kısa sürede} ortaya
  çıkmıştır.
\item
  Doğu Virginia Tıp Fakültesi zatürre ve yoğun bakım şefi, Kovid-19
  hastalarının tedavisine ilişkin
  \href{https://www.evms.edu/media/evms_public/departments/internal_medicine/EVMS_Critical_Care_COVID-19_Protocol.pdf}{yeni
  bir belgede} şöyle demektedir: ``Kovid-19'un bildiğiniz `tipik ARDS'ye
  (akciğer iflası) yol açmadığını kabul etmek önemlidir\ldots{} bu
  hastalık farklı tedavi edilmelidir. Suni solunum cihazı kaynaklı
  akciğer hasarına yol açarak bu durumu daha kötüleştiriyor
  olabiliriz.''
\item
  ABD'de, bir vali, bir bebeğin dünyadaki en genç kurban olarak,
  ``Korona'dan'' öldüğünü iddia etmişti. Ailenin tanıdıkları ise bebeğin
  evdeki trajik bir kazada boğulduğunu ve sonradan hastanede yapılan
  testin pozitif çıktığını
  \href{https://www.washingtonexaminer.com/news/candace-owens-accuses-connecticut-governor-of-lying-about-coronavirus-death-calls-for-resignation}{belirtmişti.}
  Nedeni bilinmeyen ve kuşkulu ölümleri soruşturmaktan sorumlu memur,
  bunu bir Kovid ölümü olarak
  \href{https://www.dailymail.co.uk/news/article-8193487/Coroner-refuses-rule-COVID-19-cause-death-six-week-old-Connecticut-baby.html}{açıklamamıştır}.
\item
  ABD'nin Montana eyaletinden bir doktor
  \href{https://www.youtube.com/watch?v=V0lIWZpiRU0}{yaptığı konuşmada},
  yeni kurallara göre Kovid-19 olduğu kuşkulu vakalara verilen ölüm
  belgelerinin nasıl manipüle edilmekte olduğunu anlatmıştır.
\end{itemize}

\hypertarget{bakux131m-evleri}{%
\subparagraph{\texorpdfstring{\textbf{Bakım
Evleri}}{Bakım Evleri}}\label{bakux131m-evleri}}

\begin{itemize}
\tightlist
\item
  Beş Avrupa ülkesinden gelen verilerin bir analizi, bakım evi
  sakinlerinin şu ana kadar,
  \href{https://ltccovid.org/2020/04/12/mortality-associated-with-covid-19-outbreaks-in-care-homes-early-international-evidence/}{tüm
  `Kovid-19 ölümleri'nin \%42 ile \%57 arasındaki bölümünü}
  oluşturduğunu gösteriyor. Aynı zamanda, ABD'de yapılan üç ayrı
  çalışma, testleri pozitif çıkan tüm bakım evi sakinlerinin \%50'ye
  varan bölümünün testler yapılırken (henüz) belirti göstermediğini
  ortaya koymuştur. Bundan iki sonuç çıkartılabilir: Bir yandan, yeni
  korona virüsünün tehlikesi -- zaten kuşkulanıldığı gibi -- daha da iyi
  korunması gereken küçük ve çok kırılgan bir nüfus grubunda
  yoğunlaşıyor. Diğer yandan, bu insanların bazılarının ölmeyebileceği,
  veya yalnızca korona yüzünden değil, ama aynı zamanda mevcut durumla
  bağlantılı aşırı düzeydeki stresten ölebileceği düşünülebilir. Almanya
  ve İtalya'dan gelen haberler zaten belirti göstermeksizin aniden ölen
  bakım evi sakinleri olduğundan söz ediyordu.
\item
  Bir alman palyatif (yatıştırıcı) tıp uzmanı
  \href{https://www.deutschlandfunk.de/palliativmediziner-zu-covid-19-behandlungen-sehr-falsche.694.de.html?dram:article_id=474488}{kendisiyle
  bu yakınlarda}
  \href{https://www.deutschlandfunk.de/palliativmediziner-zu-covid-19-behandlungen-sehr-falsche.694.de.html?dram:article_id=474488}{yapılan
  bir söyleşide}, Kovid-19 hastalarının tedavisinde ``çok yanlış
  öncelikler belirlendiğini ve tüm etik ilkelerin çiğnendiğini'' ileri
  sürmektedir. ``Fayda zarar dengesi'' genellikle iyi olmasa da ``yoğun
  bakım lehine aşırı tek yanlı bir yönelim'' vardır. Yeni bir tanı (yani
  Kovid-19), geçmişte çoğunlukla palyatif olarak ele alınan yaşlı
  hastaları, yoğun bakım hastalarına dönüştürecek ve onları, can yakıcı,
  ama çoğunlukla umutsuz bir tedaviye (yani suni solunuma) maruz
  bırakacaktır. Bu uzmana göre, tedavi daima hastanın onayıyla
  yapılmalıdır.
\end{itemize}

\includegraphics{https://swprs.files.wordpress.com/2020/04/c19-nursing-homes.jpg?w=700\&h=218}

\hypertarget{politik-geliux15fmeler}{%
\subparagraph{\texorpdfstring{\textbf{Politik
Gelişmeler}}{Politik Gelişmeler}}\label{politik-geliux15fmeler}}

\begin{itemize}
\tightlist
\item
  Almanya'da, Federal Anayasa Mahkemesi'nde korona önlemlerine karşı bir
  suç duyurusu yapan ve gösteriler yapılması çağrısında bulunan
  \href{http://beatebahner.de/}{tıp alanında uzman bir avukat}
  tutuklanmış ve iki günlüğüne
  \href{https://www.rnz.de/nachrichten/heidelberg_artikel,-nach-aufruf-zu-corona-demo-heidelberger-anwaeltin-in-psychiatrischer-einrichtung-update-_arid,508747.html}{bir
  hapishanenin psikiyatri koğuşuna gönderilmiştir}. Savcı ``halkı suç
  işlemeye tahrik'' `ten soruşturma açmıştır. Bir başka avukat ise
  Almanya Federal Barosu'na yazdığı bir
  \href{https://www.nachrichtenspiegel.de/2020/04/14/brief-an-die-bundesrechtsanwaltskammer-in-causa-bahmer/}{açık
  mektupta şu soruyu sormaktadır}: ``Protesto yüzünden psikiyatri
  hastanelerine gönderilen avukatlar? Almanya'da yine o zamanlar mı
  geldi?
\item
  İsviçre'de ``korona eleştirisi yapan'' bir doktor sözde ``akrabalara
  ve polise yönelik tehditler'' yüzünden, özel bir polis birimince
  tutuklanmış ve
  \href{https://www.srf.ch/news/regional/aargau-solothurn/festnahme-von-corona-kritiker-verschwoerung-oder-normale-intervention-der-aargauer-behoerden}{bir
  psikiyatri kliniğine gönderilmiştir}. Sonra ailesi akrabalara yönelik
  hiçbir tehdit olmadığını açıklamıştır. Doktor da sorgu sırasında
  ``yetkilere yönelik tehditler'' nedeniyle suçlanmadığını belirtmiştir.
  Polis, doktorun silahlı olduğunu varsayarak özel birimin harekete
  geçirilişini haklı göstermiştir, fakat bu İsviçre ordusuna ait
  mühimmatsız bir sıhhiye tabancasıydı. Doktorun bir psikiyatri
  kliniğine götürülmesi, (emzikli annelerde olduğu gibi) sözde ``hapis
  cezası muafiyeti'' `ne dayanarak haklı gösterilmişti ki bu bile bir
  bahane olarak görülebilir. Şu anki bilgimize dayanarak, alınan önlemin
  gerçekten de en azından kısmen politik motivasyonla alındığını
  söylemek bu nedenle mümkündür. ABD eski kongre üyesi Cynthia McKinney,
  eski Sovyetler Birliği'ndeki uygulamaları çağrıştıran İsviçre'deki bu
  olaya daha önce
  \href{https://twitter.com/cynthiamckinney/status/1250075810838581248}{dikkat
  çekmişti}.
\item
  İtalya tecrit süresince halkı izlemek için artık
  \href{https://www.ansa.it/english/news/2020/04/06/coronavirus-italy-activates-satellite-to-monitor-nation-3_f2ffb30c-d550-42f5-82fc-ec1f82c5c625.html}{Avrupa
  uydu verilerini} kullanmaktadır.
\item
  İngiliz polisi ``sosyal buluşmalar'' peşindeyken,
  \href{https://twitter.com/BanTheBBC/status/1249598512427347969}{bir
  ikametgahın kapısını kırmıştır}.
\end{itemize}

\hypertarget{12-nisan-2020}{%
\paragraph{12 Nisan 2020}\label{12-nisan-2020}}

\hypertarget{yeni-uxe7alux131ux15fmalar}{%
\subparagraph{\texorpdfstring{\textbf{Yeni
çalışmalar}}{Yeni çalışmalar}}\label{yeni-uxe7alux131ux15fmalar}}

\begin{itemize}
\tightlist
\item
  Stanford Üniversitesi tıp fakültesinden profesör John Ioannidis,
  \href{https://www.medrxiv.org/content/10.1101/2020.04.05.20054361v1}{yeni
  bir çalışmada}, 65 yaşının altındaki insanlar için, dünyada hastalığın
  ``en yoğun'' görüldüğü yerlerde bile, Kovid-19'dan ölme riskinin,
  günde 15-650 km yol katedenlerin ölümlü bir trafik kazasına kurban
  gitme riskine eşit olduğu sonucuna varmıştır.
\item
  Alman virolog Hendrick Streeck, yürütülen bir
  \href{https://www.land.nrw/sites/default/files/asset/document/zwischenergebnis_covid19_case_study_gangelt_0.pdf}{serolojik
  pilot çalışmada}, Kovid-19'un ölümcüllüğünün \%0.37, (toplam nüfusa
  göre) ölüm oranının ise \%0,06 olduğu ara sonucuna varmıştır. Bu
  değerler, Dünya Sağlık Örgütü (WHO) değerlerinden 10 kat, Johns
  Hopkins Üniversitesi'ninkilerden 5 kat daha düşüktür.
\item
  Danimarka'da kan bağışı yapan 1.500 kişiyle ilgili bir çalışmada,
  Kovid-19'un ölümcüllüğünün
  \href{https://www.dr.dk/nyheder/indland/doedelighed-skal-formentlig-taelles-i-promiller-danske-blodproever-kaster-nyt-lys}{yalnızca
  binde 1,6}, yani, başlangıçta WHO tarafından varsayılan değerden en az
  20 kat daha düşük ve güçlü bir grip (salgını) aralığında olduğu
  bulunmuştur. Aynı zamanda Danimarka, gelecek hafta okulları ve
  anaokullarını yeniden açmaya
  \href{https://www.thelocal.dk/20200406/denmark-to-reopen-schools-and-kindergartens-next-week}{karar
  vermiştir}.
\item
  ABD'nin Kolorado eyaletinde yapılan bir serolojik çalışmanın
  \href{https://reason.com/2020/04/08/mass-antibody-testing-in-this-rural-colorado-county-sheds-light-on-covid-19s-prevalence-and-lethality/}{ilk
  sonuçlarına göre}, Kovid-19'un ölümcüllüğü, 5-20 kat abartılı tahmin
  edilmiş olup, normal bir grip ile, bir grip salgını aralığında
  kalacağı muhtemeldir.
\item
  Viyana Tıp Fakültesi tarafından yürütülen bir çalışma, Kovid-19
  ölümlerinin yaş ve risk profilinin
  \href{https://www.vienna.at/analyse-zeigt-covid-19-opferkurve-entspricht-normaler-mortalitaet/6581246}{normal
  ölüm sayılarına benzer} olduğu sonucuna varmıştır.
\item
  Tıbbi Viroloji Dergisi'nde (Journal of Medical Virology) yayınlanan
  bir çalışma, uluslararası olarak kullanılan korona virüsü testlerinin
  \href{https://www.ncbi.nlm.nih.gov/pubmed/32219885}{güvenilmez olduğu}
  sonucuna varmıştır: Önceden bilinen hatalı pozitif sonuçlar sorununa
  ek olarak, bir de ``potansiyel olarak yüksek'' hatalı negatif sonuçlar
  oranı vardır, yani, testler bazı hastalarda bir kez çalışıp, sonra
  çalışmazken, belirtiler gösteren bireylerde bile yanıt vermemektedir.
  Bu durum, başka grip benzeri hastalıkları, sonuçların dışında tutmayı
  daha da zorlaştırmaktadır.
\item
  İsviçreli bir biyofizikçi, ABD, Almanya ve İsviçre'deki pozitif
  testlerin artış hızını şu ana kadar ilk kez değerlendirmiş ve
  \href{https://swprs.org/rate-of-positive-covid19-tests/}{grafik olarak
  göstermiştir}. Sonuçta, bu ülkelerdeki pozitiflerin sayısı
  eksponansiyel (giderek hızla artan) değil, yalnızca hafifçe
  yükselmektedir.
\item
  ABD'li araştırmacılar, yerel hava kirliliğinin Kovid-19'dan ölme
  riskini
  \href{https://www.medrxiv.org/content/10.1101/2020.04.05.20054502v1}{büyük
  ölçüde artırdığı} sonucuna varmıştır. Bu da daha önce İtalya ve Çin'de
  yapılmış olan çalışmaları doğrulamaktadır.
\item
  WHO, Mart ayı sonunda, daha önceki varsayımların aksine, Kovid-19'un
  (``havadan'')
  \href{https://www.who.int/news-room/commentaries/detail/modes-of-transmission-of-virus-causing-covid-19-implications-for-ipc-precaution-recommendations}{aerozollerle
  bulaşmadığı} sonucuna varmıştır. Bulaşma esasen doğrudan dokunma veya
  damlacık enfeksiyonu (öksürme, hapşırma) yoluyla olmaktadır.
\item
  Alman-Amerikalı epidemiyoloji profesörü Knut Wittkowski,
  \href{https://www.youtube.com/watch?v=ARTf4bpiXuI}{yeni bir
  söyleşide}, Kovid-19 salgınının birçok ülkede çoktan düşüşe geçmiş ya
  da hatta ``çoktan bitmiş'' olduğunu'', sokağa çıkma yasaklarının da
  çok geç geldiğini ve ters etki yaptığını ileri sürmektedir.
\end{itemize}

\hypertarget{avrupa-uxf6luxfcm-sayux131larux131-izleme}{%
\subparagraph{\texorpdfstring{\textbf{Avrupa Ölüm Sayıları
İzleme}}{Avrupa Ölüm Sayıları İzleme}}\label{avrupa-uxf6luxfcm-sayux131larux131-izleme}}

\href{https://www.euromomo.eu/outputs/zscore_country_total.html}{Avrupa
ölüm sayıları izleme}, bir takım Avrupa ülkelerinde şimdi, 65 yaş-üstü
gruptaki ölümlerin sayısında net bir artış tahmini gösteriyor. Almanya
ve Avusturya dahil bazı ülkelerde ise bu yaş grubundaki ölüm sayıları
hala normal aralıktadır (hatta normalin daha altındadır).

Kısmen artan ölüm sayılarının tek başına korona virüsünden mi yoksa
(örneğin, izolasyon, stres, iptal edilen ameliyatlar, vb.) bazen aşırıya
kaçan önlemlerden de mi kaynaklandığı ve ölüm sayılarının yıllık olarak
bakıldığında yine de artmış olup olmayacağı sorusu halen cevaplanmayı
bekliyor.

65 yaş-altı gruplar arasında, şu ana kadar yalnızca İngiltere'deki ölüm
sayılarında, daha önceki grip dalgalarına oranla daha fazla bir tahmini
artış vardır. Testleri pozitif çıkmış olarak ölenlerin yaş ortalaması,
İtalya'da 80, Almanya'da 83, İsviçre'de 84'tür.

\hypertarget{isviuxe7re}{%
\subparagraph{\texorpdfstring{\textbf{İsviçre}}{İsviçre}}\label{isviuxe7re}}

\begin{itemize}
\tightlist
\item
  Federal Halk Sağlığı Ofisi'nin
  \href{https://www.bag.admin.ch/bag/de/home/krankheiten/ausbrueche-epidemien-pandemien/aktuelle-ausbrueche-epidemien/novel-cov/situation-schweiz-und-international.html}{son
  raporu} na göre, testleri pozitif çıkmış olarak ölenlerin yaş
  ortalaması şimdi 84'tür. Hastaneye yatırılmış hastaların sayısı ise
  sabit kalmıştır.
\item
  Zürih'teki ETH üniversitesinde yapılan bir çalışmada, İsviçre'deki
  enfeksiyon oranının, tahminen genel hijyen ve gündelik önlemlere bağlı
  olarak, ``tecrit'' ilanından birkaç gün önce 1 sabit değerine düştüğü
  \href{https://www.tagesanzeiger.ch/ansteckungsraten-flachten-bereits-vor-dem-lockdown-ab-809893127675}{bulunmuştur}.
  Eğer bu sonuç doğruysa, ``tecrit'' 'in manasının temelden
  sorgulanmasını gerektirir.
  (\href{https://bsse.ethz.ch/cevo/research/sars-cov-2/real-time-monitoring-in-switzerland.html}{Araştırma
  ile ilgili})
\item
  İsviçre'deki Infosperber dergisi, yetkililerin ve medyanın
  bilgilendirme politikasını eleştirmektedir:
  ``\href{https://www.infosperber.ch/Artikel/Gesundheit/Corona-Statt-zu-informieren-fuhren-Behorden-eine-PR-Kampagne}{Yetkililer
  bilgilendir}\href{https://www.infosperber.ch/Artikel/Gesundheit/Corona-Statt-zu-informieren-fuhren-Behorden-eine-PR-Kampagne}{me
  yapmak yerine bir Halkla İlişkiler kampanyası yürütüyor}''. Yanıltıcı
  rakamlar ve grafikler, en azından kısmen gerekçesi olmayan bir korkuyu
  yaymak üzere kullanılmaktadır.
\item
  İsviçre tüketiciyi koruma dergisi Ktipp de bilgilendirme politikasını
  ve medya haberlerini eleştirmektedir:
  ``\href{https://www.ktipp.ch/artikel/artikeldetail/behoerden-informieren-irrefuehrend/}{Yetkililer
  yanıltıcı bilgi veriyor}''.
\item
  İsviçreli bir araştırmacı Federal Halk Sağlığı Ofisi'nin en son
  Kovid-19 raporunu analiz etmiş ve
  \href{https://covid-19-fakten.blogspot.com/2020/04/die-analyse-des-aktuellen.html}{son
  derece eleştirel bir sonuca} varmıştır: Bu rapor ``bilimsel olarak
  dengesiz, dayatmacı ve yanıltıcı'' `dır. Gerçekler hesaba
  katıldığında, yetkililerin aldığı önlemler ``sorumsuzdur ve korku
  uyandırıcı'' `dır.
\item
  \href{https://www.rontalpraxis.ch/aktuelles}{İsviçre Sağlık Bakanı'na
  yazdıkları bir açık mektupta}, İsviçreli doktorlar, ``herşeyden önce
  medya tarafından pompalanan tehdit senaryosu ile yaşanan gerçek
  arasındaki tutarsızlık'' `tan söz ediyorlar. Genel nüfus içinde
  gözlenen Kovid-19 vakaları çok az sayıda ve çoğunlukla hafif
  seyrediyor, fakat toplumdaki ``anksiyete bozuklukları ve panik
  ataklar'' artıyor ve birçok hasta önemli muayene randevularına gelmeye
  cesaret edemez halde. ``Anladığımız kadarıyla, bu da tehlikelilik
  boyutu, İsviçre'de yalnızca medyada ve kafalarımızda varolan bir virüs
  yüzünden.''
\item
  Çok düşük hasta yükü nedeniyle,
  \href{https://www.20min.ch/schweiz/news/story/Spitaeler-28949526}{İsviçre}
  ve
  \href{https://www.spiegel.de/wirtschaft/unternehmen/trotz-corona-pandemie-warum-kliniken-jetzt-kurzarbeit-anmelden-a-3dc61bc9-fb12-4298-8022-bb4c2be39d7d}{Almanya}'da
  bulunan birçok klinik artık kısaltılmış mesai yapacaklarını duyurdu.
  Hasta sayısındaki düşüş \%80'e varmış durumda.
\item
  İsviçreli fizyoloji ve nöroşirürji emeritus profesörü Dr. Daniel
  Jeanmonod, yaptığı analizde şu tavsiyelerde bulunuyor:
  ``\href{https://off-guardian.org/2020/04/07/think-deep-do-good-science-and-do-not-panic/}{Enine
  boyuna düşünün, doğru düzgün bilim yapın ve paniklemeyin!}''
\item
  İsviçreli doktor Dr. Paul Robert Vogt, Kovid-19 konusunda
  \href{https://www.mittellaendische.ch/2020/04/07/covid-19-eine-zwischenbilanz-oder-eine-analyse-der-moral-der-medizinischen-fakten-sowie-der-aktuellen-und-zuk\%C3\%BCnftigen-politischen-entscheidungen/}{çok
  kez paylaşılan bir makale} yazmıştır. ``Sansasyon peşindeki basın'' `ı
  eleştiriyor, ama aynı zamanda bunun ``sıradan bir grip'' olmadığı
  uyarısında bulunuyor. Halbuki doktor Vogt, bazı noktalarda yanılıyor:
  ölümcüllük oranı ve yaş ortalaması çoğunlukla ana değişkenler olup,
  korona virüslü / korona virüsünden ayrımının yapılması gereklidir;
  solunum maskeleri ve solunum cihazları bir çok vaka için uygun
  değildir (aşağıya bkz.) ve sokağa çıkma yasakları sorgulanabilir
  nitelikte ve olasılıkla aksi etki yapan önlemlerdir.
\end{itemize}

\hypertarget{almanya-ve-avusturya}{%
\subparagraph{\texorpdfstring{\textbf{Almanya ve
Avusturya}}{Almanya ve Avusturya}}\label{almanya-ve-avusturya}}

\begin{itemize}
\tightlist
\item
  Alman sağlık uzmanları yeni bir çalışmada Federal Hükümet'in
  \href{https://www.tagesschau.de/investigativ/ndr-wdr/corona-experten-thesenpapier-101.html}{kriz
  politikasını eleştiriyor}. Uygulanan kısmi kapatma yüzünden toplumdaki
  uzun vadeli tahribattan söz ediyorlar. Robert Koch Enstitüsü
  tarafından yayınlanan rakamlar ``yalnızca sınırlı bir önem taşıyor''.
\item
  Alman Patologlar Federe Birliği, ``korona ölümleri'' `nde (gerçek ölüm
  nedenini belirlemek üzere) zorunlu otopsi
  \href{https://www.pathologie-dgp.de/die-dgp/aktuelles/meldung/pressemitteilung-an-corona-verstorbene-sollten-obduziert-werden/}{talebinde
  bulunuyor} ve böylece sözde aşırı tehlikeli olduğu için otopsi
  yapılmaması yönündeki ``Robert Koch Enstitüsü' tavsiyesi'' `ne
  katılmayıp aksini iddia ediyor.
\item
  Dr. Martin Sprenger, ``sivil ve bilimsel fikir özgürlüğünü geri
  kazanmak'' için Avusturya Sağlık Bakanlığı'nın Korona Uzman
  Konseyi'ndeki
  \href{https://mailchi.mp/addendum/fles-home-office-260342}{konumundan
  istifa etti}. Dr. Sprenger, daha önce hükümeti, başka konular
  nedeniyle de virüsün farklı toplumsal gruplar için oluşturduğu riskler
  arasında yeterince ayrım gözetmediği ve
  \href{https://www.addendum.org/coronavirus/interview-sprenger/}{aşırı
  geniş kapsamlı önlemler} aldığı için de eleştirmişti: ``Başka akut ve
  kronik hastalıklara yetersiz bakım yapılması yüzünden sağlıklı
  yaşanacak yıl kaybının, Kovid-19 yüzünden sağlıklı yaşanacak yıl
  kaybının 10 katını aşmamasına dikkat etmeliyiz.
\item
  Almanya'daki bir bakım evinde yaşayan 84 yaşında bir erkeğe yapılan
  Kovid-19 testi pozitif çıkmış, sonra da bütün bakım evi karantinaya
  alınmış ve herkese test yapılmıştı. Halbuki, sonradan ilk test
  sonucunun hatalı olduğu ortaya çıkmıştır.
\end{itemize}

\hypertarget{iskandinavya}{%
\subparagraph{\texorpdfstring{\textbf{İskandinavya}}{İskandinavya}}\label{iskandinavya}}

\begin{itemize}
\tightlist
\item
  Norveç'te Tabip Odası'nın Sağlık Bakanı'na yazdığı açık mektupta,
  normal hastalar artık muayene ve tedavi edilmediği için, alınan
  önlemlerin
  \href{https://www.abcnyheter.no/helse-og-livsstil/helse/2020/04/06/195667780/nesten-halvparten-av-sengene-pa-oslo-universitetssykehus-star-tomme}{virüsten
  daha tehlikeli} olabileceği kaygısı dile getiriliyor.
\item
  İsveçli bir yazar,
  \href{https://www.spectator.co.uk/article/no-lockdown-please-w-re-swedish}{İngiliz
  yayın organı Spectator'da şöyle demektedir}: ``Kitlesel bir deney
  yürüten İsveç değildir. Bunu yapan diğer bütün ülkelerdir.''
\item
  Hamburg Üniversite Hastanesi Müdürü Profesör Ansgar Lohse,
  \href{https://www.abendblatt.de/hamburg/article228880917/uke-professor-shutdown-lohse-deutschland-hamburg-corona-virus-infektion-covid-19-impfstoff-coronavirus-krise-patienten-immunitaet-krankenhaeuser-kontaktverbot-kliniken-infektionsrate.html}{bir
  söyleşide şunları söylemiştir}: ``Benim görüşüme göre, İsveç'in aldığı
  önlemler dünyadaki en rasyonel olanlarıdır. Tabii ki bunun psikolojik
  olarak sürdürülüp sürdürülemeyeceği sorusu ortaya çıkıyor.
  Başlangıçta, İsveçlilerin önemli ölçüde fazla ölümle başa çıkması
  gerekecek, ama sonra orta ve uzun vadede bunlar önemli ölçüde
  azalacak. İsveçliler tutumlarını sürdürebilirlerse, bunun faturası bir
  yıl içinde ortaya çıkacak. Ne yazık ki virüsün yarattığı korku,
  çoğunlukla politikacıları hiç de mantıklı olmayan hareketlere
  zorluyor. Politikalar medyadaki imgelerle de yönlendiriliyor.''
\item
  İsveç baş epidemiyoloğu Anders Tegnell, Kovid enfeksiyonları açısından
  Stokholm, artık bir ``düzlük'' `e ulaşmıştır.
  (\href{https://www.thelocal.se/20200310/timeline-how-the-coronavirus-has-developed-in-sweden}{İsveç
  ile ilgili daha fazla haber})
\end{itemize}

\hypertarget{abd-ve-asya}{%
\subparagraph{\texorpdfstring{\textbf{ABD ve
Asya}}{ABD ve Asya}}\label{abd-ve-asya}}

\begin{itemize}
\tightlist
\item
  ABD'de yetkililer şimdi, tüm pozitif testli ölümlerin de pozitif test
  sonucu olmayan kuşkulu vakaların da ``Kovid ölümleri'' olarak
  kaydedilmesini
  \href{https://nypost.com/2020/04/07/feds-classify-all-coronavirus-patient-deaths-as-covid-19-deaths/?link=TD_mansionglobal_new_mansion_global.11147f181987fd93}{tavsiye
  ediyor}. Minnesota senatörü bir doktor, bunun ölümleri manipüle
  etmekle eşdeğer olacağını
  \href{https://www.valleynewslive.com/content/misc/Sen-Dr-Jensens-Shocking-Admission-About-Coronavirus-569458361.html}{açıkladı}.
  Dahası, hastalarını Kovid-19 hastası olarak bildirmeleri için
  hastanelere maddi teşvik verilecek. (Bu konu ile ilgili biraz
  \href{https://swprs.files.wordpress.com/2020/04/cv-2019-2020.jpg}{mizah}).
\item
  Washington eyaletinde Seattle yakınlarındaki Kovid-19 sahra hastanesi,
  açıldıktan yalnızca 3 gün sonra
  \href{https://www.yahoo.com/news/armys-seattle-field-hospital-closes-165646379.html}{hiçbir
  hasta kabul etmeden kapatıldı}. Bu Wuhan yakınlarında çok kısa zamanda
  kurulan, çoğunlukla düşük kapasitede kullanılan ya da boş bile kalan,
  sonra da
  \href{https://www.theguardian.com/world/2020/feb/12/what-chinas-empty-new-coronavirus-hospitals-say-about-its-secretive-system}{sökülen}
  hastanelerin durumunu hatırlatıyor.
\item
  Birçok medya kuruluşu, New York yakınlarındaki Hart Adası'nda sözde
  ``korona toplu mezarları'' olduğunu bildirdi. Bu haberler iki açıdan
  yanıltıcıdır: birincisi, Hart Adası çok uzun zamandır ABD'nin en iyi
  bilinen ``fakir mezarlıkları'' `ndan biridir, ikincisi ise New York
  valisi toplu mezar planlarının olmadığını, fakat ``sahiplenilmeyen''
  (yani, akrabaları olmayan) ölülerin Hart Adası'nda gömüleceğini
  \href{https://www.independent.co.uk/news/world/americas/new-york-coronavirus-cases-burials-bodies-covid-19-hart-island-a9459956.html}{açıklamıştır}.
\item
  Hindistan'ın önde gelen epidemiyologlarından biri
  ``\href{https://www.business-standard.com/article/current-affairs/we-cannot-run-away-to-the-moon-need-to-develop-herd-immunity-dr-muliyil-120040601232_1.html}{Ay'a
  kaçmamız mümkün değil}'' demiştir. Kendisi toplumda doğal bağışıklığın
  hızla geliştirilmesini önermektedir.
\end{itemize}

\hypertarget{kuzey-italya}{%
\subparagraph{\texorpdfstring{\textbf{Kuzey
İtalya}}{Kuzey İtalya}}\label{kuzey-italya}}

Kuzey İtalya'ya ilişkin, birçok potansiyel risk etkeni daha önce burada
tartışılmıştı.

Lombardiya'da, en başta da daha sonra hastalığın en yoğun görüldüğü
bölgeler olan Bergamo ve Brescia'da, Kovid-19 salgının ortaya
çıkmasından hemen önceki aylarda
\href{https://www.bergamonews.it/2019/10/21/vaccinazione-antinfluenzale-a-bergamo-ordinate-185-000-dosi-di-vaccino/332164/}{grip}
ve
\href{https://www.bsnews.it/2020/01/18/meningite-vaccinate-34mila-persone-tra-brescia-e-bergamo/}{menenjit}
hastalıklarına karşı iki büyük aşılama kampanyası yürütüldüğü doğrudur.
Bu tür aşıların korona virüsü enfeksiyonu ile etkileşimi teorik olarak
muhtemel olsa da bu şu anda kabul edilmiş değildir.

Geçmişte kuzey İtalya'da, kanserli akciğer hastalığı riskini artıran
\href{https://www.spiegel.de/panorama/justiz/asbest-prozess-in-italien-nun-sind-alle-krank-a-666421.html}{yüksek
düzeyde asbeste maruz kalma} olayı da yaşanıyordu. Fakat bunun da
Kovid-19 ile doğrudan bağlantısı yoktur.

Yine de kuzey İtalya'daki halkın, akciğer sağlığını bozan ve solunum
yolu hastalıklarına
\href{https://www.thelocal.it/20170131/our-lungs-are-breaking-smog-levels-way-above-safe-limits-in-northern-italy}{özellikle
yatkın} hale getiren yüksek düzeyde
\href{https://twitter.com/esa/status/1238480433047916545}{hava
kirliliği} ve zararlı başka etkenlere uzun süredir maruz kaldığı genel
olarak doğrudur.

\includegraphics{https://swprs.files.wordpress.com/2020/03/italy-smog.png?w=500\&h=281}

\hypertarget{isviuxe7reli-baux15fhekim-pietro-vernazza}{%
\subparagraph{\texorpdfstring{\textbf{İsviçreli Başhekim Pietro
Vernazza}}{İsviçreli Başhekim Pietro Vernazza}}\label{isviuxe7reli-baux15fhekim-pietro-vernazza}}

İsviçreli Enfeksiyoloji başhekimi Profesör Pietro Vernazza, Kovid-19 ile
ilgili çalışmalar üzerine dört yeni makale yayınlamıştır.

\begin{itemize}
\tightlist
\item
  \href{https://infekt.ch/2020/04/schulen-schliessen-hilfreich-oder-nicht/}{İlk
  makale}; çocukların Kovid hastalığına yakalanmamaları ve (gripteki
  durumun aksine) virüsün taşıyıcısı da olmamaları nedeniyle, okul
  kapatmaların yararına ilişkin bir tıbbi kanıtın hiçbir zaman olmadığı
  gerçeği ile ilgilidir.
\item
  \href{https://infekt.ch/2020/04/atemschutzmasken-fuer-alle-medienhype-oder-unverzichtbar/}{İkinci
  makale}; maskelerin belirtiler gösteren (özellikle öksürük) hastaların
  virüs yaymalarını azaltabilmek dışında belirgin bir etkisi olmadığı
  gerçeği ile ilgilidir. Bunu dışında maskeler daha çok semboliktir veya
  bir ``medya yutturmacası'' `dır.
\item
  \href{https://infekt.ch/2020/04/corona-testen-testen-und-kein-ende/}{Üçüncü
  makale}; kitlesel testler yapılması konusunu ele almaktadır. Profesör
  Vernazza'nın vardığı sonuç şöyledir: ``Bir solunum yolu hastalığı
  belirtileri gösteren kim varsa evde kalır. Grip için de bu böyledir.
  Test yapmanın ek bir değeri yoktur.''
\item
  \href{https://infekt.ch/2020/03/immunschwaeche-und-schwangerschaft-kein-covid-19-risikofaktor/}{Dördüncü
  makale}; Kovid-19 risk grupları konusunu ele almaktadır. Mevcut
  bilgiye göre, bunlar yüksek tansiyonu olanlardır -- Kovid-19 virüsünün
  kan basıncını ayarlamaktan da sorumlu olan hücre reseptörlerini
  kullandığından kuşkulanılmaktadır. Buna karşın, şaşırtıcı biçimde,
  bağışıklık zafiyeti olanlar ve (bağışıklık sistemleri doğal olarak
  düşük olan) hamile kadınlar risk altında değildir. Aksine, Kovid-19'un
  riski, genelde bağışıklık sisteminin aşırı tepki vermesidir.
\end{itemize}

\hypertarget{youx11fun-bakux131m-veya-palyatif-yatux131ux15ftux131rux131cux131-bakux131m}{%
\subparagraph{\texorpdfstring{\textbf{Yoğun Bakım veya Palyatif
(yatıştırıcı)
Bakım}}{Yoğun Bakım veya Palyatif (yatıştırıcı) Bakım}}\label{youx11fun-bakux131m-veya-palyatif-yatux131ux15ftux131rux131cux131-bakux131m}}

Almanya'da bir doktor
\href{https://www.ruhr24.de/ruhrgebiet/coronavirus-behandlung-intensivstation-nrw-lungenentzuendung-matthias-thoens-witten-zr-13645038.html}{kendisiyle
yapılan söyleşide şunları anlatmıştır}: Ciddi biçimde etkilenenler
genelde, daha önceden birden çok hastalığı bulunan yaşlılar olduğu için,
Kovid-19 ``bir yoğun bakım hastalığı değildir''. Bu insanlara, zatürreye
yakalandıkları zaman, ``her zaman palyatif bakım yapılmıştır (yani,
ölümlerine eşlik edilmiştir)''. Buna karşın, bir Kovid-19 tanısı ile
artık yoğun bakım vakaları olarak ele alınıyorlar, ama ``tabii ki
hastalar yine de kurtarılamıyor.''

Uzman, birçok karar vericinin şu anki davranışlarını ``panik modu''
olarak tarif ediyor. Mevcut durumda, Almanya'daki yoğun bakım yatakları
halen görece boş olup, suni solunum cihazları da boştadır. Hastane
yöneticileri, pek yakında maddi nedenlerle yaşlıları kabul etme fikrini
keşfedebilirler. ``Koğuşlar 14 gün içinde, kurtarılması mümkün olmayan,
birden çok kronik hastalıktan muzdarip yaşlılarla dolup taşacak. Bir kez
makinelere bağlandıklarındaysa bir daha o makineleri kimin durduracağı
meselesi ortaya çıkacak, çünkü bu cinayet olarak kabul edilecek.''
Doktor, bu durumu açgözlülükten kaynaklanan bir ``etik facia'' `nın
izleyebileceği uyarısında bulunuyor.

\hypertarget{kovid-19-ile-suni-solunum-1}{%
\subparagraph{\texorpdfstring{\textbf{Kovid-19 ile Suni
Solunum}}{Kovid-19 ile Suni Solunum}}\label{kovid-19-ile-suni-solunum-1}}

Kovid-19 hastaları için dünya çapında bir suni solunum cihazı telaşı
olmuş ve hala olagelmektedir. Bu site; sert bir müdahale olan suni
solunumun (entübasyon) çoğu vakada aksi etki yapabildiği ve hastalara
yarardan çok zarar verebildiği gerçeğine dünyada ilk dikkat çekenlerden
biriydi.

Sert bir müdahale olan suni solunum başta tavsiye edilmişti, çünkü düşük
oksijen düzeyleri, hatalı olarak, akut solunum (akciğer) iflası olduğu
sonucuna varılmasına yol açmıştı ve daha nazik, sert-olmayan, teknikler
kullanılırsa virüsün aerozoller aracılığıyla yayılabileceği korkusu
vardı.

Bu arada, ABD'de ve Avrupa'da önde gelen birçok göğüs hastalıkları
uzmanı ve yoğun bakım doktoru, sert bir müdahale olan suni solunuma
karşı görüş bildirmiş ve daha nazik yöntemler veya hatta Güney Kore'de
başarıyla kullanılmış olan oksijen terapisini önermiştir.

\begin{itemize}
\tightlist
\item
  \href{https://time.com/5818547/ventilators-coronavirus/}{Bazı
  Doktorlar Suni Solunum Tedavilerinden Artık Neden Uzak Duruyor} (TIME)
\item
  \href{https://www.spectator.co.uk/article/Ventilators-aren-t-a-panacea-for-a-pandemic-like-coronavirus}{Suni
  Solunum Cihazları korona virüsü gibi bir pandemi için çare değil} (Dr.
  Matt Strauss)
\item
  \href{https://www.statnews.com/2020/04/08/doctors-say-ventilators-overused-for-covid-19/}{Suni
  Solunum Cihazları tükenirken, doktorlar Kovid-19 için bu makinaların
  aşırı kullanıldığını söylüyor} (SN)
\item
  \href{https://www.atsjournals.org/doi/pdf/10.1164/rccm.202003-0817LE}{Kovid-19
  ``Tipik'' bir Akut Solunum Yolları Sıkıntısı Sendromu'na Yol Açmıyor}
  (ATSJ)
\item
  \href{https://www.medscape.com/viewarticle/928156}{KOVID-19 Suni
  Solunum Protokollarının Yeniden İncelenmesi}
  \href{https://www.medscape.com/viewarticle/928156}{mi
  Gerekiyor}\href{https://www.medscape.com/viewarticle/928156}{?}
  (Medscape)
\item
  Almanca: \href{https://archive.is/KX5IQ}{``Entübasyon ile sert bir
  müdahale olan suni solunum aşırı sıklıkta kullanılıyor''} (Dr. Thomas
  Voshaar)
\item
  Almanca:
  \href{https://www.doccheck.com/de/detail/articles/26271-covid-19-beatmung-und-dann}{KOVID-19:
  Suni solunum -- peki ya sonrası?} (DocCheck)
\item
  Almanca: \href{https://www.youtube.com/watch?v=JWlouv9QafU}{Bir yoğun
  bakım doktorunun covid-19 deneyim raporu} (Dr. Tobias Schindler)
\end{itemize}

\hypertarget{politik-geliux15fmeler-1}{%
\subparagraph{\texorpdfstring{\textbf{Politik
Gelişmeler}}{Politik Gelişmeler}}\label{politik-geliux15fmeler-1}}

\begin{itemize}
\tightlist
\item
  Amerikan Güvenlik Kurumu'nu (NSA) medyaya ifşa eden Edward Snowden
  kendisiyle yapılan yeni bir söyleşide, hükümetlerin korona virüsünü
  bir
  ``\href{https://www.vice.com/en_us/article/bvge5q/snowden-warns-governments-are-using-coronavirus-to-build-the-architecture-of-oppression}{tahakküm
  mimarisi}'' inşa etmek amacıyla kullandığı uyarısında bulunuyor.
\item
  Apple ve Google, yetkililerin toplum içindeki bağlantıları izlemesini
  sağlayacak olan ve
  ``\href{https://www.bloomberg.com/news/articles/2020-04-10/apple-google-bring-covid-19-contact-tracing-to-3-billion-people}{temas
  izleme}'' denilen bir özelliği, kendi mobil işletim sistemlerine
  katmak üzere ulusal yetkililerle birlikte çalışacaklarını duyurdular.
\item
  Alman anayasa hukuku uzmanı Uwe Volkmann, ARD televizyonuna verdiği
  demeçte, meslektaşları arasında Korona önlemlerinin anayasaya uygun
  olduğunu düşünen
  ``\href{https://www.youtube.com/watch?v=DvzrGLvzllU}{hiç kimse}''
  olmadığını söyledi.
\item
  İtalyan hükümeti, İnternet'te Kovid ile ilgili yalan haberleri
  ``\href{https://www.faz.net/aktuell/feuilleton/medien/corona-in-italien-das-virus-und-die-wahrheit-16714529.html}{elimine
  etmek}'' için bir ``görev gücü'' oluşturdu. Buna karşın, ifade
  özgürlüğünün ``dokunulmadan'' kaldığı söylendi.
\item
  Fransa, Kovid yüzünden, izin verilen duruşma-öncesi tutukluluk
  sürelerini uzattı ve hakim incelemesini
  \href{https://www.lefigaro.fr/politique/coronavirus-le-conseil-d-etat-sur-la-ligne-de-crete-des-libertes-publiques-20200406}{askıya
  aldı}. Baroların şikayetleri reddedildi.
\item
  Danimarka Nisan ayı başında
  \href{https://www.fr.de/politik/coronavirus-sars-cov-2-daenemark-notfalls-militaer-13598503.html}{``daha
  önce görülmemiş sertlikte acil durum yasaları}'' `nı yürürlüğe soktu:
  ``Sağlık yetkilileri, artık zorunlu testler, zorunlu aşılar ve zorunlu
  tedaviler yapılmasını emredebilir ve emirlerini yürürlüğe sokmak
  üzere, polise ek olarak askeri ve özel güvenlik hizmetlerini
  kullanabilir.''
\item
  Almanya'nın Kuzey Rhine-Westphalia eyaleti polisi, özellikle yasaklı
  insan gruplarının aranması için ``korona operasyonu'' dahilinde
  \href{https://rp-online.de/nrw/panorama/nrw-polizei-testet-drohnen-bei-einsaetzen-wegen-corona-massnahmen_aid-50006143}{drone
  testleri yapıyor}.
\item
  Almanya'nın Saksonya eyaleti, karantinayı reddedenleri
  \href{https://www.welt.de/politik/deutschland/article207198029/Coronavirus-Sachsen-will-Quarantaene-Verweigerer-in-Psychiatrien-sperren.html}{psikiyatri
  hastanelerine} yatırmak istiyor.
\item
  İsviçreli bir doktor korona önlemlerini eleştirdiği ve yetkilileri
  sözde tehdit ettiği için tutuklanıp
  \href{https://www.blick.ch/news/schweiz/mittelland/in-baden-ag-polizei-in-vollmontur-im-einsatz-id15841510.html}{psikiyatrik
  muayeneye} gönderildi.
\item
  Almanya'da tıp hukuku uzmanı bir avukat, Korona önlemlerine karşı
  anayasaya aykırılık nedeniyle suç duyurusunda bulundu ve bir polis
  devleti haline gelme tehlikesine karşı uyarıda bulunan ve gösteriler
  düzenlenmesini isteyen bir açık mektup yayınladı. Savcılık ve polis
  bunun üzerine avukat hakkında ``ceza alması isteğiyle''
  \href{https://www.morgenweb.de/mannheimer-morgen_artikel,-coronavirus-aufruf-zu-straftaten-ermittlungen-gegen-heidelberger-rechtsanwaeltin-_arid,1627078.html}{soruşturmalar
  açtı} ve kendisine ait websitesi geçici olarak kapatıldı. Anayasaya
  aykırılıkla ilgili suç duyurusu da reddedildi.
\item
  Avusturya'da da bir kaç avukat Korona önlemlerine karşı Anayasa
  Mahkemesi'nde \href{https://wien.orf.at/stories/3043172/}{suç
  duyurularında bulundu}. Avukatlar bu önlemlerle temel hakların ve
  güçler ayrılığının çiğnendiğini ileri sürüyor.
\item
  Los Angeles belediye başkanı, sokağa çıkma yasaklarını ihlal ederlerse
  komşularını yetkililere ihbar eden ``muhbirler'' `e
  \href{https://townhall.com/tipsheet/bethbaumann/2020/04/04/la-mayor-garcetti-says-snitches-get-rewards-for-ratting-out-their-neighbors-n2566348}{bir
  ödül sözü verdi}.
\item
  ABD'de, çalışan nüfusun \%10'unu oluşturan 16 milyonu aşkın insan
  zaten
  \href{https://www.nytimes.com/2020/04/09/us/coronavirus-us-news.html}{tecrit
  yüzünden işsizdir}. Uluslararası Çalışma Örgütü'ne göre, dünyadaki 3,3
  milyar işçinin \%80'i şu anda önlemlerden etkilenmiş durumda olup 1,25
  milyar işçi
  \href{https://www.ilo.org/global/about-the-ilo/newsroom/news/WCMS_740893/lang--en/index.htm}{şiddetli
  veya faciaya varan boyuttaki} sonuçlardan etkilenecektir.
\end{itemize}

\includegraphics{https://swprs.files.wordpress.com/2020/04/us-jobless-claims.png?w=600\&h=375}

\hypertarget{7-nisan-2020}{%
\paragraph{7 Nisan 2020}\label{7-nisan-2020}}

\begin{itemize}
\tightlist
\item
  Almanya'daki Robert Koch Enstitüsü'nün hazırladığı
  \href{https://multipolar-magazin.de/artikel/coronavirus-regierung-ignoriert-daten}{özel
  bir raporda yer alan en son test rakamları}, pozitiflerin oranının
  (yani, pozitif çıkanların test sayısının yapılan test sayısına oranı)
  medyada gösterilen ekponansiyel (git gide hızlanan) eğrilerden çok
  daha yavaş yükseldiğini ve Mart ayı sonunda korona virüsleri için
  oldukça tipik bir değer olan yalnızca \%10 olduğunu gösteriyor.
  Multipolar dergisine göre, bu nedenle, ``virüsün tehlike yaratacak bir
  hızla yayılması gibi bir durum söz konusu olamaz''.
\item
  Hamburg'daki adli tıp başkanı Profesör Klaus Püschel,
  \href{https://www.pressreader.com/germany/hamburger-morgenpost/20200403/281487868456736}{Kovid-19
  hakkında şu açıklamada bulunmuştur}: ``Bu virüs hayatımızı aşırı bir
  biçimde etkiliyor. Bu durum virüsün yarattığı tehditle orantısızdır.
  Şu anda neden olunan astronomik ekonomik yıkım da virüsün oluşturduğu
  tehlike ile orantılı değil. Korona ölüm sayılarının yıllık ölüm
  sayıları içinde bir doruk olarak bile görünmeyeceğine ikna olmuş
  durumdayım.'' Örneğin Hamburg'da, ``daha önceden hasta olmayan tek bir
  kişi bile'' virüsten ölmüş değil: ``Şu ana kadar incelediğimiz
  kişilerin hepsinde kanser, kronik bir akciğer hastalığı, aşırı sigara
  tüketimi veya obezite vardı; diyabet veya bir kalp damar hastalığından
  muzdariptiler. Bu virüs adeta bardağı taşıran son damladır. ``Kovid-19
  yalnızca sıradışı vakalarda ölümcül bir hastalıktır, ama çoğu vakada
  ağırlıklı olarak zararsız bir viral enfeksiyondur.''Buna ek
  olarak,\href{https://www.abendblatt.de/hamburg/article228828787/rechtsmedizin-pueschel-hamburg-corona-virus-infektion-covid-19-coronavirus-krise-patienten-krankenhaeuser-kliniken-infektionsrate-krankheit-pandemie-test-lungenkrankheit-sars-cov-epidemie-sars-cov-2.html}{Dr.
  Püschel}\href{https://www.abendblatt.de/hamburg/article228828787/rechtsmedizin-pueschel-hamburg-corona-virus-infektion-covid-19-coronavirus-krise-patienten-krankenhaeuser-kliniken-infektionsrate-krankheit-pandemie-test-lungenkrankheit-sars-cov-epidemie-sars-cov-2.html}{şunları
  söylemiştir}: Çok az sayıdaki vakada ise şu anki korona enfeksiyonunun
  ölümcül sonuçlarla hiçbir ilgisi olmadığını, çünkü örneğin bir beyin
  kanaması veya bir kalp krizi gibi, başka ölüm nedenleri olduğunu
  gördük. Adli tıp uzmanı Dr. Püschel, Korona'nın kendisinin ``özellikle
  tehlikeli bir viral hastalık olmadığını'' söylüyor. Somut muayene
  sonuçlarına dayanan istatistikler yapılmasını istiyor.
\item
  ``Uzmanlarca incelenmemiş herbir ölüm ile ilgili bütün spekülasyonlar
  yalnızca tedirginliği artırıyor.'' Robert Koch Enstitüsü kurallarının
  aksine, Hamburg son zamanlarda ``korona virüslü'' ve ``korona
  virüsünden'' ölümler arasında ayrım yapmaya başladı. Bu ise Kovid-19
  ölüm sayılarının düşmesine yol açmaktadır.
\item
  Alman virolog Hendrik Streeck, şu anda Kovid-19 patojeninin dağılımı
  ve bulaşma yollarını belirlemek üzere bir pilot çalışma yürütüyor.
  \href{https://www.zeit.de/wissen/gesundheit/2020-04/hendrik-streeck-covid-19-heinsberg-symptome-infektionsschutz-massnahmen-studie/komplettansicht}{Bir
  söyleşide şu bilgileri vermektedir}: ``Heinsberg bölgesinde ölmüş olan
  40 kişiden 31'inin vakalarını yakından inceledim ve bu insanların
  ölmelerine fazla şaşırmadım. Ölenlerden biri 100 yaşını aşmıştı, bu
  nedenle de basit bir soğuk algınlığı bile ölümüne yol açabilirdi.''
  Başlangıçtaki varsayımların aksine, Streeck, kapı kollarından ve
  benzerinden bulaşma (yani, ``smear'' enfeksiyonu) olduğunu
  kanıtlayabilmiş değildir.
\item
  İsviçre'de hastaneler çok düşük kapasite kullanımı yüzünden birbiri
  ardına
  \href{https://www.engadinerpost.ch/2020/4/04/Engadiner-Spitaeler-haben-freie-Kapazitaeten}{kısaltılmış
  mesai yapılacağını duyurmaya başladı}: ``Tüm bölümlerdeki çalışanların
  çok az işi var ve bu durum bir ilk adım olarak fazla mesaiyi ortadan
  kaldırdı. Artık kısaltılmış mesai de kayıtlara geçiyor. Bunun maddi
  sonuçları çok ciddi.`` Bir hatırlatma olarak, Zürih'teki ETH
  Üniversitesi tarafından yapılmış gerçekçi olmayan aşırı varsayımlara
  dayalı bir çalışma, İsviçre'deki kliniklerde 2 Nisan itibariyle
  \href{https://www.toponline.ch/news/coronavirus/detail/news/studie-bestaetigt-engpass-bei-spitalbetten-steht-kurz-bevor-00131333/}{darboğaza
  girileceğini öngörmüştü.} Şu ana kadar hiçbir yerde böyle bir şey
  olmadı.
\item
  İsviçre'de, 2017 yılı başında dikkat çekici bir grip dalgası vardı. O
  sırada, yılın ilk 6 ayında 65-yaş-üstü nüfusta neredeyse
  \href{https://www.srf.ch/news/schweiz/todesursachen-statistik-woran-die-meisten-schweizerinnen-und-schweizer-sterben}{1500
  fazladan ölüm} oldu. Normalde, İsviçre'de her yıl \%95'i 65 yaşının
  üzerinde olan yaklaşık
  \href{https://www.nzz.ch/lungenentzuendung-1.4550285}{1300 kişi}
  zatürreden ölmektedir. Bununla kıyaslandığında, şu anda İsviçre'de
  Kovid-19'lu (Kovid-19'un neden olduğu değil) toplam
  \href{https://www.corona-data.ch/}{762 ölüm} olduğu bildirilmiştir.
\item
  Almanya çevre laboratuvarı müdürü; kuzey İtalya'daki Lombardiya
  bölgesi sakinlerinin, herkesçe bilinen yüksek bir lejyonella (çev.
  notu: Lejyoner Hastalığı'na yol açan bakteri) kirliliğinden dolayı,
  Kovid-19 gibi
  \href{https://m.apotheke-adhoc.de/nachrichten/detail/coronavirus/erhoehen-legionellen-die-todesrate-einer-corona-infektion/}{viral
  enfeksiyonlardan özellikle kolay etkilendiklerinden kuşkulanıyor}:
  ``Akciğerler, mevcut durumda olduğu gibi, bir viral enfeksiyon
  yüzünden zayıf düşmüşse, bakterilerin işi kolaylaşır, hastalığın
  seyrini olumsuz yönde etkileyebilir, komplikasyonlara yol açabilir.''
  Lombardiya'da bölgesel zatürre salgınları, lejyonella bakterisi
  bulaşmış buharlaşma yoluyla çalışan soğutma sitemleri yüzünden zaten
  geçmişte de olmuştur.
\item
  Çin'den alınan haberlere dayanarak, pozitif testli yoğun bakım
  hastaları için vakit kaybetmeden uygulanması öngörülen sert bir
  müdahale olan entübasyon (tüp/hortum yerleştirilmesi) ile suni
  solunuma dayanan tıbbi protokoller dünyanın dört bir yanında kabul
  edilmişti. Protokollerde daha yumuşak bir uygulama olan maske
  aracılığıyla yapılan suni solunumun yeterince güçlü olmadığı
  belirtilirken, bu yöntemle ``tehlikeli virsüslerin'' aerozoller
  (havada asılı kalan çok küçük tanecikler) yoluyla yayılma riskinin de
  bulunduğu vurgulanıyor.\\
  Halbuki Mart ayı gibi erken bir tarihte bile, Alman doktorlar
  entübasyonun akciğerlerde ek tahribata yol açabileceğine ve toplamda
  başarı şansının düşük olduğuna
  \href{https://www.doccheck.com/de/detail/articles/26271-covid-19-beatmung-und-dann}{işaret
  etmişti}. Bu sırada, ABD'deki doktorlar da entübasyonu, hastalara
  ``\href{https://www.youtube.com/watch?v=k9GYTc53r2o}{faydadan çok
  zarar getiren bir yöntem''}olarak tarif etmiştir. Hastalar çoğunlukla
  akut akciğer iflasından değil de artırılan basınçla yapılan suni
  solunum nedeniyle daha kötüye giden bir tür ``yüksek irtifa
  hastalığı'' `ndan zarar görmektedir. Şubat ayında
  \href{https://www.upi.com/Top_News/World-News/2020/02/14/Oxygen-therapy-working-for-coronavirus-patient-Seoul-says/6651581696794/}{Güney
  Kore'li doktorların bildirdiğine göre}, durumu kritik olan Kovid-19
  hastaları, bir suni solunum cihazı olmaksızın yapılan, oksijen
  tedavisine iyi cevap veriyor. Yukarda söz edilen Amerikalı doktor, ek
  akciğer tahribatına yol açmamak için, suni solunum cihazı kullanımının
  acilen yeniden gözden geçirilmesi gerektiği konusunda uyarıda
  bulunuyor.
\item
  ABD'nin resmi Kovid-19 öngörüleri şu ana kadar, hastaneye yatacak
  hasta sayısını 8 kat, ihtiyaç duyulacak yoğun bakım yatağı sayısını
  6,4 kat ve ihtiyaç duyulacak suni solunum cihazı sayısını 40,5 kat
  \href{https://twitter.com/NikolovScience/status/1246823479820693505}{abartılı
  tahmin etmiştir}.
\end{itemize}

\hypertarget{ek-notlar}{%
\subparagraph{\texorpdfstring{\textbf{Ek
notlar}}{Ek notlar}}\label{ek-notlar}}

\begin{itemize}
\tightlist
\item
  Kovid-19 paniğini eleştirenlerin, uluslararası tanınmış ve ilk
  örneklerinden biri olan \href{http://wodarg.com/}{Dr. Wolfgang
  Wodarg'ın websitesi}, Alman hizmet sağlayıcı Jimdo tarafından bugün
  birkaç saatliğine \href{https://twitter.com/wodarg}{silinmiş}, fakat
  güçlü protestoların ardından yeniden online olmuştur. Geçici
  kapatmanın genel şikayetlerden mi yoksa siyasi bir talimat yüzünden mi
  olduğu bilinmiyor.
\item
  \href{https://swprs.org/prof-dr-sucharit-bhakdiden-basbakan-dr-angela-merkele-acik-mektup/}{Başbakan
  Angela Merkel'e bir Açık Mektup} yazmış olan emeritus profesör Dr.
  Sucharit Bhakdi'nin üniversitedeki eposta adresi, daha önce
  kapatılmış, protestolardan sonra ise yeniden açılmıştır.
\item
  Danimarka Parlamentosu 2 Nisan'da, hükümetin koymuş olduğu kurallara
  uymayan bilgilerin yayınlanmasını yasaklayan ve websitelerinin
  silinmesine ve yazarlarının hapse atılmasına izin veren
  \href{https://newsvoice.se/2020/04/danmark-forbjuder-corona-policy/}{yeni
  bir yasa çıkarmıştır}. Bazı yorumcular bunun sonucunda hemen geri
  çekilmiştir.
\item
  Almanya'da bilim ve tıp konularında yazan gazeteci Robert Schröder,
  bir makalesinde
  \href{https://www.nachrichten-fabrik.de/news/harald-wiesendanger-ueber-die-massenmedien-waehrend-der-corona-krise-ich-schaeme-mich---meines-berufsstands-152103}{mesleğinin
  mevcut krizde tümüyle sekteye uğramakta olduğunu} belirtmiştir:
  ``Eleştirel, tarafsız bir Dördüncü Güç olarak, erk sahiplerini
  denetlemesi gereken bir meslek; nasıl olur da izleyicilerinin içinde
  bulunduğu kolektif isteriye, ışık hızında ve neredeyse oybirliği ile
  yenik düşebilir; sahibinin sesi olmaya, hükümet propagandasına ve
  uzmanların tanrılaştırılmasına kendini böyle teslim edebilir. Bu benim
  için anlaşılmaz bir hal, midemi bulandırıyor, artık canıma yetti,
  utanç içinde bu değersiz gösteri ile aramdaki ilişkiyi kopartıyorum.
\item
  Şu anda,
  \href{https://www.sciencealert.com/one-third-of-the-world-s-population-are-now-restricted-in-where-they-can-go}{insanlığın
  üçte biri ``tecrit'' altında}, ki bu da İkinci Dünya Savaşı sırasında
  dünyada yaşayan insan sayısından daha fazladır.
\item
  ABD'de, işsizlik parası başvurularının sayısı
  \href{https://www.reuters.com/article/us-health-coronavirus-usa-layoffs/us-weekly-jobless-claims-seen-at-record-high-again-idUSKBN21K0FX}{hızla
  artarak 6 milyonu aşmıştır} (bkz. Grafik), bu ise 1929'daki Büyük
  Bunalım'dan bu yana eşi benzeri olmayan bir rakamdır.
\item
  Yüzü aşkın insan hakları ve sivil özgürlükler kuruluşu, ``korona
  krizi'' `nin
  \href{https://www.dailymail.co.uk/news/article-8181381/World-sleepwalking-surveillance-state-rights-groups-warn.html}{insanlığı
  bir sürveyans (gözetleme) devleti haline getirdiği} uyarısında
  bulunuyor. Twitter'da \#covid19 hashtag'inin yerini, kısmen
  \#covid1984 hashtag'i aldı.
\item
  ABD'li jeostratejist Henry Kissinger, Wall Street gazetesinde şöyle
  yazıyor,
  ``\href{https://www.wsj.com/articles/the-coronavirus-pandemic-will-forever-alter-the-world-order-11585953005}{Korona
  virüsü pandemisi dünya düzenini sonsuza dek değiştirecek}''. ABD, bir
  yandan ``yeni bir dönem için acil planlama çalışmaları'' `na
  başlarken, yurttaşlarını hastalıktan ``korumak'' zorundadır.
\end{itemize}

\hypertarget{5-nisan-2020}{%
\paragraph{5 Nisan 2020}\label{5-nisan-2020}}

\begin{itemize}
\tightlist
\item
  New York'ta yaşayan uluslararası üne sahip epidemiyoloji profesörü
  Knut Wittkowski, kendisiyle yapılan
  \href{https://www.youtube.com/watch?v=lGC5sGdz4kg}{40 dakikalık bir
  söyleşide}, Kovid-19 ile ilgili olarak alınan önlemlerin tümünün ters
  etkili olduğunu açıklıyor. ``Sosyal mesafe'', okulların kapatılması,
  ``tecrit'', maskeler, kitlesel testler ve aşılar yerine, hayat
  olabildiğince sekteye uğratılmadan devam etmeli ve bağışıklık
  olabildiğince hızlı oluşturulmalıdır. Şu ana kadarki tüm bulgulara
  göre, Kovid-19 daha önceki grip salgınlarından daha tehlikeli
  değildir. Şimdiki izolasyon sadece daha sonra ``ikinci bir dalga'' `ya
  neden olacaktır.
\item
  İngiliz tıp dergisi British Medical Journal (BMJ), Çin'e ait en son
  verilere göre testleri yeni pozitif çıkmış bireylerin \%78'inin hiçbir
  belirti göstermediğini
  \href{https://www.bmj.com/content/369/bmj.m1375}{bildiriyor}. Oxford
  Üniversitesi'nden bir epidemiyolog, bu bulguların ``çok çok önemli''
  olduğunu söylemiş ve sözlerine ek olarak şunu söylemiştir. Eğer bu
  sonuçların temsil gücü varsa ``o zaman şu soruyu sormak zorundayız:
  `Ne demeye tecrit altındayız?'''.
\item
  Viyana Tıp Fakültesi Genel ve Aile Hekimliği Bölümü Başkanı ve Kanıta
  Dayalı Tıp Ağı yönetim kurulu başkanı Dr. Andreas Sönnichsen, şu ana
  kadar dayatılan önlemleri
  \href{https://www.diepresse.com/5794224/was-machen-wir-da-auf-den-intensivstationen-eigentlich}{``delilik''
  olarak} değerlendiriyor. ``Virüsün etkileyeceği az sayıdaki kişiyi
  korumak'' için tüm ülke felç edilmektedir.
\item
  Dünyada bir ilk olarak, İsveç hükümeti, korona virüslü ölümler ile
  korona virüsünden ölümleri resmen ayırdedeceklerini
  \href{https://www.telegraph.co.uk/news/2020/04/03/coronavirus-swedish-experiment-could-prove-britain-wrong/}{açıklamıştır},
  bunun ise bildirilen ölüm sayısında bir azalmaya yol açması
  gerekecektir. Bu arada, her nedense, serbestlikten yana olan
  stratejisini terketmesi için İsveç'in üzerindeki uluslararası baskı
  sürekli olarak artıyor.
\item
  Hamburg sağlık müdürlüğü,
  \href{https://www.t-online.de/nachrichten/deutschland/id_87636856/coronavirus-hamburg-will-nur-echte-covid-19-tote-zaehlen.html}{yalnızca
  ``gerçek'' korona ölümlerini} saymak üzere artık pozitif-testli
  ölümleri adli tıbba inceletmektedir. Bunun sonucunda, ölümlerin sayısı
  şimdiden Robert Koch Enstitüsü'nün resmi rakamlarıyla
  karşılaştırıldığında \%50'ye varan bir oranda azalmıştır.
\item
  2018 gibi erken bir tarihte bile, Alman Doktorlar Dergisi'nde, kuzey
  İtalya'da yetkilileri endişelendiren
  \href{https://www.aerzteblatt.de/nachrichten/97750/Vielzahl-an-Lungenentzuendungen-beunruhigen-Behoerden-in-Norditalien}{``çok
  sayıda zatürre vakası''} `ndan söz ediliyordu. O zamanlar, buna yol
  açabilecek nedenlerden biri olarak kirlenmiş içme sularından
  kuşkulanılmıştı.
\item
  Alman Eczacılık Gazetesi, mevcut durumda hastaların sık sık ``daha
  önceden solunum yollarında belirtiler ortaya çıkmadan ciddi biçimde
  hastalandıklarına, hatta öldüklerine''
  \href{https://www.pharmazeutische-zeitung.de/atemstillstand-koennte-auch-zentrale-ursache-haben-116664/}{işaret
  etmektedir}. Nörologlar bu durumda, korona virüslerinin aynı zamanda
  sinir hücrelerini de tahrip edebildiğinden kuşkulanıyor. Halbuki,
  bunun bir diğer açıklaması, çoğunlukla bakıma muhtaç olan bu
  hastaların aşırı stresten öldükleri olabilir.
\item
  \href{https://www.bag.admin.ch/dam/bag/de/dokumente/mt/k-und-i/aktuelle-ausbrueche-pandemien/2019-nCoV/covid-19-lagebericht.pdf.download.pdf/COVID-19_Epidemiologische_Lage_Schweiz.pdf}{İsviçre'ye
  ait en son rakamlara} göre, testleri pozitif çıkan hastalarda en çok
  görülen belirtiler, ateş, öksürük ve solunum güçlüğüyken, bunların
  \%43'ü ya da yaklaşık 900 kişi zatürre geçirmektedir. Halbuki bu
  vakalarda bile, soruna korona virüsünün mü yoksa başka hastalık
  yapıcıların mı yol açtığı a priori (test öncesi) açık değildir.
  Testleri pozitif çıkanların yaş ortalaması 83 olup, yaş aralığının üst
  sınırı 101'e çıkmaktadır.
\item
  \href{http://inproportion2.talkigy.com/}{``In Proportion''} adlı
  İngiliz projesinde Kovid-19 ``ile'' ölüm sayıları, gripten ölümlerin
  sayısı ve tüm nedenlerden ölümlerin sayısı ile karşılaştırmalı olarak
  izlenmektedir.
\item
  ABD'nin Indiana eyaletindeki zihin sağlığı ve intihar acil yardım
  hattına gelen aramalar, tecrit ve ekonomik etkileri nedeniyle
  \%2000'den fazla artarak, günde 1000'den 25.000'e çıkmıştır.
\item
  Tıp uzmanları portalı Rxisk, çeşitli ilaçların bazı vakalarda korona
  virüsleri kapma riskini \%200'e kadar artırabildiğine
  \href{https://rxisk.org/medications-compromising-covid-infections/}{işaret
  ediyor}.
\end{itemize}

\hypertarget{yeni-notlar}{%
\subparagraph{\texorpdfstring{\textbf{Yeni
Notlar}}{Yeni Notlar}}\label{yeni-notlar}}

\begin{itemize}
\tightlist
\item
  İngiliz gazeteci Peter Hitchens,
  \href{https://www.firstthings.com/web-exclusives/2020/04/we-love-big-brother}{``Büyük
  Biraderi Seviyoruz''} başlıklı bir makalede, önceleri eleştiride
  bulunan insanların bile, ortada tıbbi kanıt olmamasına rağmen, nasıl
  ``korkuyla enfekte olduğunu'' anlatıyor. Söyleşide, temel haklar
  tehdit altında olduğu için eleştiri yapmanın
  \href{https://www.spiked-online.com/podcast-episode/in-this-lockdown-dissent-is-a-moral-duty/}{``ahlaki
  bir görev''} olduğunu açıklıyor.
\item
  Alman tarihçi René Schlott,
  \href{https://www.spiegel.de/politik/deutschland/corona-krise-und-buergerrechte-rendezvous-mit-dem-polizeistaat-a-68611322-f4d4-453f-aba5-5ec5a49ae329}{``Polis
  devletiyle randevu''} ile ilgili şunları yazmıştır: ``Bir kitap satın
  almak, parktaki bir bankta oturmak, arkadaşlarla buluşmak, -- ki
  bunlar şu anda yasak -- kontrol altında ve ihbar ediliyor. Demokratik
  hakların korunumu baltalanmış durumda. Bu iş nerede nasıl bitecek?``
\item
  Birçok alman hukuk firması yürürlüğe sokulan önlemler ve kurallara
  karşı davalar açmaya hazırlanıyor. Bir tıp yasası uzmanı
  \href{http://beatebahner.de/lib.medien/aktualisierte\%20Pressemitteilung.pdf}{yaptığı
  basın açıklamasında şöyle demektedir:} ``Federal ve merkezi
  hükümetlerin aldığı önlemler açıkça anayasaya aykırıdır ve
  Almanya'daki yurttaşların bir dizi temel hakkını daha önce olmadığı
  ölçüde çiğnemiştir. Bu, 16 federal eyaletin tüm korona düzenlemeleri
  için geçerlidir. Esas olarak, bu önlemler birkaç gün önce bir çırpıda
  değişitirilmiş olan, Enfeksiyon Korunum Yasası'nca haklı kılınamaz.
  (\ldots{}) Çünkü elimizdeki rakamlar ve istatistikler, korona
  enfeksiyonunun toplumun \%95'inden fazlası için zararsız olduğunu, bu
  nedenle de genel kamuoyuna ciddi bir tehdit oluşturmadığını
  gösteriyor.''
\item
  Alman hükümetine ait dışarıya sızdırılmış bir
  \href{https://fragdenstaat.de/dokumente/4123-wie-wir-covid-19-unter-kontrolle-bekommen/}{gizli
  strateji belgesi} Alman hükümetinin görünüşe göre, medya ve bazı
  biliminsanları ile beraber, insanları bir ``en kötü durum senaryosu''
  `ndan korkutarak bir ``şok stratejisi'' izlediğini göstermektedir.
  Virüsün kendileri için büyük ölçüde zararsız olduğu genel nüfus ``acı
  çekerek boğulma'' `ya karşı uyarılmalıdır; oyun bahçelerinde oynayan
  çocuklar ebeveynlerinin ``acılar içinde ölüm'' `üne yol açabilirler.
\item
  Profesör Sucharit Bhakdi'nin Başbakan Merkel'e yazdığı
  \href{https://swprs.org/open-letter-from-professor-sucharit-bhakdi-to-german-chancellor-dr-angela-merkel/}{Açık
  Mektup} artık Almanca, İngilizce, Fransızca, İspanyolca, Rusça,
  Türkçe, Hollandaca ve Estonyaca olarak mevcuttur ve diğer diller de
  bunları izleyecektir. 
\item
  Amerikan Güvenlik Kurumu'nu (NSA) medyaya ifşa eden Edward Snowden,
  \href{https://www.youtube.com/watch?v=-pcQFTzck_c}{kendisiyle yapılan
  yeni bir söyleşide}, temel hakların uğradığı tahribat ölümcül ve
  kalıcıyken, Kovid-19'un tehlikeli ama geçici olduğu uyarısında
  bulunuyor.
\end{itemize}

\hypertarget{3-nisan-2020}{%
\paragraph{3 Nisan 2020}\label{3-nisan-2020}}

\textbf{ABD }:
\href{https://www.youtube.com/watch?v=5pIMD1enwd4}{Yurttaş gazeteciliği
yapanların çektiği yeni videolar}da da ABD medyası tarafından ``savaş
alanı'' diye tarif edilen hastanelerin aslında hala nasıl çok sakin
olduğu görülüyor.

\textbf{Avusturya}:
\href{https://www.heute.at/s/osterreich-bei-corona-todesstatistik-sehr-liberal-48665863}{Medyada
çıkan bir haberde belirtildiği gibi}, Avusturya'da da ``korona
ölümleri'' sanki ``fazla özgürce'' tanımlanıyor: ``Virüsü kapmışsanız,
ama başka bir şeyden ölürseniz, yine de koronadan ölmüş sayılıyor
musunuz? Avusturya Sağlık Bakanlığı Korona Görev Gücü üyelerinden Rudi
Anschober ve Bernhard Benka bu soruyu, evet, diye yanıtlıyor. ``Şu anda
açık bir kural var: Koronalı ölümler de koronadan ölümler de aynı
istatistik başlığı altında sayılıyor.'' Hastanın gerçekten neden öldüğü
konusunda hiç ayrım yapılmıyor. Başka bir deyişle, femur kemiği boynu
kırığı ile ölen 90 yaşında bir insan, ölümünden birkaç saat önce korona
virüsü kapmış bile olsa koronodan ölmüş sayılıyor. Bu sadece bir
örnek.''

\textbf{Almanya}: Almanya'daki Robert Koch Enstitüsü, aerozollerle
damlacık enfeksiyonu riskinin fazla yüksek olduğunu ileri sürerek,
testleri pozitif çıkmış ölülerin,
\href{https://www.youtube.com/watch?v=gSn_YaOYYcY}{otopsilerinin
yapılmamasını tavsiye ediyor}. Bu ise çoğu vakada, gerçek ölüm nedeninin
artık belirlenemeyeceği anlamına geliyor.

 Bir patoloji uzmanının
\href{https://www.youtube.com/watch?v=gSn_YaOYYcY}{bu konudaki yorumu}
şöyledir: ``Kimin aklına bunda bir kötülük olduğu gelirdi ki! Şu ana
kadar, patoloji uzmanları için; HİV/AİDS, hepatit, tüberküloz, PRİON
hastalıkları, vb. gibi bulaşıcı hastalıklarda bile, uygun güvenlik
önlemleriyle otopsi yapmak sıradan bir işti. Dünyanın dörtbir yanında
binlerce insanı öldüren ve ülkelerin ekonomilerini gerçekten durma
noktasına getiren bir hastalıkta, elimizde yalnızca çok az sayıda otopsi
bulgusunun (Çin'de altı hasta) mevcut olması çok dikkat çekici. Hem
salgın polisinin, hem de bilim dünyasının bakış açısından, otopsi
bulgularına özellikle yüksek düzeyde kamuoyu ilgisi olmalıydı. Halbuki
tam tersi oluyor. Testleri pozitif çıkmış bulunan ölülerin gerçek ölüm
nedenlerini öğrenmekten mi korkuyorsunuz? Otopsiler yapılsaydı koronaya
bağlı bu ölüm sayıları, ilkbahar güneşinde eriyen karlar gibi eriyip
gider miydi yoksa?''

\textbf{İtalya}: Rus uzmanlar, Lombardiya'daki bakımevlerinde
``\href{https://de.sputniknews.com/panorama/20200402326767475-fachpersonal-todesfaelle-lombardei-zeitung/}{garip
ölümler''} olduğunu farkettiler: ``Gazete haberlerine göre, Gromo
kasabasında korona virüslü olduğu söylenen insanların, sadece uyuyup bir
daha uyanmadıkları birkaç vaka kaydedilmişti. Ölenlerde o zamana dek
hastalığın hiçbir gerçek belirtisi gözlenmemişti. (\ldots{}) Bakımevi
müdürünün daha sonra, Rusya'nın resmi haber ajansı RIA Novosti'deki bir
söyleşide netleştirdiği gibi, ölenlerin gerçekten korona virüsüyle
enfekte olup olmadıkları belirsizdir, çünkü bakımevindeki kimse daha
önce teste tabi tutulmuş değildir. (\ldots{}) Rusya'dan gelen doktor ve
hemşire ekiplerinin çalıştığı bakımevlerinde, koridorlar, yatak odaları
ve yemek odaları dezenfekte edilmektedir.''

Benzer vakalara ait haberler Almanya'dan da gelmiştir: Bakımevinde bu
hastalığın belirtilerini göstermeyen hastalar, mevcut sıradışı durumda,
birden bire ölüyor ve sonra da ``korona ölüleri'' olarak kabul ediliyor.
Burada da yine şu ciddi soru ortaya çıkmaktadır: Kim virüs nedeniyle
ölüyor, kim bazen aşırıya kaçan önlemlerden ölüyor?

\textbf{Bakım hizmetlileri}: Süddeutsche Zeitung gazetesi
\href{https://www.sueddeutsche.de/politik/coronavirus-pflegekraefte-ausland-1.4866124}{verdiği
habere göre}: ``Avrupa'nın dört bir yanında bakım hizmetlilerinin artık
yaşlıları ziyaret edememesi -- ya da çalıştıkları ülkeleri telaşla
terkederek evlerine dönmüş olmaları nedeniyle pandemi, yaşlıların evde
bakımını tehdit ediyor.''

\textbf{Son olarak}: Stanford Üniversitesi Tıp profesörü Dr. Jay
Bhattacharya kendisiyle ile
\href{https://www.youtube.com/watch?v=-UO3Wd5urg0}{yapılan yarım saatlik
söyleşide} Kovid-19'a ilişkin ``konvansiyonel (klasik) bilgeliği''
sorgulamaktadır. Şu anki önlemlere, çok belirsiz ve kısmen güvenilir
olmayan veriler esas alınarak karar verilmiştir.

\hypertarget{2-nisan-2020}{%
\paragraph{2 Nisan 2020}\label{2-nisan-2020}}

\hypertarget{abd}{%
\subparagraph{\texorpdfstring{\textbf{ABD}}{ABD}}\label{abd}}

Biyofizikçi Scholkmann, ABD'de (dünyanın geri kalanında olduğu gibi) git
gide hızlanarak artan şeyin, ``enfekte olmuş'' insanların sayısı değil,
testlerin sayısı olduğu gerçeğini
\href{https://swprs.org/rate-of-positive-covid19-tests/}{görselleştirmişti}.
``Enfekte olmuş'' insan sayısının test sayısına oranı (\%10 ile 20
arasında salınım göstererek) sabit kalmaktadır ki bu da mevcut bir viral
salgın iddiasının aksini gösterir.

\includegraphics{https://swprs.files.wordpress.com/2020/04/ud-data-2-fs.png?w=736}

\hypertarget{almanya}{%
\subparagraph{\texorpdfstring{\textbf{Almanya}}{Almanya}}\label{almanya}}

Almanya'daki Robert Koch Enstitüsü'nün
\href{https://influenza.rki.de/Wochenberichte/2019_2020/2020-13.pdf}{e}\href{https://influenza.rki.de/Wochenberichte/2019_2020/2020-13.pdf}{n}\href{https://influenza.rki.de/Wochenberichte/2019_2020/2020-13.pdf}{s}\href{https://influenza.rki.de/Wochenberichte/2019_2020/2020-13.pdf}{on
yayınlanan grip raporu}'na göre, akut solunum yolları enfeksiyonlarının
sayısı ülke çapında ``keskin bir düşüş göstermiştir. ``Tüm yaş
gruplarına'' ait değerler düşmüştür.

20 Mart'a gelindiğinde, hastaneye yatan akut solunum yolları vakalarının
toplamı da önemli ölçüde azalmıştır. 80 yaş ve üstü grubunda, vaka
sayısı önceki haftaya oranla neredeyse yarıya inmiştir.

İncelenen 73 hastanede, bütün solunum yolları vakalarının \%7'sine
Kovid-19 tanısı konmuştur. Kovid-19 tanısı oranı 35-59 yaş arası
gruplarda \%16, 60-79 yaş arası gruplarda ise \%13'tür.

Bu rakamlar, diğer ülkelerdeki rakamlarla da korona virüslerinin tipik
görülme oranına (\%5 ile 15) da uymaktadır.

\href{https://swprs.files.wordpress.com/2020/04/rki-ili-kw13.png}{}

\includegraphics{https://swprs.files.wordpress.com/2020/04/rki-ili-kw13.png?w=279\&h=171}

Chřipková onemocnění (RKI, 13.kal. týden)

\href{https://swprs.files.wordpress.com/2020/04/rki-sari-kw12.png}{}

\includegraphics{https://swprs.files.wordpress.com/2020/04/rki-sari-kw12.png?w=449\&h=171}

Akutní onemocnění dýchacích cest v nemocnicích

Genel olarak grip-benzeri hastalıklar (solda) ve hastanelerdeki akut
solunum yolu hastalıkları (sağda) (Robert-Koch-Enstitüsü, 13. ve 12.
haftalar)

``Şu anda politikacılar, uzmanlar ve birçok yurttaş, enfekte olan
insanların her gün artmakta olan sayısını endişeyle izliyor. Yine de
korona krizinin ne kadar kötü olduğunu ve Almanya'yı ne zaman vuracağını
değerlendirmek için gereken nihai gösterge bu değildir. Çünkü bu
rakamlar başından beri, haftalardır artmakta olan test sayısı nedeniyle
çarpıtılmış durumdadır.

Sağlık sistemi üzerindeki yükün ölçülebilmesi için, suni solunuma
ihtiyaç duyacak kadar ciddi hasta olanların sayısı özellikle önemlidir.
Yeterli suni solunum kapasitesi bulunduğu sürece, bu kişilerin çoğunluğu
kurtarılabilir. Yalnızca bu yataklar azaldığında, İtalya'daki gibi bir
durum tehdit oluşturabilir.

DIVI kayıtları, şu anda Almanya'daki yoğun bakım ünitelerindeki durumun
şimdilik rahatladığını gösteriyor. Grabenhenrich, ``Hala rahat bir
durumdayız,'' demektedir. Ciddi hastaların sayısı, enfekte olan hastalar
kadar büyük bir hızla artmıyor, öyle de olsa çok iyi ekipmana sahip çok
sayıda yoğun bakım yatağı sunulması yine de mümkün olurdu.''

\hypertarget{isviuxe7re-1}{%
\subparagraph{\texorpdfstring{\textbf{İsviçre}}{İsviçre}}\label{isviuxe7re-1}}

İsviçre Kamu Sağlığı Ofisi, şu ana kadar \%15'inin sonucu pozitif çıkan,
yaklaşık 139.330 Kovid-19 testi yapılmış olduğunu
\href{https://www.bag.admin.ch/bag/de/home/krankheiten/ausbrueche-epidemien-pandemien/aktuelle-ausbrueche-epidemien/novel-cov/situation-schweiz-und-international.html}{bildiriyor}.
Bu sayı da diğer ülkelerin bilinen değerlerine uygun olup, İsviçre'de de
artıyor gibi görünmemektedir.

Sadece medyada sık sık söz edilen test sayısı git gide hızlanarak
artıyor, ama ``enfekte olan'', hasta ve hatta ölenlerin sayısı artmıyor.

Buna karşın, 31 Mart'ta yeni bir
\href{https://www.bfs.admin.ch/bfs/de/home/statistiken/gesundheit/gesundheitszustand/sterblichkeit-todesursachen.html}{haftalık
ölüm sayısı istatistiği} yayınlanmıştır ve bu istatistikte ilk kez, 12.
takvim haftası için (22 Mart'a kadar) İsviçre'deki 65 yaş üstü grubun
toplam ölüm sayısının artacağı tahmini yapılmaktadır (bkz. aşağıdaki
grafik). Özellikle, toplam ölüm sayısının haftada 200'e kadar artması
beklenmektedir.

Federal Ofis'e göre, bu artış ``mevcut salgının bir göstergesi'' 'dir.
Burada şu sorun ortaya çıkıyor: 22 Mart'a kadar İsviçre'de toplam 106
pozitif-testli ölüm vardır. Haftada 200 ölüm gibi bir artış, ek
ölümlerin büyük bölümünün virüsten değil, ``karşı önlemler'' `den
kaynaklandığı anlamına gelir.

Bir diğer açıklama ise bir sonraki haftaya (13. hafta) ait yaklaşık 200
pozitif-testli ölümün zaten hesaba katıldığıdır. Bu ise bütün
pozitif-testli ölümlerin ek ölümler olarak kabul edildiği anlamına
gelir. Halbuki, yaş ve hastalık profiline olduğu gibi
\href{https://swprs.org/rki-relativiert-corona-todesfaelle/}{uluslalarası
deneyime} de bakıldığında, bu çok kuşkulu bir varsayım olacaktır.

Nitekim de bu sorumluluğu reddetmek üzere rapora şu satırlar
eklenmiştir: ``Bu ilk tahminler hala çok şüphelidir, bu nedenle de
hiçbir kesin rakam yayınlanamaz''.

Eğer bu pozitif-testli ölümlerin büyük bir bölümünün (yaş ortalaması:
83) ek ölümler olmadığı ortaya çıkarsa, ya toplam ölüm sayıları
artmayacak ya da
\href{https://swprs.org/open-letter-from-professor-sucharit-bhakdi-to-german-chancellor-dr-angela-merkel/}{bazı
uzmanların korktuğu} gibi, aslında alınmış sert önlemler yüzünden artmış
olacaktır.

\includegraphics{https://swprs.files.wordpress.com/2020/04/bfs-mortaliaet-22-03.png?w=600\&h=400}

İsviçre'de bir gazete, mevcut toplam ölüm sayılarını önceki yıllarla
karşılaştırmalı olarak göstermiştir (bkz. aşağıdaki grafik). Bu ise
mevcut ölüm sayısının, eğer gerçekten artmış olsa bile, hala geçmiş
yıllarda şiddetli griplerin olduğu kış mevsimi rakamlarının altında
olduğunu gösteriyor.

\includegraphics{https://swprs.files.wordpress.com/2020/04/mortalitc3a4t-schweiz.png?w=720\&h=339}

\hypertarget{ek-bilgi}{%
\subparagraph{\texorpdfstring{\textbf{Ek
bilgi}}{Ek bilgi}}\label{ek-bilgi}}

\begin{itemize}
\tightlist
\item
  Büyük Britanya'ya gitmek üzere yola çıkmış olan virüs test kitleri,
  korona virüsü bileşenleri içerdikleri için
  \href{https://www.telegraph.co.uk/news/2020/03/30/uks-attempt-ramp-coronavirus-testing-hindered-key-components/}{iade
  edilmek zorunda kalınmıştır}.
\item
  Yüzbinlerce fazladan ölüm olacağı tahmininde bulunan, ama herhangi bir
  dergi tarafından yayınlanmamış veya gözden geçirilmemiş olan, British
  Imperial College'a ait inceleme, şu anda artık görüldüğü gibi,
  \href{https://judithcurry.com/2020/04/01/imperial-college-uk-covid-19-numbers-dont-seem-to-add-up/}{büyük
  ölçüde gerçekçi olmayan varsayımlara dayanıyor}.
\item
  BBC, ``\href{https://www.bbc.com/news/health-51979654}{Ölümlere Korona
  virüsü mü yol açıyor?}'' sorusunu sormuş ve şu yanıtı vermiştir:
  ``Ölümlerin esas nedeni olabilir, katkıda bulunan bir etken olabilir
  ya da sadece insanlar başka bir şey den ölürlerken vücutlarında
  bulunuyor olabilir.'' Örneğin, 18 yaşında bir erkek ölümünden bir gün
  önce yapılan virüs testi pozitif çıktıktan sonra, ``en genç korona
  kurbanı'' diye, haber yapılmıştı. Halbuki, hastane daha sonra bu
  gencin önceden var olan ciddi bir hastalıktan öldüğünü açıkladı.
\item
  Avrupa sağlık otoritesi ECDC, testleri pozitif olan ya da ``pozitif
  olduğu varsayılan'' cesetlerin ele alınmasına yönelik
  \href{https://www.ecdc.europa.eu/sites/default/files/documents/COVID-19-safe-handling-of-bodies-or-persons-dying-from-COVID19.pdf}{çok
  sıkı kurallar} yayınlamıştır. Bugüne dek seyreden çok düşük ölüm
  oranları göz önüne alındığında, bu tür kurallar tıbbi açıdan
  sorgulanabilir görünüyor. Buna rağmen, sağlık ve cenaze işleri
  üzerindeki yükü önemli ölçüde artırmış, aynı zamanda da medyada
  şiddetli bir etki yapmıştır.
\item
  Almanya'da devlete ait bir medya kuruluşu, Profesör Sucharit
  Bhakdi'nin Başbakan Merkel'e yazdığı Açık Mektup'u
  \href{https://www.br.de/nachrichten/wissen/bhakdis-brief-an-die-kanzlerin-was-ist-dran-an-seinen-fragen,RutYDhd}{eleştiren
  bir yorum} yayınlamıştır.
\item
  2009 yılında yayınlanan ARTE televizyonuna ait
  ``\href{https://vimeo.com/403175258}{Korku Tacirleri}'' adlı belgesel;
  ağırlıklı olarak özel sektörce finanse edilen Dünya Sağlık Örgütü
  WHO'nun, birkaç milyar dolar değerindeki aşılar dünyanın dört bir
  yanındaki devletlere satılabilsin diye, (``domuz gribi'' denilen)
  hafif bir grip dalgasını nasıl küresel bir pandemi düzeyine
  ``yükselttiğini'' anlatıyor. O zamanın başkahramanlarının bazıları, şu
  anki durumda da yine
  \href{https://www.nature.com/articles/news.2009.424}{belirgin biçimde
  rol oynamaktadır}.
\item
  Britanya Yüksek Mahkemesi eski yargıcı Jonathan Sumption,
  \href{https://www.spectator.co.uk/article/former-supreme-court-justice-this-is-what-a-police-state-is-like-}{BBC
  ta}\href{https://www.spectator.co.uk/article/former-supreme-court-justice-this-is-what-a-police-state-is-like-}{rafından
  yapılan bir söyleşide} İngiltere'deki önlemler konusunda, ``İşte bir
  polis devleti tam da buna benzer'' açıklamasında bulunmuştur.
\end{itemize}

\hypertarget{-2-nisan-2020-ii}{%
\paragraph{ 2 Nisan 2020 (II)}\label{-2-nisan-2020-ii}}

\begin{itemize}
\tightlist
\item
  Guardian gazetesi 2018 yılında bile şöyle yazıyordu:
  ``\href{https://www.theguardian.com/society/2018/dec/09/steep-rise-lung-related-illness-hospitals-nhs}{Kirlilik
  ve grip, akciğer hastalıklarında hızlı bir yükselişe yol açıyor''}.
  Solunum yolları hastalıklarındaki tırmanışın hastane acil servisleri
  üzerinde oluşturduğu baskıdan duyulan endişeye bir de uzman doktor
  eksikliği eklenmiştir.
\item
  Mikrobiyoloji, viroloji ve enfeksiyon epidemiyolojisi uzmanı Profesör
  Martin Haditsch,
  \href{https://www.youtube.com/watch?v=PtzHH8DhgZM}{Kovid-19
  önlemlerini şiddetle eleştiriyor}. Bunlar ``tümüyle temelsiz'' olup
  ``sağduyulu muhakemeyi ve etik ilkeleri ayaklar altına almaktadır''.
\item
  Almanya'daki bakımevlerinin temsilcileri bile şimdi kısıtlayıcı
  önlemlerden ve Kovid-19 ile ilgili uygunsuz medya haberlerinden
  \href{https://pflege-prisma.de/2020/03/31/sterbezahlen-in-pflegeheimen/}{yakınıyor}.
\item
  Kuzey İtalya'daki Treviso kentinden (Venedik yakınında) gelen
  rakamlar, Mart ayı sonuna kadar olan 109 pozitif-testli ölüme rağmen,
  belediye hastanelerindeki toplam ölüm sayısının önceki yıllarla
  \href{https://swprs.files.wordpress.com/2020/04/reppublica-treviso.jpg}{kabaca
  aynı kaldığını} gösteriyor. Bu ise bazı yerlerde geçici olarak
  yükselmiş olan ölüm sayılarının, tek başına korona virüsünden
  kaynaklamaktan çok, olasılıkla panik ve çöküş gibi dış etkenlere bağlı
  olduğunun ek bir göstergesidir.
\item
  Londra'daki Queen Mary Üniversitesi'nden ve dünyanın önde gelen
  virolog ve grip uzmanlarından Profesör John Oxford,
  \href{https://novuscomms.com/2020/03/31/a-view-from-the-hvivo-open-orphan-orph-laboratory-professor-john-oxford/}{Kovid-19
  ile ilgili şu sonuca} varmıştır: ``Kişisel olarak, en iyi tavsiyenin
  TV haberlerini izlemeye daha az zaman ayırmak olduğunu söyleyebilirim,
  onlar sansasyonel ve pek iyi değil. Bana göre, bu Kovid-19 salgını,
  kötü bir kış gribi salgınından farksız. 
\item
  Geçen yıl bu durumda `riskli' gruplarda 8.000 ölüm yaşamıştık. Yani
  \%65'ten fazlasının kalp vb. hastalığı bulunan gruplarda. Mevcut
  Kovid-19'un bu rakamı aşacağını tahmin etmiyorum. Bir medya
  salgınından muzdaribiz!''
\end{itemize}

\hypertarget{1-nisan-2020}{%
\paragraph{1 Nisan 2020}\label{1-nisan-2020}}

\hypertarget{italyadaki-duruma-dair}{%
\subparagraph{\texorpdfstring{\textbf{İtalya'daki Duruma
Dair}}{İtalya'daki Duruma Dair}}\label{italyadaki-duruma-dair}}

İtalyan doktorlar, geçtiğimiz yılın sonunda İtalya'nın kuzeyinde ciddi
zatürre vakalarına
\href{https://www.scmp.com/news/china/society/article/3076334/coronavirus-strange-pneumonia-seen-lombardy-november-leading}{rastlamış
olduklarını} bildirmişlerdir. Buna rağmen, genetik analizler şu anda
Kovid-19 virüsünün İtalya'da sadece bu yılın Ocak ayında ortaya
çıktığını gösteriyor.
\href{https://www.nzz.ch/wissenschaft/coronavirus-der-stammbaum-verraet-woher-es-kommt-ld.1548271}{Bir
viroloğa göre}, ``Bu yüzden İtalya'da Kasım ve Aralık aylarında tanısı
konulan ciddi zatürre vakalarının başka bir patojenden kaynaklanmış
olması gerekir.'' Bu ise yine, Kovid-19 virüsünün veya diğer etkenlerin
İtalya'da yaşananlarda gerçekten nasıl bir rol oynadığı sorusunu ortaya
getirmektedir.

30 Mart'ta, ``Korona krizi sırasında'' ölen ve çoğunun yaşı 90'a yakın
olup bu krizde hiç aktif göreve katılmamış İtalyan doktorların
listesinden söz etmiştik. Bugün, listedeki tüm doğum tarihleri
\href{https://portale.fnomceo.it/elenco-dei-medici-caduti-nel-corso-dellepidemia-di-covid-19/}{kaldırılmış
durumdadır}, (yine de listenin son
\href{https://web.archive.org/web/20200328152430/https://portale.fnomceo.it/elenco-dei-medici-caduti-nel-corso-dellepidemia-di-covid-19/}{arşiv
versiyonu} burada görülüyor). Garip bir süreç.

İtalya'daki, açıkçası bir virüsten çok daha fazlasından kaynaklanan,
çarpıcı duruma ilişkin daha çok ayrıntı veren bir gözlemciden aşağıdaki
mesajı da almış bulunuyoruz:

``Son haftalarda, İtalya'da bakıma muhtaç insanları destekleyen ve 7/24
çalışan doğu Avrupalı hemşirelerin çoğu telaşla ülkeyi terketti. Bu
durumda ``acil durum yönetimi'' `nin panik salması, sokağa çıkma
yasakları ve sınır kapatma tehditlerinin etkisi hiç de azımsanamaz.
Sonuç olarak, kiminin akrabası bile bulunmayan bakıma muhtaç yaşlılar ve
engelliler bakıcıları tarafından çaresiz bırakıldı.

Terkedilmiş bu insanların çoğu bir kaç gün sonra, farklı nedenlerle su
kaybettikleri için kendilerini yıllardır sürekli aşırı yük altında
çalışan hastanelerde buldular. Hastaneler ise ne yazık ki, okullar ve
anaokulları kapatılmış olduğu için apartman dairelerinde kapalı kalan
çocuklarına bakmak zorunda olan personelden de mahrum kaldı. Bu da
özellikle daha sıkı ``önlemler'' `in zorunlu kılındığı bölgelerde
engelli ve yaşlı bakımının tamamen çökmesine ve kaotik koşullara yol
açtı.

Hastanelerde bakım konusundaki paniğin neden olduğu bu acil durum,
bakıma muhtaç olanlar ve gittikçe de daha genç hastalar arasında geçici
olarak birçok ölüme yol açtı. Bu ölümler ise dizi dizi askeri kamyon ve
tabut fotoğraflarıyla, örneğin ``475 ölüm daha'', ``Ölüler hastanelerden
ordu tarafından alınıyor'' diye haber yapan medyada ve sorumlular
arasında daha da fazla panik yaratılmasına hizmet etti.

Bu durum, cenaze işleri görevlilerinin ``öldürücü virüs'' korkusunun ve
bu nedenle hizmet vermeyi reddetmesinin bir sonucuydu. Dahası, bir
taraftan aynı anda çok fazla ölüm olurken, diğer taraftan da hükümet
korona virüslü cesetlerin yakılmasını zorunlu tutan bir yasa geçirdi.
Katolik İtalya'da geçmişte çok az sayıda ölü yakma uygulaması
yapılmıştı. Bu yüzden az sayıdaki küçük krematoryumların hemen
kapasiteleri doldu. Bu nedenle ölülerin farklı kiliselerde gömülmesi
gerekti.

Aslında, bu gelişmeler tüm ülkelerde aynıdır. Fakat sağlık sisteminin
niteliği yaşananlarda önemli bir etkiye sahiptir. İşte bu yüzden
Almanya, Avusturya veya İsviçre'de İtalya, İspanya veya ABD'ye göre daha
az sorun vardır. Buna karşın, resmi rakamlarda görülebileceği gibi, ölüm
oranlarında önemli bir artış olmayıp yalnızca bu trajediden kaynaklanan
bir doruk olmuştur.''\\

\hypertarget{abd-almanya-ve-isviuxe7redeki-hastanelerin-durumu}{%
\subparagraph{\texorpdfstring{\textbf{ABD, Almanya ve İsviçre'deki
Hastanelerin
Durumu}}{ABD, Almanya ve İsviçre'deki Hastanelerin Durumu}}\label{abd-almanya-ve-isviuxe7redeki-hastanelerin-durumu}}

\begin{itemize}
\tightlist
\item
  Amerikan televizyon kanalı CBS, New York'taki mevcut duruma ait bir
  haberinde bir İtalyan yoğun bakım ünitesinden görüntüler kullanırken
  \href{https://www.theblaze.com/news/cbs-news-footage-italy-hospital-nyc}{yakalandı}.
  Gerçekte, \href{https://www.youtube.com/watch?v=K0z8NhxNTaU}{yurttaş
  gazeteciliği yapanların çektiği düzinelerce
  görüntü}\href{https://www.youtube.com/watch?v=K0z8NhxNTaU}{de} ABD'nin
  Doğu Yakası'nda ve Batı Yakası'ndaki hastanelerin şu sıralarda
  \href{https://twitter.com/mil_ops/status/1244286621475844097}{çok
  sakin} olduğu görülüyor. Medyada öne çıkarılan ``soğutmalı ceset
  kamyonları'' bile kullanılmadan boş durmaktadır. 
\item
  Medya raporlarının aksine, Almanya'daki yoğun bakım ünitelerinin
  kayıtları da
  \href{https://www.divi.de/register/intensivregister}{doluluk
  oranlarında artış göstermiyor}. Yurttaş gazeteciliği yapanlar
  Berlin'deki kliniklerde
  \href{https://www.youtube.com/watch?v=WiJszJmGdxY}{tümüyle terkedilmiş
  Kovid-19 kabul merkezlerini} ziyaret etmişlerdir. Münih'teki bir
  klinik çalışanı, ``haftalardır dalganın vurmasını'' beklediklerini,
  ama ``hasta sayısında bir artış olmadığını'' belirtmiş,
  politikacıların açıklamalarının kendi yaşadıkları deneyimle
  örtüşmediğini ve ``öldürücü virüs efsanesi'' `nin ``teyit edilmediği''
  `ni söylemiştir. 
\item
  İsviçre'deki kliniklerde de şu ana kadar doluluk oranında bir artış
  gözlenmemektedir. Lüzern'deki kanton hastanesine giden bir ziyaretçi
  ``normal zamankinden daha az aktivite'' olduğunu belirtmiştir.
  Kliniğin bazı katları tümüyle Kovid-19 için kapatılmışsa da çalışanlar
  ``hala hastaların gelmesini bekliyor'' demiştir. Bern, Bazel, Zug ve
  Zürih'teki hastaneler de ``boşaltılıp hazırlanmıştır''. Ticino'da
  (Çev. Notu: İsviçre'nin İtalyan kesiminde bulunan) bile, yoğun bakım
  üniteleri
  \href{https://www.nzz.ch/schweiz/tessin-verlegt-erste-corona-patienten-in-deutschschweizer-spitaeler-ld.1549417}{kapasitelerinin
  altında çalışırken}, hastalar şimdi İsviçre'nin Alman kesiminde
  bulunan birimlere gönderilmektedir. Bu durum tıbbi bir bakış açısından
  hiç anlamlı değildir.
\end{itemize}

\hypertarget{diux11fer-tux131bbi-notlar}{%
\subparagraph{\texorpdfstring{\textbf{Diğer Tıbbi
Notlar}}{Diğer Tıbbi Notlar}}\label{diux11fer-tux131bbi-notlar}}

\begin{itemize}
\tightlist
\item
  Hamburg Üniversitesi Tıp Merkezi yöneticisi Dr. Ansgar Lohse,
  \href{https://www.mopo.de/hamburg/uke-infektiologe-fordert-es-muessen-sich-mehr-menschen-mit-corona-infizieren-36483636}{sokağa
  çıkma yasağı ve temas yasaklarının acilen kaldırılmasını talep
  ediyor}. Daha fazla insanın korona ile enfekte olması gerektiğini
  savunuyor. Anaokulları ve okullar en kısa zamanda açılmalıdır ki
  çocuklar ve ebeveynleri korona virüsüyle enfeksiyon yoluyla bağışıklık
  geliştirebilsinler. Doktor Lohse, katı önlemlerin devam ettirilmesinin
  can kayıplarına da neden olabilecek bir ekonomik krize yol açacağını
  belirtmiştir.
\item
  İspanya'da
  \href{https://www.heise.de/tp/features/Das-ist-keine-Krise-sondern-eine-Katastrophe-4694104.html}{testleri
  pozitif çıkan kişilerin \%15'i doktor ve hemşirelerdir}. Çoğunda
  belirti olmasa da karantina altına alınmaları gerekmekte, bu da
  İspanyol sağlık sisteminin çökmesine yol açmaktadır. 
\item
  Patoloji dalında çalışan Emeritus Profesör Dr. John Lee, İngiliz yayın
  organı Spectator'daki son derece yanıltıcı ``korona ölümleri''
  tanımlaması ve iletişimi konusunda
  \href{https://www.spectator.co.uk/article/how-to-understand-and-report-figures-for-covid-19-deaths-}{yazıyor}.
\item
  Çevresel toksikoloji dalında doktora sahibi bir uzman tarafından
  yorumlanmış
  \href{https://swprs.files.wordpress.com/2020/04/die-lage-in-norwegen.pdf}{Norveç'ten
  gelen en son veriler} de yine -- bir salgın durumunda bekleneceği gibi
  -- testleri pozitif çıkan kişi oranlarının artmadığını, fakat korona
  virüsleri için geçerli normal aralık olan \%2 ile \%10 arasında
  değiştiğini gösteriyor. Testleri pozitif olup ölenlerin yaş ortalaması
  84 olup, ölüm nedenleri halka açıklanmamıştır, ölüm oranlarında da
  fazlalık yoktur. 
\item
  Şu ana kadar (Japonya veya Güney Kore gibi Asya ülkeleriyle benzer
  biçimde) radikal önlemler olmadan idare etmiş olan ve ölüm oranlarında
  artış bildirmeyen İsveç, stratejisini değiştirmesi için uluslararası
  medya tarafından dikkat çekici bir biçimde
  \href{https://www.theguardian.com/world/2020/mar/30/catastrophe-sweden-coronavirus-stoicism-lockdown-europe}{baskıya
  uğramaktadır}. 
\item
  New York Eyaleti'nden gelen veriler, testleri pozitif çıkan bireylerin
  hastanede tedaviye alınma oranının başlangıçta varsayılana göre
  \href{https://www.nytimes.com/2020/03/27/nyregion/new-rochelle-coronavirus.html}{yirmi
  kattan daha düşük} olduğunu gösteriyor.
\item
  \href{https://www.doccheck.com/de/detail/articles/26271-covid-19-beatmung-und-dann}{Tıp
  uzmanları portalı DocCheck'te yer alan bir makale,} ``Testleri pozitif
  çıkan hastaların suni solunum sorunu'' 'nu ele almıştır. Resmi öneri,
  korona virüsünün havanın strerilizasyonu (mikroplardan arındırma) için
  kullanılan aerozollerle (havada asılı kalan çok küçük tanecikler)
  yayılmasını önlemek dahil farklı nedenler yüzünden, testleri pozitif
  çıkan hastalara maskeyle basit suni solunum yapılmaması yönündedir. Bu
  nedenle, testleri pozitif çıkan yoğun bakım hastaları çoğunlukla
  doğrudan entübe (tüp/hortum yerleştirilmesi) edilmektedir. Buna
  karşın, entübasyonun başarı oranı düşüktür ve çoğunlukla genelde
  akciğerlerde (suni solunum cihazı-kaynaklı akciğer hasarı diye tabir
  edilen) ilave hasarlara yol açar. İlaçlarla ilgili olarak da hastalara
  daha yumuşak bir tedavi uygulanmasının tıbbi olarak daha mantıklı olup
  olmadığı sorusu da ortaya çıkmaktadır.
\end{itemize}

\hypertarget{politik-geliux15fmelere-dair}{%
\subparagraph{\texorpdfstring{\textbf{Politik Gelişmelere
Dair}}{Politik Gelişmelere Dair}}\label{politik-geliux15fmelere-dair}}

\begin{itemize}
\tightlist
\item
  Bir Alman devlet bakanı,
  \href{https://de.nachrichten.yahoo.com/strobl-bürger-verstöße-gegen-corona-regeln-polizei-melden-095746341.html}{halka
  şu çağrıda bulunmuştur}: ``Uyanık olun ve korona salgınını sınırlamaya
  yönelik kuralları çiğneyenleri polise ihbar edin''.
  ``\href{https://www.br.de/nachrichten/bayern/buerger-melden-eifrig-verstoesse-gegen-corona-regeln,RuGXp1h}{Hevesle
  ihbar edilenler}'' örneğin ``grup olarak dolaşanlar, oyun
  bahçelerindeki çocuklar, partiler'' ve yürüyüşçülerdir. 
\item
  Alman anayasa hukuku uzmanları ``temel haklara ciddi tecavüzler
  olduğu'' konusunda
  \href{https://www.focus.de/politik/deutschland/corona-regelungen-der-regierung-medizin-darf-nicht-gefaehrlicher-sein-als-die-krankheit_id_11827625.html}{alarm
  vermektedir}. Anayasa hukuku uzmanı Hans Michael Heinig, ``demokratik
  anayasal devletin birden bire faşist-isterik bir hijyen devletine
  dönebileceği'' uyarısında bulunuyor. Berlin'deki Humboldt
  Üniversitesi'nden Profesör Christoph Möllers'e göre, enfeksiyondan
  korunma yasası, ``yurttaşların özgürlük hakkına yönelik bu denli uzun
  boylu kısıtlamalara temel oluşturmak için kullanılamaz''. Alman
  Federal Anayasa Mahkemesi eski başkanı Hans Jürgen Papier'e göre,
  ``Acil durum önlemleri otoriter bir gözetleme devleti yararına sivil
  özgürlüklerin askıya alınmasını haklı kılmaz''. 
\item
  Birçok ülkede sokağa çıkma yasaklarına ve temel haklara yapılan diğer
  tecavüzlere son verilmesi için ``online'' imza kampanyaları
  başlatılmıştır. Aynı zamanda, eleştirel videolarla sunulan katkılar,
  doktorlar tarafından yapılmış bile olsa, gittikçe daha fazla
  silinmektedir. Berlin'de, Alman anayasasının savunulduğu ``temel
  haklar'' konulu izinli bir etkinlik,
  \href{https://kenfm.de/berliner-corona-demo-strafbar-aufgeloest-aber-froehlich/}{polis
  tarafından dağıtılmıştır}.
\end{itemize}

\hypertarget{31-mart-2020-i}{%
\paragraph{31 Mart 2020 (I)}\label{31-mart-2020-i}}

Dr. Richard Capek ve diğer araştırmacılar, şu ana kadar incelenen tüm
ülkelerde, testleri pozitif çıkmış insan sayısının, yapılmış test
sayısına oranının sabit kalmaya devam ettiğini
\href{https://coronadaten.wordpress.com/}{zaten göstermiştir}. Bu ise
virüsün eksponansiyel yayılımı (``salgın'') olduğunun aksini ve yalnızca
yapılan test sayısındaki eksponansiyel bir artışı göstermektedir.

Ülkesine bağlı olarak, testleri pozitif çıkan bireylerin oranı, korona
virüslerinin her zamanki yayılımına denk düşen, \%5 ile 15 arasındadır.
İlginçtir ki sabit sayısal değerler yetkililer ve medya tarafından aktif
olarak halka iletilmemekte
(\href{https://multipolar-magazin.de/artikel/coronavirus-irrefuhrung-fallzahlen}{veya
ortadan bile kaldırılmaktadır}). Bunun yerine, ilgisi olmayan ve
yanıltıcı eksponansiyel eğriler herhangi bir bağlam olmaksızın
gösterilmektedir.

Bu tür davranışlar; Almanya'daki Robert Koch Enstitüsü'nün geleneksel
\href{https://influenza.rki.de/Saisonberichte/2017.pdf}{grip raporu}na
kısaca göz atıldığında açıkça anlaşılacağı gibi (bkz. s. 30, alttaki
grafik), tabii ki profesyonel tıp standartlarına uymamaktadır. Bu
grafikte, tanı adedine ek olarak (sağda), alınan örneklerin sayısı
(solda, gri çubuklar) ve pozitif çıkma oranı (solda, mavi eğri)
gösterilmektedir.

Bu grafik, bir grip mevsiminde alınan örneklerde, pozitif çıkma oranının
0'dan \%10'a oradan da \%80'e kadar yükseldiğini ve birkaç hafta sonra
da normal değere düştüğünü göstermektedir. Kovid-19 testleri bununla
karşılaştırıldığında, normal aralıkta kalan sabit bir ``pozitif çıkma
oranı'' gösteriyor (bkz. en alttaki grafik).

\includegraphics{https://swprs.files.wordpress.com/2020/03/rki-influenza-report-2017.png?w=650\&h=530}

Örnek olarak ABD verileri kullanılmış olan ``Kovid19-pozitif oranı''
(Dr. Richard Capek).\\
Bu oran, şu anda ``örneklerin sayısına ait verileri'' mevcut olan tüm
diğer ülkeler için de benzer şekilde geçerlidir.

\includegraphics{https://swprs.files.wordpress.com/2020/03/infizierte-pro-test2603.jpg?w=600\&h=325}

\hypertarget{31-mart-2020-ii}{%
\paragraph{31 Mart 2020 (II)}\label{31-mart-2020-ii}}

\begin{itemize}
\tightlist
\item
  \href{https://off-guardian.org/2020/03/30/covid19-yet-to-impact-europes-overall-mortality/}{Avrupa
  izleme verilerinin grafik bir analizi} etkileyici bir biçimde şunu
  göstermiştir: Avrupa çapında genel ölüm sayıları, alınan önlemlerden
  bağımsız olarak, 25 Mart gününe kadar normal aralıkta veya altında,
  çoğunlukla da önemli ölçüde önceki yıllardaki düzeylerin altında
  kalmıştır. Yalnızca İtalya'da (65+) genel ölüm oranı bir miktar
  (olasılıkla birkaç nedenden dolayı) artmıştı, ama hala bir önceki grip
  mevsiminden daha düşüktü.
\item
  Alman Robert Koch Enstitüsü başkanı, daha önceden var olan koşulların
  ve gerçek ölüm nedenlerinin sözde ``korona ölümleri'' tanımında
  \href{https://swprs.org/rki-relativiert-corona-todesfaelle/}{bir rol
  oynamadığını} yeniden teyit etmiştir. Tıbbi açıdan, böyle bir
  tanımlama açıkça yanıltıcıdır. Beklenen ve genel olarak bilinen etkisi
  ise siyasete ve topluma korku aşılamaktır.
\item
  İtalya'da durum şimdi
  \href{https://www.tagesspiegel.de/politik/die-verlangsamung-ist-da-in-italien-zeichnet-sich-die-wende-in-der-coronakrise-ab/25698124.html}{sakinleşmeye
  başlıyo}r. Bilindiği kadarıyla, geçici olarak artan ölüm oranları
  (65+) çoğunlukla kitlesel panik ve sağlık hizmetlerinde bir çöküşle
  beraber gelen, daha ziyade yerel etkilerdir. Kuzey İtalya'dan bir
  politikacı örneğin şunu sormaktadır, ``Brescia'daki Kovid hastaları
  Verona yakınlarındaki yoğun bakım yataklarının üçte ikisi boş dururken
  nasıl olup da Almanya'ya taşınmıştır?''
\item
  Avrupa Klinik Araştırmalar Dergisi'nde (European Journal of Clinical
  Investigation) yayınlanan bir makalede, Stanford tıp profesörü John C.
  Ioannidis ``abartılan bilgilerin ve kanıta-dayalı-olmayan önlemlerin
  verdiği zararları''
  \href{https://onlinelibrary.wiley.com/doi/abs/10.1111/eci.13222}{eleştirmektedir}.
  Başlangıçta dergiler bile kuşkulu iddialar yayınlamıştır.
\item
  Mart ayı başında Çin Epidemiyoloji Dergisi'nde yayınlanan ve Kovid-19
  virüs testlerinin güvenilir olmadığını (belirti göstermeyen
  hastalardaki yaklaşık \%50 oranında hatalı-pozitif sonuçları)
  gösteren, Çin'de yapılmış bir araştırma sonradan yayından geri
  çekilmiştir. Bir NPR muhabirine göre, bu çalışmanın, bir tıp
  fakültesinin dekanı olan başyazarı, yayından çekilme olayının nedenini
  açıklamak istememiş, siyasi baskıya işaret edebilecek bir deyişle
  ``\href{https://www.npr.org/sections/health-shots/2020/03/26/822084429/in-defense-of-coronavirus-testing-strategy-administration-cited-retracted-study}{hassas
  bir konu}'' demiştir. Halbuki, bu çalışmadan bağımsız olarak, PCR diye
  bilinen virüs testlerinin güvenilir olmadığı zaten uzun zamandır
  bilinmekteydi: Örneğin 2006'da, Kanada'daki bir bakımevinde kitlesel
  bir enfeksiyona yol açan SARS korona virüslerine ``rastlanmış'', daha
  sonra ise (risk grupları için yine ölümcül olabilecek) adi soğuk
  algınlığı korona virüsleri olduğu
  \href{https://www.ncbi.nlm.nih.gov/pmc/articles/PMC2095096/}{ortaya
  çıkmıştı}.
\item
  Alman Risk Yönetimi Ağı RiskNET yazarları,
  \href{https://www.risknet.de/themen/risknews/covid-19-und-der-blindflug/}{bir
  Kovid-19 analizinde} bir ''kör uçuş'' `tan ve ``veri kompetansında ve
  veri etiğinde yetersizlik'' `ten söz ediyorlar. Gittikçe daha fazla
  sayıda test ve önlemler yerine temsil edici özellikte bir örnek
  gereklidir. Alınan önlemlerin ``anlamı ve oranı'' eleştirel biçimde
  sorgulanmak zorundadır.
\item
  Uluslararası üne sahip Arjantinli-Fransız virolog Pablo Goldschmidt
  ile yapılan İspanyolca bir söyleşi
  \href{https://www.rubikon.news/artikel/der-corona-totalitarismus}{Almancaya
  çevrilmiştir}. Goldschmidt dayatılan önlemlerin tıbbi açıdan ters etki
  yaptığı görüşünde olup, ``totaliterliğin kökenleri'' `nin kavranması
  için şu anda ``Hannah Arendt okunmalı'' diyor.
\item
  Macar Başbakanı Viktor Orban, kendisinden önce gelmiş diğer
  başbakanlar ve cumhurbaşkanları gibi, Macar Parlamentosu'nu şu anda
  bir ``acil durum yasası'' ile
  \href{https://www.krone.at/2127086}{büyük ölçüde güçsüzleştirmiş}
  olup, ülkeyi kararnamelerle yönetiyor.
\end{itemize}

\hypertarget{30-mart-2020-i}{%
\paragraph{30 Mart 2020 (I)}\label{30-mart-2020-i}}

\begin{itemize}
\tightlist
\item
  Almanya'da bazı klinikler artık hasta kabul edemiyor -- çok fazla
  hasta ve çok az yatak olduğundan değil, ama çoğu vakada neredeyse
  hiçbir belirti göstermedikleri
  halde~\href{https://www.sueddeutsche.de/panorama/coronavirus-news-deutschland-wolfsburg-laschet-1.4828033}{sağlık
  çalışanlarının testleri pozitif
  çıktı}\href{https://www.sueddeutsche.de/panorama/coronavirus-news-deutschland-wolfsburg-laschet-1.4828033}{ğı
  için}. Bu durum sağlık sistemlerinin nasıl ve neden felç olduğunu bir
  kez daha göstermiştir.
\item
  Almanya'da ileri demans hastalarının bulunduğu yaşlılar evi ve
  bakımevlerinde testleri pozitif çıkan 15
  kişi~\href{https://www.sueddeutsche.de/panorama/coronavirus-news-deutschland-wolfsburg-laschet-1.4828033}{ölmüştür}:
  ``Şaşırtıcı olan, çoğunun~korona belirtileri göstermeksizin~ölmüş
  olmalarıdır.'' Bir alman uzman hekimin bize verdiği bilgi şöyledir:
  ``Benim tıbbi bakış açımdan, bu insanların bazılarının alınan önlemler
  sonucunda ölmüş olabileceğine yönelik bazı kanıtlar vardır. Demans
  hastaları gündelik hayatlarında izolasyon, fiziksel bağlantı
  eksikliği, olasılıkla maskeli hademeler gibi, belirleyici
  değişiklikler yapıldığında büyük strese girmektedir.''
\item
  \href{https://twitter.com/sneatio/status/1244157986832101376}{İsviçreli
  bir farmakoloğa göre}, Bern Üniversite Hastanesi (Inselspital)
  Kovid-19 korkusu yüzünden çalışanlarını zorunlu izne çıkartmış,
  terapileri durdurmuş ve ameliyatları ertelemiştir.
\item
  Alman Helmholtz Enfeksiyon Araştırma Merkezi`nin Epidemiyoloji
  Departmanı başkanı Profesör Gérard Krause, Alman devlet televizyonu
  ZDF'te korona-karşıtı önlemlerin
  ``\href{https://www.zdf.de/nachrichten/politik/coronavirus-epidemiologe-folgen-helmholtz-100.html}{virüsten
  daha fazla ölüme yol açabileceği}'' uyarısında bulunmaktadır.
\item
  Çeşitli medya kuruluşları, şimdiden İtalya'da 40'ı aşkın doktorun
  ``korona krizi sırasında'' öldüğünü bildirmiştir.
  Halbuki,~\href{https://portale.fnomceo.it/elenco-dei-medici-caduti-nel-corso-dellepidemia-di-covid-19/?ref=drnweb.repubblica.scroll-1}{ilgili
  listeye}~göz atıldığında, ölen doktorların çoğunun, 90 yaşında
  psikiyatrlar ve çocuk doktorları dahil, her tür uzmanlık alanından
  çoktan emekli olmuş, muhtemelen çoğu da doğal nedenlerle ölmüş
  kişilerden oluştuğu görülmektedir.
\item
  \href{https://www.government.is/news/article/?newsid=c65cf658-6eb6-11ea-9462-005056bc4d74}{İzlanda'da
  yapılmış kapsamlı bir araştırma}~testleri pozitif çıkmış kişilerin
  \%50'sinin ``hiçbir belirti'' göstermediğini doğrulamaktadır. Yine
  İzlanda verilerine göre, Kovid-19 ölüm oranları bin kişide görülen
  grip vakaları aralığında veya bunun altındadır. Testleri pozitif çıkıp
  ölen iki kişinin biri ``bilinen belirtileri gösteren bir turisttir''.
  (\href{https://www.covid.is/data}{İzlanda'ya ait daha fazla veri})
\item
  British Daily Mail muhabiri Peter
  Hitchens~\href{https://hitchensblog.mailonsunday.co.uk/2020/03/theres-powerful-evidence-this-great-panic-is-foolish-yet-our-freedom-is-still-broken-and-our-economy.html}{yazısında},
  ``Bu büyük Paniğin Aptalca Olduğuna Dair Güçlü Kanıtlar Var. Buna
  karşın özgürlüğümüz zedelenmiş, ekonomimiz sakatlanmıştır.'' diyor.
  Hitchens, Birleşik Krallık'ın bazı yerlerinde polis dronlarının doğada
  yapılan ``gereksiz
  yürüyüşler''i~\href{https://www.youtube.com/watch?v=fHNxDzLsPeg}{izlediği
  ve rapor ettiği}'ne işaret ediyor. Bazı durumlarda, polis dronları
  evlerine gitmeleri ve ``hayat kurtarmaları''
  yönünde~\href{https://www.youtube.com/watch?v=D4GEZjUTkqc}{halka
  hoparlörlerden çağrıda bulunmaktadır}. (Not: George Orwell bile bu
  kadar ilerisini düşünmemişti.)
\item
  İtalyan gizli servisi sosyal huzursuzluk ve ayaklanmalara
  karşı~\href{https://www.focus.de/panorama/welt/sorge-vor-sozialen-unruhen-supermaerkte-gepluendert-apotheken-ueberfallen-italiens-geheimdienst-warnt-vor-aufstaenden_id_11826664.html}{uyarıda
  bulunmaktadır}. Süpermarketler zaten yağmalanmış, eczaneler
  basılmıştır.
\item
  Profesör Sucharit Bhakdi bu sırada, Başbakan Dr. Angela Merkel'e
  yazdığı
  \href{https://swprs.org/open-letter-from-professor-sucharit-bhakdi-to-german-chancellor-dr-angela-merkel/}{Açık
  Mektup}~ile ilgili
  (Almanca/İngilizce)~\href{https://www.youtube.com/watch?v=LsExPrHCHbw\&feature=emb_title}{bir
  video yayınlamıştır}.
  (\href{https://www.coronagercegi.com/post/prof-dr-sucharit-bhakdi-den-merkel-e-a\%C3\%A7\%C4\%B1k-mektup}{Mektubun
  Türkçesi})
\end{itemize}

\hypertarget{30-mart-2020-ii}{%
\paragraph{30 Mart 2020 (II)}\label{30-mart-2020-ii}}

Bazı ülkelerde Kovid-19 konusunda, ``tedavinin hastalıktan daha kötü''
olabileceğini gösteren artan sayıda kanıt ortaya çıkmıştır.

Öte yandan, hafif bir hastalığı bulunanların hastanede kaptıkları
mikroplarla
gelişen~\href{https://en.wikipedia.org/wiki/Hospital-acquired_infection}{nozokomiyal
enfeksiyonlar}~riski vardır. Avrupa'da yılda ortalama 2,5 milyon
nozokomiyal enfeksiyon ve 50.000 ölüm olduğu hesaplanmaktadır.
Almanya'daki yoğun bakım ünitelerinde bile hastaların \%15 kadarı, yapay
solunum cihazındayken gelişen zatürre dahil, bir nozokomiyal enfeksiyon
yaşamaktadır. Dikkate değer başka bir sorun ise hastanelerdeki
antibiyotiklere-dayanıklı mikropların sayısında görülen artıştır.

Bir diğer konu da Kovid-19 hastalarında kullanılan kesinlikle iyi
niyetli ama bazen çok agresif tedavi yöntemleridir. Bunlar özellikle
steroidlerin, antibiyotiklerin ve anti-viral ilaçların (veya bunların
bir bileşiminin) hastaya verilmesini içermektedir. SARS-1 hastalarının
tedavisinde ortaya konulduğu gibi, böyle tedavilerin kullanılmalarının
sonuçları, hiç kullanılmamasına
göre~\href{https://www.sciencedaily.com/releases/2020/02/200206110703.htm}{sıklıkla
daha kötü ve ölümcül} olmaktadır.

\hypertarget{29-mart-2020}{%
\paragraph{29 Mart 2020}\label{29-mart-2020}}

\begin{itemize}
\tightlist
\item
  Almanya'nın Mainz kentinde yaşayan, Medikal Mikrobiyoloji uzmanı
  Emeritus Profesör Dr. Sucharit Bhakdi, Almanya Başbakanı Dr. Angela
  Merkel'e bir
  \href{https://swprs.org/offener-brief-von-professor-sucharit-bhakdi-an-bundeskanzlerin-dr-angela-merkel/}{Açık
  Mektup} yazdı ve Kovid-19'a verilen yanıtın acilen yeniden
  değerlendirilmesi çağrısında bulunarak beş yaşamsal soru sordu.
  (\href{https://swprs.org/open-letter-from-professor-sucharit-bhakdi-to-german-chancellor-dr-angela-merkel/}{İngilizce
  çevirisi})
\item
  \href{https://multipolar-magazin.de/artikel/coronavirus-irrefuhrung-fallzahlen}{Alman
  Robert Koch Enstitüsü'nün en son verileri}, testleri pozitif çıkan
  kişilerin sayısındaki artışın testlerin sayısındaki artışla doğru
  orantılı olduğunu, yani yüzde olarak kabaca aynı kaldığını
  göstermektedir. Bu, vaka sayısındaki artışın devam eden bir epidemiye
  değil, esasen test sayısındaki artışa bağlı olduğuna işaret ediyor
  olabilir.
\item
  Milanolu mikrobiyolog Maria Rita Gismondo, rakamlar ``sahte'' olup
  halkı gereksiz paniğe sürüklediği için, İtalyan hükümetini ``korona
  pozitifler''in gündelik sayısını
  \href{https://www.secoloditalia.it/2020/03/coronavirus-la-gismondo-ammonisce-duramente-basta-snocciolare-numeri-sui-positivi-sono-dati-falsati/}{halka
  açıklamaktan vazgeçmeye çağırıyor}. ``Korona pozitifler''in sayısı,
  test türlerine ve sayılarına çok yakından bağlı olup sağlık durumuna
  ilişkin hiçbir şey ifade etmemektedir.
\item
  Stanford Tıp ve Epidemiyoloji Profesörü Dr. John Ioannidis, alınan
  Kovid-19 önlemleri için veri eksikliği konusunda
  \href{https://www.youtube.com/watch?v=d6MZy-2fcBw}{bir saatlik
  ayrıntılı bir söyleşi} yapmıştır.
\item
  Fransa'da yaşayan Arjantinli virolog Pablo Goldschmidt, Kovid-19'a
  verilen politik tepkiyi ``tamamen abartılmış'' kabul etmekte ve alınan
  \href{https://www.infobae.com/coronavirus/2020/03/28/para-un-prestigioso-cientifico-argentino-el-coronavirus-no-merece-que-el-planeta-este-en-un-estado-de-parate-total/}{``totaliter
  önlemler''}e karşı uyarıda bulunmaktadır. Fransa'nın bazı kesimlerinde
  insanların hareketleri şimdiden dronlar tarafından izlenmektedir.
\item
  1934 doğumlu İtalyan yazar Fulvio Grimaldi, İtaya'da şu anda devlet
  tarafından uygulanan önlemlerin
  \href{https://www.youtube.com/watch?v=O3BuNp01vpc}{``faşizm iktidarı
  zamanındakinden daha beter''} olduğunu açıklamaktadır. Parlamento ve
  toplum tamamen güçsüz kılınmıştır.
\end{itemize}

\hypertarget{28-mart-2020}{%
\paragraph{28 Mart 2020}\label{28-mart-2020}}

\begin{itemize}
\tightlist
\item
  \href{https://news.yahoo.com/oxford-study-suggests-millions-people-221100162.html}{Oxford
  Üniversite'si tarafından yapılan yeni bir çalışmanın} sonucuna göre,
  Kovid-19 Birleşik Krallık'ta 2020 Ocak ayından bu yana mevcut olabilir
  ve nüfusun yarısı hiçbir belirti göstermeksizin ya da yalnızca hafif
  belirtiler yaşayarak buna bağışıklık geliştirmiş olabilir. Bunun
  anlamı, Kovid-19 nedeniyle 1000 kişide bir kişinin hastanede
  tedavisinin gerekeceğidir.
  (\href{https://www.medrxiv.org/content/10.1101/2020.03.24.20042291v1}{ilgili
  çalışma}) 
\item
  İngiliz medyası 21 yaşındaki bir kadının ``önceden herhangi bir
  hastalığı olmadığını ve Kovid-19'dan öldüğünü''
  \href{https://www.bbc.com/news/uk-england-beds-bucks-herts-52041709}{bildirmişti}.
  Halbuki, daha sonra bu kadının Kovid-19 pozitif bile olmadığı ve başka
  bir nedenle ölmüş olduğu
  \href{https://archive.is/20200329015127/https:/www.theguardian.com/world/2020/mar/27/chloe-middleton-death-21-year-old-not-recorded-nhs-covid-19-related}{ortaya
  çıkmıştır}. Kovid-19 dedikodusu ``hafif bir öksürüğü olduğu için''di.
\item
  Alman medyasında görüş bildiren biliminsanı Profesör Otfried Jarren,
  medya kuruluşlarının çoğunun,
  \href{https://www.deutschlandfunk.de/covid-19-scharfe-kritik-an-ard-und-zdf-wegen.2849.de.html?drn:news_id=1114517}{tehditleri
  ve siyasi gücü değerlendirmeksizin olduğu gibi aktaran bir
  gazetecilik} yaptığı eleştirisinde bulundu. Profesör Jarren'a göre,
  uzmanlar arasında hiçbir ayıredici özellik ve gerçek bir tartışma
  bulunmuyor.
\end{itemize}

\hypertarget{27-mart-2020-i}{%
\paragraph{27 Mart 2020 (I)}\label{27-mart-2020-i}}

\textbf{İtalya}: İtalyan Sağlık Bakanlığı tarafından
yayınlanan~\href{http://www.salute.gov.it/portale/caldo/SISMG_sintesi_ULTIMO.pdf}{en
son verilere göre}, 65 yaş üzeri tüm yaş grupları için ortalama ölüm
oranları, yumuşak geçen kış nedeniyle ortalamanın altında seyrettikten
sonra, şimdi önemli ölçüde yükselmiştir.\\
14 Mart'a kadar, ortalama ölüm oranları hala 2016/2017 grip
mevsimindekinin altındaydı, ama bu arada aşmış olabilir. Ölüm oranındaki
bu fazlalığın büyük bölümü kuzey İtalya kaynaklıdır. Yine de panik,
sağlık hizmetindeki çöküş ve tecritin kendisi gibi başka etkenlerle
karşılaştırıldığında, Kovid-19'un kesin rolü henüz açık değildir.\\

\includegraphics{https://swprs.files.wordpress.com/2020/03/italia-mortalita-marzo-14.png?w=600\&h=343}

~

\textbf{Fransa}:~\href{https://www.santepubliquefrance.fr/maladies-et-traumatismes/maladies-et-infections-respiratoires/infection-a-coronavirus/documents/bulletin-national/covid-19-point-epidemiologique-du-24-mars-2020}{Fransa'dan
gelen en son verilere göre}, ortalama ölüm oranları yumuşak geçen bir
grip mevsiminden sonra, ulusal düzeyde normal aralık içinde kalmaya
devam ediyor. Buna karşın, bazı bölgelerde, özellikle de Fransa'nın
kuzeydoğusunda, 65 yaş üstü gruptaki ortalama ölüm oranları Kovid-19'a
bağlı olarak keskin bir artış göstermiştir (bkz. alttaki şekil).~

\includegraphics{https://swprs.files.wordpress.com/2020/03/france-mortality.png?w=650\&h=400}

~

Fransa aynı zamanda yaş dağılımı ve testleri pozitif çıkan yoğun bakım
hastalarının ve ölen hastaların yaş dağılımı ve daha önceden mevcut
hastalıkları
konusunda~\href{https://www.santepubliquefrance.fr/maladies-et-traumatismes/maladies-et-infections-respiratoires/infection-a-coronavirus/documents/bulletin-national/covid-19-point-epidemiologique-du-24-mars-2020}{ayrıntılı
bilgi}~vermektedir (bkz. alttaki şekil):

\begin{itemize}
\tightlist
\item
  Ölenlerin~yaş ortalaması 81,2'dir.
\item
  Ölenlerin \%78'i 75 yaşının üzerinde; \%93'ü 65 yaşının üzerindedir.
\item
  Ölenlerin \%2,4'ü 65 yaşının altında olup daha önceden (bilinen)
  hastalıkları yoktur.
\item
  Yoğun bakım hastalarının yaş ortalaması~65'tir.
\item
  Yoğun bakım hastalarının \%26'sı 75 yaşının üzerindedir; \%67'sinin
  önceden hastalıkları yoktur.
\item
  Yoğun bakım hastalarının \%17'si 65 yaşının altında olup daha önceden
  sahip oldukları hastalıkları yoktur.
\end{itemize}

 Fransız yetkililer, ``salgının (Kovid-19) toplam ölüm oranlarındaki
payının henüz belirlenmediğini'' eklemişlerdir.

\includegraphics{https://swprs.files.wordpress.com/2020/03/france-age-distribution-march-24.png?w=736}

\includegraphics{https://swprs.files.wordpress.com/2020/03/us-pneumonia-deaths.png?w=400\&h=360}

\hypertarget{buxfcyuxfck-britanya}{%
\subparagraph{\texorpdfstring{\textbf{Büyük
Britanya:}}{Büyük Britanya:}}\label{buxfcyuxfck-britanya}}

\begin{itemize}
\tightlist
\item
  Londra'daki Imperial College'dan Neil Ferguson, İngiltere'nin Kovid-19
  hastalarını tedavi etmek için yoğun bakım ünitelerinde yeterli
  kapasiteye sahip
  olduğunu~\href{https://www.newscientist.com/article/2238578-uk-has-enough-intensive-care-units-for-coronavirus-expert-predicts/}{şu
  anda varsaymaktadır}.
\item
  Patoloji alanında çalışan Emeritus Profesör John Lee, Kovid-19
  vakalarına uygulanan özel kayıt şeklinin, normal grip ve soğuk
  algınlığı vakalarıyla karşılaştırıldığında, Kovid-19'un yol açtığı
  riskin olduğundan daha yüksek görülmesine yol
  açtığını~\href{https://www.spectator.co.uk/article/The-evidence-on-Covid-19-is-not-as-clear-as-we-think}{iddia
  etmektedir}.
\end{itemize}

\hypertarget{diux11fer-konular}{%
\subparagraph{\texorpdfstring{\textbf{Diğer
Konular:}}{Diğer Konular:}}\label{diux11fer-konular}}

\begin{itemize}
\tightlist
\item
  Stanford Üniversitesi'ndeki araştırmacıları yaptığı
  bir~\href{https://medium.com/@nigam/higher-co-infection-rates-in-covid19-b24965088333}{ön
  çalışma}~Kovid-19 pozitif hastaların \%20 ila 25'inin, diğer grip veya
  soğuk algınlığı virüsleri için de pozitif çıktıklarını göstermiştir.
\item
  ABD'de işsizlik sigortasına başvuru sayısı hızla yükselmiş ve bir
  rekor
  kırarak~\href{https://www.businessinsider.com/us-weekly-jobless-claims-record-coronavirus-unemployment-insurance-labor-recession-2020-3}{üç
  milyonu aşmıştır}. Bu
  bağlamda,~\href{https://twitter.com/KoenSwinkels/status/1243066532390977544}{intiharlarda
  da bir artış}~beklenmektedir.
\item
  Almanya'da testi pozitif çıkan ilk hasta şimdi iyileşmiş durumdadır.
  33 yaşındaki bu kişi hastalığı, kendi
  ifadesiyle,~\href{https://www.br.de/nachrichten/bayern/coronavirus-patient-nummer-1-wie-ich-die-quarantaene-erlebte,Rrm4Ul8}{``grip
  kadar ağır''}~yaşamamıştır.
\item
  İspanyol medyası, Kovid-19 için hızlı antikor testlerinin en az \%80
  olması gerekirken yalnızca \%30 hassasiyeti
  olduğunu~\href{https://elpais.com/sociedad/2020-03-25/los-test-rapidos-de-coronavirus-comprados-en-china-no-funcionan.html}{bildirmiştir}.
\item
  \href{https://ehjournal.biomedcentral.com/articles/10.1186/1476-069X-2-15}{2003'te
  Çin'de yapılmış bir araştırma}, SARS'tan ölüm olasılığının, orta
  düzeyde kirli havaya maruz kalanlarda, havası temiz bölgelerdeki
  hastalara oranla \%84 daha yüksek olduğu sonucuna varmıştır. Ağır hava
  kirliliği olan bölgelerde yaşayan insanlar arasında ise \%200 daha
  yüksektir.
\item
  Alman Kanıta-Dayalı Tıp Ağı (EbM) Kovid-19
  konusunda~\href{https://www.ebm-netzwerk.de/en/publications/covid-19}{medyanın
  verdiği haberleri eleştiriyor}: ``Medya haberleri talep ettiğimiz
  kanıta-dayalı risk iletişimi ölçütlerini hiçbir biçimde hesaba
  katmıyor. (\ldots{}) Diğer ölüm nedenlerine değinmeden ham verilerin
  sunulması riskin olduğundan fazla gösterilmesine yol açıyor''.
\end{itemize}

\hypertarget{27-mart-2020-ii}{%
\paragraph{27 Mart 2020 (II)}\label{27-mart-2020-ii}}

\begin{itemize}
\tightlist
\item
  Alman araştırmacı Dr. Richard
  Capek~\href{https://coronadaten.wordpress.com/}{yaptığı bir sayısal
  analizde şunu ileri sürüyor}: ``Korona salgını'' aslında bir ``test
  salgını''dır. Capek, test sayısı eksponansiyel olarak artarken,
  enfeksiyonların oranının sabit kaldığını ve ölüm oranının düşmüş
  olduğunu gösteriyor ki bu da virüsün kendisinin eksponansiyel olarak
  yayıldığının aksini anlatır.
\item
  Würzburg Üniversitesi'nden alman Viroloji profesörü Dr. Carsten
  Scheller,~\href{https://www.youtube.com/watch?v=w-uub0urNfw}{bir
  podcastta şunları açıklıyor:}~Kovid-19 kesinlikle griple
  kıyaslanabilecek bir hastalık olup şu ana kadar ondan daha az ölüme
  yol açmıştır. Profesör Scheller, medyada sık sık gösterilen
  eksponansiyel eğrilerin, virüsün kendisinin sıradışı bir yayılımından
  çok,~gittikçe artan test sayısı~ile ilgili olduğundan
  kuşkulanıyor.~Almanya gibi ülkeler için İtalya, Japonya ve Güney Kore
  gibi örnek alınacak bir ülke değildir. Bu ülkeler, milyonlarca Çinli
  turist ve yalnızca en alt seviyede toplumsal kısıtlama ile bir
  Kovid-19 krizi yaşamadılar. Bunun bir nedeni ağız maskeleri takılması
  olabilir: Bu ise enfeksiyona karşı korunmakta pek etkili olmasa da
  enfekte olmuş kişilerin virüs yaymasını sınırlayabiliyor.
\item
  \href{https://www.ecodibergamo.it/stories/bergamo-citta/a-bergamo-decessi-4-volte-oltre-la-medialeco-lancia-unindagine-nei-comuni_1346651_11/}{Bergamo'dan
  gelen en son rakamlar}~ortalama ölüm oranlarının orada 2020 yılı Mart
  ayında, normalde 200 ile 300'den yaklaşık 900 kişiye çıkarak,
  neredeyse 4 kat arttığını gösteriyor. Bunun ne oranda Kovid-19, ne
  oranda kitlesel panik, sistemik çöküş ve tecritin kendisi gibi başka
  nedenlerden kaynaklandığı henüz belli değildir. Görünen o ki kent
  hastanesi bütün bölgeden gelenlerle dolup taşmış ve sistemi çökmüştür.
\item
  Stanford Üniversitesi'nden iki tıp profesörü, Dr. Eran Bendavid ve Dr.
  Jay
  Bhattacharya,~\href{https://web.archive.org/web/20200325103650/https:/www.wsj.com/articles/is-the-coronavirus-as-deadly-as-they-say-11585088464}{makalelerinde}~Kovid-19'un
  ölümcüllüğünün~kat kat~fazla~gösterildiğini ve olasılıkla İtalya'da
  bile yalnızca~\%0.01 ile \%0.06 olup gripten daha düşük olduğunu
  anlatıyor. Bu abartının nedeni çoktan enfekte olmuş (belirti
  göstermeyen) insanların sayısının önemli ölçüde azımsanmasıdır.
  İtalya'da nüfusunun tamamına test yapılmış olan Vo
  belediyesinde,~\href{https://www.repubblica.it/salute/medicina-e-ricerca/2020/03/16/news/coronavirus_studio_il_50-75_dei_casi_a_vo_sono_asintomatici_e_molto_contagiosi-251474302/}{\%50
  ile 75'inin testleri pozitif çıkan ve belirti göstermeyen}~kişiler
  buna bir örnek olarak veriliyor.
\item
  Dr. Gerald Gass, Alman Hastaneler Birliği
  Başkanı,~\href{https://www.handelsblatt.com/politik/deutschland/coronakrise-deutsche-krankenhausgesellschaft-wir-sind-besser-vorbereitet-als-italien/25651268.html}{Handelsblatt
  gazetesi ile yaptığı bir söyleşide}~``İtalya'daki aşırı durumun çok
  düşük yoğun bakım kapasitelerinden kaynaklandığını'' anlatmıştır.
\item
  ``Kovid-19
  paniği''ni~\href{https://www.youtube.com/watch?v=p_AyuhbnPOI}{en başta
  ve yüksek perdeden eleştirenlerden}~biri olan Dr. Wolfgang Wodarg,
  sağlık grubuna başkanlık ettiği~Transparency International
  Germany~adlı
  kuruştan~\href{https://www.transparency.de/aktuelles/detail/article/in-eigener-sache-vorstand-beschliesst-ruhen-der-mitgliedschaft-von-wolfgang-wodarg-1/}{koşullu
  olarak ihraç edilmiştir}.~Wodarg, eleştirileri nedeniyle zaten medya
  tarafından da ciddi biçimde saldırıya uğramıştı.
\item
  Amerikan Güvenlik Kurumu'nu (NSA) medyaya ifşa eden Edward Snowden,
  hükümetlerin mevcut durumu sürveyans devletinin alanını genişletmek ve
  temel hakları kısıtlamak için
  kullandığı~\href{https://www.cnet.com/news/snowden-warns-government-surveillance-amid-covid-19-could-be-long-lasting/}{uyarısında
  bulunuyor}. Şu anda yürürlüğe konulan kontrol önlemleri krizden sonra
  kaldırılmayacaktır.
\end{itemize}

\href{https://swprs.org/a-swiss-doctor-on-covid-19/anzahl-infizierte-und-tests-2603/}{}

\includegraphics{https://swprs.files.wordpress.com/2020/03/anzahl-infizierte-und-tests-2603.jpg?w=356\&h=202}

Number of tests and test-positives (proportional)

\href{https://swprs.org/covid-19-hinweis-ii/infizierte-pro-test2603/}{}

\includegraphics{https://swprs.files.wordpress.com/2020/03/infizierte-pro-test2603.jpg?w=372\&h=202}

Test-positives per number of tests (constant)

Enfeksiyon sayısı, yapılan testlerin artan sayısıyla aynı oranda olup,
devam eden bir virüssalgını olmadığını gösteriyor (Dr. Richard Capek,
ABD verilerine göre)

\hypertarget{26-mart-2020-i}{%
\paragraph{26 Mart 2020 (I)}\label{26-mart-2020-i}}

\begin{itemize}
\tightlist
\item
  ABD: \href{https://healthweather.us/}{25 Mart'taki ABD verileri}, ülke
  çapında azalan sayıda grip benzeri hastalık olduğunu göstermektedir.
  Hükümet önlemleri, bir haftadan az bir süredir yürürlükte oldukları
  için, bu duruma bağlı olarak ortadan kaldırılabilir.
\item
  Almanya: 24 Mart'taki Robert Robert Koch Enstitüsü'nün
  \href{https://influenza.rki.de/Wochenberichte/2019_2020/2020-12.pdf}{en
  son grip raporu}, ``akut solunum yolu hastalıklarının aktivitesinde
  ülke çapında bir azalma'' olduğunu belgeliyor: Grip benzeri
  hastalıkların sayısı ve bunların neden olduğu hastanede kalış sayısı
  bir önceki seviyenin altında yıl ve halen düşüş devam ediyor. RKI
  şöyle devam ediyor: ``Şu anda doktora yapılan ziyaret sayısındaki
  artış nüfus içinde dolaşan grip virüsleri ya da SARS-CoV-2 ile
  açıklanamıyor.''
\item
  İtalya: Ünlü İtalyan virolog Giulio Tarro, Covid19'un ölüm oranının
  İtalya'da bile \%1'in altında olduğunu ve bu nedenle griple
  karşılaştırılabilir olduğunu
  \href{https://www.cybermednews.eu/index.php/it/health/70871-interview-to-the-virologist-giulio-tarro-the-death-rate-of-covid-19-is-less-than-1-as-confirmed-by-the-national-institute-of-allergy-and-infectious-diseases}{savunuyor}.
  Daha yüksek değerlerin yalnızca Covid19 ile ve Covid19 sebebiyle
  meydana gelen ölümler arasında bir ayrım yapılmadığından ve
  (semptomsuz) enfekte olmuş kişilerin sayısının büyük ölçüde hafife
  alındığı için ortaya çıktığı belirtilmiştir. 
\item
  İngiltere: 500.000'e kadar ölümü öngören British Imperial College
  çalışmasının yazarları tahminlerini tekrar düşürüyor. Test pozitif
  ölümlerin büyük bir kısmının normal ölüm oranının bir parçası olduğunu
  zaten \href{https://www.bbc.com/news/health-51979654}{kabul ettikten}
  sonra, şimdi hastalığın zirvesine zaten
  \href{https://www.thetimes.co.uk/article/nhs-now-likely-to-cope-with-coronavirus-says-key-scientist-rn5m6nggk}{iki
  ila üç hafta içinde ulaşılabileceğini} belirtiyorlar. 
\item
  İngiltere: İngiliz Guardian,
  \href{https://www.theguardian.com/society/2019/feb/20/britons-urged-to-get-flu-vaccine-as-critical-cases-rise-above-2000}{Şubat
  2019'da}, 2018/2019'daki grip sezonunda bile, İngiltere'de yoğun bakım
  ünitelerine 2180'den fazla griple ilgili kabulün olduğunu
  bildirmiştir.
\item
  İsviçre: İsviçre'de Covid19'a bağlı aşırı ölüm sayısı hala sıfırdır.
  Medya tarafından
  \href{https://www.nau.ch/ort/basel/drei-weitere-covid-19-todesfalle-in-basel-stadt-65684099}{öne
  sürülen} en son ``ölüm vakası'' 100 yaşında bir kadın. Bununla
  birlikte, İsviçre hükümeti kısıtlayıcı tedbirleri sıkılaştırmaya devam
  etmektedir.\\
\end{itemize}

\hypertarget{26-mart-2020-ii}{%
\paragraph{26 Mart 2020 (II)}\label{26-mart-2020-ii}}

\begin{itemize}
\tightlist
\item
  İsveç: İsveç, şu ana kadar
  \href{https://www.zeit.de/politik/ausland/2020-03/coronavirus-schweden-stockholm-oeffentliches-leben/komplettansicht}{iki
  ilkeye} dayanan Covid19 ile başa çıkmada en liberal stratejiyi
  izlemiştir: Risk grupları korunur ve grip belirtileri olan insanlar
  evde kalır. Baş epidemiyolog Anders Tegnell, ``Bu iki kurala
  uyarsanız, etkisi yine de marjinal olan başka önlemlere gerek yoktur''
  dedi. Sosyal ve ekonomik yaşam normal şekilde devam edecektir.
  Tegnell, hastanelere yapılan büyük telaşın bugüne kadar
  gerçekleşmediğini söyledi.
\item
  Alman ceza ve anayasa hukuku uzmanı Dr. Jessica Hamed, genel sokağa
  çıkma yasağı ve temas yasakları gibi önlemlerin, temel özgürlük
  haklarına büyük ve orantısız bir haksızlık olduğunu ve bu nedenle
  muhtemelen ``tüm yasa dışı'' olduğunu
  \href{https://www.fr.de/politik/coronakrise-deutschland-sind-kontaktsperren-ausgangsbeschraenkungen-rechtswidrig-13611821.html}{savunuyor}.
\item
  Genel ölüm oranıyla ilgili
  \href{https://www.euromomo.eu/index.html}{en son Avrupa izleme
  raporu}, tüm ülkelerde ve tüm yaş gruplarında normal veya ortalamanın
  altında değerleri göstermeye devam etmektedir. Bununla beraber
  \href{https://www.euromomo.eu/outputs/zscore_country65.html}{bir
  istisna} bulunmaktadır: İtalya'daki 65+ yaş grubunda şu anda artan bir
  toplam ölüm öngörülmektedir fakat yine de bu durum hala 2017 ve 2018
  influenza değerlerinin altında gözükmektedir.
\end{itemize}

\hypertarget{25-mart-2020}{%
\paragraph{25 Mart 2020}\label{25-mart-2020}}

\begin{itemize}
\tightlist
\item
  Alman immünolog ve toksikolog Profesör Stefan Hockertz,
  \href{https://www.youtube.com/watch?v=7wfb-B0BWmo}{radyo röportajında}
  Covid19'un gripten (grip) daha tehlikeli olmadığını, ancak daha
  yakından gözlemlendiğini açıklıyor. Virüsten daha tehlikeli olan,
  medyanın yarattığı korku ve panik ve birçok hükümetin ``otoriter
  tepkisidir''. Profesör Hockertz ayrıca ``korona ölümleri''
  denilenlerin çoğunun aslında koronavirüsler için pozitif olanların
  diğer nedenlerden öldüğüne de işaret eder. Hockertz, rapor edilenden
  on kat daha fazla insanın zaten Covid19'a sahip olduğuna, ancak hiçbir
  şey fark veya çok az şey fark ettiğine inanıyor. 
\item
  Arjantinli virolog ve biyokimyacı Pablo Goldschmidt, Covid19'un kötü
  bir soğuk algınlığı veya
  \href{https://www.clarin.com/buena-vida/coronavirus-panico-injustificado-dice-virologo-argentino-francia_0_yVcmJ4RM.html}{gripten
  daha tehlikeli olmadığını} açıklıyor. Covid19 virüsünün daha önceki
  yıllarda yayılmış olması mümkündür, ancak keşfedilmemiştir çünkü kimse
  aramıyordu. Dr. Goldschmidt medya ve siyaset tarafından yaratılan bir
  ``küresel terör'' den söz ediyor. Her yıl, dünya çapında üç milyon
  yenidoğanın ve sadece ABD'de 50.000 yetişkinin zatürreden öldüğünü
  söylüyor.
\item
  Bonn Üniversitesi Hijyen Enstitüsü başkanı Profesör Martin Exner, bir
  röportajda şu ana kadar Almanya'da hasta sayısında neredeyse hiç artış
  olmamasına rağmen sağlık personelinin neden baskı altında olduğunu
  \href{https://www.youtube.com/watch?v=9mI9trSm3PY}{açıklıyor}. Bir
  yandan, Covid19 pozitif~ doktor ve hemşirelerin karantinaya alınması
  ve yerlerine başkalarını getirmek genellikle zordur. Diğer yandan,
  bakımın önemli bir bölümünü sağlayan komşu ülkelerden gelen
  hemşireler, şu anda kapalı sınırlar nedeniyle ülkeye girememektedir.
\item
  Eski Alman Kültür Bakanı ve Etik Profesörü Profesör Julian
  Nida-Ruemelin, Covid19'un sağlıklı genel nüfus için herhangi bir risk
  oluşturmadığını ve sokağa çıkma yasağı gibi aşırı önlemlerin yerinde
  olmadığını
  \href{https://www.zdf.de/nachrichten/zdf-morgenmagazin/julian-nida-ruemelin-zur-corona-krise-100.html}{belirtiyor}.
\item
  Cruise gemisi Diamond Princess'ten alınan verileri kullanarak Stanford
  Profesörü John Ioannidis, Covid19'un yaş düzeltmeli ölümcüllüğünün\%
  0.025 ila\% 0.625 arasında, yani güçlü bir soğuk algınlığı veya grip
  aralığında olduğunu
  \href{https://www.statnews.com/2020/03/17/a-fiasco-in-the-making-as-the-coronavirus-pandemic-takes-hold-we-are-making-decisions-without-reliable-data/}{gösterdi}.
  Dahası,
  \href{https://www.niid.go.jp/niid/en/2019-ncov-e/9407-covid-dp-fe-01.html}{bir
  Japon çalışması}, tüm Covid19~ pozitif yolcuların yüksek ortalama
  yaşlara rağmen\% 48'inin tamamen semptomsuz kaldığını gösterdi; 80-89
  yaşları arasında bile\% 48'i semptomsuz kalırken, 70 ila 79 yaşları
  arasında \%60 ı nın hiç semptom geliştirmemesi şaşırtıcıydı. Bu da,
  önceden var olan hastalıkların virüsün kendisinden daha önemli bir
  faktör olup olmadığı sorusunu gündeme getirmektedir. İtalya, covid19
  pozitif
  \href{https://www.bloomberg.com/news/articles/2020-03-18/99-of-those-who-died-from-virus-had-other-illness-italy-says}{ölümlerin
  \% 99'unun} önceden mevcut bir veya daha fazla duruma sahip olduğunu
  ve hatta bunlar arasında ölüm sertifikalarında
  \href{https://web.archive.org/web/20200324214448/https:/www.telegraph.co.uk/global-health/science-and-disease/have-many-coronavirus-patients-died-italy/}{sadece
  \%12'sinin Covid19'dan nedensel bir faktör} olarak bahseder.
\end{itemize}

\hypertarget{24-mart-2020}{%
\paragraph{24 Mart 2020}\label{24-mart-2020}}

\begin{itemize}
\tightlist
\item
  İngiltere Covid 19'u, ölüm-mortalite oranları genel itibariyle düşün
  olduğunu belirterek Yüksek Oranda Bulaşıcı Hastalıklar -HCID
  listesinden
  \href{https://www.radioeins.de/programm/sendungen/die_profis/archivierte_sendungen/beitraege/corona-virus-kein-killervirus.html}{çıkarmıştır}.
\item
  Alman Ulusal Sağlık Enstitüsü-RKI direktörü gerçek ölüm nedenine
  bakılmaksızın tüm pozitif testleri ``koronavirüs ölümleri `` olarak
  saydıklarını
  \href{https://swprs.org/rki-relativiert-corona-todesfaelle/}{kabul
  etmiştir}. Ölen kişilerin yaş ortalaması 82'dir ve önceden varolan
  ciddi medikal tıbbi probleme sahiptir. Diğer birçok ülkede olduğu gibi
  Covid 19'a bağlı fazla ölüm oranının Almanya'da sıfıra yakın olması
  muhtemeldir. 
\item
  İsviçre'de Covid 19 hastaları için yoğun bakım ünitelerinde ayrılmış
  yataklar
  \href{https://www.aargauerzeitung.ch/aargau/kanton-aargau/erst-3-von-100-aargauer-betten-der-intensivstationen-sind-belegt-so-ruesten-sich-die-spitaeler-auf-die-epidemie-137332716}{``çoğunlukla
  boştur ``}. 
\item
  Zürih Üniversitesi Tıbbi Viroloji eski başkanı Profesör Karin
  Moelling, bir röportajda Covid 19'un
  \href{https://www.radioeins.de/programm/sendungen/die_profis/archivierte_sendungen/beitraege/corona-virus-kein-killervirus.html}{``öldürücü
  virüs olmadığını `` ve ``paniğin sona ermesi gerektiğini``}
  söylemiştir.
\end{itemize}

\hypertarget{23-mart-2020-i}{%
\paragraph{23 Mart 2020 (I)}\label{23-mart-2020-i}}

\begin{itemize}
\tightlist
\item
  ``Journal of Antimicrobial Agents'' dergisinin
  \href{https://www.sciencedirect.com/science/article/abs/pii/S0924857920300972}{SARS-CoV-2:
  korkuya karşı veri}, başlıklı yazısında Fransa'da bir hastanede
  yapılan çalışmaya istinaden, SARS-CoV-2 probleminin abartıldığı;~
  SARS-CoV-2'deki ölüm oranının diğer yaygın koronavirsülerden anlamlı
  ölçüde farklı olmadığı belirlenmiştir.
\item
  2019 yılında
  \href{https://www.ijidonline.com/article/S1201-9712(19)30328-5/fulltext}{İtalya'da
  yapılan bir çalışma} sonucu son yıllarda gripten ölümün 7.000 ile
  25.000 kişi arasında olduğu belirlenmiştir. İtalya'daki yaşlı nüfusun
  fazla olması sebebiyle bu değer diğer Avrupa ülkeleri ile
  kıyaslandığında ve şimdiye kadar Covid-19 sebebiyle gerçekleşenden
  daha yüksek bir rakamdır.~ 
\item
  WHO (Dünya Sağlık Örgütü) 'nün
  \href{https://www.who.int/news-room/q-a-detail/q-a-similarities-and-differences-covid-19-and-influenza}{yeni
  yayınlanan belgesinde} Covid-19'un Influenza'ya kıyasla yaklaşık \%50
  oranında hızlı değil, yavaş yayıldığı bildirilmiştir. Ayrıca,
  Covid-19'da hastalığın semptomları belirmeden önce bulaşma oranı
  Influenza'dan (gripten) daha düşüktür.
\item
  İtalya'nın önde gelen doktorlarından biri Lombardei'de görülen
  \href{https://www.scmp.com/news/china/society/article/3076334/coronavirus-strange-pneumonia-seen-lombardy-november-leading}{``garip
  pnömoni-zatürre vakalarının''} Kasım 2019 yılında görülmeye
  başlandığını ve yeni virüsten mi (resmi olarak Şubat 2020'de görülen)
  yoksa Kuzey İtalya'daki tehlikeli
  \href{https://www.thelocal.it/20170131/our-lungs-are-breaking-smog-levels-way-above-safe-limits-in-northern-italy}{yüksek
  duman seviyesine} sahip fabrikalardan mı kaynaklandığı sorusunu
  gündeme getirmiştir.
\item
  Cochrane Medical Collaboration kurucusu, Danimarkalı araştırmacı Peter
  Gøtzsche, koronanın
  \href{https://www.deadlymedicines.dk/corona-an-epidemic-of-mass-panic/}{''kitle
  paniği salgını``} olduğunu ve ''İlk kurbanlardan birinin Mantık``
  olduğunu yazıyor.\\
\end{itemize}

\hypertarget{23-mart-2020-ii}{%
\paragraph{23 Mart 2020 (II)}\label{23-mart-2020-ii}}

\begin{itemize}
\tightlist
\item
  İsrail eski Sağlık Bakanı Profesör Yoram Lass, yeni koronavirüsün
  ``gripten daha az tehlikeli'' olduğunu ve tecrit tedbirlerinin
  \href{https://en.globes.co.il/en/article-lockdown-lunacy-1001322696}{``virüse
  kıyasla daha öldürücü''} olduğunu söylemektedir. Ve şöyle
  eklemektedir; ``sayılar panikle eşleşmemektedir'' ve ``psikoloji,
  bilime hakim olmaktadır''. ``İtalya solunum problemleri sebebiyle
  diğer Avrupa ülkelerine kıyasla üç kat fazla ölüm oranına sahiptir''
  diye belirtmektedir.
\item
  İsviçreli Enfeksiyon Hastalıkları Uzmanı Pietro Vernazza, uygulanan
  tedbirlerin birçoğunun
  \href{https://www.tagblatt.ch/leben/ostschweizer-infektiologe-pietro-vernazza-die-zahlen-zu-den-jungen-corona-virus-erkrankten-sind-irrefuehrend-ld.1206440}{bilime
  dayanmadığını} ve tersine çevrilmesi gerektiğini savunmaktadır.
  Vernazza'ya göre, çok fazla test yapmanın bir anlamı yoktur çünkü
  nüfusun \%90'ında hiçbir belirti göstermeyecektir. Okulların
  kapanması, hatta tecritler ``ters etki yapacaktır''. Sadece risk
  gruplarını korumayı, ekonomiyi ve toplumu rahatsız etmemeyi
  önermektedir.
\item
  Dünya Doktorları Federasyon Başkanı Frank Ulrich Montgomery, İtalya'da
  uygulanan tecrit önlemlerinin
  \href{https://www.general-anzeiger-bonn.de/news/politik/deutschland/interview-mit-weltaerztepraesident-montgomery-ueber-corona-pandemie-ist-chaos_aid-49609561}{``mantıksız''
  ve ``zarar verici''} olduğunu ve tersine çevrilmesi gerektiğini
  söylüyor.``
\item
  İsviçre'de, medyanın paniğine rağmen, ölüm oranı neredeyse sıfıra
  yakındır; en son test pozitif
  \href{https://www.bluewin.ch/de/newsregional/zuerich/1068-bestatigte-corona-falle-und-funf-todesfalle-im-kanton-zurich-371873.html}{kurbanları}
  96 yaşındaki palyatif bakım hastası ve 97 yaşındaki altta yatan sağlık
  sorunları olan bir hastadır.
\item
  İtalya Ulusal Sağlık Enstitüsü istatistik raporları
  \href{https://www.epicentro.iss.it/coronavirus/bollettino/Report-COVID-2019_20_marzo_eng.pdf}{İngilizce
  olarak yayınlanmıştır}.
\end{itemize}

\hypertarget{22-mart-2020-i}{%
\paragraph{22 Mart 2020 (I)}\label{22-mart-2020-i}}

\begin{itemize}
\tightlist
\item
  İtalya'daki durumla ilgili olarak: Medyanın büyük çoğunluğu,
  İtalya'nın koronavirüsten günde 800'e kadar ölüme sahip olduğunu
  yanlış bir şekilde bildirmektedir. Gerçekte, İtalyan Sivil Savunma
  Servisi başkanı, bunların ``koronavirüsten değil koronavirüs ile
  ölüm'' olduğunu vurgulamaktadır
  (\href{https://youtu.be/0M4kbPDHGR0?t=210}{basın toplantısının} 03:30
  dakikası). 
\item
  Profesörler Ioannidis ve Bhakdi'nin de
  \href{https://www.statnews.com/2020/03/17/a-fiasco-in-the-making-as-the-coronavirus-pandemic-takes-hold-we-are-making-decisions-without-reliable-data/}{gösterdiği}
  gibi, Güney Kore ve Japonya gibi karantina kararı almayan ülkeler
  Covid-19 ile bağlantılı olarak sıfıra yakın aşırı ölüm yaşadı, Diamond
  Princess gemisi ise mil başına aralıkta ekstrapole (tahmin olarak
  yürütülmüş) ölüm oranı yaşadı, yani bu da mevsimsel grip seviyesinde
  veya altında bir oran ortaya koymaktadır.
\item
  İtalya'daki şu anki test pozitif ölüm rakamları, İtalya'daki günlük
  toplam (\textasciitilde{}1.800) ölümlerin \% 50'sinden daha azdır.
  Dolayısıyla, normal günlük ölümlerin büyük bir kısmının artık sadece
  ``Covid19'' ölümleri (pozitif test ettikleri gibi) olarak sayılıyor
  olması muhtemeldir. İtalyan Sivil Savunma Hizmetleri Başkanı'nın
  vurguladığı nokta budur.
\item
  Bununla birlikte, şimdiye kadar, İtalya'nın kuzeyindeki bazı
  bölgelerin, yani
  \href{https://en.wikipedia.org/wiki/2020_Italy_coronavirus_lockdown}{en
  zorlu karantina önlemleriyle} karşı karşıya kalan bölgelerin, günlük
  ölüm rakamlarında belirgin bir artış yaşadıkları açıktır. Lombardiya
  bölgesinde, test pozitif ölümlerin \%90'ının yoğun bakım ünitelerinde
  değil,
  \href{https://www.tgcom24.mediaset.it/cronaca/coronavirus-in-lombardia-9-morti-su-10-mai-giunti-in-terapia-intensiva_16362350-202002a.shtml}{daha
  çok evde} meydana geldiği bilinmektedir. Ve \%99'dan fazlasının
  önceden mevcut ciddi sağlık sorunları vardır.
\item
  Profesör Sucharit Bhakdi karantina önlemlerini ``işe yaramaz'',
  ``kendini yıkıcı'' olarak
  \href{https://www.youtube.com/watch?v=JBB9bA-gXL4}{nitelendirmektedir}.
  Bu nedenle, son derece rahatsız edici soru, önceden mevcut birden
  fazla sağlık durumuna sahip bu yaşlı, izole, yüksek stresli insanların
  artan ölüm oranının aslında hala yürürlükte olan haftalarca süren
  karantina önlemlerinden mi kaynaklanabileceği sorusudur.
\item
  Eğer öyleyse, tedavinin hastalıktan daha kötü olduğu vakalardan biri
  olabilir. (III. bölümdeki güncellemeye bakın: ölüm sertifikalarının
  sadece \%12'si neden olarak koronavirüsü göstermektedir.)
\end{itemize}

\hypertarget{22-mart-2020-ii}{%
\paragraph{22 Mart 2020 (II)}\label{22-mart-2020-ii}}

\begin{itemize}
\tightlist
\item
  İsviçre'de şu anda, ileri yaşlı ve / veya önceden var olan sağlık
  durumları nedeniyle
  ``\href{https://www.nzz.ch/schweiz/coronavirus-in-der-schweiz-die-neusten-entwicklungen-ld.1542664\#subtitle-wie-viele-infizierte-und-todesf-lle-gibt-es-second}{yüksek
  riskli hastalar}'' olan 56 ölüm vardır. Gerçek ölüm nedenleri olarak
  ise virüs bağlantılı bir durum ortaya çıkmamıştır.
\item
  İsviçre hükümeti, İsviçre'nin güneyindeki (İtalya'nın yanında) durumun
  ``dramatik'' olduğunu iddia etti, ancak yerel doktorlar bunu
  \href{https://www.nzz.ch/schweiz/punkto-intensivbetten-sind-wir-im-tessin-besser-ausgeruestet-als-der-rest-der-schweiz-ld.1547728}{reddetti}
  ve her şeyin normal olduğunu söyledi.
\item
  \href{https://www.blick.ch/news/schweiz/nicht-nur-beatmungsgeraete-werden-knapp-im-kampf-gegen-corona-es-droht-ein-engpass-beim-sauerstoff-id15808185.html}{Basın
  raporlarına} göre, oksijen şişelerinin az olduğu söylendi. Bununla
  birlikte, bu durum yüksek kullanımdan değil gelecekteki olası kıtlığa
  karşı yapılan stoklamalardandır.
\item
  Birçok ülkede, zaten doktor ve hemşire sıkıntısı
  \href{https://www.washingtonpost.com/health/covid-19-hits-doctors-nurses-emts-threatening-health-system/2020/03/17/f21147e8-67aa-11ea-b313-df458622c2cc_story.html}{artmaktadır}.
  Bunun nedeni, testleri pozitif çıkan sağlık çalışanlarının, birçok
  durumda tamamen veya büyük ölçüde semptomsuz (hastalık belirtisi
  göstermeden) kalsalar bile, kendilerini karantinaya almaları
  nedeniyledir.
\end{itemize}

\hypertarget{22-mart-2020-iii}{%
\paragraph{22 Mart 2020 (III)}\label{22-mart-2020-iii}}

\begin{itemize}
\tightlist
\item
  Imperial College London'da yapılan bir model çalışmayla, İngiltere'de
  ``Covid-19'dan 250.000 ila 500.000 arasında ölüm vakası ön
  görülmüştür. Ancak çalışmayı yapanlar bu ölümlerin çoğunun normal
  yıllık ölümlere ek olarak değil, yıllık İngiltere'de yaklaşık 600.000
  kişi olan ölüm sayısının içinde bir sayı olduğunu
  \href{https://www.bbc.com/news/health-51979654}{söylemişlerdir}. Başka
  bir deyişle, aşırı (beklenin üzerindeki) ölüm sayısı düşük kalmıştır.
\item
  İtalyan Profesör
  \href{https://www.nytimes.com/2020/03/20/opinion/coronavirus-pandemic-social-distancing.html}{Walter
  Ricciardi'ye göre}, ``ölüm sertifikalarının sadece \%12'si
  koronavirüsten doğrudan bir nedensellik gösterdi'', ancak kamuya
  açıklanan raporlarda ``hastanelerde ölen herkes koronavirüsten
  ölüyor'' şeklinde bir yaklaşım vardır.
\end{itemize}

\hypertarget{21-mart-2020-i}{%
\paragraph{21 Mart 2020 (I)}\label{21-mart-2020-i}}

\begin{itemize}
\tightlist
\item
  İspanya
  \href{https://www.20minutos.es/noticia/4193883/0/media-edad-coronavirus-espana/}{65
  yaşın altında} sadece üç pozitif test ölümü bildirmiştir (toplam 1000
  kişiden). Önceden var olan sağlık koşulları ve gerçek ölüm nedenleri
  henüz bilinmemektedir.
\item
  20 Mart'ta İtalya, bir gün içinde ülke çapında 627 test pozitif ölüm
  \href{https://www.msn.com/en-au/news/coronavirus/italy-coronavirus-deaths-surge-by-627-in-a-day-lifting-total-death-toll-to-4032/ar-BB11tDnS}{bildirdi}.
  Karşılaştırıldığında, İtalya'daki normal toplam mortalite (ölüm
  sayısı) günde yaklaşık 1800 ölümdür. 21 Şubat'tan bu yana, İtalya
  yaklaşık 4000 test pozitif ölüm bildirmiştir. Bu zaman dilimi içinde
  normal toplam ölüm 50.000 ölüme kadar çıkmaktadır. Normal toplam
  mortalitenin ne ölçüde arttığı veya test pozitifliğini ne ölçüde
  arttırdığı henüz bilinmemektedir.
\item
  \href{https://www.tgcom24.mediaset.it/cronaca/coronavirus-in-lombardia-9-morti-su-10-mai-giunti-in-terapia-intensiva_16362350-202002a.shtml}{İtalyan
  medya haberlerine göre}, Lombardiya bölgesinde ölen test
  pozitiflerinin\% 90'ı yoğun bakım ünitelerinin dışında, çoğunlukla
  evde veya genel bakım bölümlerinde öldü. Ölüm nedenleri ve ölümlerinde
  karantina önlemlerinin olası rolü belirsizliğini korumaktadır. Test
  pozitif 2168 kişiden sadece 260'ı yoğun bakım ünitelerinde öldü.
\item
  Bloomberg, ``İtalya, Virüsten Ölenlerin\% 99'unun Başka Hastalıkları
  Olduğunu Vurguladı''
  \href{https://www.bloomberg.com/news/articles/2020-03-18/99-of-those-who-died-from-virus-had-other-illness-italy-says}{demektedir}.
\end{itemize}

\includegraphics{https://swprs.files.wordpress.com/2020/03/covid-iss-stat-bloomberg.png?w=550\&h=301}

\hypertarget{21-mart-2020-ii}{%
\paragraph{21 Mart 2020 (II)}\label{21-mart-2020-ii}}

\begin{itemize}
\tightlist
\item
  Japan Times şu soruya cevap aramaktadır:
  \href{https://www.japantimes.co.jp/news/2020/03/20/national/coronavirus-explosion-expected-japan/}{Japonya
  bir koronavirüs patlaması bekliyordu. Nerde kaldı?} Pozitif test
  sonuçları alan ve karantina ilan eden ilk ülkelerden biri olmasına
  rağmen, Japonya en az etkilenen ülkelerden birisidir. Alıntı olarak şu
  cümle dikkat çekicidir: ``Japonya enfekte olanların tümünü saymasa
  bile, hastanelerde yoğunluk olmamakta ve zatürre vakalarında ani artış
  olmamıştır.''
\item
  İtalyan araştırmacılar, Avrupa'nın en kötüsü olan Kuzey İtalya'daki
  aşırı dumanın, daha önce Wuhan'da olduğu gibi, oradaki mevcut pnömoni
  salgında nedensel bir rol oynayabileceğini
  \href{https://www.heise.de/tp/features/Feinstaubpartikel-als-Viren-Vehikel-4687454.html}{savunuyorlar}.
\item
  \href{https://www.youtube.com/watch?v=JBB9bA-gXL4}{Yeni bir
  röportajda}, tıbbi mikrobiyoloji alanında dünyaca ünlü bir uzman olan
  Profesör Sucharit Bhakdi, ölümler için yeni koronavirüsü tek başına
  suçlamanın ``yanlış'' ve ``tehlikeli olarak yanıltıcı'' olduğunu, buna
  karşı daha önemli faktörlerin olduğunu belirterek bunların Çin ve
  Kuzey İtalya şehirlerindeki kötü hava koşulları ve mevcut sağlık
  kondisyonları olduğunun altını çizmektedir. Profesör Bhakdi, şu anda
  tartışılan veya dayatılan önlemleri ``grotesk'', ``yararsız'',
  ``kendini yıkıcı'' ve yaşlıların ömrünü kısaltacak ve toplum
  tarafından kabul edilmemesi gereken ``toplu intihar'' olarak
  tanımlamaktadır.
\end{itemize}

\hypertarget{20-mart-2020}{%
\paragraph{20 Mart 2020}\label{20-mart-2020}}

\begin{itemize}
\tightlist
\item
  En son \href{https://www.euromomo.eu/index.html}{Avrupa izleme
  raporuna} göre, tüm ülkelerdeki (İtalya dahil) ve tüm yaş
  gruplarındaki genel mortalite (ölüm sayısı) şu ana kadar normal
  aralığın içinde veya hatta altında kalmaktadır.
\item
  Almanya'dan gelen
  \href{https://de.wikipedia.org/wiki/COVID-19-Pandemie_in_Deutschland\#Todesf\%C3\%A4lle_in_den_Medien}{en
  son istatistiklere} göre, pozitif çıkan testlerin sonucunda ortaya
  çıkan ölümlerin ortalama yaşı yaklaşık 83'tür. Bu vakaların çoğunda
  önceden var olan ve ölüme sebep olabilecek sağlık kondisyonu
  belirlenmiştir.
\item
  Stanford Profesörü John Ioannidis tarafından atıfta bulunulan
  \href{https://www.ncbi.nlm.nih.gov/pmc/articles/PMC2095096/}{2006
  tarihli bir Kanada araştırması}, genel olarak görülen soğuk algınlığı
  koronavirüsünün bakım evinde kalan kişileri içeren risk gruplarında \%
  6'ya ulaşan ölümlere yol açabileceğini belirtmiştir. 
\end{itemize}

\hypertarget{19-mart-2020-i}{%
\paragraph{19 Mart 2020 (I)}\label{19-mart-2020-i}}

İtalyan Ulusal Sağlık Enstitüsü (ISS), pozitif testli ölümler hakkında
\href{https://www.epicentro.iss.it/coronavirus/bollettino/Report-COVID-2019_17_marzo-v2.pdf}{yeni
bir rapor} yayınladı:

\begin{itemize}
\tightlist
\item
  Ortalama yaş 80.5'dir (erkekler için 79.5, kadınlar için 83.7).
\item
  Ölenlerin\% 10'u 90 yaşın üzerindeydi; Ölenlerin\% 90'ı 70 yaşın
  üzerindeydi.
\item
  Ölenlerin en fazla\% 0.8'inde önceden mevcut kronik hastalık yoktu.
\item
  Ölenlerin yaklaşık\% 75'inin önceden mevcut iki veya daha fazla durumu
  vardı,\% 50'sinin önceden mevcut üç durumu daha vardı, özellikle kalp
  hastalığı, diyabet ve kanser.
\item
  Ölenlerin beşi, hepsi önceden mevcut olan ciddi sağlık koşullarına
  (örneğin, kanser veya kalp hastalığı) sahip olan 31 ila 39 yaşları
  arasındaydı.
\item
  Ulusal Sağlık Enstitüsü, hastaların nihayetinde ne sebeple öldüğünü
  henüz belirlememiştir ve genel olarak bunlara Covid19-pozitif ölümler
  olarak atıfta bulunmaktadır.
\end{itemize}

\hypertarget{mart-19-2020-ii}{%
\paragraph{Mart 19, 2020 (II)}\label{mart-19-2020-ii}}

\begin{itemize}
\tightlist
\item
  \href{https://milano.corriere.it/notizie/cronaca/18_gennaio_10/milano-terapie-intensive-collasso-l-influenza-gia-48-malati-gravi-molte-operazioni-rinviate-c9dc43a6-f5d1-11e7-9b06-fe054c3be5b2.shtml}{İtalyan
  Corriere della Sera gazetesinde yayınlanan bir raporda}, İtalyan yoğun
  bakım ünitelerinin 2017/2018'de belirgin grip dalgası altında
  çöktüğüne dikkat çekiliyor. Bu süreçte operasyonları ertelemek ve
  izinde olan sağlık personelini izinden çağırmak zorunda kalmışlardır
  ve ayrıca kan stoklarında büyük düşüşler yaşamışlardır.
\item
  Alman virolog Hendrik Streeck, Covid19'un normalde günde yaklaşık 2500
  kişi olan Almanya'daki toplam ölüm oranını artırma olasılığının düşük
  olduğunu
  \href{https://www.faz.net/aktuell/gesellschaft/gesundheit/coronavirus/virologe-hendrik-streeck-ueber-corona-neue-symptome-entdeckt-16681450.html?printPagedArticle=true\#pageIndex_2}{savunuyor}.
  Streeck, kalp yetmezliğinden ölen, daha sonra Covid19 için pozitif
  şekilde test edilen ve böylece Covid19 ölüm istatistiklerine dahil
  edilen 78 yaşındaki bir erkek vakadan bahsetmektedir.
\item
  Stanford Profesörü John Ioannidis'e göre, yeni koronavirüs, yaşlılarda
  bile, bazı ortak koronavirüslerden daha tehlikeli olmayabilir.
  Ioannidis, şu anda kararlaştırılan önlemleri destekleyen
  \href{https://www.statnews.com/2020/03/17/a-fiasco-in-the-making-as-the-coronavirus-pandemic-takes-hold-we-are-making-decisions-without-reliable-data/}{güvenilir
  tıbbi verilerin olmadığını savunmaktadır}.
\end{itemize}

\hypertarget{18-mart-2020}{%
\paragraph{18 Mart 2020}\label{18-mart-2020}}

\begin{itemize}
\tightlist
\item
  \href{https://www.medrxiv.org/content/10.1101/2020.02.12.20022434v2}{Yeni
  bir epidemiyolojik çalışma} (baskı öncesi), Çin'in Wuhan şehrinde bile
  Covid19'un ölüm oranının sadece\% 0.04 ila\% 0.12 arasında ve
  dolayısıyla yaklaşık\% 0.1'lik bir ölüm oranına sahip mevsimsel
  gripten daha düşük olduğu sonucuna varıyor. Covid19'un aşırı bir
  şekilde abartılan ölümcüllüğünün bir nedeni olarak, araştırmacılar
  başlangıçta Wuhan'da sadece az sayıda vakanın kaydedildiğinden
  şüpheleniyor, çünkü hastalık muhtemelen birçok insanda asemptomatik
  veya hafifti.
\item
  Çinli araştırmacılar, Wuhan kentindeki aşırı kış kirli havanın pnömoni
  patlamasında nedensel bir rol oynadığını
  \href{https://www.eurasiareview.com/01022020-polluted-air-could-be-an-important-cause-of-wuhan-pneumonia-oped/}{savunuyorlar}.
  Bununla ilgili olarak 2019 yazında, kötü hava kalitesi nedeniyle
  Wuhan'da
  \href{https://www.cnn.com/2019/07/10/asia/china-wuhan-pollution-problems-intl-hnk/index.html}{halk
  protestoları} gerçekleştiğinin altını çizmek gerekmektedir.
\item
  Yeni uydu görüntüleri Kuzey İtalya'nın Avrupa'daki
  \href{https://twitter.com/esa/status/1238480433047916545}{en yüksek
  hava kirliliğine sahip} olduğunu ve bu hava kirliliğinin karantina
  sonrasında nasıl büyük ölçüde azaldığını gösteriyor.
\item
  Bir covi-19 test kiti üreticisi kitin, henüz klinik olarak
  doğrulanmadığından, teşhis uygulamaları için değil,
  \href{https://www.creative-diagnostics.com/sars-cov-2-coronavirus-multiplex-rt-qpcr-kit-277854-457.htm}{yalnızca
  araştırma amaçlı} kullanılması gerektiğini belirtmektedir.\\
\end{itemize}

\includegraphics{https://swprs.files.wordpress.com/2020/03/covid-testkit.png?w=550\&h=149}

\hypertarget{17-mart-2020-i-}{%
\paragraph{17 Mart 2020 (I) }\label{17-mart-2020-i-}}

\begin{itemize}
\tightlist
\item
  Ölümcüllük profili virolojik açıdan şaşırtıcı olmaya devam etmektedir,
  çünkü influenza-grip virüslerinin aksine çocuklar etkilenmezken,
  erkekler kadınlardan iki kat daha fazla etkilenmektedir. Öte yandan,
  \href{https://insideparadeplatz.ch/2020/03/16/notfall-stationen-bereits-seit-tagen-am-anschlag/}{bu
  görünüm}, çocuklar için sıfıra yakın ve 75 yaşındaki erkekler için
  aynı yaştaki kadınların neredeyse iki katı olan doğal ölümcüllük
  profiline karşılık gelir.
\item
  Test sonucu pozitif çıkan gençlerin önceden var olan hastalıkları
  vardı. Örneğin 21 yaşındaki İspanyol futbol koçu öldüğünde test
  pozitifti ve uluslararası manşetlere taşındı. Ancak doktorlar tipik
  komplikasyonları şiddetli pnömoni-zatürre içeren fark edilmeyen lösemi
  \href{https://sports.yahoo.com/spanish-football-coach-francisco-garcia-163153573.html}{teşhisi}
  koydu.
\item
  Hastalığın tehlikesini değerlendirmedeki belirleyici faktör, medyada
  sık sık bahsedildiği gibi, test pozitif kişilerin ve ölenlerin sayısı
  değil, aslında beklenmedik şekilde gelişen veya zatürreden ölen (aşırı
  mortalite-ölüm) insanların sayısıdır. Şimdiye kadar bu değer çoğu
  ülkede çok düşük kalmaktadır.
\item
  İsviçre'de bazı acil servisler çok sayıda kişinin
  \href{https://insideparadeplatz.ch/2020/03/16/notfall-stationen-bereits-seit-tagen-am-anschlag/}{test
  yaptırmak istemesi sebebiyle} dolmuştur. Bu da mevcut durumun ilave
  olarak psikolojik ve lojistik (hizmetin ve bilgi akışının çıkış
  noktasından varış noktasına kadar taşınması) bileşenine işaret
  etmektedir.
\end{itemize}

\hypertarget{17-mart-2020-ii}{%
\paragraph{17 Mart 2020 (II)}\label{17-mart-2020-ii}}

\begin{itemize}
\tightlist
\item
  Floransa Üniversitesi'nden İtalyan immünoloji profesörü Sergio
  Romagnani, 3000 kişi üzerinde yapılan bir çalışmada, her yaştan test
  pozitif insanların \% 50 ila 75'inin
  \href{https://www.repubblica.it/salute/medicina-e-ricerca/2020/03/16/news/coronavirus_studio_il_50-75_dei_casi_a_vo_sono_asintomatici_e_molto_contagiosi-251474302/}{tamamen
  semptomsuz} (belirti vermeyen) kaldığı sonucuna varmıştır-tahmin
  edilenden çok daha fazla.
\item
  Kuzey İtalyan yoğun bakım ünitelerinin kış aylarında doluluk oranı
  tipik olarak zaten
  \href{https://jamanetwork.com/journals/jama/fullarticle/2763188}{\% 85
  ila 90}`dır. Mevcut hastaların bazıları veya birçoğu şimdiye kadar
  test pozitif olabilir. Bununla birlikte, beklenmeyen ek
  pnömoni-zatürre vakalarının sayısı henüz bilinmemektedir.
\item
  İspanya'nın Malaga şehrindeki bir hastane doktoru,
  \href{https://twitter.com/NeurologaenSAS/status/1239498772570308609}{Twitter'da}
  insanların şu anda panik ve sistemik çöküşten dolayı virüsden daha
  fazla ölme olasılığının olduğunu yazıyor. Hastaneler soğuk algınlığı,
  grip ve muhtemelen Covid19 tarafından aşırı yüklendi ve bu sebeple
  doktorlar kontrolü kaybetmiş olabilir.
\end{itemize}

\hypertarget{14-mart-2020-}{%
\paragraph{14 Mart 2020 }\label{14-mart-2020-}}

İtalyan Ulusal Sağlık Enstitüsü İSS'nin yayınladığı
\href{https://www.epicentro.iss.it/coronavirus/sars-cov-2-decessi-italia}{son
verilere}~göre, İtalya'da testleri pozitif çıkmış ve ölmüş olan
kişilerin ortalama yaşı yaklaşık 81'dir. Ölenlerin \%10'u 90 yaşının
üzerindedir. \%90'ı ise 70 yaşının üzerindedir.

Ölenlerin \%80'i, iki ya da daha fazla sayıda kronik hastalıktan
muzdarip kişiler olmuştur. Bu kronik hastalıklar, özellikle kalp-damar
hastalıkları, diyabet türleri, solunum yolu hastalıkları ve kanseri
içermektedir.

Ölenlerin \%1'inden daha azı sağlıklı insanlardır; yani daha önceden
kronik hastalıklara sahip olmayan kişilerdir. Ölenlerin yalnızca \%30
kadarı kadındır.

İtalyan Ulusal Sağlık Enstitüsü, koronavirüsünden ölenler ile
koronavirüsü taşıyarak ölenler arasında da
bir~\href{https://youtu.be/0M4kbPDHGR0?t=210}{ayırım yapmaktadır}.
Birçok vakada insanların virüsten mi, daha önceden sahip oldukları
kronik hastalıklardan mı yoksa ikisinin bileşiminden mi öldükleri henüz
açık değildir.

40 yaşının altında ölen iki İtalyan vatandaşından (ikisi de 39
yaşındadır) biri bir kanser hastası diğeri ise bir diyabet hastası olup
ek komplikasyonlara sahipti. Bu vakalarda da kesin ölüm nedenleri (yani,
virüsten mi yoksa önceden sahip oldukları hastalıklardan mı öldükleri)
henüz belirlenmemiştir.

Hastanelerin kısmi olarak aşırı yüklenmiş oluşu, genel hasta akınına ve
özel ya da yoğun bakım hizmeti gereksinenlerin sayısındaki artıştır.
Hedef özellikle solunum işlevlerini dengelemek ve ciddi vakalarda
anti-viral terapiler sunmaktır.

(Güncelleme: İtalyan Ulusal Sağlık Enstitüsü, testleri pozitif çıkan
hastalar ve ölenler konusunda yukardaki verileri doğrulayan
bir~\href{https://www.epicentro.iss.it/coronavirus/bollettino/Report-COVID-2019_17_marzo-v2.pdf}{istatistik
raporu}~yayınlamıştır.)

\textbf{Aşağıdaki hususlar da dikkate alınmalıdır:}

Kuzey İtalya, Avrupa'nın en yaşlı nüfusuna
ve~\href{https://twitter.com/esa/status/1238480433047916545}{en kötü
hava kalitesine}~sahiptir ki bu da zaten
geçmişte~\href{https://www.thelocal.it/20170131/our-lungs-are-breaking-smog-levels-way-above-safe-limits-in-northern-italy}{artan
sayıda}~solunum yolları hastalıklarına ve ölümlere yol açmıştır ve
olasılıkla mevcut salgında da ek bir risk faktörü olmaktadır.

Örneğin, Güney Kore'de salgın İtalya'dakinden daha hafif bir seyir
izlemiş ve tepe noktasını şimdiden aşmış durumdadır. Güney Kore'de şu
ana kadar testleri pozitif çıkmış 70 ölüm vakası bildirilmiştir.
İtalya'da olduğu gibi, etkilenenler çoğunlukla yüksek-risk grubundaki
hastalardır.

İsviçre'de de testleri pozitif çıkmış birkaç düzine ölüm vakası, şu ana
kadar kronik hastalıkları olan yüksek-riskli hastalardandır. Bu
kişilerin ortalama yaşı 80'in üzerinde olup, en yüksek yaş 97'dir ve
kesin ölüm nedenleri, yani virüsten mi yoksa zaten mevcut
hastalıklarından mı öldükleri bilinmemektedir.

Dahası, yapılan incelemeler, uluslararası olarak kullanılan virüs test
kitlerinin bazı
\href{https://www.ncbi.nlm.nih.gov/pmc/articles/PMC2095096/}{vakalarda
hatalı pozitif sonuçlar} verebildiğini göstermiştir. Bu vakalarda,
kişiler yeni korona virüsüne yakalanmamış olabiliyorken, yıllık (ve şu
anda da devam eden) adi soğuk algınlığı ve grip salgınlarına yol açan
çok sayıdaki mevcut insan koronavirüs türlerinin birine de yakalanmış
olabiliyor.\\

Bu nedenle, hastalığın tehlikesine karar vermek için en önemli gösterge,
testleri pozitif çıkan insanlar ve ölümler~değil, fakat gerçekten ve
beklenmedik biçimde~zatürreye yakalanan veya bu nedenle~ölen kişilerin
(fazladan ölümler diye tabir edilen) sayısıdır.

Bütün mevcut verilere göre, okul ve çalışma çağındaki sağlıklı genel
nüfus için, hafif ile orta şiddette seyreden bir Kovid-19 beklenebilir.
Yaşlılar ve mevcut kronik hastalıkları olan kişiler korunmalıdır. Tıbbi
kapasiteler, optimum düzeyde hazırlanmalıdır.

\hypertarget{ana-madde-covid19-hakkux131nda-geruxe7ekler-1}{%
\paragraph{\texorpdfstring{\href{https://swprs.org/a-swiss-doctor-on-covid-19/}{Ana
madde: Covid19 hakkında
gerçekler}}{Ana madde: Covid19 hakkında gerçekler}}\label{ana-madde-covid19-hakkux131nda-geruxe7ekler-1}}

\begin{center}\rule{0.5\linewidth}{\linethickness}\end{center}

\hypertarget{swiss-policy-research}{%
\subsubsection{Swiss Policy Research}\label{swiss-policy-research}}

\begin{itemize}
\tightlist
\item
  \href{https://swprs.org/kontakt/}{Kontakt}
\item
  \href{https://swprs.org/uebersicht/}{Übersicht}
\item
  \href{https://swprs.org/donationen/}{Donationen}
\item
  \href{https://swprs.org/disclaimer/}{Disclaimer}
\end{itemize}

\hypertarget{english}{%
\subsubsection{English}\label{english}}

\begin{itemize}
\tightlist
\item
  \href{https://swprs.org/contact/}{About Us / Contact}
\item
  \href{https://swprs.org/media-navigator/}{The Media Navigator}
\item
  \href{https://swprs.org/the-american-empire-and-its-media/}{The CFR
  and the Media}
\item
  \href{https://swprs.org/donations/}{Donations}
\end{itemize}

\hypertarget{follow-by-email}{%
\subsubsection{Follow by email}\label{follow-by-email}}

Follow

\href{https://wordpress.com/?ref=footer_custom_com}{WordPress.com}.

\protect\hyperlink{}{Up ↑}

Post to

\protect\hyperlink{}{Cancel}

\includegraphics{https://pixel.wp.com/b.gif?v=noscript}
