\protect\hyperlink{content}{Skip to content}

\href{https://swprs.org/}{}

\protect\hyperlink{search-container}{Search}

Search for:

\href{https://swprs.org/}{\includegraphics{https://swprs.files.wordpress.com/2020/05/swiss-policy-research-logo-300.png}}

\href{https://swprs.org/}{Swiss Policy Research}

Geopolitics and Media

Menu

\begin{itemize}
\tightlist
\item
  \href{https://swprs.org}{Start}
\item
  \href{https://swprs.org/srf-propaganda-analyse/}{Studien}

  \begin{itemize}
  \tightlist
  \item
    \href{https://swprs.org/srf-propaganda-analyse/}{SRF / ZDF}
  \item
    \href{https://swprs.org/die-nzz-studie/}{NZZ-Studie}
  \item
    \href{https://swprs.org/der-propaganda-multiplikator/}{Agenturen}
  \item
    \href{https://swprs.org/die-propaganda-matrix/}{Medienmatrix}
  \end{itemize}
\item
  \href{https://swprs.org/medien-navigator/}{Analysen}

  \begin{itemize}
  \tightlist
  \item
    \href{https://swprs.org/medien-navigator/}{Navigator}
  \item
    \href{https://swprs.org/der-propaganda-schluessel/}{Techniken}
  \item
    \href{https://swprs.org/propaganda-in-der-wikipedia/}{Wikipedia}
  \item
    \href{https://swprs.org/logik-imperialer-kriege/}{Kriege}
  \end{itemize}
\item
  \href{https://swprs.org/netzwerk-medien-schweiz/}{Netzwerke}

  \begin{itemize}
  \tightlist
  \item
    \href{https://swprs.org/netzwerk-medien-schweiz/}{Schweiz}
  \item
    \href{https://swprs.org/netzwerk-medien-deutschland/}{Deutschland}
  \item
    \href{https://swprs.org/medien-in-oesterreich/}{Österreich}
  \item
    \href{https://swprs.org/das-american-empire-und-seine-medien/}{USA}
  \end{itemize}
\item
  \href{https://swprs.org/bericht-eines-journalisten/}{Fokus I}

  \begin{itemize}
  \tightlist
  \item
    \href{https://swprs.org/bericht-eines-journalisten/}{Journalistenbericht}
  \item
    \href{https://swprs.org/russische-propaganda/}{Russische Propaganda}
  \item
    \href{https://swprs.org/die-israel-lobby-fakten-und-mythen/}{Die
    »Israel-Lobby«}
  \item
    \href{https://swprs.org/geopolitik-und-paedokriminalitaet/}{Pädokriminalität}
  \end{itemize}
\item
  \href{https://swprs.org/migration-und-medien/}{Fokus II}

  \begin{itemize}
  \tightlist
  \item
    \href{https://swprs.org/covid-19-hinweis-ii/}{Coronavirus}
  \item
    \href{https://swprs.org/die-integrity-initiative/}{Integrity
    Initiative}
  \item
    \href{https://swprs.org/migration-und-medien/}{Migration \& Medien}
  \item
    \href{https://swprs.org/der-fall-magnitsky/}{Magnitsky Act}
  \end{itemize}
\item
  \href{https://swprs.org/kontakt/}{Projekt}

  \begin{itemize}
  \tightlist
  \item
    \href{https://swprs.org/kontakt/}{Kontakt}
  \item
    \href{https://swprs.org/uebersicht/}{Seitenübersicht}
  \item
    \href{https://swprs.org/medienspiegel/}{Medienspiegel}
  \item
    \href{https://swprs.org/donationen/}{Donationen}
  \end{itemize}
\item
  \href{https://swprs.org/contact/}{English}
\end{itemize}

\protect\hyperlink{}{Open Search}

\hypertarget{zur-behandlung-von-covid-19}{%
\section{Zur Behandlung
von~Covid-19}\label{zur-behandlung-von-covid-19}}

\textbf{Publiziert}: 2. Juli 2020; \textbf{Aktualisiert}: 30. Juli
2020\\
\textbf{Sprachen}:
\href{https://swprs.org/zur-behandlung-von-covid-19/}{DE},
\href{https://swprs.org/on-the-treatment-of-covid-19/}{EN};
\textbf{Teilen auf}:
\href{https://twitter.com/intent/tweet?url=https://swprs.org/zur-behandlung-von-covid-19/}{Twitter}
/
\href{https://www.facebook.com/share.php?u=https://swprs.org/zur-behandlung-von-covid-19/}{Facebook}

Immunologische und serologische Studien
\href{https://swprs.org/studies-on-covid-19-lethality/}{zeigen}, dass
die meisten Menschen durch das neue Coronavirus keine oder nur milde
Symptome entwickeln, während es bei einigen Menschen zu einem
ausgeprägten oder kritischen Krankheitsverlauf kommen kann.

Auf Basis der verfügbaren wissenschaftlichen Evidenz und der bisherigen
klinischen Erfahrungen (siehe Referenzen) empfiehlt die
SPR-Kollaboration Ärzten und Behörden, das folgende Protokoll zur
\textbf{frühzeitigen Behandlung} bei Personen mit hohem Risiko oder
hoher Exposition zu beachten.

US-Ärzte berichteten auf Basis dieses Protokolls von einem Rückgang der
Hospitali­sierungs­rate
\href{https://www.preprints.org/manuscript/202007.0025/v1}{um 84\%}, von
einem Rückgang der Sterberate bei bereits hospitalisierten Patienten
\href{https://www.henryford.com/news/2020/07/hydro-treatment-study}{um
50\%}, und von einer Verbesserung des Gesund­heits­zustandes oftmals
innerhalb von \href{https://www.youtube.com/watch?v=eVs_EWVCVPc}{wenigen
Stunden}.

\textbf{Hinweis}: Patienten werden gebeten, sich an einen Arzt zu
wenden.

\hypertarget{behandlungsprotokoll}{%
\paragraph{Behandlungsprotokoll}\label{behandlungsprotokoll}}

\begin{enumerate}
\def\labelenumi{\arabic{enumi}.}
\tightlist
\item
  Zink (75mg bis 100mg pro Tag)
\item
  Hydroxychloroquin (400mg pro Tag)
\item
  Quercetin (500mg bis 1000mg pro Tag)
\item
  Azithromycin (bis 500mg pro Tag)
\item
  Heparin (übliche Dosierung)
\end{enumerate}

Die primäre Komponente ist das \emph{Zink}, das die
RNA-Polymerase-Aktivität von Coronaviren und damit die Viren-Replikation
hemmt (siehe Referenzen unten). \emph{Hydroxychloroquin} und
\emph{Quercetin} fördern die zelluläre Aufnahme von Zink.
\emph{Azithromycin} beugt bakteriellen Superinfektionen vor.
\emph{Heparin} beugt bei Risikopatienten infektionsbedingten Thrombosen
und Embolien vor.

\textbf{Hinweis}: Quercetin kann als Ergänzung zu oder als Ersatz für
HCQ verwendet werden. Kontra­indi­kationen für HCQ (z.B. Favismus oder
Herzprobleme) und Azithromycin sind zu beachten.

\hypertarget{erluxe4uterungen}{%
\paragraph{Erläuterungen}\label{erluxe4uterungen}}

Entscheidend ist in allen Fällen eine \textbf{frühzeitige Behandlung}
bereits bei Auftreten der ersten typischen Symptome und auch ohne
PCR-Test, um eine Progression der Erkrankung zu verhindern. Zink, HCQ
und Quercetin können zudem auch
\href{https://www.mohfw.gov.in/pdf/AdvisoryontheuseofHydroxychloroquinasprophylaxisforSARSCoV2infection.pdf}{prophylaktisch}
verwendet werden.

Im Gegensatz dazu kann die Isolierung von bereits infizierten
Risikopersonen ohne frühzeitige Behandlung, wie dies durch Lockdowns
oftmals geschah, kontraproduktiv sein und zu einer Progression der
Erkrankung bis hin zur Entwicklung schwerer Atembeschwerden führen.

Die angeblich oder tatsächlich negativen Resultate mit HCQ im Rahmen
einiger Studien beruhten auf einem
\href{https://c19study.com/}{verspäteten Einsatz} (bei
Intensivpatienten),
\href{http://www.francesoir.fr/politique-monde/oxford-recovery-et-solidarity-overdosage-two-clinical-trials-acts-considered}{stark
überhöhten Dosen} (bis zu 2400mg/T),
\href{https://www.theguardian.com/world/2020/jun/03/covid-19-surgisphere-who-world-health-organization-hydroxychloroquine}{manipulierten
Datensätzen} (der Surgisphere-Skandal), oder ignorierten
\href{https://www.iss.it/documents/20126/0/Report+ISS+COVID-19_14.pdf/8a94daca-f6eb-ae95-dad7-68b9c03c8fb6}{Kontraindikationen}.

Die frühzeitige Behandlung sollte eine Hospitalisierung
\textbf{verhindern}. Kommt es dennoch zu einer Hospitalisierung, so
empfehlen erfahrene Fachleute, eine invasive Beatmung (Intubation) wann
immer möglich
\href{https://www.evms.edu/covid-19/covid_care_for_clinicians/}{zu
vermeiden} und stattdessen eine Sauerstofftherapie (HFNC) zu verwenden.

Es ist denkbar, dass das obige Behandlungsprotokoll, das einfach,
\href{https://swprs.files.wordpress.com/2020/07/hcq-white-paper-dr-simone-gold.pdf}{sicher}
und kostengünstig ist, komplexere Medikamente, Impfungen und Maßnahmen
\href{https://www.newsweek.com/key-defeating-covid-19-already-exists-we-need-start-using-it-opinion-1519535}{weitgehend
obsolet} machen könnte.

\hypertarget{hintergrund}{%
\paragraph{Hintergrund}\label{hintergrund}}

Dass HCQ gegen Infektionen mit SARS-Coronaviren wirkt, wurde bereits
2005 im Zuge der SARS-1-Epidemie
\href{https://www.ncbi.nlm.nih.gov/pmc/articles/PMC1232869/}{festgestellt}.
Dass Zink die RNA-Replikation von Coronaviren blockiert, wurde 2010 von
Ralph Baric
\href{https://www.ncbi.nlm.nih.gov/pmc/articles/PMC2973827/}{entdeckt},
einem der weltweiten führenden SARS-Virologen. Dass HCQ die zelluläre
Aufnahme von Zink fördert, wurde 2014 im Rahmen von Krebsforschung
\href{https://journals.plos.org/plosone/article?id=10.1371/journal.pone.0109180}{entdeckt}.
Dass auch das Flavonoid Quercetin die zelluläre Aufnahme von Zink
fördert, wurde ebenfalls 2014
\href{https://pubs.acs.org/doi/10.1021/jf5014633}{entdeckt}.

\hypertarget{referenzen}{%
\paragraph{Referenzen}\label{referenzen}}

\textbf{Allgemein}

\begin{itemize}
\tightlist
\item
  \href{https://www.evms.edu/covid-19/covid_care_for_clinicians/}{EVMS
  Critical Care Covid-19 Management Protocol} (Paul Marik, MD, June
  2020)
\end{itemize}

\textbf{Zink}

\begin{enumerate}
\def\labelenumi{\arabic{enumi}.}
\tightlist
\item
  \textbf{Study}: Effect of Zinc Salts on Respiratory Syncytial Virus
  Replication
  (\href{https://www.ncbi.nlm.nih.gov/pmc/articles/PMC353050/}{Suara \&
  Crowe, AAC}, 2004)
\item
  \textbf{Study}:~Zinc Inhibits Coronavirus and Arterivirus RNA
  Polymerase Activity \emph{In Vitro} and Zinc Ionophores Block the
  Replication of These Viruses in Cell Culture
  (\href{https://www.ncbi.nlm.nih.gov/pmc/articles/PMC2973827/}{Velthuis
  et al, PLOS Path}, 2010)
\item
  \textbf{Study}: Zinc for the common cold
  (\href{https://www.cochranelibrary.com/cdsr/doi/10.1002/14651858.CD001364.pub4/full}{Cochrane
  Systematic Review}, 2013)
\item
  \textbf{Study}: Hydroxychloroquine and azithromycin plus zinc vs
  hydroxychloroquine and azithromycin alone: outcomes in hospitalized
  COVID-19 patients
  (\href{https://www.medrxiv.org/content/10.1101/2020.05.02.20080036v1}{Carlucci
  et al., MedRxiv}, May 2020)
\item
  \textbf{Review}: Does zinc supplementation enhance the clinical
  efficacy of chloroquine/ hydroxychloroquine to win today's battle
  against COVID-19?
  (\href{https://www.sciencedirect.com/science/article/pii/S0306987720306435}{Derwand
  \& Scholz, MH}, 2020)
\item
  \textbf{Review}: Zinc supplementation to improve treatment outcomes
  among children diagnosed with respiratory infections
  (\href{https://www.who.int/elena/titles/bbc/zinc_pneumonia_children/en/}{WHO,
  Technical Report}, 2011)
\item
  \textbf{Article}: Can Zinc Lozenges Help with Coronavirus Infections?
  (\href{https://www.mcgill.ca/oss/article/health/can-zinc-lozenges-help-coronavirus-infections}{McGill
  University}, March 2020)
\end{enumerate}

\textbf{Hydroxychloroquin}

\begin{enumerate}
\def\labelenumi{\arabic{enumi}.}
\tightlist
\item
  \textbf{Studies}: Overview of more than 50 international HCQ studies
  (\href{https://c19study.com/}{C19Study.com})
\item
  \textbf{Study}: Chloroquine is a potent inhibitor of SARS coronavirus
  infection and spread
  (\href{https://www.ncbi.nlm.nih.gov/pmc/articles/PMC1232869/}{Vincent
  et al., Virology Journal}, 2005)
\item
  \textbf{Study}: Chloroquine Is a Zinc Ionophore
  (\href{https://journals.plos.org/plosone/article?id=10.1371/journal.pone.0109180}{Xue
  et al, PLOS One}, 2014)
\item
  \textbf{Study}: Physicians work out treatment guidelines for
  coronavirus
  (\href{http://www.koreabiomed.com/news/articleView.html?idxno=7428}{Korean
  Biomedical Review}, February 2020)\\
\item
  \textbf{Study}: Expert consensus on chloroquine phosphate for the
  treatment of novel coronavirus pneumonia
  (\href{https://pubmed.ncbi.nlm.nih.gov/32075365/}{Guangdong Health
  Commission}, February 2020)
\item
  \textbf{Study}: Clinical Efficacy of Chloroquine derivatives in
  COVID-19 Infection: Comparative meta-analysis between the Big data and
  the real world
  (\href{https://www.sciencedirect.com/science/article/pii/S2052297520300615}{Million
  et al, NMNI}, June 2020)
\item
  \textbf{Study}: Treatment with Hydroxychloroquine, Azithromycin, and
  Combination in Patients Hospitalized with COVID-19
  (\href{https://www.henryford.com/news/2020/07/hydro-treatment-study}{Arshad
  et al, Int. Journal of Infect. Diseases}, July 2020)
\item
  \textbf{Study}: COVID-19 Outpatients -- Early Risk-Stratified
  Treatment with Zinc Plus Low Dose Hydroxychloroquine and Azithromycin
  (\href{https://www.preprints.org/manuscript/202007.0025/v1}{Scholz et
  al., Preprints}, July 2020)
\item
  \textbf{Study}: Effectiveness of HCQ in COVID-19 disease
  (\href{https://www.ijidonline.com/article/S1201-9712(20)30600-7/fulltext}{Monforte
  et al.}, IJID, July 2020)
\item
  \textbf{Protocol}: Advisory on the use of HCQ as prophylaxis for
  SARS-CoV-2 infection
  (\href{https://www.mohfw.gov.in/pdf/AdvisoryontheuseofHydroxychloroquinasprophylaxisforSARSCoV2infection.pdf}{Indian
  Council of Medical Research}, March 2020)
\item
  \textbf{Review}: White Paper on Hydroxychloroquine
  (\href{https://swprs.files.wordpress.com/2020/07/hcq-white-paper-dr-simone-gold.pdf}{Dr.
  Simone Gold}, AFD, July 2020)
\item
  \textbf{Article}: The Key to Defeating COVID-19 Already Exists. We
  Need to Start Using It.
  (\href{https://www.newsweek.com/key-defeating-covid-19-already-exists-we-need-start-using-it-opinion-1519535}{Professor
  Harvey A. Risch}, Newsweek, July 2020)
\item
  \textbf{Article}: Using Hydroxychloroquine and Other Drugs to Fight
  Pandemic (\href{https://medicine.yale.edu/news-article/25085/}{Yale
  School of Medicine})
\item
  \textbf{Article}: Morocco's Chloroquine Success Reveals European
  Failures
  (\href{https://www.moroccoworldnews.com/2020/06/306587/moroccan-scientist-moroccos-chloroquine-success-reveals-european-failures/}{Morocco
  World News, June 2020}) Prof. Zemmouri believes 78\% of Europe's
  coronavirus-related deaths could have been avoided if European states
  had mirrored Morocco's chloroquine strategy.
\item
  \textbf{Article} (IT): Covid: none of my patients are dead, and only
  5\% had to be hospitalized
  \href{https://www.italiaoggi.it/news/covid-nessuno-dei-miei-e-morto-2454154}{(Italia
  Oggi, June 2020}) Dr. Cavanna treated the affected by the virus by
  intervening promptly and at home.
\end{enumerate}

\textbf{Quercetin}

\begin{enumerate}
\def\labelenumi{\arabic{enumi}.}
\tightlist
\item
  \textbf{Study}: Small molecules blocking the entry of severe acute
  respiratory syndrome coronavirus into host cells
  (\href{https://jvi.asm.org/content/78/20/11334.long}{Ling Yi et al.},
  Journal of Virology, 2004)
\item
  \textbf{Study}: Zinc Ionophore Activity of Quercetin and
  Epigallocatechin-gallate: From Hepa 1-6 Cells to a Liposome Model
  (\href{https://pubs.acs.org/doi/10.1021/jf5014633}{Dabbagh et al.,
  JAFC}, 2014)
\item
  \textbf{Study}: Quercetin as an Antiviral Agent Inhibits Influenza A
  Virus Entry
  (\href{https://www.ncbi.nlm.nih.gov/pmc/articles/PMC4728566/}{Wu et
  al, Viruses}, 2016)
\item
  \textbf{Study}: Quercetin and Vitamin C: An Experimental, Synergistic
  Therapy for the Prevention and Treatment of SARS-CoV-2 Related Disease
  (\href{https://www.frontiersin.org/articles/10.3389/fimmu.2020.01451/full}{Biancatelli
  et al, Front. in Immun.}, June 2020)
\item
  \textbf{Report}: EVMS Critical Care Covid-19 Management Protocol
  (\href{https://www.evms.edu/media/evms_public/departments/internal_medicine/EVMS_Critical_Care_COVID-19_Protocol.pdf}{Paul
  Marik, MD}, June 2020)
\end{enumerate}

\textbf{Heparin}

\begin{enumerate}
\def\labelenumi{\arabic{enumi}.}
\tightlist
\item
  \textbf{Commentary}: The versatile heparin in COVID‐19
  (\href{https://onlinelibrary.wiley.com/doi/10.1111/jth.14821}{Thachil,
  JTH}, April 2020)
\item
  \textbf{Study}: Anticoagulant Treatment Is Associated With Decreased
  Mortality in Severe Coronavirus Disease 2019 Patients With
  Coagulopathy (\href{https://pubmed.ncbi.nlm.nih.gov/32220112/}{Tang et
  al, JTH}, May 2020)
\item
  \textbf{Study}: Autopsy Findings and Venous Thromboembolism in
  Patients With COVID-19
  (\href{https://www.acpjournals.org/doi/10.7326/M20-2003}{Wichmann et
  al., Annals of Internal Medicine}, May 2020)
\item
  \textbf{Article}: Anticoagulation Guidance Emerging for Severe
  COVID-19
  (\href{https://www.medpagetoday.com/infectiousdisease/covid19/85865}{Medpage
  Today})
\end{enumerate}

\hypertarget{siehe-auch}{%
\paragraph{Siehe auch}\label{siehe-auch}}

\begin{itemize}
\tightlist
\item
  \href{https://swprs.org/covid-19-hinweis-ii/}{Fakten zu Covid-19}
\item
  \href{https://swprs.org/studies-on-covid-19-lethality/}{Studies on the
  lethality of Covid-19}
\end{itemize}

\begin{center}\rule{0.5\linewidth}{\linethickness}\end{center}

\textbf{Teilen auf}:
\href{https://twitter.com/intent/tweet?url=https://swprs.org/zur-behandlung-von-covid-19/}{Twitter}
/
\href{https://www.facebook.com/share.php?u=https://swprs.org/zur-behandlung-von-covid-19/}{Facebook}

\hypertarget{swiss-policy-research}{%
\subsubsection{Swiss Policy Research}\label{swiss-policy-research}}

\begin{itemize}
\tightlist
\item
  \href{https://swprs.org/kontakt/}{Kontakt}
\item
  \href{https://swprs.org/uebersicht/}{Übersicht}
\item
  \href{https://swprs.org/donationen/}{Donationen}
\item
  \href{https://swprs.org/disclaimer/}{Disclaimer}
\end{itemize}

\hypertarget{english}{%
\subsubsection{English}\label{english}}

\begin{itemize}
\tightlist
\item
  \href{https://swprs.org/contact/}{About Us / Contact}
\item
  \href{https://swprs.org/media-navigator/}{The Media Navigator}
\item
  \href{https://swprs.org/the-american-empire-and-its-media/}{The CFR
  and the Media}
\item
  \href{https://swprs.org/donations/}{Donations}
\end{itemize}

\hypertarget{follow-by-email}{%
\subsubsection{Follow by email}\label{follow-by-email}}

Follow

\href{https://wordpress.com/?ref=footer_custom_com}{WordPress.com}.

\protect\hyperlink{}{Up ↑}

\includegraphics{https://pixel.wp.com/b.gif?v=noscript}
