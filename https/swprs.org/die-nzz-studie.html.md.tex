\protect\hyperlink{content}{Skip to content}

\href{https://swprs.org/}{}

\protect\hyperlink{search-container}{Search}

Search for:

\href{https://swprs.org/}{\includegraphics{https://swprs.files.wordpress.com/2020/05/swiss-policy-research-logo-300.png}}

\href{https://swprs.org/}{Swiss Policy Research}

Geopolitics and Media

Menu

\begin{itemize}
\tightlist
\item
  \href{https://swprs.org}{Start}
\item
  \href{https://swprs.org/srf-propaganda-analyse/}{Studien}

  \begin{itemize}
  \tightlist
  \item
    \href{https://swprs.org/srf-propaganda-analyse/}{SRF / ZDF}
  \item
    \href{https://swprs.org/die-nzz-studie/}{NZZ-Studie}
  \item
    \href{https://swprs.org/der-propaganda-multiplikator/}{Agenturen}
  \item
    \href{https://swprs.org/die-propaganda-matrix/}{Medienmatrix}
  \end{itemize}
\item
  \href{https://swprs.org/medien-navigator/}{Analysen}

  \begin{itemize}
  \tightlist
  \item
    \href{https://swprs.org/medien-navigator/}{Navigator}
  \item
    \href{https://swprs.org/der-propaganda-schluessel/}{Techniken}
  \item
    \href{https://swprs.org/propaganda-in-der-wikipedia/}{Wikipedia}
  \item
    \href{https://swprs.org/logik-imperialer-kriege/}{Kriege}
  \end{itemize}
\item
  \href{https://swprs.org/netzwerk-medien-schweiz/}{Netzwerke}

  \begin{itemize}
  \tightlist
  \item
    \href{https://swprs.org/netzwerk-medien-schweiz/}{Schweiz}
  \item
    \href{https://swprs.org/netzwerk-medien-deutschland/}{Deutschland}
  \item
    \href{https://swprs.org/medien-in-oesterreich/}{Österreich}
  \item
    \href{https://swprs.org/das-american-empire-und-seine-medien/}{USA}
  \end{itemize}
\item
  \href{https://swprs.org/bericht-eines-journalisten/}{Fokus I}

  \begin{itemize}
  \tightlist
  \item
    \href{https://swprs.org/bericht-eines-journalisten/}{Journalistenbericht}
  \item
    \href{https://swprs.org/russische-propaganda/}{Russische Propaganda}
  \item
    \href{https://swprs.org/die-israel-lobby-fakten-und-mythen/}{Die
    »Israel-Lobby«}
  \item
    \href{https://swprs.org/geopolitik-und-paedokriminalitaet/}{Pädokriminalität}
  \end{itemize}
\item
  \href{https://swprs.org/migration-und-medien/}{Fokus II}

  \begin{itemize}
  \tightlist
  \item
    \href{https://swprs.org/covid-19-hinweis-ii/}{Coronavirus}
  \item
    \href{https://swprs.org/die-integrity-initiative/}{Integrity
    Initiative}
  \item
    \href{https://swprs.org/migration-und-medien/}{Migration \& Medien}
  \item
    \href{https://swprs.org/der-fall-magnitsky/}{Magnitsky Act}
  \end{itemize}
\item
  \href{https://swprs.org/kontakt/}{Projekt}

  \begin{itemize}
  \tightlist
  \item
    \href{https://swprs.org/kontakt/}{Kontakt}
  \item
    \href{https://swprs.org/uebersicht/}{Seitenübersicht}
  \item
    \href{https://swprs.org/medienspiegel/}{Medienspiegel}
  \item
    \href{https://swprs.org/donationen/}{Donationen}
  \end{itemize}
\item
  \href{https://swprs.org/contact/}{English}
\end{itemize}

\protect\hyperlink{}{Open Search}

\hypertarget{die-nzz-studie}{%
\section{Die NZZ-Studie}\label{die-nzz-studie}}

Die \emph{Neue Zürcher Zeitung} ist die führende Schweizer Tages­zeitung
für internationale Themen. Doch wie objektiv und kritisch berichtet die
\emph{NZZ} über geopolitische Konflikte? Um dies zu überprüfen, wurden
während je eines Monats alle \emph{NZZ}-Berichte zur Ukraine-Krise ~und
zum Syrienkrieg analysiert und anhand des Modells von Professor
Anne~Morelli auf Muster von Kriegs­propaganda hin ausgewertet.

Die Resultate sind eindeutig: Die \emph{NZZ} verbreitet in ihren
Berichten überwiegend Propaganda der Konfliktpartei
USA/NATO.~Gast­kommentare und Meinungs­beiträge geben nahezu durchgehend
die Sicht dieser Konflikt­partei wieder, während Propaganda
ausschließlich auf der Gegenseite verortet wird. Die verwendeten
Drittquellen sind unausgewogen und teilweise nicht überprüfbar.

Insgesamt muss von einer einseitigen, selektiv-unkritischen und wenig
objektiven Berichterstattung durch die \emph{Neue Zürcher Zeitung}
gesprochen werden. Verschiedene Erklärungs­versuche für diesen Befund
werden diskutiert.

\href{https://swprs.files.wordpress.com/2017/12/nzz-propaganda-studie-2016tp.pdf}{Studie
als PDF herunterladen}

\includegraphics{https://swprs.files.wordpress.com/2016/02/nzz-propaganda-insgesamt1.png?w=450\&h=309}

(Hinweis: Bei Interesse an der Studie bitte auf diese Seite verlinken.
Obige Zusammen­fassung und einzelne Auszüge können übernommen werden.
Keine Volltext-Kopie.)

\begin{center}\rule{0.5\linewidth}{\linethickness}\end{center}

\hypertarget{qualituxe4t-der-geopolitischen-berichterstattung-in-der-neuen-zuxfcrcher-zeitung-nzz}{%
\subsection{Qualität der geopolitischen Berichterstattung in der Neuen
Zürcher Zeitung
(NZZ)}\label{qualituxe4t-der-geopolitischen-berichterstattung-in-der-neuen-zuxfcrcher-zeitung-nzz}}

--

*Eine Studie von \href{https://swprs.org}{Swiss Propaganda Research}~*

März 2016

Inhaltsübersicht

\begin{enumerate}
\def\labelenumi{\arabic{enumi}.}
\tightlist
\item
  \protect\hyperlink{kapitel1}{Das Ponsonby-Morelli-Modell für
  Kriegspropaganda}
\item
  \protect\hyperlink{kapitel2}{Untersuchungsmethode}
\item
  \protect\hyperlink{kapitel3}{Resultate}
\item
  \protect\hyperlink{kapitel4}{Erklärungsversuche}
\item
  \protect\hyperlink{kapitel5}{Schlussfolgerungen}
\item
  \protect\hyperlink{kapitel6}{Anmerkungen und Literatur}
\end{enumerate}

\hypertarget{1-das-ponsonby-morelli-modell-fuxfcr-kriegspropaganda}{%
\subsubsection{1. Das Ponsonby-Morelli-Modell für
Kriegspropaganda}\label{1-das-ponsonby-morelli-modell-fuxfcr-kriegspropaganda}}

Lord Arthur Ponsonby (1871-1946) war britischer Diplomat, Politiker und
Friedensaktivist. Er veröffentlichte 1928 das Buch \emph{Falsehood in
War-Time}, in welchem er auf rund 200 Seiten die britische,
französische, deutsche, italienische und amerikanische Propaganda aus
dem Ersten Weltkrieg dokumentierte und nach Themen und Techniken
sortierte. Ponsonbys Buch gilt bis heute als Klassiker der Literatur zu
Kriegs­propaganda (Ponsonby 1928).

2001 griff Anne Morellli, Professorin für Historische Quellenkritik an
der Freien Universität Brüssel, Ponsonbys Schrift erneut auf und
destillierte daraus die \emph{Prinzipien der Kriegspropaganda}. Ihr
gleichnamiges Buch erschien 2004 in erster und 2014 in zweiter Auflage
auf deutsch. Morelli zeigt darin auf, dass die
Kriegspropaganda-Techniken, die Ponsonby 1928 im Rückblick auf den
Ersten Weltkrieg identifizierte, nichts von ihrer Gültigkeit und
Aktualität eingebüßt haben, sondern auch bei späteren heißen wie kalten
Kriegen im 20. und jungen 21. Jahrhundert von allen Konflikt­parteien
rege benutzt wurden (Morelli 2004).

Das Ponsonby-Morelli-Modell umfasst die folgenden zehn Prinzipien der
Kriegspropaganda:

\begin{enumerate}
\def\labelenumi{\arabic{enumi}.}
\tightlist
\item
  Wir wollen keinen Krieg
\item
  Das feindliche Lager trägt die alleinige Schuld am Krieg
\item
  Der Feind hat dämonische Züge
\item
  Wir kämpfen für eine gute Sache und nicht für eigennützige Ziele
\item
  Der Feind begeht mit Absicht Grausamkeiten; bei uns ist es Versehen
\item
  Der Feind verwendet unerlaubte Waffen
\item
  Unsere Verluste sind gering, die des Gegners aber enorm
\item
  Unsere Sache wird von Künstlern und Intellektuellen unterstützt
\item
  Unsere Mission ist heilig
\item
  Wer unsere Berichterstattung in Zweifel zieht, ist ein Verräter
\end{enumerate}

Mit Blick auf die geopolitischen Krisen und Kriege der letzten
Jahrzehnte ist leicht zu erkennen, dass die meisten dieser Prinzipien,
in jeweils unterschiedlicher Ausgestaltung und Betonung, im Rahmen von
Propaganda verwendet wurden -- und zwar oftmals von allen
Konfliktparteien gleichzeitig.

Eine Besonderheit des Ponsonby-Morelli-Modells liegt darin, dass nicht
von vornherein zwischen Wahrheit und Lüge unterschieden werden muss. Ob
eine Behauptung richtig oder falsch ist, lässt sich oft erst im
Nachhinein feststellen; mitunter erst Jahrzehnte später im Rahmen von
geschichtlicher Forschung. Für den unmittelbaren Effekt der
Kriegspropaganda ist der Wahrheitsgehalt einer Behauptung jedoch
grundsätzlich unerheblich. Aus historischer Sicht ist freilich ein
Großteil vergangener Kriegspropaganda aller Konfliktparteien in gewissem
Grade übertrieben, unvollständig oder schlicht falsch gewesen.

\hypertarget{2-untersuchungsmethode}{%
\subsubsection{2. Untersuchungsmethode}\label{2-untersuchungsmethode}}

Untersucht wurde die NZZ-Berichterstattung vom April 2014 zur
Ukraine-Krise sowie vom Oktober 2015 zum Syrienkrieg. Dies waren die
jeweils ersten Kalendermonate nach Eskalation des Konfliktes zwischen
den beiden geopolitischen Akteuren USA/NATO und Russland (Einbindung der
Krim in die Russische Föderation am 21.~März 2014 und Eintritt Russlands
in den Syrienkrieg am 30. September 2015).

Die Artikel wurden mittels einer Volltext-Stichwortsuche nach
``Ukraine'' bzw. ``Syrien'' abgerufen. Berücksichtigt wurden sämtliche
Artikel in den genannten Zeiträumen, bei denen aus dem Titel, Untertitel
oder der Einleitung hervorging, dass sie sich direkt mit dem jeweiligen
Konflikt befassen würden. Nicht berücksichtigt wurden Agenturmeldungen,
reine Börsen- bzw. Finanzmeldungen, Sportberichte, sowie kurze
Einleitungs- und Übersichtsartikel, die lediglich auf einen Haupttext
verwiesen. Nicht berücksichtigt wurden ferner Artikel der
Sonntags­ausgabe (\emph{NZZ am Sonntag}), da diese von einer eigenen
Redaktion erstellt wird.

Den genannten Kriterien entsprachen insgesamt 133 Artikel (99 zur
Ukraine, 34 zu Syrien), darunter 13 Meinungs­beiträge und ein Interview.
Anhand des in Kapitel eins beschriebenen Ponsonby-Morelli-Modells wurden
diese Artikel sodann Satz für Satz auf~ Propaganda-Botschaften hin
untersucht. Entsprechende Sätze oder Formulierungen wurden in NATO- und
NATO-kritische Propaganda eingeteilt, nach den Prinzipien eins bis zehn
kategorisiert und pro Artikel aufsummiert.

Die geopolitischen Konflikt­parteien wurden dabei wie folgt definiert.
Die Konfliktpartei \emph{USA/NATO} oder kurz \emph{NATO} umfasst die von
den USA angeführte Militärallianz, deren Mitgliedsländer, sowie deren
unmittelbare Verbündete. Zu den Verbündeten wurden gezählt: in der
Ukraine-Krise die pro-westliche Opposition, die im Februar 2014 in Kiew
an die Macht kam; im Syrienkrieg die oppositionellen Gruppierungen, die
von NATO-Mitgliedern offiziell unterstützt wurden. Die Konfliktpartei
\emph{Russland} umfasst die russische Föderation und deren unmittelbare
Verbündete: in der Ukraine-Krise die pro-russischen Gruppierungen auf
der Krim und in der Ostukraine; im Syrienkrieg das syrische Regime von
Präsident Al-Asad und regimetreue Gruppierungen (bspw. die Hisbollah).
Im Syrienkrieg nicht berücksichtigt wurde der sogenannte ``Islamische
Staat'' (ISIS), da diese Gruppierung keiner der geopolitischen
Konfliktparteien eindeutig zugeordnet werden konnte. Propaganda für oder
gegen ISIS wurde deshalb nicht erfasst.

Zusätzlich wurde mittels Volltext-Stichwortsuche nach ``Propaganda''
oder ``Propagandist'' untersucht, bei welchen Gelegenheiten~die
\emph{NZZ} selbst~Propaganda von einer der Konflikt­parteien
identifizierte. Schließlich wurden die von der \emph{NZZ} verwendeten
Drittquellen auf ihre Ausrichtung und Überprüfbarkeit hin analysiert.

Mit der gewählten Methode nicht untersucht werden konnten allfällige
Bilder und Fotografien, obschon auch diese natürlich
Propaganda-Botschaften transportieren können. Ebenfalls nicht erfasst
wurde allfällige Propaganda, die durch \emph{Weglassen} von
Informationen entstehen kann. Schließlich sei nochmals daran erinnert,
dass das Ponsonby-Morelli-Modell nicht zwischen ``wahrer'' und
``falscher'' Propaganda unterscheidet (sofern sich dies überhaupt
bestimmen lässt), sondern lediglich die verschiedenen
Propaganda-Botschaften und -Prinzipien erfasst. Zur Propaganda-Theorie
siehe auch das Literaturverzeichnis zu dieser Studie (u.a. Baines 2013,
Bussemer 2008, Starkulla 2015).

\hypertarget{3-resultate}{%
\subsubsection{3. Resultate}\label{3-resultate}}

Im Folgenden werden die Resultate der verschiedenen Analysen präsentiert
und grafisch dargestellt.

\hypertarget{31-verwendung-von-propaganda-insgesamt}{%
\paragraph{3.1 Verwendung von Propaganda
insgesamt}\label{31-verwendung-von-propaganda-insgesamt}}

Die Auswertung aller 133 NZZ-Artikel zum Ukraine- und Syrienkonflikt
ergab insgesamt 833 Kriegspropaganda-Botschaften, d.h. pro Artikel
durchschnittlich 6.3 Botschaften. Davon waren 739 Botschaften oder 89\%
NATO-Propaganda und 94 Botschaften oder 11\% NATO-kritische Propaganda.
Wie Abbildung 1 zeigt, unterscheiden sich diese Werte zwischen dem
Ukraine- und Syrienkonflikt nur um wenige Prozentpunkte.

\includegraphics{https://swprs.files.wordpress.com/2016/02/nzz-propaganda-insgesamt1.png?w=600\&h=412}\emph{Abbildung
1: Verwendung von Propaganda-Botschaften in der NZZ.}

\paragraph{}

\hypertarget{32-ausgewogenheit-der-artikel}{%
\paragraph{3.2 Ausgewogenheit der
Artikel}\label{32-ausgewogenheit-der-artikel}}

Die folgende Abbildung 2 illustriert die Ausgewogenheit der
\emph{NZZ}-Artikel anhand der Propaganda-Botschaften, unterteilt in
NATO-lastige, NATO-kritische, ausgewogene, und Propaganda-freie Artikel.

\includegraphics{https://swprs.files.wordpress.com/2017/08/nzz-ausgewogenheit2.png?w=600\&h=364}

*Abbildung 2: Ausgewogenheit der NZZ-Artikel nach Anzahl
Propaganda-Botschaften.\\
*

Immerhin 11\% der untersuchten \emph{NZZ}-Artikel enthielten keinerlei
Propaganda. Oftmals handelte es sich dabei um Berichte, die den
jeweiligen Konflikt nur indirekt berührten und beispielsweise eine
Reaktion in der Schweiz, eine Veranstaltung von Ölfirmen in Genf oder
einen Transfer von Museumsstücken in die Ukraine behandelten. Es gab
aber auch einige wenige Artikel, die sich direkt mit dem Ukraine- oder
Syrienkonflikt befassten und dennoch ganz ohne Propaganda auskamen,
indem sie die Vorgänge oder Sichtweisen sehr sachlich beschrieben.

Weitere 5\% der untersuchten Artikel enthielten zwar
Propaganda-Botschaften, jedoch in einem ausgeglichenen Verhältnis
zwischen NATO- und NATO-kritischer Propaganda. Dies wird üblicherweise
erzielt, wenn einer Propaganda-Behauptung jeweils eine Gegenbehauptung
gegenübergestellt wird, oder wenn Propaganda-Behauptungen kritisch
hinterfragt werden. Solche Artikel können sich insgesamt dennoch negativ
oder kritisch zu einer Konfliktpartei (z.B. zu Russland) äußern, aber
die Anzahl der Propaganda-Botschaften ist ausgeglichen.

Überwiegend NATO-kritische Propaganda enthielt im Beobachtungszeitraum
keiner der \emph{NZZ}-Artikel. Ein solcher Artikel hätte beispielsweise
mehrheitlich Propaganda der russischen Konfliktpartei transportieren
oder NATO-Propaganda mit mehreren Gegen­argumenten hinterfragen müssen.
Dies wurde in der~ Bericht­erstattung und Kommentierung durch die
\emph{NZZ} nicht beobachtet.

Vielmehr wurde bei 84\% aller Artikel beobachtet, dass diese überwiegend
NATO-Propaganda transportierten. Das Spektrum reichte dabei von einem
leichten Propaganda-Überhang in ansonsten sachlichen Berichten bis hin
zu vergleichsweise schrillen Artikeln mit ein bis zwei Dutzend
Propaganda-Botschaften der Konfliktpartei USA/NATO.

\hypertarget{33-propaganda-nach-zeitungsressort}{%
\paragraph{3.3 Propaganda nach
Zeitungsressort}\label{33-propaganda-nach-zeitungsressort}}

Auf Ebene der Zeitungsressorts wurde untersucht, wie einseitig oder
ausgeglichen die jeweiligen Artikel durchschnittlich waren. Dazu wurde
die Differenz gebildet aus der Anzahl NATO- und NATO-kritischer
Propagandabotschaften und dieser Wert ins Verhältnis zur Anzahl der
Artikel pro Ressort gesetzt. Abbildung 3 zeigt die Resultate.

\includegraphics{https://swprs.files.wordpress.com/2016/02/nzz-ressort1.png?w=600\&h=364}

\emph{Abbildung 3: Differenzwert aus der Anzahl
NATO-Propagandabotschaften minus NATO-kritischer Propagandabotschaften,
dargestellt nach Zeitungsressort insgesamt (linke Skala) und
durch­schnitt­lich pro Artikel (rechte Skala), als Maß für die
Einseitigkeit der Zeitungsressorts.}

Aufgrund der Anzahl Artikel führt insgesamt das Ressort International
mit einem Nettowert von 330 NATO-Propagandabotschaften. Pro Artikel ist
jedoch das Ressort ``Meinung und Debatte'' bei weitem am einseitigsten:
Der durchschnittliche Überhang an NATO-Propaganda beträgt hier 11.1
Botschaften pro Artikel. Auf den weiteren Plätzen folgen das
überraschend propagandistische Feuilleton (6.5 Botschaften pro Artikel),
die Frontseite (4.8), das Ressort International (4.2) sowie die
Wirtschaft (3.4). Die Ressorts Schweiz und Panorama trugen hingegen kaum
zur Propaganda bei.

\hypertarget{34-ausrichtung-von-meinungs--und-gastbeitruxe4gen}{%
\paragraph{3.4 Ausrichtung von Meinungs- und
Gastbeiträgen}\label{34-ausrichtung-von-meinungs--und-gastbeitruxe4gen}}

Um das Bild von der Ausgewogenheit und Objektivität der
\emph{NZZ}-Berichterstattung zu vervollständigen, wurde zusätzlich die
Ausrichtung aller Meinungs- und Gastbeiträge im Beobachtungszeitraum
untersucht. Im Vergleich zur gewöhnlichen Berichterstattung erlauben
Meinungsbeiträge einen freieren Ausdruck von Positionen, während
Gastbeiträge den Einbezug externer Sichtweisen und Fachkompetenzen
ermöglichen. Die Resultate sind in Abbildung 4 ersichtlich.

\includegraphics{https://swprs.files.wordpress.com/2016/01/nzz-meinungen2.png?w=600\&h=365}

*Abbildung 4: Ausrichtung der Gastbeiträge und der Meinungsartikel
gemessen an der Anzahl Propaganda-Botschaften.\\
*

Erneut zeigt sich ein ausgesprochen einseitiges Bild: Von den insgesamt
14 Gastbeiträgen enthielten 12 überwiegend NATO-Propaganda (86\%), zwei
waren bezüglich der Propaganda­botschaften ausgeglichen, und keiner war
überwiegend NATO-kritisch. Noch einseitiger sah es im Ressort ``Meinung
\& Debatte'' aus: 12 von 13 Beiträgen (92\%) waren NATO-lastig, nur eine
Meinung (gleichzeitig ein Gastbeitrag) war ausgewogen, und keine war
NATO-kritisch. Von einer grundsätzlichen ``Debatte'', wie der
Ressorttitel suggeriert, kann insofern eigentlich keine Rede sein.
Bezüglich Geopolitik wird in der \emph{NZZ} höchstens über nachrangige
Fragestellungen debattiert -- ein Befund, der sich mit Studien und
Auswertungen zur Presse in Deutschland deckt (vgl. Krüger 2013).

\hypertarget{35-huxe4ufigkeit-der-einzelnen-propaganda-prinzipien}{%
\paragraph{3.5 Häufigkeit der einzelnen
Propaganda-Prinzipien}\label{35-huxe4ufigkeit-der-einzelnen-propaganda-prinzipien}}

In einem nächsten Schritt wurde ausgewertet, wie häufig die einzelnen
Propaganda-Prinzipien gemäß Ponsonby-Morelli in der
\emph{NZZ}-Bericht­erstattung verwendet wurden. Dabei zeigten sich
deutliche Unterschiede zwischen NATO- und NATO-kritischer Propaganda,
wie Abbildung 5 weiter unten illustriert.

Bei der NATO-Propaganda dominierten die Prinzipien 3 (``Der Feind hat
dämonische Züge'') und 2 (``Das feindliche Lager trägt die alleinige
Schuld am Krieg'') mit je 24\%. Im ersten Fall handelte es sich
insbesondere um die häufigen Botschaften und Formulierungen zur
Dämonisierung des russischen wie auch des syrischen Präsidenten
(``\emph{\ldots{} die Lage eskaliert stündlich, während Putin das alles
im Hintergrund still geniesst}``; ``\emph{Millionen sind vertrieben,
Hunderttausende tot, Städte und Kulturerbe liegen in Trümmern -- nur
einen lässt die Situation kalt: Bashar al-Asad \ldots{}}''
{[}\protect\hyperlink{anm1}{1}{]}). Im zweiten Fall wurde die Schuld für
den Ausbruch, die Eskalation oder die Fortdauer der Krise bzw. des
Krieges jeweils pauschal der Konfliktpartei Russland \& Verbündete
zugeschoben.

Auf dem nächsten Platz folgt das Prinzip 4 (``Wir kämpfen für eine gute
Sache und nicht für eigennützige Ziele'') mit 17\%. Hierbei wurde
üblicherweise betont, dass die Konfliktpartei USA/NATO für
demokratisch-rechtsstaatliche Werte, das Völkerrecht, die Sicherheit
(Ost-)Europas oder die Freiheit des syrischen Volkes kämpfe, und nicht
etwa für eine Ausdehnung des eigenen Machtbereichs oder die Kontrolle
von Ressourcen, wie dies der gegnerischen Konfliktpartei zugeschrieben
wurde.

Auf den weiteren Plätzen folgen die Prinzipien 1 (``Wir wollen keinen
Krieg'') und das beachtlich häufig verwendete Prinzip 10 (``Wer unsere
Berichterstattung in Zweifel zieht, ist ein Verräter'') mit je 8\%. Bei
ersterem wurde betont, dass die Konfliktpartei NATO den Frieden wolle
und lediglich auf die Aggression, Provokation oder Bedrohung der
Gegenseite reagieren müsse. Bei zweiterem wurden Kritiker als Verräter
oder Propagandisten der Feindseite dargestellt. Die später
bekanntgewordenen Begriffe ``Putinversteher'' und ``Putin-Troll'' wurden
dabei im Beobachtungs­zeitraum zur Ukraine-Krise (April 2014) noch nicht
verwendet.

Mit je 7\% folgen das Prinzip 5 (``Der Feind begeht mit Absicht
Grausamkeiten; bei uns ist es Versehen'') und das Prinzip 7 (``Unsere
Verluste sind gering, die des Gegners aber enorm''). Bei ersterem wurde
etwa eine (angeblich) brutale Kriegsführung der Konfliktpartei Russland
\& Verbündete angeprangert (z.B. die Bombardierung ziviler
Einrichtungen), oder aber Vorkommnisse auf Seiten der Konfliktpartei
NATO \& Verbündete als Versehen oder Einzelfälle dargestellt. Im zweiten
Fall wurde insbesondere von der diplomatischen Isolierung der
Konfliktpartei Russland gesprochen oder die wirtschaftlichen
Auswirkungen von Sanktionen gegen diese Konfliktpartei thematisiert. Um
militärische Verluste ging es im Beob­ach­tungs­zeitraum nur vereinzelt.

Schließlich folgen noch die Prinzipien 6 (``Der Feind verwendet
unerlaubte Waffen'') mit 4.5\% und 8 (``Unsere Sache wird von Künstlern
und Intellektuellen unterstützt'') mit 1.4\%. Die Verwendung unerlaubter
Waffen oder Kriegstechniken durch die russische Konfliktpartei wurde
sowohl in Syrien wie auch in der (Ost-)Ukraine behauptet (Verwendung von
Giftgas und Fassbomben, illegale Waffenlieferungen etc.). Die
``Unterstützung durch Künstler und Intellektuelle'' fällt zwar
prozentual nicht ins Gewicht, sollte als Propagandatechnik jedoch nicht
unterschätzt werden: Die \emph{NZZ} ließ insbesondere zur Ukraine-Krise
gleich mehrere Gastbeiträge von Künstlern und Intellektuellen schreiben
oder griff in Beiträgen auf sie zurück. Das Prinzip 9 (``Unsere Mission
ist heilig'') wurde auf Seiten USA/NATO hingegen kaum verwendet
(Verteidigung des Christentums, Segnung des Krieges durch Geistliche
etc.).

Bei der NATO-kritischen Propaganda ergab sich ein anderes Bild: Hier
spielte mit 54\% fast nur das Prinzip 2 (``Das feindliche Lager trägt
die alleinige Schuld am Krieg'') eine Rolle. Dieses kam insbesondere
dann zum Zuge, wenn Mitteilungen der russischen Regierung oder des
russischen Militärs zitiert wurden, die die Konfliktpartei NATO \&
Verbündete für die Krise oder den Krieg verantwortlich machten. Mit
Prinzip 4 (``Wir kämpfen für eine gute Sache und nicht für eigennützige
Ziele'', 14\%) betonte die russische Konfliktpartei ihrerseits, selbst
für eine gute Sache zu kämpfen (z.B. Sicherheit Russlands,
Selbstbestimmung der Völker in der Ukraine, Bekämpfung des Terrorismus
in Syrien). Die übrigen Prinzipien kamen nur vereinzelt vor, was auch
daran liegt, dass die NATO-kritische Propaganda insgesamt nur 11\% der
Propaganda in der \emph{NZZ} ausmachte (siehe oben).

\includegraphics{https://swprs.files.wordpress.com/2016/02/nzz-prinzipien6.png?w=600\&h=364}

\emph{Abbildung 5: Häufigkeit der verwendeten Propaganda-Prinzipien
(gemäß Ponsonby-Morelli), unterteilt in NATO- und NATO-kritische
Propaganda.}

\hypertarget{36-identifikation-von-propaganda-durch-die-nzz}{%
\paragraph{\texorpdfstring{3.6 Identifikation von Propaganda durch die
\emph{NZZ}}{3.6 Identifikation von Propaganda durch die NZZ}}\label{36-identifikation-von-propaganda-durch-die-nzz}}

In einer zusätzlichen Analyse wurde untersucht, in welchen
Zusammenhängen die \emph{NZZ}-Autoren selbst von ``Propaganda'' oder
``Propagandisten'' sprachen. Die Auswertung ergab 37 Nennungen in
insgesamt 20 Artikeln. In 86\% der Fälle wurde die Propaganda auf Seiten
der Konfliktpartei Russland und Verbündete verortet, in 14\% der Fälle
war die Nennung neutral oder unbestimmt, und in 0\% der Fälle wurde
Propaganda auf Seiten der Konfliktpartei USA/NATO verortet (siehe
Abbildung 6).

\includegraphics{https://swprs.files.wordpress.com/2016/01/nzz-verortung2.png?w=600\&h=361}

\emph{Abbildung 6: Verortung von Propaganda durch die NZZ.}

Hier ist zunächst positiv festzuhalten, dass sich die \emph{NZZ}
wenigstens mit Propaganda von einer der Konfliktparteien, nämlich der
russischen, intensiv und kritisch auseinandersetzt. Im
Beobachtungs­zeitraum waren sowohl zur Ukraine-Krise wie auch zum
Syrienkrieg gleich mehrere Artikel der russischen Propaganda gewidmet
{[}\protect\hyperlink{anm2}{2}{]}.

Einschränkend muss jedoch angemerkt werden, dass die \emph{NZZ} zwar
ausgiebig \emph{über} die russische Konfliktpartei spricht, aber kaum je
\emph{mit} ihr (oder ihren Verbündeten, etwa in Syrien). Dies ist umso
bemerkenswerter, als die \emph{NZZ} meist über eigene Korrespondenten
vor Ort in Russland oder der jeweiligen Konfliktregion verfügte. Die
Korrespondenten lieferten der Leserschaft indes wenig Informationen, die
nicht auch von zuhause aus zugänglich gewesen wären (z.B. via Internet).
Und wenn doch lokale Quellen genutzt wurden, so waren diese wiederum
mehrheitlich der westlichen Konfliktpartei USA/NATO zuzuordnen
(Oppositions­medien, Regimekritiker etc.; vgl. hierzu die separate
Auswertung der Drittquellen).

Der ehemalige niederländische Nahost-Korrespondent und
Kriegsberichterstatter \emph{Joris Luyendijk} konstatierte in diesem
Zusammenhang, dass es bei der sogenannten \emph{Ortszeile} (``berichtet
aus Moskau'' etc.) oftmals eher um Prestige und vermeintliche
Glaubwürdigkeit~ gehe, und weniger darum, den Lesern durch die Präsenz
vor Ort tatsächlich einen Mehrwert zu bieten oder gar das Verständnis
für die ``gegnerische'' Konfliktpartei zu fördern (Luyendijk 2015).

Noch bedenklicher ist jedoch der Umstand, dass die \emph{NZZ} im
Beobachtungs­zeitraum keinerlei Propaganda auf Seiten der Konfliktpartei
USA/NATO thematisiert, sondern diese im Gegenteil weitgehend unkritisch
an die Leserschaft weiterreicht. Dies wirkt wenig glaubwürdig, zumal
allein in den letzten zwei Jahrzehnten zahlreiche Propaganda-Operationen
dieser Konfliktpartei gut dokumentiert sind (Bittermann 1994,
Becker/Beham 2008, Kutz 2014, Tilgner 2003). Mitunter werden selbst
offensichtlich fragwürdige Angaben ohne Vorbehalt rapportiert
{[}\protect\hyperlink{anm3}{3}{]}.

Bezeichnend für dieses Verhalten der \emph{NZZ} ist ein Artikel vom 14.
April 2014 mit folgender Passage: \emph{``Dass die Amerikaner einst die
falsche Behauptung verbreiteten, Saddam Hussein habe
Massen­vernichtungs­waffen, fand man ja auch nicht so chic. Es gab viel
Lärm damals.''} {[}\protect\hyperlink{anm4}{4}{]} Hier wird eine
bekannte und wahrheitswidrige Propaganda-Behauptung der Konfliktpartei
USA/NATO, die zentral war für die öffentliche Begründung des
Irak-Krieges im Jahre 2003, vom \emph{NZZ}-Autor indirekt (``fand man'')
als ``nicht so chic'' beschrieben und Kritik an dieser Behauptung als
``viel Lärm'' bezeichnet (den es dem Shakespeare'schen Sprichwort
zufolge bekanntlich ``um nichts'' gibt). Die Kritik der \emph{NZZ}
richtet sich also nicht gegen die Propaganda der Konfliktpartei
USA/NATO, sondern im Gegenteil gegen die Kritik an dieser Propaganda.
Erklärungs­versuche für ein solch auffälliges Verhalten werden in
Kapitel 4 diskutiert.

\hypertarget{37-drittquellen}{%
\paragraph{3.7 Drittquellen}\label{37-drittquellen}}

In einer weiteren Analyse wurden die von den \emph{NZZ}-Autoren
verwendeten Drittquellen untersucht, d.h. Quellen, die nicht direkt
einer der Konfliktparteien angehören. Dies schließt Regierungs- und
Militär­angehörige aus, beinhaltet jedoch beispielsweise internationale
Organisationen wie die UNO und OSZE, Menschen­rechts­organisationen,
Beobachter, Experten oder andere Medien. Hierbei wurde anhand von
öffentlich zugänglichen Informationen bezüglich Finanzierung, Leitung,
Zusammen­setzung, Mitgliedschaften und dergleichen untersucht, ob die
verwendeten Drittquellen von ihrer grundsätzlichen Ausrichtung her einer
der beiden Konflikt­parteien zuzuordnen, neutral oder aber unbestimmbar
waren, und ob dies von der \emph{NZZ} transparent gemacht wurde.
Abbildung 7 stellt die Ergebnisse dar.

\includegraphics{https://swprs.files.wordpress.com/2017/08/nzz-drittquellen.png?w=600\&h=353}

\emph{Abbildung 7: Ausrichtung der von der NZZ verwendeten
Drittquellen.}

Die Auswertung ergab, dass 63\% der von der \emph{NZZ} verwendeten
Drittquellen von ihrer grundsätzlichen Ausrichtung her der
Konfliktpartei NATO \& Verbündete zuzuordnen waren, 15\% der
Konfliktpartei Russland \& Verbündete, 9\% als neutral anzusehen waren,
und 13\% nicht zugeordnet werden konnten. Erneut zeigt sich also eine
deutliche Einseitigkeit zugunsten der Konfliktpartei USA/NATO.

Ein ähnliches Bild ergibt sich mit Blick auf die Kennzeichnung der
verwendeten Drittquellen durch die NZZ. Bei den grundsätzlich neutralen
Quellen wie der UNO oder der OSZE ist die Bezeichnung im Allgemeinen
unproblematisch. Auch Quellen, die von ihrer Ausrichtung grundsätzlich
der Konfliktpartei Russland \& Verbündete zuzurechnen sind, werden von
der NZZ meist deutlich als solche gekennzeichnet (``regimetreu'',
``umstritten und besonders regimenah'', ``Staatsmedien'' und
dergleichen).

Anders sieht es hingegen bei den Drittquellen im Umfeld der
Konfliktpartei NATO \& Verbündete aus. Hier fehlt vielfach eine klare
Angabe, oder es wird sogar vermeintliche Neutralität suggeriert:
Regime-kritische und oppositionelle Medien werden nicht als solche
gekennzeichnet, die einschlägige Finanzierung von Organisationen wird
nicht erwähnt, oder es wird neutral von ``Journalisten­kollektiv'',
``Terror­experte'', ``Menschen­rechts­organisation'' und dergleichen
gesprochen. Ohne eigene Recherchen hat die Leserschaft keine Möglichkeit
herauszufinden, wer sich etwa hinter dem ``Kiewer Institut für
Massen­information'', der ``Syrischen Beobachtungs­stelle für
Menschen­rechte'', dem ``Internetportal Nowosti Donbasa'', der
``Organisation \emph{Adopt a Revolution''}, oder auch der ``Deutschen
Gesellschaft für Auswärtige Politik'' verbirgt
{[}\protect\hyperlink{anm5}{5}{]}.

Teilweise sind jedoch nicht einmal solche Recherchen möglich, da die
Quellenangabe völlig unklar und damit unüberprüfbar ist: Die \emph{NZZ}
beruft sich etwa auf ``verschiedene Kommentatoren'', ``internationale
Beobachter'', ``Medienberichte'', ``Meinungsumfragen'', ``einige
Analytiker'', ``anonyme Beamte'', ``verschiedene Medien unter Berufung
auf anonyme Quellen in Washington'', oder einfach nur ``manche'' ohne
irgendwelchen Kontext. Eine solche Dissimulierungs- oder
Verbergungs­strategie geht oftmals mit einer stillschweigenden
Delegierung der Informations­überprüfung an die jeweilige Quelle einher
und ist in der medien­kritischen Literatur vielfach belegt (siehe z.B.
Dirks 2010).

Auf diese Weise begünstigt die NZZ im Endeffekt die Verbreitung von
Propagandabotschaften der Konfliktpartei USA/NATO. Denn bereits 2003
warnte Ulrich Tilgner, der langjährige Nahost-Korrespondent des
\emph{ZDF} und \emph{Schweizer Fernsehens}, im Rückblick auf den
Irak-Krieg:

\emph{``Mit Hilfe der Medien bestimmen die Militärs zugleich die
öffentliche Wahrnehmung und nutzen sie für ihre Planungen. Sie schaffen
es, Erwartungen zu wecken und Szenarien und Täuschungen zu verbreiten.
In dieser neuen Art von Krieg erfüllen die PR-Strategen der
US-Administration eine ähnliche Funktion wie sonst die Bomberpiloten.
Die Spezial-Abteilungen für Öffentlichkeits­arbeit im Pentagon und in
den Geheim­diensten sind zu Kombattanten im Informationskrieg geworden.
\ldots{}} *Dabei nutzen die amerikanischen Militärs die mangelnde
Transparenz der Berichterstattung in den Medien gezielt für ihre
Täuschungs­manöver. Die von ihnen gestreuten Informationen, die von
Zeitungen und Rundfunk aufgenommen und verbreitet werden, können Leser,
Zuhörer oder Zuschauer unmöglich bis zur Quelle zurückverfolgen. Somit
gelingt es ihnen nicht, die ursprüngliche Absicht der Militärs zu
erkennen. \ldots{} Journalisten werden so als Mittel genutzt, den
Kriegsgegner in die Irre zu führen. Information wird zum Bestandteil der
Kriegsführung: zum Informationskrieg.'' (Tilgner 2003, S. 132ff)\\
*

Aufgrund ihrer Quellenverwendung befindet sich die \emph{NZZ} somit
weitgehend im ``in sich geschlossenen Informations­kreislauf westlicher
Demokratien während Kriegszeiten'' (Becker 2008), wie er in Abb. 8
dargestellt ist. Von außerhalb dringen kaum Informationen durch, und die
eigene investigative Leistung ist minimal. Die \emph{NZZ} macht diesen
Kreislauf jedoch nicht transparent, denn Meta-Informationen zur
Quellenauswahl und -interpretation sucht man in ihren Artikeln meist
vergebens. Dies deckt sich mit früheren Untersuchungen zur deutschen
Presse (Dirks 2010, Krüger 2013, Kutz 2014).

\includegraphics{https://swprs.files.wordpress.com/2016/01/nzz-informationskreislauf.png?w=600\&h=414}

\emph{Abbildung 8: Der in sich geschlossene Informations­kreislauf
westlicher Demokratien während Kriegszeiten. Quelle: Becker/Beham,
2008.}

\hypertarget{4-erkluxe4rungsversuche}{%
\subsubsection{4. Erklärungsversuche}\label{4-erkluxe4rungsversuche}}

Die Resultate in Kapitel 3 sind eindeutig: Die \emph{NZZ} verwendet in
ihrer Berichterstattung zu geopolitischen Konflikten überwiegend
Propaganda der Konfliktpartei USA/NATO. Gastkommentare und
Meinungsbeiträge geben nahezu durchgehend die Sicht dieser
Konfliktpartei wieder, während Propaganda ausschließlich auf der
Gegenseite verortet wird. Die genutzten Drittquellen sind unausgewogen
und teilweise nicht überprüfbar. Insgesamt muss von einer einseitigen,
selektiv-unkritischen und wenig objektiven Berichterstattung durch die
\emph{NZZ} gesprochen werden.

Wie können diese bemerkenswerten Resultate erklärt werden? Es bieten
sich hierzu verschiedene Ansätze an:

\textbf{1. Naivität oder Fahrlässigkeit}: ``\emph{Wer glaubt, im Krieg
von Militärs wirklich informiert zu werden, ist naiv''}, schrieb Ulrich
Tilgner 2003 im Rückblick auf den Irak-Krieg (Tilgner 2003). Die
\emph{NZZ} verbreitet ebensolche Informationen der Konfliktpartei
USA/NATO ausgiebig und oftmals vorbehaltslos, doch sind die \emph{NZZ}
und ihre Autoren deshalb naiv? Dies erscheint wenig plausibel. Die
umfassenden Propaganda­strategien der USA/NATO sind spätestens seit den
Golf- und Balkankriegen auch unter Journalisten bestens bekannt
(Bittermann 1994), sodass Naivität als mögliche Begründung ausscheiden
muss. Auch Fahrlässigkeit, etwa aufgrund von Zeit- und Kostendruck,
kommt als Erklärung nicht in Frage: Hierfür ist die Stoßrichtung der
Berichterstattung und der Meinungsbeiträge in der \emph{NZZ} viel zu
konsistent.

\textbf{2. Auflagendruck und Lesererwartungen:} Denkbar wäre, dass die
\emph{NZZ} ihre Berichterstattung an den (vermeintlichen) Erwartungen
der Leserschaft ausrichtet, um die Verkaufszahlen möglichst hoch zu
halten. Mit dieser Überlegung wird etwa die zunehmende Boulevardisierung
des modernen Journalismus zu begründen versucht. Im vorliegenden Kontext
ist eine solche Erklärung jedoch wenig plausibel: So erntete die
\emph{NZZ} für ihre geopolitische Berichterstattung teils heftige
Leserkritik {[}\protect\hyperlink{anm6}{6}{]}, und die Auflagenzahlen
gingen zuletzt markant zurück {[}\protect\hyperlink{anm7}{7}{]}.
Insofern macht es eher den Anschein, dass die \emph{NZZ} sogar
\emph{entgegen} einer verbreiteten Lesererwartung an ihrer einseitigen
und wenig objektiven Berichterstattung festhält. Dieses auch
markt­wirt­schaftlich erstaunliche Verhalten müsste umso mehr erklärt
werden.

\textbf{3. Ideologie:} Die \emph{NZZ} verpflichtet sich in ihren
Statuten {[}\protect\hyperlink{anm8}{8}{]} auf eine
``freisinnig-demokratische Grundhaltung'' (wobei das historische
``freisinnig'' fälschlicherweise oft als ``wirtschaftsliberal''
aufgefasst wird).~ Nun ist es denkbar, dass die \emph{NZZ} mit dieser
Grundhaltung primär die Konfliktpartei USA/NATO assoziiert und letztere
deshalb bei geopolitischen Konflikten publizistisch unterstützen möchte.
Diese Überlegung könnte in der Tat zur Erklärung der vorliegenden
Studienresultate beitragen.

Dennoch überzeugt der ideologische Ansatz nur bedingt: Einerseits hat
die genannte Konfliktpartei in den vergangenen Jahrzehnten selbst
vielfach und in eklatanter Weise gegen die statutarische Grundhaltung
der \emph{NZZ} verstoßen (beispielsweise durch Sturz demokratischer
Regierungen und wiederholter Verletzung von Völker- und Menschenrechten
(Sylvan/Majeski 2009)). Die pauschale Zuschreibung einer
freisinnig-demokratischen Grundhaltung erscheint deshalb fragwürdig.
Doch selbst in diesem Fall bliebe unklar, weshalb die \emph{NZZ} die
Propaganda dieser Konfliktpartei weitgehend vorbehaltslos transportiert
und damit die eigene journalistische Glaubwürdigkeit unterminiert.

\textbf{4. Militärische, politische und ökonomische Abhängigkeit der
Schweiz:} Die Schweiz ist politisch offiziell neutral, faktisch jedoch
in vielerlei Hinsicht von der Konfliktpartei USA/NATO abhängig.
Militärisch liegt das Land vollständig im Einflussgebiet der NATO und
ist mit dieser in den 1990er Jahren eine Partnerschaft eingegangen.
Politisch ist die Schweiz von der Europäischen Union umgeben, deren
Mitglieder wiederum größtenteils der NATO angehören oder von dieser
abhängig sind. Und auch ökonomisch besteht eine hohe Dependenz, da der
überwiegende Teil des Schweizerischen Außenhandels (Importe und Exporte)
mit Mitgliedern der NATO stattfindet und das Schweizer Finanzsystem auf
das Wohlwollen der USA angewiesen ist {[}\protect\hyperlink{anm9}{9}{]}.

Insofern ist es denkbar, dass diese Abhängigkeit zu einer grundsätzlich
wohlgesinnten Berichterstattung der Konfliktpartei USA/NATO gegenüber
führt, zumal gerade die \emph{NZZ} und ihre Aktionäre die Schweizer
Wirtschaft repräsentieren, die naturgemäß ein hohes Interesse an
ungetrübten Beziehungen mit einem solch dominanten Partner haben muss.
Dies könnte womöglich~ erklären, warum die \emph{NZZ} im Sinne einer
freiwilligen Selbstzensur eine einseitige, selektiv-unkritische und
wenig objektive Berichterstattung bei geopolitischen Konflikten
bevorzugt.

\textbf{5. Eliten-Netzwerke:} In den letzten Jahren wurde vermehrt die
publizistische Bedeutung von Eliten-Netzwerken thematisiert und deren
Einfluss auf den Journalismus untersucht, insbesondere im Falle
deutsch-transatlantischer Netzwerke und Vereinigungen. Dabei konnte
gezeigt werden, dass Journalisten, die Mitglied in solchen Netzwerken
sind, zumeist wohlgesinnt berichten und kommentieren, wenn sich die
Konfliktpartei USA/NATO in geopolitischen Auseinandersetzungen befindet
(Krüger 2013). Für die Schweiz liegen hierzu bislang jedoch kaum
empirische Befunde vor.

Bekannt ist immerhin, dass auch die Verleger und Chefredakteure der
etablierten Schweizer Medien regelmäßig an Konferenzen der
transatlantischen Elite aus Politik, Militär und Wirtschaft teilnehmen,
meist ohne dass sie darüber berichten würden
{[}\protect\hyperlink{anm10}{10}{]}. Im Falle der \emph{NZZ} ist zudem
bekannt, dass der vormalige Auslandschef und heutige Chefredakteur, Eric
Gujer, als ``Atlantiker'' gilt {[}\protect\hyperlink{anm11}{11}{]} und
Kontakte pflegt zu neokonservativen Kreisen in den USA, etwa dem
\emph{Project for the New American Century} (PNAC)
{[}\protect\hyperlink{anm12}{12}{]}, welches vom ehemaligen
US-Vizepräsidenten Dick Cheney mitgegründet wurde.

--

Insgesamt erscheint es plausibel, dass eine Kombination aus der
militärisch-politisch-ökonomischen Abhängigkeit der Schweiz, der
ideologischen Ausrichtung der \emph{NZZ}, sowie der Einbindung von
Schlüssel­personen der \emph{NZZ} in transatlantische Elite-Netzwerke in
Summe zu einer Berichterstattung führen, die der Konfliktpartei USA/NATO
wohlgesinnt ist und deren Kriegspropaganda weitgehend unkritisch
transportiert, wie in dieser Studie nachgewiesen wurde.

\hypertarget{5-schlussfolgerungen}{%
\subsubsection{5. Schlussfolgerungen}\label{5-schlussfolgerungen}}

Die vorliegende Untersuchung kam anhand des Propaganda-Modells von
Ponsonby-Morelli zum Ergebnis, dass die \emph{NZZ} bei geopolitischen
Konflikten überwiegend Propaganda der Konfliktpartei USA/NATO
verbreitet, Propaganda nur auf der Gegenseite identifiziert,
unausgewogene und teilweise wenig transparente Drittquellen verwendet
und damit insgesamt einseitig und wenig objektiv berichtet und
kommentiert. Als Erklärung für diesen Befund wurde eine Kombination aus
ideologischer Ausrichtung, transatlantischer Eliten-Netzwerke, und
militärisch-politisch-ökonomischer Abhängigkeit der Schweiz von der
Konfliktpartei USA/NATO vorgeschlagen.

Die gewonnen Resultate stimmen weitgehend mit medienkritischen
Untersuchungen etwa in Deutschland überein, wo solche Effekte in der
Berichterstattung zu geopolitischen Konflikten wiederholt nachgewiesen
wurden (Bilke 2008, Dirks 2010, Krüger 2013, Zagala 2007). In der
Schweiz dürfte die vorliegende Untersuchung hingegen ein Novum
darstellen.

Zu prüfen wäre, inwiefern dieser Befund auch für die anderen etablierten
Medien in der Schweiz Gültigkeit hat. Sollte die NATO-konforme
Berichterstattung der \emph{NZZ} wie dargelegt einer freiwilligen
Selbstzensur aus oben genannten Gründen geschuldet sein, so wäre zu
erwarten, dass ähnliche Muster auch bei den übrigen Schweizer Medien
nachgewiesen werden können, womöglich in etwas abgeschwächter Form. Denn
nur landesweit könnte eine solche Maßnahme die gewünschte Wirkung
erzielen. Dabei ist zu bedenken, dass inzwischen über 90\% des
konventionellen Schweizer Medienmarktes von nur fünf großen
Verlagshäusern abgedeckt werden und somit eine hohe Medien­konzentration
vorliegt {[}\protect\hyperlink{anm13}{13}{]}.

Eine ``freiwillige Selbstzensur'' entspräche zudem im Wesentlichen dem
\emph{Modus Operandi} der Schweizer Medien im 20. Jahrhundert: Um die
Schweiz während geopolitischer Konflikte keinen unnötigen Risiken
auszusetzen, hatten die Medien und sogar Buchverlage im Ersten und
Zweiten Weltkrieg sowie im Kalten Krieg bei ihrer Berichterstattung und
Kommentierung einen politisch vorgegebenen Meinungskorridor zu beachten,
der sich an den jeweiligen geopolitischen Kräfte­verhältnissen
orientierte (Bollmann/Oppenheim 2004, Keller 2009, Kreis 1973).

In Anbetracht dieser Tatsachen wäre es nicht weiter erstaunlich, wenn
die erwähnten publizistischen Mechanismen und Vorsichts­maßnahmen auch
im heutigen ``Global War on Terror''~ zur Anwendung kommen. Indes dürfte
ihre Akzeptanz in der Bevölkerung pro­blema­tischer sein als ehedem, da
die mediale Darstellung heutzutage aufgrund techno­logischer
Möglich­keiten leichter überprüft und mit Gegen­darstellungen verglichen
werden kann. Die Anforderungen an eine glaubwürdige geopolitische
Berichterstattung sind in diesem Sinne klar gestiegen.

\begin{center}\rule{0.5\linewidth}{\linethickness}\end{center}

\hypertarget{studie-inklusive-anhang-als-pdf-herunterladen}{%
\subparagraph{\texorpdfstring{\href{https://swprs.files.wordpress.com/2017/12/nzz-propaganda-studie-2016tp.pdf}{Studie
inklusive Anhang als PDF
herunterladen}}{Studie inklusive Anhang als PDF herunterladen}}\label{studie-inklusive-anhang-als-pdf-herunterladen}}

\begin{center}\rule{0.5\linewidth}{\linethickness}\end{center}

\hypertarget{anmerkungen}{%
\subsubsection{Anmerkungen}\label{anmerkungen}}

Bei allen NZZ-Referenzen ist der gedruckte Zeitungsartikel maßgebend.
Die Online-Versionen können in Titel wie Inhalt davon abweichen und
werden hier nur zur Information verlinkt.

{[}1\protect\hyperlink{anmb1}{\^{}}{]} Zitate aus den NZZ-Artikeln
\href{http://www.nzz.ch/staus-und-offensiven-in-berlins-aussenpolitik-1.18293143}{Staus
und Offensiven in Berlins Aussenpolitik} vom 30. April 2014 und
\href{http://www.nzz.ch/meinung/kommentare/die-spieler-von-damaskus-ld.2506}{Die
Spieler von Damaskus} vom 16. Oktober 2015.

{[}2\protect\hyperlink{anmb2}{\^{}}{]} Siehe z.B. die NZZ-Artikel
\href{http://www.nzz.ch/das-luegen-karussell-dreht-sich-immer-schneller-1.18285557}{Auf
Lügen gebaut} und
\href{http://www.nzz.ch/moskaus-einpeitscher-vom-dienst-1.18285883}{Moskaus
Einpeitscher vom Dienst} vom 17. April 2014,
\href{http://www.nzz.ch/feuilleton/medien/die-putin-show-1.18314288}{Putins
Show} vom 19. April 2014,
\href{http://www.nzz.ch/international/europa/von-gemuesesuppe-und-hoeflichen-piloten-1.18629861}{Von
Gemüsesuppe und höflichen Piloten} vom 15. Oktober 2015.

{[}3\protect\hyperlink{anmb3}{\^{}}{]} Beispielsweise erwähnt die NZZ
unter dem Titel
\href{http://www.nzz.ch/kiew-identifiziert-die-todesschuetzen-1.18277076}{Kiew
identifiziert die Todesschützen vom Maidan} am 4. April 2014 zunächst,
dass die vormalige Opposition selbst zu den Verdächtigen der
Maidan-Morde gehört habe*.* Dennoch präsentiert sie im Folgenden die
Untersuchungs­resultate eben dieser vormaligen Opposition als Fakten
ohne jeden Vorbehalt und Zweifel.

{[}4\protect\hyperlink{anmb4}{\^{}}{]}
\href{http://www.nzz.ch/besetzung-luegen-und-fiese-visa-tricks-1.18283463}{Besetzung,
Lügen und fiese Visa-Tricks,} NZZ vom 14. April 2014.

{[}5\protect\hyperlink{anmb5}{\^{}}{]} Das
\href{http://imi.org.ua/en/}{Kiewer Institut für Massen­information}
wird u.a. von der USAID, der US NED und dem Washingtoner Freedom House
unterstützt. Die
\href{http://www.theguardian.com/commentisfree/2012/jul/12/syrian-opposition-doing-the-talking}{Syrische
Beobachtungs­stelle für Menschen­rechte} wird (angeblich) von einem
einzigen syrischen Exilanten in London betrieben. Das
\href{https://cpj.org/blog/2014/07/mission-journal-attacks-on-journalists-in-ukraine-.php}{Internetportal
Nowosti Donbasa} wird von einem aus dem Donbass nach Kiew geflüchteten
Journalisten betrieben. Die Organisation
\href{https://www.adoptrevolution.org/}{Adopt a Revolution} wurde von
der syrischen Opposition mit Unterstützung der USA gegründet. Die
\href{https://dgap.org/de}{Deutsche Gesellschaft für Auswärtige Politik}
wurde vom US Council on Foreign Relations mitgegründet

{[}6\protect\hyperlink{anmb6}{\^{}}{]}
\href{http://www.nzz.ch/meinung/blogs/medienblog/312/2014/05/02/rebellion-unter-den-lesern/}{Rebellion
unter den Lesern}, NZZ vom 2. Mai 2014.

{[}7\protect\hyperlink{anmb7}{\^{}}{]}
\href{http://www.persoenlich.com/medien/zeitungen-und-zeitschriften-verlieren-markant-an-lesern-319976}{Zeitungen
und Zeitschriften verlieren markant an Lesern}, WEMF Studie 2014.

{[}8\protect\hyperlink{anmb8}{\^{}}{]} NZZ:
\href{https://swprs.files.wordpress.com/2019/03/statuten_nzz_2016.pdf}{Statuten
der Aktiengesellschaft für die Neue Zürcher Zeitung} (2016)

{[}9\protect\hyperlink{anmb9}{\^{}}{]} Vgl. die Schweizerische
Außenhandelsbilanz:
\href{http://www.aussenhandel.admin.ch}{www.aussenhandel.admin.ch}

{[}10\protect\hyperlink{anmb10}{\^{}}{]} Die Verleger und Chefredakteure
der wichtigsten Schweizer Medienhäuser nehmen beispielsweise im Turnus
an der sogenannten
\href{https://swprs.org/die-konferenz/}{Bilderberg-Konferenz} teil, an
der sich die trans­atlantische Elite aus Politik, Wirtschaft und Militär
im privaten Rahmen trifft.

{[}11\protect\hyperlink{anmb11}{\^{}}{]}
\href{http://bazonline.ch/schweiz/Ein-Atlantiker-an-der-Spitze/story/18216373}{Ein
Atlantiker an der Spitze}, Basler Zeitung vom 12. März 2015.

{[}12\protect\hyperlink{anmb12}{\^{}}{]} Siehe z.B. das Buch
\href{https://www.amazon.com/Safety-Liberty-Islamist-Terrorism-Counterterrorism/dp/084474333X}{Safety,
Liberty, and Islamist Terrorism} des American Enterprise Institutes, an
welchem Eric Gujer zusammen mit Gary J. Schmitt schrieb, dem ehemaligen
Executive Director des Project for the New American Century (PNAC).

{[}13\protect\hyperlink{anmb13}{\^{}}{]} Es sind dies die SRG, Tamedia,
Ringier, NZZ Medien und AZ Medien. Siehe das
\href{http://www.foeg.uzh.ch/jahrbuch.html}{Jahrbuch Qualität der
Medien} des Forschungsinstituts für Öffentlichkeit und Medien der
Universität Zürich.

\begin{center}\rule{0.5\linewidth}{\linethickness}\end{center}

\hypertarget{literatur}{%
\subsubsection{Literatur}\label{literatur}}

Baines, Paul R (2013): Propaganda. Volume I-IV. \emph{SAGE Library of
Military and Strategic Studies}, London.

Bilke, Nadine (2008): Qualität in der Krisen- und
Kriegsberichterstattung. Ein Modell für einen konfliktsensitiven
Journalismus. \emph{VS Verlag für Sozialwissenschaften}, Wiesbaden.

Becker, Jörg \& Beham, Miram (2008): Operation Balkan: Werbung für Krieg
und Tod. \emph{Nomos}, Baden-Baden.

Bittermann, Klaus (1994): Serbien muss sterbien. Wahrheit und Lüge im
jugoslawischen Bürgerkrieg. \emph{Edition TIAMAT}, Berlin.

Bollmann, Ulrich \& Oppenheim, Roy (2004): Die Stimme, die durch Beton
geht. \emph{Buag}, Baden.

Bussemer, Thymian (2008): Propaganda. Konzepte und Theorien. \emph{VS
Verlag für Sozialwissenschaften}, Wiesbaden.

Dirks, Una (2010): Der Irak-Konflikt in den Medien. Eine sprach-,
politik- und kommunikationswissenschaftliche Analyse. \emph{UVK},
Konstanz.

Keller, Stefan (2009): Im Gebiet des Unneutralen. Schweizerische
Buchzensur im Zweiten Weltkrieg zwischen Nationalsozialismus und
Geistiger Landesverteidigung. \emph{Chronos}, Zürich.

Kreis, Georg (1973): Zensur und Selbstzensur: Die schweizerische
Pressepolitik im Zweiten Weltkrieg. \emph{Huber \& Co.}, Frauenfeld.

Krüger, Uwe (2013): Meinungsmacht. Der Einfluss von Eliten auf
Leitmedien und Alpha-Journalisten -- eine kritische Netzwerkanalyse.
\emph{Halem}, Köln.

Kutz, Magnus-Sebastian (2014): Öffentlichkeitsarbeit in Kriegen.
Legitimation von Kosovo-, Afghanistan- und Irakkrieg in Deutschland und
den USA. \emph{Springer VS}, Wiesbaden.

Luyendijk, Joris (2015): Von Bildern und Lügen in Zeiten des Krieges:
Aus dem Leben eines Kriegsberichterstatters -- Aktualisierte Neuausgabe.
\emph{Tropen}, Stuttgart.

Morelli, Anne (2004): Die Prinzipien der Kriegspropaganda. \emph{zu
Klampen}, Springe.

Ponsonby, Arthur (1928): Falsehood in War-Time. \emph{George Allen \&
Unwin}, London.

Starkulla, Heinz jr. (2015): Propaganda: Begriffe, Typen, Phänomene.
\emph{Nomos}, Baden-Baden.

Sylvan, David \& Majeski, Stephen (2009): U.S. Foreign Policy in
Perspective: Clients, enemies and empire. \emph{Routledge}, London.

Tilgner, Ulrich (2003): Der inszenierte Krieg -- Täuschung und Wahrheit
beim Sturz Saddam Husseins. \emph{Rowohlt}, Reinbek.

Zagala, Samera (2007): Kulturkampf in den Medien. Wie Fernsehnachrichten
die arabische Welt abbilden. \emph{VDM Verlag Dr. Müller}, Saarbrücken.

\hypertarget{swiss-policy-research}{%
\subsubsection{Swiss Policy Research}\label{swiss-policy-research}}

\begin{itemize}
\tightlist
\item
  \href{https://swprs.org/kontakt/}{Kontakt}
\item
  \href{https://swprs.org/uebersicht/}{Übersicht}
\item
  \href{https://swprs.org/donationen/}{Donationen}
\item
  \href{https://swprs.org/disclaimer/}{Disclaimer}
\end{itemize}

\hypertarget{english}{%
\subsubsection{English}\label{english}}

\begin{itemize}
\tightlist
\item
  \href{https://swprs.org/contact/}{About Us / Contact}
\item
  \href{https://swprs.org/media-navigator/}{The Media Navigator}
\item
  \href{https://swprs.org/the-american-empire-and-its-media/}{The CFR
  and the Media}
\item
  \href{https://swprs.org/donations/}{Donations}
\end{itemize}

\hypertarget{follow-by-email}{%
\subsubsection{Follow by email}\label{follow-by-email}}

Follow

\href{https://wordpress.com/?ref=footer_custom_com}{WordPress.com}.

\protect\hyperlink{}{Up ↑}

Post to

\protect\hyperlink{}{Cancel}

\includegraphics{https://pixel.wp.com/b.gif?v=noscript}
