\protect\hyperlink{content}{Skip to content}

\href{https://swprs.org/}{}

\protect\hyperlink{search-container}{Search}

Search for:

\href{https://swprs.org/}{\includegraphics{https://swprs.files.wordpress.com/2020/05/swiss-policy-research-logo-300.png}}

\href{https://swprs.org/}{Swiss Policy Research}

Geopolitics and Media

Menu

\begin{itemize}
\tightlist
\item
  \href{https://swprs.org}{Start}
\item
  \href{https://swprs.org/srf-propaganda-analyse/}{Studien}

  \begin{itemize}
  \tightlist
  \item
    \href{https://swprs.org/srf-propaganda-analyse/}{SRF / ZDF}
  \item
    \href{https://swprs.org/die-nzz-studie/}{NZZ-Studie}
  \item
    \href{https://swprs.org/der-propaganda-multiplikator/}{Agenturen}
  \item
    \href{https://swprs.org/die-propaganda-matrix/}{Medienmatrix}
  \end{itemize}
\item
  \href{https://swprs.org/medien-navigator/}{Analysen}

  \begin{itemize}
  \tightlist
  \item
    \href{https://swprs.org/medien-navigator/}{Navigator}
  \item
    \href{https://swprs.org/der-propaganda-schluessel/}{Techniken}
  \item
    \href{https://swprs.org/propaganda-in-der-wikipedia/}{Wikipedia}
  \item
    \href{https://swprs.org/logik-imperialer-kriege/}{Kriege}
  \end{itemize}
\item
  \href{https://swprs.org/netzwerk-medien-schweiz/}{Netzwerke}

  \begin{itemize}
  \tightlist
  \item
    \href{https://swprs.org/netzwerk-medien-schweiz/}{Schweiz}
  \item
    \href{https://swprs.org/netzwerk-medien-deutschland/}{Deutschland}
  \item
    \href{https://swprs.org/medien-in-oesterreich/}{Österreich}
  \item
    \href{https://swprs.org/das-american-empire-und-seine-medien/}{USA}
  \end{itemize}
\item
  \href{https://swprs.org/bericht-eines-journalisten/}{Fokus I}

  \begin{itemize}
  \tightlist
  \item
    \href{https://swprs.org/bericht-eines-journalisten/}{Journalistenbericht}
  \item
    \href{https://swprs.org/russische-propaganda/}{Russische Propaganda}
  \item
    \href{https://swprs.org/die-israel-lobby-fakten-und-mythen/}{Die
    »Israel-Lobby«}
  \item
    \href{https://swprs.org/geopolitik-und-paedokriminalitaet/}{Pädokriminalität}
  \end{itemize}
\item
  \href{https://swprs.org/migration-und-medien/}{Fokus II}

  \begin{itemize}
  \tightlist
  \item
    \href{https://swprs.org/covid-19-hinweis-ii/}{Coronavirus}
  \item
    \href{https://swprs.org/die-integrity-initiative/}{Integrity
    Initiative}
  \item
    \href{https://swprs.org/migration-und-medien/}{Migration \& Medien}
  \item
    \href{https://swprs.org/der-fall-magnitsky/}{Magnitsky Act}
  \end{itemize}
\item
  \href{https://swprs.org/kontakt/}{Projekt}

  \begin{itemize}
  \tightlist
  \item
    \href{https://swprs.org/kontakt/}{Kontakt}
  \item
    \href{https://swprs.org/uebersicht/}{Seitenübersicht}
  \item
    \href{https://swprs.org/medienspiegel/}{Medienspiegel}
  \item
    \href{https://swprs.org/donationen/}{Donationen}
  \end{itemize}
\item
  \href{https://swprs.org/contact/}{English}
\end{itemize}

\protect\hyperlink{}{Open Search}

\hypertarget{ux3b1ux3bdux3bfux3b9ux3c7ux3c4ux3ae-ux3b5ux3c0ux3b9ux3c3ux3c4ux3bfux3bbux3ae-ux3b1ux3c0ux3cc-ux3c4ux3bfux3bd-ux3b4ux3c1-sucharit-bhakdi-ux3c0ux3c1ux3bfux3c2-ux3c4ux3b7-ux3b3ux3b5ux3c1ux3bcux3b1ux3bdux3afux3b4ux3b1-ux3baux3b1ux3b3ux3baux3b5ux3bbux3acux3c1ux3b9ux3bf-ux3b4ux3c1-angela-merkel}{%
\section{Ανοιχτή επιστολή από τον Δρ. Sucharit Bhakdi προς τη Γερμανίδα
Καγκελάριο Δρ
Angela~Merkel}\label{ux3b1ux3bdux3bfux3b9ux3c7ux3c4ux3ae-ux3b5ux3c0ux3b9ux3c3ux3c4ux3bfux3bbux3ae-ux3b1ux3c0ux3cc-ux3c4ux3bfux3bd-ux3b4ux3c1-sucharit-bhakdi-ux3c0ux3c1ux3bfux3c2-ux3c4ux3b7-ux3b3ux3b5ux3c1ux3bcux3b1ux3bdux3afux3b4ux3b1-ux3baux3b1ux3b3ux3baux3b5ux3bbux3acux3c1ux3b9ux3bf-ux3b4ux3c1-angela-merkel}}

\includegraphics{https://swprs.files.wordpress.com/2020/03/bakhdi-letter-header.png?w=736\&h=297}

\textbf{Γλώσσες}:
\href{https://swprs.org/offener-brief-von-professor-sucharit-bhakdi-an-bundeskanzlerin-dr-angela-merkel/}{DE},
\href{https://swprs.org/open-letter-from-professor-sucharit-bhakdi-to-german-chancellor-dr-angela-merkel/}{EN};
\href{https://swprs.org/professor-sucharit-bhakdi-avalik-kiri-angela-merkelile/}{EE},
\href{http://piensachile.com/2020/03/carta-abierta-a-angela-merkel/}{ES},
\href{https://swprs.org/covid-19-lettre-ouverte-du-professeur-sucharit-bhakdi-a-la-chanceliere-allemande-dre-angela-merkel/}{FR},
\href{https://swprs.org/professor-bhakdi-open-letter-greek/}{GR},
\href{https://yanivhamo.com/open-letter-from-professor-sucharit-bhakdi-to-german-chancellor-dr-angela-merkel-hebrew/}{HE},
\href{https://swprs.org/lettera-aperta-del-professor-sucharit-bhakdi-al-cancelliere-tedesco-dr-angela-merkel/}{IT},
\href{https://swprs.org/open-brief-van-professor-sucharit-bhakdi-aan-de-duitse-bondskanselier-dr-angela-merkel/}{NL},
\href{https://swprs.org/carta-aberta-do-professor-sucharit-bhakdi-a-chanceler-alema-dra-angela-merkel/}{PT},
\href{https://swprs.org/\%d0\%be\%d1\%82\%d0\%ba\%d1\%80\%d1\%8b\%d1\%82\%d0\%be\%d0\%b5-\%d0\%bf\%d0\%b8\%d1\%81\%d1\%8c\%d0\%bc\%d0\%be-\%d0\%bf\%d1\%80\%d0\%be\%d1\%84\%d0\%b5\%d1\%81\%d1\%81\%d0\%be\%d1\%80\%d0\%b0-\%d1\%81\%d1\%83\%d1\%87\%d0\%b0\%d1\%80\%d0\%b8\%d1\%82\%d0\%b0/}{RU},
\href{https://alatyr.sk/open-letter-from-professor_sk.htm}{SK},
\href{https://swprs.org/prof-dr-sucharit-bhakdiden-basbakan-dr-angela-merkele-acik-mektup/}{TR}

Ανοιχτή επιστολή από τον Δρ. Sucharit Bhakdi, Ομότιμο Καθηγητή Ιατρικής
Μικροβιολογίας στο Πανεπιστήμιο Johannes Gutenberg Mainz, προς τη
Γερμανίδα Καγκελάριο Δρ Angela Merkel. Ο καθηγητής Bhakdi ζητά επείγουσα
επανεκτίμηση της ανταπόκρισης στον Covid-19 και θέτει στην Καγκελάριο
πέντε κρίσιμα ερωτήματα. Η επιστολή έχει ημερομηνία 26 Μαρτίου. Αυτή
είναι μια ανεπίσημη μετάφραση. δείτε το αρχικό γράμμα
\href{https://swprs.org/offener-brief-von-professor-sucharit-bhakdi-an-bundeskanzlerin-dr-angela-merkel/}{στα
γερμανικά ως PDF.}

\hypertarget{ux3b1ux3bdux3bfux3b9ux3c7ux3c4ux3ae-ux3b5ux3c0ux3b9ux3c3ux3c4ux3bfux3bbux3ae}{%
\subsubsection{Ανοιχτή
επιστολή}\label{ux3b1ux3bdux3bfux3b9ux3c7ux3c4ux3ae-ux3b5ux3c0ux3b9ux3c3ux3c4ux3bfux3bbux3ae}}

Αγαπητή καγκελάριε,

Ως ομότιμος καθηγητής του Πανεπιστημίου Johannes-Gutenberg στο Mainz και
επί μακρόν διευθυντής του Ινστιτούτου Ιατρικής Μικροβιολογίας,
αισθάνομαι υποχρεωμένος να αμφισβητήσω κριτικά τους εκτεταμένους
περιορισμούς στη δημόσια ζωή, με τους οποίους έχουμε έρθει αντιμέτωποι
επί του παρόντος, για να μειώσουμε τη διάδοση του Ιού covid-19.

Δεν είναι σαφώς πρόθεσή μου να υποτιμήσω τους κινδύνους του ιού ή να
διαδώσω ένα πολιτικό μήνυμα. Ωστόσο, πιστεύω ότι είναι καθήκον μου να
συνεισφέρω επιστημονικά στην τοποθέτηση των τρεχόντων δεδομένων και
γεγονότων -- και, επιπλέον, να υποβάλλω ερωτήσεις που κινδυνεύουν να
χαθούν στην εν θερμώ συζήτηση.

Ο λόγος για την ανησυχία μου έγκειται κυρίως στις πραγματικά απρόβλεπτες
κοινωνικοοικονομικές συνέπειες των δραστικών μέτρων περιορισμού που
εφαρμόζονται επί του παρόντος σε μεγάλα μέρη της Ευρώπης και τα οποία
ήδη εφαρμόζονται επίσης σε μεγάλη κλίμακα στη Γερμανία.

Επιθυμία μου είναι να συζητήσω κριτικά -- και με την απαραίτητη
προοπτική -- τα πλεονεκτήματα και τα μειονεκτήματα του περιορισμού της
δημόσιας ζωής και των συνακόλουθων μακροπρόθεσμων αποτελεσμάτων.

Προς τούτο,~ έρχομαι αντιμέτωπος με πέντε ερωτήσεις που δεν έχουν
απαντηθεί επαρκώς μέχρι στιγμής, αλλά είναι απαραίτητες για μια
ισορροπημένη ανάλυση.

Θα ήθελα να σας ζητήσω να σχολιάσετε γρήγορα και, ταυτόχρονα, να
ζητήσετε από την Ομοσπονδιακή Κυβέρνηση να αναπτύξει στρατηγικές που
προστατεύουν αποτελεσματικά τις ομάδες κινδύνου χωρίς να περιορίζουν τη
δημόσια ζωή σε γενικές γραμμές και να σπέρνουν τους σπόρους για μια
ακόμη πιο εντατική πόλωση της κοινωνίας από αυτήν που ήδη λαμβάνει χώρα.

Με~ μέγιστο σεβασμό,

\textbf{Καθ. Em. Δρ. Med. Sucharit Bhakdi}

\hypertarget{1-ux3c3ux3c4ux3b1ux3c4ux3b9ux3c3ux3c4ux3b9ux3baux3adux3c2}{%
\subparagraph{\texorpdfstring{\textbf{1.
Στατιστικές}}{1. Στατιστικές}}\label{1-ux3c3ux3c4ux3b1ux3c4ux3b9ux3c3ux3c4ux3b9ux3baux3adux3c2}}

Στην Μολυσματολογία~ -- που καθιερώθηκε από τον ίδιο τον Robert Koch --
γίνεται μια παραδοσιακή διάκριση μεταξύ μόλυνσης και ασθένειας. Μια
ασθένεια απαιτεί κλινική εκδήλωση. {[}1{]} Ως εκ τούτου, μόνο οι
ασθενείς με συμπτώματα όπως πυρετός ή βήχας πρέπει να περιλαμβάνονται
στα στατιστικά στοιχεία ως νέες περιπτώσεις.

Με άλλα λόγια, μια νέα λοίμωξη -- όπως μετράται από τo test COVID-19 --
δεν σημαίνει απαραίτητα ότι αντιμετωπίζουμε έναν πρόσφατα άρρωστο ασθενή
που χρειάζεται ένα νοσοκομειακό κρεβάτι. Ωστόσο, επί του παρόντος
θεωρείται ότι πέντε τοις εκατό όλων των μολυσμένων ατόμων αρρωσταίνουν
σοβαρά και χρειάζονται διασωλήνωση. Οι προβλέψεις που βασίζονται σε
αυτήν την εκτίμηση υποδηλώνουν ότι το σύστημα υγειονομικής περίθαλψης θα
μπορούσε να είναι υπερφορτωμένο.

\textbf{Η ερώτησή μου}: Οι προβολές έκαναν διάκριση μεταξύ μολυσμένων
ατόμων χωρίς συμπτώματα και πραγματικών ασθενών~ -- δηλαδή ατόμων που
εμφανίζουν συμπτώματα;

\hypertarget{2-ux3b5ux3c0ux3b9ux3baux3b9ux3bdux3b4ux3c5ux3bdux3ccux3c4ux3b7ux3c4ux3b1}{%
\subparagraph{\texorpdfstring{\textbf{2.
Επικινδυνότητα}}{2. Επικινδυνότητα}}\label{2-ux3b5ux3c0ux3b9ux3baux3b9ux3bdux3b4ux3c5ux3bdux3ccux3c4ux3b7ux3c4ux3b1}}

Ένας αριθμός κορωναϊών κυκλοφορεί εδώ και πολύ καιρό -- σε μεγάλο βαθμό
απαρατήρητο από τα μέσα ενημέρωσης. {[}2{]} Εάν αποδειχθεί ότι στον ιό
COVID-19 δεν πρέπει να αποδοθεί σημαντικά υψηλότερο δυναμικό κινδύνου
από τους~ κορωναϊούς, που κυκλοφορούν ήδη, προφανώς όλα τα αντίμετρα θα
καταστούν περιττά.

Η διεθνώς αναγνωρισμένη Διεθνής Εφημερίδα των Αντιμικροβιακών Παραγόντων
θα δημοσιεύσει σύντομα ένα έγγραφο που πραγματεύεται ακριβώς αυτό το
ζήτημα. Τα προκαταρκτικά αποτελέσματα της μελέτης είναι ήδη ορατά σήμερα
και οδηγούν στο συμπέρασμα ότι ο νέος ιός ΔΕΝ διαφέρει από τους
παραδοσιακούς κορωναϊούς~ όσον αφορά την επικινδυνότητα. Οι συγγραφείς
το εκφράζουν στον τίτλο της εργασίας τους «SARS-CoV-2: Fear vs. Data».
{[}3{]}

\textbf{Η ερώτησή μου}: Πώς συγκρίνεται ο τρέχων φόρτος εργασίας των
μονάδων εντατικής θεραπείας με ασθενείς με διαγνωσμένο COVID-19 με άλλες
λοιμώξεις κορωναϊού και σε ποιο βαθμό αυτά τα δεδομένα θα ληφθούν υπόψη
κατά τη λήψη περαιτέρω αποφάσεων από την ομοσπονδιακή κυβέρνηση;
Επιπλέον: έχει ληφθεί υπόψη η παραπάνω μελέτη στον προγραμματισμό μέχρι
τώρα; Και εδώ, φυσικά, η «διάγνωση» σημαίνει ότι ο ιός παίζει
καθοριστικό ρόλο στην κατάσταση της ασθένειας του ασθενούς και όχι ότι
οι προηγούμενες ασθένειες παίζουν μεγαλύτερο ρόλο.

\hypertarget{3-ux3b4ux3b9ux3acux3b4ux3bfux3c3ux3b7}{%
\subparagraph{\texorpdfstring{\textbf{3.
Διάδοση}}{3. Διάδοση}}\label{3-ux3b4ux3b9ux3acux3b4ux3bfux3c3ux3b7}}

Σύμφωνα με μια έκθεση στην Süddeutsche Zeitung, ούτε καν το
πολυαναφερόμενο Robert Koch Institute δεν ξέρει ακριβώς πόσα τεστ
γίνονται για τον COVID-19. Είναι γεγονός, ωστόσο, ότι στη Γερμανία
παρατηρήθηκε πρόσφατα ραγδαία αύξηση του αριθμού των περιπτώσεων καθώς ο
όγκος των τεστ αυξάνεται. {[}4{]}Επομένως, είναι λογικό να υποψιαζόμαστε
ότι ο ιός έχει ήδη εξαπλωθεί απαρατήρητα στον υγιή πληθυσμό. Αυτό θα
είχε δύο συνέπειες: πρώτον, θα σήμαινε ότι το επίσημο ποσοστό θανάτου --
για παράδειγμα, στις 26 Μαρτίου 2020, υπήρχαν 206 θάνατοι από περίπου
37.300 μολύνσεις ή 0,55 τοις εκατό {[}5{]} -- είναι πολύ υψηλό. Και
δεύτερον, θα σήμαινε ότι θα ήταν σχεδόν αδύνατο να αποφευχθεί η εξάπλωση
του ιού στον υγιή πληθυσμό.

\textbf{Η ερώτησή μου}: Υπήρξε ήδη ένα τυχαίο δείγμα του υγιούς γενικού
πληθυσμού για την πιστοποίηση της πραγματικής εξάπλωσης του ιού ή
σχεδιάζεται αυτό στο εγγύς μέλλον;

\hypertarget{4-ux3b8ux3bdux3b7ux3c3ux3b9ux3bcux3ccux3c4ux3b7ux3c4ux3b1}{%
\subparagraph{\texorpdfstring{\textbf{4.
Θνησιμότητα}}{4. Θνησιμότητα}}\label{4-ux3b8ux3bdux3b7ux3c3ux3b9ux3bcux3ccux3c4ux3b7ux3c4ux3b1}}

Ο φόβος για αύξηση του ποσοστού θανάτων στη Γερμανία (επί του παρόντος
0,55\%) αποτελεί επί του παρόντος αντικείμενο ιδιαίτερα έντονης προσοχής
των μέσων ενημέρωσης. Πολλοί άνθρωποι ανησυχούν ότι θα μπορούσε να
εκτοξευτεί όπως στην Ιταλία (10 τοις εκατό) και στην Ισπανία (7 τοις
εκατό), εάν δεν ληφθεί έγκαιρα δράση.

Ταυτόχρονα, γίνεται το λάθος παγκοσμίως~ να αναφέρονται θάνατοι που
σχετίζονται με ιούς μόλις διαπιστωθεί ότι ο ιός υπήρχε τη στιγμή του
θανάτου -- ανεξάρτητα από άλλους παράγοντες. Αυτό παραβιάζει μια βασική
αρχή της μολυσματολογίας: μόνο όταν είναι βέβαιο ότι ένας παράγοντας
έχει διαδραματίσει σημαντικό ρόλο στην ασθένεια ή στον θάνατο μπορεί να
γίνει διάγνωση. Ο Σύνδεσμος Επιστημονικών Ιατρικών Εταιρειών της
Γερμανίας γράφει ρητά στις κατευθυντήριες γραμμές του: «Εκτός από την
αιτία θανάτου, πρέπει να δηλωθεί μια αιτιώδης αλυσίδα, με την αντίστοιχη
υποκείμενη ασθένεια στην τρίτη θέση στο πιστοποιητικό θανάτου.
Περιστασιακά, πρέπει επίσης να δηλώνονται τετρασυνδεδεμένες αλυσίδες
αιτιότητας. ``{[}6{]}

Προς το παρόν δεν υπάρχουν επίσημες πληροφορίες σχετικά με το εάν,
τουλάχιστον εκ των υστέρων, έχουν γίνει πιο κριτικές αναλύσεις ιατρικών
αρχείων για να προσδιοριστεί ο αριθμός των θανάτων που προκλήθηκαν στην
πραγματικότητα από τον ιό.

\textbf{Η ερώτησή μου}: Η Γερμανία απλώς ακολούθησε αυτήν την τάση
γενικής υποψίας για τον COVID-19; Και: σκοπεύει να συνεχίσει αυτήν την
κατηγοριοποίηση με κριτικό τρόπο όπως σε άλλες χώρες; Πώς, λοιπόν,
πρέπει να γίνει διάκριση μεταξύ γνήσιων θανάτων που σχετίζονται με τον
κορώνα και τυχαίας παρουσίας του ιού κατά τη στιγμή του θανάτου;

\hypertarget{5-ux3c3ux3c5ux3b3ux3baux3c1ux3b9ux3c3ux3b9ux3bcux3ccux3c4ux3b7ux3c4ux3b1}{%
\subparagraph{\texorpdfstring{\textbf{5.
Συγκρισιμότητα}}{5. Συγκρισιμότητα}}\label{5-ux3c3ux3c5ux3b3ux3baux3c1ux3b9ux3c3ux3b9ux3bcux3ccux3c4ux3b7ux3c4ux3b1}}

Η τρομακτική κατάσταση στην Ιταλία χρησιμοποιείται επανειλημμένα ως
σενάριο αναφοράς. Ωστόσο, ο πραγματικός ρόλος του ιού σε αυτήν τη χώρα
είναι εντελώς ασαφής για πολλούς λόγους -- όχι μόνο επειδή εδώ ισχύουν
επίσης τα σημεία 3 και 4, αλλά και επειδή υπάρχουν εξαιρετικοί
εξωτερικοί παράγοντες που καθιστούν αυτές τις περιοχές ιδιαίτερα
ευάλωτες.

Ένας από αυτούς τους παράγοντες είναι η αυξημένη ατμοσφαιρική ρύπανση
στο βόρειο τμήμα της Ιταλίας. Σύμφωνα με εκτιμήσεις του ΠΟΥ, αυτή η
κατάσταση, ακόμη και χωρίς τον ιό, οδήγησε σε πάνω από 8.000 επιπλέον
θανάτους το χρόνο το 2006 μόνο στις 13 μεγαλύτερες πόλεις της Ιταλίας.
{[}7{]} Από τότε η κατάσταση δεν έχει αλλάξει σημαντικά. {[}8{]} Τέλος,
αποδείχθηκε επίσης ότι η ατμοσφαιρική ρύπανση αυξάνει σημαντικά τον
κίνδυνο ιογενών πνευμονικών παθήσεων σε πολύ νέους και ηλικιωμένους.
{[}9{]}

Επιπλέον, το 27,4 τοις εκατό του ιδιαίτερα ευάλωτου πληθυσμού σε αυτήν
τη χώρα ζει με νέους, και στην Ισπανία έως και 33,5 τοις εκατό. Στη
Γερμανία, το ποσοστό είναι μόνο επτά τοις εκατό {[}10{]}. Επιπλέον,
σύμφωνα με τον καθηγητή Dr. Reinhard Busse, επικεφαλής του Τμήματος
Διαχείρισης Υγείας στο TU Berlin, η Γερμανία είναι πολύ καλύτερα
εξοπλισμένη από την Ιταλία σε σχέση με τις μονάδες εντατικής θεραπείας
-- με συντελεστή περίπου 2,5 {[}11{]}.

Η ερώτησή μου: Ποιες προσπάθειες καταβάλλονται για να ευαισθητοποιηθεί ο
πληθυσμός για αυτές τις στοιχειώδεις διαφορές και για να μπορέσουν οι
ανθρώποι να καταλάβουν ότι σενάρια όπως αυτά στην Ιταλία ή την Ισπανία
δεν ανταποκρίνονται στην εδώ πραγματικότητα;

\hypertarget{references}{%
\subparagraph{\texorpdfstring{\textbf{References:}}{References:}}\label{references}}

{[}1{]} Fachwörterbuch Infektionsschutz und Infektionsepidemiologie.
\href{https://www.rki.de/DE/Content/Service/Publikationen/Fachwoerterbuch_Infektionsschutz.html}{Fachwörter
-- Definitionen -- Interpretationen}. Robert Koch-Institut, Berlin 2015.
(abgerufen am 26.3.2020)

{[}2{]} Killerby et al., Human Coronavirus Circulation in the United
States 2014--2017. J Clin Virol. 2018, 101, 52-56

{[}3{]} Roussel et al. SARS-CoV-2: Fear Versus Data. Int. J. Antimicrob.
Agents 2020, 105947

{[}4{]} Charisius, H.
\href{https://www.sueddeutsche.de/gesundheit/covid-19-coronavirus-testverfahren-1.4855487}{Covid-19:
Wie gut testet Deutschland?} Süddeutsche Zeitung. (abgerufen am
27.3.2020)

{[}5{]} Johns Hopkins University,
\href{https://coronavirus.jhu.edu/map.html}{Coronavirus Resource
Center}. 2020. (abgerufen am 26.3.2020)

{[}6{]} S1-Leitlinie 054-001,
\href{https://www.awmf.org/uploads/tx_szleitlinien/054-002l_S1_Regeln-zur-Durchfuehrung-der-aerztlichen-Leichenschau_2018-02_01.pdf}{Regeln
zur Durchführung der ärztlichen Leichenschau}. AWMF Online (abgerufen am
26.3.2020)

{[}7{]} Martuzzi et al. Health Impact of PM10 and Ozone in 13 Italian
Cities. World Health Organization Regional Office for Europe. WHOLIS
number E88700 2006

{[}8{]} European Environment Agency,
\href{https://www.eea.europa.eu/themes/air/country-fact-sheets/2019-country-fact-sheets}{Air
Pollution Country Fact Sheets 2019}, (abgerufen am 26.3.2020)

{[}9{]} Croft et al. The Association between Respiratory Infection and
Air Pollution in the Setting of Air Quality Policy and Economic Change.
Ann. Am. Thorac. Soc. 2019, 16, 321--330.

{[}10{]} United Nations, Department of Economic and Social Affairs,
Population Division. Living Arrange­ments of Older Persons: A Report on
an Expanded International Dataset (ST/ESA/SER.A/407). 2017

{[}11{]} Deutsches Ärzteblatt,
\href{https://www.aerzteblatt.de/nachrichten/111029/Ueberlastung-deutscher-Krankenhaeuser-durch-COVID-19-laut-Experten-unwahrscheinlich}{Überlastung
deutscher Krankenhäuser durch COVID-19 laut Experten unwahrscheinlich},
(abgerufen am 26.3.2020)

\begin{center}\rule{0.5\linewidth}{\linethickness}\end{center}

Share this letter on:
\href{https://twitter.com/intent/tweet?url=https://swprs.org/professor-bhakdi-open-letter-greek/}{Twitter}
/
\href{https://www.facebook.com/share.php?u=https://swprs.org/professor-bhakdi-open-letter-greek/}{Facebook}\\
Back to main article:
\href{https://swprs.org/a-swiss-doctor-on-covid-19/}{Facts about
Covid-19}

\hypertarget{swiss-policy-research}{%
\subsubsection{Swiss Policy Research}\label{swiss-policy-research}}

\begin{itemize}
\tightlist
\item
  \href{https://swprs.org/kontakt/}{Kontakt}
\item
  \href{https://swprs.org/uebersicht/}{Übersicht}
\item
  \href{https://swprs.org/donationen/}{Donationen}
\item
  \href{https://swprs.org/disclaimer/}{Disclaimer}
\end{itemize}

\hypertarget{english}{%
\subsubsection{English}\label{english}}

\begin{itemize}
\tightlist
\item
  \href{https://swprs.org/contact/}{About Us / Contact}
\item
  \href{https://swprs.org/media-navigator/}{The Media Navigator}
\item
  \href{https://swprs.org/the-american-empire-and-its-media/}{The CFR
  and the Media}
\item
  \href{https://swprs.org/donations/}{Donations}
\end{itemize}

\hypertarget{follow-by-email}{%
\subsubsection{Follow by email}\label{follow-by-email}}

Follow

\href{https://wordpress.com/?ref=footer_custom_com}{WordPress.com}.

\protect\hyperlink{}{Up ↑}

Post to

\protect\hyperlink{}{Cancel}

\includegraphics{https://pixel.wp.com/b.gif?v=noscript}
