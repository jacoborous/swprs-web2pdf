\protect\hyperlink{content}{Skip to content}

\href{https://swprs.org/}{}

\protect\hyperlink{search-container}{Search}

Search for:

\href{https://swprs.org/}{\includegraphics{https://swprs.files.wordpress.com/2020/05/swiss-policy-research-logo-300.png}}

\href{https://swprs.org/}{Swiss Policy Research}

Geopolitics and Media

Menu

\begin{itemize}
\tightlist
\item
  \href{https://swprs.org}{Start}
\item
  \href{https://swprs.org/srf-propaganda-analyse/}{Studien}

  \begin{itemize}
  \tightlist
  \item
    \href{https://swprs.org/srf-propaganda-analyse/}{SRF / ZDF}
  \item
    \href{https://swprs.org/die-nzz-studie/}{NZZ-Studie}
  \item
    \href{https://swprs.org/der-propaganda-multiplikator/}{Agenturen}
  \item
    \href{https://swprs.org/die-propaganda-matrix/}{Medienmatrix}
  \end{itemize}
\item
  \href{https://swprs.org/medien-navigator/}{Analysen}

  \begin{itemize}
  \tightlist
  \item
    \href{https://swprs.org/medien-navigator/}{Navigator}
  \item
    \href{https://swprs.org/der-propaganda-schluessel/}{Techniken}
  \item
    \href{https://swprs.org/propaganda-in-der-wikipedia/}{Wikipedia}
  \item
    \href{https://swprs.org/logik-imperialer-kriege/}{Kriege}
  \end{itemize}
\item
  \href{https://swprs.org/netzwerk-medien-schweiz/}{Netzwerke}

  \begin{itemize}
  \tightlist
  \item
    \href{https://swprs.org/netzwerk-medien-schweiz/}{Schweiz}
  \item
    \href{https://swprs.org/netzwerk-medien-deutschland/}{Deutschland}
  \item
    \href{https://swprs.org/medien-in-oesterreich/}{Österreich}
  \item
    \href{https://swprs.org/das-american-empire-und-seine-medien/}{USA}
  \end{itemize}
\item
  \href{https://swprs.org/bericht-eines-journalisten/}{Fokus I}

  \begin{itemize}
  \tightlist
  \item
    \href{https://swprs.org/bericht-eines-journalisten/}{Journalistenbericht}
  \item
    \href{https://swprs.org/russische-propaganda/}{Russische Propaganda}
  \item
    \href{https://swprs.org/die-israel-lobby-fakten-und-mythen/}{Die
    »Israel-Lobby«}
  \item
    \href{https://swprs.org/geopolitik-und-paedokriminalitaet/}{Pädokriminalität}
  \end{itemize}
\item
  \href{https://swprs.org/migration-und-medien/}{Fokus II}

  \begin{itemize}
  \tightlist
  \item
    \href{https://swprs.org/covid-19-hinweis-ii/}{Coronavirus}
  \item
    \href{https://swprs.org/die-integrity-initiative/}{Integrity
    Initiative}
  \item
    \href{https://swprs.org/migration-und-medien/}{Migration \& Medien}
  \item
    \href{https://swprs.org/der-fall-magnitsky/}{Magnitsky Act}
  \end{itemize}
\item
  \href{https://swprs.org/kontakt/}{Projekt}

  \begin{itemize}
  \tightlist
  \item
    \href{https://swprs.org/kontakt/}{Kontakt}
  \item
    \href{https://swprs.org/uebersicht/}{Seitenübersicht}
  \item
    \href{https://swprs.org/medienspiegel/}{Medienspiegel}
  \item
    \href{https://swprs.org/donationen/}{Donationen}
  \end{itemize}
\item
  \href{https://swprs.org/contact/}{English}
\end{itemize}

\protect\hyperlink{}{Open Search}

\hypertarget{open-letter-from-professor-sucharit-bhakdi-to-german-chancellor-dr-angela-merkel}{%
\section{Open Letter from Professor Sucharit Bhakdi to German Chancellor
Dr.
Angela~Merkel}\label{open-letter-from-professor-sucharit-bhakdi-to-german-chancellor-dr-angela-merkel}}

\includegraphics{https://swprs.files.wordpress.com/2020/03/bakhdi-letter-header.png?w=736\&h=297}

\textbf{Languages}:
\href{https://swprs.org/offener-brief-von-professor-sucharit-bhakdi-an-bundeskanzlerin-dr-angela-merkel/}{DE},
\href{https://swprs.org/open-letter-from-professor-sucharit-bhakdi-to-german-chancellor-dr-angela-merkel/}{EN};
\href{https://swprs.org/professor-sucharit-bhakdi-avalik-kiri-angela-merkelile/}{EE},
\href{http://piensachile.com/2020/03/carta-abierta-a-angela-merkel/}{ES},
\href{https://swprs.org/covid-19-lettre-ouverte-du-professeur-sucharit-bhakdi-a-la-chanceliere-allemande-dre-angela-merkel/}{FR},
\href{https://swprs.org/professor-bhakdi-open-letter-greek/}{GR},
\href{https://yanivhamo.com/open-letter-from-professor-sucharit-bhakdi-to-german-chancellor-dr-angela-merkel-hebrew/}{HE},
\href{https://swprs.org/lettera-aperta-del-professor-sucharit-bhakdi-al-cancelliere-tedesco-dr-angela-merkel/}{IT},
\href{https://swprs.org/open-brief-van-professor-sucharit-bhakdi-aan-de-duitse-bondskanselier-dr-angela-merkel/}{NL},
\href{https://swprs.org/carta-aberta-do-professor-sucharit-bhakdi-a-chanceler-alema-dra-angela-merkel/}{PT},
\href{https://swprs.org/\%d0\%be\%d1\%82\%d0\%ba\%d1\%80\%d1\%8b\%d1\%82\%d0\%be\%d0\%b5-\%d0\%bf\%d0\%b8\%d1\%81\%d1\%8c\%d0\%bc\%d0\%be-\%d0\%bf\%d1\%80\%d0\%be\%d1\%84\%d0\%b5\%d1\%81\%d1\%81\%d0\%be\%d1\%80\%d0\%b0-\%d1\%81\%d1\%83\%d1\%87\%d0\%b0\%d1\%80\%d0\%b8\%d1\%82\%d0\%b0/}{RU},
\href{https://alatyr.sk/open-letter-from-professor_sk.htm}{SK},
\href{https://swprs.org/prof-dr-sucharit-bhakdiden-basbakan-dr-angela-merkele-acik-mektup/}{TR}

An Open Letter from Dr. Sucharit Bhakdi, Professor Emeritus of Medical
Microbiology at the Johannes Gutenberg University Mainz, to the German
Chancellor Dr. Angela Merkel. Professor Bhakdi calls for an urgent
reassessment of the response to Covid-19 and asks the Chancellor five
crucial questions. The let­ter is dated March 26. This is an inofficial
translation; see the
\href{https://swprs.org/offener-brief-von-professor-sucharit-bhakdi-an-bundeskanzlerin-dr-angela-merkel/}{original
letter in German as a PDF}.

\hypertarget{open-letter}{%
\subsubsection{Open Letter}\label{open-letter}}

Dear Chancellor,

As Emeritus of the Johannes-Gutenberg-University in Mainz and longtime
director of the Institute for Medical Microbiology, I feel obliged to
critically question the far-reaching restrictions on public life that we
are currently taking on ourselves in order to reduce the spread of the
COVID-19 virus.

It is expressly not my intention to play down the dangers of the virus
or to spread a political message. However, I feel it is my duty to make
a scientific contribution to putting the current data and facts into
perspective -- and, in addition, to ask questions that are in danger of
being lost in the heated debate.

The reason for my concern lies above all in the truly unforeseeable
socio-economic consequences of the drastic containment measures which
are currently being applied in large parts of Europe and which are also
already being practiced on a large scale in Germany.

My wish is to discuss critically -- and with the necessary foresight --
the advantages and disadvantages of restricting public life and the
resulting long-term effects.

To this end, I am confronted with five questions which have not been
answered sufficiently so far, but which are indispensable for a balanced
analysis.

I would like to ask you to comment quickly and, at the same time, appeal
to the Federal Government to develop strategies that effectively protect
risk groups without restricting public life across the board and sow the
seeds for an even more intensive polarization of society than is already
taking place.

With the utmost respect,

\textbf{Prof. em. Dr. med. Sucharit Bhakdi}

\hypertarget{1-statistics}{%
\subparagraph{\texorpdfstring{\textbf{1.
Statistics}}{1. Statistics}}\label{1-statistics}}

In infectiology -- founded by Robert Koch himself -- a traditional
distinction is made between infection and disease. An illness requires a
clinical manifestation. {[}1{]} Therefore, only patients with symptoms
such as fever or cough should be included in the statistics as new
cases.

In other words, a new infection -- as measured by the COVID-19 test --
does not necessarily mean that we are dealing with a newly ill patient
who needs a hospital bed. However, it is currently assumed that five
percent of all infected people become seriously ill and require
ventilation. Projections based on this estimate suggest that the
healthcare system could be overburdened.

\textbf{My question}: Did the projections make a distinction between
symptom-free infected people and actual, sick patients -- i.e. people
who develop symptoms?

\hypertarget{2-dangerousness}{%
\subparagraph{\texorpdfstring{\textbf{2.
Dangerousness}}{2. Dangerousness}}\label{2-dangerousness}}

A number of coronaviruses have been circulating for a long time --
largely unnoticed by the media. {[}2{]} If it should turn out that the
COVID-19 virus should not be ascribed a significantly higher risk
potential than the already circulating corona viruses, all
countermeasures would obviously become unnecessary.

The internationally recognized \emph{International Journal of
Antimicrobial Agents} will soon publish a paper that addresses exactly
this question. Preliminary results of the study can already be seen
today and lead to the conclusion that the new virus is NOT different
from traditional corona viruses in terms of dangerousness. The authors
express this in the title of their paper ``SARS-CoV-2: Fear versus
Data''. {[}3{]}

\textbf{My question}: How does the current workload of intensive care
units with patients with diagnosed COVID-19 compare to other coronavirus
infections, and to what extent will this data be taken into account in
further decision-making by the federal government? In addition: Has the
above study been taken into account in the planning so far?~ Here too,
of course, ``diagnosed'' means that the virus plays a decisive role in
the patient's state of illness, and not that previous illnesses play a
greater role.

\hypertarget{3-dissemination}{%
\subparagraph{\texorpdfstring{\textbf{3.
Dissemination}}{3. Dissemination}}\label{3-dissemination}}

According to a report in the Süddeutsche Zeitung, not even the
much-cited Robert Koch Institute knows exactly how much is tested for
COVID-19. It is a fact, however, that a rapid increase in the number of
cases has recently been observed in Germany as the volume of tests
increases. {[}4{]}

It is therefore reasonable to suspect that the virus has already spread
unnoticed in the healthy population. This would have two consequences:
firstly, it would mean that the official death rate -- on 26 March 2020,
for example, there were 206 deaths from around 37,300 infections, or
0.55 percent {[}5{]} -- is too high; and secondly, it would mean that it
would hardly be possible to prevent the virus from spreading in the
healthy population.

\textbf{My question}: Has there already been a random sample of the
healthy general population to validate the real spread of the virus, or
is this planned in the near future?

\hypertarget{4-mortality}{%
\subparagraph{\texorpdfstring{\textbf{4.
Mortality}}{4. Mortality}}\label{4-mortality}}

The fear of a rise in the death rate in Germany (currently 0.55 percent)
is currently the subject of particularly intense media attention. Many
people are worried that it could shoot up like in Italy (10 percent) and
Spain (7 percent) if action is not taken in time.

At the same time, the mistake is being made worldwide to report
virus-related deaths as soon as it is established that the virus was
present at the time of death -- regardless of other factors. This
violates a basic principle of infectiology: only when it is certain that
an agent has played a significant role in the disease or death may a
diagnosis be made. The \emph{Association of the Scientific Medical
Societies of Germany} expressly writes in its guidelines: ``In addition
to the cause of death, a causal chain must be stated, with the
corresponding underlying disease in third place on the death
certificate. Occasionally, four-linked causal chains must also be
stated.'' {[}6{]}

At present there is no official information on whether, at least in
retrospect, more critical analyses of medical records have been
undertaken to determine how many deaths were actually caused by the
virus.

\textbf{My question}: Has Germany simply followed this trend of a
COVID-19 general suspicion? And: is it intended to continue this
categorisation uncritically as in other countries? How, then, is a
distinction to be made between genuine corona-related deaths and
accidental virus presence at the time of death?

\hypertarget{5-comparability}{%
\subparagraph{\texorpdfstring{\textbf{5.
Comparability}}{5. Comparability}}\label{5-comparability}}

The appalling situation in Italy is repeatedly used as a reference
scenario. However, the true role of the virus in that country is
completely unclear for many reasons -- not only because points 3 and 4
above also apply here, but also because exceptional external factors
exist which make these regions particularly vulnerable.

One of these factors is the increased air pollution in the north of
Italy. According to WHO estimates, this situation, even without the
virus, led to over 8,000 additional deaths per year in 2006 in the 13
largest cities in Italy alone. {[}7{]} The situation has not changed
significantly since then. {[}8{]} Finally, it has also been shown that
air pollution greatly increases the risk of viral lung diseases in very
young and elderly people. {[}9{]}

Moreover, 27.4 percent of the particularly vulnerable population in this
country live with young people, and in Spain as many as 33.5 percent. In
Germany, the figure is only seven percent {[}10{]}. In addition,
according to Prof. Dr. Reinhard Busse, head of the Department of
Management in Health Care at the TU Berlin, Germany is significantly
better equipped than Italy in terms of intensive care units -- by a
factor of about 2.5 {[}11{]}.

\textbf{My question}: What efforts are being made to make the population
aware of these elementary differences and to make people understand that
scenarios like those in Italy or Spain are not realistic here?

\hypertarget{references}{%
\subparagraph{\texorpdfstring{\textbf{References:}}{References:}}\label{references}}

{[}1{]} Fachwörterbuch Infektionsschutz und Infektionsepidemiologie.
\href{https://www.rki.de/DE/Content/Service/Publikationen/Fachwoerterbuch_Infektionsschutz.html}{Fachwörter
-- Definitionen -- Interpretationen}. Robert Koch-Institut, Berlin 2015.
(abgerufen am 26.3.2020)

{[}2{]} Killerby et al., Human Coronavirus Circulation in the United
States 2014--2017. J Clin Virol. 2018, 101, 52-56

{[}3{]} Roussel et al. SARS-CoV-2: Fear Versus Data. Int. J. Antimicrob.
Agents 2020, 105947

{[}4{]} Charisius, H.
\href{https://www.sueddeutsche.de/gesundheit/covid-19-coronavirus-testverfahren-1.4855487}{Covid-19:
Wie gut testet Deutschland?} Süddeutsche Zeitung. (abgerufen am
27.3.2020)

{[}5{]} Johns Hopkins University,
\href{https://coronavirus.jhu.edu/map.html}{Coronavirus Resource
Center}. 2020. (abgerufen am 26.3.2020)

{[}6{]} S1-Leitlinie 054-001,
\href{https://www.awmf.org/uploads/tx_szleitlinien/054-002l_S1_Regeln-zur-Durchfuehrung-der-aerztlichen-Leichenschau_2018-02_01.pdf}{Regeln
zur Durchführung der ärztlichen Leichenschau}. AWMF Online (abgerufen am
26.3.2020)

{[}7{]} Martuzzi et al. Health Impact of PM10 and Ozone in 13 Italian
Cities. World Health Organization Regional Office for Europe. WHOLIS
number E88700 2006

{[}8{]} European Environment Agency,
\href{https://www.eea.europa.eu/themes/air/country-fact-sheets/2019-country-fact-sheets}{Air
Pollution Country Fact Sheets 2019}, (abgerufen am 26.3.2020)

{[}9{]} Croft et al. The Association between Respiratory Infection and
Air Pollution in the Setting of Air Quality Policy and Economic Change.
Ann. Am. Thorac. Soc. 2019, 16, 321--330.

{[}10{]} United Nations, Department of Economic and Social Affairs,
Population Division. Living Arrange­ments of Older Persons: A Report on
an Expanded International Dataset (ST/ESA/SER.A/407). 2017

{[}11{]} Deutsches Ärzteblatt,
\href{https://www.aerzteblatt.de/nachrichten/111029/Ueberlastung-deutscher-Krankenhaeuser-durch-COVID-19-laut-Experten-unwahrscheinlich}{Überlastung
deutscher Krankenhäuser durch COVID-19 laut Experten unwahrscheinlich},
(abgerufen am 26.3.2020)

\begin{center}\rule{0.5\linewidth}{\linethickness}\end{center}

Share this letter on:
\href{https://twitter.com/intent/tweet?url=https://swprs.org/open-letter-from-professor-sucharit-bhakdi-to-german-chancellor-dr-angela-merkel/}{Twitter}
/
\href{https://www.facebook.com/share.php?u=https://swprs.org/open-letter-from-professor-sucharit-bhakdi-to-german-chancellor-dr-angela-merkel/}{Facebook}\\
Back to main article:
\href{https://swprs.org/a-swiss-doctor-on-covid-19/}{Facts about
Covid-19}

\hypertarget{swiss-policy-research}{%
\subsubsection{Swiss Policy Research}\label{swiss-policy-research}}

\begin{itemize}
\tightlist
\item
  \href{https://swprs.org/kontakt/}{Kontakt}
\item
  \href{https://swprs.org/uebersicht/}{Übersicht}
\item
  \href{https://swprs.org/donationen/}{Donationen}
\item
  \href{https://swprs.org/disclaimer/}{Disclaimer}
\end{itemize}

\hypertarget{english}{%
\subsubsection{English}\label{english}}

\begin{itemize}
\tightlist
\item
  \href{https://swprs.org/contact/}{About Us / Contact}
\item
  \href{https://swprs.org/media-navigator/}{The Media Navigator}
\item
  \href{https://swprs.org/the-american-empire-and-its-media/}{The CFR
  and the Media}
\item
  \href{https://swprs.org/donations/}{Donations}
\end{itemize}

\hypertarget{follow-by-email}{%
\subsubsection{Follow by email}\label{follow-by-email}}

Follow

\href{https://wordpress.com/?ref=footer_custom_com}{WordPress.com}.

\protect\hyperlink{}{Up ↑}

Post to

\protect\hyperlink{}{Cancel}

\includegraphics{https://pixel.wp.com/b.gif?v=noscript}
