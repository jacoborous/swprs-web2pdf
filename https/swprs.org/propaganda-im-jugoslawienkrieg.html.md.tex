\protect\hyperlink{content}{Skip to content}

\href{https://swprs.org/}{}

\protect\hyperlink{search-container}{Search}

Search for:

\href{https://swprs.org/}{\includegraphics{https://swprs.files.wordpress.com/2020/05/swiss-policy-research-logo-300.png}}

\href{https://swprs.org/}{Swiss Policy Research}

Geopolitics and Media

Menu

\begin{itemize}
\tightlist
\item
  \href{https://swprs.org}{Start}
\item
  \href{https://swprs.org/srf-propaganda-analyse/}{Studien}

  \begin{itemize}
  \tightlist
  \item
    \href{https://swprs.org/srf-propaganda-analyse/}{SRF / ZDF}
  \item
    \href{https://swprs.org/die-nzz-studie/}{NZZ-Studie}
  \item
    \href{https://swprs.org/der-propaganda-multiplikator/}{Agenturen}
  \item
    \href{https://swprs.org/die-propaganda-matrix/}{Medienmatrix}
  \end{itemize}
\item
  \href{https://swprs.org/medien-navigator/}{Analysen}

  \begin{itemize}
  \tightlist
  \item
    \href{https://swprs.org/medien-navigator/}{Navigator}
  \item
    \href{https://swprs.org/der-propaganda-schluessel/}{Techniken}
  \item
    \href{https://swprs.org/propaganda-in-der-wikipedia/}{Wikipedia}
  \item
    \href{https://swprs.org/logik-imperialer-kriege/}{Kriege}
  \end{itemize}
\item
  \href{https://swprs.org/netzwerk-medien-schweiz/}{Netzwerke}

  \begin{itemize}
  \tightlist
  \item
    \href{https://swprs.org/netzwerk-medien-schweiz/}{Schweiz}
  \item
    \href{https://swprs.org/netzwerk-medien-deutschland/}{Deutschland}
  \item
    \href{https://swprs.org/medien-in-oesterreich/}{Österreich}
  \item
    \href{https://swprs.org/das-american-empire-und-seine-medien/}{USA}
  \end{itemize}
\item
  \href{https://swprs.org/bericht-eines-journalisten/}{Fokus I}

  \begin{itemize}
  \tightlist
  \item
    \href{https://swprs.org/bericht-eines-journalisten/}{Journalistenbericht}
  \item
    \href{https://swprs.org/russische-propaganda/}{Russische Propaganda}
  \item
    \href{https://swprs.org/die-israel-lobby-fakten-und-mythen/}{Die
    »Israel-Lobby«}
  \item
    \href{https://swprs.org/geopolitik-und-paedokriminalitaet/}{Pädokriminalität}
  \end{itemize}
\item
  \href{https://swprs.org/migration-und-medien/}{Fokus II}

  \begin{itemize}
  \tightlist
  \item
    \href{https://swprs.org/covid-19-hinweis-ii/}{Coronavirus}
  \item
    \href{https://swprs.org/die-integrity-initiative/}{Integrity
    Initiative}
  \item
    \href{https://swprs.org/migration-und-medien/}{Migration \& Medien}
  \item
    \href{https://swprs.org/der-fall-magnitsky/}{Magnitsky Act}
  \end{itemize}
\item
  \href{https://swprs.org/kontakt/}{Projekt}

  \begin{itemize}
  \tightlist
  \item
    \href{https://swprs.org/kontakt/}{Kontakt}
  \item
    \href{https://swprs.org/uebersicht/}{Seitenübersicht}
  \item
    \href{https://swprs.org/medienspiegel/}{Medienspiegel}
  \item
    \href{https://swprs.org/donationen/}{Donationen}
  \end{itemize}
\item
  \href{https://swprs.org/contact/}{English}
\end{itemize}

\protect\hyperlink{}{Open Search}

\hypertarget{propaganda-im-jugoslawienkrieg}{%
\section{Propaganda im
Jugoslawienkrieg}\label{propaganda-im-jugoslawienkrieg}}

\textbf{Publiziert}: Dezember 2019 (akt.)\\
\textbf{Sprachen}:
\href{https://swprs.org/propaganda-in-the-war-on-yugoslavia/}{EN},
\href{https://swprs.org/propaganda-im-jugoslawienkrieg/}{DE};
\href{https://mondo.ba/Info/Region/a923284/Trnopolje-Markale-Srebrenica-Racak-Podvale-oo-kojima-bruji-svijet.html}{BS},
\href{https://prima.iprima.cz/zpravodajstvi/svycari-o-valce-v-byvale-jugoslavii-dezinformacich-a-fake-news-medii-boju-se-ucastnila}{CZ},
\href{http://www.politika.rs/sr/clanak/444725/Na-Zapadu-se-urusava-mit-o-zlim-Srbima}{SR}

Beim Jugoslawienkrieg der 1990er Jahre ging es aus geopolitischer Sicht
um eine Neuordnung Süd­ost­europas nach dem Ende des Kalten Krieges. Die
USA nutzten dazu auch jene
\href{https://www.theguardian.com/world/2002/apr/22/warcrimes.comment}{Milizen},
mit denen sie zuvor in Afghanistan die UdSSR bekämpften, und die sie
später »Al Kaida« nennen sollten.

Die politische und mediale Propaganda zum Jugoslawien­krieg ist
inzwischen gut erforscht. Interes­santer­weise versuchen dennoch
zahlreiche Medien und Kommentatoren bis heute die offizielle Darstellung
von damals zu verteidigen, im Unterschied etwa zum Irak­krieg.

Hierfür mag es verschiedene Gründe geben. Einerseits stammt die
fragliche Propaganda noch aus der Frühzeit des Internets und ist deshalb
in der Öffentlichkeit im Allgemeinen etwas weniger bekannt. Andererseits
sind die Implikationen für Europa in diesem Fall besonders groß.

Aus heutiger Sicht ist es eine triviale Feststellung, dass die meisten
westlichen Medien den Jugo­sla­wien­krieg der NATO unterstützten, doch
damals glaubten selbst Kritiker noch an ein mediales »Versagen«, zumal
die entscheidenden
\href{https://swprs.org/netzwerk-medien-deutschland/}{Medienstrukturen}
noch nicht allgemein bekannt waren.

Es folgt eine Übersicht der bekanntesten Propagandabeispiele aus dem
Jugoslawienkrieg sowie Hinweise auf weiterführende Literatur und
Dokumentation. Die Analyse stellt dabei weder regionale Aspekte des
Krieges noch tatsächliche Kriegsverbrechen, egal auf welcher Seite,
infrage.

\textbf{Siehe auch}:

\begin{itemize}
\tightlist
\item
  \href{https://www.heise.de/tp/features/Der-Staat-Jugoslawien-stand-dem-strategischen-Entwurf-der-USA-im-Wege-3378506.html}{»Jugoslawien
  stand dem strategischen Entwurf der USA im Wege«} (Telepolis, 2016)
\item
  \href{https://www.heise.de/tp/features/800-Millionen-Mark-fuer-einen-Buergerkrieg-3430677.html}{Jugoslawien:
  800 Millionen Mark für einen Bürgerkrieg} (Interview, Telepolis, 2003)
\item
  \href{https://swprs.org/logik-imperialer-kriege/}{Die Logik imperialer
  Kriege} und \href{https://swprs.org/das-gewuenschte-narrativ/}{Das
  gewünschte Narrativ} (SPR, 2016/2018)
\end{itemize}

\hypertarget{propagandabeispiele}{%
\subsubsection{Propagandabeispiele}\label{propagandabeispiele}}

\hypertarget{1-das-serbische-todeslager-1992}{%
\paragraph{1. Das serbische »Todeslager«
(1992)}\label{1-das-serbische-todeslager-1992}}

Eines der bekanntesten Propagandabeispiele aus dem Jugoslawienkrieg
betrifft das angebliche Todeslager von
\href{https://en.wikipedia.org/wiki/Trnopolje_camp}{Trnopolje} in
Bosnien. Dabei besuchten drei britische Journalisten im August 1992 ein
Flüchtlingslager, dessen Insassen betonten, sehr gut behandelt zu werden
(siehe Video unten).

Die Journalisten begaben sich indes auf ein abgesperrtes Trafo-Areal
direkt neben dem Flüchtlings­lager und filmten die Männer durch einen
Stachel­draht­zaun hindurch, was den Eindruck erweckte, die Männer seien
eingesperrt. Sodann baten die Journalisten einen aufgrund von Krankheit
oder kriegsbedingter Mangelernährung abgemagerten Mann, sein T-Shirt
auszuziehen.

Das so entstandene Foto landete -- sorgfältig zurecht­geschnitten -- auf
den Titelseiten der meisten westlichen Medien als »Beweis« für serbische
»Todeslager«, die wiederum als Begründung für die nachfolgende
\href{https://en.wikipedia.org/wiki/NATO_intervention_in_Bosnia_and_Herzegovina}{NATO-Intervention}
in Bosnien dienten, beginnend mit einer Flugverbotszone.

Die Trnopolje-Täuschung wurde 1997 von einem deutschen Journalisten
\href{https://www.novo-argumente.com/artikel/es_war_dieses_bild_das_die_welt_in_alarmbereitschaft_versetzte_penny_marsha}{aufgedeckt}.
Ein Magazin in England, das seinen Artikel veröffentlichte, wurde von
den drei britischen Journalisten wegen Verleumdung verklagt und
verurteilt, da es ihnen keine Absicht nachweisen konnte.

Der Chef einer amerikanischen PR-Agentur, die die Falschmeldung der
Todeslager aktiv verbreitete, erklärte in einem späteren
\href{https://www.sourcewatch.org/index.php/James_Harff}{Interview}:
»Wir sind Professionals. Wir hatten einen Auftrag und wir erledigten
ihn. Wir sind nicht dafür bezahlt, moralisch zu sein.«

\textbf{Siehe auch:} Der
\href{https://www.novo-argumente.com/artikel/es_war_dieses_bild_das_die_welt_in_alarmbereitschaft_versetzte_penny_marsha}{Originalartikel
zur Trnopolje-Täuschung} von Th. Deichmann (Novo, 1997)

Vollständige Doku:
\href{https://www.youtube.com/watch?v=xtQ-PJLIpcE}{Yugoslavia: The
Pictures that Fooled the World} (2000)

\href{https://swprs.files.wordpress.com/2019/12/trnopolje_tv_screenshot.jpg}{}

\includegraphics{https://swprs.files.wordpress.com/2019/12/trnopolje_tv_screenshot.jpg?w=241\&h=241\&crop=1}

TV screenshot

\href{https://swprs.files.wordpress.com/2019/12/trnopolje_collage.jpg}{}

\includegraphics{https://swprs.files.wordpress.com/2019/12/trnopolje_collage.jpg?w=241\&h=241\&crop=1}

Front pages about Trnopolje

\href{https://swprs.files.wordpress.com/2019/12/trnopolje_lageplan.jpg}{}

\includegraphics{https://swprs.files.wordpress.com/2019/12/trnopolje_lageplan.jpg?w=241\&h=241\&crop=1}

Lageplan des Lagers. Der abgesperrte Bereich ist unten im Bild.

TV, Presse und Lageplan zum Lager bei Trnopolje

\hypertarget{2-die-marktplatz-massaker-in-sarajewo-1992-1995}{%
\paragraph{2. Die Marktplatz-Massaker in Sarajewo
(1992-1995)}\label{2-die-marktplatz-massaker-in-sarajewo-1992-1995}}

Ein weiteres bekanntes Propaganda­beispiel betrifft die sogenannten
Marktplatz-Massaker während der vierjährigen
\href{https://en.wikipedia.org/wiki/Siege_of_Sarajevo}{Belagerung}
Sarajewos, darunter insbesondere das Bäckerei-Massaker vom Mai 1992
sowie die beiden sogenannten
\href{https://en.wikipedia.org/wiki/Markale_massacres}{Markale-Massaker}
vom Februar 1994 und August 1995.

Diese Vorfälle sollen durch Granatenbeschuss von außerhalb der Stadt
erfolgt sein und fanden zumeist kurz vor wichtigen politischen
Beratungen der UNO oder EU statt. Sie führten letztlich zu einem
direkten militärischen
\href{https://en.wikipedia.org/wiki/Operation_Deliberate_Force}{Eingreifen}
der NATO und damit zur Wende im Bosnienkrieg.

In den genannten sowie einigen weiteren Fällen kamen Untersuchungen
durch Offiziere der UNO-Schutz­mission zum
\href{https://swprs.files.wordpress.com/2019/12/anatomy-of-a-massacre_david-binder_foreign-policy_1994.pdf}{Ergebnis},
dass diese Vorfälle womöglich von der bosnischen Seite selbst verübt
wurden, um die öffentliche Meinung im Westen zu beeinflussen (sog.
False-Flag-Angriff).

Die entsprechenden UNO-Berichte wurden jedoch
\href{https://swprs.files.wordpress.com/2019/12/dpa_un-report-sarajevo_1996.pdf}{geheim
gehalten}. Stattdessen behaupteten amerikanische Medien -- insbesondere
CNN -- sowie die US-Regierung meist unmittelbar nach den Vorfällen, dass
der jeweilige Angriff vermutlich von serbischer Seite erfolgt sei (vgl.
Video unten).

Der kanadische General Lewis MacKenzie, Kommandeur der UNO-Truppen in
Sarajewo,
\href{https://archive.org/details/peacekeeperroadt0000mack/page/194}{notierte}
zum Vorfall von 1992: »Laut unseren Leuten passte einiges nicht. Die
Straße wurde kurz zuvor abgesperrt. Dann positionierten sich die
Personen und die Medien tauchten auf, hielten aber noch Abstand. Der
Angriff erfolgte und die Medien begannen sofort zu filmen.«

Zum Vorfall von 1994
\href{http://news.bbc.co.uk/2/hi/europe/3459965.stm}{erinnerte} sich ein
BBC-Journalist, dass »TV-Crews bereits Sekunden nach der Explosion vor
Ort filmten«, während UNO-Helfern und selbst Ärzten der Zutritt
\href{https://swprs.files.wordpress.com/2019/12/dpa_un-report-sarajevo_1996.pdf}{verweigert}
wurde und die offiziell 197 Opfer innerhalb von 25 Minuten
abtransportiert waren. Beobachtern fiel zudem auf, dass der Markt zur
Zeit des Vorfalls geschlossen war (siehe Video unten).

Zum Vorfall von 1995
\href{https://swprs.files.wordpress.com/2019/12/sunday-times_serbs-not-guilty-of-massacre_1995.pdf}{berichtete}
die Londoner Sunday Times, dass britische und französische
Munitions­experten die Serben für »unschuldig« hielten, jedoch »von
einem hohen US-Offizier überstimmt« wurden. Die NATO-Luftangriffe
begannen innerhalb von weniger als 48 Stunden.

US-Professor Yossef Bodansky, langjähriger Direktor der \emph{US
Congressional Task Force on Terrorism and Unconventional Warfare},
\href{https://swprs.files.wordpress.com/2019/12/bodansky_offensive-in-the-balkans_1995.pdf\#page=40}{beschrieb}
diese Vorfälle später als »professionell inszenierte Horror­spektakel«
mit Einsatz von Leichen kürzlich gefallener bosnischer Soldaten.

Im Folgenden finden sich die wichtigsten Artikel aus der damaligen Zeit
von Journalisten, die die unveröffentlichten UNO-Berichte einsehen oder
mit Beteiligten darüber sprechen konnten:

\begin{itemize}
\tightlist
\item
  \href{https://www.independent.co.uk/news/muslims-slaughter-their-own-people-bosnia-bread-queue-massacre-was-propaganda-ploy-un-told-1541801.html}{Bosnia
  bread queue massacre was propaganda ploy, UN told}~ (Independent,
  1992)
\item
  \href{https://swprs.files.wordpress.com/2019/12/dateline-yugoslavia-the-partisan-press_peter-brock_foreign-policy_1994.pdf}{Dateline
  Yugoslavia: The Partisan Press} (Peter Brock, Foreign Policy, 1994,
  Archiv)
\item
  \href{https://swprs.files.wordpress.com/2019/12/anatomy-of-a-massacre_david-binder_foreign-policy_1994.pdf}{Anatomy
  of a Massacre} (David Binder, Foreign Policy, 1994, zu Markale I,
  Archiv)
\item
  \href{https://swprs.files.wordpress.com/2019/12/bosnias-bombers_david-binder_the-nation-1995.pdf}{Bosnia's
  Bombers} (David Binder, The Nation, 1995, zu Markale II, Archiv)
\item
  \href{https://swprs.files.wordpress.com/2019/12/sunday-times_serbs-not-guilty-of-massacre_1995.pdf}{»Serbs
  not guilty of massacre«} (Hugh McManners, The Sunday Times, 1995,
  Archiv)
\item
  \href{https://swprs.files.wordpress.com/2019/12/dpa_un-report-sarajevo_1996.pdf}{Senior
  official admits to secret U.N. report on Sarajevo massacre} (DPA,
  1996, Archiv)
\item
  2004: \href{http://news.bbc.co.uk/2/hi/europe/3459965.stm}{Sarajevo
  massacre remembered} (BBC; siehe Zitat von General Michael Rose)
\end{itemize}

Als der Auslandschef der Schweizer \emph{Weltwoche} den obigen Text von
Peter Brock unter dem Titel »Bosnien: So logen Fernsehen und Presse uns
an« 1994 auf Deutsch veröffentlichte, gab es derart starke Proteste
durch andere Medien, dass er ein vorläufiges
\href{https://swprs.org/das-gewuenschte-narrativ/}{Schreibverbot} zu
Bosnien erhielt.

Zwanzig Jahre später wurden die bosnischen Markale-Massaker von 1994/95
wieder in Erinnerung
\href{https://swprs.files.wordpress.com/2019/12/sarajevo-1995-damscus-2013-mass-attack-deceptions_world-tribune.pdf}{gerufen},
als sich Giftgas-Angriffe im Rahmen des Syrienkrieges als fragwürdig
herausstellten und wie damals Unter­suchungs­ergebnisse der UNO bzw.
OPCW
\href{https://www.dailymail.co.uk/news/article-7793253/PETER-HITCHENS-reveals-evidence-watchdog-suppressed-report-casting-doubt-Assad-gas-attack.html}{unterdrückt}
wurden.

Der Markale-Vorfall vom Februar 1994\\
Quelle: BBC,
\href{https://www.youtube.com/playlist?list=PLJvRFxihL4d03IzmoxyhU1C-kn27lxVvB}{The
Death of Yugoslavia}, 1995

\hypertarget{3-der-genozid-von-srebrenica-1995}{%
\paragraph{3. Der »Genozid von Srebrenica«
(1995)}\label{3-der-genozid-von-srebrenica-1995}}

Als trauriger Höhepunkt des Bosnienkriegs gilt der »Genozid von
Srebrenica« im Juli 1995. Dabei sollen laut westlichen Angaben, die
ursprünglich auf einen
\href{https://www.nytimes.com/1995/08/11/world/us-seeks-to-prove-mass-killings.html}{Bericht}
der US-Regierung zurück­gehen, mehr als 8000 bosnische Zivilisten
umgebracht worden sein.

Laut Phillip Corwin, dem ranghöchsten zivilen UNO-Vertreter in Bosnien
während des Krieges, deutet die tatsächliche Evidenz jedoch auf einen
komplexeren Sachverhalt und Kontext hin. Corwin
\href{https://swprs.files.wordpress.com/2019/12/the-srebrenica-massacre_edward-herman_2011.pdf\#page=7}{nannte}
die offizielle westliche Darstellung zu Srebrenica eine »Verzerrung«.

Der US-Politologe Edward Herman und der ehemalige CIA-Offizier Robert
Baer, der damals in Jugoslawien operierte,
\href{https://www.heise.de/tp/features/Nobelpreis-fuer-einen-Genozid-Leugner-4608176.html?seite=all}{sprachen}
in diesem Zusammenhang sogar von einem »Betrug«.

Für weitere Details sei beispielsweise auf folgende Artikel und
Dokumentationen verwiesen:

\begin{itemize}
\tightlist
\item
  \href{https://www.globalresearch.ca/the-politics-of-the-srebrenica-massacre/660}{The
  Politics of the Srebrenica Massacre} (Edward S. Herman, Global
  Research, 2005)
\item
  \href{https://www.counterpunch.org/2005/10/12/srebrenica-revisited/}{Srebrenica
  Revisited} (Diana Johnstone, Counterpunch Magazine, 2005)
\item
  \href{https://www.youtube.com/watch?v=FvqHWS_4AuM}{Srebrenica: A Town
  Betrayed} (Norwegische Dokumentation, 60 Minuten, 2010)
\item
  \href{https://swprs.files.wordpress.com/2019/12/the-srebrenica-massacre_edward-herman_2011.pdf}{The
  Srebrenica Massacre: Evidence, Context, Politics} (Edward S. Herman,
  ed., 2011)
\item
  \href{https://www.youtube.com/watch?v=KmkV7zHiHoQ}{Evakuierung von
  Flüchtlingen aus Srebrenica} (AP, Originalaufnahmen, 12. Juli 1995)
\item
  \textbf{DE}:
  \href{https://swprs.files.wordpress.com/2019/12/srebrenica_edward-herman_junge-welt_interview_2008.pdf}{»In
  Bosnien hat kein Völkermord stattgefunden«} (Edward Herman in JW,
  2008)
\item
  \textbf{DE}:
  \href{https://www.heise.de/tp/features/Nobelpreis-fuer-einen-Genozid-Leugner-4608176.html?seite=all}{Nobelpreis
  für einen »Genozid-Leugner«} (Michael Ewert, Telepolis, 2019)
\end{itemize}

Generell müssen auch Ereignisse mit sehr hohen berichteten Opferzahlen
bisweilen kritisch hinter­fragt werden. Dies zeigte etwa das
\href{https://www.france24.com/en/20091220-twenty-years-later-timisoara-affair-exposes-media-credulity}{»Timisoara-Massaker«}
von 1989 mit angeblich 4630 Toten, das sich später als psychologische
Operation im Rahmen der rumänischen Revolution herausstellte.

Srebrenica: A Town Betrayed (Norwegische Doku, 60 Minuten, 2010)

\hypertarget{4-kosovo-hufeisenplan-raux10dak-massaker-und-mehr-1999}{%
\paragraph{4. Kosovo: »Hufeisenplan«, »Račak-Massaker«, und mehr
(1999)}\label{4-kosovo-hufeisenplan-raux10dak-massaker-und-mehr-1999}}

Nach der Abtrennung von Slowenien, Kroatien und Bosnien starteten die
USA und die NATO 1999 einen weiteren Krieg gegen das verbleibende
Jugoslawien bzw. Serbien zur Abtrennung der Provinz Kosovo. Auch dieser
Krieg musste durch Propaganda und Desinformation begründet werden.

Dazu wurden insbesondere angebliche Vertreibungspläne,
Konzentrationslager und Massaker medial thematisiert, die sich später
jedoch als falsch oder fragwürdig herausstellten. Beispiele hierfür sind
etwa der angebliche
\href{https://de.wikipedia.org/wiki/Hufeisenplan}{»Hufeisenplan«} sowie
die Vorfälle von
\href{https://de.wikipedia.org/wiki/Massaker_von_Ra\%C4\%8Dak}{Račak}
und \href{https://de.wikipedia.org/wiki/Rogovo-Vorfall}{Rogovo}.

Im Falle von Račak etwa kamen finnische Forensiker zum
\href{https://swprs.files.wordpress.com/2019/12/racak-massacre_peter-worthington_toronto-sun_2001.pdf}{Ergebnis},
dass Gefechtstote der UCK-Miliz umplatziert, umgekleidet und als zivile
Exekutionsopfer ausgegeben wurden.

Nach dem Krieg erklärte der Chef einer amerikanischen PR-Agentur, die
solche zweifelhaften Darstellungen aus dem Kosovo verbreitet hatte, in
einem
\href{https://www.hintergrund.de/globales/kriege/operation-balkan-werbung-fuer-krieg-und-tod/}{Interview}:
»Ich muss sagen, als die NATO 1999 angriff, haben wir eine Flasche
Champagner aufgemacht.«

Für weitere Details wird im Folgenden die WDR-Doku »Es begann mit einer
Lüge« von 2001 gezeigt. Diese dokumentiert, wie westliche Politiker und
Militärs bewusst Falschinformationen veröffentlichten, um den Krieg auch
ohne UNO-Mandat legitimieren zu können.

\textbf{Siehe auch:}

\begin{itemize}
\tightlist
\item
  \href{https://www.youtube.com/watch?v=s7JU4cwtYZU}{Der Kosovo-Krieg --
  Eine gesteuerte Debatte} (Kurt Gritsch, Video, 2018)
\item
  \href{https://swprs.files.wordpress.com/2019/12/blaetter_wimmer_interview_2001.pdf}{Strategische
  Konfliktmuster auf dem Balkan} (Willy Wimmer, 2001, Archiv)
\item
  \href{https://swprs.files.wordpress.com/2019/12/meet-mister-massacre_ames-taibbi_the-exile_2000.pdf}{Meet
  Mr. Massacre} (Mark Ames and Matt Taibbi, The Exile, 2000, Archiv)
\item
  \href{https://swprs.files.wordpress.com/2019/12/racak-massacre_peter-worthington_toronto-sun_2001.pdf}{The
  hoax that started a war} (Peter Worthington, Toronto Sun, 2001,
  Archiv)
\item
  \href{https://www.spiegel.de/politik/ausland/kosovo-krieg-keine-beweise-fuer-massaker-von-racak-a-112775.html}{Kosovo-Krieg:
  Keine Beweise für Massaker von Racak} (Spiegel, 2001)
\end{itemize}

»Es begann mit einer Lüge« (Doku, WDR, 2001)\\
\href{https://swprs.files.wordpress.com/2019/12/es_begann_mit_einer_luege_monitor_doku_mit_kritik.pdf}{Dokumentation
zur Sendung} (AKF Heidelberg)

\hypertarget{weiterfuxfchrende-literatur}{%
\paragraph{Weiterführende Literatur}\label{weiterfuxfchrende-literatur}}

Deutsch:

\begin{itemize}
\tightlist
\item
  Bittermann (ed., 1994):
  \href{https://edition-tiamat.de/serbien-muss-sterbien/}{Serbien muss
  sterbien: Wahrheit und Lüge im jugo­slaw. Bürger­krieg}
\item
  Schneider (ed., 1997):
  \href{https://www.amazon.de/Bei-Andruck-Mord-deutschen-Balkankrieg/dp/3930786095}{Bei
  Andruck Mord: Die deutsche Propaganda und der Balkankrieg}
\item
  Elsässer (2007):
  \href{https://www.amazon.de/Kriegsl\%C3\%BCgen-NATO-Angriff-Jugoslawien-J\%C3\%BCrgen-Els\%C3\%A4sser/dp/3897065118}{Kriegslügen:
  Der NATO-Angriff auf Jugoslawien}
\item
  Becker/Beham (2008):
  \href{https://www.amazon.de/Operation-Balkan-Werbung-Vorwort-Norman/dp/3832935916}{Operation
  Balkan: Werbung für Krieg und Tod}
\item
  Gritsch (2016):
  \href{https://www.heise.de/tp/buch/telepolis_buch_3186181.html}{Nie
  wieder Krieg (ohne uns)! Grüne, Linke und Medien im Kosovo-Krieg}
\item
  \textbf{Video}: Gritsch (2018):
  \href{https://www.youtube.com/watch?v=N-5yxP1Wyao}{Die Kriege in
  Jugoslawien und die Remilitarisierung Deutschlands}
\end{itemize}

Englisch:

\begin{itemize}
\tightlist
\item
  Larry E. Craig (1997)
  \href{https://web.archive.org/web/20110206110107/http://rpc.senate.gov/releases/1997/iran.htm}{Bericht
  des US Senats zum Jugoslawienkrieg} (Archiv)
\item
  Herman/Peterson (2007):
  \href{https://monthlyreview.org/2007/10/01/the-dismantling-of-yugoslavia/}{The
  Dismantling of Yugoslavia} (Monthly Review)
\item
  \textbf{Video}: Pierre Gallois (2009)
  \href{https://archive.org/details/french-general-pierre-marie-gallois-on-yugoslavia-war-2009}{Französischer
  General zur westlichen Balkan-Strategie}
\end{itemize}

\hypertarget{siehe-auch}{%
\paragraph{Siehe auch}\label{siehe-auch}}

\begin{itemize}
\tightlist
\item
  \href{https://swprs.org/logik-imperialer-kriege/}{Die Logik imperialer
  Kriege}
\item
  \href{https://swprs.org/syrienkrieg-geopolitik-medien/}{Syrienkrieg:
  Geopolitik und Medien}
\item
  \href{https://swprs.org/ruanda-was-geschah-wirklich/}{Ruanda: Was
  geschah 1994 wirklich?}
\end{itemize}

\begin{center}\rule{0.5\linewidth}{\linethickness}\end{center}

Publiziert: Dezember 2019\\
Beitrag teilen auf:
\href{https://twitter.com/intent/tweet?url=https://swprs.org/propaganda-im-jugoslawienkrieg/}{Twitter}
/
\href{https://www.facebook.com/share.php?u=https://swprs.org/propaganda-im-jugoslawienkrieg/}{Facebook}

\hypertarget{swiss-policy-research}{%
\subsubsection{Swiss Policy Research}\label{swiss-policy-research}}

\begin{itemize}
\tightlist
\item
  \href{https://swprs.org/kontakt/}{Kontakt}
\item
  \href{https://swprs.org/uebersicht/}{Übersicht}
\item
  \href{https://swprs.org/donationen/}{Donationen}
\item
  \href{https://swprs.org/disclaimer/}{Disclaimer}
\end{itemize}

\hypertarget{english}{%
\subsubsection{English}\label{english}}

\begin{itemize}
\tightlist
\item
  \href{https://swprs.org/contact/}{About Us / Contact}
\item
  \href{https://swprs.org/media-navigator/}{The Media Navigator}
\item
  \href{https://swprs.org/the-american-empire-and-its-media/}{The CFR
  and the Media}
\item
  \href{https://swprs.org/donations/}{Donations}
\end{itemize}

\hypertarget{follow-by-email}{%
\subsubsection{Follow by email}\label{follow-by-email}}

Follow

\href{https://wordpress.com/?ref=footer_custom_com}{WordPress.com}.

\protect\hyperlink{}{Up ↑}

Post to

\protect\hyperlink{}{Cancel}

\includegraphics{https://pixel.wp.com/b.gif?v=noscript}
