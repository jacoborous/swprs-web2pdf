\protect\hyperlink{content}{Skip to content}

\href{https://swprs.org/}{}

\protect\hyperlink{search-container}{Search}

Search for:

\href{https://swprs.org/}{\includegraphics{https://swprs.files.wordpress.com/2020/05/swiss-policy-research-logo-300.png}}

\href{https://swprs.org/}{Swiss Policy Research}

Geopolitics and Media

Menu

\begin{itemize}
\tightlist
\item
  \href{https://swprs.org}{Start}
\item
  \href{https://swprs.org/srf-propaganda-analyse/}{Studien}

  \begin{itemize}
  \tightlist
  \item
    \href{https://swprs.org/srf-propaganda-analyse/}{SRF / ZDF}
  \item
    \href{https://swprs.org/die-nzz-studie/}{NZZ-Studie}
  \item
    \href{https://swprs.org/der-propaganda-multiplikator/}{Agenturen}
  \item
    \href{https://swprs.org/die-propaganda-matrix/}{Medienmatrix}
  \end{itemize}
\item
  \href{https://swprs.org/medien-navigator/}{Analysen}

  \begin{itemize}
  \tightlist
  \item
    \href{https://swprs.org/medien-navigator/}{Navigator}
  \item
    \href{https://swprs.org/der-propaganda-schluessel/}{Techniken}
  \item
    \href{https://swprs.org/propaganda-in-der-wikipedia/}{Wikipedia}
  \item
    \href{https://swprs.org/logik-imperialer-kriege/}{Kriege}
  \end{itemize}
\item
  \href{https://swprs.org/netzwerk-medien-schweiz/}{Netzwerke}

  \begin{itemize}
  \tightlist
  \item
    \href{https://swprs.org/netzwerk-medien-schweiz/}{Schweiz}
  \item
    \href{https://swprs.org/netzwerk-medien-deutschland/}{Deutschland}
  \item
    \href{https://swprs.org/medien-in-oesterreich/}{Österreich}
  \item
    \href{https://swprs.org/das-american-empire-und-seine-medien/}{USA}
  \end{itemize}
\item
  \href{https://swprs.org/bericht-eines-journalisten/}{Fokus I}

  \begin{itemize}
  \tightlist
  \item
    \href{https://swprs.org/bericht-eines-journalisten/}{Journalistenbericht}
  \item
    \href{https://swprs.org/russische-propaganda/}{Russische Propaganda}
  \item
    \href{https://swprs.org/die-israel-lobby-fakten-und-mythen/}{Die
    »Israel-Lobby«}
  \item
    \href{https://swprs.org/geopolitik-und-paedokriminalitaet/}{Pädokriminalität}
  \end{itemize}
\item
  \href{https://swprs.org/migration-und-medien/}{Fokus II}

  \begin{itemize}
  \tightlist
  \item
    \href{https://swprs.org/covid-19-hinweis-ii/}{Coronavirus}
  \item
    \href{https://swprs.org/die-integrity-initiative/}{Integrity
    Initiative}
  \item
    \href{https://swprs.org/migration-und-medien/}{Migration \& Medien}
  \item
    \href{https://swprs.org/der-fall-magnitsky/}{Magnitsky Act}
  \end{itemize}
\item
  \href{https://swprs.org/kontakt/}{Projekt}

  \begin{itemize}
  \tightlist
  \item
    \href{https://swprs.org/kontakt/}{Kontakt}
  \item
    \href{https://swprs.org/uebersicht/}{Seitenübersicht}
  \item
    \href{https://swprs.org/medienspiegel/}{Medienspiegel}
  \item
    \href{https://swprs.org/donationen/}{Donationen}
  \end{itemize}
\item
  \href{https://swprs.org/contact/}{English}
\end{itemize}

\protect\hyperlink{}{Open Search}

\hypertarget{der-basler-tierkreis}{%
\section{Der Basler Tierkreis}\label{der-basler-tierkreis}}

Beim 1998 aufgeflogenen »Basler Tierkreis« handelte es sich um einen
pädokriminellen Ring unter anderem aus Akademikern, Anwälten, Ärzten und
Geschäftsleuten aus den »besseren Kreisen Basels«, die sich zur Tarnung
Tiernamen gaben und seit circa den 1960er Jahren Strichjungen sowie
minderjährige Knaben an sadistischen »Sex-Partys« missbrauchten.

Internationale Verbindungen nach Spanien und Thailand waren belegt,
solche zum belgischen Dutroux-Netzwerk wurden vermutet. Die Hintergründe
des Tierkreises wurden nie gänzlich aufgedeckt, Anklagen erfolgten unter
anderem aufgrund von Verjährung keine und der Fall verschwand nach dem
Sommer 1998 vollständig aus den Medien.

Im Internet finden sich in neuerer Zeit auch Spekulationen bezüglich
Verbindungen zur Schweizer Bundespolitik und der P-26 Geheimarmee, für
die bislang jedoch keinerlei Belege vorliegen.

Es folgt eine Übersicht zu Medienberichten über den »Basler Tierkreis«
vom Juli/August 1998. Danach folgten keine weiteren Berichte zu diesem
bis heute weitgehend unaufgeklärten Komplex.

\textbf{Zum Hauptartikel}:
\href{https://swprs.org/geopolitik-und-paedokriminalitaet/}{Geopolitik
und Pädokriminalität}

\hypertarget{basel-grauenvolle-sexspiele}{%
\paragraph{Basel: Grauenvolle
Sexspiele}\label{basel-grauenvolle-sexspiele}}

\emph{FACTS, 9. Juli 1998, Seiten 33/34, Simon Hubacher}

\includegraphics{https://swprs.files.wordpress.com/2019/02/focus-tierkreis-1998-09-07.png?w=600\&h=400}

~

\textbf{Der Männerring mit dem Decknamen Tierkreis hätte schon viel
früher ausgehoben werden können.}

Das zweistöckige Reihenhaus in der Nähe des Basler Schützenmattparks
leuchtet in grüner Farbe. Ein ungewöhnlicher Anstrich, der sich in dem
beschaulichen Quartier von den übrigen Fassaden abhebt. Ungewöhnlich ist
auch das kleine Klebebild, das die Eingangstüre ziert. Es zeigt ein
Krokodil. Nachbarn erinnern sich daran, dass in dem Haus die Rollläden
oft schon am frühen Nachmittag heruntergelassen wurden. Als bedrohlich
empfand das aber niemand. Der nächste Polizeiposten ist nur ein paar
Meter entfernt.

Da hatten die Beamten am vergangenen Montag nicht weit. Um 17 Uhr
tauchten Ermittler des Basler Kriminalkommissariats mit einem
Hausdurchsuchungsbeschluss vor dem Gebäude auf. Zuvor war die
Liegenschaft beobachtet worden. Es waren nicht zuletzt die grüne Farbe
und der Krokodil-Kleber, welche die Polizisten dazu bewogen, in die
ominöse Liegenschaft einzudringen.

Besitzer ist der 64-jährige Basler Geschäftsmann R. F., von Freunden
«Kroko» genannt. Er wird verdächtigt, in dem Haus seit Anfang der
achtziger Jahre unter dem Decknamen Tierkreis einen geheimen Männerring
geführt zu haben, dessen Mitglieder sich mit Tiernamen tarnen. Sie
sollen Strichjungen aus dem Basler Schwulenmilieu sowie minderjährige
Knaben zu widerlichen, von Gewaltpraktiken und Drogenmissbrauch
geprägten Sex-Partys gezwungen haben.

Die Beamten stellten im grünen Haus nach eigenen Angaben «umfangreiches
Beweismaterial» sicher, darunter zahlreiche Videobänder, 98 ältere
Super-8-Filme sowie kistenweise private Fotos von nackten Männern und
Jugendlichen. Sie stiessen zudem auf Namenslisten möglicher Mitglieder
des Sexringes. Zum Tierkreis gehören nach den ersten Erkenntnissen
ausschliesslich Männer, vorwiegend Akademiker, Geschäftsleute, Ärzte und
Wirte -- allesamt aus den besseren Kreisen Basels.

Den Behörden waren schon vor der Hausdurchsuchung die Identitäten von
einem halben Dutzend Tierkreis-Mitgliedern bekannt, darunter zwei
Mediziner: ein Universitätsdozent, Tarnname Elefant, sowie ein Arzt
namens Eisbär mit eigener Praxis.

Dass die Beamten überhaupt fündig wurden, ist spektakulär. Lange Zeit
schenkten sie den Aussagen ihres einzigen Zeugen keinen Glauben. Der
frühere Drogenabhängige und Strichjunge Lucky verbrachte nach eigenen
Angaben zu Beginn der neunziger Jahre zwei Jahre in den Fängen des
Männerringes. Vergangenen April schilderte er der Basler Staatsanwältin
Judie Melzl während einer vierstündigen Einvernahme ausführlich, welche
Gräueltaten sich in den Räumen des grünen Hauses abgespielt hatten.

Entstanden war ein Protokoll des Unfassbaren. Lucky war als 22-Jähriger
erstmals an R. F. geraten und von ihm in den Ring eingeführt worden. Er
avancierte rasch zum Lieblingsknaben der Partybesucher. Jeden Mittwoch
ging es los. «Um 15 Uhr trank Krokodil einen Appenzeller, dann drehte er
den ersten Joint und wartete auf Gäste», erinnert sich Lucky. Im Keller
des grünen Hauses befindet sich ein Weinkeller, daneben eine Bar und
eine kleine Disco. An der Wand eine grosse Leinwand, in der Nähe die
üppige Filmsammlung -- verbotene Videos mit Kindersex, sadistischen und
sodomistischen Quälereien. Auch solche mit echten Todesszenen sollen
darunter gewesen sein. An der Wand hatte das Krokodil in einem
Bilderrahmen eine polizeiliche Vorladung hängen, wegen des Verdachts auf
Misshandlungen. «Kroko lachte, wenn er das Schreiben sah», sagt Lucky,
«man habe ihm nie etwas nachweisen können.»

Das Krokodil zeigte Lucky ein Buch, voll mit Fotos junger Männer,
jeweils drei Aufnahmen: Gesicht, Gesäss und Penis. Wie ein Sklave musste
Lucky härtesten, schmerzhaften Männersex ertragen. Gegen Bares, um seine
Drogensucht zu stillen. Lucky trank Urin und malträtierte Brustwarzen.
Als Stammgäste Lucky mit Zigaretten verstümmelten und ihm ein K -- wie
Krokodil -- auf den Rücken brennen wollten, stieg er aus.

Laut Aussagen von Lucky und seinem Vormund haben die Behörden bereits
vor zwei Jahren erste Hinweise auf einen Missbrauch von Minderjährigen
und Abhängigen im organisierten Stil erhalten. Anzeichen existierten --
nur wurde ihnen lange kein Glauben geschenkt. Wohl aus Angst vor einem
grossen Skandal. Auch nach Luckys Aussage vom April dieses Jahres blieb
die Staatsanwaltschaft zunächst untätig. Bis zur Hausdurchsuchung
vergingen zehn Wochen.

Dass doch noch Funde im grünen Haus gemacht wurden, lässt auch einen
anderen Schluss zu: dass die Aktivitäten des Tierkreises zwar seit
einiger Zeit eingestellt waren, sich die Mitglieder aber in Sicherheit
wiegten und es nicht für nötig befanden, Beweismaterial wegzuschaffen.

Einer, der vom obskuren Sexring zumindest ansatzweise seit längerer Zeit
wusste, ist der Basler Kantonsarzt Hanspeter Rohr. Er hatte Ende 1996
während eines anderen Verfahrens von obskuren geheimen Treffen erfahren.
Doch Rohr war überzeugt, dass es sich ausschliesslich um Treffen unter
homosexuellen Erwachsenen handelte. Auch dass Rohr den Namen einer der
beiden involvierten Ärzte kannte, blieb ohne Folgen. Er vertraute
darauf, dass die Staatsanwaltschaft ebenso Kenntnis von den Vorwürfen
erhielt. Davon will Markus Melzl, Sprecher der Staatsanwaltschaft,
nichts wissen: «Es gab keine solchen Informationen.» Inzwischen ist
Kantonsarzt Rohr zur Überzeugung gelangt, dass die damaligen
Ermittlungen voreilig eingestellt wurden.

Bei der begonnenen Spurensuche gegen den Tierkreis werden die Behörden
wohl weite Kreise ziehen müssen. Mitglieder des Sexklubs sollen sich
regelmässig im spanischen Ferienhaus des «Eisbären» getroffen haben.
«Krokodil» habe breite internationale Kontakte zu sadistisch veranlagten
Männern gepflegt. Vorerst Spekulation bleiben Hinweise, wonach
Verbindungen zum belgischen Kinderschänder Marc Dutroux bestanden.
Kenntnis haben die Untersuchungsbehörden auch von einer Reise nach
Thailand erhalten. Dort wollten sich die Tiere den ultimativen Kick
verschaffen: zusehen, wie ein Mann vor Publikum zu Tode gefoltert wird.

\begin{center}\rule{0.5\linewidth}{\linethickness}\end{center}

\hypertarget{der-chef-nennt-sich-krokodil-krimineller-puxe4dophilenring-tierkreis-in-basel-mit-vielen-kindern-und-jugendlichen-als-opfern}{%
\paragraph{Der Chef nennt sich Krokodil: Krimineller Pädophilenring
``Tierkreis'' in Basel mit vielen Kindern und Jugendlichen als
Opfern}\label{der-chef-nennt-sich-krokodil-krimineller-puxe4dophilenring-tierkreis-in-basel-mit-vielen-kindern-und-jugendlichen-als-opfern}}

\emph{Sonntagsblick, 5. Juli 1998, Seite A-2}

Basel -- \textbf{Das ``Sumpfhuhn'' singt -- und bringt angesehene Basler
Bürger um den Schlaf: Ärzte, Chemiker und Manager betreiben am Rheinknie
einen perversen ``Tierkreis''. Die Opfer für ihre Sado-Maso-Spiele
besorgen sich die hohen Tiere auf dem Babystrich!}

``Lucky'' war gerade 14 und hing an der Nadel. Das Geld für sein Heroin
beschaffte sich der Jung-Junkie als Stricher -- in den einschlägigen
Parks und Bars der Rheinstadt.

``Eines Abends bat man mich im Restaurant `Dupf' an einen Tisch. Das
`Krokodil' wolle mich sprechen, hiess es'', erinnert sich Lukas Frey,
heute 29, an die Nacht, in der alles anfing.

\textbf{``Frisches Fleisch für den Tierkreis''}

Das Krokodil, das sich für Lucky interessierte, war der Prokurist R.F.
Dieser stammt aus einer angesehenen Basler Fabrikantenfamilie und lebt
heute vom Ertrag seiner Immobilien.

Das Krokodil und der blutjunge Stricher waren sich schnell handelseinig.
``Ich brauche Geld für die Drogen -- und er frisches Fleisch für seinen
Tierkreis'', sagt Lucky.

Der Prokurist Kroko betreibe seit Anfang der 80er Jahre einen geheimen
Sexring, getarnt als Tierkreis. Der Staatsanwältin Judie Melzl nannte er
auch die Namen von angesehenen Baslern, die dem perversen Kreis
angehören:

-- Der ``Elefant'' im Tierkreis ist der Arzt und Privatdozent Dr. Y.Z.
Er sollte einmal für Lucky 10.000 Franken bezahlen, weigerte sich aber
im letzten Moment.\\
-- Als ``Eisbär'' gehört der Pharmakologe U.J. zum Sexring.\\
-- Der ``Weisse Affe'' ist ein bekannter Friseur, der aus Basel stammt
und heute in Frankreich arbeitet.\\
-- Der Wirt X.Y. nennt sich ``Wieseli''.\\
-- Als ``Adler'' fungiert der Geschäftsmann P.L.

``Ich musste jeden Dienstag zu Kroko'', berichtet Lucky dem
SonntagsBlick. ``Im Keller liefen wüste Partys -- mit viel Alkohol und
schmerzhaften Sado-Maso-Spielen. Kroko, Eisbär, Wieseli und Co. geilten
sich an Kinderpornos auf -- und liessen sich von mir und den andern
Buben befriedigen. Regelmässig reiste der Tierkreis nach Thailand und
auf die Philippinen. In Fernost seien die Buben halt billiger, erklärten
sie.''

Lucky avancierte schnell zum Lieblings-Stricher von Kroko. Der Tierkreis
ehrte ihn mit einem Pokal für ``das süsseste Schleckmaul''. Schliesslich
belohnte Kroko den Strichjungen mit der Aufnahme in den Tierkreis --
Lucky erhielt den Namen ``Ralle'', den Namen eines Sumpfhuhns.

\textbf{Das perverse Archiv des Krokodils}

Lucky: ``Kroko führt auch einen Katalog mit Fotos des Hintern und der
Genitalien jedes Buben. Jeder Stricher musste eine getragene Unterhose
in sein `Archiv' abgeben.''

Die Quälereien, der Whisky und die Drogen forderten ihren Tribut. Völlig
am Ende wandte sich Lucky an den Arzt Dr. K. -- mit Hilfe von Methadon
wollte er vom Heroin wegkommen. Dr. K. gehört zwar nicht zum Tierkreis,
trotzdem kam es auch bei ihm zu Sex. ``Ich musste den Arzt oral
befriedigen, damit ich mein Methadon bekam'', behauptete Lucky.

Immerhin: Mit Hilfe von Methadon schaffte Lucky den Ausstieg aus den
Drogen, ein Sozialarbeiter verhilft ihm zu einer IV-Rente. Jetzt wagte
er auch den Absprung aus dem Tierkreis: ``Die Praktiken wurden immer
brutaler -- und ohne die Drogen spürte ich die Schmerzen doppelt und
dreifach'', erklärt Lucky.

Endgültig genug vom Tierkreis aber hatte Lucky, als die Herren wieder
einmal eine Reise planten: ``Sie wollten nach Thailand, um dort
zuzusehen, wie Pädophile ein Kind zu Tode quälten'', sagt Lucky.

SonntagsBlick konfrontierte auch Prokurist R.F. mit den Vorwürfen. Doch
das Krokodil verweigerte das Gespräch: ``Lucky ist ein Trudi Gerster,
eine Märlitante'', erklärte er.

Doch wird das Krokodil wohl demnächst trotzdem Stellung nehmen müssen.
Die Aussagen, die Lucky gegenüber SonntagsBlick gemacht hat, deponierte
er auch bei der Basler Justiz.

``Wir ermitteln seit dem 21. April {[}1998{]}. Zurzeit machen wir eine
Auslegeordnung. Wenn sich der Verdacht erhärtet, schlagen wir auch zu'',
erklärt Kriminalkommissär Markus Melzl, Sprecher der Basler
Staatsanwaltschaft.

\begin{center}\rule{0.5\linewidth}{\linethickness}\end{center}

\hypertarget{perverse-sexpartys-mit-buben--bis-heute-geduldet}{%
\paragraph{Perverse Sexpartys mit Buben -- bis heute
geduldet}\label{perverse-sexpartys-mit-buben--bis-heute-geduldet}}

\emph{Blick, 6. Juli 1998, Seite 2, Beat Alder}

BASEL -- \textbf{Angesehene Biedermänner trafen sich im Geheimzirkel
``Tierkreis'' zu perversen Sexpartys. Bis ein Opfer im SonntagsBlick
auspackte. Jetzt kam aus: Die Basler Strafverfolger sahen dem Treiben
des Zirkels seit langem zu. Ohne einzuschreiten!}

Bislang interessierten sich die Basler Strafverfolger nicht für den
Sexzirkel. Dabei besassen die Behörden seit fast zwei Jahren
Informationen über den Tierkreis.

``Wir ermitteln jetzt mit aller Härte'', sagte Staatsanwältin Judie
Melzl vor zwei Wochen gegenüber BLICK. Und fuhr in die Ferien!

Weshalb die Behörden den Fall nicht längst mit allem Nachdruck
verfolgten, konnte sie nicht erklären.

Die Mitglieder des Tierkreises sind angesehene Bürger: Ärzte, Chemiker,
Pharmakologen, Anwälte und Geschäftsleute. Allen gemein ist die Vorliebe
für perversen Sex mit Knaben.

An der Spitze des Tierkreises steht das ``Krokodil'', ein über
60-jähriger Geschäftsmann. Sein dreistöckiges Reihenhaus an bester
Wohnlage in Basel-West diente dem Geheimzirkel jahrelang als
Anlaufstelle.

``Elefant'' ist ein renommierter Spezialarzt, ``Eisbär'' ist
Pharmakologe. Ein Baselbieter Wirt war als ``Wieseli'' bekannt. Ein
Geschäftsmann nannte sich ``Adler'', ein Prominenten-Coiffeur nahm als
``Weisser Affe'' an den Partys teil.

Eines ihrer Opfer: Lukas ``Lucky'' Frey, heute 29. Er kennt den
Tierkreis aus leidvoller Erfahrung. Als 14-Jähriger geriet Lucky in die
Fänge von ``Kroko''.

``Ich war jeden Dienstag dran'', sagte der Methadon-abhängige Ex-Junkie
dem SonntagsBlick. Auch die anderen Opfer der ``Tiere'' waren gefangen
zwischen Drogensucht und Sklaverei: Die ``Tiere'' bezahlten gut.

Lucky wollte aussteigen. Er wandte sich an den Kinderarzt Dr. K. Von ihm
erhielt er Methadon. Und musste mit Sex bezahlen. Dr. K. wusste alles
über die ``Tiere''. Lucky: ``Er war richtig scharf darauf, sich von mir
die Details erzählen zu lassen.''

Die Basler Staatsanwaltschaft sah keinen Grund, den renommierten Arzt
und damaligen Präsidenten der Kinderspitalkommission anzuklagen: Das
Verfahren wurde eingestellt!

Jetzt wird es neu aufgerollt.

\begin{center}\rule{0.5\linewidth}{\linethickness}\end{center}

\hypertarget{der-tierkreis-war-ein-sexzirkel}{%
\paragraph{Der ``Tierkreis'' war ein
Sexzirkel}\label{der-tierkreis-war-ein-sexzirkel}}

*Tages-Anzeiger, 8. Juli 1998, Seite 14, Thomas Renold\\
*

Basel. -- In Basel ist eine Organisation unter dem Decknamen
``Tierkreis'' wegen Verdachts auf sexuellen Missbrauch von Jugendlichen
ausgehoben worden. Auf die Spur des ``Tierkreises'' führten die
Ermittlungen gegen einen 70jährigen Arzt. Ein ehemaliger Patient habe
den Arzt wegen sexueller Übergriffe angezeigt, teilte am Dienstag die
Staatsanwaltschaft in Basel mit. Bereits vor zwei Jahren wurde gegen den
Arzt er­mit­telt. Weil sich damals die Vorwürfe nicht erhärten liessen,
stellte die Staatsanwaltschaft das Verfahren ein.

Geleitet wurde der ``Tierkreis'' von einem 64jährigen Mann mit dem
Übernamen ``Krokodil''. Bei einer Hausdurchsuchung seien Bilder von
nackten Männern, pornographisches Material und Namenslisten gefunden
worden, sagte Kriminalkommissär Markus Melzl. Die Mitglieder des
Sexzirkels würden alle aus der Region Basel stammen. Ob sie bei ihren
Treffs strafbare sexuelle Praktiken oder sexuelle Handlungen mit
Minderjährigen durchführten, müsse die Untersuchung zeigen.

Laut Aussagen des Patienten, der wegen seiner Heroinsucht in ärztlicher
Behandlung stand, gehören dem ``Tierkreis'' prominente Männer an. Buben
und Drogenabhängige seien von ihnen während Jahrzehnten missbraucht
worden.

\begin{center}\rule{0.5\linewidth}{\linethickness}\end{center}

\hypertarget{tierkreis-porno-material-entdeckt}{%
\paragraph{``Tierkreis'': Porno-Material
entdeckt}\label{tierkreis-porno-material-entdeckt}}

\emph{Basler Zeitung, 8. Juli 1998, Seite 27, Valentin Kressler}

\textbf{Bei den Untersuchungen gegen die Organisation ``Tierkreis'' ist
die Staatsanwaltschaft fündig geworden: Am Montag Abend hat sie bei
einer Hausdurchsuchung eine grosse Menge an pornographischem Material
entdeckt. Ob sich darunter auch strafrechtlich verbotene Sachen
befinden, wird derzeit noch abgeklärt.}

``Basler Sex-Ring aufgeflogen! -- Ärzte und Manager missbrauchten
Kinder'', titelte der ``SonntagsBlick'' am vergangenen Wochenende und
brachte damit eine Untersuchung der Staatsanwaltschaft Basel-Stadt gegen
eine Organisation namens ``Tierkreis'' an die Öffentlichkeit. Deren
``prominente'' Basler Mitglieder sollen angeblich sexuelle Praktiken und
Handlungen ausüben, die verboten sind, wie auch Markus Melzl, der Chef
Medien und Information bei der Staatsanwaltschaft, bestätigt hat (vgl.
BaZ vom 6. Juli).

\textbf{Eine Fülle von Material}

Gestern Dienstag ist nun die Basler Staatsanwaltschaft aus eigener
Initiative an die Öffentlichkeit gelangt: In einem Communiqué teilte die
Untersuchungsbehörde mit, dass sie am Montag Abend bei einem als
Mitglied des ``Tierkreises'' bezeichneten 64jährigen Mann eine
Hausdurchsuchung vorgenommen und dabei eine grosse Menge an
pornographischem Material und umfangreiche Unterlagen im Zusammenhang
mit der Organisation ``Tierkreis'' sichergestellt hat. Man habe dabei
eine ``Fülle von Material'' gefunden, präzisierte Melzl im Gespräch mit
der BaZ. Einerseits handelt es sich um eine umfangreiche Sammlung von
Filmen und Fotos, darunter befinden sich alleine schon 98 Rollen
Super-8-Filme, andererseits um Namen und Daten von Mitgliedern der
Organisation.

``Auf den ersten Blick enthält das Material der Hausdurchsuchungen aber
nur Bilder von nackten jungen Männern und nicht von Minderjährigen'',
erklärte der Sprecher der Staatsanwaltschaft. In den nächsten Tagen
gelte es jetzt zuerst einmal abzuklären, ob es sich bei den
sichergestellten Gegenständen um legale Pornographie handle, oder ob
sich auch strafrechtlich verbotene Sachen darunter befinden würden. Im
Mittelpunkt der Ermittlungen der Gruppe für Sexualdelikte des
Kriminalkommissariats stehen die Straftatbestände der sexuellen
Handlungen mit Kindern, der sexuellen Nötigung sowie der verbotenen
Pornographie. Wie schon in der BaZ vom Montag betonte Melzl, dass die
Untersuchungsbehörden erst seit kurzem Kenntnis von der Organisation
hätten. ``Basel ist keine Millionenmetropole. Wir hätten es ziemlich
sicher gewusst, wenn über Jahre hinweg sexuelle Handlungen mit Kindern
begangen worden wären.''

Beim 64jährigen Mann, bei dem das Material am Montag Abend
sichergestellt wurde, soll es sich nach den Angaben von Melzl um den
Kopf der Organisation ``Tierkreis'' und bei dessen Privatwohnung im
Grossbasel um das Hauptquartier der Gruppe handeln. Der früherer
Prokurist ist unter dem Decknamen ``Krokodil'' bekannt und nicht
``prominent''. Der Mann befindet sich derzeit auf freiem Fuss und nicht
etwa in Unter­suchungs­haft, weil nach Ansicht der Staatsanwaltschaft
keinerlei Haftgründe vorliegen. ``Er hätte das Material ja schon lange
verschwinden lassen oder selbst flüchten können'', begründete Melzl das
Vorgehen.

\textbf{Drei weitere Mitglieder}

Bei den anderen drei bis heute ermittelten Mitgliedern der Organisation
``Tierkreis'' handelt es sich neben einem Apotheker und einem Arzt aus
dem Kanton Basel-Stadt auch um einen Wirt aus dem Kanton Baselland. Eine
allfällige Hausdurchsuchung bei diesen drei Personen werde allerdings
erst dann zur Diskussion stehen, wenn bei der Sichtung des nun
gefundenen Materials strafrechtlich verbotene Sachen entdeckt würden,
sagte Melzl auf eine entsprechende Frage der BaZ. ``Die weiteren
Schritte machen wir von der Auswertung des Materials abhängig.''

Die Basler Staatsanwaltschaft ist gemäss Melzl auf die Organisation
``Tierkreis'' im Rahmen der Wiedereinleitung eines vor zweieinhalb
Jahren eingestellten Strafverfahrens gegen einen mittlerweile
pensionierten 70jährigen Basler Arzt aufmerksam geworden. Dieser soll
dem Geschädigten, einem in einem Methadonprogramm stehenden 29-jährigen
Mann das Methadon angeblich erst dann ausgehändigt haben, wenn dieser
ihn zuvor sexuell befriedigt hatte. Bei diesem Verfahren gibt es im
übrigen zurzeit keine Neuigkeiten zu vermelden, wie Melzl erklärte.

\begin{center}\rule{0.5\linewidth}{\linethickness}\end{center}

\hypertarget{tierkreis-jahrelang-aktiv}{%
\paragraph{Tierkreis jahrelang aktiv}\label{tierkreis-jahrelang-aktiv}}

\emph{Tages-Anzeiger, 15. Juli 1998, Seite 12}

Basel. -- \textbf{Die in Basel wegen Sexualdelikten verdächtigte
Organisation ``Tierkreis'' (TA vom 8. Juli) ist offenbar bereits vor 30
Jahren aktiv gewesen.}

Bei Hausdurchsuchungen in der vergangenen Woche hat die Basler
Staatsanwaltschaft 98 Super-8-Filmspulen, Unmengen an Videokassetten,
Heftchen und Privatphotos sichergestellt.

Ein Teil des sichergestellten Filmmaterials sei mit Sicherheit 30 bis 35
Jahre alt; allfällige strafbare Handlungen seien somit verjährt. Bei der
Visionierung des Bildmaterials wurden bisher jedoch keine strafrechtlich
relevanten Handlungen festgestellt.

\textbf{Kauf von Porno nicht strafbar}

Ebenfalls sichergestellte Aufnahmen von Kinderpornographie und
Pornographie mit Tieren seien aufgrund der bisherigen Ermittlungen nur
gekauft und nicht selber produziert worden, was in der Schweiz nicht
strafbar ist.

Die Basler Untersuchungsbehörden haben laut Melzl eine Liste von rund 20
Mitgliedern der Organisation im Alter von heute 60 bis 70 Jahren,
darunter ein Anwalt, ein Apotheker, ein Arzt und der als treibende Kraft
der Gruppe geltende Prokurist. Die Vertreter der Gruppe hätten
zugegeben, sich zu Sexspielen getroffen zu haben. Die Untersuchungen
laufen weiter.

Unabhängig davon läuft das Verfahren gegen einen Arzt (60), der nicht
zum ``Tierkreis'' gehört hat. Ein 29jähriger wirft ihm vor, er habe
diesen oral befriedigen müssen, um von ihm Methadon zu erhalten. (AP)

\begin{center}\rule{0.5\linewidth}{\linethickness}\end{center}

\hypertarget{basler-sexring-das-ist-das-krokodil}{%
\paragraph{Basler Sexring: Das ist das
Krokodil}\label{basler-sexring-das-ist-das-krokodil}}

\emph{Sonntagsblick, 12. Juli 1998,~ Seite A10}

Lucky (29) -- er war gerade vierzehn, als er in die Fänge des Krokodils
geriet. Dem SonntagsBlick erzählte er exklusiv, wie er vom
Multimillionär missbraucht wurde. Wie ihn der Tierkreis mit der Aufnahme
in den Männerring ehrte und er den Namen ``Ralle'', den eines
Sumpfhuhns, erhielt.

Das Krokodil -- es heisst mit bürgerlichem Namen René F. (64). Und: Es
ist der Boss des Tierkreises. Dem gehören angesehene Männer an; Ärzte,
Chemiker, Manager, vor allem aus dem Basler Daig, aber nicht nur: Der
Tierkreis ist international. Einige der 120 Mitglieder kommen aus
Marokko, Thailand, Spanien, England und Deutschland. Und sie alle
zittern. Denn Lucky ist kein Einzelfall. Immer mehr Opfer sprechen offen
über die traumatischen Ereignisse ihrer Kindheit.

Beat* (35) kämpft mit den Tränen. Seine Stimme stockt: ``Ich war damals
erst elf. Ich musste Bier trinken, Hasch rauchen, Poppers sniffen, mir
die widerlichsten Sex-Filme ansehen, bevor mich das Krokodil drannahm.''
Mit Malen versucht er heute, die Missbräuche zu verarbeiten. Beat: ``Die
Orgien dauerten manchmal zwanzig Stunden lang.''

Auch Reto* (32), ein bärenstarker Mann, redet sich jetzt die Seelenpein
vom Leibe. ``Mit dreizehn musste ich mir einen Film anschauen, in dem
eine Rockerbande ein 13-jähriges Mädchen vergewaltigte. Es lag auf einem
Altar. Seither kann ich keine Kirche mehr betreten.'' Reto weint: ``Das
Krokodil hat mich mit zwei Kollegen vergewaltigt. Ich musste Urin
trinken und wurde an den Geschlechtsteilen mit Stecknadeln traktiert.''

\textbf{Das Krokodil leugnet alles}

Und was sagt dazu das Krokodil? ``Alles Hirngespinste. Ich habe nie mit
Buben geschlafen.''

Die Basler Polizei führte diese Woche in den Räumlichkeiten des
Krokodils drei Hausdurchsuchungen durch. Beschlagnahmt wurden 98
Super-8-Streifen, darunter Kinderpornos und Filme, die Sex mit Tieren
beinhalten. Weiter Hunderte von Nacktfotos, die Tagebücher des Krokodils
der vergangenen zwanzig Jahre und die Namensliste der
Tierkreismitglieder.

Kriminalkommissar Markus Melzl, Sprecher der Staatsanwaltschaft: ``Wir
entdeckten auch eine Falltür, die mit einem Teppich und einem Stuhl
zugedeckt war.'' Doch Melzl befürchtet, dass die meisten mutmasslichen
Straftaten bereits verjährt sind. Dazu die Basler SP-Nationalrätin
Margrith von Felten: ``Die Verjährungsfristen müssen unbedingt
verlängert werden. Konkret: Sie dürfen erst mit der Mündigkeit der Opfer
zu laufen beginnen. Wer Kinder missbraucht, begeht Seelenmord.''

* Name der Redaktion bekannt

\textbf{Die Polizei sucht weitere Opfer des Krokodils}

Die Staatsanwaltschaft Basel-Stadt ruft sämtliche Opfer des Tierkreises
dazu auf, sich mit ihr umgehend in Verbindung zu setzen (061-267 71 71).
Kriminalkommissar Markus Melzl: ``Unsere Beamtinnen und Beamten sind in
der Lage, die heiklen Befragungen einfühlsam durchzuführen. Dem
Opferhilfegesetz wird voll Rechnung getragen. Die Befragungen werden
durch Personen gleichen Geschlechts durchgeführt. Die Opfer können zur
Ein­ver­nahme eine selbstgewählte Vertrauensperson mitnehmen. Notfalls
erhalten sie psychologische Betreuung.''

\begin{center}\rule{0.5\linewidth}{\linethickness}\end{center}

\hypertarget{viele-animalische-geschichten-uxfcber-den-basler-tierkreis}{%
\paragraph{Viele animalische Geschichten über den Basler
``Tierkreis''}\label{viele-animalische-geschichten-uxfcber-den-basler-tierkreis}}

\emph{Basler Zeitung, 16. Juli 1998, Seite 21, Markus Sutter}

(Anmerkung: Ein relativierender und teilweise irreführender Beitrag der
\emph{Basler Zeitung}.)

\textbf{Seit Anfang Juli berichtet die auflagenstärkste schweizer
Zeitung serienmässig über ein regionales Ereignis, das aber aus
strafrechtlicher Sicht noch keine Zeile wert ist: Denn bis jetzt fehlen
Anhaltspunkte dafür, dass Pädophile in einem ``Tierkreis-Sexclub'' ihre
perversen Neigungen ausgelebt haben. Hingegen muss ein Arzt mit einer
Anzeige rechnen.}

\emph{Basel. msu.} Unter dem schlagzeilenträchtigen Titel ``Basler
Sex-Ring aufgeflogen! -- Ärzte und Manager missbrauchten Kinder'' sorgte
der ``Sonntagsblick'' zu Beginn der Sommerferien für etliche Aufregung
in der traditionell nachrichtenarmen Zeit. Kronzeuge war ein 29-jähriger
Mann. Nach Aussagen des Geschädigten sollen zahlreiche prominente Basler
verbotene sexuelle Praktiken mit Kindern ausgeübt haben.

Zumindest eine Aussage stimmt im letzten Satz bestimmt nicht: Auf der
rund 120-köpfigen Mitgliederliste, welche sich in den Händen der
Staatsanwaltschaft befindet, tauche niemand auf, welcher mit der ``High
Society'' irgendwie in Verbindung gebracht werden könne, stellte
Kriminalkommissär Markus Melzl gestern gegenüber der BaZ klar. Wer
beispielsweise einen Politiker ausfindig machen wolle, suche völlig
vergebens.

\textbf{Fälle durcheinandergebracht}

Im weiteren stört sich der Sprecher der Basler Staatsanwaltschaft vor
allem am Umstand, dass hier zwei Vorkommnisse, welche grundsätzlich
nichts miteinander zu tun hätten, von den Medien immer wieder
durcheinandergebracht würden. Einzig das Opfer spiele beide Male eine
zentrale Rolle. Der erste Fall dreht sich um einen älteren Arzt, welcher
mit dem 29-jährigen Kronzeugen Lucky sexuelle Kontakte pflegte und
dessen Methadonabhängigkeit ausgenützt haben soll. Ein erstes Verfahren
sei dann aber am 16. Januar 1996 eingestellt worden, weil der jüngere
Mann von einer freiwillig gewählten Liebesbeziehung zum Arzt sprach.
Doch ein Betreuer scheine diese Ausführungen nicht für bare Münze
genommen zu haben, so dass in der Folge eine Neuauflage des Verfahrens
eingeleitet worden sei.

\textbf{Frühere Aussagen korrigiert}

Lucky widerrief dabei seine früheren Aussagen und sprach neuerdings von
Druckversuchen und einem Abhängigkeitsverhältnis. Nach dem gegenwärtigen
Stand der Dinge müsse der Arzt jetzt entweder mit einer Anklage wegen
sexueller Nötigung oder der Ausnützung einer Notlage rechnen, sagte
Melzl. Es gebe keinen Grund, dem jungen Mann nicht zu glauben.

So viel zum ersten Fall. Der zweite Fall dreht sich um den sogenannten
``Tierkreis'', worauf Lucky die Behörden im Laufe des zweiten
Ermittlungsverfahrens aufmerksam gemacht hatte. Schwer wogen in diesem
Zusammenhang seine Vorwürfe, dass ein sadomasochistisch veranlagter
Männerring Jugendliche missbraucht und gequält haben soll. In zwei
Hausdurchsuchungen wurde zwar eine Fülle von porno­gra­phi­schem
Material -- auch über Kinder -- gefunden. Da die sichergestellte Ware
aufgrund der bisherigen Ermittlungen nur gekauft und nicht selber
produziert worden war, liegt laut Melzl jedoch kein strafrechtlich
relevanter Tatbestand vor. Denn der blosse Besitz solcher Aufnahmen
werde in der Schweiz toleriert.

\textbf{Wohl überall verjährt}

Strafbar seien aber sexuelle Handlungen mit Kindern unter 16 Jahren.
Beweise dafür, dass irgendwelche Pädophile am Werk waren, fehlten
indessen nach wie vor. Und wenn, dann dürften die meisten Vorfälle laut
Melzl sowieso verjährt sein. Das ist nach spätestens zehn oder in
Ausnahmefällen nach 15 Jahren der Fall (sofern sich der Täter
beispielsweise vorübergehend in Haft befand). Der 29-jährige Lucky wäre
also streng rechtlich gesehen kein Opfer mehr, genauso wenig wie ein
54-jähriger Mann, der sich schriftlich bei den Behörden gemeldet habe
und vorgab, im Alter von 14 Jahren von ``Tierkreis-Mitgliedern'' sexuell
misshandelt worden zu sein.

``Wir wären aber eine schlechte Strafverfolgungsbehörde, wenn wir immer
mit der Verjährung kommen würden'', machte Melzl gegenüber der BaZ klar.
Geschädigte, welche bei den Befragungen eine Person ihres Vertrauens
beiziehen könnten, sollten sich in jedem Fall bei den Behörden melden.
Die Verfahren würden streng nach Opferhilfegesetz durchgeführt.

Nicht auszuschliessen ist allerdings generell, dass die
``Tierkreis''-Story insbesondere in Zürcher Medien nicht nur
hochgespielt, sondern auch in völlig verzerrter Weise dargestellt wurde.
Ein Paradebeispiel dafür lieferte die Zeitung ``Facts'', welcher
``gerüchteweise'' zu Ohren kam, dass der ``Basler Männerring Kontakte
zum Kinderschänder Dutroux pflegte'' und im selben Artikel auch gleich
ein Bild des verhafteten Belgiers veröffentlichte.

\textbf{Keine Moralapostel}

Soweit würde sich Melzl niemals auf die Äste hinauslassen. Zwar steht
für den Kriminalkommissär einwandfrei fest, dass sich am Ort des
Geschehens kein Literaturclub ein Stelldichein gab. Vielmehr scheint es
sich um einen bevorzugten Treffpunkt für Schwule zu handeln. Die Basler
Staatsanwaltschaft mache aber weder Jagd auf Schwule, noch masse sie
sich das Recht an, eine Moralbehörde zu sein, betonte Melzl.

\textbf{``Tierkreis-Mitglied'' meldete sich}

Aus Verärgerung über die Berichterstattung in der Sonntagspresse hat
sich ein ``Tierkreis''-Mitglied, welches über die ``Szene'' seit langem
Bescheid zu wissen glaubt, direkt bei der BaZ gemeldet. Stricher seien
etliche anzutreffen. Er kenne aber absolut niemanden, welcher mit
Kinderpornographie etwas am Hut habe, beteuerte er. Unter den 120 Leuten
figurierten zudem mehrere Frauen und Nichtschwule. Sie alle seien vom
Chef der losen Verbindung spasseshalber mit einem Tiernamen ausgestattet
worden.

Seine Schwester beispielsweise habe das ``Krokodil'', welches in der
Öffentlichkeit bereits mehrmals als Kopf der Organisation bezeichnet
wurde, auf den Namen ``Pferd'' getauft.

Weil findige Internet-Freaks aufgrund der veröffentlichten Initialen
teilweise die Namen von ``Tierkreis''-​Mitgliedern ausfindig machen
konnten, fühlen sich nun einige von ihnen bedrängt und haben auch
bereits mit rechtlichen Schritten gedroht. Markus Melzl wäre froh, wenn
sich diese Betroffenen vorgängig mit ihm oder der Gruppe für
Sexualdelikte beim Kriminalkommissariat in Verbindung setzen würden.

\begin{center}\rule{0.5\linewidth}{\linethickness}\end{center}

\hypertarget{neue-vorwuxfcrfe-gegen-basler-tierkreis}{%
\paragraph{Neue Vorwürfe gegen Basler
``Tierkreis''}\label{neue-vorwuxfcrfe-gegen-basler-tierkreis}}

\emph{Basler Zeitung, 25. Juli 1998, Seite 29, Valentin Kressler}

**Bei den Ermittlungen gegen den Basler ``Tierkreis'' sind drei neue
potentiell Geschädigte aufgetaucht. Von strafrechtlich relevanten
Handlungen kann aber nach wie vor keine Rede sein.\\
**\\
Die Affäre um die Organisation ``Tierkreis'', deren Mitgliedern
sexueller Missbrauch von Jugendlichen vorgeworfen wird, zieht weitere
Kreise: Seit dem BaZ-Artikel vom 16. Juli hat die Staatsanwaltschaft
neben dem 29-jährigen ``Lucky'', der die ganze Angelegenheit ins Rollen
gebracht hat, nämlich von drei weiteren potentiell Geschädigten Kenntnis
bekommen, wie Mediensprecher Markus Melzl auf Anfrage erklärte. Die
Untersuchungsbehörde hat gestern und vorgestern die Aussagen von zwei
dieser Opfer aufgenommen. Beim jetzigen Stand der Dinge, so Melzl,
würden sich die von ``Lucky'' erhobenen Vorwürfe bestätigen.

``Der Hauptvorwurf ist, dass es zwischen Mitgliedern des Tierkreises und
den Geschädigten zu sexuellen Handlungen gekommen ist, zum Zeitpunkt,
als diese noch nicht 16 Jahre alt waren'', erläuterte der Sprecher der
Staatsanwaltschaft. Womit der Artikel 187 des Schweizerischen
Strafgesetzbuches (Sexuelle Handlungen mit Kindern) unter Umständen
erfüllt wäre. Wäre, denn die zur Diskussion stehenden Handlungen sind --
wie schon bei ``Lucky'' -- längst verjährt. Die für solche Fälle
massgebende Verjährungsfrist beträgt zehn Jahre seit der letzten Tat.
Der Staatsanwaltschaft liegt also zum jetzigen Zeitpunkt keine einzige
strafrechtlich relevante Handlung vor. Aus diesem Grund könne gegen die
Tatverdächtigen, so hält Melzl klar fest, kein Strafverfahren
eingeleitet werden, weshalb es bisher auch zu keiner Einvernahme
gekommen sei.

Bei den drei neu aufgetauchten Geschädigten handelt es sich im übrigen
um zwei Basler (32 und 54 Jahre alt) sowie um einen im Raum Konstanz
wohnhaften älteren Deutschen. Dessen Aussage konnte noch nicht
aufgenommen werden, weil er sich zurzeit ferienhalber in den USA
aufhält.

Im Gespräch mit der BaZ zeigte sich Melzl denn auch unzufrieden mit der
geltenden Gesetzgebung: ``Die heutige Regelung der Verjährung bei den
Sexualdelikten ist unbefriedigend. Eine zehnjährige Frist, die mit der
Volljährigkeit der Betroffenen zu laufen beginnt, wäre sinnvoller. Das
hätte zudem eine grosse prä­ven­tive Wirkung.'' Angesprochen auf die
nächsten Ermittlungsschritte nennt der Mediensprecher in erster Linie
die nun laufenden Zeugenbefragungen. Auch hofft er, dass sich bei der
Staatsanwaltschaft weitere Personen melden, die über nicht verjährte,
strafrechtlich relevante Handlungen Auskunft geben können.

Der mutmassliche Kopf des ``Tierkreises'', ein früherer Prokurist mit
dem Decknamen ``Krokodil'', wollte gegenüber der BaZ keine Stellungnahme
zu den jüngsten Vorwürfen abgeben. Der Mann wies jedoch darauf hin, dass
es zu keinerlei sexuellen Handlungen mit Minderjährigen gekommen sei.

\begin{center}\rule{0.5\linewidth}{\linethickness}\end{center}

\hypertarget{offensive-gegen-tierkreis-grosse-namenslisten-und-geplante-massenverhuxf6re}{%
\paragraph{Offensive gegen ``Tierkreis'': Grosse Namenslisten und
geplante
Massenverhöre}\label{offensive-gegen-tierkreis-grosse-namenslisten-und-geplante-massenverhuxf6re}}

\emph{Sonntagsblick, 9. August 1998, Seite A15}

BASEL -- Die Basler Polizei bläst zur Offensive gegen den
Sex-``Tierkreis'': Geplant sind Massenverhöre. ``Bei Hausdurchsuchungen
stiessen wir auf Dutzende von Namen'', erklärt Markus Melzl von der
Staats­an­walt­schaft. ``Diese Personen werden wir jetzt alle
kontaktieren.'' Bis anhin eruierte die Polizei vier Opfer, die bezeugen,
als Kinder von Mitgliedern des ``Tierkreises'' sexuell missbraucht
worden zu sein. Melzl: ``Die Taten sind aber alle verjährt. Wir suchen
jetzt Geschädigte, deren Missbrauch weniger lang zurückliegt.''

\textbf{Letzter bekannter Medienbericht zum »Tierkreis«.}

\begin{center}\rule{0.5\linewidth}{\linethickness}\end{center}

Publiziert: Februar 2019

\hypertarget{swiss-policy-research}{%
\subsubsection{Swiss Policy Research}\label{swiss-policy-research}}

\begin{itemize}
\tightlist
\item
  \href{https://swprs.org/kontakt/}{Kontakt}
\item
  \href{https://swprs.org/uebersicht/}{Übersicht}
\item
  \href{https://swprs.org/donationen/}{Donationen}
\item
  \href{https://swprs.org/disclaimer/}{Disclaimer}
\end{itemize}

\hypertarget{english}{%
\subsubsection{English}\label{english}}

\begin{itemize}
\tightlist
\item
  \href{https://swprs.org/contact/}{About Us / Contact}
\item
  \href{https://swprs.org/media-navigator/}{The Media Navigator}
\item
  \href{https://swprs.org/the-american-empire-and-its-media/}{The CFR
  and the Media}
\item
  \href{https://swprs.org/donations/}{Donations}
\end{itemize}

\hypertarget{follow-by-email}{%
\subsubsection{Follow by email}\label{follow-by-email}}

Follow

\href{https://wordpress.com/?ref=footer_custom_com}{WordPress.com}.

\protect\hyperlink{}{Up ↑}

Post to

\protect\hyperlink{}{Cancel}

\includegraphics{https://pixel.wp.com/b.gif?v=noscript}
