\protect\hyperlink{content}{Skip to content}

\href{https://swprs.org/}{}

\protect\hyperlink{search-container}{Search}

Search for:

\href{https://swprs.org/}{\includegraphics{https://swprs.files.wordpress.com/2020/05/swiss-policy-research-logo-300.png}}

\href{https://swprs.org/}{Swiss Policy Research}

Geopolitics and Media

Menu

\begin{itemize}
\tightlist
\item
  \href{https://swprs.org}{Start}
\item
  \href{https://swprs.org/srf-propaganda-analyse/}{Studien}

  \begin{itemize}
  \tightlist
  \item
    \href{https://swprs.org/srf-propaganda-analyse/}{SRF / ZDF}
  \item
    \href{https://swprs.org/die-nzz-studie/}{NZZ-Studie}
  \item
    \href{https://swprs.org/der-propaganda-multiplikator/}{Agenturen}
  \item
    \href{https://swprs.org/die-propaganda-matrix/}{Medienmatrix}
  \end{itemize}
\item
  \href{https://swprs.org/medien-navigator/}{Analysen}

  \begin{itemize}
  \tightlist
  \item
    \href{https://swprs.org/medien-navigator/}{Navigator}
  \item
    \href{https://swprs.org/der-propaganda-schluessel/}{Techniken}
  \item
    \href{https://swprs.org/propaganda-in-der-wikipedia/}{Wikipedia}
  \item
    \href{https://swprs.org/logik-imperialer-kriege/}{Kriege}
  \end{itemize}
\item
  \href{https://swprs.org/netzwerk-medien-schweiz/}{Netzwerke}

  \begin{itemize}
  \tightlist
  \item
    \href{https://swprs.org/netzwerk-medien-schweiz/}{Schweiz}
  \item
    \href{https://swprs.org/netzwerk-medien-deutschland/}{Deutschland}
  \item
    \href{https://swprs.org/medien-in-oesterreich/}{Österreich}
  \item
    \href{https://swprs.org/das-american-empire-und-seine-medien/}{USA}
  \end{itemize}
\item
  \href{https://swprs.org/bericht-eines-journalisten/}{Fokus I}

  \begin{itemize}
  \tightlist
  \item
    \href{https://swprs.org/bericht-eines-journalisten/}{Journalistenbericht}
  \item
    \href{https://swprs.org/russische-propaganda/}{Russische Propaganda}
  \item
    \href{https://swprs.org/die-israel-lobby-fakten-und-mythen/}{Die
    »Israel-Lobby«}
  \item
    \href{https://swprs.org/geopolitik-und-paedokriminalitaet/}{Pädokriminalität}
  \end{itemize}
\item
  \href{https://swprs.org/migration-und-medien/}{Fokus II}

  \begin{itemize}
  \tightlist
  \item
    \href{https://swprs.org/covid-19-hinweis-ii/}{Coronavirus}
  \item
    \href{https://swprs.org/die-integrity-initiative/}{Integrity
    Initiative}
  \item
    \href{https://swprs.org/migration-und-medien/}{Migration \& Medien}
  \item
    \href{https://swprs.org/der-fall-magnitsky/}{Magnitsky Act}
  \end{itemize}
\item
  \href{https://swprs.org/kontakt/}{Projekt}

  \begin{itemize}
  \tightlist
  \item
    \href{https://swprs.org/kontakt/}{Kontakt}
  \item
    \href{https://swprs.org/uebersicht/}{Seitenübersicht}
  \item
    \href{https://swprs.org/medienspiegel/}{Medienspiegel}
  \item
    \href{https://swprs.org/donationen/}{Donationen}
  \end{itemize}
\item
  \href{https://swprs.org/contact/}{English}
\end{itemize}

\protect\hyperlink{}{Open Search}

\hypertarget{the-propaganda-multiplier}{%
\section{The Propaganda Multiplier}\label{the-propaganda-multiplier}}

\textbf{Languages}:
\href{https://swprs.org/der-propaganda-multiplikator/}{DE},
\href{https://www.bibliotecapleyades.net/sociopolitica2/sociopol_mediacontrol225.htm}{ES},
\href{https://swprs.org/le-multiplicateur-de-propagande/}{FR},
\href{https://www.bibliotecapleyades.net/sociopolitica2/sociopol_mediacontrol228.htm}{IT},
\href{https://swprs.files.wordpress.com/2019/12/propaganda-multiplier-dutch.pdf}{NL},
\href{https://midtifleisen.wordpress.com/2018/01/04/en-titt-pa-nyhetsbyraenes-rolle/}{NO},
\href{https://wolnemedia.net/powielacze-propagandy/}{PL},
\href{https://revistaopera.com.br/2019/04/23/a-propagacao-hegemonica-como-as-agencias-globais-e-a-midia-ocidental-cobrem-a-geopolitica-parte-1/}{PT},
\href{https://csa.pnzgu.ru/infopswars/ipw1}{RU}

It is one of the most important aspects of our media system, and yet
hardly known to the public: most of the international news coverage in
Western media is provided by only three global news agencies based in
New York, London and Paris.

The key role played by these agencies means Western media often report
on the same topics, even using the same wording. In addition,
governments, military and intelligence services use these global news
agencies as multipliers to spread their messages around the world.

A study of the Syria war coverage by nine leading European newspapers
clearly illustrates these issues: 78\% of all articles were based in
whole or in part on agency reports, yet 0\% on investigative research.
Moreover, 82\% of all opinion pieces and interviews were in favor of a
US and NATO intervention, while propaganda was attributed exclusively to
the opposite side.

Share this study on:
\href{https://twitter.com/intent/tweet?url=https://swprs.org/the-propaganda-multiplier/}{Twitter}
/
\href{https://www.facebook.com/share.php?u=https://swprs.org/the-propaganda-multiplier/}{Facebook}

\includegraphics{https://swprs.files.wordpress.com/2019/02/propaganda-multiplier.png?w=512\&h=588}

\begin{center}\rule{0.5\linewidth}{\linethickness}\end{center}

\hypertarget{the-propaganda-multiplier-1}{%
\subsection{The Propaganda
Multiplier:}\label{the-propaganda-multiplier-1}}

\hypertarget{how-global-news-agencies-and}{%
\subsection{How Global News Agencies
and}\label{how-global-news-agencies-and}}

Western Media Report on Geopolitics

A Study by \href{https://swprs.org/contact/}{Swiss Propaganda
Research}*\\
*

Translated by Terje Maloy

2016 / 2019

``Therefore, you always have to ask yourself: Why do I get this\\
specific information, in this specific form, at this specific moment?\\
Ultimately, these are always questions about power.''
(\href{http://www.nzz.ch/wer-lustvoll-schreibt-der-schreibt-auch-gut-1.11329756}{*})\\
Dr. Konrad Hummler, Swiss banking and media executive

Contents

\begin{enumerate}
\def\labelenumi{\arabic{enumi}.}
\tightlist
\item
  \protect\hyperlink{k1}{Part 1: The Propaganda Multiplier}
\item
  \protect\hyperlink{k2}{Part 2: Case Study on Syria War Coverage}
\item
  \protect\hyperlink{k3}{Notes and Literature}
\end{enumerate}

\hypertarget{introduction-something-strange}{%
\paragraph{Introduction: ``Something
strange''}\label{introduction-something-strange}}

``How does the newspaper know what it knows?'' The answer to this
question is likely to surprise some newspaper readers: ``The main source
of information is stories from news agencies. The almost anonymously
operating news agencies are in a way the key to world events. So what
are the names of these agencies, how do they work and who finances them?
To judge how well one is informed about events in East and West, one
should know the answers to these questions.'' (Höhne 1977, p. 11)

A Swiss media researcher points out: ``The news agencies are the most
important suppliers of material to mass media. No daily media outlet can
manage without them. () So the news agencies influence our image of the
world; above all, we get to know what they have selected.'' (Blum 1995,
p. 9)

In view of their essential importance, it is all the more astonishing
that these agencies are hardly known to the public: ``A large part of
society is unaware that news agencies exist at all \ldots{} In fact,
they play an enormously important role in the media market. But despite
this great importance, little attention has been paid to them in the
past.'' (Schulten-Jaspers 2013, p. 13)

Even the head of a news agency noted: ``There is something strange about
news agencies. They are little known to the public. Unlike a newspaper,
their activity is not so much in the spotlight, yet they can always be
found at the source of the story.'' (Segbers 2007, p. 9)

\hypertarget{the-invisible-nerve-center-of-the-media-system}{%
\paragraph{``The Invisible Nerve Center of the Media
System''}\label{the-invisible-nerve-center-of-the-media-system}}

So what are the names of these agencies that are ``always at the source
of the story''? There are now only three global news agencies left:

\begin{enumerate}
\def\labelenumi{\arabic{enumi}.}
\tightlist
\item
  The American \textbf{Associated Press}
  (\href{https://en.wikipedia.org/wiki/Associated_Press}{AP}) with over
  4000 employees worldwide. The AP belongs to US media companies and has
  its main editorial office in New York. AP news is used by around
  12,000 international media outlets, reaching more than half of the
  world's population every day.
\item
  The quasi-governmental French \textbf{Agence France-Presse}
  (\href{https://en.wikipedia.org/wiki/Agence_France-Presse}{AFP}) based
  in Paris and with around 4000 employees. The AFP sends over 3000
  stories and photos every day to media all over the world.
\item
  The British agency \textbf{Reuters} in London, which is privately
  owned and employs just over 3000 people. Reuters was acquired in 2008
  by Canadian media entrepreneur Thomson -- one of the 25 richest people
  in the world -- and merged into
  \href{https://en.wikipedia.org/wiki/Reuters}{Thomson Reuters},
  headquartered in New York.
\end{enumerate}

In addition, many countries run their own news agencies. These include,
for instance, the German DPA, the Austrian APA, and the Swiss SDA. When
it comes to international news, however, national agencies usually rely
on the three global agencies and simply copy and translate their
reports.

\includegraphics{https://swprs.files.wordpress.com/2017/01/logos_agenturen.png?w=600\&h=259}

\emph{The three global news agencies Reuters, AFP and AP, and the three
national agencies of the German-speaking countries of Austria (APA),
Germany (DPA) and Switzerland (SDA).}

Wolfgang Vyslozil, former managing director of the Austrian APA,
described the key role of news agencies with these words: ``News
agencies are rarely in the public eye. Yet they are one of the most
influential and at the same time one of the least known media types.
They are key institutions of substantial importance to any media system.
They are the invisible nerve center that connects all parts of this
system.'' (Segbers 2007, p.10)

\hypertarget{small-abbreviation-great-effect}{%
\paragraph{Small abbreviation, great
effect}\label{small-abbreviation-great-effect}}

However, there is a simple reason why the global agencies, despite their
importance, are virtually unknown to the general public. To quote a
Swiss media professor: ``Radio and television usually do not name their
sources, and only specialists can decipher references in magazines.''
(Blum 1995, P. 9)

The motive for this discretion, however, should be clear: news outlets
are not particularly keen to let readers know that they haven't
researched most of their contributions themselves.

The following figure shows some examples of source tagging in popular
European newspapers. Next to the agency abbreviations we find the
initials of editors who have edited the respective agency report.

\includegraphics{https://swprs.files.wordpress.com/2019/02/agenturen-quellen.png?w=650}

\emph{News agencies as sources in newspaper articles}

Occasionally, newspapers use agency material but do not label it at all.
A study in 2011 from the Swiss \emph{Research Institute for the Public
Sphere and Society} at the University of Zurich came to the following
conclusions (FOEG 2011):

``Agency contributions are exploited integrally without labeling them,
or they are partially rewritten to make them appear as an editorial
contribution. In addition, there is a practice of `spicing up' agency
reports with little effort: for example, unpublished agency reports are
enriched with images and graphics and presented as comprehensive
articles.''

The agencies play a prominent role not only in the press, but also in
private and public broadcasting. This is
\href{http://www.heise.de/tp/artikel/47/47821/3.html}{confirmed} by
Volker Braeutigam, who worked for the German state broadcaster ARD for
ten years and views the dominance of these agencies critically:

``One fundamental problem is that the newsroom at ARD sources its
information mainly from three sources: the news agencies DPA/AP, Reuters
and AFP: one German/American, one British and one French. () The editor
working on a news topic only needs to select a few text passages on the
screen that he considers essential, rearrange them and glue them
together with a few flourishes.''

Swiss Radio and Television (SRF), too, largely bases itself on reports
from these agencies. Asked by viewers why a peace march in Ukraine was
not reported, the editors
\href{http://www.srf.ch/sendungen/hallosrf/warum-berichtet-srf-nicht-ueber-den-friedensmarsch-in-der-ukraine}{said}:
``To date, we have not received a single report of this march from the
independent agencies Reuters, AP and AFP.''

In fact, not only the text, but also the images, sound and video
recordings that we encounter in our media every day, are mostly from the
very same agencies. What the uninitiated audience might think of as
contributions from their local newspaper or TV station, are actually
copied reports from New York, London and Paris.

Some media have even gone a step further and have, for lack of
resources, outsourced their entire foreign editorial office to an
agency. Moreover, it is well known that many news portals on the
internet mostly publish agency reports (see e.g., Paterson 2007,
Johnston 2011, MacGregor 2013).

In the end, this dependency on the global agencies creates a striking
similarity in international reporting: from Vienna to Washington, our
media often report the same topics, using many of the same phrases -- a
phenomenon that would otherwise rather be associated with »controlled
media« in authoritarian states.

The following graphic shows some examples from German and international
publications. As you can see, despite the claimed objectivity, a slight
(geo-)political bias sometimes creeps in.

\includegraphics{https://swprs.files.wordpress.com/2019/02/agency-headlines.png?w=650}

*``Putin threatens'', ``Iran provokes'', ``NATO concerned'', ``Assad
stronghold'': Similarities in content and wording due to reports by
global news agencies.\\
*

\hypertarget{the-role-of-correspondents}{%
\paragraph{The role of
correspondents}\label{the-role-of-correspondents}}

Much of our media does not have own foreign correspondents, so they have
no choice but to rely completely on global agencies for foreign news.
But what about the big daily newspapers and TV stations that have their
own international correspondents? In German-speaking countries, for
example, these include newspapers such NZZ, FAZ, Sueddeutsche Zeitung,
Welt, and public broadcasters.

First of all, the size ratios should be kept in mind: while the global
agencies have several thousand employees worldwide, even the Swiss
newspaper NZZ, known for its international reporting, maintains only 35
foreign correspondents (including their business correspondents). In
huge countries such as China or India, only one correspondent is
stationed; all of South America is covered by only two journalists,
while in even larger Africa no-one is on the ground permanently.

Moreover, in war zones, correspondents rarely venture out. On the Syria
war, for example, many journalists ``reported'' from cities such as
Istanbul, Beirut, Cairo or even from Cyprus. In addition, many
journalists lack the language skills to understand local people and
media.

How do correspondents under such circumstances know what the ``news'' is
in their region of the world? The main answer is once again: from global
agencies. The Dutch Middle East correspondent Joris Luyendijk has
impressively described how correspondents work and how they depend on
the world agencies in his book
\href{https://www.amazon.com/People-Like-Us-Misrepresenting-Middle/dp/1593762569}{``People
Like Us: Misrepresenting the Middle East''}:

``I had imagined correspondents to be historians-of-the-moment. When
something important happened, they would go after it, find out what was
going on, and report on it. But I didn't go off to find out what was
going on; that had been done long before. I went along to present an
on-the-spot report.

The editors in the Netherlands called when something happened, they
faxed or emailed the press releases, and I'd retell them in my own words
on the radio, or rework them into an article for the newspaper. This was
the reason my editors found it more important that I could be reached in
the place itself than that I knew what was going on. The news agencies
provided enough information for you to be able to write or talk your way
through any crisis or summit meeting.

That's why you often come across the same images and stories if you leaf
through a few different newspapers or click the news channels.

Our men and women in London, Paris, Berlin and Washington bureaus -- all
thought that wrong topics were dominating the news and that we were
following the standards of the news agencies too slavishly.

The common idea about correspondents is that they `have the story', ()
but the reality is that the news is a conveyor belt in a bread factory.
The correspondents stand at the end of the conveyor belt, pretending
we've baked that white loaf ourselves, while in fact all we've done is
put it in its wrapping.

Afterwards, a friend asked me how I'd managed to answer all the
questions during those cross-talks, every hour and without hesitation.
When I told him that, like on the TV-news, you knew all the questions in
advance, his e-mailed response came packed with expletives. My friend
had relalized that, for decades, what he'd been watching and listening
to on the news was pure theatre.'' (Luyendjik 2009, p. 20-22, 76, 189)

In other words, the typical correspondent is in general not able to do
independent research, but rather deals with and reinforces those topics
that are already prescribed by the news agencies -- the notorious
``mainstream effect''.

In addition, for cost-saving reasons many media outlets nowadays have to
share their few foreign correspondents, and within individual media
groups, foreign reports are often used by several publications -- none
of which contributes to diversity in reporting.

\hypertarget{what-the-agency-does-not-report-does-not-take-place}{%
\paragraph{``What the agency does not report, does not take
place''}\label{what-the-agency-does-not-report-does-not-take-place}}

The central role of news agencies also explains why, in geopolitical
conflicts, most media use the same original sources. In the Syrian war,
for example, the
\href{https://en.wikipedia.org/wiki/Syrian_Observatory_for_Human_Rights}{``Syrian
Observatory for Human Rights''} -- a dubious one-man organization based
in London --~ featured prominently. The media rarely inquired directly
at this ``Observatory'', as its operator was in fact difficult to reach,
even for journalists.

Rather, the ``Observatory'' delivered its stories to global agencies,
which then forwarded them to thousands of media outlets, which in turn
``informed'' hundreds of millions of readers and viewers worldwide. The
reason why the agencies, of all places, referred to this strange
``Observatory'' in their reporting -- and who really financed it -- is a
question that was rarely asked.

The former chief editor of the German news agency DPA, Manfred Steffens,
therefore states in his book ``The Business of News'':

``A news story does not become more correct simply because one is able
to provide a source for it. It is indeed rather questionable to trust a
news story more just because a source is cited. () Behind the protective
shield such a `source' means for a story, some people are inclined to
spread rather adventurous things, even if they themselves have
legitimate doubts about their correctness; the responsibility, at least
morally, can always be attributed to the cited source.'' (Steffens 1969,
p. 106)

Dependence on global agencies is also a major reason why media coverage
of geopolitical conflicts is often superficial and erratic, while
historic relationships and background are fragmented or altogether
absent. As put by Steffens: ``News agencies receive their impulses
almost exclusively from current events and are therefore by their very
nature ahistoric. They are reluctant to add any more context than is
strictly required.'' (Steffens 1969, p. 32)

Finally, the dominance of global agencies explains why certain
geopolitical issues and events -- which often do not fit very well into
the US/NATO narrative or are too ``unimportant'' -- are not mentioned in
our media at all: if the agencies do not report on something, then most
Western media will not be aware of it. As pointed out on the occasion of
the 50th anniversary of the German DPA: ``What the agency does not
report, does not take place.'' (Wilke 2000, p. 1)

\hypertarget{adding-questionable-stories}{%
\paragraph{``Adding questionable
stories''}\label{adding-questionable-stories}}

While some topics do not appear at all in our media, other topics are
very prominent -- even though they shouldn't actually be: ``Often the
mass media do not report on reality, but on a constructed or staged
reality. () Several studies have shown that the mass media are
predominantly determined by PR activities and that passive, receptive
attitudes outweigh active-researching ones.'' (Blum 1995, p. 16)

In fact, due to the rather low journalistic performance of our media and
their high dependence on a few news agencies, it is easy for interested
parties to spread propaganda and disinformation in a supposedly
respectable format to a worldwide audience. DPA editor Steffens warned
of this danger:

``The critical sense gets more lulled the more respected the news agency
or newspaper is. Someone who wants to introduce a questionable story
into the world press only needs to try to put his story in a reasonably
reputable agency, to be sure that it then appears a little later in the
others. Sometimes it happens that a hoax passes from agency to agency
and becomes ever more credible.'' (Steffens 1969, p. 234)

Among the most active actors in ``injecting'' questionable geopolitical
news are the military and defense ministries. For example, in 2009 the
head of the American news agency AP, Tom Curley,
\href{https://harpers.org/blog/2009/02/pentagon-targeted-and-mistreated-journalists-ap-head-charges/}{made
public} that the Pentagon employs more than 27,000 PR specialists who,
with a budget of nearly \$ 5 billion a year, are working the media and
circulating targeted manipulations. In addition, high-ranking US
generals had threatened that they would ``ruin'' him and the AP if the
journalists reported too critically on the US military.

Despite -- or because of? -- such threats our media regularly publish
dubious stories sourced to some unnamed~ ``informants'' from ``US
defense circles''.

Ulrich Tilgner, a veteran Middle East correspondent for German and Swiss
television, warned in 2003, shortly after the Iraq war, of acts of
deception by the military and the role played by the media:

``With the help of the media, the military determine the public
perception and use it for their plans. They manage to stir expectations
and spread deceptive scenarios. In this new kind of war, the PR
strategists of the US administration fulfill a similar function as the
bomber pilots. The special departments for public relations in the
Pentagon and in the secret services have become combatants in the
information war.

For their deception maneuvers, the US military specifically uses the
lack of transparency in media coverage. The way they spread information,
which is then picked up and distributed by newspapers and broadcasters,
makes it impossible for readers, listeners or viewers to trace the
original source. Thus, the audience will fail to recognize the actual
intention of the military.'' (Tilgner 2003, p. 132)

What is known to the US military, would not be foreign to US
intelligence services. In a
\href{https://swprs.org/video-the-cia-and-the-media/}{remarkable~
report} by British Channel 4, former CIA officials and a Reuters
correspondent spoke candidly about the systematic dissemination of
propaganda and misinformation in reporting on geopolitical conflicts:

Former CIA officer and whistleblower
\href{https://en.wikipedia.org/wiki/John_Stockwell}{John Stockwell} said
of his work in the Angolan war: ``The basic theme was to make it look
like an {[}enemy{]} aggression. So any kind of story that you could
write and get into the media anywhere in the world, that pushed that
line, we did. One third of my staff in this task force were
propagandists, whose professional career job was to make up stories and
finding ways of getting them into the press. () The editors in most
Western newspapers are not too skeptical of messages that conform to
general views and prejudices. () So we came up with another story, and
it was kept going for weeks. () But it was all fiction.''

\href{https://en.wikipedia.org/wiki/Fred_Bridgland}{Fred Bridgland}
looked back on his work as a war correspondent for the Reuters agency:
``We based our reports on official communications. It was not until
years later that I learned that a little CIA disinformation expert had
sat in the US embassy and had composed these communiqués that bore
absolutely no relationship at all to truth. () Basically, and to put it
very crudely, you can publish any old crap and it will get into the
newspaper.''

And former CIA analyst
\href{https://en.wikipedia.org/wiki/David_MacMichael}{David MacMichael}
described his work in the Contra War in Nicaragua with these words:
``They said our intelligence of Nicaragua was so good that we could even
register when someone flushed a toilet. But I had the feeling that the
stories we were giving to the press came straight out of the toilet.''
(\href{https://swprs.org/video-the-cia-and-the-media/}{Hird 1985})

Of course, the intelligence services also have a large number of
\href{http://www.carlbernstein.com/magazine_cia_and_media.php}{direct
contacts} in our media, which can be ``leaked'' information to if
necessary. But without the central role of the global news agencies, the
worldwide synchronization of propaganda and disinformation would never
be so efficient.

Through this ``propaganda multiplier'', dubious stories from PR experts
working for governments, military and intelligence services reach the
general public more or less unchecked and unfiltered. The journalists
refer to the news agencies and the news agencies refer to their sources.
Although they often attempt to point out uncertainties (and hedge
themselves) with terms such as ``apparent'', ``alleged'' and the like --
by then the rumor has long been spread to the world and its effect has
taken place.

\includegraphics{https://swprs.files.wordpress.com/2019/02/propaganda-multiplier.png?w=550}

*The Propaganda Multiplier: Governments, military and intelligence
services using global news agencies to disseminate their messages to a
worldwide audience.\\
*

\hypertarget{as-the-new-york-times-reported-}{%
\paragraph{As the New York Times reported
\ldots{}}\label{as-the-new-york-times-reported-}}

In addition to global news agencies, there is another source that is
often used by media outlets around the world to report on geopolitical
conflicts, namely the major publications in Great Britain and the US.

News outlets like the New York Times or the BBC may have up to 100
foreign correspondents and additional external employees. However, as
Middle East correspondent Luyendijk points out:

``Our news teams, me included, fed on the selection of news made by
quality media like \emph{CNN,} the \emph{BBC,} and the \emph{New York
Times}. We did that on the assumption that their correspondents
understood the Arab world and commanded a view of it -- but many of them
turned out not to speak Arabic, or at least not enough to be able to
have a conversation in it or to follow the local media. Many of the top
dogs at CNN, the BBC, the Independent, the Guardian, the New Yorker, and
the NYT were more often than not dependent on assistants and
translators.'' (Luyendijk p. 47)

In addition, the sources of these media outlets are often not easy to
verify (``military circles'', ``anonymous government officials'',
``intelligence officials'' and the like) and can therefore also be used
for the dissemination of propaganda. In any case, the widespread
orientation towards the major Anglo-Saxon publications leads to a
further convergence in the geopolitical coverage in our media.

The following figure shows some examples of such citation based on the
Syria coverage of the largest daily newspaper in Switzerland,
Tages-Anzeiger. The articles are all from the first days of October
2015, when Russia for the first time intervened directly in the Syrian
war (US/UK sources are highlighted):

\includegraphics{https://swprs.files.wordpress.com/2017/01/us-medien.png?w=620\&h=616}

\emph{Frequent citation of major British and US media, exemplified by
the Syria war coverage of Swiss daily newspaper Tages-Anzeiger in
October 2015.}

\hypertarget{the-desired-narrative}{%
\paragraph{The desired narrative}\label{the-desired-narrative}}

But why do journalists in our media not simply try to research and
report independently of the global agencies and the Anglo-Saxon media?
Middle East correspondent Luyendijk describes his experiences:

``You might suggest that I should have looked for sources I could trust.
I did try, but whenever I wanted to write a story without using news
agencies, the main Anglo-Saxon media, or talking heads, it fell apart.
() Obviously I, as a correspondent, could tell very different stories
about one and the same situation. But the media could only present one
of them, and often enough, that was exactly the story that confirmed the
prevailing image.'' (Luyendijk p.54ff)

Media researcher Noam Chomsky has described this effect in his essay
\href{https://chomsky.info/199710__/}{``What makes the mainstream media
mainstream''} as follows: ``If you get off line, if you're producing
stories that the big press doesn't like, you'll hear about it pretty
soon. () So there are a lot of ways in which power plays can drive you
right back into line if you move out. If you try to break the mold,
you're not going to last long. That framework works pretty well, and it
is understandable that it is just a reflection of obvious power
structures.'' (Chomsky 1997)

Nevertheless, some of the leading journalists continue to believe that
nobody can tell them what to write. How does this add up? Media
researcher Chomsky \href{https://chomsky.info/199710__/}{clarifies} the
apparent contradiction:

``{[}T{]}he point is that they wouldn't be there unless they had already
demonstrated that nobody has to tell them what to write because they are
going say the right thing. If they had started off at the Metro desk, or
something, and had pursued the wrong kind of stories, they never would
have made it to the positions where they can now say anything they like.
The same is mostly true of university faculty in the more ideological
disciplines. They have been through the socialization system.'' (Chomsky
1997)

Ultimately, this ``socialization system'' leads to a journalism that no
longer independently researches and critically reports on geopolitical
conflicts (and some other topics), but seeks to consolidate the desired
narrative through appropriate editorials, commentary, and interviews.

\hypertarget{conclusion-the-first-law-of-journalism}{%
\paragraph{Conclusion: The ``First Law of
Journalism''}\label{conclusion-the-first-law-of-journalism}}

Former AP journalist Herbert Altschull called it the First Law of
Journalism: ``In all press systems, the news media are instruments of
those who exercise political and economic power. Newspapers,
periodicals, radio and television stations do not act independently,
although they have the possibility of independent exercise of power.''
(Altschull 1984/1995, p. 298)

In that sense, it is logical that our traditional media -- which are
predominantly financed by advertising or the state -- represent the
geopolitical interests of the transatlantic alliance, given that both
the advertising corporations as well as the states themselves are
dependent on the transatlantic economic and security architecture led by
the United States.

In addition, the key people of our leading media are -- in the spirit of
Chomsky's ``socialization system'' --~ often themselves part of
transatlantic elite networks. Some of the most important institutions in
this regard include the US Council on Foreign Relations (CFR), the
Bilderberg Group, and the Trilateral Commission, all of which feature
many prominent journalists (see
\href{https://swprs.org/the-american-empire-and-its-media/}{in-depth
study of these groups}).

Most well-known publications, therefore, may indeed be seen as a kind of
``establishment media''. This is because, in the past, the freedom of
the press was rather theoretical, given significant entry barriers such
as broadcasting licenses, frequency slots, requirements for financing
and technical infrastructure, limited sales channels, dependence on
advertising, and other restrictions.

It was only due to the Internet that Altschull's First Law has been
broken to some extent. Thus, in recent years a high-quality,
reader-funded journalism \href{https://swprs.org/media-navigator/}{has
emerged}, often outperforming traditional media in terms of critical
reporting. Some of these ``alternative'' publications already reach a
very large audience, showing that the ``mass'' does not have to be a
problem for the quality of a media outlet.

Nevertheless, up to now the traditional media has been able to attract a
solid majority of online visitors, too. This, in turn, is closely linked
to the hidden role of news agencies, whose up-to-the-minute reports form
the backbone of most online news sites.

Will ``political and economic power'', according to Altschull's Law,
retain control over the news, or will ``uncontrolled news'' change the
political and economic power structure? The coming years will show.

\href{https://swprs.org/contact/}{Read more from SPR →}\\
Share this study on:
\href{https://twitter.com/intent/tweet?url=https://swprs.org/the-propaganda-multiplier/}{Twitter}
/
\href{https://www.facebook.com/share.php?u=https://swprs.org/the-propaganda-multiplier/}{Facebook}

\begin{center}\rule{0.5\linewidth}{\linethickness}\end{center}

\hypertarget{case-study-syria-war-coverage}{%
\subsubsection{Case study: Syria war
coverage}\label{case-study-syria-war-coverage}}

As part of a case study, the Syria war coverage of nine leading daily
newspapers from Germany, Austria and Switzerland were examined for
plurality of viewpoints and reliance on news agencies. The following
newspapers were selected:

\begin{itemize}
\tightlist
\item
  \textbf{For Germany:} Die Welt, Süddeutsche Zeitung (SZ), and
  Frankfurter Allgemeine Zeitung (FAZ)
\item
  \textbf{For Switzerland:} Neue Zürcher Zeitung (NZZ), Tagesanzeiger
  (TA), and Basler Zeitung (BaZ)
\item
  \textbf{For Austria:} Standard, Kurier, and Die Presse
\end{itemize}

The investigation period was defined as October 1 to 15, 2015, i.e. the
first two weeks after Russia's direct intervention in the Syrian
conflict. The entire print and online coverage of these newspapers was
taken into account. Any Sunday editions were not taken into account, as
not all of the newspapers examined have such. In total, 381 newspaper
articles met the stated criteria.

In a first step, the articles were classified according to their
properties into the following groups:

\begin{enumerate}
\def\labelenumi{\arabic{enumi}.}
\tightlist
\item
  \textbf{Agencies}: Reports from news agencies (with agency code)
\item
  \textbf{Mixed}: Simple reports (with author names) that are based in
  whole or in part on agency reports
\item
  \textbf{Reports}: Editorial background reports and analyses
\item
  \textbf{Opinions/Comments}: Opinions and guest comments
\item
  \textbf{Interviews}: Interviews with experts, politicians etc.
\item
  \textbf{Investigative}: Investigative research that reveals new
  information or context
\end{enumerate}

The following \textbf{Figure 1} shows the composition of the articles
for the nine newspapers analyzed in total. As can be seen, 55\% of
articles were news agency reports; 23\% editorial reports based on
agency material; 9\% background reports; 10\% opinions and guest
comments; 2\% interviews; and 0\% based on investigative research.

\includegraphics{https://swprs.files.wordpress.com/2019/02/figure1.png?w=650}

\emph{Figure 1: Types of articles (total; n=381)}

The pure agency texts -- from short notices to the detailed reports --
were mostly on the Internet pages of the daily newspapers: on the one
hand, the pressure for breaking news is higher than in the printed
edition, on the other hand, there are no space restrictions. Most other
types of articles were found in both the online and printed editions;
some exclusive interviews and background reports were found only in the
printed editions. All items were collected only once for the
investigation.

The following \textbf{Figure 2} shows the same classification on a per
newspaper basis. During the observation period (two weeks), most
newspapers published between 40 and 50 articles on the Syrian conflict
(print and online). In the German newspaper \emph{Die Welt} there were
more (58), in the \emph{Basler Zeitung} and the Austrian \emph{Kurier},
however, significantly less (29 or 33).

Depending on which newspaper, the share of agency reports is almost 50\%
(Welt, Süddeutsche, NZZ, Basler Zeitung), just under 60\% (FAZ,
Tagesanzeiger), and 60 to 70\% (Presse, Standard, Kurier). Together with
the agency-based reports, the proportion in most newspapers is between
approx. 70\% and 80\%. These proportions are consistent with previous
media studies (e.g., Blum 1995, Johnston 2011, MacGregor 2013, Paterson
2007).

In the background reports, the Swiss newspapers were leading (five to
six pieces), followed by \emph{Welt}, \emph{Süddeutsche} and
\emph{Standard} (four each) and the other newspapers (one to three). The
background reports and analyzes were in particular devoted to the
situation and development in the Middle East, as well as to the motives
and interests of individual actors (for example Russia, Turkey, the
Islamic State).

However, most of the commentaries were to be found in the German
newspapers (seven comments each), followed by \emph{Standard} (five),
\emph{NZZ} and \emph{Tagesanzeiger} (four each). \emph{Basler Zeitung}
did not publish any commentaries during the observation period, but two
interviews. Other interviews were conducted by \emph{Standard} (three)
and \emph{Kurier} and \emph{Presse} (one each). Investigative research,
however, could not be found in any of the newspapers.

In particular, in the case of the three German newspapers, a
journalistically problematic blending of opinion pieces and reports was
noted. Reports contained strong expressions of opinion even though they
were not marked as commentary. The present study was in any case based
on the article labeling by the newspaper.

\includegraphics{https://swprs.files.wordpress.com/2019/03/figure2c.png?w=650}

*Figure 2: Types of articles per newspaper\\
*

The following \textbf{Figure 3} shows the breakdown of agency stories
(by agency abbreviation) for each news agency, in total and per country.
The 211 agency reports carried a total of 277 agency codes (a story may
consist of material from more than one agency). In total, 24\% of agency
reports came from the AFP; about 20\% each by the DPA, APA and Reuters;
9\% of the SDA; 6\% of the AP; and 11\% were unknown (no labeling or
blanket term ``agencies'').

In Germany, the DPA, AFP and Reuters each have a share of about one
third of the news stories. In Switzerland, the SDA and the AFP are in
the lead, and in Austria, the APA and Reuters.

In fact, the shares of the global agencies AFP, AP and Reuters are
likely to be even higher, as the Swiss SDA and the Austrian APA obtain
their international reports mainly from the global agencies and the
German DPA cooperates closely with the American AP.

It should also be noted that, for historical reasons, the global
agencies are represented differently in different regions of the world.
For events in Asia, Ukraine or Africa, the share of each agency will
therefore be different than from events in the Middle East.

\includegraphics{https://swprs.files.wordpress.com/2019/02/figure3.png?w=650}

\emph{Figure 3: Share of news agencies, total (n=277) and per country}

In the next step, central statements were used to rate the orientation
of editorial opinions (28), guest comments (10) and interview partners
(7) (a total of 45 articles). As \textbf{Figure 4} shows, 82\% of the
contributions were generally US/NATO friendly, 16\% neutral or balanced,
and 2\% predominantly US/NATO critical.

The only predominantly US/NATO-critical contribution was an op-ed in the
Austrian \emph{Standard} on October 2, 2015, titled: ``The strategy of
regime change has failed. A distinction between `good' and `bad'
terrorist groups in Syria makes the Western policy untrustworthy.''

\includegraphics{https://swprs.files.wordpress.com/2019/02/figure4.png?w=650}

\emph{Figure 4: Orientation of editorial opinions, guest comments, and
interviewees (total; n=45).}

The following \textbf{Figure 5}~ shows the orientation of the
contributions, guest comments and interviewees, in turn broken down by
individual newspapers. As can be seen, \emph{Welt, Süddeutsche Zeitung,
NZZ, Zürcher Tagesanzeiger} and the Austrian newspaper \emph{Kurier}
presented exclusively US/NATO-friendly opinion and guest contributions;
this goes for \emph{FAZ} too, with the exception of one neutral/balanced
contribution. \emph{The Standard} brought four US/NATO friendly, three
balanced/neutral, as well as the already mentioned US/NATO critical
opinion contributions.

\emph{Presse} was the only one of the examined newspapers to
predominantly publish neutral/balanced opinions and guest contributions.
The \emph{Basler Zeitung} published one US/NATO-friendly and one
balanced contribution. Shortly after the observation period (October 16,
2015), \emph{Basler Zeitung} also published an interview with the
President of the Russian Parliament. This would of course have been
counted as a contribution critical of the US/NATO.

\includegraphics{https://swprs.files.wordpress.com/2019/02/figure5.png?w=650}

\emph{Figure 5: Basic orientation of opinion pieces and interviewees per
newspaper}

In a further analysis, a full-text keyword search for ``propaganda''
(and word combinations thereof) was used to investigate in which cases
the newspapers themselves identified propaganda in one of the two
geopolitical conflict sides, USA/NATO or Russia (the participant
``IS/ISIS'' was not considered). In total, twenty such cases were
identified. \textbf{Figure 6} shows the result: in 85\% of the cases,
propaganda was identified on the Russian side of the conflict, in 15\%
the identification was neutral or unstated, and in 0\% of the cases
propaganda was identified on the USA/NATO side of the conflict.

It should be noted that about half of the cases (nine) were in the Swiss
\emph{NZZ}, which spoke of Russian propaganda quite frequently
(``Kremlin propaganda'', ``Moscow propaganda machine'', ``propaganda
stories'', ``Russian propaganda apparatus'' etc.), followed by German
\emph{FAZ} (three), \emph{Welt} and \emph{Süddeutsche Zeitung} (two
each) and the Austrian newspaper \emph{Kurier} (one). The other
newspapers did not mention propaganda, or only in a neutral context (or
in the context of IS).

\includegraphics{https://swprs.files.wordpress.com/2019/02/figure6.png?w=650}

\emph{Figure 6: Attribution of propaganda to conflict parties (total;
n=20).}

\hypertarget{conclusion}{%
\paragraph{Conclusion}\label{conclusion}}

In this case study, the geopolitical coverage in nine leading European
newspapers was examined for diversity and journalistic performance using
the example of the Syrian war.

The results confirm the high dependence on the global news agencies (63
to 90\%, excluding commentaries and interviews) and the lack of own
investigative research, as well as the rather biased commenting on
events in favor of the US/NATO side (82\% positive; 2\% negative), whose
stories were not checked by the newspapers for any propaganda.

*About the authors: Swiss Propaganda Research (SPR) is an independent
research group investigating geopolitical propaganda in Swiss and
international media. You can contact us
\href{https://swprs.org/contact/}{here}.\\
*

\emph{English translation provided by Terje Maloy.}

\begin{center}\rule{0.5\linewidth}{\linethickness}\end{center}

\hypertarget{literature}{%
\subsubsection{Literature}\label{literature}}

Altschull, Herbert J. (1984/1995): Agents of power. The media and public
policy. \emph{Longman,} New York.

Becker, Jörg (2015): Medien im Krieg -- Krieg in den Medien.
\emph{Springer Verlag für Sozialwissenschaften,} Wiesbaden.

Blum, Roger et al. (Hrsg.) (1995): Die AktualiTäter.
Nachrichtenagenturen in der Schweiz. \emph{Verlag Paul Haupt,} Bern.

Chomsky, Noam (1997): What Makes Mainstream Media Mainstream. \emph{Z
Magazine,} MA. (\href{https://chomsky.info/199710__/}{PDF})

Forschungsinstitut für Öffentlichkeit und Gesellschaft der Universität
Zürich (FOEG) (2011): Jahrbuch Qualität der Medien, Ausgabe 2011.
\emph{Schwabe,} Basel.

Gritsch, Kurt (2010): Inszenierung eines gerechten Krieges?
Intellektuelle, Medien und der ``Kosovo-Krieg'' 1999. \emph{Georg Olms
Verlag,} Hildesheim.

Hird, Christopher (1985): Standard Techniques. \emph{Diverse Reports,
Channel 4 TV.} 30. Oktober 1985.
(\href{https://swprs.org/video-the-cia-and-the-media/}{Link})

Höhne, Hansjoachim (1977): Report über Nachrichtenagenturen. Band 1: Die
Situation auf den Nachrichtenmärkten der Welt. Band 2: Die Geschichte
der Nachricht und ihrer Verbreiter. \emph{Nomos Verlagsgesellschaft,}
Baden-Baden.

Johnston, Jane \& Forde, Susan (2011): The Silent Partner: News Agencies
and 21st Century News. \emph{International Journal of Communication 5
(2011),} p. 195--214.
(\href{https://ijoc.org/index.php/ijoc/article/view/928}{PDF})

Krüger, Uwe (2013): Meinungsmacht. Der Einfluss von Eliten auf
Leitmedien und Alpha-Journalisten -- eine kritische Netzwerkanalyse.
\emph{Herbert von Halem Verlag,} Köln.

Luyendijk, Joris (2015): Von Bildern und Lügen in Zeiten des Krieges:
Aus dem Leben eines Kriegsberichterstatters -- Aktualisierte Neuausgabe.
\emph{Tropen,} Stuttgart.

MacGregor, Phil (2013): International News Agencies. Global eyes that
never blink. In: Fowler-Watt/Allan (ed.): Journalism: New Challenges.
\emph{Centre for Journalism \& Communication Research,} Bournemouth
University.
(\href{https://microsites.bournemouth.ac.uk/cjcr/files/2013/10/JNC-2013-Chapter-3-MacGregor.pdf}{PDF})

Mükke, Lutz (2014): Korrespondenten im Kalten Krieg. Zwischen Propaganda
und Selbstbehauptung. \emph{Herbert von Halem Verlag,} Köln.

Paterson, Chris (2007): International news on the internet. \emph{The
International Journal of Communication Ethics.} Vol 4, No 1/2 2007.
(\href{http://www.communicationethics.net/journal/v4n1-2/v4n1-2_12.pdf}{PDF})

Queval, Jean (1945): Première page, Cinquième colonne. \emph{Arthème
Fayard,} Paris.

Schulten-Jaspers, Yasmin (2013): Zukunft der Nachrichtenagenturen.
Situation, Entwicklung, Prognosen. \emph{Nomos,} Baden-Baden.

Segbers, Michael (2007): Die Ware Nachricht. Wie Nachrichtenagenturen
ticken. \emph{UVK,} Konstanz.

Steffens, Manfred {[}Ziegler, Stefan{]} (1969): Das Geschäft mit der
Nachricht. Agenturen, Redaktionen, Journalisten. \emph{Hoffmann und
Campe}, Hamburg.

Tilgner, Ulrich (2003): Der inszenierte Krieg -- Täuschung und Wahrheit
beim Sturz Saddam Husseins. \emph{Rowohlt}, Reinbek.

Wilke, Jürgen (Hrsg.) (2000): Von der Agentur zur Redaktion.
\emph{Böhlau}, Köln.

\begin{center}\rule{0.5\linewidth}{\linethickness}\end{center}

Share this study on:
\href{https://twitter.com/intent/tweet?url=https://swprs.org/the-propaganda-multiplier/}{Twitter}
/
\href{https://www.facebook.com/share.php?u=https://swprs.org/the-propaganda-multiplier/}{Facebook}\\
Published (EN): March 2019

\hypertarget{swiss-policy-research}{%
\subsubsection{Swiss Policy Research}\label{swiss-policy-research}}

\begin{itemize}
\tightlist
\item
  \href{https://swprs.org/kontakt/}{Kontakt}
\item
  \href{https://swprs.org/uebersicht/}{Übersicht}
\item
  \href{https://swprs.org/donationen/}{Donationen}
\item
  \href{https://swprs.org/disclaimer/}{Disclaimer}
\end{itemize}

\hypertarget{english}{%
\subsubsection{English}\label{english}}

\begin{itemize}
\tightlist
\item
  \href{https://swprs.org/contact/}{About Us / Contact}
\item
  \href{https://swprs.org/media-navigator/}{The Media Navigator}
\item
  \href{https://swprs.org/the-american-empire-and-its-media/}{The CFR
  and the Media}
\item
  \href{https://swprs.org/donations/}{Donations}
\end{itemize}

\hypertarget{follow-by-email}{%
\subsubsection{Follow by email}\label{follow-by-email}}

Follow

\href{https://wordpress.com/?ref=footer_custom_com}{WordPress.com}.

\protect\hyperlink{}{Up ↑}

Post to

\protect\hyperlink{}{Cancel}

\includegraphics{https://pixel.wp.com/b.gif?v=noscript}
