\protect\hyperlink{content}{Skip to content}

\href{https://swprs.org/}{}

\protect\hyperlink{search-container}{Search}

Search for:

\href{https://swprs.org/}{\includegraphics{https://swprs.files.wordpress.com/2020/05/swiss-policy-research-logo-300.png}}

\href{https://swprs.org/}{Swiss Policy Research}

Geopolitics and Media

Menu

\begin{itemize}
\tightlist
\item
  \href{https://swprs.org}{Start}
\item
  \href{https://swprs.org/srf-propaganda-analyse/}{Studien}

  \begin{itemize}
  \tightlist
  \item
    \href{https://swprs.org/srf-propaganda-analyse/}{SRF / ZDF}
  \item
    \href{https://swprs.org/die-nzz-studie/}{NZZ-Studie}
  \item
    \href{https://swprs.org/der-propaganda-multiplikator/}{Agenturen}
  \item
    \href{https://swprs.org/die-propaganda-matrix/}{Medienmatrix}
  \end{itemize}
\item
  \href{https://swprs.org/medien-navigator/}{Analysen}

  \begin{itemize}
  \tightlist
  \item
    \href{https://swprs.org/medien-navigator/}{Navigator}
  \item
    \href{https://swprs.org/der-propaganda-schluessel/}{Techniken}
  \item
    \href{https://swprs.org/propaganda-in-der-wikipedia/}{Wikipedia}
  \item
    \href{https://swprs.org/logik-imperialer-kriege/}{Kriege}
  \end{itemize}
\item
  \href{https://swprs.org/netzwerk-medien-schweiz/}{Netzwerke}

  \begin{itemize}
  \tightlist
  \item
    \href{https://swprs.org/netzwerk-medien-schweiz/}{Schweiz}
  \item
    \href{https://swprs.org/netzwerk-medien-deutschland/}{Deutschland}
  \item
    \href{https://swprs.org/medien-in-oesterreich/}{Österreich}
  \item
    \href{https://swprs.org/das-american-empire-und-seine-medien/}{USA}
  \end{itemize}
\item
  \href{https://swprs.org/bericht-eines-journalisten/}{Fokus I}

  \begin{itemize}
  \tightlist
  \item
    \href{https://swprs.org/bericht-eines-journalisten/}{Journalistenbericht}
  \item
    \href{https://swprs.org/russische-propaganda/}{Russische Propaganda}
  \item
    \href{https://swprs.org/die-israel-lobby-fakten-und-mythen/}{Die
    »Israel-Lobby«}
  \item
    \href{https://swprs.org/geopolitik-und-paedokriminalitaet/}{Pädokriminalität}
  \end{itemize}
\item
  \href{https://swprs.org/migration-und-medien/}{Fokus II}

  \begin{itemize}
  \tightlist
  \item
    \href{https://swprs.org/covid-19-hinweis-ii/}{Coronavirus}
  \item
    \href{https://swprs.org/die-integrity-initiative/}{Integrity
    Initiative}
  \item
    \href{https://swprs.org/migration-und-medien/}{Migration \& Medien}
  \item
    \href{https://swprs.org/der-fall-magnitsky/}{Magnitsky Act}
  \end{itemize}
\item
  \href{https://swprs.org/kontakt/}{Projekt}

  \begin{itemize}
  \tightlist
  \item
    \href{https://swprs.org/kontakt/}{Kontakt}
  \item
    \href{https://swprs.org/uebersicht/}{Seitenübersicht}
  \item
    \href{https://swprs.org/medienspiegel/}{Medienspiegel}
  \item
    \href{https://swprs.org/donationen/}{Donationen}
  \end{itemize}
\item
  \href{https://swprs.org/contact/}{English}
\end{itemize}

\protect\hyperlink{}{Open Search}

\hypertarget{der-absturz-swissair-111}{%
\section{Der Absturz: Swissair~111}\label{der-absturz-swissair-111}}

\includegraphics{https://swprs.files.wordpress.com/2018/12/swissair-111.png?w=736}

\textbf{Publiziert}: Dezember 2018\\
\textbf{Sprachen}:
\href{https://swprs.org/der-absturz-swissair-111/}{Deutsch},
\href{https://swprs.org/the-crash-of-swissair-111/}{Englisch}

Es ist die größte Katastrophe der Schweizer Luftfahrtgeschichte: Am 2.
September 1998 stürzte der
\href{https://en.wikipedia.org/wiki/Swissair_Flight_111}{Swissair-Flug
111} von New York nach Genf mit 229 Menschen an Bord beim kanadischen
Halifax in den Atlantik. Die Absturzursache wurde bis heute nicht
überzeugend aufgeklärt. Doch für Schweizer Medien ist der Fall ein Tabu.

Die kanadische Transportsicherheitsbehörde TSB kam in ihrem
\href{http://www.tsb.gc.ca/eng/rapports-reports/aviation/1998/a98h0003/a98h0003.pdf}{Untersuchungsbericht}
von 2003 zum Ergebnis, dass vermutlich ein Kurzschluss in der
Verkabelung des Unterhaltungssystems zu einem Brand im Cockpit geführt
habe, der das Flugzeug binnen weniger Minuten abstürzen ließ.

Doch 2011 trat ein an der Untersuchung beteiligter Forensiker der
kanadischen Bundespolizei RCMP als Whistleblower an die Öffentlichkeit
und präsentierte neue Fakten, die den Untersuchungsbericht infrage
stellten und Sabotage bzw. einen Anschlag als Absturzursache vermuten
ließen. Sechs Jahre später publizierte er seine Erkenntnisse sowie
Original­dokumente in \href{http://www.swissair111.ca/}{Buchform}.

Die folgende Auflistung nennt seine Hauptaussagen erstmals in deutscher
Sprache:

\begin{enumerate}
\def\labelenumi{\arabic{enumi}.}
\tightlist
\item
  Bereits nach wenigen Tagen und trotz erster Hinweise auf ein mögliches
  Verbrechen wurde von der Untersuchungsleitung entschieden, den Absturz
  ausschließlich als Unfall zu behandeln und keine kriminalistische
  Untersuchung einzuleiten.
\item
  Ein externer Metallexperte, der mit der Untersuchung der Brandstelle
  beauftragt wurde, fand ungewöhnlich hohe Magnesiumrückstände, die auf
  die Verwendung eines Hoch­temperatur-Brand­satzes aus
  \href{https://en.wikipedia.org/wiki/Thermite}{Thermit} hindeuteten.
  Der Experte musste sein Gutachten jedoch auf Geheiß der
  Unter­suchungs­leitung mehrmals umschreiben, bis alle Hinweise auf
  einen möglichen Brandsatz entfernt waren. Im Untersuchungsbericht
  wurde das Magnesium nicht mehr erwähnt.
\item
  Der Brand-Experte der amerikanischen Luftfahrtbehörde FAA sowie
  weitere Fachleute vermuteten aufgrund des Schadensbildes ebenfalls die
  Verwendung eines Brandsatzes.
\item
  Testversuche, die belegen sollten, dass allein das brennende
  Isoliermaterial den Aluminium­rahmen des Flugzeuges schmelzen konnte,
  mussten auf Anweisung der Untersuchungs­leitung solange angepasst
  werden, bis gänzlich unrealistische Bedingungen erreicht waren.
\item
  Ein geschmolzenes Metallstück aus dem Flugzeug wurde während der
  Untersuchung entsorgt.
\item
  Die aufgezeichneten Stimmdaten deuteten nicht auf einen Ausfall des
  Unterhaltungssystems hin, wie er bei einem Kurzschluss zu erwarten
  wäre. Die Untersuchung der Verkabelung deutete vielmehr darauf hin,
  dass die Fehlstellen eine Folge und nicht die Ursache des Feuers
  waren.
\item
  Die Überprüfung der Flugzeugwartung ergab, dass bei der Reinigung und
  Abfertigung der Maschine am Flughafen in New York ein neuer
  Mitarbeiter unter nachweislich falscher Identität anwesend war, der
  danach nicht mehr zur Arbeit erschien.
\item
  Gemäß Frachtdokumenten sollte das Flugzeug Diamanten im
  Versicherungswert von 300 Millionen Dollar transportieren, doch bei
  der Bergung des Wracks waren sie
  \href{https://www.cbc.ca/news/canada/the-mystery-of-swissair-flight-111-s-diamond-cargo-1.1051965}{unauffindbar}.
\item
  Mangels Kriminaluntersuchung wurde kein ausführliches Profiling der
  Passagiere durchgeführt, obschon einige
  \href{https://www.cbc.ca/news/canada/swissair-crash-may-not-have-been-an-accident-ex-rcmp-1.1019738}{exponierte}
  Personen auf dem Flug von New York nach Genf an Bord waren.
\item
  Der Whistleblower wurde während der Untersuchung von seinen
  Vorgesetzten aufgefordert, seine polizeilichen Untersuchungsprotokolle
  zu manipulieren.
\end{enumerate}

Das kanadische Fernsehen CBC und das Schweizer Fernsehen SRF
produzierten 2011 einen gemeinsamen
\href{https://www.youtube.com/watch?v=s9rVKWsMv_g}{Dokumentarfilm}, in
dem der Whistleblower und der Metallexperte erstmals zu Wort kamen. Der
zuständige SRF-Journalist sprach im Vorfeld zudem mit zwei renommierten
europäischen Flugunfall-Experten, die den offiziellen
Untersuchungsbericht ebenfalls kritisierten.

Dennoch
\href{http://www.news.ch/SRF+strahlt+Halifax+Doku+nicht+aus/508488/detail.htm}{entschied}
sich das Schweizer Fernsehen überraschend gegen die Ausstrahlung der
Doku: Man wolle keine Spekulationen verbreiten, so der damalige
Chefredakteur. Andere Schweizer Medien versuchten den kanadischen
Ermittler als
\href{https://www.nzz.ch/ein_einzelkaempfer_will_keine_ruhe_geben-1.12534482}{»Einzelkämpfer«}
oder
\href{https://www.tagesanzeiger.ch/kultur/fernsehen/verschwoererische-doku-ueber-swissairtodesflug/story/17320616}{»Verschwörungs­theoretiker«}
darzustellen. Die damalige Sprecherin der Swissair, die Einblick in die
umfangreichen Dokumente des Whistleblowers hatte, zeigte sich hingegen
\href{http://www.news.ch/SRF+strahlt+Halifax+Doku+nicht+aus/508488/detail.htm}{schockiert}:
»Was er sagte, war unantastbar.«

Der Grund für das bis heute anhaltende Desinteresse der Schweizer Medien
ist nicht offensichtlich. Die Jahre von 1996 bis 1999 waren für die
Schweizer Außenpolitik eine
\href{https://interaktiv.bernerzeitung.ch/2018/schweizer-privatagenten/}{besonders
schwierige Zeit}. Auch der --
\href{https://en.wikipedia.org/wiki/Luxor_massacre}{ebenfalls
unaufgeklärte} -- Luxor-Anschlag vom November 1997 auf eine
hauptsächlich schweizerische Reisegruppe fiel in diese Phase.

Und drei Jahre nach Halifax ereigneten sich die Anschläge vom 11.
September 2001, bei denen, laut Kritikern der offiziellen Darstellung,
ebenfalls Thermit zum Einsatz
\href{https://en.wikipedia.org/wiki/World_Trade_Center_controlled_demolition_conspiracy_theories}{gekommen}
sein könnte.

\emph{Swissair 111: The Untold Story (CBC, 2011, 40 Minuten)}

\hypertarget{weiterfuxfchrende-literatur}{%
\paragraph{Weiterführende Literatur}\label{weiterfuxfchrende-literatur}}

\begin{itemize}
\tightlist
\item
  Die \href{http://www.swissair111.ca/}{Website} des kanadischen
  Forensikers mit zahlreichen Originaldokumenten
\item
  Sein
  \href{https://www.amazon.com/Twice-Far-SwissAir-Airplane-Investigation/dp/1540879593/}{Buch},
  \emph{Twice as Far: The Swissair 111 Airplane Crash Investigation
  (2017)}
\item
  NZZ-Beitrag
  \href{https://www.nzz.ch/schweiz/todesflug-sr-111-ld.1416442}{Todesflug
  SR 111} vom

  \begin{enumerate}
  \def\labelenumi{\arabic{enumi}.}
  \tightlist
  \item
    September 2018
  \end{enumerate}
\end{itemize}

\begin{center}\rule{0.5\linewidth}{\linethickness}\end{center}

Beitrag teilen auf:
\href{https://twitter.com/intent/tweet?url=https\%3A\%2F\%2Fswprs.org/der-absturz-swissair-111/}{Twitter}
/
\href{https://www.facebook.com/share.php?u=https://swprs.org/der-absturz-swissair-111/}{Facebook}

Publiziert: Dezember 2018

\hypertarget{swiss-policy-research}{%
\subsubsection{Swiss Policy Research}\label{swiss-policy-research}}

\begin{itemize}
\tightlist
\item
  \href{https://swprs.org/kontakt/}{Kontakt}
\item
  \href{https://swprs.org/uebersicht/}{Übersicht}
\item
  \href{https://swprs.org/donationen/}{Donationen}
\item
  \href{https://swprs.org/disclaimer/}{Disclaimer}
\end{itemize}

\hypertarget{english}{%
\subsubsection{English}\label{english}}

\begin{itemize}
\tightlist
\item
  \href{https://swprs.org/contact/}{About Us / Contact}
\item
  \href{https://swprs.org/media-navigator/}{The Media Navigator}
\item
  \href{https://swprs.org/the-american-empire-and-its-media/}{The CFR
  and the Media}
\item
  \href{https://swprs.org/donations/}{Donations}
\end{itemize}

\hypertarget{follow-by-email}{%
\subsubsection{Follow by email}\label{follow-by-email}}

Follow

\href{https://wordpress.com/?ref=footer_custom_com}{WordPress.com}.

\protect\hyperlink{}{Up ↑}

Post to

\protect\hyperlink{}{Cancel}

\includegraphics{https://pixel.wp.com/b.gif?v=noscript}
