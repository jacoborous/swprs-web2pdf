\protect\hyperlink{content}{Skip to content}

\href{https://swprs.org/}{}

\protect\hyperlink{search-container}{Search}

Search for:

\href{https://swprs.org/}{\includegraphics{https://swprs.files.wordpress.com/2020/05/swiss-policy-research-logo-300.png}}

\href{https://swprs.org/}{Swiss Policy Research}

Geopolitics and Media

Menu

\begin{itemize}
\tightlist
\item
  \href{https://swprs.org}{Start}
\item
  \href{https://swprs.org/srf-propaganda-analyse/}{Studien}

  \begin{itemize}
  \tightlist
  \item
    \href{https://swprs.org/srf-propaganda-analyse/}{SRF / ZDF}
  \item
    \href{https://swprs.org/die-nzz-studie/}{NZZ-Studie}
  \item
    \href{https://swprs.org/der-propaganda-multiplikator/}{Agenturen}
  \item
    \href{https://swprs.org/die-propaganda-matrix/}{Medienmatrix}
  \end{itemize}
\item
  \href{https://swprs.org/medien-navigator/}{Analysen}

  \begin{itemize}
  \tightlist
  \item
    \href{https://swprs.org/medien-navigator/}{Navigator}
  \item
    \href{https://swprs.org/der-propaganda-schluessel/}{Techniken}
  \item
    \href{https://swprs.org/propaganda-in-der-wikipedia/}{Wikipedia}
  \item
    \href{https://swprs.org/logik-imperialer-kriege/}{Kriege}
  \end{itemize}
\item
  \href{https://swprs.org/netzwerk-medien-schweiz/}{Netzwerke}

  \begin{itemize}
  \tightlist
  \item
    \href{https://swprs.org/netzwerk-medien-schweiz/}{Schweiz}
  \item
    \href{https://swprs.org/netzwerk-medien-deutschland/}{Deutschland}
  \item
    \href{https://swprs.org/medien-in-oesterreich/}{Österreich}
  \item
    \href{https://swprs.org/das-american-empire-und-seine-medien/}{USA}
  \end{itemize}
\item
  \href{https://swprs.org/bericht-eines-journalisten/}{Fokus I}

  \begin{itemize}
  \tightlist
  \item
    \href{https://swprs.org/bericht-eines-journalisten/}{Journalistenbericht}
  \item
    \href{https://swprs.org/russische-propaganda/}{Russische Propaganda}
  \item
    \href{https://swprs.org/die-israel-lobby-fakten-und-mythen/}{Die
    »Israel-Lobby«}
  \item
    \href{https://swprs.org/geopolitik-und-paedokriminalitaet/}{Pädokriminalität}
  \end{itemize}
\item
  \href{https://swprs.org/migration-und-medien/}{Fokus II}

  \begin{itemize}
  \tightlist
  \item
    \href{https://swprs.org/covid-19-hinweis-ii/}{Coronavirus}
  \item
    \href{https://swprs.org/die-integrity-initiative/}{Integrity
    Initiative}
  \item
    \href{https://swprs.org/migration-und-medien/}{Migration \& Medien}
  \item
    \href{https://swprs.org/der-fall-magnitsky/}{Magnitsky Act}
  \end{itemize}
\item
  \href{https://swprs.org/kontakt/}{Projekt}

  \begin{itemize}
  \tightlist
  \item
    \href{https://swprs.org/kontakt/}{Kontakt}
  \item
    \href{https://swprs.org/uebersicht/}{Seitenübersicht}
  \item
    \href{https://swprs.org/medienspiegel/}{Medienspiegel}
  \item
    \href{https://swprs.org/donationen/}{Donationen}
  \end{itemize}
\item
  \href{https://swprs.org/contact/}{English}
\end{itemize}

\protect\hyperlink{}{Open Search}

\hypertarget{fakten-zu-covid-19-archiv}{%
\section{Fakten zu Covid-19~(Archiv)}\label{fakten-zu-covid-19-archiv}}

\hypertarget{zur-hauptseite-fakten-zu-covid-19}{%
\paragraph{\texorpdfstring{\href{https://swprs.org/covid-19-hinweis-ii/}{Zur
Hauptseite: Fakten zu
Covid-19}}{Zur Hauptseite: Fakten zu Covid-19}}\label{zur-hauptseite-fakten-zu-covid-19}}

\hypertarget{mai-2020}{%
\paragraph{Mai 2020}\label{mai-2020}}

\hypertarget{interviews-mit-experten}{%
\subparagraph{**Interviews mit Experten}\label{interviews-mit-experten}}

**

\begin{itemize}
\tightlist
\item
  Stanford-Professor \textbf{John Ioannidis} erklärt
  in\href{https://twitter.com/cnn/status/1256579248342564865}{einem
  Interview mit CNN}, dass Covid19 eine ``verbreitete und milde
  Erkrankung'' sei, die für die Allgemeinbevölkerung gleich gefährlich
  oder sogar weniger gefährlich als die Influenza (Grippe) sei. Zu
  schützen seien insbesondere Patienten in Pflegeheimen und
  Krankenhäusern.
\item
  Stanford-Professor \textbf{Dr. Scott Atlas} erklärt in
  \href{https://www.facebook.com/cnn/posts/10160799274796509}{einem
  Interview mit CNN}, dass man ``durch die falsche Idee, Covid19 stoppen
  zu müssen, eine katastrophale Situation im Gesundheitsbereich
  geschaffen'' habe. Es seien irrationale Ängste erzeugt worden, denn
  die Erkrankung sei ``insgesamt mild''. Deshalb gebe es auch ``absolut
  keinen Grund'' für umfangreiche Testungen in der Allgemeinbevölkerung,
  diese seien nur gezielt in Krankenhäusern und Pflegeheimen
  erforderlich. Professor Atlas verfasste Ende April einen Artikel mit
  dem Titel
  \href{https://thehill.com/opinion/healthcare/494034-the-data-are-in-stop-the-panic-and-end-the-total-isolation}{``Die
  Daten sind da -- Stoppt die Panik und beendet die totale Isolation''},
  der über 15.000 Kommentare erzeugte.
\item
  Epidemiologe \textbf{Dr. Knut Wittkowski} erklärt in
  \href{https://www.thepressandthepublic.com/post/perspectives-on-the-pandemic-v}{einem
  neuen Interview}, dass die Gefährlichkeit von Covid19 vergleichbar mit
  einer Influenza sei und der Höhepunkt in den meisten Ländern bereits
  vor dem Lockdown überschritten war. Der Lockdown ganzer Gesellschaften
  sei eine katastrophale Entscheidung ohne Nutzen aber mit enormen
  Schäden gewesen. Die wichtigste Maßnahme sei der Schutz von
  Pflegeheimen. Die Aussagen von Bill Gates zu Covid19 seien ``absurd''
  und hätten ``nichts mit der Realität zu tun'', eine Impfung gegen
  Covid19 sei nicht erforderlich. Das einflussreiche Covid19-Modell des
  britischen Epidemiologen Neil Ferguson sei ein ``völliger Fehlschlag''
  gewesen.
  (\href{https://vitalstoff.blog/2020/05/01/wir-brauchen-keinen-impfstoff/}{Deutsches
  Transkript des Interviews}) (\textbf{Hinweis}: Das Interview wurde von
  Youtube entfernt.)
\item
  Der deutsche Virologe \textbf{Hendrik Streeck} erklärt in
  \href{https://www.youtube.com/watch?v=vrL9QKGQrWk}{einem neuen
  Interview} die finalen Resultate seiner
  \href{https://medicalxpress.com/news/2020-05-team-covid-infection-fatality.html}{Antikörper-Studie}.
  Streeck fand eine Covid19-Letalität von 0.36\%, erklärt jedoch, dass
  dies eine Obergrenze sei und die Letalität vermutlich im Bereich 0.24
  bis 0.26\% oder sogar darunter liege. Das Durchschnittsalter der
  testpositiven Verstorbenen lag bei ca. 81 Jahren. Laut Professor
  Streeck ist es keine gute Strategie, auf einen Impfstoff zu warten, da
  die Machbarkeit und Wirksamkeit eines Impfstoffes unsicher sind.
\item
  Biologieprofessor und Nobelpreisträger \textbf{Michael Levitt}, der
  sich seit Februar mit der Ausbreitung von Covid19 befasst, beschreibt
  den allgemeinen Lockdown
  \href{https://www.youtube.com/watch?v=bl-sZdfLcEk}{als einen
  ``riesigen Fehler''} und fordert gezieltere Maßnahmen, insbesondere
  zum Schutz der Risikogruppen.
\item
  Der emeritierte Mikrobiologie-Professor \textbf{Sucharit Bhakdi}
  erklärt\href{https://www.servustv.com/videos/aa-23ud73pbh1w12/}{in
  einem neuen Interview}, dass Politik und Medien zu Covid19 eine
  ``unerträgliche Angstmacherei'' und eine ``unverantwortliche
  Desinformation'' gegenüber der Bevölkerung betreiben. Atemschutzmasken
  für die Allgemein­bevölkerung seien gesund­heits­schädliche
  Keimfänger. Die gegenwärtige Krise sei von den Politikern selbst
  herbeigeführt worden und habe wenig mit dem Virus zu tun. Ein
  Impfstoff gegen Coronaviren sei wie schon bei der Schweinegrippe
  ``unnötig und gefährlich''. Die WHO übernehme für ihre vielen
  Fehlentscheidungen seit Jahren keine Verantwortung.
\item
  Der Schweizer Chefarzt für Infektiologie, \textbf{Dr. Pietro
  Vernazza}, erklärt in
  \href{https://www.saldo.ch/artikel/artikeldetail/fuer-die-allermeisten-menschen-verlaeuft-die-erkrankung-mild/}{einem
  neuen Interview}, dass die Covid19-Erkrankung ``für die allermeisten
  Menschen mild verläuft''. Die ``Zählerei von Infizierten und der Ruf
  nach mehr Tests'' würden nicht viel bringen. Zudem würden die meisten
  Leute, die in der Corona­statistik aufgeführt sind, nicht nur an
  Covid-19 sterben. Die Sterblichkeit von Covid-19 liege nach bisherigen
  Erkenntnissen ``in der Größenordnung einer saisonalen Grippe''. Für
  den Nutzen von Atemschutzmasken bei Menschen, die selbst keine
  Symptome zeigen, gebe es keine Belege.
  (\href{https://swprs.files.wordpress.com/2020/05/saldo-interview-pietro-vernazza-14-04-2020.pdf}{Archivversion})
\end{itemize}

\hypertarget{medizinische-studien}{%
\subparagraph{\texorpdfstring{\textbf{Medizinische
Studien}}{Medizinische Studien}}\label{medizinische-studien}}

\begin{itemize}
\tightlist
\item
  Eine \href{https://swprs.org/studies-on-covid-19-lethality/}{neue
  Übersicht aller bisherigen PCR- und Antikörper-Studien} zeigt, dass
  der Medianwert der Covid19-Letalität (IFR) bei ca. 0.2\% und damit im
  Bereich einer starken Influenza liegt.
\item
  Eine
  \href{https://www.medrxiv.org/content/10.1101/2020.04.24.20075291v1}{neue
  Antikörper-Studie mit dänischen Blutspendern} ergab eine sehr tiefe
  Covid19-Letalität (IFR) von 0.08\% für Personen unter 70 Jahren.
\item
  Eine
  \href{https://www.medrxiv.org/content/10.1101/2020.04.26.20079244v1}{neue
  Antikörper-Studie} aus dem Iran, einem der am frühesten und am
  stärksten von Covid19 betroffenen Länder, kommt ebenfalls auf eine
  sehr tiefe Letalität von 0.08\% bis 0.12\%.
\item
  Eine neue
  \href{https://www.medrxiv.org/content/10.1101/2020.04.26.20079822v1}{Antikörper-Studie
  aus Japan} kommt zum Ergebnis, dass dort ca. 400 bis 800 mal mehr
  Menschen mit dem neuen Coronavirus Kontakt hatten als bisher
  angenommen, jedoch keine oder kaum Symptome zeigten. Japan hatte
  bisher relativ wenig getestet.
\item
  Eine
  \href{https://www.medrxiv.org/content/10.1101/2020.04.17.20061440v1}{neue
  Studie aus Deutschland} mit Beteiligung des Virologen Christian
  Drosten zeigt, dass rund ein Drittel der Bevölkerung bereits eine
  gewisse \textbf{zelluläre Immunität} gegen das Covid19-Coronavirus
  aufweist, vermutlich durch den Kontakt mit früheren Coronaviren
  (Erkältungsviren). Diese zelluläre Immunität durch sogenannte T-Zellen
  liegt deutlich höher als die PCR- und Antikörper-Tests vermuten ließen
  und dürfte teilweise erklären, warum viele Menschen beim neuen
  Coronavirus keine oder kaum Symptome entwickeln.
\item
  In einem Gefängnis im US-Bundesstaat Tennesse
  \href{https://www.tennessean.com/story/news/politics/2020/05/01/tennessee-testing-all-inmates-prison-staff-after-multiple-outbreaks/3067388001/}{zeigten}
  nur zwei von 1349 testpositiven Personen überhaupt Symptome.
\item
  Auf dem französischen \textbf{Flugzeugträger} Charles de Gaulle
  verstarb von 1046 testpositiven Matrosen bisher
  \href{https://en.wikipedia.org/wiki/COVID-19_pandemic_on_Charles_de_Gaulle}{keiner}.
  Auf dem US-Flugzeugträger Theodore Roosevelt verstarb von 969
  testpositiven Matrosen bisher
  \href{https://en.wikipedia.org/wiki/COVID-19_pandemic_on_USS_Theodore_Roosevelt}{einer}
  (Vorerkrankungen und Todesursache nicht bekannt). Insgesamt ergibt
  sich daraus eine Letalität von 0 bis 0.1\% für diese
  Bevölkerungsgruppe.
\item
  Zahlreiche Medien berichteten von angeblichen
  \textbf{``Neu-Infektionen''} bereits genesener Personen in Südkorea.
  Forscher kamen nun aber
  \href{https://www.independent.co.uk/news/world/asia/coronavirus-south-korea-patients-infected-twice-test-a9491986.html}{zum
  Ergebnis}, dass es sich bei diesen 290 Verdachtsfällen um falsche
  positive Testresultate handelte, ausgelöst durch ``nicht-infektiöse
  Virenfragmente''. Das Ergebnis wirft auch ein Schlaglicht auf die
  \href{https://www.ncbi.nlm.nih.gov/pubmed/32219885}{bekannte
  Unzuverlässigkeit} der Virentests.
\end{itemize}

\hypertarget{weitere-medizinische-meldungen}{%
\subparagraph{\texorpdfstring{\textbf{Weitere medizinische
Meldungen}}{Weitere medizinische Meldungen}}\label{weitere-medizinische-meldungen}}

\begin{itemize}
\tightlist
\item
  Im Rahmen einer weiteren Panikmeldung berichteten zahlreiche Medien
  davon, dass im Zusammenhang mit Covid19 zunehmend Kinder an der
  sogenannten \textbf{Kawasaki-Krankheit} (eine Gefäßentzündung)
  erkranken würden. Die britische Kawasaki Disease Foundation teilte
  \href{https://www.societi.org.uk/kawasaki-disease-and-covid-19/}{in
  einer Pressemitteilung} indes mit, dass derzeit \emph{weniger}
  Kawasaki-Fälle als üblich gemeldet werden, und dass von den wenigen
  gemeldeten Fällen nur rund die Hälfte überhaupt positiv auf
  Coronaviren getestet haben.
\item
  Ein \textbf{französischer Arzt} spricht in
  einem\href{https://covidinfos.net/covid19/la-lettre-dun-praticien-hospitalier-adressee-au-ministere-de-la-sante-denonce-une-arnaque-sanitaire/604/}{Offenen
  Brief an das französische Gesundheits­ministerium} bezüglich Covid19
  vom ``größten Gesundheitsbetrug des 21. Jahrhunderts''. Die
  Gefährlichkeit des Virus für die Allgemeinbevölkerung liege im Bereich
  der Influenza und die Folgen des Lockdowns seien gefährlicher als das
  Virus selbst.
\item
  In \textbf{Frankreich} wurde im Rahmen einer nachträglichen
  Untersuchung
  \href{https://www.reuters.com/article/us-health-coronavirus-france-retests/frances-early-covid-19-case-may-hold-clues-to-pandemics-start-idUSKBN22H15R}{bekannt},
  dass der erste Covid19-positive Patient \emph{bereits Ende Dezember
  2019} und damit einen Monat früher als bisher angenommen behandelt
  wurde. Der Mann wurde damals wegen einer scheinbar grippebedingten
  Lungenentzündung behandelt. Dieser Fall zeigt, dass die neuen
  Coronaviren entweder bereits früher als angenommen in Europa waren,
  oder dass sie nicht so neu sind wie angenommen, oder dass es sich um
  ein falsches positives Testresultat handelt. Zudem ist nicht klar, ob
  der derweil wieder gesunde Mann nun an Grippe- oder an Coronaviren
  oder an beiden erkrankt war.
\item
  Der Exekutivdirektor der WHO nannte \textbf{Schweden} zuletzt
  \href{https://www.nau.ch/news/schweiz/coronavirus-who-nennt-schweden-ein-vorbild-65701044}{ein
  ``Vorbild''} für den Umgang mit Covid19. Schweden habe seine
  Gesundheitspolitik erfolgreich und ``in Partnerschaft mit der
  Bevölkerung'' umgesetzt. Zuvor wurde Schweden während Wochen von
  ausländischen Medien und Politikern für seinen entspannten Umgang mit
  Covid19 heftig kritisiert.
\item
  \textbf{Weißrussland}, das von allen europäischen Ländern am wenigsten
  Maßnahmen gegen Covid19 ergriff und selbst Sportturniere und andere
  Großveranstaltungen nicht absagte, zählt nach über zwei Monaten erst
  103 testpositive oder vermutete
  \href{https://en.wikipedia.org/wiki/COVID-19_pandemic_in_Belarus}{Covid19-Todesfälle}.
  Der weißrussiche Langzeit-Präsident Lukashenko nannte Corona
  \href{https://www.baltictimes.com/lukashenko__coronavirus_is_psychosis/}{eine
  ``Psychose''}. Kritiker vermuten indes, er gebe nicht alle Todesfälle
  bekannt.
\item
  Eine
  \href{https://covidinfos.net/wp-content/uploads/2020/05/MasksDon-twork-4.pdf}{umfangreiche
  Literaturauswertung} durch einen kanadischen Forscher ergab, dass
  \textbf{Atemschutzmasken} keinen nachweisbaren Schutz gegen
  Erkältungen und Influenza bieten.
\item
  Ein Schweizer Chefarzt für \textbf{Psychiatrie} rechnet aufgrund des
  Lockdowns und der Arbeitslosigkeit mit einer starken Zunahme an
  psychischen Problemen und
  \href{https://www.aargauerzeitung.ch/aargau/kanton-aargau/aargauer-psychiatrie-chefarzt-kawohl-warnt-arbeitslosigkeit-erhoeht-das-suizidrisiko-137742663}{über
  10,000 Suiziden} weltweit.
\item
  Die sogenannte \textbf{Reproduktionszahl}, die die Weiterverbreitung
  von Covid angibt, wird zunehmend
  \href{https://www.tagesschau.de/faktenfinder/corona-reproduktionszahl-101.html}{zum
  Politikum}. An der Realität ändert das jedoch nichts: Der Höhepunkt
  der Ausbreitung war in den meisten Ländern bereits vor dem Lockdown
  erreicht und die Reproduktionszahl fiel durch simple Alltags- und
  Hygienemaßnahmen auf oder unter den stabilen Wert von eins. Der
  Lockdown war epidemiologisch mithin
  \href{https://infekt.ch/2020/04/sind-wir-tatsaechlich-im-blindflug/}{unnötig
  und unsinnig}.
\item
  Das \textbf{Krankheitsbild und die Risikogruppen} von
  Covid19-Coronaviren dürften mit der Nutzung des sogenannten
  ACE2-Zellrezeptors
  \href{https://www.news-medical.net/news/20200420/ACE2-TMPRSS2-profiling-indicates-tissue-vulnerability-to-SARS-CoV-2-infection.aspx}{zusammenhängen},
  der in den Bronchien und Lungen, aber auch in Blutgefäßen, dem Darm
  und Nieren vorkommt. Allerdings nutzen auch
  \href{https://www.ncbi.nlm.nih.gov/pmc/articles/PMC4369385/}{andere
  Coronaviren} wie etwa das Erkältungsvirus NL63 den ACE2-Rezeptor.
  Einige Forscher
  \href{https://www.ncbi.nlm.nih.gov/pmc/articles/PMC7074995/}{rechnen
  deshalb damit}, dass auch die Covid19-Coronaviren mittelfristig als
  gewöhnliche Corona-Erkältungsviren zu sehen sind.
\item
  Die genaue \textbf{Herkunft des neuen Coronavirus} ist weiterhin
  unklar. Am naheliegendsten ist eine natürliche Übertragung oder
  Mutation, wie sie häufig vorkommt. Es ist indes richtig, dass das
  virologische Labor in Wuhan im Rahmen eines
  \href{https://www.newsweek.com/dr-fauci-backed-controversial-wuhan-lab-millions-us-dollars-risky-coronavirus-research-1500741}{von
  den USA mitfinanzierten Forschungs­programms} Coronaviren aus
  Fledermäusen untersuchte und dabei auch die Übertragbarkeit auf andere
  Säugetiere prüfte, was von einigen Forschern seit Jahren als zu
  riskant kritisiert wurde. Die Leiterin des Labors
  \href{https://www.scientificamerican.com/article/how-chinas-bat-woman-hunted-down-viruses-from-sars-to-the-new-coronavirus1/}{erklärte
  indes}, das neue Virus entspreche nicht den im Labor untersuchten
  Coronaviren. Frühere Gerüchte bezüglich ``Biowaffen'' oder
  ``HIV-Sequenzen'' sind angesichts der relativen Harmlosigkeit des
  Coronavirus indes als
  \href{https://onlinelibrary.wiley.com/doi/full/10.1111/eci.13222}{Desinformation}
  anzusehen.
\end{itemize}

\hypertarget{pflegeheime}{%
\subparagraph{\texorpdfstring{\textbf{Pflegeheime}}{Pflegeheime}}\label{pflegeheime}}

Pflegeheime spielen eine
\href{https://ltccovid.org/2020/04/12/mortality-associated-with-covid-19-outbreaks-in-care-homes-early-international-evidence/}{absolute
Schlüsselrolle} in der aktuellen Corona-Situation. In den meisten
westlichen Ländern ereigneten sich 30\% bis 70\% aller Todesfälle ``im
Zusammenhang mit Covid'' in Pflegeheimen (in einzelnen Regionen sogar
bis zu 90\%). Auch aus Norditalien ist bekannt, dass die dortige Krise
durch einen
\href{https://swprs.org/covid19-bericht-aus-italien/}{panikbedingten
Zusammenbruch der Alterspflege} begann.

Pflegeheime erfordern einen gezielten Schutz und profitieren nicht von
einem allgemeinen Lockdown der Gesellschaft. Betrachtet man nur die
Todesfälle in der Allgemeinbevölkerung, so liegen diese in den meisten
Ländern im Rahmen einer gewöhnlichen oder sogar milden Grippewelle.

In vielen Fällen ist zudem nicht klar, woran die Menschen in den
Pflegeheimen wirklich starben, d.h. ob an Covid19 oder an Stress, Angst
und Einsamkeit. Aus Belgien ist beispielsweise
\href{https://web.archive.org/web/20200424212535/https://covid-19.sciensano.be/sites/default/files/Covid19/Meest\%20recente\%20update.pdf}{bekannt},
dass ca. 94\% aller Todesfälle in Pflegeheimen ungetestete
``Verdachtsfälle'' sind.

Eine
\href{https://covidinfos.net/covid19/deces-dus-au-covid-19-le-nombre-officiel-de-morts-en-france-est-il-surestime/502/}{neue
Analyse der französischen Statistiken} zeigt zudem Folgendes: Sobald es
in einem Pflegeheim einen ``Verdachtsfall'' gibt (z.B. durch Husten),
gelten alle Todesfälle als ``Covid19-Verdachtsfälle'', und sobald es in
einem Pflegeheim einen ``bestätigten Fall'' gibt (selbst wenn
symptomlos), gelten alle Todesfälle als ``bestätigte
Covid19-Todesfälle''.

Ein Bericht aus Deutschland
\href{https://www.rnd.de/gesundheit/corona-ist-mir-egal-warum-helga-witt-kronshage-86-lieber-sterben-will-als-eingesperrt-zu-sein-3MEBDIOBEFA6BDULC4N5WGZJG4.html}{beschreibt
eindringlich}, unter welch extremen Bedingungen hundert­tausende
Patienten in Alters- und Pflegeheimen in den vergangenen Wochen leben
mussten, und zwar oftmals auch gegen ihren Willen. Viele der Patienten
durften ihr Zimmer kaum mehr verlassen, durften nicht mehr an die
frische Luft und keinen Besuch von Angehörigen empfangen.

In mehreren Pflegeheimen kam es aufgrund des fehleranfälligen
PCR-Virentests zudem zu
\href{https://www.wjhl.com/local-coronavirus-coverage/agape-nursing-home-says-residents-who-tested-positive-for-covid-19-were-false-positives/}{folgenreichen
Fehlalarmen}. In einem kanadischen Pflegeheim flüchteten die Mitarbeiter
aus Angst vor dem Coronavirus, sodass in der Folge 31 Menschen
\href{https://orf.at/stories/3162365/}{mangels Pflege verstarben}.

Der frühere New York Times Journalist und Corona-Kritiker Alex Berenson
\href{https://twitter.com/AlexBerenson/status/1257496834043531267}{schreibt
dazu auf Twitter}: ``Let's be clear: the fact the nursing home deaths
are not front and center every day in elite media coverage of \#COVID
tells you everything you need to know about the media's priority --
which is instilling panic (and punishing Trump), not driving good health
policy.''

\textbf{Zur Analyse}:
\href{https://ltccovid.org/2020/04/12/mortality-associated-with-covid-19-outbreaks-in-care-homes-early-international-evidence/}{Mortality
associated with COVID-19 outbreaks in care homes: early international
evidence} (LTC Covid, Mai 2020)

\includegraphics{https://swprs.files.wordpress.com/2020/05/care-homes-covid.png?w=700\&h=462}

\hypertarget{grouxdfbritannien}{%
\subparagraph{\texorpdfstring{\textbf{Großbritannien}}{Großbritannien}}\label{grouxdfbritannien}}

\begin{itemize}
\tightlist
\item
  Die kumulierte Gesamtsterblichkeit in Großbritannien liegt derzeit im
  Bereich der fünf \href{http://inproportion2.talkigy.com/}{stärksten
  Grippewellen} der letzten 25 Jahre. Der Höhepunkt der täglichen
  Todesfälle in Krankenhäusern war bereits am 8. April erreicht (siehe
  Grafik unten).
\item
  Neue
  \href{https://www.telegraph.co.uk/politics/2020/05/01/evidence-rising-britains-lockdown-could-deadly-mistake/}{statistische
  Daten} zeigen, dass Mitte April von ca. 12.000 \emph{zusätzlichen}
  Todesfällen ca. 9000 ``im Zusammenhang mit Covid'' waren (inkl.
  ``Verdachtsfälle''), aber ca. 3000 ``nicht im Zusammenhang mit
  Covid''. Zudem seien von den insgesamt ca. 7300 Todesfällen in
  Pflegeheimen nur ca. 2000 ``im Zusammenhang mit Covid'' erfolgt.
  Sowohl bei den ``Covid19-Todesfällen'' als auch bei den
  Nicht-Covid19-Todesfällen ist
  \href{https://www.hsj.co.uk/commissioning/thousands-of-extra-deaths-outside-hospital-not-attributed-to-covid-19/7027459.article}{oftmals
  unklar}, woran diese Menschen wirklich starben. Der Verband der
  britischen Pathologen hat deshalb einen
  \href{https://www.hsj.co.uk/coronavirus/systematic-reviews-to-discover-true-cause-of-outbreak-deaths/7027491.article}{``systematischen
  Review der wirklichen Todesursachen''} gefordert.
\item
  Die zusätzlich erichteten temporären Krankenhäuser blieben
  bisher~\href{https://www.telegraph.co.uk/news/0/do-many-nhs-nightingale-hospitals-remain-empty/}{weitgehend
  leer}. Ein ähnliches Bild zeigte sich zuvor bereits in China, den USA
  und anderen Ländern.
\item
  Ende April wurde bekannt, dass der Lockdown offenbar nicht, wie
  offiziell dargestellt, allein von einer wissenschaftlichen Kommission
  empfohlen wurde, sondern dass ein hoher Regierungs­berater die
  Wissenschaftler
  \href{https://www.bloomberg.com/news/articles/2020-04-28/top-aide-to-u-k-s-johnson-pushed-scientists-to-back-lockdown}{zur
  Unterstützung des Lockdowns bewegte}.
\item
  Peter Hitchens:
  \href{https://hitchensblog.mailonsunday.co.uk/2020/05/peter-hitchens-were-destroying-the-nations-wealth-and-the-health-of-millions.html}{We're
  destroying the nation's wealth -- and the health of millions} ``If you
  don't defend your most basic freedom, the one to go lawfully where you
  wish when you wish, then you will lose it for ever. And that is not
  all you will lose. Look at the censorship of the internet, spreading
  like a great dark blot, the death of Parliament, the conversion of the
  police into a state militia.''
\end{itemize}

\includegraphics{https://swprs.files.wordpress.com/2020/05/england-deaths.png?w=736\&h=457}

\hypertarget{vereinigte-staaten}{%
\subparagraph{**Vereinigte Staaten}\label{vereinigte-staaten}}

**

\begin{itemize}
\tightlist
\item
  Der
  \href{https://www.cdc.gov/coronavirus/2019-ncov/covid-data/covidview/index.html}{neueste
  Bericht der US-Gesundheitsbehörde CDC} zeigt, dass die
  Covid19-Hospitali­sierungs­rate bei den über 65-Jährigen im Bereich
  starker Grippewellen liegt. Bei den 18- bis 64-Jährigen liegt sie
  etwas darüber, bei den unter 18-Jährigen liegt sie deutlich darunter.
\item
  \textbf{Video}: Eine Krankenschwester aus New York City
  \href{https://www.dailymail.co.uk/news/article-8262351/Nurse-New-York-claims-city-killing-COVID-19-patients-putting-ventilators.html}{erklärte
  in einem dramatischen Beitrag}, New York würde die Covid19-Patienten
  ``umbringen'', in dem sie an invasive Beatmungs­maschinen
  angeschlossen und ihre Lungen dadurch zerstört werden. Die Verwendung
  der invasiven Beatmungs­maschinen (statt einfache Beatmungsmasken)
  geschehe ``aus Angst vor einer Verbreitung des Virus''. Es sei ``ein
  Horrorfilm'', ``nicht wegen der Krankheit, sondern wegen der Art, wie
  damit umgegangen wird''. Fachleute haben bereits seit März vor der
  Intubation von Covid19-Patienten
  \href{https://off-guardian.org/2020/05/06/covid19-are-ventilators-killing-people/}{gewarnt}.
\item
  Dr. Daniel Murphy, der Leiter der Notfallmedizin eines stark
  betroffenen Krankenhauses in New York City, empfiehlt eine
  \href{https://nypost.com/2020/04/27/ive-worked-the-coronavirus-front-line-and-i-say-its-time-to-start-opening-up/}{rasche
  Beendigung des Lockdowns}. Die Covid19-Welle habe bereits am 7. April
  ihren Höhepunkt erreicht. Covid19 sei eine ernste Angelegenheit, aber
  die Angst davor sei übertrieben, da die große Mehrheit der Bevölkerung
  höchtens mild erkranke. Seine größte Sorge sei nun der starke Rückgang
  der Versorgung von Notfallpatienten und Kindern durch den Lockdown und
  die verbreitete Angst in der Bevölkerung.
\item
  \textbf{Video}: Die konservative Enthüllungs­plattform Project Veritas
  sprach mit \href{https://www.youtube.com/watch?v=g5f_6ltv7oI}{Chefs
  von New Yorker Bestattungs­­unter­­nehmen}, die erklärten, dass
  derzeit ``auf alle Totenscheine'' (von Verdachts­fällen) Covid
  geschrieben werde, egal ob es einen Test gab oder nicht. Viele
  Menschen würden derzeit zuhause sterben, und oft werde die
  Todesursache gar nicht mehr überprüft. Die Covid19-Statistiken würden
  aus politischen oder finanziellen Gründen aufgeblasen, da es für
  Covid-Patienten und -Todesfälle zusätzliche Bundesgelder gebe.
\item
  Der Direktor des Gesundheitsdepartements von Illinois
  \href{https://week.com/2020/04/20/idph-director-explains-how-covid-deaths-are-classified/}{bestätigte},
  dass selbst todkranke Menschen, die eindeutig an einer anderen Ursache
  sterben, aber vor oder nach dem Tod positiv auf Covid19-Viren getestet
  werden, als Covid19-Todesfälle erfasst werden.
\item
  In den USA haben aufgrund des Lockdowns bis Ende April bereits 30
  Millionen Menschen
  \href{https://edition.cnn.com/2020/04/30/economy/unemployment-benefits-coronavirus/index.html}{Arbeitslosenhilfe
  beantragt} -- das sind deutlich mehr, als die Internationale
  Arbeitsagentur ILO ursprünglich für die ganze Welte annahm.
\item
  Tesla-Chef Elon Musk bezeichnete die Ausgangssperren in Kalifornien
  als
  \href{https://www.theguardian.com/technology/2020/apr/29/tesla-quarterly-earnings-coronavirus-shares}{``faschistisch''}.
  Die ``gewaltsame Inhaftierung'' von Menschen in ihren Häusern verstoße
  gegen all ihre verfassungs­mäßigen Rechte, erklärte Musk in einer
  Telefonkonferenz.
\item
  \textbf{Video}: In den USA wurde eine Mutter zuhause
  \href{https://twitter.com/AlexBerenson/status/1256219418343981056}{von
  der Polizei konfrontiert}, weil ihre Kinder unerlaubterweise mit den
  Nachbarkindern gespielt haben.
\item
  \textbf{Video}: US-Medien wurden Ende April bei der
  \href{https://twitter.com/talialikeitis/status/1253126254942773248}{teilweisen
  Inszenierung} eines Protests von Pflegemitarbeitern gegen
  Anti-Lockdown-Demonstranten erwischt.
  (\href{https://www.buzzfeednews.com/article/tasneemnashrulla/photos-denver-nurses-block-anti-lockdown-protest}{Mehr
  dazu}).
\end{itemize}

\hypertarget{schweiz}{%
\subparagraph{\texorpdfstring{\textbf{Schweiz}}{Schweiz}}\label{schweiz}}

\begin{itemize}
\tightlist
\item
  Die
  \href{https://swprs.files.wordpress.com/2020/05/schweiz-todesfaelle-2010-2020_woche_17.pdf}{kumulierten
  Todesfälle} seit Anfang Jahr liegen in der Schweiz weiterhin im
  Bereich einer üblichen Grippewelle und weit unter der starken
  Grippewelle von 2015 (siehe Grafik unten). Rund
  \href{https://www.bluewin.ch/de/news/schweiz/sp-chef-levrat-will-die-reichen-schropfen-383977.html}{50\%
  der Todesfälle} erfolgten in Alters- und Pflegeheimen.
\item
  Die Schweizer Regierung plant, die aktuellen Corona-Notverordnungen in
  ein permanentes
  \href{https://www.admin.ch/gov/de/start/dokumentation/medienmitteilungen.msg-id-78929.html}{dringliches
  Bundesgesetz zu überführen}. Die meisten Schweizer Medien haben über
  diese folgenreiche Ankündigung nicht oder nur am Rande berichtet.
\item
  Die Schweizer Armee begann mit der Testung einer
  \href{https://uncut-news.ch/wp-content/uploads/2020/05/Schweizer-Soldaten-k\%C3\%A4mpfen-bewaffnet-mit-Bluetooth-App-gegen-COVID-19.pdf}{App
  zur Kontaktverfolgung}, die ab dem 11. Mai in Zusammenarbeit mit
  Google und Apple eingeführt werden soll. Ein Schweizer
  ``Datenschützer''
  \href{https://www.nzz.ch/zuerich/coronavirus-in-zuerich-tracing-app-braucht-keine-freiwilligkeit-ld.1553964}{erklärte
  derweil}: ``Wenn die Contact-Tracing-App geeignet und erforderlich
  ist, braucht es keine Freiwilligkeit''.
\item
  Auf dem Schweizer Bundesplatz in Bern kam es zu verschiedenen
  \href{https://www.zeitpunkt.ch/index.php/mahnwache-fuer-grundrechte-400-menschen-auf-dem-bundesplatz-wie-aus-dem-nichts}{Mahnwachen
  mit rund 400 Teilnehmern}, die sich gegen Einschränkungen der
  Verfassungsrechte aussprachen. Die Kundgebungen wurde jeweils von der
  Polizei geräumt.
\item
  Im Rahmen von Covid19 kam nicht die seit langem bestehende Schweizer
  \href{https://www.srf.ch/news/schweiz/bag-verzichtete-auf-beratung-was-macht-die-pandemie-kommission-in-der-krise}{Pandemie-Kommission}
  zum Einsatz, sondern eine eigens gegründete
  \href{https://ncs-tf.ch/de/organisation}{``Covid-19 Task Force''},
  deren Mitglieder teilweise Interessenskonflikte im Bereich der
  Pharmazie aufweisen.
\item
  \textbf{Video}:
  \href{https://www.youtube.com/watch?v=RyZGkdeQ6CY}{``Gehört der
  Bundesrat ins Gefängnis?''} Der Schweizer Journalist Reto Brennwald
  interviewte den Unternehmer
  \href{https://www.youtube.com/user/timturpis/videos}{Daniel Stricker},
  der Mitte März für einige Wochen aus der Schweiz nach Schweden
  flüchtete und die Corona-Politik des Schweizer Bundesrates stark
  kritisiert.
\item
  Eine Schweizer Pflegefachfrau hat einen vielbeachteten
  \href{https://www.facebook.com/simone.christinat/posts/10221314943115981}{Beitrag
  zur aktuellen Situation} verfasst. Sie erklärt, dass die Schweizer
  Krankenhäuser weitgehend leer blieben und teilweise Kurzarbeit
  anmelden mussten. Es sei zudem sehr ungewöhnlich, Menschen 80+ wegen
  Grippe oder Lungenentzündung auf die Intensivstation zu verlegen, wo
  sie dann einsam statt im Kreise ihrer Familie sterben müssen. Würde
  man dies tun, wären die Intensivstationen fast jeden Winter
  überlastet. Die Pflegefachfrau kritisiert, dass die meisten Medien die
  neueren wissenschaftlichen Erkenntnisse zur insgesamt geringen Gefahr
  durch Covid19 nicht ausreichend thematisiert haben.
\end{itemize}

\includegraphics{https://swprs.files.wordpress.com/2020/05/schweiz-todesfaelle-2010-2020_woche_17-1.png?w=736\&h=347}

\hypertarget{deutschland-und-uxf6sterreich}{%
\subparagraph{\texorpdfstring{\textbf{Deutschland und
Österreich}}{Deutschland und Österreich}}\label{deutschland-und-uxf6sterreich}}

\begin{itemize}
\tightlist
\item
  In Deutschland und Österreich besteht ähnlich wie in Dänemark,
  Finnland, Griechenland, Ungarn, Irland, Luxemburg, Malta, Norwegen und
  Portugal bisher \href{https://www.euromomo.eu/graphs-and-maps/}{keine
  Übersterblichkeit}.
\item
  Laut einem geleakten Protokoll der österreichischen Corona-Taskforce
  soll Kanzler Kurz im März
  \href{https://www.derstandard.de/story/2000117131591/sitzungsprotokoll-der-taskforce-corona-ueber-zu-wenig-angst-in-der}{gefordert
  haben}, dass die Bevölkerung ``mehr Angst'' vor einer Ansteckung oder
  dem Tod der Eltern oder Großeltern haben müsse. Bereits zuvor wurde
  ein
  \href{https://fragdenstaat.de/dokumente/4123-wie-wir-covid-19-unter-kontrolle-bekommen/}{Strategiepapier}
  des deutschen Bundes­­innen­­mini­steriums bekannt, das ebenfalls eine
  psychologische Angstkampagne forderte, die von Politik und Medien
  tatsächlich auch umgesetzt wurde. Rückblickend stellt sich die Frage,
  wieviele Menschen an den Folgen dieser weitgehend unbegründeten Angst
  gestorben sind.
\item
  Ein
  \href{https://www.change.org/p/bundeskanzlerin-corona-sch\%C3\%BCtzen-sie-\%C3\%A4ltere-nicht-um-diesen-preis-selbstbestimmt-altern-und-sterben}{Offener
  Brief mit bereits rund 5000 Unterschriften} von Menschen über 64
  Jahren fordert: ``Corona: Schützen Sie uns Ältere nicht um diesen
  Preis! Selbstbestimmt altern und sterben!'' Für den Schutz einer
  Risikogruppe dürften nicht die Grundrechte der gesamten Gesellschaft
  außer Kraft gesetzt werden, fordern die Autoren.
\item
  In Österreich (und womöglich auch in anderen Ländern) ist ein Kuss
  unter Verliebten, aber nicht zusammenlebenden Menschen
  \href{https://www.unsertirol24.com/2020/04/29/oeffentliches-kuessen-wird-unter-strafe-gestellt/}{weiterhin
  verboten}. Dies gelte sowohl in der Öffentlichkeit als auch in den
  eigenen vier Wänden, erklärte der österreichische
  Gesundheitsminister.\\
\item
  Eine deutsche Rechtsanwältin klagt derzeit vor mehreren Gerichten
  gegen die eingeführten Corona-Maßnahmen, da diese
  \href{https://www.rubikon.news/artikel/eklatant-verfassungswidrig-2}{``eklatant
  verfassungswidrig''} seien.
\item
  \textbf{Videos}: In Deutschland kam es zuletzt zu teilweise
  erheblichen Übergriffen durch die Polizei. Eine junge Frau wurde beim
  Einkaufen von mehreren Polizisten auf rabiate Weise
  \href{https://web.archive.org/web/20200509235301/https://www.youtube.com/watch?v=TZrKv4-jkK8}{festgenommen},
  da sie einer Polizistin offenbar ``20cm zu nahe gekommen sei''. Eine
  andere Frau wurde von der Polizei auf einer Kundgebung
  \href{https://twitter.com/ChristianFritze/status/1256609660318224385}{angewiesen},
  das deutsche Grundgesetz nicht vor der Brust zu halten, da dies eine
  ``unerlaubte politische Botschaft'' sei. Auch der Organisator einer
  friedlichen Kundgebung in Berlin wurde auf eher brachiale Weise
  \href{https://web.archive.org/web/20200506155035/https://www.youtube.com/watch?v=NbV2OH3uYxI}{verhaftet}.
  Selbst ältere Frauen wurden auf
  \href{https://www.youtube.com/watch?v=Bn11jXTjh_Y}{unverhältnismäßige
  Weise festgenommen}. (\textbf{Vorsicht}: Verstörende Bilder von
  Polizeigewalt).
\end{itemize}

\hypertarget{weitere-meldungen}{%
\subparagraph{**Weitere Meldungen}\label{weitere-meldungen}}

**

\begin{itemize}
\tightlist
\item
  Die Chefin von \textbf{Youtube} erklärte Ende April in einem
  Interview, dass Videobeiträge zum Coronavirus, die den Vorgaben der
  WHO oder der nationalen Gesundheitsbehörden widersprechen,
  \href{https://www.businessinsider.com/youtube-will-ban-anything-against-who-guidance-2020-4}{entfernt
  werden}. So wurde beispielsweise das kritische Video der beiden
  kalifornischen Notfallärzte mit über fünf Millionen Ansichten
  \href{https://www.turnto23.com/news/coronavirus/video-interview-with-dr-dan-erickson-and-dr-artin-massihi-taken-down-from-youtube}{gelöscht}.
  Ebenso wurde das weiter oben verlinkte Interview mit Professor
  Sucharit Bhakdi von Youtube zeitweise entfernt.
\item
  Im US-Magazin \emph{The Atlantic} verfassten zwei Rechtsprofessoren
  einen
  \href{https://www.theatlantic.com/ideas/archive/2020/04/what-covid-revealed-about-internet/610549/}{Beitrag
  mit dem Titel}: ``Internet Speech Will Never Go Back to Normal. In the
  debate over freedom versus control of the global network, China was
  largely correct, and the U.S. was wrong.''
\item
  Mathias Döpfner, Vorstandsvorsitzender von Axel Springer und einer der
  einflussreichsten Medienmanager Deutschlands, fordert im Zuge der
  Corona-Krise eine
  \href{https://www.german-foreign-policy.com/news/detail/8263/}{``Abkopplung
  von China''} und eine Stärkung des transatlantischen Bündnisses mit
  den USA.
\item
  Washington Post:
  \href{https://www.washingtonpost.com/history/2020/05/01/vaccine-swine-flu-coronavirus/}{``The
  last time the government sought a `warp speed' vaccine, it was a
  fiasco''}. Die Schweinegrippe-Express-Impfung von 1976 führte zu
  Lähmungen und Todesfällen.
\item
  Rückblick:
  \href{https://www.aier.org/article/woodstock-occurred-in-the-middle-of-a-pandemic/}{Woodstock
  Occurred in the Middle of a Pandemic}. Zum vergleichsweise entspannten
  Umgang mit der weltweiten Grippepandemie von 1968.
  (\href{https://www.britannica.com/event/Hong-Kong-flu-of-1968}{Mehr
  dazu}).
\end{itemize}

\hypertarget{covid-19-und-die-medien}{%
\subparagraph{\texorpdfstring{\textbf{Covid-19 und die
Medien}}{Covid-19 und die Medien}}\label{covid-19-und-die-medien}}

Viele Menschen sind erstaunt und irritiert über die unseriöse und
oftmals angstverstärkende Berichterstattung vieler Medien. Es handelt
sich dabei offenkundig nicht um eine ``gewöhnliche Berichterstattung'',
sondern um klassische und massive Propaganda, wie sie üblicherweise im
Zusammenhang mit
\href{https://swprs.org/propaganda-im-jugoslawienkrieg/}{Angriffskriegen}
oder angeblichem
\href{https://www.motherjones.com/politics/2013/01/terror-factory-fbi-trevor-aaronson-book/}{Terrorismus}
zum Einsatz kommt.

SPR hat die Mediennetzwerke, die für die Verbreitung solcher Propaganda
zuständig sind, in früheren Infografiken für
\href{https://swprs.org/das-american-empire-und-seine-medien/}{die USA},
für \href{https://swprs.org/netzwerk-medien-deutschland/}{Deutschland}
und für \href{https://swprs.org/netzwerk-medien-schweiz/}{die Schweiz}
dargestellt. Auch das Internetlexikon Wikipedia ist ein
\href{https://swprs.org/wikipedia-disinformation-operation/}{integraler
Bestandteil} dieser geopolitisch-medialen Netzwerke.

In einem Mediennavigator für
\href{https://swprs.org/media-navigator/}{englische Medien} und für
\href{https://swprs.org/medien-navigator/}{deutsche Medien} wurde die
politische und geopolitische Ausrichtung verschiedener Medien
dargestellt. Diese Medien-Navigatoren können auch bezüglich der
Covid19-Bericht­erstattung eine erste Orientierungshilfe bieten.

Wenn im Fernsehen beispielsweise Bilder von Soldaten in Schutzanzügen zu
sehen sind, die ganze Straßenzüge desinfizieren, dann belegt das eben
nicht die Gefährlichkeit des Coronavirus, sondern es belegt -- wie
Professor Giesecke es wohlwollend
\href{https://www.addendum.org/coronavirus/interview-johan-giesecke/}{formulierte}
-- nutzlosen ``politischen Aktivismus''. Oder wie andere es formulieren
würden: Propaganda.

\hypertarget{covid-19-und-massenuxfcberwachung}{%
\subparagraph{\texorpdfstring{\textbf{Covid-19 und
Massenüberwachung}}{Covid-19 und Massenüberwachung}}\label{covid-19-und-massenuxfcberwachung}}

Die bei weitem bedeutendste und aus zivilgesellschaftlicher Sicht
gefährlichste Entwicklung in Zusammenhang mit dem Coronavirus ist der
politische Versuch, die Massenüberwachung und Kontrolle der Gesellschaft
massiv auszubauen. NSA-Whistleblower Edward Snowden warnte in diesem
Zusammenhang vor der Entstehung einer
\href{https://www.vice.com/en_us/article/bvge5q/snowden-warns-governments-are-using-coronavirus-to-build-the-architecture-of-oppression}{``Architektur
der Unterdrückung''}.

Das grippeähnliche Coronavirus dient dabei als Anlass oder als Vorwand
für die Einführung von
\href{https://norberthaering.de/die-regenten-der-welt/id2020-ktdi-apple-google/}{strategischen
Maßnahmen} zur erweiterten Kontrolle einer zunehmend unruhigen
Bevölkerung. Zu den wichtigsten derzeit diskutierten Instrumenten in
diesem Zusammenhang gehören:

\begin{enumerate}
\def\labelenumi{\arabic{enumi}.}
\tightlist
\item
  Die Einführung von Applikationen zur gesamtgesellschaftlichen
  Kontaktverfolgung
\item
  Der Aufbau von Einheiten zur Durchsetzung der Verfolgung und
  Isolierung von Bürgern
\item
  Die Einführung von digitalen biometrischen Ausweisen, über die die
  Teilnahme an gesellschaftlichen und beruflichen Aktivitäten
  kontrolliert und reguliert werden kann.
\item
  Die erweiterte Kontrolle von Reiseverkehr und Zahlungsverkehr
  (Bargeldabschaffung).
\item
  Die Schaffung von gesetzlichen Grundlagen für einen Zugriff und
  Eingriff in die biologischen Systeme der Bürger durch Regierungen oder
  Konzerne (durch sog. ``Pflichtimpfungen'').
\end{enumerate}

In den USA hat im April der ehemalige US-Präsident Bill Clinton die
Einführung eines nationalen Netzwerks von ``Kontaktverfolgern'' mit
Gouverneuren verschiedener Bundesstaaten
\href{https://www.youtube.com/watch?v=-Ug9XHT9JQQ}{diskutiert}. Der
Gouverneur von New York, Andrew Cuomo, kündigte daraufhin an, zusammen
mit dem Milliardär und ehemaligen Bürgermeister von New York City,
Michael Bloomberg, eine
\href{https://www.cbsnews.com/news/contact-tracing-new-york-cuomo-plan/}{``Kontaktverfolgungs-Armee''}
mit bis zu 17,000 Kontaktverfolgern für New York aufzubauen.

In Großbritannien und vielen weiteren Ländern wird derweil von
Regierungen die Einführung biometrischer ``Immunitätsausweise''
\href{https://www.msn.com/en-us/money/news/the-uk-just-published-blueprints-for-covid-19-immunity-passports-a-controversial-potential-route-out-of-lockdown/ar-BB13qr8L}{gefordert}
und als angeblich ``einziger Ausweg'' aus dem primär politisch
motivierten Lockdown dargestellt. Das britische \emph{Tony Blair
Insitute} forderte zudem den
\href{https://www.theguardian.com/world/2020/apr/24/surveillance-a-price-worth-paying-to-beat-coronavirus-says-blair-thinktank}{``Ausbau
der technologischen Überwachung''}, um ``das Coronavirus bekämpfen zu
können''.

In den USA soll das kalifornische Datenanalyse-Unternehmen Palantir eine
Schlüsselrolle beim
\href{https://www.msn.com/en-us/news/us/team-trump-turns-to-peter-thiel-s-palantir-to-track-virus/ar-BB130qfE}{Aufbau
der Datenplattform} zur Überwachung der (bereits abklingenden)
Ausbreitung des Coronavirus spielen. Palantir ist bekannt für seine
Informatikprojekte mit Geheimdiensten und dem Militär und wurde von
US-Milliardär und Trump-Unterstützer Peter Thiel gegründet.

In Israel wird die Kontaktüberwachung der Zivilbevölkerung durch den
Inlandsgeheimdienst Shin Bet
\href{https://www.techdirt.com/articles/20200402/14261944226/controversial-spyware-vendor-nso-group-is-helping-israeli-government-spy-own-citizens.shtml}{durchgeführt}
auf Basis einer Software der NSO Group, die bekannt ist für ihre
weltweit zur Überwachung von Aktivisten und Menschenrechtlern genutzten
Spionageprogramme.

Länder wie Russland und China möchten die Überwachung der Bevölkerung im
Zuge von ``Corona'' ebenfalls massiv
\href{https://www.npr.org/sections/coronavirus-live-updates/2020/04/01/825329399/moscow-launches-new-surveillance-app-to-track-residents-in-coronavirus-lockdown}{ausbauen},
werden dies aber vermutlich unabhängig von den USA tun.

Die Idee, dass eine Pandemie für den Ausbau von Überwachung und
Kontrolle der Bevölkerung genutzt werden kann, ist nicht neu: Bereits
2010 beschrieb die amerikanische Rockefeller Foundation in einem
\href{https://swprs.files.wordpress.com/2020/04/rockefeller-foundation-scenarios-2010.pdf}{Bericht
zu technologischen und gesellschaftlichen Zukunftsentwicklungen} ein
„Lock Step Szenario``, in dem die heutigen Entwicklungen überraschend
präzise antizipiert wurden (ab Seite 18). Das Szenario war damals als
eine Art autoritärer ``worst case'' gedacht.

Über 500 Wissenschaftler haben bereits in einem Offenen Brief vor einer
\href{https://www.esat.kuleuven.be/cosic/sites/contact-tracing-joint-statement/}{``beispiellosen
Überwachung der Gesellschaft''} durch Applikationen zur
Kontaktverfolgung gewarnt.

Auch das sogenannte \emph{Center for Health Security} der Johns Hopkins
Universität, das im Zentrum des Covid19-Pandemie-Managements steht und
durch seine irreführenden Darstellungen stark zur weltweiten Eskalation
beitrug, ist sehr eng
\href{http://unlimitedhangout.com/2020/04/investigative-series/all-roads-lead-to-dark-winter/}{mit
dem US-Sicherheitsapparat verbunden} und war bereits in dessen frühere
Simulationen und Operationen involviert.

Generell ist die Kooperation mit privaten Akteuren zur Erreichung
geostrategischer Ziele kein neues oder ungewöhnliches Phänomen in der
amerikanischen Außen- und Sicherheitspolitik.

Microsoft-Gründer Bill Gates, der
\href{https://www.youtube.com/watch?v=wQSYdAX_9JY}{wichtigste private
Sponsor} von WHO, Impf­stoff­industrie und biometrischen Projekten,
finanzierte z.B. bereits 2003 ein
\href{https://www.cfr.org/news-releases/council-establishes-senior-fellowship-global-health-and-foreign-policy-grant-bill}{Global
Health Program} des \emph{US Council on Foreign Relations}, bei dem es
um die Frage geht, wie Gesundheitspolitik die Geopolitik beeinflusst und
umgekehrt für die Erreichung geostrategischer Ziele genutzt werden kann.

\hypertarget{25-april-2020}{%
\paragraph{25. April 2020}\label{25-april-2020}}

\hypertarget{medizinische-updates}{%
\subparagraph{\texorpdfstring{\textbf{Medizinische
Updates}}{Medizinische Updates}}\label{medizinische-updates}}

\begin{itemize}
\tightlist
\item
  Professor Detlef Krüger, der direkte Vorgänger des bekannten deutschen
  Virologen Christian Drosten an der Charité-Klinik Berlin, erklärt
  \href{https://de.sputniknews.com/interviews/20200425326953541-corona-gefahr-virologe/}{in
  einem neuen Interview}, dass Covid19 ``in vieler Hinsicht mit der
  Grippe vergleichbar'' und ``nicht gefährlicher als bestimmte Varianten
  des Grippevirus'' sei. Der ``von der Politik entdeckte
  Mund-Nasen-Schutz'' hält Professor Krüger für ``Aktionismus'' und eine
  potentielle ``Keimschleuder''. Zugleich warnt er vor ``massiven
  Kollateralschäden'' durch die getroffenen Maßnahmen.
\item
  Der ehemalige schwedische und europäische Chefepidemiologe Professor
  Johan Giesecke gab dem österreichischen Magazin Addendum ein
  \href{https://www.addendum.org/coronavirus/interview-johan-giesecke/}{sehr
  offenes Interview}. Professor Giesecke sagt, 75 bis 90\% der Epidemie
  sei ``unsichtbar'', weil so viele Personen keine oder kaum Symptome
  entwickeln. Ein Lockdown sei daher ``sinnlos'' und schade der
  Gesellschaft. Die Grundlage der schwedischen Strategie sei gewesen,
  dass ``die Leute nicht dumm sind''. Giesecke rechnet mit einer
  Sterberate zwischen 0.1 und 0.2\%, ähnlich einer Influenza. Italien
  und New York seien sehr schlecht auf das Virus vorbereitet gewesen und
  hätten ihre Risikogruppen nicht geschützt.
\item
  Die
  \href{https://www.epicentro.iss.it/coronavirus/bollettino/Bollettino-sorveglianza-integrata-COVID-19_16-aprile-2020.pdf\#page=13}{neuesten
  Zahlen aus Italien} zeigen (S. 12/13), dass von knapp 17,000 positiv
  getesten Ärzten und Krankenpflegern 60 verstarben. Bei den unter
  50-Jährigen ergibt sich daraus eine Covid19-Letalität von unter 0.1\%,
  bei den 50- bis 60-Jährigen von 0.27\%, bei den 60- bis 70-Jährigen
  von 1.4\%, und bei den 70- bis 80-Jährigen von 12.6\%. Selbst diese
  Werte dürften noch zu hoch sein, da es sich um Todesfälle \emph{mit}
  und nicht unbedingt \emph{durch} Coronaviren handelt, und da bis zu
  80\% der Personen asymptomatisch bleiben und einige von ihnen
  womöglich nicht getestet wurden. Insgesamt stimmen die Werte jedoch
  mit jenen z.B. aus Südkorea überein und ergeben für die
  Allgemeinbevölkerung eine Sterblichkeit im Bereich der Influenza.
\item
  Der Chef des italienischen Zivilschutzes
  \href{https://www.theguardian.com/world/2020/apr/16/italian-police-broaden-care-home-coronavirus-milan}{erklärte
  Mitte April}, dass in der Lombardei über 1800 Menschen in Pflegeheimen
  starben und die Todesursache in vielen Fällen noch nicht klar sei.
  Bereits zuvor wurde bekannt, dass die Alters- und Pflegeversorgung und
  in der Folge die gesamte Krankenversorgung in Teilen der Lombardei
  unter anderem aufgrund der Angst vor dem Virus und dem Lockdown
  \href{https://swprs.org/covid19-bericht-aus-italien/}{zusammengebrochen
  war}.
\item
  Die
  \href{https://covid-19.sciensano.be/sites/default/files/Covid19/Meest\%20recente\%20update.pdf}{neuesten
  Zahlen aus Belgien} zeigen, dass sich auch dort etwas über 50\% aller
  zusätzlichen Todesfälle in Alters- und Pflegeheimen ereigneten, die
  durch einen allgemeinen Lockdown nicht besser geschützt werden. Bei
  6\% dieser Todesfälle war Covid19 ``bestätigt'', bei 94\% der
  Todesfälle wurde es ``vermutet''. Etwa 70\% der testpositiven Personen
  (Mitarbeiter und Bewohner) zeigten zum Zeitpunkt des Tests keine
  Symptome.
\item
  Der deutsche Impfexperte Professor Dr. Siegwart Bigl
  \href{https://www.pressreader.com/germany/dresdner-neueste-nachrichten/20200423/281496458428447}{hält
  den Coronaschutz für ``überzogen''}. Es liege ``keine Pandemie'' (mit
  besonders vielen Todesfällen) vor, der Lockdown sei unnötig und falsch
  gewesen. Der Vergleich mit der Influenza sei durchaus zulässig.
\item
  Der britische Guardian
  \href{https://www.theguardian.com/environment/2020/apr/20/air-pollution-may-be-key-contributor-to-covid-19-deaths-study?utm_medium}{zitiert
  eine neue Studie}, wonach die Luftverschmutzung ein
  ``Schlüsselfaktor'' für Covid19-Todesfälle sein könnte. So seien 80\%
  der Todesfälle in vier Ländern in den am stärksten verschmutzten
  Regionen erfolgt (darunter die Lombardei und Madrid).
\item
  Der kalifornische Arzt Dr. Dan Erickson berichtete in
  einer~\href{https://www.turnto23.com/news/coronavirus/video-interview-with-dr-dan-erickson-and-dr-artin-massihi-taken-down-from-youtube}{vielbeachteten
  Pressekonferenz} von seinen bisherigen Beobachtungen bezüglich
  Covid19. In Kalifornien und anderen Bundesstaaten seien die
  Krankenhäuser und Intensivstationen bisher weitgehend leer geblieben.
  Dr. Erickson berichtet von Ärzten aus mehreren US-Bundesstaaten, die
  ``unter Druck gesetzt'' wurden, Totenscheine auf Covid19 auszustellen,
  obwohl sie selbst nicht dieser Ansicht waren. Dr. Erickson empfiehlt,
  nur die Kranken und nicht die Gesunden bzw. die ganze Gesellschaft
  unter Quarantäne zu stellen, da dies auch negative Auswirkungen auf
  die Gesundheit und Psyche haben könne. So sei bereits eine deutliche
  Zunahme von ``Sekundäreffekten'' wie Alkoholismus, Depressionen,
  Suiziden sowie Missbrauch von Kindern und Ehepartnern zu beobachten.
  Dr. Erickson schätzt die Letalität von Covid-19 aufgrund der
  bisherigen Zahlen aus verschiedenen Ländern auf ca. 0.1\%, ähnlich
  einer Influenza. Ein Mundschutz mache nur in akuten Situationen wie im
  Krankenhaus Sinn, nicht jedoch im Alltag. (\textbf{Hinweis}: Youtube
  löschte die Pressekonferenz nach über 5 Millionen Views.)
\item
  Die deutsche ZEIT thematisiert die
  \href{https://www.zeit.de/2020/18/kliniken-coronavirus-intensivbetten-patienten-behandlung-notaufnahme}{hohen
  Leerstände in deutschen Kliniken}, die in manchen Abteilungen bis zu
  70\% betragen. Selbst Krebsuntersuchungen und nicht akut
  überlebens­notwendige Organ­trans­plan­tationen seien abgesagt worden,
  um Platz für Covid19-Patienten zu schaffen, die bisher jedoch
  größteneils ausblieben.
\item
  Eine neue Analyse aus Großbritannien
  \href{https://www.telegraph.co.uk/global-health/science-and-disease/two-new-waves-deaths-break-nhs-new-analysis-warns/}{kommt
  zum Ergebnis}, dass dort derzeit circa 2000 Personen pro Woche
  \emph{ohne Covid19} zuhause sterben, weil sie das Gesundheitssystem
  nicht nutzen können oder möchten. Es handelt sich dabei insbesondere
  um Notfallpatienten mit Herzinfarkten und Hirnschlägen sowie um
  chronisch erkrankte Menschen.
\item
  Forscher in Österreich sind
  \href{https://academic.oup.com/eurheartj/advance-article/doi/10.1093/eurheartj/ehaa314/5820829}{zum
  Ergebnis gelangt}, dass dort im März mehr Menschen an einem
  unbehandelten Herzinfarkt starben als an Covid19.
\item
  In Deutschland wurde eine Maskenpflicht im öffentlichen Nahverkehr und
  in Einzel­handels­geschäften eingeführt. Der Präsident des
  Weltärzteverbands, Frank Montgomery, hat dies als ``falsch'' und die
  vorgesehene Verwendung von Schals und Tüchern als ``lächerlich''
  \href{https://www.aerztezeitung.de/Politik/Montgomery-haelt-Maskenpflicht-fuer-falsch-408844.html}{kritisiert}.
  Tatsächlich zeigen Studien, dass die Verwendung von Masken im Alltag
  bei gesunden und asymptomatischen Menschen keinen messbaren Nutzen
  bringt, weshalb der Schweizer Infektiologe Dr. Vernazza von einem
  \href{https://infekt.ch/2020/04/atemschutzmasken-fuer-alle-medienhype-oder-unverzichtbar/}{``Medienhype''}
  sprach. Andere Kritiker sprechen von einem Symbol des
  \href{https://multipolar-magazin.de/artikel/maskenpflicht-gesellschaftliches-klima}{``erzwungenen,
  öffentlich sichtbaren Gehorsams''}.
\item
  Eine WHO-Studie fand 2019 für die Wirksamkeit von Maßnahmen wie
  ``Social Distancing'', Reisebeschränkungen und Ausgangssperren
  \href{https://www.heise.de/tp/features/COVID-19-WHO-Studie-findet-kaum-Belege-fuer-die-Wirksamkeit-von-Eindaemmungsmassnahmen-4706446.html}{``wenig
  bis keine wissenschaftlichen Belege''}.
\item
  Ein deutsches Labor
  \href{http://www.labor-augsburg-mvz.de/de/aktuelles/coronavirus}{erklärte
  Anfang April}, dass Covid19-Virentests laut Empfehlung der WHO neu
  auch dann als positiv gelten, wenn die spezifische Zielsequenz des
  Covid19-Virus \emph{negativ} und nur die \emph{allgemeinere}
  Coronaviren-Zielsequenz positiv ist. Dies kann jedoch dazu führen,
  dass auch andere Coronaviren (Erkältungsviren) ein falsches positives
  Testergebnis auslösen. Das Labor erklärte außerdem, dass
  Covid19-Antikörper oftmals erst zwei bis drei Wochen nach
  Symptombeginn nachweisbar sind. Dies muss berücksichtigt werden, damit
  die tatsächliche Anzahl der Personen, die bereits gegen Covid19 immun
  ist, nicht unterschätzt wird.
\item
  Sowohl in der
  \href{https://www.20min.ch/schweiz/news/story/-rzte-und-Politiker-fordern-Corona-Impfzwang-20853917}{Schweiz}
  als auch in
  \href{https://www.t-online.de/nachrichten/deutschland/id_87757340/corona-krise-markus-soeder-spricht-sich-fuer-deutschlandweite-impfpflicht-aus.html}{Deutschland}
  haben einzelne Politiker eine ``Impfpflicht gegen Corona'' gefordert.
  Allerdings führte beispielsweise die Impfung gegen die sogenannte
  ``Schweinegrippe'' von 2009/2010 zu teilweise
  \href{https://www.ibtimes.co.uk/brain-damaged-uk-victims-swine-flu-vaccine-get-60-million-compensation-1438572}{schweren
  neurologischen Schäden} insbesondere bei Kindern und zu
  Schaden­­ersatz­forderungen in Millionenhöhe.
\item
  Professor Christof Kuhbandner:
  \href{https://www.heise.de/tp/features/Von-der-fehlenden-wissenschaftlichen-Begruendung-der-Corona-Massnahmen-4709563.html?seite=all}{Von
  der fehlenden wissenschaftlichen Begründung der Corona-Maßnahmen}:
  ``Die berichteten Zahlen zu den Neuinfektionen überschätzen die wahre
  Ausbreitung des Coronavirus sehr dramatisch. Der beobachtete rasante
  Anstieg in den Neuinfektionen geht fast ausschließlich auf die
  Tatsache zurück, dass die Anzahl der Tests mit der Zeit rasant
  gestiegen ist (siehe Abbildung unten). Es gab also zumindest laut den
  berichteten Zahlen in Wirklichkeit nie eine exponentielle Ausbreitung
  des Coronavirus. Die berichteten Zahlen zu den Neuinfektionen
  verbergen die Tatsache, dass die Anzahl der Neuinfektionen bereits
  seit in etwa Anfang bis Mitte März sinkt.''
\end{itemize}

\includegraphics{https://swprs.files.wordpress.com/2020/04/zunahme-infektionen-tests-tag.png?w=550\&h=404}

\hypertarget{schweden-mediendarstellung-versus-realituxe4t}{%
\subparagraph{\texorpdfstring{\textbf{Schweden: Mediendarstellung versus
Realität}}{Schweden: Mediendarstellung versus Realität}}\label{schweden-mediendarstellung-versus-realituxe4t}}

Einige Leser waren überrascht über die Abnahme der Todesfälle in
Schweden, denn in den meisten Medien wird eine steil \emph{ansteigende}
Kurve gezeigt. Woran liegt das? Die meisten Medien zeigen
\emph{kumulierte} Zahlen nach \emph{Meldedatum}, während die
schwedischen Behörden die deutlich aussagekräftigeren \emph{täglichen}
Zahlen nach \emph{Todesdatum} veröffentlichen.

Die schwedischen Behörden betonen stets, dass die neu gemeldeten Fälle
\emph{nicht} alle innerhalb der letzten 24 Stunden verstorben sind, doch
viele Medien ignorieren das (siehe Grafik unten). Die jüngsten
schwedischen Zahlen können zwar wie in allen Ländern noch etwas
zunehmen, aber am grundsätzlich abnehmenden Trend ändert das nichts
mehr.

Hinzu kommt noch, dass auch diese Zahlen Todesfälle \emph{mit} und nicht
zwingend \emph{an} Coronaviren darstellen. Das Durchschnittsalter liegt
auch in Schweden bei über 80 Jahren, ca. 50\% der Todesfälle geschahen
in Pflegeheimen, der Effekt auf die Allgemeinbevölkerung blieb minimal,
obschon Schweden über eine der tiefsten
\href{https://link.springer.com/article/10.1007/s00134-012-2627-8}{intensivmedizinischen
Kapazitäten} Europas verfügt.

Allerdings hat auch die schwedische Regierung im Zuge von ``Corona''
neue
\href{https://www.tagesschau.de/faktenfinder/ausland/corona-kursaenderung-schweden-103.html}{Notstands­befug­nisse}
erhalten und könnte an späteren Programmen zu Kontakt­verfolgungen
teilnehmen.

\includegraphics{https://swprs.files.wordpress.com/2020/04/sweden-corona-media-vs-reality.png?w=736\&h=338}

\hypertarget{situation-in-grouxdfbritannien}{%
\subparagraph{\texorpdfstring{\textbf{Situation in
Großbritannien}}{Situation in Großbritannien}}\label{situation-in-grouxdfbritannien}}

Die Todesfälle in Großbritannien sind in den letzten Wochen sehr stark
angestiegen, bewegen sich indes immer noch im Bereich der
\href{http://inproportion2.talkigy.com/}{schwersten Grippewellen} der
letzten fünfzig Jahre (siehe Grafik unten). Auch in Großbritannien
fallen
\href{https://ltccovid.org/2020/04/12/mortality-associated-with-covid-19-outbreaks-in-care-homes-early-international-evidence/}{bis
zu 50\%} der zusätzlichen Todesfälle in Alters- und Pflegeheimen an, die
von einem allgemeinen Lockdown nicht profitieren.

Bei
\href{https://www.thetimes.co.uk/edition/news/coronavirus-record-weekly-death-toll-as-fearful-patients-avoid-hospitals-bm73s2tw3}{bis
zu 50\%} der zusätzlichen Todesfälle soll es sich zudem \emph{nicht} um
Covid19-Verdachtsfälle handeln, und
\href{https://www.ft.com/content/67e6a4ee-3d05-43bc-ba03-e239799fa6ab}{bis
zu 25\%} der zusätzlichen Todesfälle ereignen sich zuhause. Auch in
Großbritannien stellt sich mithin die Frage, ob der allgemeine Lockdown
mehr nützt oder mehr schadet.

Frasor Nelson, der Editor des britischen \emph{Spectator},
\href{https://www.telegraph.co.uk/politics/2020/04/09/boris-worried-lockdown-has-gone-far-can-end/}{berichtet
davon}, dass Regierungs­stellen durch den Lockdown mittelfristig mit bis
zu 150,000 zusätzlichen Todesfällen rechnen, deutlich mehr, als Covid19
voraussichtlich verursachen wird. Zuletzt wurde der Fall einer
17-jährigen Schülerin und Sängerin bekannt, die sich wegen des Lockdowns
\href{https://sports.yahoo.com/coronavirus-bethany-palmer-teenager-death-suicide-152707750.html}{das
Leben nahm}.

Auffallend ist, dass England im Unterschied zu anderen Ländern auch eine
deutlich erhöhte Sterblichkeit bei den 15- bis 64-Jährigen
\href{https://www.euromomo.eu/}{aufweist}. Dies könnte womöglich an den
häufigen Herz-Kreislauf-Vorerkankungen liegen, oder aber durch die
\href{https://www.telegraph.co.uk/global-health/science-and-disease/two-new-waves-deaths-break-nhs-new-analysis-warns/}{Effekte
des Lockdowns} bedingt sein.

Das~\href{http://inproportion2.talkigy.com/}{Projekt InProportion} hat
zahlreiche neue Grafiken veröffentlicht, die die aktuelle Sterblichkeit
ins Verhältnis zu früheren Grippewellen und anderen Todesursachen
setzen. Weitere Websites, die sich kritisch mit den Maßnahmen befassen,
sind \href{https://lockdownsceptics.org/}{Lockdown Skeptics} und
\href{https://www.ukcolumn.org/}{UK Column}.

\includegraphics{https://swprs.files.wordpress.com/2020/04/inproportion2_chart5.png?w=736\&h=363}

\hypertarget{schweiz-uxfcbersterblichkeit-deutlich-unter-starken-grippewellen}{%
\subparagraph{**Schweiz: Übersterblichkeit deutlich unter starken
Grippewellen}\label{schweiz-uxfcbersterblichkeit-deutlich-unter-starken-grippewellen}}

**

\begin{itemize}
\tightlist
\item
  Eine erste serologische Studie der Universität Genf
  \href{https://www.hug-ge.ch/medias/communique-presse/seroprevalence-covid-19-premiere-estimation}{kommt
  zum Ergebnis}, dass im Kanton Genf mindestens sechsmal mehr Personen
  mit Covid19 Kontakt hatten, als bisher angenommen. Damit sinkt die
  Letalität von Covid19 auch in der Schweiz in den Promillebereich,
  während offizielle Quellen teilweise noch immer von bis zu 5\%
  sprechen.
\item
  Auch im am stärksten betroffenen Kanton Tessin sind
  \href{https://www.bluewin.ch/de/news/schweiz/sp-chef-levrat-will-die-reichen-schropfen-383977.html}{knapp
  die Hälfte} der zusätzlichen Todesfälle in Altersheimen erfolgt, die
  vom allgemeinen Lockdown nicht profitieren.
\item
  In der Schweiz wurden bereits 1.85 Millionen Menschen oder über ein
  Drittel aller Arbeitnehmenden für Kurzarbeit
  \href{https://www.bluewin.ch/de/news/schweiz/sp-chef-levrat-will-die-reichen-schropfen-383977.html}{angemeldet}.
  Die volkswirtschaftlichen Kosten werden von der ETH Zürich für die
  Periode von März bis Juni 2020 auf 32 Milliarden geschätzt.
\item
  Die ETH Zürich hat ihre Studie zur Reproduktionszahl von Covid19
  \href{https://www.nau.ch/politik/regional/coronavirus-eth-forscherin-passt-studie-an-und-stutzt-lockdown-65695817}{angepasst}
  und ``unterstützt'' nun den Lockdown des Bundesrates. Am
  grundsätzlichen Ergebnis der Studie ändert dies indes nichts: Die
  Reproduktionszahl fiel bereits vor dem Lockdown auf den stabilen Wert
  von 1, einfache Alltags- und Hygienemaßnahmen genügten hierfür und der
  Lockdown war für die Eindämmung der Epidemie somit
  \href{https://infekt.ch/2020/04/sind-wir-tatsaechlich-im-blindflug/}{unnötig}.
\item
  \textbf{Infosperber}:
  \href{https://www.infosperber.ch/Artikel/Medien/Corona-NZZ-deckt-das-Nachplappern-anderer-Medien-auf}{Corona:
  NZZ deckt das Nachplappern anderer Medien auf}. ``Grosse Medien
  verschweigen, dass sie sich bei Covid-19-Zahlen auf intransparente
  Daten stützen.''
\item
  \textbf{Ktipp}:
  \href{https://www.ktipp.ch/artikel/artikeldetail/bund-fast-alle-zahlen-ohne-gewaehr/}{Bund:
  Fast alle Zahlen ohne Gewähr}. ``Dieses Jahr starben in den ersten 14
  Wochen weniger unter 65-Jährige als in den letzten fünf Jahren. Bei
  den über 65-Jährigen war die Zahl ebenfalls verhältnismässig tief.''
\end{itemize}

Die folgende Grafik zeigt, dass die Gesamtsterblichkeit in der Schweiz
im ersten Quartal 2020 im Normalbereich und bis Mitte April immer noch
rund 2000 Personen \emph{unter} der Grippewelle von 2015 lag. 50\% der
Todesfälle ereigneten sich in
\href{https://www.nzz.ch/zuerich/coronavirus-zuerich-aendert-nun-das-testregime-in-heimenauch-viele-aeltere-covid-19-infizierte-entwickeln-keine-symptome-zuerich-aendert-nun-das-testregime-in-heimen-ld.1552089}{Altersheimen},
die von einem Lockdown nicht profitieren.

Insgesamt ereigneten sich rund 75\% der zusätzlichen Todesfälle
\href{https://www.tagesspiegel.de/wissen/woran-sterben-corona-patienten-wirklich-ein-schweizer-forscher-macht-hoffnung-im-kampf-gegen-covid-19/25750666.html}{zuhause},
während Krankenhäuser und Intensivstationen weiterhin
\href{https://swprs.files.wordpress.com/2020/04/intensivbettenbelegung-schweiz-2020-04-14.png}{stark
unterbelegt} sind und zahlreiche Operationen abgesagt wurden. Auch in
der Schweiz stellt sich mithin die sehr folgenschwere Frage, ob der
``Lockdown'' nicht mehr Existenzen und Leben gekostet als gerettet haben
könnte.

\includegraphics{https://swprs.files.wordpress.com/2020/04/schweiz-todesfaelle-2010-2020.png?w=736\&h=357}

\hypertarget{politische-meldungen}{%
\subparagraph{\texorpdfstring{\textbf{Politische
Meldungen}}{Politische Meldungen}}\label{politische-meldungen}}

\begin{itemize}
\tightlist
\item
  Die Website \href{https://kollateral.news/}{kollateral.news} einer
  deutschen Fachanwältin sammelt Berichte zu ``Lockdown-Leid'' und der
  tatsächlichen Situation in deutschen Krankenhäusern.
\item
  Deutsche Hausärzte haben einen
  \href{https://aerzteinnenvorort.de/der-appell}{Appell an Politik und
  Wissenschaft} veröffentlicht, in dem sie ``einen verantwortlicheren
  Umgang mit der Corona-Krise'' fordern.
\item
  Ein Münchner Lokalradio, das im März Corona-kritische Ärzte
  interviewte, wurde nach Beschwerden von der zuständigen Medienaufsicht
  \href{https://norberthaering.de/medienversagen/radiomuenchen-blm-meinungsvielfalt/}{informiert},
  dass ``derartige problematische Sendungen zukünftig auszubleiben''
  haben.
\item
  \textbf{Video}: Im australischen Bundesstaat Queensland hat ein
  Polizeihelikopter mit Nachtsichtgerät
  \href{https://twitter.com/Independent/status/1252911273597120513}{drei
  junge Männer aufgespürt}, die nachts auf einem Hausdach ein Bier
  tranken und damit die ``Corona-Vorschriften'' verletzten. Über ein
  Megaphon wurden die Männer informiert, dass das Gebäude ``von der
  Polizei umstellt'' ist und sie sich zum Ausgang begeben müssen. Die
  Männer wurden mit je ca. \$1000
  \href{https://www.dailystar.co.uk/news/world-news/police-helicopter-uses-night-vision-21899640}{bestraft}.
\item
  Der \textbf{Corona-kritische Schweizer Arzt}, der von einem
  Sonderkommando der Schweizer Polizei verhaftet und in die Psychiatrie
  eingeliefert wurde (siehe Update vom 15. April), ist inzwischen wieder
  frei. Eine
  \href{https://uncut-news.ch/wp-content/uploads/2020/04/Wer-l\%C3\%B6ste-den-Fehlalarm-aus.pdf}{Recherche
  des Magazins Weltwoche} ergab, dass der Arzt unter Vorgabe falscher
  Gründe verhaftet wurde: es habe keine Bedrohung von Angehörigen oder
  Behörden und kein Besitz einer geladenen Waffe vorgelegen. Damit
  erscheint eine politisch oder lokalpolitisch motivierte Einweisung
  wahrscheinlich.
\item
  Sowohl in
  \href{https://www.sn.at/panorama/oesterreich/arzt-droht-berufsverbot-wegen-kritik-an-corona-massnahmen-86594140}{Österreich}
  als auch in
  \href{https://magyarhang.org/belfold/2020/04/16/etikai-vizsgalat-indul-az-orvos-ellen-aki-szerint-nincs-jarvany-es-az-idosek-csak-a-felelemtol-halnak-meg/}{Ungarn}
  droht Ärzten, die sich kritisch zu den Corona-Maßnahmen geäußert
  haben, ein Berufsverbot.
\item
  In \textbf{Nigeria} tötete die Polizei beim Durchsetzen der
  Corona-Ausgangssperren bisher laut offiziellen Angaben
  \href{https://www.bbc.com/news/world-africa-52317196}{mehr Menschen}
  als das Coronavirus selbst.
\item
  In \textbf{Israel} kann der Inlands- und Antiterror-Geheimdienst Shin
  Bet in Zusammenarbeit mit der Polizei seit Mitte März die
  Mobiltelefone der Bevölkerung
  \href{https://www.jewishpress.com/news/the-courts/state-to-high-court-even-more-shin-bet-involvement-in-fighting-the-coronavirus/2020/04/14/}{überwachen},
  um im Kontext von Covid19 Kontakte nachzuverfolgen und Hausarrest
  anzuordnen. Diese Maßnahmen wurden zunächst ohne Zustimmung des
  Parlaments verfügt und sollen vorerst bis Ende April gelten.
\item
  \textbf{OffGuardian}:
  \href{https://off-guardian.org/2020/04/23/the-seven-step-path-from-pandemic-to-totalitarianism/}{The
  Seven Step Path from Pandemic to Totalitarianism} (Deutsche
  \href{https://uncut-news.ch/wp-content/uploads/2020/04/In-sieben-Schritte-von-der-Pandemie-zum-Totalitarismus.pdf}{Übersetzung})
\item
  \textbf{UK Column}:
  \href{https://www.ukcolumn.org/article/who-controls-british-government-response-covid19-part-one}{Who
  controls the British Government response to Covid--19?}
\end{itemize}

\hypertarget{21-april-2020}{%
\paragraph{21. April 2020}\label{21-april-2020}}

\hypertarget{medizinische-updates-1}{%
\subparagraph{\texorpdfstring{\textbf{Medizinische
Updates}}{Medizinische Updates}}\label{medizinische-updates-1}}

\begin{itemize}
\tightlist
\item
  Stanford-Medizinprofessor John Ioannidis erklärt in einem
  \href{https://www.youtube.com/watch?v=cwPqmLoZA4s}{neuen einstündigen
  Interview} mehrere neue Studien zu Covid19. Die Letalität von Covid19
  liegt laut Professor Ioannidis ``im Bereich der saisonalen Grippe''.
  Für Personen unter 65 Jahren sei das Sterberisiko selbst in den
  weltweiten ``Hotspots'' vergleichbar mit der täglichen Autofahrt zur
  Arbeit, für \emph{gesunde} Personen unter 65 Jahren sei das
  Sterberisiko ``völlig vernachläßigbar''. Lediglich in New York liege
  das Sterberisiko für Personen unter 65 Jahren im Bereich eines
  professionellen LKW-Fahrers.
\item
  Professor Carl Heneghan, Direktor des Zentrums für evidenzbasierte
  Medizin der Universität Oxford,
  \href{https://news.yahoo.com/lockdown-damage-outweighs-coronavirus-warning-121940675.html}{warnt
  in einem neuen Beitrag}, dass der Schaden durch den Lockdown größer
  sein könnte als jener durch das Virus. Der Peak der Epidemie sei in
  den meisten Ländern bereits vor dem Lockdown erreicht gewesen.
\item
  Eine
  \href{http://publichealth.lacounty.gov/phcommon/public/media/mediapubhpdetail.cfm?prid=2328}{neue
  serologische Studie} im Bezirk von Los Angeles kommt zum Ergebnis,
  dass bereits 28 bis 55 mal mehr Personen Covid19 hatten als bisher
  angenommen (ohne starke Symptome zu zeigen), wodurch sich die
  Gefährlichkeit der Erkrankung entsprechend reduziert.
\item
  In der Stadt Chelsea bei Boston hatte
  \href{https://archive.is/20200418222442/https://www.bostonglobe.com/2020/04/17/business/nearly-third-200-blood-samples-taken-chelsea-show-exposure-coronavirus/}{rund
  ein Drittel von 200 Blutspendern} Antikörper gegen den
  Covid19-Erreger. Die Hälfte davon berichtete, im letzten Monat ein
  Erkältungs­symptom erlebt zu haben. In einem Obdach­losen­heim bei
  Boston wurde etwas mehr als ein Drittel der Menschen positiv getestet,
  wobei
  \href{https://www.wsbtv.com/news/trending/coronavirus-cdc-reviewing-stunning-universal-testing-results-boston-homeless-shelter/ZADQ45HCAZEVJAZA3OTCUR7M6M/}{niemand
  Symptome zeigte}.
\item
  Schottland meldet, dass die Hälfte der (aufgestockten) Intensivbetten
  weiterhin
  \href{https://www.heraldscotland.com/news/18377095.coronavirus-scotland-half-icu-beds-empty/}{leer
  stehen}. Die Aufnahme neuer Patienten würde inzwischen stagnieren.
\item
  Die Notaufnahme im städtischen Krankenhaus von Bergamo war zu Beginn
  dieser Woche erstmals seit 45 Tagen
  \href{https://orf.at/stories/3162642/}{wieder vollständig leer}.
  Inzwischen würden wieder mehr Menschen mit anderen Krankheiten als
  ``Covid19-Patienten'' behandelt.
\item
  Ein Bericht im Fachmagazin \emph{Lancet} kam bereits Anfang April
  \href{https://www.thelancet.com/journals/lanchi/article/PIIS2352-4642(20)30095-X/fulltext}{zum
  Ergebnis}, dass Schulschließungen zur Eindämmung von Coronaviren
  keinen oder einen minimalen Effekt haben.
\item
  Ein neunjähriges französisches Kind mit Corona-Infektion hatte Kontakt
  zu 172 Personen, von denen es
  jedoch~\href{https://www.n-tv.de/panorama/172-Kontaktpersonen-von-Corona-verschont-article21727469.html}{niemanden
  angesteckt hat}. Dies bestätigt frühere Ergebnisse, wonach die
  Corona-Infektion (im Unterschied zur Influenza) nicht oder kaum von
  Kindern übertragen wird.
\item
  Der deutsche emeritierte Mikrobiologie-Professor Sucharit Bhakdi gab
  ein
  \href{https://kenfm.de/kenfm-am-set-gespraech-mit-prof-dr-sucharit-bhakdi-zu-covid-19/}{neues
  einstündiges Interview} zu Covid-19. Professor Bhakdi ist der Ansicht,
  dass die meisten Medien während der Covid19-Epidemie ``völlig
  verantwortungslos'' gehandelt haben.
\item
  Die deutsche Initiative für Pflegeethik
  \href{http://pflegeethik-initiative.de/2020/04/15/corona-krise-falsche-prioritaeten-gesetzt-und-ethische-prinzipien-verletzt/}{kritisiert
  pauschale Besuchsverbote und leidvolle Intensivbehandlungen} von
  Pflegepatienten: ``Schon vor Corona starben jeden Tag in deutschen
  Heimen rund 900 alte, pflegebedürftige Menschen, ohne nochmals kurz
  vorher ins Krankenhaus verbracht zu werden. Tatsächlich wäre bei
  diesen, falls überhaupt, eher palliative Behandlung angezeigt. () Nach
  allem was wir bisher zu Corona wissen, gibt es nicht einen einzigen
  plausiblen Grund, den Infektionsschutz weiterhin höher zu bewerten,
  als die Grundrechte der Bürger. Heben Sie die Besuchsverbote auf!
  Diese sind unmenschlich und unnötig!''
\item
  Die älteste Frau des Schweizer Kantons St. Gallen ist vergangene Woche
  mit 109 Jahren verstorben. Sie überlebte die ``Spanische Grippe'' von
  1918, war nicht Corona-infiziert und sei ``für ihr Alter sehr gut
  unterwegs gewesen''. Die ``coronabedingte Isolation'' habe ihr jedoch
  \href{https://swprs.files.wordpress.com/2020/04/tagblatt-109.jpg}{``sehr
  zugesetzt''}: ``Sie verkümmerte ohne die täglichen Besuche ihrer
  Familienangehörigen.''
\item
  Der Schweizer Kardiologe Dr. Nils Kucher berichtet, dass in der
  Schweiz derzeit rund 75\% aller zusätzlichen Todesfälle nicht im
  Krankenhaus,
  \href{https://www.tagesspiegel.de/wissen/woran-sterben-corona-patienten-wirklich-ein-schweizer-forscher-macht-hoffnung-im-kampf-gegen-covid-19/25750666.html}{sondern
  zuhause erfolgen}. Dies erklärt sicherlich die
  \href{https://swprs.files.wordpress.com/2020/04/intensivbettenbelegung-schweiz-2020-04-14.png}{weitgehend
  leeren} Schweizer Krankenhäuser und Intensivstationen. Außerdem ist
  bereits bekannt, dass rund 50\% aller zusätzlichen Todesfälle in
  Alters- und Pflegeheimen
  \href{https://www.nzz.ch/zuerich/coronavirus-zuerich-aendert-nun-das-testregime-in-heimenauch-viele-aeltere-covid-19-infizierte-entwickeln-keine-symptome-zuerich-aendert-nun-das-testregime-in-heimen-ld.1552089}{erfolgen}.
  Dr. Kucher vermutet, dass ein Teil dieser Personen an einer
  plötzlichen Lungenembolie versterben. Das ist denkbar. Dennoch stellt
  sich die Frage, welche Rolle der ``Lockdown'' bei diesen zusätzlichen
  Todesfällen spielt.
\item
  Die italienische Gesundheitsbehörde ISS
  \href{https://www.iss.it/en/rapporti-covid-19/-/asset_publisher/btw1J82wtYzH/content/id/5334891}{warnt},
  dass Covid19-Patienten aus dem Mittelmeerraum, die häufig eine
  genetische Stoffwechselbesonderheit namens Favismus aufweisen, nicht
  mit Malariamitteln wie Chloroquine behandelt werden sollten, da diese
  bei Favismus zum Tod führen können. Es ist dies ein weiterer Hinweis
  darauf, dass eine falsche oder zu aggressive Medikation die Krankheit
  \href{https://www.sciencedaily.com/releases/2020/02/200206110703.htm}{zusätzlich
  verschlimmern} kann.
\item
  Rubikon:
  \href{https://www.rubikon.news/artikel/120-expertenstimmen-zu-corona}{120
  Expertenstimmen zu Corona}. Weltweit kritisieren hochrangige
  Wissenschaftler, Ärzte, Juristen und andere Experten den Umgang mit
  dem Coronavirus.
\end{itemize}

\hypertarget{einstufung-der-pandemie}{%
\subparagraph{\texorpdfstring{\textbf{Einstufung der
Pandemie}}{Einstufung der Pandemie}}\label{einstufung-der-pandemie}}

Die US-Gesundheitsbehörden haben 2007 eine
\href{https://www.cidrap.umn.edu/news-perspective/2007/02/hhs-ties-pandemic-mitigation-advice-severity}{fünfstufige
Einteilung} für Grippe-Pandemien und entsprechende Maßnahmen definiert.
Die fünf Kategorien richten sich nach der beobachteten Letalität (CFR)
der Pandemie, von Kategorie 1 (\textless{}0.1\%) bis Kategorie 5
(\textgreater{}2\%). Die aktuelle Corona-Pandemie wäre nach diesem
Schlüssel derzeit vermutlich in Kategorie 2 (0.1\% bis 0.5\%)
einzuordnen. Für diese Kategorie war damals lediglich die ``freiwillige
Isolierung kranker Personen'' als Hauptmaßnahme vorgesehen.

Die WHO strich 2009 allerdings
\href{https://www.forbes.com/2010/02/05/world-health-organization-swine-flu-pandemic-opinions-contributors-michael-fumento.html\#5ae32fb848e8}{die
Letalität aus ihrer Pandemie-Definition}, sodass seither im Prinzip jede
weltweite Grippewelle zu einer Pandemie erklärt werden kann, wie dies
mit der sehr milden ``Schweingegrippe'' von 2009/2010 erstmals geschah,
für die Impfstoffe im Wert von rund 18 Milliarden Dollar an Regierungen
verkauft wurden.

Die Dokumentation TrustWHO (``Vertraue wem?''), die die zweifelhafte
Rolle der WHO im Rahmen der ``Schweinegrippe'' thematisierte,
\href{https://www.youtube.com/watch?v=VjQGyqVN5RM}{wurde zuletzt von
VIMEO gelöscht}.

\hypertarget{chefarzt-pietro-vernazza-einfache-mauxdfnahmen-genuxfcgen}{%
\subparagraph{\texorpdfstring{\textbf{Chefarzt Pietro Vernazza: Einfache
Maßnahmen
genügen}}{Chefarzt Pietro Vernazza: Einfache Maßnahmen genügen}}\label{chefarzt-pietro-vernazza-einfache-mauxdfnahmen-genuxfcgen}}

Der Schweizer Chefarzt für Infektiologie, Pietro Vernazza, zeigt
\href{https://infekt.ch/2020/04/sind-wir-tatsaechlich-im-blindflug/}{in
seinem neusten Beitrag} anhand der Resultate des deutschen
Robert-Koch-Instituts und der ETH Zürich, dass die Covid19-Epidemie
bereits vor Einführung der ``Lockdowns'' unter Kontrolle war:

``Diese Resultate enthalten Zündstoff: Offenbar zeigen nun diese beiden
Arbeiten mehr oder weniger identisch: Die einfachen Massnahmen, Verzicht
auf Grossveranstaltungen und die Einführung von Hygienemassnahmen sind
hoch wirksam. Die Bevölkerung ist in der Lage, diese Empfehlungen gut
umzusetzen und die Massnahmen können die Epidemie fast zum Stoppen
bringen. Auf jeden Fall sind die Massnahmen ausreichend, unser
Gesundheitssystem so zu schonen, dass die Spitäler nicht überlastet
werden.''

\includegraphics{https://swprs.files.wordpress.com/2020/04/ch-reproduktionszahl-eth-infekt.png?w=650\&h=379}

\hypertarget{schweiz-kumulierte-gesamtsterblichkeit-im-normalbereich}{%
\subparagraph{**Schweiz: Kumulierte Gesamtsterblichkeit im
Normalbereich}\label{schweiz-kumulierte-gesamtsterblichkeit-im-normalbereich}}

**

In der Schweiz lag die
\href{https://swprs.files.wordpress.com/2020/04/ch-sterblichkeit-kumuliert-q1-2020.pdf}{\emph{kumulierte}
Gesamtsterblichkeit} im ersten Quartal (bis 5. April) beim mittleren
Erwartungswert und über 1500 Personen \emph{unter} dem oberen
Erwartungswert. Die Gesamtsterblichkeit lag bis Mitte April zudem über
2000 Personen \emph{unter} dem Vergleichswert aus der schweren
Grippesaison von 2015 (siehe Abbildung). Weiterhin offen bleibt die
Frage, wie sich die Zunahme der Sterblichkeit während des Lockdowns
genau zusammensetzt.

\includegraphics{https://swprs.files.wordpress.com/2020/04/schweiz-todesfaelle-2010-2020.png?w=700\&h=339}

\hypertarget{schweden-epidemie-ohne-lockdown-zu-ende}{%
\subparagraph{\texorpdfstring{\textbf{Schweden: Epidemie ohne Lockdown
zu
Ende}}{Schweden: Epidemie ohne Lockdown zu Ende}}\label{schweden-epidemie-ohne-lockdown-zu-ende}}

Die neuesten Zahlen zu Patienten und Todesfällen zeigen, dass die
Epidemie in Schweden dem Ende entgegen geht. Auch in Schweden entstand
die Übersterblichkeit hauptsächlich in Pflegeheimen, die man nicht gut
genug geschützt habe, wie der Chefepidemiologe
\href{https://www.washingtontimes.com/news/2020/apr/15/sweden-coronavirus-rates-easing-despite-loose-rule/}{erklärte}.

Die schwedische Bevölkerung profitiert nun im Vergleich zu anderen
Ländern zudem von einer hohen natürlichen Immunität gegen das
Covid19-Virus, die sie vor einer möglichen ``zweiten Welle'' im nächsen
Winter insgesamt besser schützen dürfte.

Es ist davon auszugehen, dass Covid19 in der schwedischen
Gesamtsterblichkeit 2020 nicht sichtbar sein wird. Das schwedische
Beispiel
\href{https://www.kleinezeitung.at/international/corona/5802224/Anzeichen-fuer-Entspannung_Schweden-sieht-sich-auf-dem-richtigen-Weg}{zeigt},
dass ``Lockdowns'' medizinisch unnötig oder sogar kontraproduktiv sowie
gesellschaftlich und ökonomisch verheerend waren.

\textbf{Video}: \href{https://www.youtube.com/watch?v=bfN2JWifLCY}{Why
lockdowns are the wrong policy -- Swedish expert Professor Johan
Giesecke}

\includegraphics{https://swprs.files.wordpress.com/2020/04/sweden-deaths-day-2.png?w=736\&h=293}

\hypertarget{anekdoten-vs-evidenz}{%
\subparagraph{\texorpdfstring{\textbf{Anekdoten vs.
Evidenz}}{Anekdoten vs. Evidenz}}\label{anekdoten-vs-evidenz}}

Angesichts fehlender wissenschaftlicher Evidenz setzen manche Medien
vermehrt auf schauerliche Anekdoten, um die Angst in der Bevölkerung
aufrechtzuerhalten. Ein typisches Beispiel sind angeblich an Covid19
verstorbene ``gesunde Kinder'', bei denen sich später meist
herausstellt, dass sie
\href{https://www.dailymail.co.uk/news/article-8193487/Coroner-refuses-rule-COVID-19-cause-death-six-week-old-Connecticut-baby.html}{doch
nicht} an Covid19 starben oder aber
\href{https://www.msn.com/de-ch/news/other/spanischer-nachwuchs-trainer-stirbt-an-corona/ar-BB11gT64}{schwer
vorerkrankt} waren.

Österreichische Medien berichteten zuletzt von
\href{https://www.rainews.it/tgr/tagesschau/articoli/2020/04/tag-Coronavirus-Lungeschaden-Forschung-Uniklinik-Innsbruck-6708e11e-28dc-4843-a760-e7f926ace61c.html}{einigen
Tauchern}, die sechs Wochen nach einer Covid19-Erkrankung mit
Lungenbeteiligung noch verminderte Leistungswerte und eine auffällige
Bildgebung aufwiesen. In einem Abschnitt wird von ``irreversiblen
Schäden'' gesprochen, im nächsten erklärt, dies sei ``unklar und
spekulativ''. Unerwähnt bleibt, dass Taucher nach einer Lungenentzündung
generell 6 bis 12 Monate
\href{https://www.gesundheitsfrage.net/g/frage/tauchen-lungenentzuendung}{pausieren}
sollten.

Häufig werden auch neurologische Effekte wie der temporäre Verlust des
Geruchs- oder Geschmackssinnes angeführt. Auch hier bleibt zumeist
unerwähnt, dass dies ein
\href{https://www.ncbi.nlm.nih.gov/pubmed/25294743}{bekannter Effekt}
von Erkältungs- und Grippeviren ist, und Covid19 hierbei
\href{https://www.ncbi.nlm.nih.gov/pubmed/23948436}{eher mild} auffällt.

In anderen Berichten werden mögliche Auswirkungen auf verschiedene
Organe wie Nieren, Leber oder Gehirn hervorgehoben, ohne zu erwähnen,
dass viele der betroffenen Patienten bereits sehr alt waren und schwere
chronische Vorerkrankungen
\href{https://www.epicentro.iss.it/coronavirus/sars-cov-2-decessi-italia}{hatten}.

\hypertarget{politische-updates}{%
\subparagraph{\texorpdfstring{\textbf{Politische
Updates}}{Politische Updates}}\label{politische-updates}}

\begin{itemize}
\tightlist
\item
  WOZ:
  \href{https://www.woz.ch/2016/grundrechte/wenn-die-angst-regiert}{Wenn
  die Angst regiert.} ``Mit Drohnen, Apps und Demoverboten: Im Zuge der
  Coronakrise werden grundlegende Freiheiten ausgehebelt. Passen wir
  nicht auf, bleiben sie es auch nach dem Lockdown -- doch die
  Extremsituation bietet auch Grund zur Hoffnung.''
\item
  Multipolar:
  \href{https://multipolar-magazin.de/artikel/die-massnahmen-wirken}{Welche
  Agenda wird hier verfolgt?} ``Die Regierung lobt sich selbst,
  verbreitet Durchhalteparolen und bremst zugleich beim Erheben
  grundlegender Daten, mit denen sich die Verbreitung und Gefährlichkeit
  des Virus verlässlich messen ließe. Schnell und entschlossen handeln
  die Behörden dagegen beim Ausbau von fragwürdigen Instrumenten, wie
  neuen „Corona-Apps`` zur kollektiven Pulsmessung und
  Kontaktverfolgung.''
\item
  Professor Christian Piska, Experte für öffentliches Recht und Legal
  Tech in Wien: ``Österreich ist anders geworden. Sehr anders, auch wenn
  die meisten es einfach so hinzunehmen scheinen. Schrittweises
  Hochfahren der Wirtschaft hin oder her -- wir leben urplötzlich mit
  polizeistaatlichen Verhältnissen und gravierenden Einschränkungen
  unserer Grund- und Menschenrechte, die diktatorischen Regimes bestens
  anstünden. ()~ Das ist die Büchse der Pandora, die -- einmal geöffnet
  -- eventuell
  \href{https://kurier.at/meinung/das-juristische-totschlagargument-vom-menschenleben/400814570}{nie
  mehr geschlossen werden kann}.''
\item
  Mehr als 300 Wissenschaftler aus 26 Ländern warnen vor einer
  \href{https://www.golem.de/news/corona-app-300-wissenschaftler-warnen-vor-zentraler-datenspeicherung-2004-147973.html}{``beispiellosen
  Überwachung der Gesellschaft''} durch nichtdatenschutzkonforme
  Corona-Apps. Mehrere Wissenschaftler und Universitäten, darunter die
  ETH Zürich und EPFL-Professor Marcel Salathé, sind inzwischen aus dem
  europäischen Kontaktverfolgungs-Projekt PEPP-PT wegen mangelnder
  Transparenz ausgestiegen. Zuletzt wurde bekannt, dass die Schweizer
  Firma AGT in das Projekt involviert ist, die zuvor
  Massen­über­wachungs­systeme für arabische Staaten aufgebaut hatte.
\item
  In Israel demonstrierten rund 5000 Menschen (mit jeweils 2m Abstand)
  \href{https://edition.cnn.com/2020/04/20/middleeast/israel-protest-social-distancing-intl/index.html}{gegen
  die Maßnahmen der Regierung Netanjahu}: ``Sie sprechen von einer
  exponentiellen Zunahme der Coronafälle, aber das einzige was
  exponentiell zunimmt, das sind die Menschen die aufstehen um unser
  Land und unsere Demokratie zu schützen.''
\item
  Der in Madrid lebende irische Journalist Jason O'Toole
  \href{https://www.rt.com/op-ed/486350-spain-tough-rules-covid-19-lockdown/}{beschreibt
  die Situation in Spanien}: ``Mit dem Militär auf den Straßen ist es
  schwer, nicht von Kriegsrecht zu sprechen. George Orwells Big Brother
  ist hier lebendig und wohlauf, und die spanische Polizei überwacht
  jeden mit Hilfe von Überwachungskameras oder durch Drohnen. () Allein
  in den ersten vier Wochen wurden 650.000 Menschen zu einer Geldstrafe
  verurteilt und 5.568 verhaftet. Ich war schockiert, als ich einen
  Videoclip sah, in dem ein Polizist einen psychisch kranken jungen Mann
  mit schwerer Gewalt festnahm, der offenbar gerade mit Brot nach Hause
  ging.''
\item
  OffGuardian:
  \href{https://off-guardian.org/2020/04/18/the-disturbing-developments-in-uk-policing/}{Die
  beunruhigenden Entwicklungen bei der britischen Polizei.}
\item
  US-Investigativjournalistin Whitney Webb beschreibt in einem neuen
  Beitrag
  \href{https://www.thelastamericanvagabond.com/top-news/techno-tyranny-how-us-national-security-state-using-coronavirus-fulfill-orwellian-vision/}{``Wie
  der nationale Sicherheitsstaat der USA das Coronavirus nutzt, um eine
  orwellsche Vision zu verwirklichen''}: ``Im vergangenen Jahr forderte
  eine US-Regierungs­­kommission, ein mit künstlicher Intelligenz
  gesteuertes Massen­über­wachungs­­system einzuführen, das weit über
  das in jedem anderen Land verwendete hinausgeht, um die amerikanische
  Hegemonie im Bereich der künstlichen Intelligenz zu sichern. Nun
  werden unter dem Deckmantel der Bekämpfung der Coronavirus-Krise viele
  der identifizierten ``Hindernisse'' zur Einführung dieses Systems
  rasch beseitigt.''
  (\href{https://www.konjunktion.info/2020/04/techno-tyrannei-wie-der-nationale-sicherheitsstaat-der-usa-den-coronavirus-einsetzt-um-eine-orwellsche-vision-zu-verwirklichen/}{Deutsche
  Version})
\item
  In einem
  \href{https://www.thelastamericanvagabond.com/top-news/all-roads-lead-dark-winter/}{früheren
  Beitrag} befasste sich Whitney Webb bereits mit der zentralen Rolle
  des ``Zentrums für Gesundsheits-Sicherheit'' der Johns Hopkins
  Universität in der aktuellen Pandemie sowie dessen Rolle in früheren
  Pandemie- und Biowaffen-Simulationen und dessen enge Verbindungen zum
  US-Sicherheits- und Militärapparat.
  (\href{https://www.theblogcat.de/uebersetzungen/dark-winter-01-04-2020/}{Deutsche
  Übersetzung})
\item
  Die Idee, dass eine Pandemie für den Ausbau weltweiter Überwachungs-
  und Kontroll­instrumente genutzt werden kann, ist nicht neu. Bereits
  2010 beschrieb die amerikanische Rockefeller Foundation in einem
  \href{https://swprs.files.wordpress.com/2020/04/rockefeller-foundation-scenarios-2010.pdf}{Arbeitspapier
  zu technologischen und gesellschaftlichen Zukunftsentwicklungen} ein
  ``Lock Step Szenario'', in dem die heutigen Entwicklungen überraschend
  präzise antizipiert wurden (ab Seite 18).
\item
  ``Die Wahrheit über Fauci'': In einem
  \href{https://childrenshealthdefense.org/news/the-truth-about-fauci-featuring-dr-judy-mikovits/}{neuen
  Interview} spricht US-Virologin Dr. Judy Mikovits über ihre
  Erfahrungen mit Dr. Anthony Fauci, der als Chef der US-Seuchenbehörde
  derzeit die Covid19-Maßnahmen der US-Regierung wesentlich
  mitgestaltet.
\item
  Hilfsorganisationen warnen, dass ``ungleich mehr Menschen''
  \href{https://www.welt.de/wirtschaft/article207092745/Corona-Pandemie-Rezession-beschert-der-Welt-die-noch-groessere-Katastrophe.html}{an
  den wirtschaftlichen Folgen der Maßnahmen sterben werden} als an
  Covid-19 selbst. Prognosen gehen inzwischen von 35 bis 65 Millionen
  Menschen aus, die durch die globale Rezession in absolute Armut
  abrutschen werden. Und vielen von ihnen drohe der Hungertod.
\item
  In Deutschland wird für 2020 mit 2.35 Millionen Beschäftigten in
  Kurzarbeit
  \href{https://www.boeckler.de/pdf/p_wsi_pb_38_2020.pdf}{gerechnet},
  das sind mehr als doppelt so viele wie nach der Finanzkrise von
  2008/2009.
\end{itemize}

\includegraphics{https://swprs.files.wordpress.com/2020/04/kurzarbeit-de-corona.png?w=650\&h=461}

\hypertarget{18-april-2020}{%
\paragraph{18. April 2020}\label{18-april-2020}}

\hypertarget{medizinische-notizen}{%
\subparagraph{\texorpdfstring{\textbf{Medizinische
Notizen}}{Medizinische Notizen}}\label{medizinische-notizen}}

\begin{itemize}
\tightlist
\item
  Eine
  \href{https://www.medrxiv.org/content/10.1101/2020.04.14.20062463v1}{neue
  serologische Studie} der Universität Stanford fand im Bezirk Santa
  Clara in Kalifornien Antikörper in 50 bis 85 mal mehr Personen als
  bisher angenommen, wodurch sich eine Covid-Letalität von 0.12\% bis
  0.2\% oder sogar darunter ergibt (d.h. im Bereich einer starken
  Influenza). Professor John Ioannidis erklärt die Studie
  \href{https://www.youtube.com/watch?v=jGUgrEfSgaU}{in einem Video}.
\item
  Das Zentrum für evidenzbasierte Medizin (CEBM) der Universität Oxford
  geht
  \href{https://www.cebm.net/covid-19/global-covid-19-case-fatality-rates/}{in
  einer neuen Analyse} davon aus, dass die Letalität von Covid19 (IFR)
  zwischen 0,1\% und 0,36\% liegt (das heißt im Bereich einer starken
  Grippe). Bei über 70-Jährigen \emph{ohne schwere Vorerkrankungen}
  liege die Letalität voraussichtlich bei unter 1\%. Bei über
  80-Jährigen liege die Letalität zwischen 3\% und 15\%, je nachdem, ob
  die bisherigen Todesfälle hauptsächlich \emph{mit} oder \emph{an} der
  Krankheit erfolgten. Die Letalität bei Kindern liege -- im Unterschied
  zur Grippe -- nahe bei null. Zur hohen Sterblichkeit in Norditalien
  weist die Forschungsrupppe u.a. auf die
  \href{https://www.ansa.it/english/news/science_tecnology/2019/11/19/italy-top-in-eu-in-antibiotic-resistance_369e0123-0107-445e-8c17-f11932c9d27c.html}{europaweit
  höchste Antibiotika­resistenz} in Italien hin. Tatsächlich zeigen
  Daten der italienischen Behörden, das rund 80\% der Verstorbenen mit
  Antibiotika behandelt wurden, was auf bakterielle Superinfektionen
  hindeutet.
\item
  Der finnische Epidemiologie-Professor Mikko Paunio von der Universität
  Helsinki hat
  \href{https://lockdownsceptics.org/wp-content/uploads/2020/04/How-the-World-got-Fooled-by-COVID-ed-2c.pdf}{in
  einem Arbeitspapier} mehrere internationale Untersuchungen ausgewertet
  und kommt auf eine Covid19-Letalität (IFR) von 0.1\% oder weniger
  (d.h. im Bereich der saisonalen Grippe). Der Eindruck einer höheren
  Letalität sei entstanden, weil sich das Virus sehr schnell verbreitet
  habe, insbesondere auch in Mehr-Generationen-Haushalten in Italien und
  Spanien, aber auch in Großstädten wie New York. Die
  ``Lockdown''-Maßnahmen seien überall zu spät gekommen und hätten
  nichts mehr gebracht bzw. seien letztlich sogar kontraproduktiv
  gewesen. (Update: Professor Paunio hat ein
  \href{https://lockdownsceptics.org/wp-content/uploads/2020/04/Outbreak-continues-in-NYC.pdf}{Update
  veröffentlicht}.)
\item
  Die kumulierte Gesamtsterblichkeit in der Schweiz lag im ersten
  Quartal 2020 (1. Januar bis 5. April) gemäß den Zahlen des Bundesamtes
  für Statistik trotz Covid19
  \href{https://swprs.files.wordpress.com/2020/04/ch-sterblichkeit-kumuliert-q1-2020.pdf}{im
  mittleren Normalbereich}. Ein wesentlicher Grund hierfür dürfte der
  milde Winter und die milde diesjährige Grippesaison sein, die nun
  durch Covid19 seit März teilweise ``kompensiert'' wurde.
\item
  In der Schweiz sind laut einer Recherche vom 14. April nicht nur die
  Krankenhäuser insgesamt sehr tief ausgelastet, sondern
  \href{https://swprs.files.wordpress.com/2020/04/intensivbettenbelegung-schweiz-2020-04-14.png}{auch
  die Intensivstationen}. Es stellt sich damit weiterhin die Frage, wo
  und woran die testpositiven Schweizer Todesfälle tatsächlich
  verstorben sind.
\item
  Der Präsident der deutschen Krankenhausgesellschaft
  \href{https://www.bz-berlin.de/deutschland/kliniken-verband-schlaegt-alarm-wegen-corona-regeln}{schlägt
  Alarm}: Mehr als 50 Prozent aller deutschlandweit geplanten
  Operationen wurden abgesagt, der ``OP-Stau'' gehe in die Tausende.
  Zudem würden 30 bis 40\% weniger Patienten mit Herzinfarkt und
  Schlaganfall behandelt, da sich diese aus Angst vor Corona nicht mehr
  in die Kliniken wagen. Es gebe bundesweit 150.000 freie
  Krankenhausbetten und 10.000 freie Intensivbetten. In Berlin seien nur
  68 Intensivbetten mit Corona-Patienten belegt, die Notklinik mit 1000
  Betten werde aktuell nicht gebraucht. Es brauche einen ``viel
  breiteren öffentlichen Diskurs'' über die richtigen Maßnahmen.
\item
  Neue Daten des RKI zeigen, dass auch in Deutschland die
  Reproduktionszahl von Covid19 \emph{bereits vor dem Lockdown} unter
  den kritischen Wert von 1
  \href{https://www.rki.de/DE/Content/Infekt/EpidBull/Archiv/2020/Ausgaben/17_20_SARS-CoV2_vorab.pdf?__blob=publicationFile\#page=5}{gefallen
  war}. Allgemeine Hygiene­maß­nahmen waren mithin ausreichend, um die
  exponentielle Ausbreitung zu verhindern. Bereits zuvor wurde dies von
  der ETH Zürich auch für die Schweiz
  \href{https://www.tagesanzeiger.ch/ansteckungsraten-flachten-bereits-vor-dem-lockdown-ab-809893127675}{belegt}.
\item
  UK: Londons provisorisches Nightingale-Krankenhaus ist mit nur 19
  Patienten, die am Osterwochenende in der Einrichtung behandelt wurden,
  \href{https://www.hsj.co.uk/service-design/exclusive-nightingale-largely-empty-as-icus-handle-surge/7027398.article}{weitgehend
  leer geblieben}. Die etablierten Krankenhäuser Londons haben ihre
  Kapazität auf der Intensivstation verdoppelt und kommen bisher mit der
  Anzahl an Patienten zurecht.
\item
  In Kanada \href{https://orf.at/stories/3162365/}{verstarben 31
  Menschen in einem Altersheim}, nachdem ``fast alle Pflegekräfte die
  Einrichtung aus Angst vor einer Ausbreitung des Coronavirus
  fluchtartig verlassen hatten. Gesundheitsbehörden fanden die Menschen
  in dem Heim in Dorval bei Montreal erst Tage später vor -- viele der
  Überlebenden dehydriert, unterernährt und teilnahmslos.'' Ähnliche
  Tragödien wurden bereits
  \href{https://swprs.org/covid19-bericht-aus-italien/}{aus Norditalien
  berichtet}, wo osteuropäische Pflegekräfte das Land aufgrund der Panik
  und des angekündigten Lockdowns fluchtartig verließen.
\item
  Ein schottischer Arzt, der auch Pflegeheime betreut,
  \href{https://drmalcolmkendrick.org/2020/04/17/care-homes-and-covid19/}{schreibt}:
  ``Was war die Regierungs­strategie für Pflegeheime? Die bisherigen
  Aktionen machten die Situation viel, viel schlimmer.''
\item
  Auf einem französischen Flugzeugträger wurden 1081 Soldaten
  \href{https://www.ouest-france.fr/sante/virus/coronavirus/coronavirus-au-moins-940-marins-positifs-sur-le-charles-de-gaulle-et-son-escorte-6810816}{positiv
  getestet}. Davon blieben bisher knapp 50\% symptomlos und ca. 50\%
  zeigten milde Symptome. 24 Soldaten wurden hospitalisiert, davon einer
  auf der Intensivstation (Vorerkrankungen unbekannt).
\item
  Der deutsche Virologe Christian Drosten hält es für möglich, dass
  manche Menschen durch Kontakt mit normalen Erkältungs-Coronaviren
  bereits eine wirksame sogenannte Hintergrund-Immunität gegen das neue
  Coronavirus \href{https://www.watson.de/!324026684}{aufgebaut haben}.
\item
  Der Hamburger Rechtsmediziner Klaus Püschsel, der bereits zahlreiche
  testpositive Verstorbene untersucht hat, erklärt in einem neuen
  Beitrag:
  \href{https://www.abendblatt.de/hamburg/article228908865/hamburg-corona-virus-uke-infektion-covid-19-pueschel-coronavirus-krise-patienten-impfstoff-immunitaet-krankenhaeuser-kontaktverbot-kliniken-infektionsrate-krankheit-pandemie-test-lungenkrankheit-sars-cov-epidemie-sars-cov-2.html}{``Die
  Zahlen rechtfertigen die Angst vor Corona nicht''}. Seine
  Erkenntnisse: „Corona ist eine vergleichsweise harmlose
  Viruserkrankung. Wir müssen uns damit beschäftigten, dass Corona eine
  normale Infektion ist, und wir müssen lernen, damit zu leben, und zwar
  ohne Quarantäne.`` Die von ihm untersuchten Todes­opfer hätten alle so
  schwere Vorerkrankungen gehabt, dass sie, „auch wenn das hart klingt,
  alle im Verlauf dieses Jahres gestorben wären``. Püschel weiter: „Die
  Zeit der Virologen ist vorbei. Wir sollten jetzt andere fragen, was in
  der Coronakrise das Richtige ist, etwa die Intensivmediziner.``
\item
  Eine
  \href{https://emedicine.medscape.com/article/227820-overview}{Übersicht
  auf Medscape} zeigt, dass Coronaviren-Erkrankungen typischerweise Ende
  April zurückgehen -- mit oder ohne Lockdown.
\item
  Infosperber:
  \href{https://www.infosperber.ch/Artikel/Gesundheit/Weniger-Corona-Falle-Einfach-weniger-testen}{``Weniger
  Corona-Fälle? Einfach weniger testen!''} Die täglich gemeldete Zahl
  der «neuen Fälle» sage über den Stand der Epidemie wenig aus. Es sei
  fahrlässig, mit der Kurve der kumulierten testpositiven Todesfälle
  Angst auszulösen.
\item
  OffGuardian:
  \href{https://off-guardian.org/2020/04/17/8-more-experts-questioning-the-coronavirus-panic/}{Acht
  weitere Experten, die die Corona-Panik hinterfragen}. (Englisch)
\item
  \textbf{Video}: \href{https://www.youtube.com/watch?v=bfN2JWifLCY}{Why
  lockdowns are the wrong policy -- Swedish expert Prof. Johan Giesecke}
  Der schwedische Epidemiologie-Professor Johan Giesecke spricht von
  einem ``Tsunami einer milden Erkrankung'' und hält Ausgangssperren für
  kontraproduktiv. Das Wichtigste sei es, die Risikogruppen,
  insbesondere die Pflegeheime, effizient zu schützen.
\end{itemize}

\includegraphics{https://swprs.files.wordpress.com/2020/04/rki-reproduktion-lockdown.png?w=600\&h=421}

\hypertarget{beatmung-bei-covid19}{%
\subparagraph{\texorpdfstring{\textbf{Beatmung bei
Covid19}}{Beatmung bei Covid19}}\label{beatmung-bei-covid19}}

Weitere Fachleute in Europa und den USA haben sich zur Behandlung von
kritischen Covid19-Patienten geäußert und raten dringend von einer
invasiven Beatmung (Intubation) ab. Es liege bei Covid19-Patienten kein
akutes Lungenversagen (ARDS) vor, sondern ein Sauerstoffmangel, der
möglicherweise durch ein Diffusionsproblem (bei der Durchblutung der
Lunge) verursacht wird, ausgelöst durch das Virus oder die Immunreaktion
darauf.

\begin{itemize}
\tightlist
\item
  WELT:
  \href{https://www.welt.de/vermischtes/article207221877/Corona-Pandemie-Sterberate-bei-Beatmungspatienten-gibt-Raetsel-auf.html}{Sterberate
  bei Beatmungspatienten gibt Rätsel auf}
\item
  AP: \href{https://apnews.com/8ccd325c2be9bf454c2128dcb7bd616d}{Some
  doctors moving away from ventilators for virus patients}
\item
  FAZ:
  \href{https://www.vpneumo.de/fileadmin/pdf/f2004071.007_Voshaar.pdf}{„Es
  wird zu häufig intubiert und invasiv beatmet``} (Dr. Thomas Voshaar)
\item
  Video: \href{https://www.youtube.com/watch?v=NmRlvX3VrAQ}{New York
  intensive care doctor on Covid19 as a possible diffusion hypoxemia}
\item
  \href{https://link.springer.com/article/10.1007/s00134-020-06033-2}{COVID-19
  pneumonia: different respiratory treatments for different phenotypes?}
\end{itemize}

\hypertarget{politische-notizen}{%
\subparagraph{\texorpdfstring{\textbf{Politische
Notizen}}{Politische Notizen}}\label{politische-notizen}}

\begin{itemize}
\tightlist
\item
  Der deutsche Ökonom Norbert Haering
  \href{https://norberthaering.de/}{erklärt in mehreren Beiträgen}, wie
  die ``Corona-Krise'' genutzt wird, um seit längerem geplante
  Überwachungsinstrumente in den Bereichen Reiseverkehr,
  Zahlungsverkehr, Kontaktverfolgung und Biometrie weltweit einzuführen.
\item
  In mehreren US-Bundesstaaten kam es zu
  \href{https://news.yahoo.com/protests-draw-thousands-over-state-024328374.html}{Protesten
  gegen die Ausgangssperren}.
\item
  \textbf{Video}:
  \href{https://archive.org/details/what-in-the-world-is-actually-going-on-document-reveals-plans-step-by-step}{Polizeigewalt
  und Überwachung im Rahmen von ``Corona-Lockdowns'' in aller Welt}.
\item
  Giorgio Agamben, italienischer Philosoph,
  \href{https://www.nzz.ch/feuilleton/coronavirus-giorgio-agamben-zum-zusammenbruch-der-demokratie-ld.1551896}{zu
  den Corona-Maßnahmen}: ``Ein Land, ja eine Kultur implodiert gerade,
  und niemanden scheint es zu kümmern. Was spielt sich gerade vor
  unseren Augen in den Ländern ab, die von sich behaupten, sie seien
  zivilisiert?''
\item
  Italienische Anwälte
  \href{https://www.tvprato.it/2020/04/la-camera-civile-degli-avvocati-pratesi-chiede-lannullamento-del-dpcm-del-10-aprile-e-illegittimo/}{legen
  Beschwerde} gegen Corona-Maßnahmen der Regierung ein.
\item
  Der deutsche Wirtschaftsprofessor Stefan Homburg in der WELT:
  \href{https://www.msn.com/de-de/nachrichten/coronavirus/warum-deutschlands-lockdown-falsch-ist-\%E2\%80\%93-und-schweden-vieles-besser-macht/ar-BB12E6km}{``Warum
  Deutschlands Lockdown falsch ist -- und Schweden vieles besser
  macht.''}: ``Zusammengefasst haben Länder wie Schweden, Südkorea oder
  Taiwan mit ihrem Verzicht auf Lockdowns klug gehandelt. Die dortigen
  Virologen führten Bevölkerung und Politik mit ruhiger Hand durch die
  Krise, statt sie durch ständige Kurswechsel zu verunsichern. Das
  Coronavirus wurde ohne Schaden für Grundrechte und Arbeitsplätze
  erfolgreich eingedämmt. Deutschland sollte sich diese Politik zum
  Vorbild nehmen.'' (Siehe auch:
  \href{https://www.youtube.com/watch?v=Vy-VuSRoNPQ}{Ein Videointerview
  mit Prof. Homburg}).
\item
  Ein Schweizer Bürger hat einen
  \href{https://faktenb-covid-19-massnahmen.jimdofree.com/}{Eilantrag an
  das Bundesverwaltungsgericht und an den Bundesrat} gesandt, um die
  sofortige Aufhebung des Lockdowns zu erwirken.
\item
  \textbf{Video}:
  \href{https://www.youtube.com/watch?v=eU6IdglI-wc}{``Schweizer Ärzten
  wurden Maulkörbe verpasst, der Bundesrat ist zerstritten.''} Ein
  Interview mit Dr. med. Stephan Rietiker, dem Initianten und Betreiber
  von \href{https://www.insidecorona.ch/}{InsideCorona.ch}
\item
  \textbf{Video}:
  \href{https://www.youtube.com/watch?v=SO2JMkKtq40}{``Der Schweizer
  Bundesrat gehört ins Gefängnis. Eine Polemik.''}
\end{itemize}

\hypertarget{16-april-2020}{%
\paragraph{16. April 2020}\label{16-april-2020}}

\begin{itemize}
\tightlist
\item
  Die Londoner \emph{Times}
  \href{https://www.thetimes.co.uk/edition/news/coronavirus-record-weekly-death-toll-as-fearful-patients-avoid-hospitals-bm73s2tw3}{berichtet},
  dass bis zu 50\% der aktuellen britischen Übersterblichkeit
  \emph{nicht durch Coronaviren verursacht} werden, sondern durch die
  Effekte des Lockdowns, der allge­meinen Panik und des teilweisen
  gesellschaftlichen Zusammen­bruchs. Es handelt sich dabei um rund 3000
  Menschen
  \href{https://www.ons.gov.uk/peoplepopulationandcommunity/birthsdeathsandmarriages/deaths/bulletins/deathsregisteredweeklyinenglandandwalesprovisional/weekending3april2020}{pro
  Woche}. Tatsächlich könnte diese Zahl sogar noch höher liegen, da auch
  die britische Corona-Definition Todesfälle \emph{mit} (statt durch)
  Coronaviren sowie reine Verdachtsfälle
  \href{https://www.ons.gov.uk/news/statementsandletters/deathsrelatingtothecoronaviruscovid19}{einschließt}.
  Hinzu kommt, dass rund 50\% der ``Corona-Todesfälle'' Pflegeheime
  \href{https://ltccovid.org/2020/04/12/mortality-associated-with-covid-19-outbreaks-in-care-homes-early-international-evidence/}{betreffen},
  die durch einen allgemeinen Lockdown nicht besser geschützt werden.
\item
  In Dänemark wird der Lockdown
  \href{https://jyllands-posten.dk/debat/breve/ECE12074246/vi-skulle-aldrig-have-trykket-paa-stopknappen/}{inzwischen
  bereut}: ``Wir hätten nie den Stoppknopf drücken sollen. Das dänische
  Gesundheitssystem hatte die Situation unter Kontrolle. Der totale
  Lockdown war ein Schritt zu weit.'', argumentiert Professor Jens Otto
  Lunde Jørgensen vom Aarhus Universitätskrankenhaus. Dänemark fährt
  derzeit den Schulbetrieb wieder hoch.\\
\item
  Der Yale-Professor David Katz, der bereits früh vor den negativen
  Folgen eines Lockdowns warnte, gab ein ausführliches
  \href{https://www.youtube.com/watch?v=VK0Wtjh3HVA}{einstündiges
  Interview} zur aktuellen Situation.
\item
  Der deutsche Virologe Hendrik Streeck erklärt, dass bisher
  \href{https://today.rtl.lu/news/science-and-environment/a/1498185.html}{keine
  ``Schmierinfektionen''} in Supermärkten, Restaurants oder
  Frisiersalons nachgewiesen werden konnte.
\item
  Neue Antikörper-Daten aus der italienischen Gemeinde Robbio in der
  Lombardei zeigen, dass rund
  \href{https://www.tgcom24.mediaset.it/cronaca/a-robbio-pv-il-22-ha-o-ha-avuto-il-coronavirus-ok-del-sindaco-ai-test-per-tutti_17285128-202002a.shtml}{zehmal
  mehr Personen} den Coronavirus hatten als ursprünglich angenommen, da
  sie keine oder nur leichte Symptome entwickelten. Die
  Immunisierungrate liege bei 22\%.
\item
  Neue Daten aus dem Schweizer Kanton Zürich
  \href{https://www.nzz.ch/zuerich/coronavirus-zuerich-aendert-nun-das-testregime-in-heimenauch-viele-aeltere-covid-19-infizierte-entwickeln-keine-symptome-zuerich-aendert-nun-das-testregime-in-heimen-ld.1552089}{zeigen},
  dass sich rund 50\% aller ``Covid19-Todesfälle'' in Alters- und
  Pflegeheimen ereignet haben; dennoch blieben auch dort rund 40\% aller
  testpositiven Personen bisher symptomlos. Das Durchschnittsalter der
  testpositiven Verstorbenen liegt in der Schweiz derzeit bei 84 Jahren.
\item
  Der Schweizer Chefarzt für Infektiologie, Pietro Vernazza, äußert sich
  zur\href{https://infekt.ch/2020/04/exitstrategie-lockdown/}{``Mit-dem-Virus-Leben''-Strategie}
  und empfiehlt unter anderem einen individuell optimierten Schutz
  gefährdeter Personen. Auch die Immunität der Allgemeinbevölkerung sei
  ein Schutz für gefährdete Menschen.
\item
  Die neue britische Website
  \href{https://lockdownsceptics.org/}{Lockdown Skeptics} berichtet
  kritisch über Covid19, die getroffenen Maßnahmen und die allgemeine
  Medienberichterstattung.
\item
  Die zivilgesellschaftliche österreichische
  \href{https://www.initiative-corona.info/}{``Initiative für
  evidenzbasierte Corona-Informationen''} bietet einen Überblick über
  Studien und Analysen zu Coronaviren.
\item
  Dokumentation:
  \href{https://www.youtube.com/watch?v=dYlia_fQOLk}{``Die WHO -- Im
  Griff der Lobbyisten''} (ARTE, 2017)
\end{itemize}

\hypertarget{15-april-2020}{%
\paragraph{15. April 2020}\label{15-april-2020}}

\hypertarget{medizinische-meldungen}{%
\subparagraph{\texorpdfstring{\textbf{Medizinische
Meldungen}}{Medizinische Meldungen}}\label{medizinische-meldungen}}

\begin{itemize}
\tightlist
\item
  Professor Alexander Kekulé, einer der führenden deutschen
  Mikrobiologen und Epidemiologen, fordert in einem Interview mit dem
  britischen \emph{Telegraph}
  \href{https://www.telegraph.co.uk/news/2020/04/11/german-scientist-predicted-european-epidemic-calls-end-lockdown/}{ein
  Ende des Lockdowns}, da dieser mehr Schaden anrichte als das Virus
  selbst. Bei Menschen unter 50 Jahren seien schwere Verläufe oder
  Todesfälle ``sehr, sehr unwahrscheinlich''. Die Allgemein­bevölkerung
  solle eine rasche Immunität entwickeln, während Risikogruppen zu
  schützen seien. Man könne nicht auf einen Impfstoff warten, was
  mindestens sechs bis zwölf Monate dauern werde, sondern müsse einen
  Weg finden, mit dem Virus zu leben.
\item
  Das deutsche Netzwerk für evidenzbasierte Medizin berichtet, dass die
  Letalität einer starken saisonalen Influenza (Grippe) wie 2017/2018
  vom deutschen Robert-Koch-Institut auf 0,4\% bis 0,5\%
  \href{https://www.ebm-netzwerk.de/en/publications/covid-19}{geschätzt
  werde}, und nicht wie früher angenommen auf nur 0,1\%. Dies würde
  bedeuten, dass die Letalität von Covid19 sogar \emph{unter} derjenigen
  einer starken saisonalen Grippe liegen könnte, aber sich in einem
  deutlich kürzeren Zeitraum auswirkt.
\item
  Das Luxemburger Tageblatt
  \href{https://swprs.files.wordpress.com/2020/04/volksblatt_schweden_corona_20200414_18.pdf}{berichtet},
  dass Schwedens ``lockere Strategie zu Covid19 zu funktionieren
  scheint''. Trotz minimaler Maßnahmen und viel internationaler Kritik
  scheine sich die Lage ``derzeit deutlich zu beruhigen''. Ein riesiges
  Feldlazarett, das bei Stockholm aufgebaut wurde, bleibe mangels Bedarf
  weiterhin geschlossen. Die Anzahl der Patienten auf Intensivstationen
  sei gleichbleibend auf niedrigem Niveau oder sogar leicht rückläufig.
  ``Es gibt viele freie Plätze in den Intensivstationen in allen
  Stockholmer Krankenhäusern. Wir nähern uns der Abflachung der
  Erkrankungskurve'', erklärte ein Oberarzt der Karolinska Klinik.
  Bisher kam es in Schweden zu circa 900 Todesfällen \emph{mit} Covid19.
\item
  Ein direkter Vergleich zwischen UK (mit Lockdown) und Schweden (ohne
  Lockdown) zeigt, dass die beiden Länder bei Fallzahlen und Todesfällen
  pro Bevölkerung
  \href{http://www.theblogmire.com/a-comparison-of-lockdown-uk-with-non-lockdown-sweden/}{fast
  identisch} abschneiden.
\item
  Eine Mitteilung im \emph{New England Journal of Medicine}
  \href{https://www.nejm.org/doi/full/10.1056/NEJMc2009316}{berichtet},
  dass bei einer Untersuchung von schwangeren Frauen 88\% der
  testpositiven Frauen \emph{keine Symptome} zeigten -- ein sehr hoher
  Wert, der sich aber mit früheren Untersuchungen aus China und Island
  deckt.
\item
  Professor Dan Yamin, Direktor des Forschungslabors für Epidemien an
  der Tel Aviv Universität,
  \href{https://www.ynet.co.il/articles/0,7340,L-5714371,00.html}{erklärt
  in einem Interview}, dass das neue Coronavirus für einen Großteil der
  Bevölkerung ``wenig gefährlich'' sei und eine rasche natürliche
  Immunität das Ziel sein müsse. Der Schaden durch einen Lockdown sei
  enorm, mit diesem Geld könne man besser eine neue Klinik bauen.
\item
  Der Präsident des israelischen Nationalen Forschungsrats, Professor
  Isaac Ben-Israel,
  \href{https://www.timesofisrael.com/top-israeli-prof-claims-simple-stats-show-virus-plays-itself-out-after-70-days/}{argumentiert},
  dass die Corona-Epidemie nach bisherigen Erktenntnissen in den meisten
  Ländern nach ca. 8 Wochen vorbei sei, \emph{unabhängig davon}, welche
  Maßnahmen getroffen werden. Er empfiehlt deshalb, den ``Lockdown''
  umgehend aufzuheben.
\item
  Der britische Statistik-Professor David Spiegelhalter zeigt, dass das
  Sterberisiko durch Covid19 in etwa
  \href{https://medium.com/wintoncentre/how-much-normal-risk-does-covid-represent-4539118e1196}{der
  normalen Sterblichkeit entspricht} und nur für die Altersgruppe
  zwischen ca. 70 und 80 Jahren sichtbar erhöht ist (vgl. unterste
  Grafik im Artikel).
\item
  Professor Karin Moelling, emeritierte Direktorin des Instituts für
  Virologie der Universität Zürich und
  \href{https://www.rubikon.news/artikel/die-stimme-der-vernunft}{eine
  frühe Kritikerin überzogener Maßnahmen}, betont in einem
  \href{https://www.youtube.com/watch?v=4rl2sqLcDoQ}{neuen Interview}
  die Rolle lokaler Sonderfaktoren wie Luftverschmutzung und
  Populationsdichte.
\item
  Der britische Guardian wies 2015
  \href{https://www.theguardian.com/world/2015/aug/14/air-pollution-in-china-is-killing-4000-people-every-day-a-new-study-finds}{darauf
  hin}, dass die extreme Luftverschmutzung in chinesischen Städten 4000
  Menschen \emph{pro Tag} tötet. Dies ist mehr, als China an
  Covid19-Todesfällen bisher \emph{insgesamt} gemeldet hat.
\item
  Der deutsche Virologe Hendrik Streeck hat sich gegen Kritik an seiner
  Pilotstudie
  \href{https://www.tagesspiegel.de/wissen/virologe-streeck-zur-coronavirus-studie-die-veroeffentlichung-zu-heinsberg-war-nicht-leichtfertig/25735672.html}{verteidigt}.
  Streeck fand eine Letalität (auf Fälle bezogen) von 0,37\% und eine
  Mortalität (auf Bevölkerung bezogen) von 0,06\%, was einer starken
  saisonalen Grippe entspricht.
\item
  Ein Mitarbeiter einer Münchner Klinik berichtet: ``Wie bereits
  wiederholt beschrieben bleiben die Covid-Patienten weitgehend aus. Da
  Kliniken aber verpflichtet wurden für diese Patientengruppe eine
  vorgegebene Anzahl Betten und hier besonders Beatmungsplätze
  freizuhalten, um den ,Gesundheitsnotstand` zu verhindern, entsteht
  mehr und mehr eine groteske Situation. Wir stehen vor leeren
  Intensivbetten weil die Corona-Welle einfach nicht kommen will, müssen
  aber gleichzeitig andere lebens­bedrohlich Erkrankte oder Verletzte
  abweisen, weil wir die Betten nicht belegen dürfen. Dies bedeutet in
  der Folge eine zeitintensive Suche nach geeigneten Behandlungsplätzen
  für diese Patienten, längere Anfahrtwege der Rettungsdienste, einen
  verzögerten Behandlungsbeginn für die Patienten und oft genug die
  berühmten Rot-Kreuz-Reisen im RTW oder den Rundflug im Helikopter bis
  eine aufnahmefähige Klinik gefunden wird. Der Mangel entsteht jetzt
  einfach an anderer Stelle. Der abstrakte Mangel an Corona-Betten
  weicht dem realen Mangel für Betten der ,normalen` Patienten.''
\item
  Österreichische Internisten
  \href{https://wien.orf.at/stories/3044064}{warnen vor
  „Kollateralschäden``}: Abseits des Coronavirus werden Kontroll- und
  Operationstermine verschoben, „Kollateralschäden`` würden damit
  drohen. In Wien kommen z.B. weniger Patienten mit Herzinfarktsymptomen
  in die Spitäler.
\item
  Ein Schweizer Biophysiker hat erstmals die Positivenrate von Covid19
  in der Schweiz durchgehend
  \href{https://swprs.org/rate-of-positive-covid19-tests/}{grafisch
  dargestellt}. Das Ergebnis zeigt, dass die Positivenrate zwischen ca.
  10\% und 25\% pendelt, und dass der ``Lockdown'' keinen wesentlichen
  Einfluss hat (siehe Grafik unten). Schweizer Behörden und Medien haben
  die Positivenrate bisher nicht dargestellt.
\item
  Ein Schweizer Forscher hat den neuesten Covid19-Bericht des
  Bundesamtes für Gesundheit analysiert und kommt erneut zu einer
  \href{https://covid-19-fakten.blogspot.com/2020/04/der-bag-situationsbericht-vom-1442020.html}{sehr
  kritischen Einschätzung}: ``Der BAG-Situations­bericht ist ungeeignet
  für die Politik und eine dortige kompetente Entscheidungs­findung, ist
  erneut höchst unspezifisch, lückenhaft und mangelhaft
  aussagekräftig.''
\item
  Der Schweizer Chefarzt für Infektiologie, Dr. Pietro Vernazza, erklärt
  \href{https://infekt.ch/2020/04/hinterlaesst-coronavirus-eine-immunitaet/}{in
  einem neuen Beitrag}, dass es sich bei der angeblich fehlenden
  Immunitätsbildung bei Covid19 um ``seltene Einzelfälle oder auch nur
  Hinweise'' handle, die ``bei genauerer Betrachtung kein Problem
  darstellen'', jedoch von manchen Medien ``zu Hiobsbotschaften
  aufgebauscht und überstürzt aufgetischt'' werden.
\item
  Aus Frankreich werden
  \href{https://www.midilibre.fr/2020/04/09/coronavirus-ces-suicides-de-malades-ou-de-personnes-tenaillees-par-langoisse,8839373.php}{vermehrt
  Suizide gemeldet}, die aus Angst vor dem Coronavirus erfolgen oder aus
  Angst, jemanden mit dem Coronavirus angesteckt zu haben.
\item
  Die neue französische Seite \href{https://covidinfos.net/}{Covid
  Infos} befasst sich kritisch mit Covid19 und Medienberichten.
\end{itemize}

\includegraphics{https://swprs.files.wordpress.com/2020/04/fs-ch-pos-rate.png?w=600\&h=383}

\hypertarget{usa-und-uk}{%
\subparagraph{\texorpdfstring{\textbf{USA und
UK}}{USA und UK}}\label{usa-und-uk}}

\begin{itemize}
\tightlist
\item
  Auf dem US-Kriegsschiff Theodore Roosevelt wurden 600 Matrosen positiv
  auf Covid19 getestet, ein erster Matrose sei inzwischen
  \href{https://www.theguardian.com/world/2020/apr/14/sailor-dies-from-covid-19-and-600-test-positive-after-outbreak-on-uss-theodore-roosevelt-guam}{an
  oder mit Covid19 gestorben}. Das Kriegsschiff wird eine wichtige
  ``Fallstudie'' sein für die Wirkung auf die gesunde
  Allgemeinbevölkerung unter 65 Jahren.
\item
  Der emeritierte britische Pathologie-Professor, Dr. John Lee,
  argumentiert, es brauche eine
  \href{https://www.spectator.co.uk/article/to-understand-covid-we-need-evidence-scepticism-and-vigorous-debate}{robuste
  und evidenzbasierte Debatte}, um ``große Fehler'' zu vermeiden. Viele
  der von Regierungen und Medien verwendeten Zahlen seien nicht
  zuverlässig gewesen.
\item
  In Großbritannien sind derzeit
  \href{https://www.hsj.co.uk/acute-care/nhs-hospitals-have-four-times-more-empty-beds-than-normal/7027392.article}{40\%
  der Krankenhausbetten unbelegt}, das ist viermal mehr als üblich. Der
  Grund dafür ist der starke Rückgang der allgemeinen
  Patientenaufnahmen. Bei den Intensivbetten, deren Kapazität ausgebaut
  wurde, seien insgesamt 78\% belegt, in einigen Regionen auch mehr.
  Zudem seien 10\% der Krankenpfleger in Quarantäne.
\item
  Die temporären Corona-Stationen des US-Militärs bei New York seien
  bisher
  ``\href{https://nypost.com/2020/04/09/usns-comfort-and-javits-center-mostly-empty-amid-coronavirus/}{weitgehend
  leer}``. Die Hospitalisierungsrate in New York wurde um den Faktor
  sieben
  \href{https://www.nytimes.com/2020/04/10/nyregion/new-york-coronavirus-hospitals.html}{überschätzt}.
\item
  Eine US-Studie
  \href{https://www.medrxiv.org/content/10.1101/2020.04.01.20050542v1}{kommt
  zum Ergebnis}, dass sich das neue Coronavirus bereits viel weiter
  ausgebreitet hat als ursprünglich angenommen, bei den meisten Menschen
  jedoch keine oder nur milde Symptome hervorruft, sodass die
  Letalitätsrate bei nur 0,1\% liegen könnte, was in etwa der saisonalen
  Grippe entspricht. Wegen der leichteren Übertragbarkeit seien die
  Krankheitsfälle etwa in New York jedoch in
  \href{https://archive.is/7w2XE}{kürzerer Zeit als üblich} angefallen.
\item
  Der Chefarzt für Pneumologie und Intensivmedizin der Eastern Virginia
  Medical School erklärt in einem
  \href{https://www.evms.edu/media/evms_public/departments/internal_medicine/EVMS_Critical_Care_COVID-19_Protocol.pdf}{neuen
  Dokument} zur Behandlung von Covid19-Patienten: ``Covid19 verursacht
  kein typisches Lungenversagen \ldots{} Diese Erkrankung muss anders
  behandelt werden und es ist wahrscheinlich, dass die Situation durch
  Beatmungsschäden an der Lunge verschlimmert wird.''
\item
  In den USA behauptete ein Gouverneur, ein Kleinkind sei als weltweit
  jüngstes Opfer ``an Covid'' gestorben. Bekannte der Familie erklärten
  jedoch, dass das Kleinkind bei einem tragischen Unfall zuhause
  \href{https://www.washingtonexaminer.com/news/candace-owens-accuses-connecticut-governor-of-lying-about-coronavirus-death-calls-for-resignation}{erstickt
  sei} und im Krankenhaus nachträglich positiv getestet wurde. Der
  zuständige Rechtsmediziner erklärte
  \href{https://www.dailymail.co.uk/news/article-8193487/Coroner-refuses-rule-COVID-19-cause-death-six-week-old-Connecticut-baby.html}{keinen
  Covid-Todesfall}.
\item
  Eine Ärztin aus dem US-Bundesstaat Montana
  \href{https://www.youtube.com/watch?v=V0lIWZpiRU0}{erklärt in einem
  Vortrag}, wie Totenscheine bei Covid19-Verdachtsfällen aufgrund neuer
  Richtlinien manipuliert werden.
\end{itemize}

\hypertarget{pflegeheime-1}{%
\subparagraph{\texorpdfstring{\textbf{Pflegeheime}}{Pflegeheime}}\label{pflegeheime-1}}

\begin{itemize}
\tightlist
\item
  Eine Analyse von Daten aus fünf europäischen Ländern zeigt, dass
  Bewohner von Plegeheimen bisher
  \href{https://ltccovid.org/2020/04/12/mortality-associated-with-covid-19-outbreaks-in-care-homes-early-international-evidence/}{zwischen
  42\% und 57\% aller ``Covid19-Todesfälle''} ausmachten. Zugleich
  zeigen drei US-Studien, dass bis zu 50\% aller testpositiven Bewohner
  von Pflegeheimen zum Testzeitpunkt (noch) \emph{keine Symptome}
  zeigten. Daraus lassen sich zwei Schlüsse ziehen: Einerseits scheint
  sich die Gefährlichkeit des neuen Coronavirus -- wie bereits vermutet
  -- auf eine kleine, sehr verwundbare Bevölkerungs­gruppe zu
  konzentrieren, die es noch besser zu schützen gilt. Andererseits ist
  es denkbar, dass ein Teil dieser Menschen nicht oder nicht nur am
  Coronavirus stirbt, sondern auch am extremen, damit verbundenen
  Stress. Bereits in Deutschland und Italien wurde zuletzt von
  Pflegeheimbewohnern berichtet, die ohne Symptome plötzlich verstarben.
\item
  Ein deutscher Palliativmediziner argumentiert
  \href{https://www.deutschlandfunk.de/palliativmediziner-zu-covid-19-behandlungen-sehr-falsche.694.de.html?dram:article_id=474488}{in
  einem neuen Interview}, dass man bei der Behandlung von
  Covid19-Patienten „sehr falsche Prioritäten gesetzt und alle ethischen
  Prinzipien verletzt`` habe. Es gebe eine ``sehr einseitige Ausrichtung
  auf die Intensivbehandlung'', obschon ``das Verhältnis zwischen Nutzen
  und Schaden'' oftmals nicht stimme. Man würde aus oftmals
  schwerstpflegebedürftigen Patienten, die in der Vergangenheit zumeist
  palliativ behandelt worden seien, durch eine neue Diagnose
  Intensivpatienten machen und sie einer leidvollen, aber oftmals
  aussichtslosen Behandlung (mit künstlicher Beatmung) unterziehen. Im
  Vordergrund müsse der Wille des jeweiligen Patienten stehen.
\end{itemize}

\includegraphics{https://swprs.files.wordpress.com/2020/04/c19-nursing-homes.jpg?w=700\&h=218}

\hypertarget{politische-themen}{%
\subparagraph{\texorpdfstring{\textbf{Politische
Themen}}{Politische Themen}}\label{politische-themen}}

\begin{itemize}
\tightlist
\item
  In Deutschland wurde eine \href{http://beatebahner.de/}{Fachanwältin
  für Medizinrecht}, die eine Beschwerde gegen die Corona-Maßnahmen beim
  Bundesverfassungsgericht einreichte und zur Anmeldung von
  Demonstrationen aufrief, verhaftet und für zwei
  Tage\href{https://www.rnz.de/nachrichten/heidelberg_artikel,-nach-aufruf-zu-corona-demo-heidelberger-anwaeltin-in-psychiatrischer-einrichtung-update-_arid,508747.html}{in
  eine Gefängnispsychiatrie eingeliefert}. Die Staatsanwaltschaft
  ermittelt wegen ``öffentlicher Aufforderung zu Straftaten''. Ein
  weiterer Rechtsanwalt fragt in einem
  \href{https://archive.is/20200415172607/https://www.nachrichtenspiegel.de/2020/04/14/brief-an-die-bundesrechtsanwaltskammer-in-causa-bahmer/}{Offenen
  Brief} an die deutsche Bundes-Rechtsanwalts-Kammer: ``Rechtsanwältin
  wegen Protest in die Psychiatrie? Ist es wieder soweit in
  Deutschland?''
\item
  In der Schweiz wurde ein ``corona-kritischer'' Arzt wegen angeblicher
  ``Drohungen gegen Angehörige und Behörden'' von einer Spezialeinheit
  der Polizei
  \href{https://www.srf.ch/news/regional/aargau-solothurn/festnahme-von-corona-kritiker-verschwoerung-oder-normale-intervention-der-aargauer-behoerden}{verhaftet
  und in die Psychiatrie eingewiesen}. Die Familie erklärte inzwischen,
  dass es \emph{keine} Drohungen gegen Angehörige gab. Der Arzt erklärte
  zudem, dass ihm bei der Einvernahme \emph{keine} ``Drohungen gegen
  Behörden'' vorgehalten wurden. Die Polizei rechtfertigte den Einsatz
  der Spezialeinheit damit, dass sie beim Arzt von einem Waffenbesitz
  ausging -- dabei handelte es sich indes um die übliche Schweizer
  Sanitätspistole ohne Munition. Die Verlegung des Arztes in die
  Psychiatrie wurde mit einer angeblichen
  ``Haft­erstehungs­unfähigkeit'' begründet (wie sie z.B. bei
  Pflegefällen vorliegt) -- auch dies dürfte als Vorwand zu sehen sein.
  Nach jetzigem Kenntnisstand ist mithin von einer möglicher­weise
  \emph{politisch motivierten Psychiatrisierung} auszugehen, wie sie in
  der Schweiz bis in die 1980er Jahre eine
  \href{https://www.revue.ch/ausgaben/2019/06/detail/news/detail/News/als-die-schweiz-arme-und-unangepasste-wegsperrte-1/}{jahrzehntelange
  Tradition} hatte. Die ehemalige US-Abgeordnete
  \href{https://twitter.com/cynthiamckinney/status/1250075810838581248}{Cynthia
  McKinney} machte bereits auf den Schweizer Fall aufmerksam.
\item
  Italien verwendet nun europäische
  \href{https://www.ansa.it/english/news/2020/04/06/coronavirus-italy-activates-satellite-to-monitor-nation-3_f2ffb30c-d550-42f5-82fc-ec1f82c5c625.html}{Satellitendaten},
  um die Bewegungen der Bevölkerung während des Lockdowns zu überwachen.
\item
  Die britische Polizei
  \href{https://twitter.com/BanTheBBC/status/1249598512427347969}{schlug
  die Türe einer Privatwohnung ein}, um eine Corona-Kontrolle auf
  ``soziale Ansammlungen'' durchzuführen.
\item
  Verfassungsrechtler Professor Oliver Lepsius:
  \href{https://verfassungsblog.de/vom-niedergang-grundrechtlicher-denkkategorien-in-der-corona-pandemie/}{``Vom
  Niedergang grundrechtlicher Denkkategorien in der Corona-Pandemie''}.
\end{itemize}

\hypertarget{12-april-2020}{%
\paragraph{12. April 2020}\label{12-april-2020}}

\hypertarget{neue-studien}{%
\subparagraph{\texorpdfstring{\textbf{Neue
Studien}}{Neue Studien}}\label{neue-studien}}

\begin{itemize}
\tightlist
\item
  Der Stanford-Medizinprofessor John Ioannidis kommt in
  \href{https://www.medrxiv.org/content/10.1101/2020.04.05.20054361v1}{einer
  neuen Studie} zum Ergebnis, dass das Sterberisiko durch Covid19 für
  unter 65 Jahre alte Menschen selbst in den globalen ``Hotspots'' dem
  Risiko eines tödlichen Autounfalls für tägliche Pendler entspreche.
\item
  Der deutsche Virologe Hendrick Streeck kommt in einer
  \href{https://www.t-online.de/gesundheit/krankheiten-symptome/id_87680236/lockerung-der-corona-massnahmen-ergebnisse-der-heinsberg-studie-machen-hoffnung.html}{serologischen
  Pilotstudie} zum Zwischen­ergebnis, dass die Letaltiät von Covid19 bei
  0.37\% und die Mortalität (auf die Gesamtbevölkerung bezogen) bei
  0.06\% liegt. Diese Werte sind rund zehnmal tiefer als jene der WHO
  und rund fünfmal tiefer als jene der Johns Hopkins Universität.
\item
  Eine dänische Studie mit 1500 Blutspendern kommt zum Ergebnis, dass
  die Letalität von Covid19
  \href{https://www.dr.dk/nyheder/indland/doedelighed-skal-formentlig-taelles-i-promiller-danske-blodproever-kaster-nyt-lys}{bei
  nur 1.6 Promille liegt}, das heißt über 20 mal niedriger als von der
  WHO ursprünglich angenommen und damit im Bereich einer starken
  (pandemischen) Influenza. Zugleich hat Dänemark
  \href{https://www.thelocal.dk/20200406/denmark-to-reopen-schools-and-kindergartens-next-week}{beschlossen},
  kommende Woche die Schulen und Kindergärten wieder zu öffnen.
\item
  Eine serologische Studie im US-Bundesstaat Colorado kommt zum
  \href{https://reason.com/2020/04/08/mass-antibody-testing-in-this-rural-colorado-county-sheds-light-on-covid-19s-prevalence-and-lethality/}{vorläufigen
  Ergebnis}, dass die Letalität von Covid19 um einen Faktor 5 bis Faktor
  20 überschätzt wurde und im Bereich zwischen einer normalen und einer
  pandemischen Grippe liegen dürfte.
\item
  Eine Untersuchung der Medizinischen Universität Wien kommt
  \href{https://www.vienna.at/analyse-zeigt-covid-19-opferkurve-entspricht-normaler-mortalitaet/6581246}{zum
  Ergebnis}, dass das Alters- und Risikoprofil der Covid19-Verstorbenen
  in etwa der normalen Sterblichkeit entspricht.
\item
  Eine Studie im \emph{Journal of Medical Virology} kommt
  \href{https://www.ncbi.nlm.nih.gov/pubmed/32219885}{zum Ergebnis},
  dass der international verwendete Coronavirentest instabil sei:
  Zusätzlich zum bereits bekannten Problem der falschen positiven
  Resultate gebe es auch eine ``möglicherweise hohe'' Rate an falschen
  negativen Resultaten, d.h. der Test spricht selbst bei symptomatischen
  Personen nicht an, während er bei anderen Patienten einmal anspricht
  und dann wieder nicht. Dadurch werde die Unterscheidung von anderen
  grippeähnlichen Erkrankungen erschwert.
\item
  Ein Schweizer Biophysiker hat erstmals die Positivenrate in den USA,
  Deutschland, Frankreich und der Schweiz ausgewertet und
  \href{https://swprs.org/rate-of-positive-covid19-tests/}{grafisch
  dargestellt}. Daraus wird ersichtlich, dass die Positivenrate in
  diesen Ländern nur leicht und nicht exponentiell zunimmt.
\item
  US-Forscher kommen zum Ergebnis, dass lokale Luftverschmutzung das
  Sterberisiko an
  Covid19\href{https://www.heise.de/tp/features/Luftverschmutzung-erhoeht-Covid-19-Sterberisiko-4699306.html}{stark
  erhöht}. Dies bestätigt frühere Untersuchungen aus Italien und China.
\item
  Die WHO kam Ende März
  \href{https://www.who.int/news-room/commentaries/detail/modes-of-transmission-of-virus-causing-covid-19-implications-for-ipc-precaution-recommendations}{zum
  Resultat}, dass Covid19 entgegen früheren Vermutungen nicht durch
  Aerosole (``durch die Luft'') übertragen wird. Die Übertragung finde
  haupt­sächlich durch direkten Kontakt oder durch Tröpfcheninfektion
  (Husten, Niesen) statt.
\item
  Der deutsch-amerikanische Epidemiologie-Professor Knut Wittkowski geht
  in einem \href{https://www.youtube.com/watch?v=ARTf4bpiXuI}{Interview}
  davon aus, dass die Covid19-Epidemie in vielen Ländern bereits am
  Abklingen oder sogar ``schon vorbei'' sei. Die Ausgangssperren seien
  zu spät gekommen und kontraproduktiv gewesen.
\end{itemize}

\hypertarget{europuxe4isches-mortalituxe4tsmonitoring}{%
\subparagraph{\texorpdfstring{\textbf{Europäisches
Mortalitätsmonitoring}}{Europäisches Mortalitätsmonitoring}}\label{europuxe4isches-mortalituxe4tsmonitoring}}

Das
\href{https://www.euromomo.eu/outputs/zscore_country_total.html}{europäische
Mortalitätsmonitoring} zeigt inzwischen in mehreren europäischen Ländern
eine deutliche prognostizierte Übersterblichkeit in der Altersgruppe
über 65 Jahren. In anderen Ländern, darunter Deutschland und Österreich,
liegt die Sterblichkeit aber auch in dieser Altersgruppe noch im
Normalbereich (oder sogar darunter).

Offen bleibt weiterhin die Frage, ob die teilweise erhöhte Sterblichkeit
allein auf das Coronavirus oder auch auf die teilweise drastischen
Maßnahmen zurückzuführen ist (durch Isolation, Stress, abgesagte OPs,
etc.), und ob die Sterblichkeit auch in der Jahressicht noch erhöht sein
wird.

Bei den Altersgruppen unter 65 Jahren gibt es bisher nur in England eine
(prognostizierte) erhöhte Sterblichkeit, die über frühere Grippewellen
hinausgeht. Das Medianalter der testpositiven Verstorbenen liegt in
Italien bei 80, in Deutschland bei 83 und in der Schweiz bei 84 Jahren.

\hypertarget{schweiz-1}{%
\subparagraph{\texorpdfstring{\textbf{Schweiz}}{Schweiz}}\label{schweiz-1}}

\begin{itemize}
\tightlist
\item
  Laut dem
  \href{https://www.bag.admin.ch/bag/de/home/krankheiten/ausbrueche-epidemien-pandemien/aktuelle-ausbrueche-epidemien/novel-cov/situation-schweiz-und-international.html}{neuesten
  Bericht} des BAG liegt der Altersmedian der testpositiven Verstorbenen
  inzwischen bei 84 Jahren, 96\% hatten mindestens eine Vorerkrankung.
  Die Anzahl der hospitalisierten Patienten bleibt konstant.
\item
  Eine Studie der ETH Zürich
  \href{https://www.tagesanzeiger.ch/ansteckungsraten-flachten-bereits-vor-dem-lockdown-ab-809893127675}{kommt
  zum Ergebnis}, dass die Ansteckungsrate in der Schweiz bereits
  \emph{mehrere Tage vor} dem ``Lockdown'' auf den stabilen Wert von 1
  fiel, vermutlich aufgrund von allgemeinen Hygiene- und
  Alltagsmaßnahmen. Falls dieses Resultat korrekt ist, würde es die
  Sinnhaftigkeit eines ``Lockdowns'' grundsätzlich infrage stellen.
  (\href{https://bsse.ethz.ch/cevo/research/sars-cov-2/real-time-monitoring-in-switzerland.html}{Zur
  ETH-Studie})
\item
  Das Schweizer Magazin \emph{Infosperber} kritisiert die
  Informationspolitik von Behörden und Medien:~
  ``\href{https://www.infosperber.ch/Artikel/Gesundheit/Corona-Statt-zu-informieren-fuhren-Behorden-eine-PR-Kampagne}{Statt
  zu informieren führen Behörden eine PR-Kampagne}'' . Mit irreführenden
  Zahlen und Grafiken werde eine teilweise unberechtigte Angst
  verbreitet.
\item
  Auch das Schweizer Konsumentenschutz-Magazin \emph{Ktipp} kritisiert
  die Informations­politik und Medien­bericht­erstattung:
  \href{https://www.ktipp.ch/artikel/artikeldetail/behoerden-informieren-irrefuehrend/}{``Behörden
  informieren irreführend.''}
\item
  Ein Schweizer Forscher hat den neusten Covid19-Bericht des Bundesamtes
  für Gesundheit analysiert und kommt zu einem
  \href{https://covid-19-fakten.blogspot.com/2020/04/die-analyse-des-aktuellen.html}{sehr
  kritischen Ergebnis}: Der Bericht sei ``wissenschaftlich unausgewogen,
  tendenziell bevormundend und irreführend (oder zumindest
  verwirrend)''. Die Maßnahmen seien in Abetracht der Fakten
  ``verantwortungslos und angstverbreitend''.
\item
  Schweizer Ärzte sprechen in einem
  \href{https://www.rontalpraxis.ch/aktuelles}{Offenen Brief an den
  Schweizer Gesundheitsminister} von einer ``Diskrepanz zwischen dem vor
  allem auch von den Medien geschürten Bedrohungsszenario und unserer
  Realität.'' Die in der Allgemeinbevölkerung beobachteten Covid19-Fälle
  seien wenige und zumeist mild verlaufen, hingegen nehmen
  ``Angststörungen und Panikattacken'' in der Bevölkerung zu und viele
  Patienten würden sich nicht mehr zu wichtigen Untersuchungsterminen
  getrauen. ``Und dies im Zusammenhang mit einem Virus, dessen
  Gefährlichkeit nach unserer Wahrnehmung in der Zentralschweiz
  lediglich in den Medien und in unseren Köpfen existiert.''
\item
  Aufgrund der sehr tiefen Patientenauslastung mussten inzwischen
  mehrere Kliniken in der
  \href{https://www.20min.ch/schweiz/news/story/Spitaeler-28949526}{Schweiz}
  und in
  \href{https://www.spiegel.de/wirtschaft/unternehmen/trotz-corona-pandemie-warum-kliniken-jetzt-kurzarbeit-anmelden-a-3dc61bc9-fb12-4298-8022-bb4c2be39d7d}{Deutschland}
  Kurzarbeit anmelden. Der Rückgang an Patienten beträgt bis zu 80\%.
\item
  Dr. Daniel Jeanmonod, emeritierter Schweizer Professor für Physiologie
  und Neurochirurgie, empfiehlt in einer Analyse:
  ``\href{https://off-guardian.org/2020/04/07/think-deep-do-good-science-and-do-not-panic/}{Think
  deep, do good science, and do not panic!}``
\item
  Der Schweizer Mediziner Professor Dr. Paul Robert Vogt hat einen
  \href{https://www.mittellaendische.ch/2020/04/07/covid-19-eine-zwischenbilanz-oder-eine-analyse-der-moral-der-medizinischen-fakten-sowie-der-aktuellen-und-zuk\%C3\%BCnftigen-politischen-entscheidungen/}{vielbeachteten
  Beitrag} zu Covid19 verfasst. Er kritisiert eine „Sensationspresse``,
  warnt aber auch, dass es sich nicht um eine „gewöhnliche Grippe``
  handle. In manchen Punkten liegt der Arzt indes falsch: Letalitätsrate
  und Altersmedian sind sehr wohl zentrale Größen, die Unterscheidung
  mit/durch Coronavirus essentiell, Atemschutz­masken und
  Beatmungsgeräte in vielen Fällen ungeeignet (siehe unten),
  Ausgangssperren eine fragwürdige und womöglich kontraproduktive
  Maßnahme.
\end{itemize}

\hypertarget{deutschland-und-uxf6sterreich-1}{%
\subparagraph{**Deutschland und
Österreich}\label{deutschland-und-uxf6sterreich-1}}

**

\begin{itemize}
\tightlist
\item
  In einem Thesenpapier üben deutsche Gesundheitsexperten
  \href{https://www.tagesschau.de/investigativ/ndr-wdr/corona-experten-thesenpapier-101.html}{Kritik
  an der Krisenpolitik} der Bundesregierung. Sie sprechen von
  Langzeitschäden in der Bevölkerung, die der teilweise Shutdown
  verursache. Die vom RKI publizierten Zahlen hätten ``eine geringe
  Aussagekraft''.
\item
  Der Bundesverband deutscher
  Pathologen~\href{https://www.pathologie-dgp.de/die-dgp/aktuelles/meldung/pressemitteilung-an-corona-verstorbene-sollten-obduziert-werden/}{fordert
  in einer Mitteilung}, dass ``Corona-Todesfälle'' obduziert werden
  müssen (um die wirkliche Todesursache festzustellen) und widerspricht
  damit explizit ``der Empfehlung des Robert-Koch-Instituts'', das sich
  gegen Obduktionen aussprach, angeblich, weil sie zu gefährlich seien.
\item
  Dr. Martin Sprenger hat seine Funktion im Corona-Expertenrat des
  österreichischen Gesund­heits­ministeriums
  \href{https://mailchi.mp/addendum/fles-home-office-260342}{niedergelegt},
  um ``seine bürgerliche und wissen­schaftliche Meinungs­freiheit
  wiederzugewinnen''. Dr. Sprenger kritisierte zuvor unter anderem, dass
  die Regierung das Risiko des Virus für verschiedene
  Bevölkerungs­­gruppen nicht genügend unterscheide und
  \href{https://www.addendum.org/coronavirus/interview-sprenger/}{zu
  pauschale Maßnahmen treffe}: ``Wir müssen aufpassen, dass der Verlust
  an gesunden Lebensjahren aufgrund einer mangelhaften Versorgung
  anderer akuter und chronischer Erkrankungen nicht Faktor 10-mal höher
  ist als der durch COVID-19 verursachte Verlust an gesunden
  Lebensjahren.'' Das Coronavirus sei vor allem für ``ältere,
  hochbetagte Menschen'' gefährlich, so Sprenger.\\
\item
  In einem deutschen Pflegeheim wurde ein 84 Jahre alter Mann positiv
  auf Covid19 getestet, worauf das gesamte Heim unter Quarantäne
  gestellt und Massen­tests durchgeführt wurden. Das erste Testergebnis
  stellte sich später jedoch
  \href{https://www.schwerin.de/news/4a3e5560-78c9-11ea-b543-1967de695b51/}{als
  falsch heraus}.
\end{itemize}

\hypertarget{skandinavien}{%
\subparagraph{\texorpdfstring{\textbf{Skandinavien}}{Skandinavien}}\label{skandinavien}}

\begin{itemize}
\tightlist
\item
  Der Norwegische Ärzteverband schreibt in einem Offenen Brief an den
  Gesund­heits­minister, man sei besorgt, dass die getroffenen Maßnahmen
  \href{https://www.abcnyheter.no/helse-og-livsstil/helse/2020/04/06/195667780/nesten-halvparten-av-sengene-pa-oslo-universitetssykehus-star-tomme}{gefährlicher
  als das Virus} sein könnten, da normale Patienten nicht mehr
  untersucht und behandelt werden.
\item
  Ein schwedischer Publizist
  \href{https://www.spectator.co.uk/article/no-lockdown-please-w-re-swedish}{erklärt
  im britischen \emph{Spectator}}: ``Es ist nicht Schweden, das ein
  Massenexperiment durchführt. Es sind alle anderen Länder, die das
  tun.''
\item
  Professor Ansgar Lohse, Direktor an der Hamburger Universitätsklinik,
  \href{https://www.abendblatt.de/hamburg/article228880917/uke-professor-shutdown-lohse-deutschland-hamburg-corona-virus-infektion-covid-19-impfstoff-coronavirus-krise-patienten-immunitaet-krankenhaeuser-kontaktverbot-kliniken-infektionsrate.html}{erklärt
  in einem Interview}: ``Die schwedischen Maßnahmen sind meines
  Erachtens die rationalsten weltweit. Natürlich stellt sich die Frage,
  ob das psychologisch durchzuhalten ist. Anfänglich müssen die Schweden
  mit deutlich mehr Todesfällen rechnen, die sich aber mittel- bis
  langfristig dann deutlich reduzieren. Abgerechnet wird in einem Jahr
  -- wenn die Schweden es durchhalten. Die Angst vor der Virusinfektion
  zwingt Politiker leider oft zu Handlungen, die nicht unbedingt
  vernünftig sind. Die Politik ist getrieben, auch durch die Bilder der
  Medien.''
\item
  Laut dem schwedischen Chefepidemiologen Anders Tegnell habe Stockholm
  bezüglich Covid-Infektionen inzwischen ein ``Plateau'' erreicht.
  (\href{https://www.thelocal.se/20200310/timeline-how-the-coronavirus-has-developed-in-sweden}{Mehr
  Meldungen zu Schweden})
\end{itemize}

\hypertarget{usa-und-asien}{%
\subparagraph{\texorpdfstring{\textbf{USA und
Asien}}{USA und Asien}}\label{usa-und-asien}}

\begin{itemize}
\tightlist
\item
  In den USA empfehlen die Behörden nun ebenfalls, alle testpositiven
  Todesfälle sowie sogar Verdachtsfälle \emph{ohne} positives
  Testergebnis als ``Covid-Todesfälle''
  \href{https://nypost.com/2020/04/07/feds-classify-all-coronavirus-patient-deaths-as-covid-19-deaths/?link=TD_mansionglobal_new_mansion_global.11147f181987fd93}{zu
  registrieren}. Ein amerikanischer Arzt und Staatssenator von Minnesota
  \href{https://www.valleynewslive.com/content/misc/Sen-Dr-Jensens-Shocking-Admission-About-Coronavirus-569458361.html}{erklärte},
  dies komme einer Manipulation gleich. Außerdem würden für
  Krankenhäuser finanzielle Anreize bestehen, Patienten als
  Covid19-Patienten zu deklarieren. (Etwas
  \href{https://swprs.files.wordpress.com/2020/04/cv-2019-2020.jpg}{Humor}
  zu dieser Thematik).
\item
  Ein Covid19-Feldspital bei Seattle im US-Bundesstaat Washington wurde
  bereits nach drei Tagen
  \href{https://www.yahoo.com/news/armys-seattle-field-hospital-closes-165646379.html}{wieder
  geschlossen}, ohne dass Patienten aufgenommen wurden. Dies erinnert an
  die kurzfristig errichteten Krankenhäuser bei Wuhan, die ebenfalls nur
  sehr gering ausgelastet waren oder sogar leer blieben und nach kurzer
  Zeit wieder
  \href{https://www.theguardian.com/world/2020/feb/12/what-chinas-empty-new-coronavirus-hospitals-say-about-its-secretive-system}{abgebaut
  wurden}.
\item
  Zahlreiche Medien berichteten von angeblichen ``Corona-Massengräbern''
  auf der Hart Island bei New York. Diese Meldungen sind in doppelter
  Hinsicht irreführend: Erstens ist Hart Island seit langem einer der
  bekanntesten
  \href{https://en.wikipedia.org/wiki/Hart_Island_(Bronx)\#Cemetery}{Armenfriedhöfe}
  der USA, zweitens
  \href{https://www.independent.co.uk/news/world/americas/new-york-coronavirus-cases-burials-bodies-covid-19-hart-island-a9459956.html}{erklärte}
  der Bürgermeister von New York, dass keine Massengräber geplant sind,
  sondern dass unidentifizierte Verstorbene (d.h. ohne Angehörige) auf
  Hart Island beigesetzt werden sollen.
\item
  Einer der führenden indischen Epidemiologen erklärt,
  ``\href{https://www.business-standard.com/article/current-affairs/we-cannot-run-away-to-the-moon-need-to-develop-herd-immunity-dr-muliyil-120040601232_1.html}{Wir
  können nicht auf den Mond flüchten}``, und empfiehlt die rasche
  Entwicklung einer natürlichen Immunität in der Bevölkerung.
\end{itemize}

\hypertarget{norditalien}{%
\subparagraph{\texorpdfstring{\textbf{Norditalien}}{Norditalien}}\label{norditalien}}

Zu Norditalien wurden zuletzt verschiedene mögliche Risikofaktoren
diskutiert.

Es ist richtig, dass in der Lombardei in den Monaten unmittelbar vor
Ausbruch von Covid19 zwei umfangreiche Impfkampagnen gegen
\href{https://www.bergamonews.it/2019/10/21/vaccinazione-antinfluenzale-a-bergamo-ordinate-185-000-dosi-di-vaccino/332164/}{Influenza}
und gegen
\href{https://www.bsnews.it/2020/01/18/meningite-vaccinate-34mila-persone-tra-brescia-e-bergamo/}{Meningokokken}
durchgeführt wurden, insbesondere auch in den späteren Hotspots Bergamo
und Brescia. Es ist zwar \emph{theoretisch denkbar}, dass solche
Impfungen mit Coronaviren-Infektionen wechselwirken können, aber
medizinisch ist eine solche mögliche Wechselwirkung derzeit nicht
belegt.

Ebenfalls ist es richtig, dass in Norditalien in der Vergangenheit eine
hohe
\href{https://www.spiegel.de/panorama/justiz/asbest-prozess-in-italien-nun-sind-alle-krank-a-666421.html}{Asbestbelastung}
vorlag, die das Risiko für spätere, \emph{krebsartige}
Lungenerkrankungen erhöht. Auch hier kann aber nicht von einem
\emph{direkten Zusammenhang} mit Covid19 ausgegangen werden.

Generell ist es aber zutreffend, dass die Lungengesundheit der
norditalienischen Bevölkerung aufgrund von hoher
\href{https://www.heise.de/tp/features/Feinstaubpartikel-als-Viren-Vehikel-4687454.html}{Luftverschmutzung}
und einigen anderen Faktoren seit langem beeinträchtigt ist und sie für
Atemwegserkrankungen deshalb
\href{https://www.srf.ch/news/international/massive-schadstoffbelastung-nirgendwo-erkranken-so-viele-wegen-smog-wie-in-norditalien}{besonders
anfällig} ist.

\includegraphics{https://swprs.files.wordpress.com/2020/03/italy-smog.png?w=500\&h=281}

\hypertarget{chefarzt-pietro-vernazza}{%
\subparagraph{\texorpdfstring{\textbf{Chefarzt Pietro
Vernazza}}{Chefarzt Pietro Vernazza}}\label{chefarzt-pietro-vernazza}}

Der Schweizer Chefarzt für Infektiologie, Professor Pietro Vernazza, hat
vier neue Artikel zu Studien bezüglich Covid19 veröffentlicht.

\begin{itemize}
\tightlist
\item
  Im
  \href{https://infekt.ch/2020/04/schulen-schliessen-hilfreich-oder-nicht/}{ersten
  Artikel} geht es darum, dass es für die Wirksamkeit von
  \textbf{Schulschließungen} keine medizinische Evidenz gibt (und auch
  nie gab), da Kinder im Allgemeinen weder ernsthaft am Virus erkranken
  noch zu den Überträgern des Virus gehören (im Unterschied zur
  Influenza).
\item
  Im
  \href{https://infekt.ch/2020/04/atemschutzmasken-fuer-alle-medienhype-oder-unverzichtbar/}{zweiten
  Artikel} geht es darum, dass \textbf{Atemschutzmasken} im Allgemeinen
  keine nachweisbare Wirkung haben, mit einer Ausnahme: Erkrankte
  Menschen \emph{mit Symptomen} (d.h. insbesondere Husten) können
  dadurch die Ausbreitung des Virus reduzieren. Ansonsten seien die
  Masken eher Symbolik bzw. ein ``Medienhype''.
\item
  Im~\href{https://infekt.ch/2020/04/corona-testen-testen-und-kein-ende/}{dritten
  Artikel} geht es um die Frage der \textbf{Massentests}. Das Fazit von
  Professor Vernazza: ``Wer Symptome einer Atemwegserkrankung hat,
  bleibt zu Hause. Das gilt genauso bei Grippe. Eine Testung bringt
  keinen zusätzlichen Nutzen.''
\item
  Im~\href{https://infekt.ch/2020/03/immunschwaeche-und-schwangerschaft-kein-covid-19-risikofaktor/}{vierten
  Artikel} geht es um die \textbf{Covid19-Risikogruppen}. Dazu gehören
  nach den bisherigen Erkenntnissen Menschen mit \emph{Bluthochdruck} --
  es wird vermutet, dass das Covid19-Virus Zellrezeptoren nutzt, die
  auch für die Blutdruckregulation zuständig sind. \emph{Nicht zu den
  Risikogruppen} gehören jedoch, überraschenderweise, Menschen mit
  Immunschwäche sowie schwangere Frauen (die natürlicherweise ein
  reduziertes Immunsystem haben). Das Risiko bei Covid19 sei im
  Gegenteil oftmals eine \emph{Überreaktion} des Immunsystems.
\end{itemize}

\hypertarget{intensiv--vs-palliativmedizin}{%
\subparagraph{\texorpdfstring{\textbf{Intensiv- vs.
Palliativmedizin}}{Intensiv- vs. Palliativmedizin}}\label{intensiv--vs-palliativmedizin}}

Ein deutscher Palliativmediziner
\href{https://www.ruhr24.de/ruhrgebiet/coronavirus-behandlung-intensivstation-nrw-lungenentzuendung-matthias-thoens-witten-zr-13645038.html}{erklärt
in einem Interview}, dass Covid19 ``keine intensiv­medizinische
Erkrankung'' sei, da es sich bei den stark betroffenen Menschen
typischerweise um mehrfach vorerkrankte Menschen im hohen Alter handle.
Wenn diese Menschen eine Lungenentzündung bekommen, seien sie ``schon
immer palliativ (sterbebegleitend) versorgt worden''. Mit einer
Covid19-Diagnose mache man daraus nun aber einen Intensivfall und könne
die Patienten dann aber ``natürlich trotzdem nicht retten''.

Das aktuelle Handeln vieler Entscheider bezeichnet der Arzt als
``Panikmodus''. Derzeit seien die Intensivbetten in Deutschland noch
relativ leer. Beatmungsgeräte seien frei. Aus Umsatz­gründen könnten
Krankenhaus-Geschäftsführer auf die Idee kommen, alte Menschen
aufzunehmen. ``Wir werden in 14 Tagen die Stationen voll haben mit
nicht-rettbaren, multimorbiden Alten. Und wenn die dann an den Geräten
sind, stellt sich die Frage, wer die wieder ausschaltet. Das ist doch
dann ein Tötungsdelikt.'' Es drohe eine ``ethische Katastrophe'' aus
Geldgier, meint der Mediziner.

\hypertarget{beatmung-bei-covid19-1}{%
\subparagraph{**Beatmung bei Covid19}\label{beatmung-bei-covid19-1}}

**

Weltweit gab und gibt es einen Ansturm auf Beatmungsgeräte für
Covid19-Patienten. Diese Seite war weltweit eine der ersten, die darauf
aufmerksam machte, dass die invasive Beatmung (Intubation) in vielen
Fällen kontraproduktiv ist und den Patienten zusätzlich schadet.

Die invasive Beatmung wurde ursprünglich empfohlen, weil aufgrund tiefer
Sauerstoffwerte fälschlicherweise auf ein akutes Lungenversagen
geschlossen wurde, und weil die Angst bestand, bei einer sanfteren
nicht-invasiven Beatmung könnte sich das Virus durch Aerosole
verbreiten.

Inzwischen haben sich mehrere führende Lungenfachärzte und
Intensivmediziner aus den USA und Europa zu Wort gemeldet, die von einer
invasiven Beatmung abraten und sanftere Methoden bzw. eine
Sauerstofftherapie empfehlen, wie sie von Südkorea bereits erfolgreich
angewandt wurde.

\begin{itemize}
\tightlist
\item
  \textbf{DE}:
  \href{https://www.vpneumo.de/fileadmin/pdf/f2004071.007_Voshaar.pdf}{„Es
  wird zu häufig intubiert und invasiv beatmet``} (Dr. Thomas Voshaar,
  FAZ)
\item
  \textbf{DE}:
  \href{https://www.doccheck.com/de/detail/articles/26271-covid-19-beatmung-und-dann}{COVID-19:
  Beatmung -- und dann?} (DocCheck Fachartikel)
\item
  \textbf{DE}:
  \href{https://www.frankenpost.de/region/oberfranken/laenderspiegel/Gefahr-durch-das-Beatmungsgeraet;art2388,7210803}{Erfahrungsbericht
  eines Intensivmediziners zu Covid-19} (Dr. Tobias Schindler)
\item
  \href{https://time.com/5818547/ventilators-coronavirus/}{Why Some
  Doctors Are Now Moving Away From Ventilator Treatments} (TIME)
\item
  \href{https://www.spectator.co.uk/article/Ventilators-aren-t-a-panacea-for-a-pandemic-like-coronavirus}{Ventilators
  aren't a panacea for a pandemic like coronavirus} (Dr. Matt Strauss)
\item
  \href{https://www.statnews.com/2020/04/08/doctors-say-ventilators-overused-for-covid-19/}{With
  ventilators running out, doctors say the machines are overused for
  Covid-19} (SN)
\item
  \href{https://www.atsjournals.org/doi/pdf/10.1164/rccm.202003-0817LE}{Covid-19
  Does Not Lead to a ``Typical'' Acute Respiratory Distress Syndrome}
  (ATSJ)
\item
  \href{https://www.medscape.com/viewarticle/928156}{Do COVID-19
  Ventilator Protocols Need a Second Look?} (Medscape)
\end{itemize}

\hypertarget{politische-entwicklungen}{%
\subparagraph{**Politische
Entwicklungen}\label{politische-entwicklungen}}

**

\begin{itemize}
\tightlist
\item
  NSA-Whistleblower Edward Snowden warnt in einem neuen Interview, dass
  Regierungen den Coronavirus nutzen, um eine
  ``\href{https://www.vice.com/en_us/article/bvge5q/snowden-warns-governments-are-using-coronavirus-to-build-the-architecture-of-oppression}{Architektur
  der Unterdrückung}'' aufzubauen.
\item
  Apple und Google haben angekündigt, in Zusammenarbeit mit nationalen
  Behörden ein sogenanntes
  ``\href{https://www.bloomberg.com/news/articles/2020-04-10/apple-google-bring-covid-19-contact-tracing-to-3-billion-people}{Contact
  Tracing}'' in ihre mobilen Betriebssysteme einzubauen, mit dem sich
  die Kontakte innerhalb der Bevölkerung überwachen lassen.
\item
  Der deutsche Verfassungsrechtler Uwe Volkmann
  \href{https://www.youtube.com/watch?v=DvzrGLvzllU}{erklärte in ARD
  Monitor}, er kenne unter seinen Kollegen ``niemanden'', der die
  Corona-Maßnahmen für verfassungskonform hält.
\item
  Die italienische Regierung hat eine ``Task Force'' eingerichtet, um
  ``Falsch­meldungen'' zu Covid im Internet zu
  ``\href{https://www.faz.net/aktuell/feuilleton/medien/corona-in-italien-das-virus-und-die-wahrheit-16714529.html}{beseitigen}''
  . Die freie Meinungsäußerung bleibe aber ``unangetastet''.
\item
  Frankreich hat die erlaubte Untersuchungshaft verlängert und die
  Prüfung durch einen Richter
  \href{https://www.lefigaro.fr/politique/coronavirus-le-conseil-d-etat-sur-la-ligne-de-crete-des-libertes-publiques-20200406}{ausgesetzt}.
  Beschwerden durch Anwaltsverbände wurden abgewiesen.
\item
  Dänemark führte Anfang April
  ``\href{https://www.fr.de/politik/coronavirus-sars-cov-2-daenemark-notfalls-militaer-13598503.html}{beispiellos
  harte Ausnahmegesetze}'' ein: ``Die Gesundheits­behörden können ab
  sofort Zwangstests, Zwangs­impfungen sowie Zwangs­behandlungen
  anordnen und für die Durchsetzung ihrer Anordnungen neben der Polizei
  auch Militär sowie private Wachdienste einsetzen.''
\item
  Die Polizei im deutschen Bundesland Nordrhein-Westfalen
  \href{https://rp-online.de/nrw/panorama/nrw-polizei-testet-drohnen-bei-einsaetzen-wegen-corona-massnahmen_aid-50006143}{testet
  Drohnen} bei Corona-Einsätzen, konkret zur Suche nach verbotenen
  Menschenansammlungen.
\item
  Das deutsche Bundesland Sachsen will Quarantäne-Verweigerer
  \href{https://www.welt.de/politik/deutschland/article207198029/Coronavirus-Sachsen-will-Quarantaene-Verweigerer-in-Psychiatrien-sperren.html}{in
  Psychiatrien sperren}.
\item
  Ein ``corona-kritischer'' Schweizer Arzt wurde verhaftet und
  \href{https://www.srf.ch/news/regional/aargau-solothurn/festnahme-von-corona-kritiker-verschwoerung-oder-normale-intervention-der-aargauer-behoerden}{in
  die Psychiatrie eingewiesen}, da er ``Drohungen gegen Behörden und
  Angehörige'' geäußert habe.
\item
  In Deutschland hat eine Fachanwältin für Medizinrecht eine
  Verfassungsbeschwerde gegen die Corona-Maßnahmen eingereicht und einen
  Offenen Brief dazu veröffentlicht, in dem sie vor einem Abgleiten in
  einen Polizeistaat warnt und unter anderem zur Anmeldung von
  Demonstrationen aufrief. Die Staats­anwalt­schaft und Polizei haben
  daraufhin
  \href{https://www.morgenweb.de/mannheimer-morgen_artikel,-coronavirus-aufruf-zu-straftaten-ermittlungen-gegen-heidelberger-rechtsanwaeltin-_arid,1627078.html}{Ermittlungen}
  gegen die Anwältin aufgenommen wegen ``Aufruf zu einer Straftat'', die
  Internetseite der Anwältin wurde zeitweise abgeschaltet. Die
  Verfassungsbeschwerde wurde inzwischen abgelehnt.
\item
  Auch in Österreich haben inzwischen mehrere Anwälte Beschwerden gegen
  die Corona-Maßnahmen beim Verfassungsgerichtshof
  \href{https://wien.orf.at/stories/3043172/}{eingereicht}. Grundrechte
  und Gewaltenteilung seien durch die Maßnahmen verletzt, argumentieren
  die Anwälte.
\item
  Der Bürgermeister von Los Angeles
  \href{https://townhall.com/tipsheet/bethbaumann/2020/04/04/la-mayor-garcetti-says-snitches-get-rewards-for-ratting-out-their-neighbors-n2566348}{versprach
  eine Belohnung} für ``Petzer'' (\emph{snitches}), die ihre Nachbarn
  den Behörden melden, wenn sie die Ausgangs­sperren verletzen.
\item
  In den USA sind aufgrund des Lockdowns bereits über 16 Millionen
  Menschen
  \href{https://www.nytimes.com/2020/04/09/us/coronavirus-us-news.html}{arbeitslos},
  das entspricht rund 10\% der Arbeitsbevölkerung. Laut der
  Internationalen Arbeitsagentur ILO sind derzeit 80\% der weltweit 3,3
  Milliarden Arbeitskräfte von den Maßnahmen
  \href{https://www.ilo.org/global/about-the-ilo/newsroom/news/WCMS_740893/lang--en/index.htm}{betroffen}.
  1,25 Milliarden Arbeitskräfte könnten von ``drastischen oder
  katastrophalen'' Folgen betroffen sein.
\end{itemize}

\includegraphics{https://swprs.files.wordpress.com/2020/04/us-jobless-claims.png?w=600\&h=375}

\hypertarget{7-april-2020}{%
\paragraph{7. April 2020}\label{7-april-2020}}

\hypertarget{einschuxe4tzung-des-chefs-der-hamburger-rechtsmedizin}{%
\subparagraph{\texorpdfstring{\textbf{Einschätzung des Chefs der
Hamburger
Rechtsmedizin}}{Einschätzung des Chefs der Hamburger Rechtsmedizin}}\label{einschuxe4tzung-des-chefs-der-hamburger-rechtsmedizin}}

Professor Klaus Püschel, Chef der Hamburger Rechtsmedizin,
\href{https://www.pressreader.com/germany/hamburger-morgenpost/20200403/281487868456736}{erklärte
zu Covid19}: ``Die­ses Vi­rus be­ein­flusst in ei­ner völ­lig
über­zo­ge­nen Wei­se un­ser Le­ben. Das steht in kei­nem Ver­hält­nis
zu der Ge­fahr, die vom Vi­rus aus­geht. Und der as­tro­no­mi­sche
wirt­schaft­li­che Scha­den, der jetzt ent­steht, ist der Ge­fahr, die
von dem Vi­rus aus­geht, nicht an­ge­mes­sen. Ich bin über­zeugt, dass
sich die Co­ro­na-Sterb­lich­keit nicht mal als Peak in der
Jah­res­s­terb­lich­keit be­merk­bar ma­chen wird.'' So sei bis­her in
Ham­burg ``kein ein­zi­ger nicht vor­er­krank­ter Mensch'' an dem Vi­rus
ver­stor­ben: ``Al­le, die wir bis­her un­ter­sucht ha­ben, hat­ten
Krebs, ei­ne chro­ni­sche Lun­gen­er­kran­kung, wa­ren star­ke Rau­cher
oder schwer fett­lei­big, lit­ten an Dia­be­tes oder hat­ten ei­ne
Her­z-K­reis­lauf-Er­kran­kung. () Co­vid-19 ist nur im Aus­nah­me­fall
ei­ne töd­li­che Krank­heit, in den meis­ten Fäl­len je­doch ei­ne
über­wie­gend harm­los ver­lau­fen­de Vi­rus­in­fek­ti­on.''

\href{https://www.abendblatt.de/hamburg/article228828787/rechtsmedizin-pueschel-hamburg-corona-virus-infektion-covid-19-coronavirus-krise-patienten-krankenhaeuser-kliniken-infektionsrate-krankheit-pandemie-test-lungenkrankheit-sars-cov-epidemie-sars-cov-2.html}{Zudem
erklärte Dr. Püschel}: ``In nicht wenigen Fällen haben wir auch
festgestellt, dass die aktuelle Coronainfektion überhaupt nichts mit dem
tödlichen Ausgang zu tun hat, weil andere Todesursachen vorliegen, zum
Beispiel eine Hirnblutung oder ein Herzinfarkt.`` Corona an sich sei
eine ``nicht besonders gefährliche Viruserkrankung'', sagt der
Rechtsmediziner. Er plädiert für eine auf konkreten
Unter­suchungs­befunden beruhende Statistik. „Alle Mutmaßungen über
einzelne Todesfälle, die nicht sachkundig überprüft worden sind, schüren
nur Ängste.``

Die Freie und Hansestadt Hamburg hatte zuletzt, entgegen den Vorgaben
des Berliner Robert-Koch-Instituts, damit begonnen, zwischen Todesfällen
``mit'' und ``durch'' Coronaviren zu unterscheiden, was zu einem
\href{https://www.t-online.de/nachrichten/deutschland/id_87636856/coronavirus-hamburg-will-nur-echte-covid-19-tote-zaehlen.html}{Rückgang}
der Covid19-Todesfälle führte.

\hypertarget{weitere-medizinische-meldungen-1}{%
\subparagraph{\texorpdfstring{\textbf{Weitere medizinische
Meldungen}--}{Weitere medizinische Meldungen--}}\label{weitere-medizinische-meldungen-1}}

\begin{itemize}
\tightlist
\item
  Die\href{https://multipolar-magazin.de/artikel/coronavirus-regierung-ignoriert-daten}{neuesten
  Zahlen aus einem Spezialbericht} des deutschen Robert-Koch-Instituts
  zeigen, dass die sogenannte Positivenrate (d.h. die Anzahl
  Testpositiver pro Anzahl Tests) deutlich langsamer zunimmt als die von
  den Medien gezeigten Exponentialkurven und Ende März erst bei rund
  10\% lag, ein für Coronaviren grundsätzlich typischer Wert. Von einer
  ``gefährlich schnellen Verbreitung des Virus'' könne deshalb ``keine
  Rede sein'', so das Magazin Multipolar.
\item
  Der deutsche Virologe Hendrik Streeck führt derzeit eine Pilotstudie
  durch, um die Verbreitung und die Übertragungswege des
  Covid19-Erregers zu bestimmen. In
  \href{https://www.zeit.de/wissen/gesundheit/2020-04/hendrik-streeck-covid-19-heinsberg-symptome-infektionsschutz-massnahmen-studie/komplettansicht}{einem
  Interview erklärt er}: ``Ich habe mir die Fälle~von 31 der
  40~Verstorbenen aus dem Landkreis Heinsberg einmal genauer angeschaut
  -- und war nicht sehr überrascht, dass diese Menschen gestorben sind.
  Einer der Verstorbenen war älter als 100 Jahre, da hätte auch ein ganz
  normaler Schnupfen zum Tod führen können.'' Übertragungen durch
  Türklinken und dergleichen, d.h. sogenannte Schmier­infektionen, habe
  er bisher entgegen ursprünglicher Annahmen keine nachweisen können.
\item
  Erste Schweizer Krankenhäuser müssen aufgrund der sehr geringen
  Auslastung
  \href{https://www.engadinerpost.ch/2020/4/04/Engadiner-Spitaeler-haben-freie-Kapazitaeten}{Kurzarbeit
  anmelden}: ``Das Personal hat in allen Abteilungen zu wenig zu tun und
  hat in einem ersten Schritt Überzeiten abgebaut. Jetzt wird auch
  Kurzarbeit angemeldet. Die finanziellen Folgen sind gross.'' Zur
  Erinnerung: Eine auf unrealistischen Annahmen basierende Studie der
  ETH Zürich
  \href{https://www.toponline.ch/news/coronavirus/detail/news/studie-bestaetigt-engpass-bei-spitalbetten-steht-kurz-bevor-00131333/}{prog­nos­tizierte}
  für den 2. April erste Engpässe in Schweizer Kliniken. Dazu kam es
  bisher nirgends.
\item
  In der Schweiz gab es Anfang 2017 eine ausgeprägte Grippewelle. Damals
  kam es in den ersten sechs Wochen des Jahres zu knapp
  \href{https://www.srf.ch/news/schweiz/todesursachen-statistik-woran-die-meisten-schweizerinnen-und-schweizer-sterben}{1500
  zusätzlichen Sterbefällen} bei der über 65-jährigen Bevölkerung.
  Normalerweise sterben in der Schweiz rund
  \href{https://www.nzz.ch/lungenentzuendung-1.4550285}{1300 Personen}
  pro Jahr an den Folgen einer Lungenentzündung, wovon 95\% über 65
  Jahre alt sind. Zum Vergleich: Derzeit wird in der Schweiz von
  insgesamt \href{https://www.corona-data.ch/}{762 Todesfällen} mit
  (nicht durch) Covid19 berichtet.
\item
  Der Geschäftsführer eines deutschen Umweltlabors vermutet, dass die
  Bewohner der norditalienischen Lombardei aufgrund einer notorisch
  hohen Legionellenbelastung
  \href{https://m.apotheke-adhoc.de/nachrichten/detail/coronavirus/erhoehen-legionellen-die-todesrate-einer-corona-infektion/}{besonders
  anfällig für Vireninfektionen} wie Covid19 sind: ``Ist die Lunge wie
  in der aktuellen Situation durch eine Virusinfektion geschwächt, haben
  Bakterien leichtes Spiel, können den Krankheitsverlauf negativ
  beeinflussen und Komplikationen verursachen.'' In der Lombardei sei es
  bereits in der Vergangenheit durch mit Legionellen verseuchte
  Verdunstungs­kühlanlagen zu regionalen Pneumonie-Ausbrüchen gekommen.
\item
  Aufgrund von Angaben aus China wurden weltweit medizinische Protokolle
  definiert, die für testpositive Intensivpatienten rasch eine
  \textbf{invasive künstliche Beatmung durch Intubation} vorsehen.
  Einerseits gehen die Protokolle davon aus, dass eine schonungs­vollere
  nicht-invasive Beatmung durch eine Maske zu schwach sei, andererseits
  besteht vor allem die Befürchtung, das ``gefährliche Virus'' könne
  sich sonst duch Aerosole verbreiten. Bereits im März haben deutsche
  Mediziner aber
  \href{https://www.doccheck.com/de/detail/articles/26271-covid-19-beatmung-und-dann}{darauf
  aufmerksam gemacht}, dass die Intubation zu zusätzlichen Lungenschäden
  führen kann und eine insgesamt schlechte Erfolgsaussicht hat.
  Inzwischen haben sich auch US-Mediziner gemeldet, die beschreiben,
  dass die Intubation den Patienten
  \href{https://www.youtube.com/watch?v=k9GYTc53r2o}{``mehr schade als
  nütze''}. Die Patienten würden oftmals nicht an einem akuten
  Lungenversagen leiden, sondern eher an einer Art Höhenkrankheit, die
  durch die künstliche Beatmung mit erhöhtem Druck noch verschlimmert
  werde. Bereits im Februar
  \href{https://www.upi.com/Top_News/World-News/2020/02/14/Oxygen-therapy-working-for-coronavirus-patient-Seoul-says/6651581696794/}{meldeten
  südkoreanische Mediziner} hingegen, dass kritische Covid19-Patienten
  gut auf eine Sauerstofftherapie ohne Beatmungsgerät ansprechen. Der
  oben genannte US-Mediziner warnt, man müsse den Einsatz der
  Beatmungsgeräte dringend überdenken, um keine zusätzlichen Schäden zu
  verursachen.
\item
  Das offizielle US-Modell für Covid19 hat die Hospitalisierungen bisher
  achtfach, die Intensiv­patienten sechsfach, und die erforderlichen
  Beatmungsgeräte vierzigfach
  \href{https://twitter.com/NikolovScience/status/1246823479820693505}{überschätzt}.
\item
  Der bekannte US-Statistiker Nate Silver erklärt, warum die Angabe der
  Corona-Fallzahlen
  \href{https://fivethirtyeight.com/features/coronavirus-case-counts-are-meaningless/}{``sinnlos''}
  ist, solange man nicht mehr über die Anzahl und~ Durchführung der
  Tests wisse.
\item
  Ein \href{https://www.youtube.com/watch?v=EpSdCh1KT1A}{Beitrag von ARD
  Monitor} zur übertrieben dargestellten ``Schweinegrippe'' von 2009
  zeigt erstaunliche Parellelen zur heutigen Situation. Das Fazit des
  ARD-Beitrags lautete damals: ``Die eigentliche Pandemie ist die Angst
  vor ihr.''
\end{itemize}

\hypertarget{weitere-meldungen-1}{%
\subparagraph{\texorpdfstring{\textbf{Weitere
Meldungen}}{Weitere Meldungen}}\label{weitere-meldungen-1}}

\begin{itemize}
\tightlist
\item
  Die\href{http://wodarg.com}{Internetseite von Dr. Wolfgang Wodarg},
  einem der frühesten und international bekanntesten Kritiker der
  ``Covid19-Panik'', wurde heute vom deutschen Anbieter Jimdo für einen
  Tag \href{https://twitter.com/wodarg}{gelöscht} und erst nach starken
  Protesten wieder aufgeschaltet. Es ist nicht bekannt, ob die zeitweise
  Löschung aufgrund von allgemeinen Beschwerden oder aufgrund von einer
  politischen Anweisung erfolgte.
\item
  Bereits zuvor wurde die universitäre E-Mail-Adresse des emeritierten
  Professors Dr. Sucharit Bhakdi, der einen
  \href{https://swprs.org/offener-brief-von-professor-sucharit-bhakdi-an-bundeskanzlerin-dr-angela-merkel/}{Offenen
  Brief an Bundeskanzlerin Angela Merkel} verfasste, deaktiviert, nach
  Protesten aber ebenfalls wieder reaktiviert.
\item
  Das dänische Parlament hat am 2.
  April~\href{https://www.ft.dk/ripdf/samling/20191/lovforslag/l157/20191_l157_som_vedtaget.pdf}{ein
  neues Gesetz erlassen}, dass es den Behörden erlaubt,
  ``betrügerische'' Websites zu Covid19 zunächst auch ohne richterlichen
  Beschluss zu sperren und die Betreiber mit einem erhöhten Strafmaß zu
  belegen. Noch ist unklar, was dies für allgemein kritische Websites zu
  Covid19 und der diesbezüglichen Regierungs­politik bedeutet. (Hinweis:
  Der Absatz wurde präzisiert und das Gesetz direkt verlinkt.)
\item
  Der deutsche Publizist und Journalist Harald Wiesendanger schreibt in
  einem Artikel,
  \href{https://www.nachrichten-fabrik.de/news/harald-wiesendanger-ueber-die-massenmedien-waehrend-der-corona-krise-ich-schaeme-mich---meines-berufsstands-152103}{dass
  sein Berufsstand in der derzeitigen Krise völlig versage}: ``Wie ein
  Berufsstand, der als unabhängige, kritische, unvoreingenommene Vierte
  Gewalt die Mächtigen kontrollieren soll, ebenso blitzschnell wie
  nahezu einmütig derselben kollektiven Hysterie erliegen kann wie sein
  Publikum und sich für Hofberichterstattung, Regierungspropaganda,
  expertengläubige Vergöt­te­rung der Heiligen Kuh Wissenschaft hergibt:
  Das ist mir unbegreiflich, es widert mich an, ich habe genug davon,
  ich distanziere mich voller Fremdscham von dieser unwürdigen
  Performance.''
\item
  Derzeit befindet sich
  \href{https://www.sciencealert.com/one-third-of-the-world-s-population-are-now-restricted-in-where-they-can-go}{über
  ein Drittel der Menschheit} in einem ``Lockdown'', das sind mehr
  Menschen als zur Zeit des Zweiten Weltkriegs lebten.
\item
  In den USA sind die Gesuche für Arbeitslosengeld
  \href{https://www.reuters.com/article/us-health-coronavirus-usa-layoffs/us-weekly-jobless-claims-seen-at-record-high-again-idUSKBN21K0FX}{auf
  über sechs Millionen hochgeschnellt} (siehe Grafik darin), ein seit
  der Großen Depression von 1929 historisch einmaliger Wert.
\item
  Über einhundert Menschen- und Bürger­rechts­organisationen
  \href{https://www.dailymail.co.uk/news/article-8181381/World-sleepwalking-surveillance-state-rights-groups-warn.html}{warnen
  davor}, dass die Welt durch die Corona-Krise ``in einen
  Über­wachungs­staat schlafwandelt''. Auf Twitter hat sich inzwischen
  neben dem Hastag \#covid19 auch \#covid1984 etabliert.
\item
  Der US-Geostratege Henry Kissinger schreibt im \emph{Wall Street
  Journal},
  ``\href{https://www.wsj.com/articles/the-coronavirus-pandemic-will-forever-alter-the-world-order-11585953005}{Die
  Coronavirus-Pandemie wird für immer die Weltordnung verändern''}. Die
  USA müssten ihre Bürger ``beschützen'' und gleichzeitig ``eine neue
  Epoche planen''.
\end{itemize}

\hypertarget{5-april-2020}{%
\paragraph{5. April 2020}\label{5-april-2020}}

\begin{itemize}
\tightlist
\item
  In einem aufschlussreichen
  \href{https://www.youtube.com/watch?v=lGC5sGdz4kg}{40-minütigen
  Interview} erklärt der international renommierte
  Epidemiologie-Professor Knut Wittkowski aus New York, dass die
  getroffenen Maßnahmen zu Covid19 allesamt kontraproduktiv seien. Statt
  ``social distancing'', Schulschließungen, ``lock down'', Mundschutz,
  Massentests und Impfungen müsse das Leben möglichst ungestört
  weitergehen und möglichst rasch eine Immunität in der Bevölkerung
  aufgebaut werden. Covid-19 sei nach allen bisherigen Erkenntnissen
  nicht gefährlicher als frühere Grippeepidemien.
\item
  Das British Medical Journal (BMJ)
  \href{https://www.bmj.com/content/369/bmj.m1375}{berichtet}, dass laut
  neuesten Daten aus China 78\% der testpositiven Personen keine
  Symptome zeigen. Ein Oxford-Epidemiologe sagt dazu: ``Diese Resultate
  sind sehr, sehr wichtig. () Wenn diese Resultate repräsentativ sind,
  dann müssen wir uns fragen, warum zur Hölle wir einen Lockdown
  betreiben?''
\item
  Dr. Andreas Sönnichsen, Leiter der Abteilung für Allgemein- und
  Familienmedizin an der Medizinischen Universität Wien sowie
  Vorsitzender des Netzwerks für evidenzbasierte Medizin,
  \href{https://www.diepresse.com/5794224/was-machen-wir-da-auf-den-intensivstationen-eigentlich}{hält
  die bisher verfügten Maßnahmen für ``irre''}. Der ganze Staat werde
  lahmgelegt, nur um ``die wenigen, die es betreffen könnte, zu
  schützen''.
\item
  Die schwedische Regierung hat
  \href{https://www.telegraph.co.uk/news/2020/04/03/coronavirus-swedish-experiment-could-prove-britain-wrong/}{als
  erste weltweit angekündigt}, künftig offiziell zwischen Todesfällen
  ``durch'' und Todesfällen ``mit'' dem Coronavirus zu unterscheiden.
  Dies dürfte zu einer Reduktion der gemeldeten Todesfälle führen.
  Derweil nimmt der internationale Druck auf Schweden, seine liberale
  Strategie aufzugeben, interessanterweise laufend zu.
\item
  Das Gesundheitsamt von Hamburg lässt testpositive Sterbefälle neu
  \href{https://www.t-online.de/nachrichten/deutschland/id_87636856/coronavirus-hamburg-will-nur-echte-covid-19-tote-zaehlen.html}{durch
  die Rechtsmedizin untersuchen}, um nur noch ``echte''
  Corona-Todesfälle zu zählen. Dadurch habe sich die Anzahl der
  Todesfälle im Vergleich zu den Angaben des Robert-Koch-Instituts
  bereits um bis zu 50\% reduziert.
\item
  Das deutsche Ärzteblatt berichtete bereits 2018 von einer
  \href{https://www.aerzteblatt.de/nachrichten/97750/Vielzahl-an-Lungenentzuendungen-beunruhigen-Behoerden-in-Norditalien}{``Vielzahl
  an Lungen­ent­zündungen''} in Norditalien, die die Behörden
  beunruhigten. Damals wurde unter anderem verunreinigtes Trinkwasser
  als Grund vermutet.
\item
  Die deutsche Pharmazeutische Zeitung
  \href{https://www.pharmazeutische-zeitung.de/atemstillstand-koennte-auch-zentrale-ursache-haben-116664/}{weist
  daraufhin}, dass in der aktuellen Situation oftmals ``Patienten schwer
  erkranken, sogar versterben, ohne zuvor respiratorische Symptome
  entwickelt zu haben.'' Neurologen vermuten diesbezüglich, dass die
  Coronaviren auch Nervenzellen schädigen könnten. Eine andere Erklärung
  wäre indes, dass diese oftmals pflegebedürftigen Patienten durch den
  sehr hohen Stress versterben.
\item
  Laut den
  \href{https://www.bag.admin.ch/dam/bag/de/dokumente/mt/k-und-i/aktuelle-ausbrueche-pandemien/2019-nCoV/covid-19-lagebericht.pdf.download.pdf/COVID-19_Epidemiologische_Lage_Schweiz.pdf}{neuesten
  Zahlen aus der Schweiz} sind die häufigsten Symptome der testpositiven
  Patienten in Krankenhäusern Fieber, Husten und Atembeschwerden. Bei
  43\% oder ca. 900 Personen liegt eine Lungenentzündung vor. Auch in
  diesen Fällen ist indes nicht a priori klar, ob sie durch das
  Coronavirus oder durch andere Erreger ausgelöst wurde. Der
  Altersmedian der testpositiven Verstorbenen liegt bei 83 Jahren, die
  Spanne reicht bis 101 Jahre.\\
\item
  Das britische Projekt \href{http://inproportion2.talkigy.com/}{``In
  Proportion''} verfolgt die Sterblichkeit ``mit'' Covid19 im Vergleich
  zur Sterblichkeit durch Influenza und zur Gesamtsterblichkeit, die
  auch in Großbritannien weiterhin im Normalbereich oder darunter liegt
  und derzeit abnimmt.
\item
  Im US-Bundesstaat Indiana haben aufgrund des Lockdowns und der
  wirtschaftlichen Folgen die Anrufe bei der Hotline für psychische
  Probleme und Suizidgedanken um über 2000\% von 1000 auf 25,000 Anrufe
  pro Tag
  \href{https://twitter.com/JesseKellyDC/status/1246449878219145216}{zugenommen}.
\item
  Das medizinische Fachportal Rxisk
  \href{https://rxisk.org/medications-compromising-covid-infections/}{weist
  darauf hin}, dass verschiedene Medikamente das Infektionsrisiko für
  Coronaviren um teilweise bis zu 200\% erhöhen können.
\end{itemize}

\hypertarget{weitere-meldungen-2}{%
\subparagraph{\texorpdfstring{\textbf{Weitere
Meldungen}}{Weitere Meldungen}}\label{weitere-meldungen-2}}

\begin{itemize}
\tightlist
\item
  Der britische Journalist Peter Hitchens beschreibt in einem Artikel
  mit dem Titel
  \href{https://www.firstthings.com/web-exclusives/2020/04/we-love-big-brother}{``We
  love Big Brother''}, wie sich auch zuvor kritische Menschen trotz
  fehlender medizinischer Evidenz von der Angst anstecken ließen. In
  einem Interview erklärt er mit Blick auf die bedrohten Grundrechte,
  dass Kritik derzeit
  \href{https://www.spiked-online.com/podcast-episode/in-this-lockdown-dissent-is-a-moral-duty/}{``eine
  moralische Verpflichtung''} sei.
\item
  Der deutsche Historker René Schlott schreibt über das
  \href{https://www.spiegel.de/politik/deutschland/corona-krise-und-buergerrechte-rendezvous-mit-dem-polizeistaat-a-68611322-f4d4-453f-aba5-5ec5a49ae329}{``Rendesvouz
  mit dem Polizeistaat''}: ``Ein Buch kaufen, auf einer Parkbank sitzen,
  sich mit Freunden treffen -- das ist jetzt verboten, wird kontrolliert
  und denunziert. Die demokratischen Sicherungen scheinen durchgebrannt.
  Wo und wie soll das enden?''
\item
  In Deutschland bereiten mehrere Anwaltskanzleien Klagen gegen die
  erlassenen Maßnahmen und Verordnungen vor. Eine Fachanwältin für
  Medizinrecht
  \href{http://beatebahner.de/lib.medien/aktualisierte\%20Pressemitteilung.pdf}{schreibt
  in einer Pressemitteilung}: ``Die Maßnahmen der Bundes- und
  Landesregierung sind eklatant verfassungswidrig und verletzen in
  bisher nie gekanntem Ausmaß eine Vielzahl von Grundrechten der Bürger
  in Deutschland. Dies gilt für alle Corona-Verordnungen der 16
  Bundesländer. Insbesondere sind diese Maßnahmen nicht durch das
  Infektionsschutzgesetz gerechtfertigt, welches erst vor wenigen Tagen
  in Windeseile überarbeitet wurde. ()~ Denn die vorliegenden Zahlen und
  Statistiken zeigen, dass die Corona-Infektion bei mehr als 95 \% der
  Bevölkerung harmlos verläuft (oder vermutlich sogar bereits verlaufen
  ist) und somit keine schwerwiegende Gefahr für die Allgemeinheit
  darstellt.''
\item
  Der
  \href{https://swprs.org/offener-brief-von-professor-sucharit-bhakdi-an-bundeskanzlerin-dr-angela-merkel/}{Offene
  Brief} von Professor Sucharit Bhakdi and Bundeskanzlerin Angela Merkel
  ist inwzsichen in Deutsch, Englisch, Französisch, Spanisch, Russisch,
  Türkisch, Niederländisch und Estnisch verfügbar, weitere Sprachen
  folgen.
\item
  In einem \href{https://www.youtube.com/watch?v=-pcQFTzck_c}{Interview
  (EN/DE)} warnt NSA-Whistleblower Edward Snowden, dass Covid19
  gefährlich, aber temporär sei, die Zerstörung der Grundrechte hingegen
  tödlich und permanent.
\end{itemize}

\hypertarget{3-april-2020}{%
\paragraph{3. April 2020}\label{3-april-2020}}

\textbf{USA}: Weitere
\href{https://www.youtube.com/watch?v=5pIMD1enwd4}{Videos von
Bürgerjournalisten} zeigen, dass es in manchen von US-Medien als
``Kriegszonen'' beschriebenen Krankenhäusern in der Realität weiterhin
sehr ruhig ist. (Hinweis: Einige Autoren ziehen daraus
\href{https://www.politifact.com/factchecks/2020/apr/03/facebook-posts/hospital-beds-being-kept-empty-prepare-covid-influ/}{falsche
Folgerungen}.)

\textbf{Österreich}: Auch in Österreich werden die ``Corona-Todesfälle''
offenbar ``sehr liberal'' definiert,
\href{https://www.heute.at/s/osterreich-bei-corona-todesstatistik-sehr-liberal-48665863}{wie
Medien berichten}: ``Zählt man auch als Corona-Toter, wenn man mit dem
Virus infiziert, aber an etwas anderem gestorben ist? Ja, sagen Rudi
Anschober und Bernhard Benka, Mitglied der Corona-Task Force im
Gesundheitsministerium. ``Es gibt eine klare Regel derzeit: Gestorben
mit dem Coronavirus oder an dem Coronavirus'', führt Benka aus. Alle
diese Fälle zählen zur Statistik. Ein Unterschied, woran der Patient
tatsächlich gestorben ist, wird nicht gemacht. Flapsig formuliert zählt
also auch ein 90-Jähriger, der mit einem Oberschenkelhalsbruch stirbt
und sich in den Stunden vor seinem Tod mit Corona infiziert, als
Corona-Toter. Um nur ein Beispiel zu nennen.''

\textbf{Deutschland}: Das deutsche Robert-Koch-Institut rät neu von
Autopsien testpositiver Verstorbener ab, da das Risiko einer
Tröpfcheninfektion durch Aerosole angeblich
\href{https://www.youtube.com/watch?v=gSn_YaOYYcY}{zu hoch sei}. Dadurch
kann jedoch die wirkliche Todesursache in vielen Fällen nicht mehr
bestimmt werden.

Ein Facharzt für Pathologie
\href{https://www.youtube.com/watch?v=gSn_YaOYYcY}{kommentiert} dies wie
folgt (Brief unter Video abgedruckt): ``Ein Schelm, wer böses dabei
denkt! Bisher war es für Pathologen selbstverständlich, mit
entsprechenden Sicherheitsvorkehrungen auch bei infektiösen Erkrankungen
wie HIV/AIDS, Hepatitis, Tuberkulose, PRION-Erkrankungen usw. zu
obduzieren. Es ist schon bemerkenswert, dass bei einer Seuche, die über
den ganzen Globus hinweg Tausende von Patienten dahin rafft und die
Wirtschaft ganzer Länder nahezu zum Stillstand bringt, nur äußerst
spärliche Obduktionsbefunde (sechs Patienten aus China) vorliegen.
Sowohl aus seuchenpolizeilicher als auch aus wissenschaftlicher Sicht
sollte hier doch ein besonders großes öffentliches Interesse an
Obduktionsbefunden bestehen. Das Gegenteil ist aber der Fall. Hat man
Angst, davor, die wahren Todesursachen der positiv getesteten
Verstorbenen zu erfahren? Könnte es sein, dass die Zahlen der
Corona-Toten dann dahin schmelzen würden wie Schnee in der
Frühlingssonne.''

\textbf{Italien}: Russisches Fachpersonal habe
\href{https://de.sputniknews.com/panorama/20200402326767475-fachpersonal-todesfaelle-lombardei-zeitung/}{``merkwürdige
Todesfälle''} in Pflegeheimen in der Lombardei bemerkt: ``So wurden in
der Stadt Gromo Zeitungsangaben zufolge mehrere Fälle registriert, in
denen angebliche Coronavirus-Infizierte einfach eingeschlafen und nicht
wieder erwacht sind. Bei den Verstorbenen wurden bis dahin keine
ernstzunehmenden Symptome der Krankheit festgestellt. () Wie der
Direktor des Seniorenheims später im Gespräch mit RIA Novosti
präzisierte, sei es unklar, ob die Gestorbenen tatsächlich mit dem
Coronavirus infiziert wurden, weil niemand in dem Heim darauf getestet
worden sei. () In den Heimen, in denen Ärzte- und Pflegeteams aus
Russland tätig sind, werden Korridore, Bettenzimmer und Speiseräume
desinfiziert.''

Ähnliche Fälle wurden auch aus Deutschland bereits
\href{https://web.archive.org/web/20200330082928/https://www.sueddeutsche.de/panorama/coronavirus-news-deutschland-wolfsburg-laschet-1.4828033}{berichtet}:
Pflegepatienten \emph{ohne Krank­heits­symptome} sterben in der
aktuellen Ausnahmesituation plötzlich und gelten dann als
``Corona-Todesfälle''. Hier stellt sich erneut die folgenschwere Frage:
Wer stirbt am Virus, und wer stirbt an den teilweise extremen Maßnahmen?

\textbf{Pflegekräfte}: Die Süddeutsche Zeitung
\href{https://www.sueddeutsche.de/politik/coronavirus-pflegekraefte-ausland-1.4866124}{berichtet}:
``In ganz Europa gefährdet die Pandemie die Versorgung alter Menschen zu
Hause, weil Pflegekräfte nicht mehr zu ihnen können -- oder das
jeweilige Land fluchtartig verlassen haben Richtung Heimat.''

\textbf{Weiteres}: Stanford-Medizinprofessor Dr. Jay Bhattacharya gab
ein \href{https://www.youtube.com/watch?v=-UO3Wd5urg0}{halbstündiges
Interview}, in dem er den ``conventional wisdom'' zu Covid19 infrage
stellt. Die bisherigen Maßnahmen seien auf sehr unsicherer und teilweise
fragwürdiger Datenbasis beschlossen worden.

\hypertarget{2-april-2020-i}{%
\paragraph{2. April 2020 (I)}\label{2-april-2020-i}}

\hypertarget{deutschland}{%
\subparagraph{\texorpdfstring{\textbf{Deutschland}}{Deutschland}}\label{deutschland}}

Laut dem
\href{https://influenza.rki.de/Wochenberichte/2019_2020/2020-13.pdf}{neuesten
Influenza-Bericht} des deutschen Robert-Koch-Instituts ist die Anzahl
der akuten Atemwegs­erkrankungen zuletzt~ ``bundesweit stark gesunken''.
Die Werte seien ``in allen Altersgruppen stark zurückgegangen.''

Bis zum 20. März (KW12) sei die Gesamtzahl stationär behandelter Fälle
mit akuten Atemwegs­erkrankungen deutlich gesunken. In der Altersgruppe
ab 80 Jahre und älter habe sich die Fallzahl im Vergleich zur Vorwoche
sogar fast halbiert.

In den 73 untersuchten Krankenhäusern haben 7 \% aller Fälle mit
Atemwegserkrankungen eine COVID-19-Diagnose erhalten. In den
Altersgruppen 35-59 Jahre waren es 16\% und in der Altersgruppe 60-79
Jahre waren es 13\%, die eine COVID-19-Diagnose erhielten.

Diese Zahlen entsprechen jenen aus anderen Ländern sowie der
grundsätzlich typischen Verbreitung von Coronaviren (5 bis 15\%).

\href{https://swprs.files.wordpress.com/2020/04/rki-ili-kw13.png}{}

\includegraphics{https://swprs.files.wordpress.com/2020/04/rki-ili-kw13.png?w=279\&h=171}

Chřipková onemocnění (RKI, 13.kal. týden)

\href{https://swprs.files.wordpress.com/2020/04/rki-sari-kw12.png}{}

\includegraphics{https://swprs.files.wordpress.com/2020/04/rki-sari-kw12.png?w=449\&h=171}

Akutní onemocnění dýchacích cest v nemocnicích

Grippeähnliche Erkrankungen insgesamt und Akute Atemwegserkrankungen in
Krankenhäusern (Robert-Koch-Institut, KW13 und KW12)

Ein
\href{https://www.zeit.de/wissen/2020-04/krankenhaeuser-kapazitaeten-coronavirus-patienten-deutschland/seite-2}{Artikel
in der ZEIT} befasst sich mit der Frage der Intensivpatienten in
Deutschland:

``Zurzeit beobachten Politiker, Fachleute und viele Bürger täglich mit
Sorge die exponentiell steigende Zahl der Menschen, die sich neu
infizieren. Jedoch ist das nicht die entscheidende Kennziffer, um
einschätzen zu können, wie schwer die Corona-Krise Deutschland trifft
und treffen wird. Denn sie wird vor allem durch die Zahl der seit Wochen
immer stärker gesteigerten Tests verfälscht.

Um die Belastung des Gesundheitssystems zu messen, ist dagegen vor allem
Zahl derjenigen wichtig, die so schwer erkrankt sind, dass sie beatmet
werden müssen. Solange es genug Beatmungsplätze für sie gibt, können
sehr viele von ihnen gerettet werden. Erst wenn diese Betten knapp
werden, droht eine Situation wie in Italien.

Das DIVI-Register zeigt nun, dass die Lage auf den deutschen
Intensivstationen bisher entspannt ist. ``Noch sind wir in einem
komfortablen Bereich'', sagt Grabenhenrich. Die Zahl der schwer
Erkrankten steige längst nicht so steil wie die Zahl der Infizierten und
selbst wenn, könne man noch sehr viele Intensivbetten mit einer sehr
guten Ausstattung zur Verfügung stellen.''

\hypertarget{schweiz-2}{%
\subparagraph{\texorpdfstring{\textbf{Schweiz}}{Schweiz}}\label{schweiz-2}}

Das Schweizer Bundesamt für Gesundheit (BAG)
\href{https://www.bag.admin.ch/bag/de/home/krankheiten/ausbrueche-epidemien-pandemien/aktuelle-ausbrueche-epidemien/novel-cov/situation-schweiz-und-international.html}{meldet},
dass bisher circa 139,330 Covid19-Tests durchgeführt wurden, davon fiel
das Resultat bei 15\% positiv aus (PDF). Auch diese Zahl entspricht dem
aus anderen Ländern bekannten und für Coronaviren typischen Wert und
scheint, soweit ersichtlich, auch in der Schweiz bisher nicht
zuzunehmen.

Lediglich die in den Medien oft genannte Anzahl der Tests nimmt
exponentiell zu, nicht jedoch die Anzahl der ``Infizierten'', Erkrankten
oder gar Verstorbenen.

Am 31. März wurde indes eine
\href{https://www.bfs.admin.ch/bfs/de/home/statistiken/gesundheit/gesundheitszustand/sterblichkeit-todesursachen.html}{neue
wöchentliche Mortalitätsstatistik} publiziert, die in der Schweiz für
die 12. Kalenderwoche (bis zum 22. März) erstmals eine erhöhte
Gesamtmortalität in der Altersgruppe 65+ prognostiziert (siehe Grafik
unten). Konkret soll die Gesamtmortalität um rund 200 Todesfälle
\emph{pro Woche} zunehmen.

Diese Zunahme sei ``Ausdruck der gegenwärtigen Pandemie''. Hier ergibt
sich folgendes Problem: Bis zum 22. März gab es in der Schweiz
\emph{insgesamt}
\href{https://de.wikipedia.org/wiki/COVID-19-Pandemie_in_der_Schweiz\#Todesf\%C3\%A4lle}{106
testpositive Todesfälle}. Eine Zunahme um 200 Todesfälle \emph{pro
Woche} würde bedeuten, dass ein Großteil der zusätzlichen Sterblichkeit
nicht durch das Virus, sondern durch die ``Gegenmaßnahmen'' verursacht
wird.

Eine andere Erklärung wäre, dass die ca. 200 testpositiven Todesfälle
der \emph{Folgewoche}
(\href{https://de.wikipedia.org/wiki/COVID-19-Pandemie_in_der_Schweiz\#Todesf\%C3\%A4lle}{KW13})
bereits eingerechnet wurden. Dies würde bedeuten, dass \emph{alle}
testpositiven Todesfälle als \emph{zusätzliche} Todesfälle angenommen
werden. Angesichts des Alters- und Krankheitsprofils sowie der
\href{https://swprs.org/rki-relativiert-corona-todesfaelle/}{inernationalen
Erfahrungen} wäre dies jedoch eine durchaus zweifelhafte Annahme.

Tatsächlich wird im Bericht denn auch angemerkt: ``Diese ersten
Schätzungen sind noch sehr unsicher, sodass keine exakten Zahlen
publiziert werden können.''

Sollte sich herausstellen, dass ein Großteil der testpositiven
Todesfälle (Medianalter: 83 Jahre) \emph{keine} zusätzlichen Todesfälle
sind, so wäre die Gesamtmortalität entweder nicht erhöht, oder sie wäre
hauptsächlich wegen der drastischen Maßnahmen erhöht, wie von manchen
Experten
\href{https://swprs.org/offener-brief-von-professor-sucharit-bhakdi-an-bundeskanzlerin-dr-angela-merkel/}{befürchtet}.

\includegraphics{https://swprs.files.wordpress.com/2020/04/bfs-mortaliaet-22-03.png?w=600\&h=400}

Der Schweizer Tages-Anzeiger hat die aktuelle Gesamtmortalität im
Vergleich mit früheren Jahren
\href{https://interaktiv.tagesanzeiger.ch/2020/uebersterblichkeit-wegen-coronavirus/}{dargestellt}
(siehe Grafik unten). Dies illustriert, dass die jetzige Mortalität,
selbst falls tatsächlich erhöht, weiterhin unter den stärkeren
Grippewintern der vergangenen Jahre liegt.

\includegraphics{https://swprs.files.wordpress.com/2020/04/mortalitc3a4t-schweiz.png?w=720\&h=339}

\hypertarget{usa}{%
\subparagraph{\texorpdfstring{\textbf{USA}}{USA}}\label{usa}}

Ein Schweizer Biophysiker hat
\href{https://swprs.org/rate-of-positive-covid19-tests/}{den Umstand
visualisiert}, dass in den USA (wie im Rest der Welt) nicht die Anzahl
der ``Infizierten'' exponentiell zunimmt, sondern die Anzahl der Tests.
Die Anzahl der Test­positiven in Relation zur Anzahl an Tests bleibt
konstant oder steigt nur langsam, was im Prinzip \emph{gegen} eine
exponentielle virale Epidemie spricht.

\includegraphics{https://swprs.files.wordpress.com/2020/04/ud-data-2-fs.png?w=736}

\hypertarget{weiteres}{%
\subparagraph{\texorpdfstring{\textbf{Weiteres}}{Weiteres}}\label{weiteres}}

\begin{itemize}
\tightlist
\item
  Für Großbritannien bestimmte Viren-Testkits mussten
  \href{https://www.telegraph.co.uk/news/2020/03/30/uks-attempt-ramp-coronavirus-testing-hindered-key-components/}{zurückgerufen
  werden}, da sie bereits Coronaviren-Bestandteile enthielten.
\item
  Die Studie des britischen Imperial College, die hunderttausende
  zusätzliche Todesfälle prognostizierte, jedoch nie in einer
  Fachzeitschrift veröffentlicht oder einem Review unterzogen wurde,
  \href{https://judithcurry.com/2020/04/01/imperial-college-uk-covid-19-numbers-dont-seem-to-add-up/}{basierte
  auf weitgehend unrealistischen Annahmen}, wie sich nun zeigte.
\item
  Die BBC fragt,
  \href{https://www.bbc.com/news/health-51979654}{``Werden die
  Todesfälle durch das Coronavirus verursacht?''}, und antwortet: ``Es
  könnte eine Hauptursache sein, ein zusätzlicher Faktor, oder einfach
  auch noch da.'' So sei ein 18 Jahre alter Mann als ``jüngstes
  Corona-Opfer'' gemeldet worden, da ein Test am Tag vor seinem Tod
  positiv war. Das Krankenhaus habe später jedoch mitgeteilt, der junge
  Mann sei an einer schweren Vorerkrankung gestorben.
\item
  Die europäische Gesundheitsbehörde ECDC hat
  \href{https://www.ecdc.europa.eu/sites/default/files/documents/COVID-19-safe-handling-of-bodies-or-persons-dying-from-COVID19.pdf}{sehr
  strenge Vorgaben veröffentlicht} für den Umgang mit testpositiven oder
  ``vermutlich testpositiven'' Leichnamen. Angesichts der bisher sehr
  tiefen Mortalitätsraten erscheinen solche Vorgaben aus medizinischer
  Sicht durchaus fraglich; sie erhöhen jedoch die Belastung für das
  Gesundheits- und Bestattungswesen signifikant, und sind zugleich sehr
  medienwirksam.
\item
  Der Bayerische Rundfunk hat einen
  \href{https://www.br.de/nachrichten/wissen/bhakdis-brief-an-die-kanzlerin-was-ist-dran-an-seinen-fragen,RutYDhd}{kritischen
  Kommentar} zum Offenen Brief von Professor Sucharit Bhakdi an
  Bundeskanzlerin Angela Merkel veröffentlicht.
\item
  In der ARTE-Dokumentation
  \href{https://www.youtube.com/watch?v=1--c2SBYlMY}{``Profiteure der
  Angst''} von 2009 wird gezeigt, wie die hauptsächlich privat
  finanzierte WHO eine milde Grippewelle (die sogenannte
  ``Schweinegrippe'') zu einer globalen Pandemie hochstufte und in der
  Folge für mehrere Milliarden Dollar teilweise gefährliche Impfstoffe
  an die Regierungen verkauft wurden. Einige der damaligen Protagonisten
  sind auch in der heutigen Situation wieder
  \href{https://www.nature.com/articles/news.2009.424}{prominent
  vertreten}.
\item
  Der frühere Richter am britischen Supreme Court, Jonathan Sumption,
  erklärte
  \href{https://www.spectator.co.uk/article/former-supreme-court-justice-this-is-what-a-police-state-is-like-}{in
  einem BBC-Interview} zu den britischen Maßnahmen: ``So sieht ein
  Polizeistaat aus.''
\end{itemize}

\hypertarget{2-april-2020-ii}{%
\paragraph{2. April, 2020 (II)}\label{2-april-2020-ii}}

\begin{itemize}
\tightlist
\item
  Schon 2018 titelte der britische Guardian:
  ``\href{https://www.theguardian.com/society/2018/dec/09/steep-rise-lung-related-illness-hospitals-nhs}{Pollution
  and flu bring steep rise in lung-related illnesses}``. Shortage of
  specialists adds to worries that surge in respiratory diseases is
  putting pressure on A\&Es.
\item
  Inzwischen beschweren sich selbst
  \href{https://pflege-prisma.de/2020/03/31/sterbezahlen-in-pflegeheimen/}{Vertreter
  von Pflegeheimen} über die restriktiven Maßnahmen und die
  unangebrachte Medien­bericht­erstattung zu Covid19: ``Auch vor Corona
  kam es in Wintermonaten häufig vor, dass in relativ kurzer Zeit viele
  Heimbewohner starben, ohne dass hier Fernsehteams vor der Tür standen
  und in Schutzanzüge gehüllte Personen gezeigt werden, die sich
  heldenhaft der Infektionsgefahr aussetzen.''
\item
  Zahlen aus der norditalienischen Stadt Treviso (bei Venedig) zeigen,
  dass die Gesamtmortalität in den städtischen Krankenhäusern bis Ende
  März trotz 108 testpositiven Verstorbenen
  \href{https://swprs.files.wordpress.com/2020/04/reppublica-treviso.jpg}{in
  etwa gleich blieb} wie in den Vorjahren. Dies ist ein weiterer Hinweis
  darauf, dass die temporär erhöhte Mortalität an einigen Orten eher mit
  Drittfaktoren wie Panik und Kollaps zu tun haben dürften, als nur mit
  dem Coronavirus.
\item
  Professor Martin Haditsch, Facharzt für Mikrobiologie, Virologie und
  Infektions­epidemiologie, übt
  \href{https://www.youtube.com/watch?v=PtzHH8DhgZM}{scharfe Kritik an
  den Covid19-Maßnahmen}. Diese seien ``völlig haltlos'' und würden
  ``Augenmaß und ethische Grundsätze'' mit Füßen treten.
\item
  Professor John Oxford von der Queen Mary Universität London, ein
  weltweit führender Virologe und Influenza-Spezialist, kommt zu
  \href{https://novuscomms.com/2020/03/31/a-view-from-the-hvivo-open-orphan-orph-laboratory-professor-john-oxford/}{folgender
  Einschätzung bezüglich Covid19}: ``Persönlich würde ich sagen, dass
  der beste Ratschlag ist, weniger Zeit mit dem Anschauen von
  Fernsehnachrichten zu verbringen, die sensationell und nicht sehr gut
  sind. Ich persönlich halte diesen Covid-Ausbruch für eine schlimme
  Winter­grippe­epidemie. In diesem Fall hatten wir im letzten Jahr 8000
  Todesfälle in den Risikogruppen, d.h. über 65\% Menschen mit
  Herzkrankheiten usw. Ich glaube nicht, dass der aktuelle Covid diese
  Zahl überschreiten wird. Wir leiden unter einer Medienepidemie!''
\end{itemize}

\hypertarget{1-april-2020}{%
\paragraph{1. April 2020}\label{1-april-2020}}

\hypertarget{zur-situation-in-italien}{%
\subparagraph{\texorpdfstring{\textbf{Zur Situation in
Italien}}{Zur Situation in Italien}}\label{zur-situation-in-italien}}

Italienische Ärzte berichteten, dass sie bereits Ende letztes Jahr
schwere Lungen­ent­zündungen in Norditalien
\href{https://www.scmp.com/news/china/society/article/3076334/coronavirus-strange-pneumonia-seen-lombardy-november-leading}{beobachteten}.
Genetische Analysen zeigen nun aber, dass das ``Covid19-Virus'' offenbar
erst im Januar in Italien auftauchte. ``Die schweren
Lungen­ent­zündungen, die im November und im Dezember in Italien
diagnostiziert wurden, müssen also auf einen anderen Erreger
zurückzuführen sein'',
\href{https://www.nzz.ch/wissenschaft/coronavirus-der-stammbaum-verraet-woher-es-kommt-ld.1548271}{so
die NZZ}. Dies wirft einmal mehr die Frage auf, welche Rolle das
Covid19-Virus in der italienischen Situation tatsächlich spielt, und
welche Rolle andere Faktoren spielen.

Am 30. März wurde hier auf die Liste der ``während der Corona-Krise''
verstorbenen italienischen Ärzte aufmerksam gemacht, von denen viele in
Wirklichkeit längst pensioniert und bis 90 Jahre alt waren und mit der
Krise nicht direkt zu tun hatten. Heute nun wurden auf der Liste
\href{https://portale.fnomceo.it/elenco-dei-medici-caduti-nel-corso-dellepidemia-di-covid-19/}{alle
Geburts­jahr­e entfernt} (siehe aber die letzte
\href{https://web.archive.org/web/20200328152430/https://portale.fnomceo.it/elenco-dei-medici-caduti-nel-corso-dellepidemia-di-covid-19/}{Archiv-Version}).
Ein seltsamer Vorgang.

Außerdem erreicht uns folgende Mitteilung eines Beobachters aus Italien,
der weitere Aspekte zur dramatischen Situation in Italien anführt, die
weit über ein Virus hinausreichen dürfte:

``In den letzten Wochen haben die meisten osteuropäischen Pflegekräfte,
die im 24 Stunden Dienst 7 Tage die Woche in der Betreuung von
Pflegebedürftigen in Italien arbeiteten, fluchtartig das Land verlassen.
Dies nicht zuletzt wegen der Panikmache und den von den
„Notstandsregierungen`` angedrohten Ausgangssperren und
Grenzschließungen. Deshalb wurden alte pflegebedürftige Personen und
Behinderte, teilweise ohne Verwandte, von ihren Betreuern hilflos
zurückgelassen.

Viele von diesen verlassenen Menschen landeten dann nach einigen Tagen
in den seit Jahren permanent überlasteten Krankenhäusern, weil sie unter
anderem dehydriert waren. Leider fehlte den Spitälern jetzt auch noch
das Personal, welches eingesperrt in den Wohnungen auf die Kinder
aufpassen mussten, weil Schulen und Kindergärten geschlossen worden
waren. Dies führte dann in der Folge zum vollkommenen Zusammenbruch der
Behinderten- und Altenpflege gerade in den Gebieten, wo weitere noch
härtere „Maßnahmen`` angeordnet wurden und zu chaotischen Verhältnissen.

Der Pflegenotstand, der durch die Panik entstand, führte temporär zu
vielen Todesopfern unter den Pflegebedürftigen und zunehmend auch unter
jüngeren Patienten der Krankenhäuser. Diese Todesopfer dienten dann den
Verantwortlichen und den Medien dazu, die Leute in noch mehr Panik zu
versetzen, indem sie zum Beispiel meldeten „weitere 475 Todesopfer``,
„Die Toten werden von der Armee aus den Krankenhäusern geholt``,
untermalt mit Bildern von aufgereihten Särgen und Armeelastwagen.

Das war jedoch die Folge der Angst der Bestattungsunternehmer vor dem
„Killervirus``, die deshalb ihre Dienste verwehrten. Außerdem waren es
zum einen zu viele Todesfälle auf einmal und zum anderen wurde von der
Regierung ein Gesetz erlassen, dass die Leichen, die den Coronavirus
trugen eingeäschert werden mussten. In Italien wurden bis zu diesem
Datum nur wenige Feuerbestattungen vollzogen. Deshalb gab es nur wenige
kleine Krematorien, die sehr schnell an Ihre Grenzen stießen. Die
Verstorbenen mussten deshalb in verschiedenen Kirchen aufgebahrt werden.

Diese Entwicklung lief im Prinzip in allen Ländern gleich ab. Die
Qualität des Gesund­heits­systems hat jedoch einen erheblichen Einfluss
auf die Auswirkungen. Deshalb gibt es in Deutschland, Österreich oder
der Schweiz weniger Probleme als in Italien, Spanien oder den USA. Wie
man aber in den offiziellen Zahlen sehen kann, gibt es keine
nennenswerte Erhöhung der Mortalitätsrate. Nur einen kleiner Berg, der
von dieser Tragödie stammt.''

\includegraphics{https://swprs.files.wordpress.com/2020/03/covid-iss-stat-bloomberg.png?w=550\&h=301}

\hypertarget{kliniken-in-den-usa-deutschland-und-der-schweiz}{%
\subparagraph{\texorpdfstring{\textbf{Kliniken in den USA, Deutschland
und der
Schweiz}}{Kliniken in den USA, Deutschland und der Schweiz}}\label{kliniken-in-den-usa-deutschland-und-der-schweiz}}

\begin{itemize}
\tightlist
\item
  Der US-Fernsehsender CBS
  \href{https://nypost.com/2020/04/01/cbs-admits-to-using-footage-from-italy-in-report-about-nyc/}{wurde
  dabei ertappt}, wie er Aufnahmen einer italienischen Intensivstation
  in einem Beitrag zur aktuellen Situation in New York verwendete, ohne
  dies zu kennzeichnen.
\item
  Das Register der deutschen Intensivstationen zeigt entgegen
  Medienberichten ebenfalls
  \href{https://www.intensivregister.de/\#/intensivregister}{keine
  erhöhte Belegung}. Bürgerjournalisten berichten
  von~\href{https://www.in-opr.de/2020/03/28/coronakrise-und-ruppiner-kliniken-was-stimmt-hier-nicht/}{verlassenen
  Covid19-Aufnahme­zentren}. Ein Mitarbeiter einer Münchner Klinik
  erklärt, dass man ``seit Wochen auf die Welle warte'', aber es gebe
  ``keine Steigerung der Patienten­zahlen''. Die Aussagen der Politiker
  würden sich nicht mit den eigenen Erfahrungen decken, der ``Mythos des
  Killervirus'' könne ``nicht bestätigt'' werden.
\item
  Auch in Schweizer Kliniken ist bisher keine erhöhte Belegung zu
  erkennen. Ein Besucher des Kantons­spitals Luzern berichtet, dass dort
  ``weniger los ist als in Norma­lzeiten''. Ganze Stockwerke seien für
  Covid19 gesperrt worden, aber das Personal ``warte immer noch auf die
  Patienten''. Auch die Krankenhäuser in Bern, Basel, Zug und Zürich
  seien ``leergefegt''. Selbst im Tessin seien die Intensivstationen
  \href{https://www.nzz.ch/schweiz/tessin-verlegt-erste-corona-patienten-in-deutschschweizer-spitaeler-ld.1549417}{nicht
  ausgelastet}, dennoch würden nun Patienten in die deutschschweizer
  Abteilungen verlegt. Rein medizinisch macht dies kaum Sinn.
\end{itemize}

\hypertarget{weitere-medizinische-meldungen-2}{%
\subparagraph{\texorpdfstring{\textbf{Weitere medizinische
Meldungen}}{Weitere medizinische Meldungen}}\label{weitere-medizinische-meldungen-2}}

\begin{itemize}
\tightlist
\item
  Der Infektiologe und Direktor des Universitätsklinikums
  Hamburg-Eppendorf, Dr. Ansgar Lohse,
  \href{https://www.mopo.de/hamburg/uke-infektiologe-fordert-es-muessen-sich-mehr-menschen-mit-corona-infizieren-36483636}{fordert
  ein rasches Ende der Ausgangssperren und Kontaktverbote}. Es müssten
  sich \emph{mehr} Menschen mit Corona infizieren. Kitas und Schulen
  sollen möglichst bald wieder geöffnet werden, damit Kinder und ihre
  Eltern durch eine Ansteckung mit dem Coronavirus immun werden können.
  Die Fortdauer der strikten Maßnahmen würde zu einer Wirtschaftskrise
  führen, die ebenfalls Menschenleben kostet, so der Mediziner.
\item
  In Spanien seien
  \href{https://www.heise.de/tp/features/Das-ist-keine-Krise-sondern-eine-Katastrophe-4694104.html}{15\%
  der testpositiven Personen} Ärzte und Krankenpfleger. Diese bleiben
  zwar zumeist weitgehend symptomfrei, müssen sich jedoch in Quarantäne
  begeben, wodurch das spanische Gesundheitssystem zunehmend kollabiert.
\item
  Dr. John Lee, emeritierter Professor für Pathologie, befasst sich im
  britischen \emph{The Spectator} mit der
  \href{https://www.spectator.co.uk/article/how-to-understand-and-report-figures-for-covid-19-deaths-}{stark
  irreführenden Definition und Kommunikation} von
  ``Corona-Todesfällen''.
\item
  Die
  \href{https://swprs.files.wordpress.com/2020/04/die-lage-in-norwegen.pdf}{neuesten
  Daten aus Norwegen}, ausgewertet von einer promovierten
  Umwelt­toxikologin, zeigen ebenfalls, dass die Rate der Testpositiven
  nicht -- wie bei einer Epidemie zu erwarten wäre -- zunehmen, sondern
  im für Coronaviren normalen Bereich zwischen 2 und 10\% pendeln. Das
  Durchschnittsalter der testpositiven Verstorbenen liegt bei 84 Jahren,
  die Todesursachen werden nicht öffentlich mitgeteilt, eine
  Übersterblichkeit besteht nicht.
\item
  Schweden, das bisher ohne radikale Maßnahmen auskommt und keine
  erhöhte Mortalität meldet (ähnlich wie asiatische Länder wie Japan
  oder Südkorea), wird von internationalen Medien bemerkenswerterweise
  \href{https://www.theguardian.com/world/2020/mar/30/catastrophe-sweden-coronavirus-stoicism-lockdown-europe}{unter
  Druck gesetzt}, seine Strategie zu ändern.
\item
  Daten aus dem Bundesstaat New York zeigen, dass die
  Hospitalisierungsrate von testpositiven Personen
  \href{https://www.nytimes.com/2020/03/27/nyregion/new-rochelle-coronavirus.html}{über
  zwanzigmal tiefer liegen könnte} als ursprünglich angenommen.
\item
  Ein
  \href{https://www.doccheck.com/de/detail/articles/26271-covid-19-beatmung-und-dann}{Beitrag
  auf dem Fachportal DocCheck} thematisiert die Problematik der Beatmung
  testpositiver Patienten. Bei testpositiven Patienten wird offiziell
  von einer einfachen Beatmung durch eine Maske abgeraten. Einerseits
  wird vermutet, diese könnte zu schwach sein; andererseits wird
  befürchtet, das Coronavirus könnte sich durch Aerosole verbreiten.
  Deshalb werden testpositive Intensivpatienten oftmals direkt
  intubiert. Die Intubation habe aber schlechte Erfolgsaussichten und
  führe oft zu zusätzlichen Schäden in der Lunge (sog.
  Ventilator-induzierte Lungenschäden). Wie schon bei der Medikation, so
  stellt sich auch hier die Frage, ob eine schonungsvollere Behandlung
  der Patienten medizinisch nicht sinnvoller wäre.
\end{itemize}

\hypertarget{weitere-meldungen-3}{%
\subparagraph{**Weitere Meldungen}\label{weitere-meldungen-3}}

**

\begin{itemize}
\tightlist
\item
  Ein deutscher Landesminister hat die Bevölkerung dazu
  \href{https://de.nachrichten.yahoo.com/strobl-bürger-verstöße-gegen-corona-regeln-polizei-melden-095746341.html}{aufgerufen},
  ``wachsam zu sein und Verstöße gegen die Regeln zur Eindämmung der
  Corona-Epidemie der Polizei zu melden.''
  ``\href{https://www.br.de/nachrichten/bayern/buerger-melden-eifrig-verstoesse-gegen-corona-regeln,RuGXp1h}{Eifrig
  gemeldet}'' würden etwa ``verbotene Grüppchen-Bildung, Kinder auf
  Spielplätzen oder Feiern''. Auch Wanderer im Allgäu wurden bereits
  angezeigt.
\item
  Deutsche Verfassungsrechtler schlagen wegen
  \href{https://www.focus.de/politik/deutschland/corona-regelungen-der-regierung-medizin-darf-nicht-gefaehrlicher-sein-als-die-krankheit_id_11827625.html}{``schwerwiegender
  Grundrechtseingriffe''} Alarm. Verfassungsrechtler Hans Michael Heinig
  warnt, dass sich der ``demokratische Rechtsstaat in kürzester Frist in
  einen faschistoid-hysterischen Hygienestaat'' verwandeln könnte.
  Professor Christoph Möllers von der Berliner Humboldt Universität
  erklärt, dass das Infektionsschutzgesetz ``nicht als Grundlage für so
  weitreichende Einschränkungen der Freiheitsrechte der Bürger'' dienen
  könne. Laut dem ehemaligen Präsident des deutschen
  Bundesverfassungsgerichts, Hans Jürgen Papier, rechtfertigen
  ``Notlagen­maßnahmen nicht die Außerkraftsetzung von Freiheitsrechten
  zugunsten eines Obrigkeits- und Überwachungsstaates''.
\item
  In mehreren Ländern wurden Online-Petitionen zur Beendigung der
  Ausgangssperren und anderer Eingriffe in die Grundrechte gestartet.
  Zugleich kommt es vermehrt zur Löschung von kritischen Videobeiträgen,
  selbst von Ärzten. In Berlin wurde eine angemeldete Veranstaltung zu
  Grundrechten, auf der das deutsche Grundgesetz verteilt wurde,
  \href{https://www.heise.de/tp/features/Wenn-Demonstranten-zu-Gefaehrdern-erklaert-werden-4692869.html}{von
  der Polizei aufgelöst}.
\end{itemize}

\hypertarget{31-muxe4rz-2020-i}{%
\paragraph{31. März 2020 (I)}\label{31-muxe4rz-2020-i}}

Dr. Richard Capek und andere Forscher haben
\href{https://coronadaten.wordpress.com/}{bereits gezeigt}, dass die
Anzahl der testpositiven Personen im Verhältnis zur Anzahl der
durchgeführten Tests in allen untersuchten Ländern konstant bleibt, was
\emph{gegen} eine exponentielle Ausbreitung (``Epidemie'') des Virus
spricht und lediglich auf eine exponentielle Zunahme der Tests
hindeutet.

Je nach Land liegt der Anteil der testpositiven Personen zwischen circa
5 und 15\%, was der üblichen Verbreitung von Coronaviren entspricht.
Interessanterweise werden diese \emph{konstanten Zahlenwerte} von
Behörden und Medien nicht aktiv kommuniziert
(\href{https://multipolar-magazin.de/artikel/coronavirus-irrefuhrung-fallzahlen}{oder
sogar entfernt}). Stattdessen werden exponentielle, aber irrelevante und
irreführende Kurven ohne Kontext gezeigt.

Dies entspricht selbstverständlich nicht den professionellen
medizinischen Standards, wie auch ein Blick in den
traditionellen\href{https://influenza.rki.de/Saisonberichte/2017.pdf}{Influenza-Bericht}
des deutschen Robert-Koch-Instituts zeigt (S. 30, siehe Grafik unten).
Hier wird neben der Anzahl Nachweise (rechts) auch die \emph{Anzahl
Proben} (links, graue Balken) sowie die \emph{Positivenrate} (links,
blaue Kurve) ausgewiesen.

Dadurch wird sichbar, dass die Positivenrate während einer Grippesaison
von 0 bis 10\% auf bis zu 80\% der Proben hochschnellt und nach einigen
Wochen wieder auf den Normalwert absinkt. Im Vergleich dazu weisen
Covid19-Tests eine konstante Positivenrate im Normalbereich aus (s.u.)

\includegraphics{https://swprs.files.wordpress.com/2020/03/rki-influenza-report-2017.png?w=650\&h=530}

Kontante Covid19-Positivenrate am Beispiel der USA (Dr. Richard Capek).
Diese gilt analog für alle übrigen Länder, für die derzeit Daten zur
Anzahl der Proben verfügbar ist.

\includegraphics{https://swprs.files.wordpress.com/2020/03/infizierte-pro-test2603.jpg?w=600\&h=325}

\hypertarget{31-muxe4rz-2020-ii}{%
\paragraph{31. März 2020 (II)}\label{31-muxe4rz-2020-ii}}

\begin{itemize}
\tightlist
\item
  Eine
  \href{https://off-guardian.org/2020/03/30/covid19-yet-to-impact-europes-overall-mortality/}{graphische
  Darstellung der europäischen Monitoringdaten} zeigt eindrucksvoll,
  dass die Gesamtsterblichkeit in ganz Europa, unabhängig von den
  jeweils getroffenen Maßnahmen, bis zum 25. März im Normalbereich oder
  darunter liegt, sowie oftmals deutlich unter den Werten der Vorjahre.
  Einzig in Italien (65+) war die Gesamtsterblichkeit zuletzt erhöht
  (vermutlich aus mehreren Gründen), lag indes immer noch unter früheren
  Grippewintern.
\item
  Der Präsident des deutschen Robert-Koch-Instituts bestätigte in einer
  weiteren Presse­konferenz, dass Vorerkrankungen und wirkliche
  Todesursache für die Definition von sogenannten ``Corona-Todesfällen''
  \href{https://swprs.org/rki-relativiert-corona-todesfaelle/}{keine
  Rolle spielen} (siehe Video unten). Aus medizinischer Sicht ist eine
  solche Definition klarerweise irreführend. Sie hat den
  offensichtlichen und allgemein bekannten Effekt, dass Politik und
  Gesellschaft in Angst versetzt werden.
\item
  In Italien zeichnet sich inzwischen eine
  \href{https://www.tagesspiegel.de/politik/die-verlangsamung-ist-da-in-italien-zeichnet-sich-die-wende-in-der-coronakrise-ab/25698124.html}{Beruhigung
  der Situation} ab. Soweit bisher ersichtlich handelte es sich bei den
  temporär erhöhten Sterberaten (65+) um sehr lokale Effekte, die
  oftmals mit einer Maßenpanik und einem Zusammenbruch der
  Kranken­versorgung einhergingen. Ein norditalienischer Politiker fragt
  etwa, „wie es kommt, dass Covid-Patienten aus Brescia sogar nach
  Deutschland transportiert werden, während im nahen Venetien, in
  Verona, zwei Drittel der Intensivbetten leer stehen``.
\item
  Der Stanford-Medizinprofessor John Ioannidis kritisiert
  \href{https://onlinelibrary.wiley.com/doi/abs/10.1111/eci.13222}{in
  einem Artikel} im \emph{European Journal of Clinical Investigation}
  die ``Schäden durch übertriebene Informationen und
  nicht-evidenzbasierte Maßnahmen.'' Selbst Fachjournale hätten zu
  Beginn unseriöse Behauptungen publiziert.
\item
  Eine chinesische Studie, die Anfang März im \emph{Chinese Journal of
  Epidemiology} veröffentlicht wurde und die Unzuverlässigkeit der
  Covid-Virentests nachwies (ca. 50\% falsch-positive Resultate bei
  Asymptomatischen), wurde inzwischen wieder zurückgezogen. Der
  Hauptautor der Studie, immerhin Dekan einer medizinischen Fakultät,
  wollte den Grund für den Rückzug nicht nennen und sprach von einer
  \href{https://www.npr.org/sections/health-shots/2020/03/26/822084429/in-defense-of-coronavirus-testing-strategy-administration-cited-retracted-study}{``heiklen
  Angelegenheit''}. Unabhängig von dieser Studie ist die
  Fehleranfälligkeit von sog. PCR-Virentests indes seit langem bekannt:
  2003 wurde etwa in einem kanadischen Pflegeheim eine Masseninfektion
  mit SARS-Coronaviren ``nachgewiesen'', die sich später als gewöhnliche
  Erkältungs-Coronaviren
  \href{https://www.ncbi.nlm.nih.gov/pmc/articles/PMC2095096/}{herausstellten}
  (die für Risikogruppen auch tödlich sein können).
\item
  Autoren des deutschen \emph{Risk Management Networks RiskNET} sprechen
  \href{https://www.risknet.de/themen/risknews/covid-19-und-der-blindflug/}{in
  einer Analyse} zu Covid19 von einem ``Blindflug'' sowie ``mangelhafter
  Datenkompetenz und Datenethik''.~ Statt immer mehr Tests und Maßnahmen
  sei eine repräsentative Stichprobe erforderlich. Die ``Sinnhaftigkeit
  und Ratio'' der getroffenen Maßnahmen müsse kritisch hinterfragt
  werden.
\item
  Das spanische Interview mit dem international anerkannten,
  argentinisch-französischen Virologen Pablo Goldschmidt wurde
  \href{https://www.rubikon.news/artikel/der-corona-totalitarismus}{ins
  Deutsche übersetzt}. Goldschmidt hält die getroffenen Maßnahmen für
  medizinisch kontraproduktiv und merkt an, man müße jetzt ``Hannah
  Arendt lesen'', um die ``damaligen Ursprünge des Totalitarismus'' zu
  verstehen.
\item
  Der ungarische Premierminister Viktor Orban hat, wie bereits andere
  Premiers und Präsidenten, im Rahmen eines ``Notstandsgesetzes'' das
  ungarische Parlament \href{https://www.krone.at/2127086}{weitgehend
  entmachtet} und kann nun im Wesentlichen per Dekret regieren.
\end{itemize}

\hypertarget{30-muxe4rz-2020-i}{%
\paragraph{30. März 2020 (I)}\label{30-muxe4rz-2020-i}}

\begin{itemize}
\tightlist
\item
  In Deutschland können einige Kliniken keine Patienten mehr annehmen --
  dies jedoch nicht, weil es zuviele Patienten oder zuwenige Betten
  gäbe, sondern weil
  \href{https://web.archive.org/web/20200330082928/https://www.sueddeutsche.de/panorama/coronavirus-news-deutschland-wolfsburg-laschet-1.4828033}{das
  Pflege­personal positiv getestet wurde}, obschon es in den meisten
  Fällen kaum Symptome zeigt. Hier wird erneut deutlich, wie und warum
  das Gesundheitssystem paralysiert wird.
\item
  In einem deutschen Alters- und Pflegeheim für stark demenzerkrankte
  Menschen sind 15 test­positive Menschen
  \href{https://web.archive.org/web/20200330082928/https://www.sueddeutsche.de/panorama/coronavirus-news-deutschland-wolfsburg-laschet-1.4828033}{verstorben}:
  ``Überraschend viele Menschen seien verstorben, \textbf{ohne dass sie
  Symptome von Corona gezeigt hätten.}'' Ein deutscher Facharzt schreibt
  uns dazu: ``Aus meiner ärztlichen Sicht spricht einiges dafür, dass
  einige dieser Menschen möglicherweise an den Folgen der Maßnahmen
  gestorben sind. Demente Menschen geraten in Hochstress, wenn sich
  Entschei­dendes an ihrem Alltag ändert: Isolation, kein Körperkontakt,
  evtl. vermummte PflegerInnen.'' Dennoch werden auch diese Verstorbenen
  in den deutschen und internationalen Statistiken als ``Corona-Tote''
  gezählt. Im Zusam­men­hang mit der ``Corona-Krise'' kann man nun also
  auch an einer Krankheit sterben, ohne überhaupt deren Symptome zu
  haben.
\item
  Das Schweizer Inselspital in Bern hat laut einem Pharmakologen wegen
  der Angst vor Covid19 Personal zwangsbeurlaubt, Therapien gestoppt und
  OPs verschoben.
\item
  Professor Gérard Krause, Abteilungsleiter Epidemiologie am deutschen
  Helmholtz-Zentrum für Infektionsforschung, warnt im deutschen ZDF
  davor, dass die Anti-Corona-Maßnahmen
  \href{https://www.zdf.de/nachrichten/politik/coronavirus-epidemiologe-folgen-helmholtz-100.html}{``zu
  mehr Toten führen könnten als das Virus selbst''}.
\item
  Verschiedene Medien berichteten, dass in Italien bereits über 50 Ärzte
  ``während der Corona-Krise'' gestorben seien, wie Soldaten im Krieg.
  Ein Blick auf die
  \href{https://web.archive.org/web/20200328152430/https://portale.fnomceo.it/elenco-dei-medici-caduti-nel-corso-dellepidemia-di-covid-19/}{entsprechende
  Liste} zeigt indes, dass es sich bei vielen der Verstorbenen um
  pensionierte Doktoren verschiedener Fachrichtungen handelte, darunter
  90-jährige Psychiater und Kinderärzte, die kaum an der
  ``Corona-Front'' gefallen sein dürften.
\item
  Laut einer
  \href{https://www.buzzfeed.com/albertonardelli/coronavirus-testing-iceland}{umfangreichen
  Untersuchung in Island} zeigten 50\% aller testpositiven Personen
  ``keiner­lei Symptome'', während die anderen 50\% zumeist ``sehr
  moderate, erkältungsähnliche Symptome'' zeigten. Laut den isländischen
  Daten liegt die Sterblichkeit von Covid19 im Promille­bereich, d.h. im
  Bereich der Grippe oder darunter. Von den zwei testpositiven
  \href{https://www.government.is/news/article/?newsid=c65cf658-6eb6-11ea-9462-005056bc4d74}{Verstorbenen}
  sei zudem einer ``ein Tourist mit unüblichen Symptomen gewesen''.
  (Weitere \href{https://www.covid.is/data}{Island-Daten})
\item
  Der britische Journalist Peter Hitchens
  \href{https://hitchensblog.mailonsunday.co.uk/2020/03/theres-powerful-evidence-this-great-panic-is-foolish-yet-our-freedom-is-still-broken-and-our-economy.html}{schreibt},
  ``Es gibt deutliche Evidenz, dass diese große Panik dumm ist. Aber
  unsere Freiheiten sind noch immer beschränkt und unsere Wirtschaft
  zerstört.'' Hitchens macht darauf aufmerksam, dass in Teilen
  Großbritanniens Polizeidrohnen ``nicht-essentielle'' Spaziergänge von
  Menschen in der Natur
  \href{https://www.youtube.com/watch?v=fHNxDzLsPeg}{überwachen und
  melden}. Teilweise werden die Menschen von Polizeidrohnen
  \href{https://www.youtube.com/watch?v=D4GEZjUTkqc}{per Lautsprecher
  aufgefordert}, nach Hause zu gehen, ``um Leben zu retten.''
  (Anmerkung: Soweit hatte selbst George Orwell noch nicht gedacht.)
\item
  Der italienische Geheimdienst
  \href{https://www.focus.de/panorama/welt/sorge-vor-sozialen-unruhen-supermaerkte-gepluendert-apotheken-ueberfallen-italiens-geheimdienst-warnt-vor-aufstaenden_id_11826664.html}{warnt
  vor} sozialen Unruhen und Aufständen. Es würden bereits Supermärkte
  geplündert und Apotheken überfallen.
\item
  Professor Sucharit Bhakdi hat inzwischen
  \href{https://www.youtube.com/watch?v=LsExPrHCHbw\&feature=emb_title}{ein
  Video veröffentlicht} (deutsch/englisch), in dem er seinen
  \href{https://swprs.org/offener-brief-von-professor-sucharit-bhakdi-an-bundeskanzlerin-dr-angela-merkel/}{Offenen
  Brief} an Bundeskanzlerin Dr. Angela Merkel erläutert.
\end{itemize}

\hypertarget{30-muxe4rz-2020-ii}{%
\paragraph{30. März 2020 (II)}\label{30-muxe4rz-2020-ii}}

In mehreren Ländern mehren sich im Zusammenhang mit Covid19 die
Anzeichen, dass ``die Behandlung schlimmer als die Erkrankung'' sein
könnte.

Dabei geht es einerseits um das Risiko von sogenannten
\href{https://de.wikipedia.org/wiki/Nosokomiale_Infektion}{nosokomialen
Infektionen}, das heißt Infektionen, die sich der womöglich nur leicht
erkrankte Patient erst im Krankenhaus zuzieht. Für Europa wird mit 2.5
Millionen nosokomialen Infektionen und 50,000 damit verbundenen
Todesfällen \emph{pro Jahr} gerechnet. Selbst auf deutschen
Intensivstationen erleiden rund 15\% der Patienten eine nosokomiale
Infektion, darunter auch Lungenentzündungen bei künstlicher Beatmung.
Ein beson­deres Problem sind überdies die zunehmend
antibiotikaresistenten Keime in Krankenhäusern.

Ein weiterer Aspekt sind die sicherlich gutgemeinten, aber teilweise
sehr aggressiven Behandlungs­methoden, die bei Covid19-Erkrankten
vermehrt zum Einsatz kommen. Hierzu zählt insbesondere die Verabreichung
von Steroiden, Antibiotika und anti-viralen Medikamenten (oder eine
Kombination davon). Bereits bei der Behandlung von SARS-1 Patienten
zeigte sich, dass das Erebnis \emph{mit} einer solchen Behandlung
\href{https://www.sciencedaily.com/releases/2020/02/200206110703.htm}{oft
schlechter und tödlicher war}, als ohne eine solche Behandlung.

\hypertarget{29-muxe4rz-2020}{%
\paragraph{29. März 2020}\label{29-muxe4rz-2020}}

\begin{itemize}
\tightlist
\item
  Der emeritierte Mainzer Professor für Medizinische Mikrobiologie, Dr.
  Sucharit Bhakdi, schrieb am Donnerstag, 26. März 2020 einen
  \href{https://swprs.org/offener-brief-von-professor-sucharit-bhakdi-an-bundeskanzlerin-dr-angela-merkel/}{Offenen
  Brief an die deutsche Bundeskanzlerin Dr. Angela Merkel}, in dem er
  eine dringende Neubewertung der Reaktion auf Covid19 fordert und der
  Kanzlerin fünf entscheidende Fragen stellt.
  (\href{https://swprs.org/open-letter-from-professor-sucharit-bhakdi-to-german-chancellor-dr-angela-merkel/}{Englische
  Übersetzung})
\item
  Die
  \href{https://multipolar-magazin.de/artikel/coronavirus-irrefuhrung-fallzahlen}{neuesten
  Daten des Robert-Koch-Instituts} zeigen, dass sich die Zunahme der
  testpositiven Personen proportional zur Zunahme der Anzahl Tests
  verhält, \textbf{d.h. prozentual in etwa gleich bleibt}. Dies könnte
  darauf hindeuten, dass die Zunahme der Fallzahlen im Wesentlichen aus
  einer Zunahme der Anzahl der Tests resultiert, und nicht aus einer
  laufenden Epidemie.
\item
  Die Mailänder Mikrobiologin Maria Rita Gismondo
  \href{https://www.secoloditalia.it/2020/03/coronavirus-la-gismondo-ammonisce-duramente-basta-snocciolare-numeri-sui-positivi-sono-dati-falsati/}{ruft
  die italienische Regierung auf}, die tägliche Anzahl der
  ``Corona-Positiven'' nicht mehr zu kommunizieren, da diese Zahlen
  ``gefälscht'' seien und die Bevölkerung in eine unnötige Panik
  versetzen. Die Anzahl der Testpositiven hänge stark von der Art und
  Anzahl der Tests ab und sage nichts über den Gesundheitszustand aus.
\item
  Dr. John Ioannidis, Stanford-Professor für Medizin und Epidemiologie,
  gab ein
  \href{https://www.youtube.com/watch?v=d6MZy-2fcBw}{einstündiges
  Interview} zur fehlenden Datengrundlage bezüglich der
  Covid19-Maßnahmen.
\item
  Der in Frankreich lebende, argentinische Virologe Pablo Goldschmidt
  hält die politische Reaktion auf Covid19 für ``völlig übertrieben''
  und warnt vor
  \href{https://www.infobae.com/coronavirus/2020/03/28/para-un-prestigioso-cientifico-argentino-el-coronavirus-no-merece-que-el-planeta-este-en-un-estado-de-parate-total/}{``totalitären
  Maßnahmen''.} In Frankreich werde die Bewegung der Menschen teilweise
  bereits mit Dronen überwacht.
\item
  Der 1934 geborene, italienische Publizist Fulvio Grimaldi erklärt,
  dass die derzeit in Italien umgesetzten staatlichen Maßnahmen
  \href{https://www.youtube.com/watch?v=O3BuNp01vpc}{``schlimmer als im
  Faschismus''} seien. Parlament und Gesellschaft seien vollständig
  entmachtet worden.
\end{itemize}

\hypertarget{28-muxe4rz-2020}{%
\paragraph{28. März 2020}\label{28-muxe4rz-2020}}

\begin{itemize}
\tightlist
\item
  Eine
  \href{https://news.yahoo.com/oxford-study-suggests-millions-people-221100162.html}{neue
  Studie der Universität Oxford} kommt zum Ergebnis, dass Covid19
  vermutlich bereits seit Januar 2020 in Großbritannien existierte und
  inzwischen bereits die Hälfte der Bevölkerung infiziert und somit
  immunisiert sei, wobei die meisten Menschen keine oder nur sehr milde
  Symptome erlebten. Dies würde bedeuten, dass nur eine von eintausend
  Personen wegen Covid19 hospitalisiert werden müsse, ein
  vergleichsweise tiefer Wert.
  (\href{https://www.medrxiv.org/content/10.1101/2020.03.24.20042291v1}{Studie})
\item
  Britische Medien
  \href{https://www.bbc.com/news/uk-england-beds-bucks-herts-52041709}{berichteten}
  von einer 21 Jahre alten Frau, die ohne Vorerkrankungen an Covid19
  gestorben sei. Inzwischen
  \href{https://archive.is/20200329015127/https://www.theguardian.com/world/2020/mar/27/chloe-middleton-death-21-year-old-not-recorded-nhs-covid-19-related}{wurde
  jedoch bekannt}, dass die Frau nicht positiv auf Covid19 testete und
  an einem Herzversagen starb. Das Covid19-Gerücht sei entstanden,
  ``weil sie einen leichten Husten hatte''.
\item
  Der deutsche Medienwissenschaftler Professor Otfried Jarren
  kritisiert, viele Medien würden einen
  \href{https://www.deutschlandfunk.de/covid-19-scharfe-kritik-an-ard-und-zdf-wegen.2849.de.html?drn:news_id=1117133}{unkritischen
  Journalismus betreiben}, der Bedrohung und exekutive Macht inszeniere.
  Eine Differenzierung und eine echte Debatte zwischen Experten finde
  kaum statt.
\end{itemize}

\hypertarget{27-muxe4rz-2020-i}{%
\paragraph{27. März 2020 (I)}\label{27-muxe4rz-2020-i}}

\textbf{Italien}: Laut den
\href{http://www.salute.gov.it/portale/caldo/SISMG_sintesi_ULTIMO.pdf}{neusten
Daten} des italienischen Gesundheitsministeriums vom 14. März ist die
Gesamtsterblichkeit nun in allen Altersgruppen über 65 Jahren deutlich
erhöht, nachdem sie zuvor aufgrund des milden Winters noch
unterdurchschnittlich war. Die Gesamtsterblichkeit lag bis zum 14. März
zwar noch unter der Grippesaison von 2016/2017, könnte diese aber
inzwischen bereits übertroffen haben. Der Großteil dieser
Übersterblichkeit stammt derzeit aus Norditalien. Allerdings ist noch
nicht klar, welchen Anteil Covid19 daran hat, und welchen Anteil
Faktoren wie Panik, Systemkollaps und der Lockdown selbst haben könnten.

\includegraphics{https://swprs.files.wordpress.com/2020/03/italia-mortalita-marzo-14.png?w=600\&h=343}

\textbf{Frankeich}: In Frankreich liegt die Gesamsterblichkeit laut
\href{https://www.santepubliquefrance.fr/maladies-et-traumatismes/maladies-et-infections-respiratoires/infection-a-coronavirus/documents/bulletin-national/covid-19-point-epidemiologique-du-24-mars-2020}{den
neusten Daten} auf nationaler Ebene nach einer milden Grippesaison
weiterhin im Normalbereich. In einzelnen Departementen, insbesondere in
Nordostfankreich, hat die Gesamtsterblichkeit in den Altersgruppen über
65 Jahre im Zusammenhang mit Covid19 allerdings bereits stark zugenommen
(siehe Abbildung).

\includegraphics{https://swprs.files.wordpress.com/2020/03/france-mortality.png?w=650\&h=400}

Frankreich liefert zudem
\href{https://www.santepubliquefrance.fr/maladies-et-traumatismes/maladies-et-infections-respiratoires/infection-a-coronavirus/documents/bulletin-national/covid-19-point-epidemiologique-du-24-mars-2020}{detaillierte
Angaben} zu Altersverteilung und Vorerkrankungen der testpositiven
Intensivpatienten und Verstorbenen (siehe Abbildung unten):

\begin{itemize}
\tightlist
\item
  Das Durchschnittsalter der \textbf{Verstorbenen} liegt bei 81,2
  Jahren.
\item
  78\% der Verstorbenen waren über 75 Jahre alt; 93\% waren über 65
  Jahren alt.
\item
  2,4\% der Verstorbenen war unter 65 Jahre alt und hatte keine
  (bekannte) Vorerkrankung.
\item
  Das Durchschnittsalter der \textbf{Intensivpatienten} liegt bei 65
  Jahren.
\item
  26\% der Intensivpatienten sind über 75 Jahre alt; 67\% haben
  Vorerkrankungen.
\item
  17\% der Intensivpatienten sind unter 65 Jahre alt und haben keine
  Vorerkrankungen.
\end{itemize}

Die französischen Behörden ergänzen, dass ``der Anteil der (Covid-19)
Epidemie an der Gesamtmortalität noch zu bestimmen bleibt.''

\includegraphics{https://swprs.files.wordpress.com/2020/03/france-age-distribution-march-24.png?w=736\&h=349}

\textbf{USA}: Ein kanadischer Forscher hat die offiziellen Daten zu
\href{https://gis.cdc.gov/grasp/fluview/mortality.html}{Todesfällen
durch Lungen­ent­zündungen} in den USA ausgewertet. Diese liegen
typischerweise zwischen 3000 und 5500 Todesfälle \emph{pro Woche} und
damit deutlich über den aktuellen Zahlen zu Covid19. Die Todesfälle
insgesamt liegen in den USA bei 50,000 bis 60,000 pro Woche. (Hinweis:
In der Graphik unten sind die neusten Zahlen für März 2020 noch nicht
vollständig nachgeführt, deshalb sackt die Kurve ab.)

\includegraphics{https://swprs.files.wordpress.com/2020/03/us-pneumonia-deaths.png?w=400\&h=360}

\textbf{Großbritannien}:

\begin{itemize}
\tightlist
\item
  Neil Ferguson vom Imperial College London
  \href{https://www.newscientist.com/article/2238578-uk-has-enough-intensive-care-units-for-coronavirus-expert-predicts/}{geht
  inzwischen davon aus}, dass Großbritannien zur Behandlung von
  Covid19-Patienten über genügend Kapazitäten auf Intensivstationen
  verfügt.
\item
  Der emeritierte Professor für Pathologie, John Lee,
  \href{https://www.spectator.co.uk/article/The-evidence-on-Covid-19-is-not-as-clear-as-we-think}{argumentiert},
  dass die besondere Art der Registrierung von Covid-19 Fällen im
  Vergleich zu normalen Grippe- und Erklältungsfällen zu einer
  Überschätzung des Risikos durch Covid19 führe.
\end{itemize}

\textbf{Weiteres:}

\begin{itemize}
\tightlist
\item
  Eine
  \href{https://medium.com/@nigam/higher-co-infection-rates-in-covid19-b24965088333}{vorläufige
  Untersuchung} von Forschern der Universität Stanford zeigte, dass 20
  bis 25\% der Covid19-positiven Patienten zusätzlich positiv auf
  anderen Grippe- oder Erkältungsviren testeten.
\item
  Die Anzahl der Anträge an die Arbeitslosenversicherung schnellte in
  den USA auf einen Rekordwert
  \href{https://www.businessinsider.com/us-weekly-jobless-claims-record-coronavirus-unemployment-insurance-labor-recession-2020-3}{von
  über drei Millionen} hoch. In diesem Zusammenhang wird auch mit einer
  starken
  \href{https://twitter.com/KoenSwinkels/status/1243066532390977544}{Zunahme
  an Suiziden} gerechnet.
\item
  Der erste testpositive Patient in Deutschland ist inzwischen genesen.
  Der 33-jährige Mann hatte die Erkrankung laut eigenen Angaben
  \href{https://www.br.de/nachrichten/bayern/coronavirus-patient-nummer-1-wie-ich-die-quarantaene-erlebte,Rrm4Ul8}{``nicht
  so schlimm wie die Grippe''} erlebt.
\item
  Spanische Medien
  \href{https://elpais.com/sociedad/2020-03-25/los-test-rapidos-de-coronavirus-comprados-en-china-no-funcionan.html}{berichten},
  dass die Antikörper-Schnelltests für Covid19 nur eine Sensitivität von
  30\% aufweisen, obschon sie mindestens 80\% betragen sollte.
\item
  Eine
  \href{https://ehjournal.biomedcentral.com/articles/10.1186/1476-069X-2-15}{Untersuchung
  aus China} kam 2003 zum Schluss, dass die Wahrscheinlichkeit, an SARS
  zu sterben, bei Personen, welche einer moderaten Luftverschmutzung
  ausgesetzt sind, 84\% höher liegt als bei Patienten aus Regionen mit
  sauberer Luft. Gar ein 200\% höheres Risiko tragen Menschen aus
  Gebieten mit stark verschmutzter Luft.
\item
  Das Deutsche Netzwerk Evidenzbasierte Medizin (EbM)
  \href{https://www.ebm-netzwerk.de/de/veroeffentlichungen/covid-19}{kritisiert
  die Medienarbeit} zu Covid19: ``Die mediale Berichterstattung
  berücksichtigt in keiner Weise die von uns geforderten Kriterien einer
  evidenzbasierten Risikokommunikation. () Die Darstellung von Rohdaten
  ohne Bezug zu anderen Todesursachen führt zur Überschätzung des
  Risikos.''
\end{itemize}

\hypertarget{27-muxe4rz-2020-ii}{%
\paragraph{27. März 2020 (II)}\label{27-muxe4rz-2020-ii}}

\begin{itemize}
\tightlist
\item
  Der deutsche Forscher Dr. Richard Capek
  \href{https://coronadaten.wordpress.com/}{argumentiert in einer
  quantitativen Analyse}, dass die ``Corona-Epidemie'' in Wirklichkeit
  eine ``Epidemie der Tests'' sei. Capek zeigt, dass die Zahl der Tests
  exponentiell zugenommen hat, der Prozentsatz der Infizierten jedoch
  stabil geblieben und die Sterblichkeit zurückgegangen ist, was
  \emph{gegen} eine exponentielle Ausbreitung des Virus selbst spreche.
\item
  Virologie-Professor Dr. Carsten Scheller von der Universität Würzburg
  \href{https://www.youtube.com/watch?v=w-uub0urNfw}{erklärt in einem
  Podcast}, dass Covid19 durchaus mit der Influenza vergleichbar sei und
  bisher sogar zu weniger Todesfällen geführt habe. Professor Scheller
  vermutet, dass die in den Medien oft dargestellten Exponential­kurven
  eher mit der \emph{zunehmenden Anzahl an Tests} zu tun habe als mit
  einer ungewöhnlichen Ausbreitung des Virus selbst. Als Vorbild für
  Länder wie Deutschland diene weniger Italien als etwa Japan und
  Südkorea. Diese haben trotz Millionen chinesischer Touristen und nur
  minimaler gesellschaftlicher Einschränkungen bisher keine
  Covid19-Krise erlebt. Ein Grund dafür könne das Tragen von Mundmasken
  sein: Diese würde zwar kaum vor einer Infektion schützen, jedoch die
  Verbreitung des Virus durch erkrankte Personen einschränken.
\item
  Die
  \href{https://www.ecodibergamo.it/stories/bergamo-citta/a-bergamo-decessi-4-volte-oltre-la-medialeco-lancia-unindagine-nei-comuni_1346651_11/}{neusten
  Zahl aus Bergamo (Stadt)} zeigen, dass die Gesamtsterblichkeit im März
  2020 von typischerweise ca. 150 Personen pro Monat auf rund 450
  Personen zunahm. Dabei ist noch unklar, welchen Anteil daran Covid19
  hatte, und welchen Anteil andere Faktoren wie Massenpanik,
  Systemkollaps und Lockdown ausmachten. Offenbar wurde das städtische
  Krankenhaus von Personen aus der ganzen Region überrannt und
  kollabierte.
\item
  Die beiden Stanford-Medizinprofessoren, Dr. Eran Bendavid und Dr. Jay
  Bhattacharya, erklären in einem
  \href{https://web.archive.org/web/20200325103650/https://www.wsj.com/articles/is-the-coronavirus-as-deadly-as-they-say-11585088464}{Beitrag},
  dass die Tödlichkeit von Covid19 \emph{um mehrere Größenordnungen
  überschätzt} werde und vermutlich selbst in Italien nur bei 0,01\% bis
  0,06\% und damit unter jener der Influenza liege. Der Grund für die
  Überschätzung liege in der stark unterschätzten Anzahl der bereits
  (symptomlos) Infizierten. Als Beispiel wird etwa die vollständig
  ausgetestete italienische Gemeinde Vo genannt, die
  \href{https://www.repubblica.it/salute/medicina-e-ricerca/2020/03/16/news/coronavirus_studio_il_50-75_dei_casi_a_vo_sono_asintomatici_e_molto_contagiosi-251474302/}{50
  bis 75\% symptomlose testpositive Personen} ergab.
\item
  Dr. Gerald Gaß, der Präsident der deutschen Krankenhausgesellschaft,
  erklärte in einem
  \href{https://www.handelsblatt.com/politik/deutschland/coronakrise-deutsche-krankenhausgesellschaft-wir-sind-besser-vorbereitet-als-italien/25651268.html}{Interview
  mit dem Handelsblatt}, dass ``die extreme Situation in Italien vor
  allem an den sehr geringen Intensivkapazitäten'' liege.
\item
  Dr. Wolfgang Wodarg, einer der
  \href{https://www.youtube.com/watch?v=p_AyuhbnPOI}{frühen Kritiker}
  der Covid19-Darstellung, wurde vom Vorstand von \emph{Transparency
  Deutschland} vorläufig
  \href{https://www.transparency.de/aktuelles/detail/article/in-eigener-sache-vorstand-beschliesst-ruhen-der-mitgliedschaft-von-wolfgang-wodarg-1/}{ausgeschlossen},
  wo er die Arbeitsgruppe Gesundheit leitete. Wodarg wurde für seine
  Kritik bereits zuvor von Medien heftig
  \href{https://www.youtube.com/watch?v=xcirqmhBCvk}{angegriffen}.
\item
  Der NSA-Whistleblower Edward Snowden
  \href{https://www.futurezone.de/digital-life/article228779795/Gefaehrliche-weltweite-Entwicklung-Edward-Snowden-warnt-vor-Ueberwachung.html}{warnt},
  dass Regierungen die aktuelle Situation nutzen, um den
  Überwachungsstaat auszubauen und die Grundrechte einzuschränken. Die
  derzeit eingeführten Kontrollmaßnahmen würden nach der Krise nicht
  mehr abgebaut.
\end{itemize}

~

\href{https://swprs.org/a-swiss-doctor-on-covid-19/anzahl-infizierte-und-tests-2603/}{}

\includegraphics{https://swprs.files.wordpress.com/2020/03/anzahl-infizierte-und-tests-2603.jpg?w=356\&h=202}

Number of tests and test-positives (proportional)

\href{https://swprs.org/covid-19-hinweis-ii/infizierte-pro-test2603/}{}

\includegraphics{https://swprs.files.wordpress.com/2020/03/infizierte-pro-test2603.jpg?w=372\&h=202}

Test-positives per number of tests (constant)

Die exponentielle Zunahme an Tests findet eine \emph{proportionale}
Zunahme an Testpositiven, der Anteil bleibt \emph{konstant}, was
\emph{gegen} eine virale Epidemie spricht. (Dr. Richard Capek, US-Daten)

\hypertarget{26-muxe4rz-2020-i}{%
\paragraph{26. März 2020 (I)**}\label{26-muxe4rz-2020-i}}

**

\begin{itemize}
\tightlist
\item
  \textbf{USA}: Die \href{https://healthweather.us/}{neusten Daten aus
  den USA} vom 25. März zeigen im ganzen Land eine \emph{abnehmende
  Anzahl} an grippeähnlichen Erkankungen, deren Häufigkeit inzwischen
  sogar deutlich \emph{unter} dem mehrjährigen Durchschnitt liegt. Die
  Regierungsmaßnahmen können als Grund hierfür ausgeschlossen werden, da
  sie noch keine Woche in Kraft sind.
\end{itemize}

\href{https://swprs.org/covid-19-hinweis-ii/us-influenza-trend/}{}

\includegraphics{https://swprs.files.wordpress.com/2020/03/us-influenza-trend.png?w=404\&h=242}

US Influenza Trend (March 25, 2020)

\href{https://swprs.org/covid-19-hinweis-ii/us-illness-levels/}{}

\includegraphics{https://swprs.files.wordpress.com/2020/03/us-illness-levels.png?w=324\&h=242}

US Influenza Trend (March 25, 2020)

USA: Abnehmende grippeähnliche Erkrankungen (25. März 2020, KINSA)

\begin{itemize}
\tightlist
\item
  \textbf{Deutschland}: Der
  \href{https://influenza.rki.de/Wochenberichte/2019_2020/2020-12.pdf}{neuste
  Influenza-Bericht} des deutschen Robert-Koch-Instituts vom 24. März
  dokumentiert eine ``bundesweite sinkende Aktivität der akuten
  Atemwegs­erkran­kungen'': Die Anzahl der grippeähnlichen Erkrankungen
  und dadurch bedingter Kranken­haus­aufenthalte liege \emph{unter dem
  Wert der Vorjahre} und sei derzeit weiter \emph{rückläufig}. Das RKI
  weiter: ``Die Erhöhung der Zahl der Arztbesuche () lässt sich zurzeit
  weder durch in der Bevölkerung zirkulierende Influenzaviren noch durch
  SARS-CoV-2 erklären.''\\
\end{itemize}

\href{https://swprs.org/covid-19-hinweis-ii/rki-atemwegserkrankungen-20-2-2020/}{}

\includegraphics{https://swprs.files.wordpress.com/2020/03/rki-atemwegserkrankungen-20-2-2020.png?w=327\&h=202}

Deutschland: Atemwegserkrankungen 2019/2020 ggü. Vorjahren

\href{https://swprs.org/covid-19-hinweis-ii/rki-kliniken-belegung/}{}

\includegraphics{https://swprs.files.wordpress.com/2020/03/rki-kliniken-belegung.png?w=401\&h=202}

Deutschland: Krankenhausaufenthalte durch Atemwegserkrankungen nach
Altersgruppen

Deutschland: Abnehmende grippeähnliche Erkrankungen (20. März 2020, RKI)

\begin{itemize}
\tightlist
\item
  \textbf{Italien}: Der renommierte italienische Virologe Giulio Tarro
  \href{https://www.cybermednews.eu/index.php/it/health/70871-interview-to-the-virologist-giulio-tarro-the-death-rate-of-covid-19-is-less-than-1-as-confirmed-by-the-national-institute-of-allergy-and-infectious-diseases}{argumentiert},
  dass die Mortalität von Covid19 auch in Italien \emph{bei unter 1\%
  liege} und damit vergleichbar mit der Grippe sei. Die höheren Werte
  ergeben sich nur, weil nicht zwischen Todesfällen mit und durch
  Covid19 unterschieden werde, und weil die Anzahl der (symptomfreien)
  Infizierten stark unterschätzt werde.
\item
  \textbf{UK}: Die Autoren der britischen Imperial College Studie, die
  bis zu 500,000 Todesfälle vorhersagten, reduzieren ihre Prognosen
  erneut. Nachdem sie bereits
  \href{https://www.bbc.com/news/health-51979654}{einräumten}, dass ein
  Großteil der testpositiven Todesfälle Teil der normalen Sterblichkeit
  sind, erklären sie nun, dass die Spitze der Krankheitsfälle
  \href{https://www.thetimes.co.uk/article/nhs-now-likely-to-cope-with-coronavirus-says-key-scientist-rn5m6nggk}{bereits
  in zwei bis drei Wochen erreicht} sei.
\item
  \textbf{UK}: Der britische Guardian
  \href{https://www.theguardian.com/society/2019/feb/20/britons-urged-to-get-flu-vaccine-as-critical-cases-rise-above-2000}{berichtete
  im Februar 2019}, dass es bereits in der eigentlich schwachen
  Grippesaison 2018/2019 in Großbritannien zu über 2180 grippebedingter
  Einweisungen auf Intensivstationen kam.
\item
  \textbf{Schweiz}: In der Schweiz liegt die Übersterblichkeit durch
  Covid19 bisher offenbar immer noch bei null. Als neustes
  ``Todesopfer'' wird von den Medien eine 100 Jahre alte Frau
  \href{https://www.nau.ch/ort/basel/drei-weitere-covid-19-todesfalle-in-basel-stadt-65684099}{präsentiert}.
  Dennoch verschärft die Schweizer Regierung die restriktiven Maßnahmen
  weiter.
\end{itemize}

\hypertarget{26-muxe4rz-2020-ii}{%
\paragraph{26. März 2020 (II)}\label{26-muxe4rz-2020-ii}}

\begin{itemize}
\tightlist
\item
  \textbf{Schweden}: Schweden verfolgt bisher die liberalste Strategie
  im Umgang mit Covid19, die
  \href{https://www.zeit.de/politik/ausland/2020-03/coronavirus-schweden-stockholm-oeffentliches-leben/komplettansicht}{auf
  zwei Prinzipien} beruht: Risikogruppen werden geschützt und Personen
  mit Grippesymptomen bleiben zuhause. ``Wenn man diese beiden Regeln
  befolgt, braucht man keine weiteren Maßnahmen, deren Effekt sowieso
  nur sehr marginal ist'', erklärte Chefepidemiologe Anders Tegnell. Das
  gesellschaftliche und ökonomische Leben gehe normal weiter. Der große
  Ansturm auf die Krankenhäuser sei bisher ausgeblieben.
\item
  Die deutsche Straf- und Verfassungsrechtlerin Dr. Jessica Hamed
  \href{https://www.fr.de/politik/coronakrise-deutschland-sind-kontaktsperren-ausgangsbeschraenkungen-rechtswidrig-13611821.html}{argumentiert},
  dass Maßnahmen wie allgemeine Ausgangssperren und Kontaktverbote ein
  massiver und unverhältnismäßiger Eingriff in die Freiheitsgrundrechte
  und damit vermutlich ``allesamt rechtswidrig'' seien.
\item
  Der \href{https://www.euromomo.eu/index.html}{neuste europäische
  Monitoringbericht} zur Gesamtsterblichkeit vom 26. März zeigt
  weiterhin in allen Ländern und allen Altersgruppen normale oder
  unterdurschnittliche Werte, nun aber mit einer
  \href{https://www.euromomo.eu/outputs/zscore_country65.html}{Ausnahme}:
  Bei der Altersgruppe 65+ in Italien wird eine aktuell erhöhte
  Gesamtsterblichkeit prognostiziert (sog. delay-adjusted z-score), die
  allerdings noch unter den Werten der Grippewellen von 2016/2017 und
  2017/2018 liegt.
\end{itemize}

\hypertarget{25-muxe4rz-2020}{%
\paragraph{25. März 2020}\label{25-muxe4rz-2020}}

\begin{itemize}
\tightlist
\item
  Der deutsche Immunologe und Toxikologe Professor Stefan Hockertz
  \href{https://www.youtube.com/watch?v=7wfb-B0BWmo}{erklärt} in einem
  Interview, dass Covid19 nicht gefährlicher sei als die Influenza
  (Grippe), sondern nur viel genauer beobachtet werde. Gefährlicher als
  das Virus sei die Angst und Panik, die durch die Medien ausgelöst
  wurden, sowie die ``autoritäre Reaktion'' vieler Regierungen.
  Professor Hockertz betont zudem, dass viele der angeblichen
  ``Corona-Toten'' in Wirklichkeit an anderen Erkrankungen starben und
  zusätzlich positiv auf Coronaviren getestet wurden. Hockertz vermutet,
  dass bis zu zehnmal mehr Personen als berichtet Covid19 bereits
  hatten, davon indes kaum etwas merkten.
\item
  Der argentinische Virologe und Biochemiker Pablo Goldschmidt
  \href{https://www.clarin.com/buena-vida/coronavirus-panico-injustificado-dice-virologo-argentino-francia_0_yVcmJ4RM.html}{erklärt},
  dass Covid19 nicht gefährlicher sei als eine starke Erkältung oder die
  Grippe. Es sei sogar möglich, dass der Covid19-Erreger bereits in
  früheren Jahren zirkulierte, aber noch nicht entdeckt wurde, da man
  nicht nach ihm suchte. Dr. Goldschmidt spricht von einem ``globalen
  Terror'', der durch Medien und Politik erzeugt worden sei. Jedes Jahr
  würden weltweit drei Millionen Neugeborene und 50,000 Erwachsene
  allein in den USA an einer Lungenentzündung sterben.
\item
  Professor Martin Exner, Leiter des Instituts für Hygiene an der
  Universität Bonn,
  \href{https://www.youtube.com/watch?v=9mI9trSm3PY}{erklärt in einem
  Interview} mit dem Sender phoenix, warum das Gesundheitspersonal
  derzeit unter Druck steht, obschon es in Deutschland bisher kaum
  erhöhte Patientenzahlen gibt: Einerseits müssen positiv-getestete
  Ärzte und Pfleger in Quarantäne und sind oft kaum zu ersetzen,
  andererseits können Pfleger aus Nachbarländern, die einen wichtigen
  Teil der Versorgung übernehmen, derzeit aufgrund der Grenzschließungen
  nicht einreisen.
\item
  Professor Julian Nida-Rümelin, ehemaliger deutscher
  Kulturstaatsminister und Professor für Ethik,
  \href{https://www.zdf.de/nachrichten/zdf-morgenmagazin/julian-nida-ruemelin-zur-corona-krise-100.html}{weist
  daraufhin}, dass Covid19 für die gesunde Allgemeinbevölkerung kein
  Risiko darstelle und extreme Maßnahmen wie Ausgangssperren daher nicht
  zu rechtfertigen seien.
\item
  Stanford-Professor John Ioannidis
  \href{https://www.statnews.com/2020/03/17/a-fiasco-in-the-making-as-the-coronavirus-pandemic-takes-hold-we-are-making-decisions-without-reliable-data/}{zeigte}
  anhand der Daten des Kreuzfahrtschiffes \emph{Diamond Princess,} dass
  die alterskorrigierte Letalität von Covid19 bei 0.025\% bis 0.625\%
  liegt, das heißt im Bereich einer starken Erkältung oder einer Grippe.
  Eine
  \href{https://www.niid.go.jp/niid/en/2019-ncov-e/9407-covid-dp-fe-01.html}{japanische
  Studie} zeigt zudem, dass von allen positiv-getesteten Passagieren
  trotz des hohen Durch­schnitts­alters 48\% \emph{vollständig
  symptomfrei} blieben; selbst bei den 80-89 Jährigen blieben 48\%
  symptomfrei, bei den 70 bis 79 Jährigen waren es sogar 60\% die keine
  Symptome zeigten. Hier stellt sich somit erneut die Frage, ob nicht
  die \emph{Vorerkrankungen} als Faktor entscheidender sind als das
  Virus selbst. Der Fall Italiens zeigt, dass
  \href{https://www.bloomberg.com/news/articles/2020-03-18/99-of-those-who-died-from-virus-had-other-illness-italy-says}{99\%
  der testpositiven Verstorbenen} eine oder mehrere Vorerkrankungen
  hatten, und selbst bei diesen nannten nur
  \href{https://web.archive.org/web/20200324214448/https://www.telegraph.co.uk/global-health/science-and-disease/have-many-coronavirus-patients-died-italy/}{12\%
  der Totenscheine} Covid19 als kausalen Faktor.
\end{itemize}

\hypertarget{24-muxe4rz-2020}{%
\paragraph{24. März 2020}\label{24-muxe4rz-2020}}

\begin{itemize}
\tightlist
\item
  Der Präsident des deutschen Robert-Koch-Instituts
  \href{https://swprs.org/rki-relativiert-corona-todesfaelle/}{bestätigte},
  dass das RKI alle testpositiven Todesfälle, \emph{unabhängig von der
  wirklichen Todesursache,} als »Corona-Todesfälle« erfasse. Das
  Durch­schnitts­alter der Todesfälle liege bei 82 Jahren, die meisten
  mit Vorerkrankungen. Die Übersterblichkeit durch Covid19 dürfte somit
  auch in Deutschland nahe bei null liegen.
\item
  Die Betten in den Deutschschweizer Intensivstationen seien
  \href{https://www.aargauerzeitung.ch/aargau/kanton-aargau/erst-3-von-100-aargauer-betten-der-intensivstationen-sind-belegt-so-ruesten-sich-die-spitaeler-auf-die-epidemie-137332716}{»noch
  weitgehend leer«}.
\item
  Professor Karin Mölling, ehemalige Direktorin des Instituts für
  Medizinische Virologie an der Universität Zürich:
  \href{https://www.radioeins.de/programm/sendungen/die_profis/archivierte_sendungen/beitraege/corona-virus-kein-killervirus.html}{Kein
  Killervirus, Panikmache beenden}.
\item
  Großbritannien hat Covid19 von der Liste der gefährlichen
  Infektionskrankheiten
  \href{https://www.gov.uk/guidance/high-consequence-infectious-diseases-hcid\#status-of-covid-19}{entfernt},
  da die Mortalitätsrate »insgesamt tief« liege.
\end{itemize}

\hypertarget{23-muxe4rz-2020-i}{%
\paragraph{23. März 2020 (I)**}\label{23-muxe4rz-2020-i}}

**

\begin{itemize}
\tightlist
\item
  Eine neue französische Studie mit dem Titel
  \href{https://www.sciencedirect.com/science/article/abs/pii/S0924857920300972}{SARS-CoV-2:
  Angst versus Daten} kommt zum Ergebnis, dass ``das Problem durch
  SARS-CoV-2 vermutlich überschätzt wird'', da sich ``die Mortalität von
  SARS-CoV-2 nicht wesentlich unterscheidet von gewöhnlichen Coronaviren
  (Erkältungsviren), die in einem Krankenhaus in Frankreich untersucht
  wurden.''
\item
  Eine
  \href{https://www.ijidonline.com/article/S1201-9712(19)30328-5/fulltext}{italienische
  Studie vom August 2019} fand, dass es in Italien in den letzten Jahren
  zwischen 7000 und 25,000 Grippetote pro Jahr gegeben habe. Dieser Wert
  sei höher als in anderen europäischen Ländern aufgrund der älteren
  Bevölkerung Italiens, und er ist sehr viel höher als alles, was bisher
  mit Covid-19 in Verbindung gebracht wurde.
\item
  In einem
  \href{https://www.who.int/news-room/q-a-detail/q-a-similarities-and-differences-covid-19-and-influenza}{neuen
  Faktenblatt} schreibt die WHO, dass sich Covid-19 nach bisherigen
  Erkenntnissen \emph{langsamer} als die Influenza verbreite (um circa
  50\%), und dass die vorsymptomatische Übertragung von Covid-19
  wesentlich tiefer liege als bei der Influenza.
\item
  Ein italienischer Chefarzt berichtet von
  \href{https://www.scmp.com/news/china/society/article/3076334/coronavirus-strange-pneumonia-seen-lombardy-november-leading}{``seltsamen
  Fällen von Lungenentzündungen''} in der Lombardei bereits im November
  2019. Dies wirft erneut die Frage auf, ob dafür das neue Virus
  verantwortlich ist (das offiziell erst im Februar 2020 in Italien
  auftrat), oder andere Faktoren, wie etwa die
  \href{https://www.thelocal.it/20170131/our-lungs-are-breaking-smog-levels-way-above-safe-limits-in-northern-italy}{starke
  Luftverschmutzung} in Norditalien.
\item
  Der dänische Forscher Peter Gøtzsche, Gründer der renommierten
  Cochrane Collaboration, schreibt dass Corona eine
  \href{https://www.deadlymedicines.dk/corona-an-epidemic-of-mass-panic/}{``Epidemie
  der Panik''} sei und ``Logik eines der ersten Opfer'' war.
\end{itemize}

\includegraphics{https://swprs.files.wordpress.com/2020/03/italy-smog.png?w=550\&h=309}

\hypertarget{23-muxe4rz-2020-ii}{%
\paragraph{23. März 2020 (II)}\label{23-muxe4rz-2020-ii}}

\begin{itemize}
\tightlist
\item
  Laut dem früheren israelischen Gesundheitsminister, Professor Yoram
  Lass, ist das neue Coronavirus weniger gefährlich als die Grippe und
  die Ausgangssperren würden
  \href{https://en.globes.co.il/en/article-lockdown-lunacy-1001322696}{mehr
  Menschen töten als das Virus}. ``Die Zahlen begründen keine Panik'',
  so Lass. Es sei bekannt, dass ``Italien eine enorme Morbidität durch
  Atemwegserkrankungen habe, die mehr als drei Mal so hoch sei wie im
  restlichen Europa.''
\item
  Laut Pietro Vernazza, ein Schweizer Spezialist für
  Infektionskrankheiten, sind die verfügten Maßnahmen
  \href{https://www.tagblatt.ch/leben/ostschweizer-infektiologe-pietro-vernazza-die-zahlen-zu-den-jungen-corona-virus-erkrankten-sind-irrefuehrend-ld.1206440}{nicht
  wissenschaftlich begründet} und müssen neu überdacht werden. Laut
  Vernazza machen Massentests keinen Sinn, da bis zu 90\% der
  Bevölkerung symptomlos bleiben werde, während Ausgangssperren und
  Schulschließungen sogar ``kontraproduktiv'' seien. Vernazza empfiehlt,
  nur die Risikogruppen zu schützen und Einschränkungen rückgängig zu
  machen.
\item
  Der Präsident der Internationalen Ärzte-Gesellschaft, Frank Ulrich
  Montgomery, hält Ausgangssperren wie in Italien ebenfalls für
  \href{https://www.general-anzeiger-bonn.de/news/politik/deutschland/interview-mit-weltaerztepraesident-montgomery-ueber-corona-pandemie-ist-chaos_aid-49609561}{``unvernünftig''
  und ``kontraproduktiv''}.
\item
  Schweiz: Trotz medialer Aufregung liegt die Übersterblichkeit
  weiterhin bei oder nahe bei null: Die letzten beiden testpositiven
  \href{https://www.bluewin.ch/de/newsregional/zuerich/1068-bestatigte-corona-falle-und-funf-todesfalle-im-kanton-zurich-371873.html}{``Todesopfer''}
  waren ein 96jähriger in Palliativ­behandlung und ein 97jähriger mit
  mehreren Vorerkrankungen.
\item
  Der neuste statistische Bericht des ISS zu Italien ist nun auch
  \href{https://www.epicentro.iss.it/coronavirus/bollettino/Report-COVID-2019_20_marzo_eng.pdf}{auf
  Englisch verfügbar}.
\end{itemize}

\hypertarget{22-muxe4rz-2020-i}{%
\paragraph{22. März 2020 (I)**}\label{22-muxe4rz-2020-i}}

**

\textbf{Bezüglich der Situation in Italien}: Die meisten Medien
berichten inkorrekt, dass Italien bis zu 800 Todesfälle pro Tag
\emph{durch das Coronavirus} habe. In Wirklichkeit betont der Präsident
des italienischen Zivilschutzes, dass es sich um Todesfälle ``\emph{mit
dem} Coronavirus und \emph{nicht durch das} Coronavirus'' handelt
(Minute 03:30 der
\href{https://youtu.be/0M4kbPDHGR0?t=210}{Pressekonferenz}). Mit anderen
Worten, diese Personen starben, während sie zusätzlich positiv getestet
wurden.

Wie die beiden Professoren Ioannidis und Bhakdi
\href{https://www.statnews.com/2020/03/17/a-fiasco-in-the-making-as-the-coronavirus-pandemic-takes-hold-we-are-making-decisions-without-reliable-data/}{aufzeigten},
haben Länder wie Südkorea und Japan, \emph{die keine Sperrmaßnahmen
eingeführt haben}, im Zusammenhang mit Covid-19 eine Übersterblichkeit
von fast null erlebt, während das Kreuzfahrtschiff Diamond Princess eine
hochgerechnete Sterblichkeitszahl \emph{im Promillebereich} hatte, d.h.
auf oder unter dem Niveau der saisonalen Grippe oder einer starken
Erkältung.

Die aktuellen testpositiven Sterbezahlen in Italien liegen immer noch
unter 50\% der normalen täglichen Gesamtsterblichkeit in Italien, die
bei etwa 1800 Todesfällen pro Tag liegt. Daher ist es möglich,
vielleicht sogar wahrscheinlich, dass ein großer Teil der
\emph{normalen} täglichen Mortalität jetzt einfach als
``Covid19''-Todesfälle gezählt wird (da sie positiv getestet werden).
Dies ist der Punkt, den der Präsident des italienischen Zivilschutzes
betont hat.

Inzwischen ist jedoch bekannt, dass bestimmte Regionen in Norditalien,
d.h. diejenigen, die mit den
\href{https://en.wikipedia.org/wiki/2020_Italy_coronavirus_lockdown}{härtesten
Sperrmaßnahmen} konfrontiert sind, deutlich erhöhte
Tages­sterb­lichkeits­zahlen aufweisen. Es ist auch bekannt, dass in der
Region Lombardei 90\% der testpositiven Todesfälle \emph{nicht} auf der
Intensivstation, sondern
\href{https://www.tgcom24.mediaset.it/cronaca/coronavirus-in-lombardia-9-morti-su-10-mai-giunti-in-terapia-intensiva_16362350-202002a.shtml}{meist
zu Hause} auftreten. Und mehr als 99\% dieser Todesfälle haben schwere
gesundheitliche Vorerkrankungen (z.B. Herzprobleme, Atemprobleme,
Krebs).

Professor Sucharit Bhakdi hat die Sperrmaßnahmen als ``nutzlos'',
``selbstzerstörerisch'' und als ``kollektiven Selbstmord''
\href{https://www.youtube.com/watch?v=JBB9bA-gXL4}{bezeichnet}. Daher
stellt sich die äußerst beunruhigende Frage, inwieweit die erhöhte
Sterblichkeit dieser älteren, isolierten, stark gestressten Menschen mit
mehreren Vorerkrankungen womöglich durch die noch immer geltenden
wochenlangen Sperrmaßnahmen verursacht worden sein könnte.

Es wäre dann einer jener Fälle, in denen die Behandlung schlimmer als
die Krankheit ist. (Siehe Update unten: Nur 12\% der Todeszertifikate
geben das Coronavirus als eine Ursache an.)

\includegraphics{https://swprs.files.wordpress.com/2020/03/borrelli2.jpg?w=550\&h=309}

\hypertarget{22-muxe4rz-2020-ii}{%
\paragraph{22. März 2020 (II)}\label{22-muxe4rz-2020-ii}}

\begin{itemize}
\tightlist
\item
  In der Schweiz gebe es bisher 56 test-positive Todesfälle. Bei allen
  habe es sich aufgrund von Alter und/oder Vorerkrankungen um
  ``Risikopatienten''
  \href{https://www.nzz.ch/schweiz/coronavirus-in-der-schweiz-die-neusten-entwicklungen-ld.1542664\#subtitle-wie-viele-infizierte-und-todesf-lle-gibt-es-second}{gehandelt}.
  Zur genauen Todesursache, das heißt ob am oder nur mit dem Virus, gibt
  es weiterhin keine Angaben.
\item
  Die Schweizer Regierung behauptete, die Situation in der Südschweiz
  (direkt neben Italien) sei ``dramatisch'', aber lokale Ärzte
  \href{https://www.nzz.ch/schweiz/punkto-intensivbetten-sind-wir-im-tessin-besser-ausgeruestet-als-der-rest-der-schweiz-ld.1547728}{widersprechen}:
  die Situation sei keineswegs dramatisch.
\item
  Laut
  \href{https://www.blick.ch/news/schweiz/nicht-nur-beatmungsgeraete-werden-knapp-im-kampf-gegen-corona-es-droht-ein-engpass-beim-sauerstoff-id15808185.html}{Presseberichten}drohe
  ein Engpass bei den Sauerstoffflaschen. Der Grund sei aber nicht ein
  derzeit erhöhter Bedarf, sondern die Hortung aus Angst vor Knappheit.
\item
  In vielen Ländern besteht bereits ein
  \href{https://www.washingtonpost.com/health/covid-19-hits-doctors-nurses-emts-threatening-health-system/2020/03/17/f21147e8-67aa-11ea-b313-df458622c2cc_story.html}{zunehmender
  Mangel} an Ärzten und Pflegern. Der Hauptgrund dafür ist, dass sich
  positiv-getestete Fachkräfte in Quarantäne begeben müssen, obschon sie
  in den meisten Fällen keine oder nur leichte Symptome entwickeln.
\end{itemize}

\hypertarget{22-muxe4rz-2020-iii}{%
\paragraph{22. März 2020 (III)}\label{22-muxe4rz-2020-iii}}

\begin{itemize}
\tightlist
\item
  Ein Modell des Imperial College London prognostizierte für
  Großbritannien 250,000 bis 500,000 Todesfälle ``durch'' Covid-19. Die
  Autoren der Studie haben nun aber
  \href{https://www.bbc.com/news/health-51979654}{eingeräumt}, dass
  viele dieser Todesfälle nicht zusätzlich anfallen, sondern Teil der
  normalen jährlichen Sterblichkeit sind, die in Groß­britannien bei
  600,000 Personen pro Jahr liegt.
\item
  Dr. David Katz, der Gründungsdirektor des Yale University Prevention
  Research Center, fragt in der
  \href{https://www.nytimes.com/2020/03/20/opinion/coronavirus-pandemic-social-distancing.html}{New
  York Times}: ``Ist unser Kampf gegen den Coronavirus schlimmer als die
  Krankheit? Es gibt gezieltere Mittel, die Pandemie zu besiegen.''
\item
  Laut dem italienischen Professor Walter Ricciardi
  \href{https://web.archive.org/web/20200324214448/https://www.telegraph.co.uk/global-health/science-and-disease/have-many-coronavirus-patients-died-italy/}{geben}\textbf{``nur
  12\% der Todes­zertifikate das Coronavirus als einen Grund an''},
  während in öffentlichen Berichten ``alle Todesfälle, die im
  Krankenhaus mit dem Coronavirus sterben, als Todesfälle durch das
  Coronavirus gezählt werden.'' Somit müssen die in den Medien genannten
  italienischen Todeszahlen \emph{um mindestens einen Faktor acht}
  reduziert werden um die tatsächlich durch das Coronavirus verursachten
  Todesfälle zu erhalten. Dies ergibt höchstens einige dutzend
  Todesfälle pro Tag, verglichen mit einer normalen Gesamtsterblichkeit
  von 1800 pro Tag und bis zu 20,000 Grippetoten pro Jahr.
\end{itemize}

\hypertarget{21-muxe4rz-2020-i}{%
\paragraph{21. März 2020 (I)**}\label{21-muxe4rz-2020-i}}

**

\begin{itemize}
\tightlist
\item
  Spanien meldet bisher nur drei testpositive Todesfälle
  \href{https://www.20minutos.es/noticia/4193883/0/media-edad-coronavirus-espana/}{unter
  65 Jahren} (von total ca. 1000). Deren Vorerkrankungen und
  tatsächliche Todesursache sind bisher nicht bekannt.
\item
  Italien
  \href{https://www.msn.com/en-au/news/coronavirus/italy-coronavirus-deaths-surge-by-627-in-a-day-lifting-total-death-toll-to-4032/ar-BB11tDnS}{meldete}
  am 20. März landesweit 627 testpositive Todesfälle an einem Tag. Die
  normale Sterblichkeit liegt in Italien bei ca. 1800 Todesfällen pro
  Tag. Seit dem 21. Februar meldete Italien insgesamt ca. 4000
  testpositive Todesfälle. Im selben Zeitraum hatte Italien eine
  natürliche Gesamtmortalität von ca. 50.000 Todesfällen. Es ist noch
  nicht klar, um wie viel die Gesamt­mortalität zunahm oder aber einfach
  testpositiv wurde. Italien und Europa hatten zudem eine sehr milde
  Grippesaison 2019/2020, die viele ansonsten gefährdete Personen
  verschonte.
\item
  Laut
  \href{https://www.tgcom24.mediaset.it/cronaca/coronavirus-in-lombardia-9-morti-su-10-mai-giunti-in-terapia-intensiva_16362350-202002a.shtml}{italienischen
  Medienberichten} verstarben in der Region Lombardei bisher ca. 90\%
  der testpositiven Todesfälle nicht auf der Intensivstation, sondern
  größtenteils zuhause oder in der Allgemeinabteilung. Die
  Todes­ursachen und die mögliche Rolle der Quarantäne­maßnahmen sind
  noch nicht klar. Nur 260 von 2168 testpositiven Todesfällen seien auf
  Intensivstationen erfolgt.
\item
  Bloomberg
  \href{https://www.bloomberg.com/news/articles/2020-03-18/99-of-those-who-died-from-virus-had-other-illness-italy-says}{berichtet},
  dass 99\% der italienischen Todesfälle andere Erkrankungen hatten.
\end{itemize}

\includegraphics{https://swprs.files.wordpress.com/2020/03/covid-iss-stat-bloomberg.png?w=550\&h=301}

\hypertarget{21-muxe4rz-2020-ii}{%
\paragraph{21. März 2020 (II)}\label{21-muxe4rz-2020-ii}}

\begin{itemize}
\tightlist
\item
  Die Japan Times fragt:
  \href{https://www.japantimes.co.jp/news/2020/03/20/national/coronavirus-explosion-expected-japan/}{Japan
  erwartete eine Coronaviren-Explosion. Wo bleibt sie?} Obschon Japan
  als eines der ersten Länder positive Testresultate hatte und keinen
  ``Lockdown'' einführte, ist es bisher eines der am wenigsten
  betroffenen Länder. Es gebe keine Zunahme an Lungen­ent­zündungen und
  keine erhöhte Krankenhausbelegung.
\item
  Italienischen Forscher
  \href{https://www.heise.de/tp/features/Feinstaubpartikel-als-Viren-Vehikel-4687454.html}{argumentieren},
  dass die extreme Luftverschmutzung in Norditalien -- die stärkste in
  ganz Europa -- eine ursächliche Rolle bei der aktuellen lokalen
  Zunahme an Lungenentzündungen spielen könnte, ähnlich wie zuvor im
  chinesischen Wuhan (s.o.)
\item
  In einem \href{https://www.youtube.com/watch?v=JBB9bA-gXL4}{neuen
  Interview} erklärt Professor Sucharit Bhakdi, einer der meistzitierten
  Experten auf dem Gebiet der medizinischen Mikrobiologie, dass es
  ``falsch'' und ``gefährlich irreführend'' sei, das neue Coronavirus
  für die Todesfälle haupt­ver­ant­wortlich zu machen, da
  Vorerkrankungen und die Luftverschmutzung in chinesischen und
  nord­ita­lie­nischen Städten eine wichtigere Rolle spielten. Die
  derzeit diskutierten oder beschlossenen Maßnahmen bezeichnet Professor
  Bhakdi als ``grotesk'', ``sinnlos'', ``selbstzer­stö­rerisch'' und
  ``kollektiven Selbstmord'', der die Lebens­erwartung der Senioren
  verkürzen werde und von der Gesellschaft nicht akzeptiert werden
  dürfe.
\end{itemize}

\hypertarget{20-muxe4rz-2020}{%
\paragraph{20. März 2020**}\label{20-muxe4rz-2020}}

**

\begin{itemize}
\tightlist
\item
  Laut dem \href{https://www.euromomo.eu/index.html}{neusten
  europäischen Monitoringbericht} liegt die Gesamtmortalität in allen
  Ländern (inkl. Italien) und in allen Altersgruppen bisher im
  Normalbereich oder darunter.
\item
  Laut den
  \href{https://de.wikipedia.org/wiki/COVID-19-Pandemie_in_Deutschland\#Todesf\%C3\%A4lle_in_den_Medien}{neusten
  Zahlen} aus Deutschland liegt das Median-Alter der testpositiven
  Todesfälle bei circa 83 Jahren, die meisten davon mit chronischen
  Vorerkrankungen.
\item
  Eine von Stanford-Professor John Ioannidis angeführte
  \href{https://www.ncbi.nlm.nih.gov/pmc/articles/PMC2095096/}{kanadische
  Studie von 2006} zeigt am Fall eines Pflegeheims, dass auch
  gewöhnliche Coronaviren (Erkältungsviren) in Risikogruppen eine
  Sterblichkeit von bis zu 6\% hervorrufen können, und dass
  Virentestkits zunächst fälschlicherweise eine Infektion mit dem
  SARS-Coronavirus angaben.
\end{itemize}

\hypertarget{19-muxe4rz-2020-i}{%
\paragraph{19. März 2020 (I)**}\label{19-muxe4rz-2020-i}}

**

Das italienische ISS hat einen
\href{https://www.epicentro.iss.it/coronavirus/bollettino/Report-COVID-2019_17_marzo-v2.pdf}{neuen
Bericht} zu den testpositiven Verstorbenen publiziert:

\begin{itemize}
\tightlist
\item
  Das Medianalter liegt bei 80.5 Jahren (79.5 bei den Männern, 83.7 bei
  den Frauen).
\item
  10\% der Verstorbenen waren über 90 Jahre alt; 90\% waren über 70
  Jahre alt.
\item
  Höchstens 0.8\% der Verstorbenen hatte keine chronischen
  Vorerkrankungen.
\item
  Ca. 75\% der Verstorbenen hatten zwei oder mehr Vorerkrankungen, ca.
  50\% hatten drei oder mehr Vorerkrankungen, darunter insb.
  Herzkrankheiten, Diabetes und Krebs.
\item
  Fünf Verstorbene waren 31 bis 39 Jahre alt, alle mit schweren
  Vorerkrankungen.
\item
  Das Gesundheitsinstitut lässt weiterhin offen, woran die untersuchten
  Patienten starben, und spricht allgemein von »Covid19-positiven
  Verstorbenen«.
\end{itemize}

\hypertarget{19-muxe4rz-2020-ii}{%
\paragraph{19. März 2020 (II)**}\label{19-muxe4rz-2020-ii}}

**

\begin{itemize}
\tightlist
\item
  Ein
  \href{https://milano.corriere.it/notizie/cronaca/18_gennaio_10/milano-terapie-intensive-collasso-l-influenza-gia-48-malati-gravi-molte-operazioni-rinviate-c9dc43a6-f5d1-11e7-9b06-fe054c3be5b2.shtml}{Bericht
  der italienischen Zeitung \emph{Corriere della Sera}} beschreibt, dass
  die italienischen Intensivstationen bereits unter der markanten
  Grippewelle von 2017/2018 kollabierten, Operationen verschieben sowie
  Krankenpfleger aus dem Urlaub zurückrufen mussten.
\item
  Der deutsche Virologe Hendrik Streeck
  \href{https://www.faz.net/aktuell/gesellschaft/gesundheit/coronavirus/virologe-hendrik-streeck-ueber-corona-neue-symptome-entdeckt-16681450.html?printPagedArticle=true\#pageIndex_2}{vermutet
  in einem Interview}, dass Covid19 die Gesamtsterblichkeit in
  Deutschland nicht erhöhen werde, die normalerweise bei rund 2500
  Personen \emph{pro Tag} liege. Streeck erwähnt den Fall eines 78 Jahre
  alten Mannes mit Vorerkrankungen, der an einem Herzversagen starb,
  nachträglich positiv auf Covid19 getestet und deshalb in die Statistik
  der Covid19-Todesfälle aufgenommen wurde.
\item
  Laut Stanford-Professor John P.A. Ioannidis gebe es für die derzeit
  beschlossenen Maßnahmen
  \href{https://www.statnews.com/2020/03/17/a-fiasco-in-the-making-as-the-coronavirus-pandemic-takes-hold-we-are-making-decisions-without-reliable-data/}{keine
  ausreichende medizinische Datengrundlage}. Das neue Coronavirus sei
  womöglich selbst bei älteren Personen nicht gefährlicher als einige
  der üblichen Coronaviren.
\end{itemize}

\hypertarget{18-muxe4rz-2020}{%
\paragraph{18. März 2020}\label{18-muxe4rz-2020}}

\begin{itemize}
\tightlist
\item
  Eine
  \href{https://www.medrxiv.org/content/10.1101/2020.02.12.20022434v2}{neue
  epidemiologische Studie} (Vorabdruck) kommt zum Ergebnis, dass die
  Fatalität von Covid19 selbst in der chinesischen Stadt Wuhan bei nur
  0.04\% bis 0.12\% gelegen habe und somit eher geringer sei als bei der
  saisonalen Grippe, deren Fatalität bei ca. 0.1\% liegt. Als Grund für
  die offenbar stark überschätzte Fatalität von Covid19 vermuten die
  Forscher, dass in Wuhan ursprünglich nur ein kleiner Teil der Fälle
  erfasst worden sei, da die Krankheit bei vielen Personen wohl
  symptomlos oder mild verlief.
\item
  Chinesiche Forscher
  \href{https://www.eurasiareview.com/01022020-polluted-air-could-be-an-important-cause-of-wuhan-pneumonia-oped/}{argumentieren},
  dass extremer Wintersmog in der Stadt Wuhan eine ursächliche Rolle
  beim Ausbruch der Lungenentzündungen gespielt haben könnte. Im Sommer
  2019 kam es in Wuhan bereits zu
  \href{https://www.cnn.com/2019/07/10/asia/china-wuhan-pollution-problems-intl-hnk/index.html}{öffentlichen
  Protesten} wegen der schlechten Luftqualität.
\item
  Neue Satellitenbilder zeigen, wie Norditalien die europaweit
  \href{https://twitter.com/esa/status/1238480433047916545}{stärkste
  Luftverschmutzung} aufweist, und wie diese Luftverschmutzung durch die
  Quarantäne stark zurückging.
\item
  Ein Hersteller des Covid19-Testkits gibt an, dass dieses
  \href{https://www.creative-diagnostics.com/sars-cov-2-coronavirus-multiplex-rt-qpcr-kit-277854-457.htm}{nur
  für Forschungszwecke} und nicht für diagnostische Anwendungen
  einzusetzen ist, da noch nicht klinisch validiert.
\end{itemize}

\includegraphics{https://swprs.files.wordpress.com/2020/03/covid-testkit.png?w=550\&h=149}

\hypertarget{17-muxe4rz-2020-i}{%
\paragraph{17. März 2020 (I)**}\label{17-muxe4rz-2020-i}}

**

\begin{itemize}
\tightlist
\item
  Einige Schweizer Notfallstationen seien bereits überlastet allein
  aufgrund der hohen Anzahl an Personen, die sich
  \href{https://insideparadeplatz.ch/2020/03/16/notfall-stationen-bereits-seit-tagen-am-anschlag/}{testen
  lassen möchten}. Dies deutet auf eine zusätzliche psychologische und
  logistische Komponente der aktuellen Situation hin.
\item
  Das Sterblichkeitsprofil ist aus virologischer Sicht weiterhin
  rätselhaft, da im Unterschied zu Grippeviren Kinder verschont und
  ältere Männer etwa doppelt so häufig betroffen sind wie ältere Frauen.
  Dieses Profil entspricht andererseits der
  \href{http://www.gbe-bund.de/gbe10/abrechnung.prc_abr_test_logon?p_uid=gast\&p_aid=0\&p_knoten=FID\&p_sprache=D\&p_suchstring=820}{natürlichen
  Sterblichkeit}, die bei Kindern nahe null und bei 75-jährigen Männern
  fast doppelt so hoch wie bei gleichaltrigen Frauen liegt.
\item
  Bei den jungen testpositiven Verstorbenen handelt es sich weiterhin
  größtenteils oder sogar ausschließlich um Personen mit schwersten
  Vorerkrankungen. So sei ein 21-jähriger spanischer Fußballtrainer
  testpositiv
  \href{https://www.msn.com/de-ch/news/other/spanischer-nachwuchs-trainer-stirbt-an-corona/ar-BB11gT64}{verstorben}.
  Die Ärzte stellten indes eine unerkannte Leukämie fest, zu deren
  typischen Komplikationen eine schwere Lungen­ent­zündung gehört.
\item
  Entscheidend zur Beurteilung der Gefährlichkeit der Krankheit ist
  daher \emph{nicht} die in den Medien oft genannte Anzahl der
  testpositiven Personen und Verstorbenen, sondern die Anzahl der
  tatsächlich und unerwartet \emph{an einer Lungenentzündung} Erkrankten
  oder Verstorbenen (sog. Übersterblichkeit). Dieser Wert liegt in den
  meisten Ländern bisher \href{https://www.euromomo.eu/index.html}{sehr
  tief}.
\end{itemize}

\hypertarget{17-muxe4rz-2020-ii}{%
\paragraph{17. März 2020 (II)}\label{17-muxe4rz-2020-ii}}

\begin{itemize}
\tightlist
\item
  Der italienische Immunologie-Professor Sergio Romagnani der
  Universität Florenz kommt in einer Studie an 3000 Personen zum
  Ergebnis, dass 50 bis 75\% der testpositiven Personen aller
  Altersgruppen
  \href{https://www.repubblica.it/salute/medicina-e-ricerca/2020/03/16/news/coronavirus_studio_il_50-75_dei_casi_a_vo_sono_asintomatici_e_molto_contagiosi-251474302/}{vollständig
  symptomfrei bleiben} -- deutlich mehr als bisher angenommen.
\item
  Die Belegung der norditalienischen Intensivstationen beträgt in den
  Wintermonaten typischerweise bereits
  \href{https://jamanetwork.com/journals/jama/fullarticle/2763188}{85
  bis 90\%}. Einige oder viele dieser bestehenden Patienten könnten
  inzwischen auch testpositiv sein. Zur Anzahl der zusätzlichen
  unerwarteten Lungenentzündungen gibt es indes noch keine offiziellen
  Angaben.
\item
  Eine Krankenhausärztin in der spanischen Stadt Málaga
  \href{https://twitter.com/NeurologaenSAS/status/1239498772570308609}{schreibt
  auf Twitter}, dass die Menschen derzeit eher an der Panik und am
  Systemkollaps sterben als am Virus. Das Krankenhaus werde von Personen
  mit Erkältungen, Grippe und womöglich Covid19 überrannt und die
  Abläufe seien zusammengebrochen.
\end{itemize}

\hypertarget{14-muxe4rz-2020}{%
\paragraph{14. März 2020}\label{14-muxe4rz-2020}}

Laut den
\href{https://www.epicentro.iss.it/coronavirus/sars-cov-2-decessi-italia}{Angaben}
des italienischen Nationalen Gesundheitsinstituts ISS liegt das
Durch­schnitts­­alter der positiv-getesteten Verstorbenen in Italien
derzeit bei circa 81 Jahren. 10\% der Verstorbenen sind über 90 Jahre
alt. 90\% der Verstorbenen sind über 70 Jahre alt.

80\% der Verstorbenen hatten zwei oder mehr chronische Vorerkrankungen.
50\% der Verstorbenen hatten drei oder mehr chronische Vorerkrankungen.
Zu den chronischen Vorerkrankungen zählen insbesondere
Herz-Kreislauf-Probleme, Diabetes, Atemprobleme und Krebs.

Bei weniger als 1\% der Verstorbenen handelte es sich um gesunde
Personen, das heißt um Personen ohne chronische Vorerkrankungen. Nur
circa 30\% der Verstorbenen sind Frauen.

Das italienische Gesundheitsinstitut
\href{https://youtu.be/0M4kbPDHGR0?t=210}{unterscheidet} zudem zwischen
Verstorbenen \emph{durch das} und Verstorbenen \emph{mit dem}
Coronavirus. In vielen Fällen sei noch nicht klar, ob die Personen am
Virus starben oder an ihren chronischen Vorerkrankungen oder an einer
Kombination davon.

Bei den zwei italienischen Verstorbenen unter 40 Jahren (beide 39 Jahre
alt) habe es sich um einen Krebspatienten sowie um einen
Diabetes-Patienten mit weiteren Komplikationen gehandelt. Auch hier sei
die genaue Todesursache noch nicht klar (d.h. ob am Virus oder an den
Vorerkrankungen).

Die Überlastung der Kliniken ergebe sich durch den allgemeinen Andrang
an Patienten sowie durch die erhöhte Anzahl an Patienten, die besondere
oder intensive Betreuung benötigen. Dabei gehe es insb. um die
Stabilisierung der Atemfunktion sowie in schweren Fällen um anti-virale
Therapien.

\textbf{Außerdem sind folgende Aspekte zu beachten:}

Norditalien hat eine der ältesten Bevölkerungen sowie die
\href{https://twitter.com/esa/status/1238480433047916545}{schlechteste
Luftqualität} Europas, was bereits in der Vergangenheit zu einer
\href{https://www.srf.ch/news/international/massive-schadstoffbelastung-nirgendwo-erkranken-so-viele-wegen-smog-wie-in-norditalien}{erhöhten
Anzahl} an Atemwegserkrankungen und dadurch bedingter Todesfälle geführt
hat. Dies ist als zusätzlicher Risikofaktor zu sehen.

Südkorea beispielsweise hat einen deutlich milderen Verlauf als Italien
erlebt und den Höhepunkt der Epidemie bereits überschritten. In Südkorea
kam es bisher zu lediglich ca. 70 Todesfällen mit positivem Test.
Betroffen waren wie in Italien hauptsächlich Risikopatienten.

Bei den bisherigen testpositiven Schweizer Todesfällen hat es sich
ebenfalls um Risiko­patienten mit Vorerkrankungen und einem Medianalter
von über 80 Jahren gehandelt, deren genaue Todesursache, d.h. ob am
Virus oder an den Vorerkrankungen, noch nicht bekannt ist.

Ferner zeigen
\href{https://www.ncbi.nlm.nih.gov/pmc/articles/PMC2095096/}{Studien},
dass die weltweit verwendeten Viren-Testkits in einigen Fällen ein
falsches positives Resultat ergeben können, d.h. die Personen wären in
diesen Fällen \emph{nicht} am neuen Coronavirus erkrankt, sondern
womöglich an einem der bisherigen Coronaviren, die Teil der jährlichen
(und aktuellen) Erkältungs- und Grippewelle sind.

Zur Beurteilung der Gefährlichkeit der Krankheit ist daher \emph{nicht}
die oft genannte Anzahl der testpositiven Personen und Verstorbenen
entscheidend, sondern die Anzahl der tatsächlich und unerwartet \emph{an
einer Lungenentzündung} Erkrankten oder Verstorbenen (sog.
\href{https://www.euromomo.eu/index.html}{Übersterblichkeit}).

Für die gesunde Allgemeinbevölkerung im Schul- und Arbeitsalter ist nach
allen bisherigen Erkenntnissen bei Covid-19 mit einem milden bis
moderaten Verlauf zu rechnen. Senioren und Personen mit bestehenden
chronischen Erkrankungen sind besonders zu schützen. Die medizinischen
Kapazitäten sind optimal vorzubereiten.

\hypertarget{zur-hauptseite-fakten-zu-covid-19-1}{%
\paragraph{\texorpdfstring{\href{https://swprs.org/covid-19-hinweis-ii/}{Zur
Hauptseite: Fakten zu
Covid-19}}{Zur Hauptseite: Fakten zu Covid-19}}\label{zur-hauptseite-fakten-zu-covid-19-1}}

\begin{center}\rule{0.5\linewidth}{\linethickness}\end{center}

\hypertarget{swiss-policy-research}{%
\subsubsection{Swiss Policy Research}\label{swiss-policy-research}}

\begin{itemize}
\tightlist
\item
  \href{https://swprs.org/kontakt/}{Kontakt}
\item
  \href{https://swprs.org/uebersicht/}{Übersicht}
\item
  \href{https://swprs.org/donationen/}{Donationen}
\item
  \href{https://swprs.org/disclaimer/}{Disclaimer}
\end{itemize}

\hypertarget{english}{%
\subsubsection{English}\label{english}}

\begin{itemize}
\tightlist
\item
  \href{https://swprs.org/contact/}{About Us / Contact}
\item
  \href{https://swprs.org/media-navigator/}{The Media Navigator}
\item
  \href{https://swprs.org/the-american-empire-and-its-media/}{The CFR
  and the Media}
\item
  \href{https://swprs.org/donations/}{Donations}
\end{itemize}

\hypertarget{follow-by-email}{%
\subsubsection{Follow by email}\label{follow-by-email}}

Follow

\href{https://wordpress.com/?ref=footer_custom_com}{WordPress.com}.

\protect\hyperlink{}{Up ↑}

Post to

\protect\hyperlink{}{Cancel}

\includegraphics{https://pixel.wp.com/b.gif?v=noscript}
