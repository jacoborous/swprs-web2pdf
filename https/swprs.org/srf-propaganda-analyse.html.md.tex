\protect\hyperlink{content}{Skip to content}

\href{https://swprs.org/}{}

\protect\hyperlink{search-container}{Search}

Search for:

\href{https://swprs.org/}{\includegraphics{https://swprs.files.wordpress.com/2020/05/swiss-policy-research-logo-300.png}}

\href{https://swprs.org/}{Swiss Policy Research}

Geopolitics and Media

Menu

\begin{itemize}
\tightlist
\item
  \href{https://swprs.org}{Start}
\item
  \href{https://swprs.org/srf-propaganda-analyse/}{Studien}

  \begin{itemize}
  \tightlist
  \item
    \href{https://swprs.org/srf-propaganda-analyse/}{SRF / ZDF}
  \item
    \href{https://swprs.org/die-nzz-studie/}{NZZ-Studie}
  \item
    \href{https://swprs.org/der-propaganda-multiplikator/}{Agenturen}
  \item
    \href{https://swprs.org/die-propaganda-matrix/}{Medienmatrix}
  \end{itemize}
\item
  \href{https://swprs.org/medien-navigator/}{Analysen}

  \begin{itemize}
  \tightlist
  \item
    \href{https://swprs.org/medien-navigator/}{Navigator}
  \item
    \href{https://swprs.org/der-propaganda-schluessel/}{Techniken}
  \item
    \href{https://swprs.org/propaganda-in-der-wikipedia/}{Wikipedia}
  \item
    \href{https://swprs.org/logik-imperialer-kriege/}{Kriege}
  \end{itemize}
\item
  \href{https://swprs.org/netzwerk-medien-schweiz/}{Netzwerke}

  \begin{itemize}
  \tightlist
  \item
    \href{https://swprs.org/netzwerk-medien-schweiz/}{Schweiz}
  \item
    \href{https://swprs.org/netzwerk-medien-deutschland/}{Deutschland}
  \item
    \href{https://swprs.org/medien-in-oesterreich/}{Österreich}
  \item
    \href{https://swprs.org/das-american-empire-und-seine-medien/}{USA}
  \end{itemize}
\item
  \href{https://swprs.org/bericht-eines-journalisten/}{Fokus I}

  \begin{itemize}
  \tightlist
  \item
    \href{https://swprs.org/bericht-eines-journalisten/}{Journalistenbericht}
  \item
    \href{https://swprs.org/russische-propaganda/}{Russische Propaganda}
  \item
    \href{https://swprs.org/die-israel-lobby-fakten-und-mythen/}{Die
    »Israel-Lobby«}
  \item
    \href{https://swprs.org/geopolitik-und-paedokriminalitaet/}{Pädokriminalität}
  \end{itemize}
\item
  \href{https://swprs.org/migration-und-medien/}{Fokus II}

  \begin{itemize}
  \tightlist
  \item
    \href{https://swprs.org/covid-19-hinweis-ii/}{Coronavirus}
  \item
    \href{https://swprs.org/die-integrity-initiative/}{Integrity
    Initiative}
  \item
    \href{https://swprs.org/migration-und-medien/}{Migration \& Medien}
  \item
    \href{https://swprs.org/der-fall-magnitsky/}{Magnitsky Act}
  \end{itemize}
\item
  \href{https://swprs.org/kontakt/}{Projekt}

  \begin{itemize}
  \tightlist
  \item
    \href{https://swprs.org/kontakt/}{Kontakt}
  \item
    \href{https://swprs.org/uebersicht/}{Seitenübersicht}
  \item
    \href{https://swprs.org/medienspiegel/}{Medienspiegel}
  \item
    \href{https://swprs.org/donationen/}{Donationen}
  \end{itemize}
\item
  \href{https://swprs.org/contact/}{English}
\end{itemize}

\protect\hyperlink{}{Open Search}

\hypertarget{srf-die-propaganda-analyse}{%
\section{SRF: Die Propaganda-Analyse}\label{srf-die-propaganda-analyse}}

\includegraphics{https://swprs.files.wordpress.com/2017/01/srf-propaganda-analyse-600.png?w=736}

Das Schweizer Radio und Fernsehen (SRF) leistet mit seinen Nachrichten-
und Informations­sendungen einen wichtigen Beitrag zur öffentlichen
Meinungsbildung in der Schweiz. Doch wie objektiv und kritisch berichtet
das SRF über geopolitische Themen? Um dies zu überprüfen, wurde erstmals
eine systematische Analyse der SRF-Berichterstattung zu einem
geopolitischen Ereignis durchgeführt.

Die Resultate sind alarmierend: In allen untersuchten Beiträgen des SRF
wurden Propaganda- und Manipulationstechniken auf redaktioneller,
sprachlicher und audiovisueller Ebene festgestellt. Beispiele sind die
Zuteilung von Redezeit an nur eine Konfliktpartei, die intransparente
Kennzeichnung von Drittquellen, die Auslassung von Kontext, tendenziöse
Formulierungen, unbelegte Behauptungen und Suggestionen, manipulative
Bearbei­tungen von Film­material sowie Falschübersetzungen.

Alle verwendeten Manipulationstechniken fielen zugunsten der
Konfliktpartei USA/NATO aus. Insgesamt muss somit von einer einseitigen,
selektiv-unkritischen und wenig objektiven Berichterstattung durch das
Schweizer Radio und Fernsehen gesprochen werden. Mögliche Ursachen für
diesen Befund werden diskutiert.

\href{https://swprs.files.wordpress.com/2017/11/srf-propaganda-analyse-2016-tp.pdf}{Studie
als PDF herunterladen}

(Hinweis: Bei Interesse an der Studie bitte auf diese Seite verlinken.
Obige Zusammen­fassung und einzelne Auszüge können übernommen werden.
Keine Volltext-Kopie.)

\begin{center}\rule{0.5\linewidth}{\linethickness}\end{center}

\hypertarget{geopolitische-propaganda-im-uxf6ffentlichen-rundfunk-eine-analyse-am-beispiel-des-schweizer-radio-und-fernsehens}{%
\subsection{Geopolitische Propaganda im öffentlichen Rundfunk: Eine
Analyse am Beispiel des Schweizer Radio und
Fernsehens}\label{geopolitische-propaganda-im-uxf6ffentlichen-rundfunk-eine-analyse-am-beispiel-des-schweizer-radio-und-fernsehens}}

\emph{Eine Studie von \href{https://swprs.org/}{Swiss Propaganda
Research}}

Oktober 2016

*»Die SRG trägt zur freien Meinungsbildung des Publikums bei durch\\
umfassende, vielfältige und sachgerechte Information insbesondere\\
über politische, wirtschaftliche und soziale Zusammenhänge.«\\
*Aus der
\href{https://www.bakom.admin.ch/bakom/de/home/elektronische-medien/informationen-ueber-radio-und-fernsehveranstalter/srg-ssr/konzessionierung-und-technik-srg-ssr.html}{Konzession
des Schweizer Radio und Fernsehens}

Inhaltsübersicht

\begin{enumerate}
\def\labelenumi{\arabic{enumi}.}
\tightlist
\item
  \protect\hyperlink{kapitel1}{Das Untersuchungsmodell}
\item
  \protect\hyperlink{kapitel2}{Untersuchte Beiträge}
\item
  \protect\hyperlink{kapitel3}{Ergebnisse}

  \begin{enumerate}
  \def\labelenumii{\arabic{enumii}.}
  \tightlist
  \item
    \protect\hyperlink{kapitel3-1}{Manipulationstechniken}
  \item
    \protect\hyperlink{kapitel3-2}{Propagandabotschaften}
  \end{enumerate}
\item
  \protect\hyperlink{kapitel4}{Schlussfolgerungen}
\item
  \protect\hyperlink{ZDF}{Vergleich: Manipulationstechniken im ZDF}
\end{enumerate}

\hypertarget{1-das-untersuchungsmodell}{%
\subsubsection{1. Das
Untersuchungsmodell}\label{1-das-untersuchungsmodell}}

Für die vorliegende Studie wurde ein zweiteiliges Untersuchungsmodell
verwendet, bestehend aus rund 25 medialen Manipulationstechniken
einerseits sowie zehn Botschaften der Kriegspropaganda (basierend auf
\href{https://swprs.org/der-propaganda-schluessel/}{Ponsonby-Morelli})
andererseits:

\href{https://swprs.files.wordpress.com/2018/09/propaganda-schluessel-spr.pdf}{\includegraphics{https://swprs.files.wordpress.com/2018/09/srf-untersuchungsmodell.png?w=600}}

*Untersuchungsmodell bestehend aus Manipulationstechniken und
Propagandabotschaften.\\
\href{https://swprs.files.wordpress.com/2018/09/propaganda-schluessel-spr.pdf}{Als
PDF-Datei herunterladen.}\\
*

\hypertarget{2-untersuchte-beitruxe4ge}{%
\subsubsection{2. Untersuchte
Beiträge}\label{2-untersuchte-beitruxe4ge}}

Untersucht wurde die Berichterstattung des Schweizer Radio und
Fernsehens (SRF) vom 20. September 2016 zum Angriff auf einen
Hilfskonvoi des Syrisch-Arabischen Roten Halbmonds und der UNO in der
Nähe von Aleppo am Tag zuvor. Es handelt sich dabei um ein Ereignis im
Rahmen des Syrienkrieges, das weltweit für Schlagzeilen und Bestürzung
gesorgt hat.

\includegraphics{https://swprs.files.wordpress.com/2016/10/srf-konvoi.png?w=600\&h=337}

\emph{Ein LKW des zerstörten Hilfskonvois (SRF)}

Ausgewertet wurden folgende Sendungen des Schweizer Radio und Fernsehens
(die Links führen direkt zur entsprechenden Sendung im
SRF-Medienplayer):

\begin{itemize}
\tightlist
\item
  Die
  \emph{\href{http://www.srf.ch/play/tv/tagesschau-am-mittag/video/tagesschau-vom-20-09-2016-1245?id=72fb2535-87a4-4f59-b22e-4e9dd7a84436}{Tagesschau
  am Mittag}} (TM) um 12:45 Uhr
\item
  Die
  \emph{\href{http://www.srf.ch/play/tv/tagesschau/video/tagesschau-vom-20-09-2016-1930?id=d6397078-c449-4f65-8cf6-599e1b16c2ef}{Tagesschau-Hauptausgabe}}
  (TH) um 19:30 Uhr
\item
  Die Sendung
  \emph{\href{http://www.srf.ch/play/tv/10vor10/video/10vor10-vom-20-09-2016?id=5fbec62e-8498-4cbe-9a1b-7eb48207b2b4}{10
  vor 10}} (ZZ) um 21:50 Uhr
\item
  Das
  \emph{\href{http://www.srf.ch/play/radio/echo-der-zeit/audio/uno-stoppt-hilfsgueter-lieferungen-nach-syrien?id=2e4fd32d-6fbb-40d0-8b17-131179be6916}{Echo
  der Zeit}} (EZ) des Schweizer Radios um 18:00 Uhr
\end{itemize}

Die untersuchten Beiträge (Gesamtlaufzeit 21 Minuten und 24 Sekunden)
wurden transkribiert und schriftlich sowie audiovisuell auf die im
ersten Kapitel vorgestellten Manipulationstechniken hin ausgewertet.
Identifizierte Techniken wurden in Manipulation zugunsten oder zulasten
der Konfliktpartei USA/NATO eingeteilt. In einem zweiten Schritt wurde
ausgewertet, ob und ggf. welche Propagandabotschaften durch die Beiträge
des SRF transportiert wurden.

\includegraphics{https://swprs.files.wordpress.com/2016/10/srf-sendungen.png?w=600\&h=344}

\emph{Die vier untersuchten Sendungen des Schweizer Radio und Fernsehens
(SRF)}

\hypertarget{3-ergebnisse}{%
\subsubsection{3. Ergebnisse}\label{3-ergebnisse}}

\hypertarget{31-manipulationstechniken}{%
\paragraph{3.1.
Manipulationstechniken}\label{31-manipulationstechniken}}

Im Folgenden werden die Untersuchungsresultate sortiert nach
Manipulationstechnik präsentiert. Im Anschluss folgt eine
zusammenfassende Darstellung der transportierten Propaganda­botschaften.

\textbf{\{1\} Redaktionelle Manipulationstechniken}

\textbf{\{1a\} Themenauswahl, Gewichtung und Platzierung:} Das Schweizer
Fernsehen widmete dem Angriff auf den Hilfskonvoi in den untersuchten
Sendungen folgende Beiträge:

\begin{itemize}
\tightlist
\item
  \textbf{Tagesschau am Mittag}: Hauptthema; 04:30 Minuten: ca. 30\% der
  Sendezeit
\item
  \textbf{Tagesschau-Hauptausgabe}: Hauptthema; 06:11 Minuten; ca. 25\%
  der Sendezeit
\item
  \textbf{Echo der Zeit}: Hauptthema; 09:23 Minuten; ca. 20\% der
  Sendezeit
\item
  \textbf{10 vor 10}: Im hinteren Teil der Sendung; 01:20 Minuten; ca.
  8\% der Sendezeit
\end{itemize}

Mit Ausnahme der spätabendlichen Sendung \emph{10 vor 10} wurde über das
Thema somit sehr prominent und umfangreich berichtet. Angesichts der
Tragik des Ereignisses (Angriff auf einen Hilfskonvoi zur Versorgung des
belagerten Aleppos) und den inter­nationalen Reaktionen erscheint die
Auswahl und Priorisierung des Themas jedoch gerechtfertigt.

Für die objektive Beurteilung der Themengewichtung ist zudem ein
Vergleich mit der Berichterstattung über ähnliche Ereignisse hilfreich.
Beispielsweise wurde die Stadt Aleppo
\href{http://www.thealeppoproject.com/aleppo-conflict-timeline-2013/}{bereits
im Jahre 2013 von Juli bis Oktober für rund drei Monate belagert} und
von der Außenwelt abgeschnitten. Hilfslieferungen wurden blockiert und
Zivilisten, die Nahrungsmittel oder Medikamente in die belagerten
Stadtteile bringen wollten, bedrängt oder getötet. Verglichen mit 2016
waren damals sogar fünf bis zehn Mal mehr Menschen
\href{https://web.archive.org/web/20170623121133/https://www.al-monitor.com/pulse/originals/2013/07/aleppo-syria-rebel-siege-assad.html}{von
der Blockade betroffen}. Im Unterschied zu 2016 fand die Belagerung von
2013 indes nicht durch die Regierung, sondern durch die Rebellen statt,
die damals auf dem Vormarsch waren, 2016 jedoch in die Defensive
gerieten.

Eine Suche im
\href{http://www.srf.ch/play/tv/sendungen-nach-datum?date=15-07-2013}{Archiv}
des Schweizer Radio und Fernsehens ergibt jedoch keinen Beitrag zu
dieser Belagerung vom Sommer 2013. Auch eine
\href{http://www.srf.ch/news/international/der-buergerkrieg-in-syrien-im-zeitraffer}{chronologische
Übersicht} zum Syrienkonflikt vom Oktober 2013 auf der Internetseite des
SRF erwähnt die Belagerung nicht. Dies könnte darauf hindeuten, dass das
Schweizer Radio und Fernsehen die Berichterstattung über solche
Ereignisse unterschiedlich gewichtet, je nachdem, ob Verbündete oder
Gegner der Konfliktpartei USA/NATO betroffen sind (siehe Box). Für eine
abschließende Bewertung dieser Frage wäre indes eine
\href{https://swprs.org/srf-ombudsstelle-im-faktencheck/}{longitudinale
Analyse} über einen längeren Zeitraum erforderlich.

Über die drei­monatige \textbf{Belagerung Aleppos} durch die Rebellen im
Sommer 2013 wurde von den meisten deutsch­sprachigen Medien nicht oder
nur am Rande berichtet. Die \emph{NZZ} erwähnte die beginnende
Belagerung und Aushungerung am
\href{http://www.nzz.ch/kaempfe-in-homs-und-aleppo-eskalieren-1.18113802}{10.
Juli} mit einem Satz \emph{(``Im Norden belagern die Rebellen hingegen
die von der Armee gehaltene Hälfte von Aleppo und versuchen, diese
auszuhungern.'')} sowie am
\href{http://www.nzz.ch/machtkaempfe-im-befreiten-syrien-1.18115794}{13.
Juli} in einem Absatz \emph{(``So protestierten in Aleppo diese Woche
Hunderte gegen die Blockade, welche die Rebellen gegen die von der
Regierung kontrollierten Stadtteile verhängt hatten. In die betroffenen
Viertel gelangten keine Lebens­mittel und Medi­ka­mente, und Aktivisten
wiesen warnend darauf hin, dass dort schnell der Hunger einziehen
könnte. Aufständische Kämpfer gingen am Dienstag mit Schuss­waffen gegen
die Demonstranten vor und sollen einen von diesen getötet haben.}
\emph{Dass Kämpfer, die sich als Verteidiger des Islams aufspielen, zu
Beginn des Fasten­monats Ramadan ganze Stadt­viertel aushungern, hat sie
kaum beliebter gemacht.'')} Die folgenden drei Monate, bis zur
\href{http://www.thealeppoproject.com/aleppo-conflict-timeline-2013/}{Sprengung
der Belagerung im Oktober}, berichtete die \emph{NZZ} indes nicht mehr
über die Blockade der rund 2 Millionen Menschen.

\textbf{\{1b\} Schlagzeilen}: Das SRF wählte für die untersuchten
Beiträge folgende Schlagzeilen:

\begin{itemize}
\tightlist
\item
  ``UNO stoppt Hilfe'' (TM)
\item
  ``Grausam und verwerflich'' (ein Zitat des UN-Generalsekretärs) (TH)
\item
  Keine Schlagzeile (ZZ)
\item
  ``Die Waffenruhe in Syrien ist gescheitert, ein Hilfskonvoi wird
  bombardiert. Die USA, Russland und Syrien schieben sich die Schuld
  zu.'' (EZ)
\end{itemize}

Der Titel der \emph{Tagesschau-Hauptausgabe} ist als eher emotional
einzustufen \{4d\}, aber aus geopolitischer Sicht kann bei keinem der
gewählten Titel \emph{a priori} von einem Propagandaeffekt gesprochen
werden. Auf den Titel der Radiosendung \emph{Echo der Zeit}, der das
Ende der Waffenruhe sprachlich mit dem Angriff auf den Hilfskonvoi
verknüpft, wird bei den Assoziationen unter Punkt \{2\} noch
eingegangen.

\textbf{\{1c\} Konfliktparteien:} In den untersuchten Beiträgen des SRF
kam lediglich die Konfliktpartei USA/NATO selbst zu Wort. Dies geschah
in Person des US-Außen­ministers \emph{(TM, 02:26)} und des
US-Präsidenten \emph{(TH, 03:43),} die ihre Sicht des Vorfalls darlegen
konnten. Die Konfliktpartei Syrien/ Russland kam hingegen, ungeachtet
der gegen sie erhobenen Vorwürfe, in keiner der untersuchten Sendungen
zu Wort. Bei der Verteilung der Redezeiten auf die Konfliktparteien
besteht somit ein deutlicher Propagandaeffekt zugunsten der
Konfliktpartei USA/NATO.

\includegraphics{https://swprs.files.wordpress.com/2016/10/srf-kerry-obama.png?w=600\&h=174}

\emph{Redezeit: Vertreter der Konfliktpartei USA/NATO (SRF)}

\textbf{\{1d\}} \textbf{Drittquellen}: Das Schweizer Radio und Fernsehen
verwendete in den untersuchten Sendungen insgesamt vier verschiedene
Dritt­quellen, d.h. Quellen, die nicht direkt zu einer der
Konfliktparteien gehören:

\begin{itemize}
\tightlist
\item
  die UNO (TM 01:50; TH 01:07, 02:32, 03:15; ZZ 18:55; EZ 02:20)
\item
  Die \emph{Syrische Beobachtungsstelle für Menschenrechte} (TM 01:20;
  EZ 02:53)
\item
  Die Organisation \emph{Syrian Civil Defence} bzw. \emph{White Helmets}
  (TM 01:30)
\item
  Die Medienorganisation \emph{Aleppo24} (ZZ 18:37)
\end{itemize}

Während die UNO als neutral anzusehen ist, sind die drei anderen
Organisationen der syrischen Opposition und mithin der Konfliktpartei
USA/NATO zuzurechnen: Die
\emph{\href{https://de.wikipedia.org/wiki/Syrische_Beobachtungsstelle_f\%C3\%BCr_Menschenrechte}{Syrische
Beobach­tungs­stelle für Menschen­rechte}} hat ihren Sitz in London und
wird von einem syrischen Exilanten geleitet; die
\emph{\href{https://en.wikipedia.org/wiki/Syrian_Civil_Defense}{White
Helmets}} werden unter anderem von den USA, England und Deutschland
finanziert und operieren ausschließlich auf dem Gebiet der syrischen
Rebellen; und
\emph{\href{http://hosted2.ap.org/APDEFAULT/3d281c11a96b4ad082fe88aa0db04305/Article_2016-09-20-Syria/id-c7267984287249e3be724a1da5e14af6}{Aleppo24}}
wird von der Nachrichtenagentur AP als eine \emph{``Syrian
anti-government group''} beschrieben.

Das SRF machte die politische Zugehörigkeit der drei syrischen
Organisationen -- immerhin die einzigen Zeugen des Angriffs -- indes
nicht transparent, sondern sprach stattdessen allgemein von
\emph{``Hilfsorganisation''} oder \emph{``Amateurvideo''}, sodass dem
Publikum eine vermeintliche Neutralität suggeriert wurde. Einzige
Ausnahme war das \emph{Echo der Zeit}, welches die \emph{``Syrische
Beobachtungsstelle für Menschenrechte''} korrekt als ``oppositionsnah''
bezeichnete.

Keine der verwendeten Drittquellen war indes der Konfliktpartei
Syrien/ Russland zuzurechnen. Bei der Verwendung und Kennzeichnung von
Drittquellen durch das SRF besteht somit ebenfalls ein deutlicher
Propagandaeffekt zugunsten der Konfliktpartei USA/NATO.

\href{https://swprs.files.wordpress.com/2016/10/srf-white-helmets.png}{\includegraphics{https://swprs.files.wordpress.com/2016/10/srf-white-helmets.png?w=600\&h=337}}

\emph{Drittquelle: Die ``White Helmets'' (SRF)}

\textbf{\{1e\} Interviews}: Interviews wurden, von den SRF-eigenen
Korrespondenten abgesehen, nur eines geführt: Im \emph{Echo der Zeit}
(EZ, 08:39 bis 15:01) mit einem Vertreter der deutschen
\emph{\href{https://de.wikipedia.org/wiki/Stiftung_Wissenschaft_und_Politik}{Stiftung
Wissenschaft und Politik (SWP)}.} Die SWP wird hauptsächlich von der
deutschen Bundesregierung und somit von einem Mitglied der
Konfliktpartei USA/NATO finanziert. Der Leiter der SWP ist denn auch
\href{https://spiegelkabinett-blog.blogspot.com/2014/05/stiftung-fur-wissenschaft-und-politik.html}{Mitglied}
in zahlreichen hochkarätigen Transatlantik-Netzwerken. 2012 organisierte
die SWP zusammen mit einer US-Organisation zudem eine Serie von
Workshops mit syrischen Oppositionellen und Rebellen, um die Zeit nach
dem Regierungssturz zu planen
(\href{http://www.zeit.de/2012/31/Syrien-Bundesregierung}{Projekt
\emph{»Day After«}}). Bei der SWP muss mithin von einem
transatlantischen \emph{Think Tank} gesprochen werden, der klar der
Konfliktpartei USA/NATO zuzuordnen ist.

Die Moderatorin des Interviews machte die politische Verortung der SWP
und ihres Vertreters jedoch nicht transparent, sondern sprach neutral
vom ``Nahost-Experten der deutschen Stiftung Wissenschaft und Politik''
(vgl. auch Technik \{4a\}: Berufung auf Autorität).

Im Interview kam die geopolitische Orientierung des Gesprächspartners
allerdings deutlich zum Ausdruck und kulminierte in der Aussage, ein
sich kurz zuvor ereigneter amerikanischer Luftangriff auf syrische
Truppen sei \emph{``natürlich eine ganz große Katastrophe'',} nicht
jedoch wegen der Toten und Ver­letzten, sondern \emph{``weil dieser
Angriff der Amerikaner den Russen und auch den Syrern die Möglich­keit
gegeben hat, ihnen jetzt das Ende des Waffenstillstands in die Schuhe zu
schieben.'' (EZ, 13:03)}

Der Effekt des Interviews wurde noch verstärkt, indem die Moderatorin
die geopolitisch deutlich gefärbten Aussagen des SWP-Experten -- mit
einer Ausnahme \emph{(EZ 10:28)} -- nicht hinter­fragte. Auf den Inhalt
des Interviews wird unter Punkt \{2\} zu den sprachlichen Techniken noch
genauer eingegangen.

Neben der direkten Redezeit und den Drittquellen wurde vom SRF somit
auch das einzige Interview der Konfliktpartei USA/NATO zugeteilt, die
politische Verortung des Interviewpartners nicht transparent gemacht,
und seine Aussagen kaum hinterfragt. Insgesamt muss daher auch in der
Dimension Interviews/ Gäste von einem deutlichen Propagandaeffekt
zugunsten der Konfliktpartei USA/NATO gesprochen werden.

\textbf{\{1f\} Auslassung von Kontext:} Eine Auslassung von relevanter
Kontext- und Hintergrundinformation konnte in den untersuchten Beiträgen
des Schweizer Radio und Fernsehens verschiedentlich festgestellt werden.
Drei Beispiele:

\begin{itemize}
\tightlist
\item
  In allen Sendungen wurde davon gesprochen, dass die syrische Regierung
  die Waffenruhe beendigt hatte, jedoch nirgends erwähnt,
  \href{http://www.latimes.com/world/la-fg-syria-cease-fire-20160919-snap-story.html}{aus
  welchem Grund dies geschah}: Die syrische Regierung machte über 300
  dokumentierte Verletzungen der Waffenruhe durch die Rebellen geltend
  sowie den erwähnten US-Luftangriff auf syrische Truppen.
\item
  In keiner der Sendungen wurde erwähnt, dass die Rebellen im
  eingekesselten Ost-Aleppo die geplanten Hilfslieferungen explizit
  ablehnten und sogar
  \href{https://web.archive.org/web/20170202013724/http://ogn.news/306/}{öffentliche
  Demonstrationen} dagegen veranstalteten. Auch der Chef der
  Al-Nusra-Rebellen, die den Zielort des Konvois kontrollierten,
  erklärte bereits vorab in einem
  \href{https://www.alaraby.co.uk/english/news/2016/9/17/us-has-surrendered-to-assad-says-jfs-head}{Interview}
  mit dem Fernsehsender Al-Jazeera, dass er die Hilfslieferungen nicht
  akzeptieren werde. Der Grund für diese Ablehnung bestand darin, dass
  die Rebellen Hilfslieferungen, die von der syrischen Regierung
  genehmigt werden mussten, als Erniedrigung empfanden.
\item
  In allen Beiträgen wurde von vermuteten oder tatsächlichen
  Luftangriffen gesprochen, mögliche alternative Szenarien oder
  Erklärungen für die Zerstörung des Konvois jedoch nicht erwähnt (bspw.
  eine Zerstörung durch Artillerie, einen Brand, eine Explosion oder
  eine bewaffnete Drohne).
\end{itemize}

Durch solche Auslassungen von Kontextinformation wurde die
Aufmerksamkeit des Publikums auf ein bestimmtes Szenario bzw. Narrativ
gelenkt. Dieser Effekt ist oftmals mit sprachlichen Techniken wie
Insinuationen oder Suggestionen kombiniert und dadurch verstärkt worden
(siehe Punkt \{2c\}).

Ein weiteres Beispiel für eine Auslassung betrifft den Umstand, dass die
Konfliktpartei Syrien/ Russland ungeachtet der gegen sie erhobenen
Anschuldigungen in keinem der Beiträge selbst zu Wort kam. In der
\emph{Tagesschau am Mittag} wurde dies noch wie folgt zu begründen
versucht:

\emph{(02:47) \textbf{Sprecherin:} ``Das syrische Militär äußerte sich
zunächst nicht zu dem Vorfall.''}

Dabei ließ die Sprecherin unerwähnt, dass bereits am Vormittag das
russische Militär eine Verwicklung sowohl der russischen wie der
syrischen Luftwaffe
\href{http://www.wsj.com/articles/death-toll-rises-in-syria-aid-convoy-attack-1474370636}{dementiert
hatte}.

In allen identifizierten Fällen von ausgelassener Kontext- und
Hintergrund­information fiel dies zu Ungunsten der Konfliktpartei
Syrien/ Russland aus. Somit muss auch in dieser Kategorie von einem
deutlichen Propagandaeffekt zugunsten der Konfliktpartei USA/NATO
gesprochen werden.

\textbf{\{2\} Sprachliche Techniken}

\textbf{\{2a\} Unterstellungen, unbelegte oder falsche Behauptungen:}
Das SRF machte in seiner Bericht­erstattung zum Angriff auf den
Hilfskonvoi mehrere unbelegte und sogar einige falsche Be­haup­tungen.
Dies geschah primär im Zusammenhang mit der Frage, ob es sich beim
Angriff auf den Konvoi um einen Luftangriff gehandelt hat. Diese Frage
ist deshalb brisant, weil für einen Luftangriff ver­mut­lich nur die
syrische oder russische Luftwaffe in Betracht käme (wobei auch ein
\href{http://www.wsj.com/articles/russia-suggests-u-s-drone-may-have-hit-aid-convoy-in-syria-1474483634}{Drohnen­angriff}einer
anderen Partei denkbar wäre, was vom SRF jedoch nicht in Betracht
gezogen wurde, siehe Punkt \{1f\}).

Details zu Art und Urhebern des Angriffs blieben den ganzen
Untersuchungszeitraum über unklar. Die Konfliktpartei USA/NATO
\href{http://www.nytimes.com/2016/09/21/world/middleeast/syria-cease-fire.html}{sprach}
von Anbeginn von einem russischen oder syrischen Luftangriff. Die
Konfliktpartei Syrien/ Russland
\href{http://www.wsj.com/articles/death-toll-rises-in-syria-aid-convoy-attack-1474370636}{dementierte},
einen Luftangriff geflogen zu haben. Der Rote Halbmond machte in seiner
\href{http://sarc.sy/syria-attack-humanitarian-convoy-attack-humanity/}{schriftlichen
Pressemitteilung} zur Art des Angriffs keine Angaben. Auch die UNO
sprach in ihren
\href{http://www.unocha.org/top-stories/all-stories/syria-unsarc-convoy-hit-urum-al-kubra-northwest-aleppo-city}{Stellungnahmen}
am 19. und 20. September 2016 allgemein von einem ``Angriff''.

Am Nachmittag des 20. Septembers war in einer
\href{http://reliefweb.int/report/syrian-arab-republic/statement-attributable-massimo-diana-un-resident-and-humanitarian}{Pressemitteilung}
der UNO indes von einem ``Luftangriff'' die Rede. Dies wurde jedoch
\href{http://www.reuters.com/article/us-mideast-crisis-syria-communications-idUSKCN11Q20M}{wenige
Minuten später korrigiert}: Es habe sich um einen ``Entwurfsfehler''
gehandelt, zur Art des Angriffes könne man noch keine Angaben machen.
Dessen ungeachtet sprach das SRF in allen Beiträgen von einem
Luftangriff, meist sogar explizit von einem syrischen oder russischen
Luftangriff:

\begin{itemize}
\tightlist
\item
  \textbf{Tagesschau am Mittag:} \emph{``Die Vereinten Nationen
  reagierten mit Fassungslosigkeit auf den Luftangriff.''}
  \emph{(01:50)} und \emph{``Der Angriff kam entweder von syrischen oder
  russischen Flugzeugen. Dies steht für die USA außer Frage.'' (02:13).}
\item
  \textbf{Tagesschau am Abend:} \emph{``Allerdings deutet viel
  daraufhin, dass die syrische oder die russische Luftwaffe
  dahintersteckt.'' (01:28),} \emph{``Ein Konvoi der UNO und des Roten
  Halbmondes wird von der Luft aus bombardiert.'' (02:14)} und
  \emph{``Der Angriff soll entweder von syrischer oder russischer Seite
  gekommen sein. Für die USA ist das klar.'' (02:52).}
\item
  \textbf{10 vor 10:} \emph{``nach dem offenbar vorsätzlichen
  Luftangriff auf einen Hilfskonvoi'' (18:00),} \emph{``Der Konvoi mit
  Hilfsgütern wurde gestern Abend gezielt aus der Luft bombardiert.''
  (18:18)} und \emph{``Im Verdacht stehen die syrische Armee oder die
  verbündete russische Luftwaffe.'' (18:37).}
\item
  \textbf{Echo der Zeit:} \emph{``Vergangene Nacht wurde ein Hilfskonvoi
  in Syrien in der Nähe von Aleppo von der Luft aus angegriffen.''
  (01:07)} und \emph{``Ein Konvoi wurde vergangene Nacht sogar aus der
  Luft angegriffen.'' (08:17).}
\end{itemize}

Das SRF übernahm mithin das noch unbelegte Narrativ der Konfliktpartei
USA/NATO, ohne alternative Szenarien zu erwähnen (Auslassung von
Kontext, siehe Punkt \{1f\}). Die journalistisch erforderliche,
kritische Distanz zu allen Parteien war in den untersuchten Sendungen
nicht gegeben.

Zahlreiche Unterstellungen und unbelegte Behauptungen fanden sich zudem
im Interview mit dem Vertreter der deutschen \emph{Stiftung Wissenschaft
und Politik} im \emph{Echo der Zeit} (siehe \{1e\}). Diese wurden in der
Untersuchung jedoch nicht dem SRF angerechnet. Einige Auszüge:

\emph{(08:50) \textbf{Interviewpartner}: ``Ich denke, dass der
wichtigste Grund für das Scheitern der Waffenruhe der Unwille der
Regierung und auch ihrer Verbündeten ist, überhaupt die Waffen schweigen
zu lassen. (\ldots{})''}

\emph{(09:18) \textbf{Interviewpartner}: ``(\ldots{}) In den letzten
Tagen scheint es auf russischer Seite aber doch eher so zu sein, dass
das Interesse bei Ihnen an diesem Waffenstillstand nicht so sehr
ausgeprägt ist.''}

\emph{(09:54) \textbf{Interviewpartner}: ``(\ldots{}) Aber mein Eindruck
ist, dass sie {[}die Russen{]} das nicht so ernst genommen haben, und
dass sie nur nach einer Möglichkeit gesucht haben, das Scheitern der
Waffenruhe den Amerikanern anzuhängen.''}

\emph{(10:28) \textbf{Interviewpartner}: ``(\ldots{}) Jetzt sehen wir
aber in den letzten Tagen, dass die Russen offensichtlich gar keine
großen Bemühungen unternehmen, um ihren Verbündeten in die Schranken zu
weisen, also das Assad-Regime.''}

Unterstellungen, unbelegte oder falsche Behauptungen zulasten der
Konfliktpartei USA/NATO konnten in den Beiträgen des SRF hingegen nicht
identifiziert werden. Insgesamt muss somit auch in dieser Kategorie von
einem Propagandaeffekt zugunsten der Konfliktpartei USA/NATO gesprochen
werden.

\textbf{\{2b\} Wortwahl, Formulierungen, Bezeichnungen:} Das Schweizer
Radio und Fernsehen verwendete im Allgemeinen eine neutrale Wortwahl.
Die Konfliktpartei Syrien wurde zumeist neutral als ``Syrische
Regierung'' bezeichnet, einige Male aber auch
\href{http://www.duden.de/rechtschreibung/Regime}{abwertend} als
``Syrisches Regime'' \emph{(TH 03:00, EZ 08:17).}

Eine klar tendenziöse Wortwahl fand sich hingegen in den
Interview-Antworten des Vertreters der deutschen \emph{Stiftung
Wissenschaft und Politik} (siehe \{1e\} und \{2a\}): Dieser sprach
repetitiv von \emph{``Assad-Regime'',} pauschal von \emph{``russischer
und syrischer Propaganda'',} oder davon, dass die Konfliktpartei
Russland \emph{``die Verantwortung den Amerikanern in die Schuhe
schieben''} wolle. Diese abwertenden Formulierungen wurden in der
Untersuchung jedoch nicht dem SRF angerechnet.

In der \emph{Tagesschau am Mittag} und in der \emph{Tagesschau am Abend}
kam zudem eine \textbf{manipulative Syntax (Wort- und Satzstellung)} zur
Anwendung:

\begin{itemize}
\tightlist
\item
  Tagesschau am Mittag: \emph{(02:13) \textbf{Sprecherin}: ``Der Angriff
  kam entweder von syrischen oder russischen Flugzeugen. Dies steht für
  die USA außer Frage.''}
\item
  Tagesschau am Abend: \emph{(02:52) \textbf{Sprecherin}: ``Der Angriff
  soll entweder von syrischer oder russischer Seite gekommen sein.
  (Kunstpause). Für die USA ist das klar.''}
\end{itemize}

In beiden Fällen wird im ersten Satz eine unbelegte Behauptung dem
Publikum als \emph{Tatsache} präsentiert (im zweiten Beispiel verstärkt
durch die anschließende Kunstpause). Danach folgt keine Einschränkung,
sondern der Zusatz, dass dieser Umstand für die USA ``außer Frage
stehe'' bzw. ``klar sei'' -- wobei nicht gesagt wird, worauf sich diese
Aussage stützt bzw. dass dafür
\href{http://www.nytimes.com/2016/09/21/world/middleeast/syria-cease-fire.html}{keine
Belege vorgelegt wurden} \{1f\}. Journalistisch korrekt wäre es in
beiden Fällen, die (unbelegte) Behauptung klar als solche zu
kennzeichnen und den Autor (hier die Konfliktpartei USA) zu Beginn zu
nennen.

Variante mit gegenteiliger Wirkung: \emph{\textbf{Sprecherin}}*: ``Der
Angriff kam entweder von amerikanischen oder türkischen Flugzeugen. Dies
steht für Russland außer Frage.''*

Eine weitere manipulative Formulierung wurde im Beitrag von \emph{10 vor
10} identifiziert:

\emph{(18:37) \textbf{Sprecher}: ``Im Verdacht stehen die syrische Armee
oder die verbündete russische Luftwaffe.''}

Hier sagt der Sprecher, ``die syrische oder russische Luftwaffe''
stünden ``im Verdacht'', aber er sagt nicht, durch wen, sodass ein
``allgemeiner Verdacht'' suggeriert wird (vgl. Punkt \{2c\}:
Suggestionen). Bei obiger Satzkonstruktion wie auch bei diesem Beispiel
zeigt sich erneut, dass das Schweizer Radio und Fernsehen tendenziell
die Sichtweise der Konfliktpartei USA/NATO übernimmt und transportiert.

Manipulative Formulierungen zulasten der Konfliktpartei USA/NATO konnten
in den Beiträgen des SRF nicht identifiziert werden. Insgesamt ergibt
sich somit auch in dieser Kategorie ein Propagandaeffekt zugunsten der
Konfliktpartei USA/NATO, der jedoch -- mit Ausnahme des
Interview­partners der \emph{Stiftung Wissenschaft und Politik} --
vergleichsweise moderat ausgeprägt ist.

\textbf{\{2c\} Suggestionen, Insinuationen und Assoziationen:} Mit den
Techniken dieser Kategorie werden fragliche Behauptungen nicht explizit
ausgesprochen (siehe Kategorie \{2a\}), sondern sprachlich impliziert
oder nahegelegt. Das Schweizer Radio und Fernsehen machte hiervon
relativ häufig Gebrauch.

Eine typische sprachliche Assoziation kombinierte die Beendigung der
Waffenruhe durch die syrische Regierung mit dem Angriff auf den
Hilfskonvoi, wodurch eine mögliche oder sogar wahrscheinliche Kausalität
nahegelegt wurde. Dieser Effekt wurde noch verstärkt, indem die
tatsächlichen Hintergründe der Beendigung nicht genannt wurden
(Auslassung von Kontext, \{1f\}):

\begin{itemize}
\tightlist
\item
  \emph{\textbf{Moderatorin}}*: ``Der Angriff ereignete sich wenige
  Stunden, nachdem die syrische Regierung die Waffenruhe für beendet
  erklärt hatte.'' (TM, 01:01)*
\item
  \emph{\textbf{Moderatorin}}*: ``Nur Stunden nachdem Syrien die
  Waffenruhe für beendet erklärt, werden Hilfs­lieferungen beschossen
  und Hilfspersonal getötet.'' (TH, 04:01)*
\item
  \emph{\textbf{Moderatorin}}*: ``Kaum eine Woche dauerte die wacklige
  Waffenruhe in Syrien. Das syrische Regime erklärte sie heute für
  beendet und fliegt wieder Angriffe. () Ein Konvoi wurde vergangene
  Nacht sogar aus der Luft angegriffen, wir haben es eingangs gehört.''
  (EZ, 08:01)*
\end{itemize}

Variante mit gegenteiliger Wirkung (inhaltlich ebenfalls korrekt):
\emph{\textbf{Moderatorin}}*: ``Nur Stunden nachdem die Rebellen eine
neue Offensive gestartet haben, werden Hilfs­­lieferungen beschossen und
Hilfspersonal getötet.''*

Eine weitere Suggestion fand sich im Beitrag von \emph{10 vor 10}:

\emph{(19:07) \textbf{Sprecherin}: ``Das russische
Verteidigungs­ministerium ließ heute Abend verlauten, dass der Konvoi
von einem bewaffneten Fahrzeug begleitet worden sei. Dies könnte als
indirektes Eingeständnis verstanden werden.''}

Hier wird explizit ein ``indirektes (Schuld-)Eingeständnis'' der
Konfliktpartei Russland suggeriert, obschon es ein solches keineswegs
gegeben hat. Die Suggestion wird wiederum verstärkt durch das Weglassen
des Kontextes (Technik \{1f\}), denn tatsächlich hatte das russische
Verteidigungs­ministerium
\href{https://sputniknews.com/middleeast/201609201045521488-aleppo-un-convoy/}{das
Gegenteil nahegelegt}: dass womöglich das bewaffnete Fahrzeug der
Rebellen den Hilfskonvoi angriff.

In den Beiträgen des SRF fanden sich auch ganze
\textbf{Suggestions-Ketten}, bei denen mehrere Suggestionen aufeinander
aufbauen. Bei­spiels­­weise wurde in der \emph{Tagesschau-Hauptausgabe}
zunächst suggeriert, die Konfliktpartei Syrien/ Russland sei für den
(Luft-)Angriff auf den Hilfskonvoi verantwortlich, um dann zu
suggerieren, dies sei absichtlich erfolgt, um dann weiter zu
suggerieren, die Konfliktpartei Syrien/ Russland sei somit für den
Zusammenbruch der Waffenruhe und für das humanitäre Leid in Syrien
verantwortlich.

\emph{(01:28) \textbf{Moderatorin}: ``Allerdings deutet viel daraufhin,
dass die syrische oder die russische Luftwaffe dahintersteckt.''}

\emph{(02:52) \textbf{Sprecherin}: ``Der Angriff soll entweder von
syrischer oder russischer Seite gekommen sein. (Pause) Für die USA ist
das klar.''}

\emph{(03:53) Einspielung wartender UN-Konvoi. \textbf{Sprecherin}:
``Hunderttausende Syrer sind dringend auf humanitäre Hilfe angewiesen.
Ihre Lage wird nun noch verzweifelter.''}

\emph{(04:01) \textbf{Moderatorin}: ``Nur Stunden nachdem Syrien die
Waffenruhe für beendet erklärt, werden Hilfslieferungen beschossen und
Hilfspersonal getötet. () Kann das ein Unglück gewesen sein? Denn alles
deutet ja eigentlich auf einen gezielten Angriff hin.''}

Suggestionen, Insinuationen oder Assoziationen zulasten der
Konfliktpartei USA/NATO konnten in den Beiträgen des SRF nicht
identifiziert werden. Insgesamt ergibt sich somit auch in dieser
Kategorie ein deutlicher Propagandaeffekt zugunsten der Konfliktpartei
USA/NATO.

\textbf{\{2d\} Übersetzungen und Zitierungen:} In den untersuchten
Beiträgen des SRF wurde ein Fall einer manipulativen Übersetzung
identifiziert. Es handelt sich dabei um eine Sequenz aus der
\href{http://webtv.un.org/media/geneva-press-briefings/watch/geneva-press-briefing-hrc-ifrc-ocha-who-iom-unhcr-ohchr-ilo-gavi/5131061146001}{UNO-Pressekonferenz},
die der Sprecher in der \emph{Tagesschau-Hauptausgabe} aus dem
Englischen übersetzte. Dabei wird das Gesagte jedoch subtil verändert,
so dass aus einer neutralen Klarstellung durch die UNO eine
Anschuldigung an die syrische Regierung wird (Verschiebung der
Illokution bzw. Sprechabsicht).

\includegraphics{https://swprs.files.wordpress.com/2016/10/srf-uno-translation.png?w=600\&h=337}

\emph{Der Pressesprecher der UNO (SRF)}

Im
\href{http://webtv.un.org/media/geneva-press-briefings/watch/geneva-press-briefing-hrc-ifrc-ocha-who-iom-unhcr-ohchr-ilo-gavi/5131061146001}{englischen
Original} lautete die Aussage des UNO-Sprechers wie folgt (09:48):
\emph{``{[}First of all, it's important to stress that this particular
convoy was fully deconflicted.{]} That means, all approvals had been
obtained by the government and by the authorities, and every single
partner or party to the conflict -- men wearing weapons or having access
to lethal weapons -- had been notified, duly notified about this.
{[}This notification of the convoy also extends to and through the
Russians and the Americans.{]}''}

Auf Deutsch: \emph{``{[}Zunächst ist es wichtig zu betonen, dass dieser
spezifische Konvoi vollständig von Konflikten befreit wurde.{]} Das
bedeutet: Alle Bewilligungen wurden erhalten von der Regierung und von
den Autoritäten, und jeder einzelne Partner und jede Konfliktpartei --
bewaffnete Gruppen oder Gruppen mit Zugang zu tödlichen Waffen -- wurden
darüber benachrichtigt, ordentlich benachrichtigt. {[}Diese
Benachrichtigung über den Konvoi schließt auch die Russen und die
Amerikaner mit ein.{]}''}

Es handelt sich bei dieser Aussage wie ersichtlich um eine neutrale
Klarstellung oder sogar Rechtfertigung durch die UNO, und nicht um eine
Anschuldigung an eine der Konfliktparteien.

Der Nachrichtensprecher übersetzte diese Passage indes wie folgt:

\emph{(02:32) Einspielung UNO-Pressekonferenz. \textbf{Sprecher
übersetzt}: ``Diese Hilfsoperation war von der syrischen Regierung doch
bewilligt worden! Und sämtliche Konfliktparteien, alle, die im Besitz
von Waffen sind, waren in Kenntnis darüber, wann und wo sich der Konvoi
bewegt.''}

Hier wurde folgendes gemacht: Der Ausdruck ``Regierung und Autoritäten''
-- der auch Autoritäten auf Rebellenseite umfassen kann -- wurde auf
``syrische Regierung'' verkürzt. Dann wurde der im Original nicht
vorhandene Modalpartikel ``doch'' eingefügt und die Betonung auf das
Ende des ersten Satzes verschoben (vgl.
\href{http://www.srf.ch/play/tv/tagesschau/video/tagesschau-vom-20-09-2016-1930?id=d6397078-c449-4f65-8cf6-599e1b16c2ef}{Tonspur
der Tagesschau}; in der Transkription durch das Ausrufezeichen
gekenn­zeichnet). Dadurch wurde die neutrale Klarstellung der UNO in
einen Vorwurf an die syrische Regierung transformiert und sprachlich
impliziert, dass aus Sicht der UNO vermutlich die syrische Regierung für
den Angriff verantwortlich sei. Dieser Effekt wird noch verstärkt, da in
der Einspielung der erste Satz \emph{(``fully deconflicted'')} und der
letzte Satz \emph{(``to and through the Russians and the Americans'')}
fehlt.

Da die übersetzende Stimme des Nachrichtensprechers das englische
Original vollständig überdeckt, haben die Zuschauer keine Möglichkeit,
diese wörtliche und semantische Diskrepanz festzustellen. Sie müssten
dazu das
\href{http://webtv.un.org/media/geneva-press-briefings/watch/geneva-press-briefing-hrc-ifrc-ocha-who-iom-unhcr-ohchr-ilo-gavi/5131061146001}{75-minütige
Video} der Pressekonferenz auf der Internetseite der UNO aufsuchen.

Weitere manipulative Übersetzungen oder Zitierungen, inklusive solche
zugunsten der Konfliktpartei Syrien/ Russland, wurden nicht gefunden.
Somit ergibt sich auch in dieser Kategorie ein Propagandaeffekt
zugunsten der Konfliktpartei USA/NATO.

\textbf{\{3\} Audiovisuelle Techniken}

\textbf{\{3a\} Manipulative \emph{Verwendung} von Filmmaterial:} In den
untersuchten Beiträgen des Schweizer Radio und Fernsehens wurde ein Fall
identifiziert, in dem Filmmaterial manipulativ verwendet wurde. In der
\emph{Tagesschau-Hauptausgabe} erwähnte die Sprecherin zunächst, dass
Russland jede Verantwortung für den Angriff auf den Konvoi zurückweise,
suggerierte dann jedoch durch die Einspielung von russischen
Drohnenaufnahmen des Konvois und die Einfügung des Wortes ``aber'', dass
Russland den Konvoi womöglich ``im Visier'' gehabt habe:

\emph{(02:52) \textbf{Sprecherin}: Der Angriff soll entweder von
syrischer oder russischer Seite gekommen sein. (Pause) Für die USA ist
das klar. Das syrische Regime aber dementiert, und auch dessen
Verbündeter Russland weist jede Verantwortung zurück.'' Einspielung
russisches Drohnenvideo. \textbf{Sprecherin}: ``Das russische
Staatsfernsehen hat aber Drohnenaufnahmen des Verteidigungs­ministeriums
veröffentlicht, die den Konvoi zeigen sollen.''}

Der Effekt wird dadurch verstärkt, dass die Sprecherin nicht erwähnte,
in welchem Kontext Russland die Drohnenaufnahmen veröffentlicht hat
(Technik \{1f\}): Russland wollte damit
\href{https://sputniknews.com/middleeast/20160920/1045521488/aleppo-un-convoy.html}{dokumentieren},
dass sich bewaffnete Rebellenfahrzeuge in der Nähe des Konvois
aufgehalten haben.

\includegraphics{https://swprs.files.wordpress.com/2016/10/srf-konvoi-drohne.png?w=600\&h=337}

\emph{Russische Drohnenaufnahmen vom Konvoi (SRF)}

\textbf{\{3b\} Manipulative \emph{Bearbeitung} von Filmmaterial:} Die
manipulative \emph{Bearbeitung} von Filmmaterial konnte ebenfalls in
einem Fall festgestellt werden. Dabei handelt es sich um eine
sinnverändernde Videoschnitttechnik bei einer
\href{https://gadebate.un.org/en/71/secretary-general-united-nations}{Rede
des UNO-Generalsekretärs Ban Ki-moon}. In der
\emph{Tagesschau-Hauptausgabe} wurde gleich nach der Anmoderation ein
erster Ausschnitt aus dieser Rede eingespielt:

\emph{(01:06) Einspielung Ban Ki-moon. \textbf{Sprecher übersetzt}:
``Als wir dachten, dass es nicht mehr schlimmer werden kann, wird die
Schwelle der Verwerflichkeit noch einmal gesenkt. Der abscheuliche,
grausame und offenbar vorsätzliche Angriff auf einen Hilfskonvoi der UNO
und des Roten Halbmonds ist das jüngste Beispiel.''}

Im weiteren Verlauf des Beitrags folgen verschiedene Suggestionen, dass
die Konfliktpartei Syrien/ Russland für den genannten Angriff
verantwortlich ist:

\emph{(01:28) \textbf{Moderatorin}: ``Allerdings deutet viel daraufhin,
dass die syrische oder die russische Luftwaffe dahintersteckt}.''

\emph{(02:14) \textbf{Sprecherin}: ``Ein Konvoi der Uno und des Roten
Halbmondes wird von der Luft aus bombardiert.''}

\emph{(02:52) \textbf{Sprecherin}: ``Der Angriff soll entweder von
syrischer oder russischer Seite gekommen sein. Für die USA ist dies
klar.''}

Sodann folgt das eben erwähnte russische Drohnenvideo, bei dem
suggeriert wurde, Russland habe den Konvoi ``im Visier'' gehabt. Gleich
danach wird ein zweiter Ausschnitt aus der Rede von UNO-Generalsekretär
Ban Ki-moon eingespielt:

\emph{(03:14) \textbf{Sprecherin}: ``UNO-Generalsekretär Ban Ki-moon
äußert in New York scharfe Kritik -- nicht nur an den Kriegsparteien.''
Rede Ban Ki-moon. \textbf{Sprecher übersetzt}: ``Mächtige Gönner, die
die Kriegsmaschinerie weiter füttern, haben auch Blut an ihren
Händen.''}

Die Manipulation besteht hier darin, dass Ban Ki-Moon dieses zweite
Statement in seiner
\href{https://gadebate.un.org/en/71/secretary-general-united-nations}{Rede}
\emph{vor} dem ersten Statement gemacht hat, als er noch allgemein vom
Syrienkrieg und \emph{allen} daran Beteiligten sprach. Das Schweizer
Fernsehen hat dieses erste Statement jedoch heraus­geschnitten und nach
hinten verschoben, sodass es direkt im Anschluss an die Einspielung der
russischen Drohnen­aufnahmen vom Konvoi zu sehen ist, von dem zuvor
suggeriert wurde, Russland oder Syrien haben ihn angegriffen. Dadurch
wird der Eindruck vermittelt, der UN-General­sekretär mache hier gezielt
die Konfliktpartei Syrien/ Russland für den Angriff auf den Konvoi
verantwortlich.

\includegraphics{https://swprs.files.wordpress.com/2016/10/srf-ban-ki-moon.png?w=600\&h=337}

\emph{Rede des UNO-Generalsekretärs (SRF)}

Eine irreführende \emph{Bearbeitung} von Bild-, Ton- oder Filmmaterial
zulasten der Konfliktpartei USA/NATO konnten in den Beiträgen des SRF
nicht identifiziert werden. Insgesamt ergibt sich somit auch in dieser
Kategorie ein Propagandaeffekt zugunsten der Konfliktpartei USA/NATO.

\textbf{\{3c\} Hintergrundmusik:} Durch den Einsatz von mehr oder
weniger subtiler Hintergrundmusik bei Einspielungen kann die Stimmung
der Zuschauer je nach Bedarf positiv oder negativ beeinflusst werden. In
einer professionellen Nachrichtensendung sollte ein solcher Effekt
grundsätzlich nicht verwendet werden. Dennoch sind beim Schweizer
Fernsehen beispielsweise
\href{http://www.srf.ch/play/tv/10vor10/video/warum-assad-bleibt?id=a6d267c9-52b3-470b-868e-95bb919a0b96}{Einspielungen
von leiser Gruselmusik} bei Beiträgen über den syrischen Präsidenten
belegt.

In den hier untersuchten Beiträgen wurde hingegen keine Verwendung
manipulativer Hintergrund­musik erkannt. Dies könnte indes auch dadurch
bedingt sein, dass die Konfliktpartei Syrien/ Russland in keinem der
untersuchten Beiträge zu Wort oder ins Bild kam. Bereits am nächsten Tag
-- und damit außerhalb der vorliegenden Untersuchung -- wurde
bei­spiels­­weise in der Sendung \emph{10 vor 10} ein
\href{http://www.srf.ch/play/tv/10vor10/video/10vor10-vom-21-09-2016?id=81934d9e-0e10-4ad1-81b0-32da68a8b777\&startTime=945}{dunkler,
bedrohlicher Klang eingespielt}, als eine kurze Videosequenz aus Moskau
zu sehen war \emph{(10 vor 10 vom 21. September 2016, 15:45).}

\textbf{\{3d\} Mimik, Gestik, Intonation}: Abgesehen von der unter Punkt
\{2d\} identifizierten Intonation, durch die eine neutrale Aussage der
UNO in eine Anschuldigung an die syrische Regierung transformiert wurde,
ist in den untersuchten Beiträgen kein manipulativer Einsatz von Mimik,
Gestik oder Intonation festgestellt worden.

\textbf{\{4\} Weitere Techniken}

\textbf{\{4a\}} Eine \textbf{manipulative Berufung auf Autorität} wurde
in zwei Fällen festgestellt: Einerseits im Interview mit dem Vertreter
der deutschen \emph{Stiftung Wissenschaft und Politik}, den die
Moderatorin als ``Nahost-Experten'' bezeichnete, ohne die politische
Zugehörigkeit der Stiftung zu erwähnen. Andererseits indem Vertretern
der UNO durch Falsch­über­setzungen und Videoschnitt­techniken Aussagen
in den Mund gelegt wurden, die sie nicht getätigt hatten (siehe \{2d\}
und \{3b\}). Beide Fälle wirkten zugunsten der Konfliktpartei USA/NATO.

\textbf{\{4b\}} Ein spezifisches \textbf{Diffamieren, Diskreditieren
oder Verhöhnen} konnte in den untersuchten Beiträgen -- mit Ausnahme der
Interview-Antworten des Vertreters der deutschen \emph{Stiftung
Wissenschaft und Politik} -- nicht festgestellt werden.

\textbf{\{4c\}} Hingegen konnte ein deutliches \textbf{Idealisieren} der
Konfliktpartei USA/NATO festgestellt werden, indem diese Partei durch
Einspielungen und Kommentare der Moderatoren nicht als Konfliktpartei,
sondern als Friedenspartei dargestellt wurde, die als einzige gegen
Gewalt und für Diplomatie sei:

\begin{itemize}
\tightlist
\item
  \emph{\textbf{Sprecherin}}*: ``US-Außenminister John Kerry nimmt
  insbesondere Russland in die Pflicht.'' Einspielung US-Außen­minister.
  \textbf{Sprecherin übersetzt}: ``Nicht mit Syrien, sondern mit
  Russland haben wir die Waffenruhe vereinbart. Russland muss dem
  syrischen Machthaber Assad ultimativ auf die Finger schauen. Denn
  dieser lässt offenbar weiterhin bombardieren, selbst Hilfskonvois.''
  (TM 02:26)*
\item
  \emph{Einspielung US-Präsident. \textbf{Sprecher übersetzt}: ``In
  einem Land wie Syrien, wo kein militärischer Sieg möglich ist, hier
  müssen wir den mühsamen Weg der Diplomatie wählen, um die Gewalt zu
  stoppen.'' (TH 03:43)}
\item
  \emph{\textbf{Moderatorin}}*: (Frage an den Korrespondenten) ``Wir
  haben es im Beitrag gehört, US-Präsident Barack Obama will den Weg der
  Diplomatie weiter gehen. Wie stehen die Chancen für eine erneute
  Feuerpause aus Ihrer Sicht?'' (TH 05:57)*
\item
  \emph{\textbf{Sprecher}}*: ``Trotz allem wollen die USA und die
  Vereinten Nationen die Feuerpause wieder zum Funktionieren bringen.
  Die Waffenruhe ist nicht tot, erklärte US-Außenminister John Kerry
  nach einem Treffen im Rahmen der Syrien-Unterstützer-Gruppe in New
  York.'' (EZ 03:25)*
\end{itemize}

Ein \textbf{Bagatellisieren} konnte lediglich in den Interview-Antworten
des Vertreters der \emph{Stiftung Wissenschaft und Politik} festgestellt
werden, als es um den US-Luftangriff auf syrische Truppen ging.

\textbf{\{4d\}} \textbf{Emotionalisieren}: Das Schweizer Radio und
Fernsehen verzichtete in den untersuchten Beiträgen weitgehend darauf,
das Ereignis durch audiovisuelle oder sprachliche Techniken zusätzlich
zu emotionalisieren. Zu nennen ist allenfalls die Titelsetzung in der
\emph{Tagesschau-Hauptausgabe}, ``Grausam und verwerflich'' (siehe Punkt
\{1b\}). Ein Angriff auf einen Hilfskonvoi ist natürlich an sich schon
ein emotionales Ereignis (zerstörte Hilfsgüter, leidende Bevölkerung)
und insofern anfällig für die politische Instrumenta­lisierung. Auf
diese Möglichkeit bzw. Gefahr ging das Schweizer Fernsehen hingegen
nicht ein (siehe \{1f\} und Schluss­folgerungen).

\hypertarget{uxfcbersicht-der-verwendeten-manipulationstechniken}{%
\subparagraph{\texorpdfstring{\textbf{Übersicht der verwendeten
Manipulationstechniken}}{Übersicht der verwendeten Manipulationstechniken}}\label{uxfcbersicht-der-verwendeten-manipulationstechniken}}

Das folgende Diagramm stellt die vom Schweizer Radio und Fernsehen
verwendeten Manipulationstechniken zusammenfassend dar. Die relative
Intensität (keine bis stark) basiert auf Häufigkeit, Ausprägung und
Wirkung der jeweiligen Technik und ist als Größen­ordnung zu verstehen.
Aufgrund der konsistenten Wirkung aller Manipulationstechniken zugunsten
der Konfliktpartei USA/NATO kann von einem signifikanten
(nicht-zufälligen) Ergebnis ausgegangen werden.

\href{https://swprs.files.wordpress.com/2016/10/srf_manipulationstechniken_diagramm.png}{\includegraphics{https://swprs.files.wordpress.com/2016/10/srf_manipulationstechniken_diagramm.png?w=736\&h=438}}

\emph{Vom Schweizer Radio und Fernsehen verwendete
Manipulationstechniken}

\hypertarget{32-propagandabotschaften}{%
\paragraph{3.2. Propagandabotschaften}\label{32-propagandabotschaften}}

Im Folgenden wird dargestellt, welche der zehn eingangs präsentierten
Kriegspropaganda-Botschaften durch die untersuchten Beiträge des
Schweizer Radio und Fernsehens transportiert wurden, und in welcher Form
dies geschah.

\begin{enumerate}
\def\labelenumi{\arabic{enumi}.}
\tightlist
\item
  \textbf{Das feindliche Lager trägt die alleinige Schuld am Krieg:}
  Hierbei handelt es sich um die Hauptbotschaft, die in den untersuchten
  Beiträgen des Schweizer Radio und Fernsehens vermittelt wurde. Dies
  geschah hauptsächlich durch eine Kombination von drei Suggestionen:
  Erstens, dass die Konfliktpartei Syrien/ Russland für den Angriff auf
  den Hilfskonvoi verantwortlich sei; zweitens, dass dieser Angriff
  absichtlich erfolgt sei; und drittens, dass der Angriff auf den
  Hilfskonvoi für die Beendigung der Waffenruhe und des
  Friedensprozesses verantwortlich sei. Im Interview der Sendung
  \emph{Echo der Zeit} wurde überdies behauptet, dass die Konfliktpartei
  Syrien/ Russland die Waffenruhe vermutlich gar nie ernst gemeint habe.
  Die Botschaft der alleinigen Schuld wurde noch verstärkt, indem das
  SRF in keinem der Beiträge erwähnte, womit die syrische Regierung die
  Beendigung der Waffenruhe tatsächlich begründet hatte: Nämlich mit der
  wiederholten Verletzung der Waffenruhe durch die Rebellen und zuletzt
  durch die US-Luftwaffe.
\item
  \textbf{Wir sind unschuldig und friedliebend:} Diese Botschaft wurde
  insbesondere durch die Einspielungen des US-Präsidenten und des
  US-Außenministers transportiert, deren Aussagen von den Moderatoren
  und Sprechern des SRF noch verstärkt wurden \emph{(``Wir haben es im
  Beitrag gehört, US-Präsident Barack Obama will den Weg der Diplomatie
  weiter gehen.'')}
\item
  \textbf{Der Feind hat dämonische Züge}: Diese Botschaft wurde insofern
  transportiert, als dass suggeriert wurde, die Konfliktpartei
  Syrien/ Russland habe den Hilfskonvoi (absichtlich) bombardiert, und
  eine solche Bombardierung eine ``dämonische'' Tat darstellt (vgl. den
  Titel der Tagesschau-Hauptausgabe: \emph{``Grausam und verwerflich''})
\item
  \textbf{Wir kämpfen für eine gute Sache, der Feind für eigennützige
  Ziele:} Diese Botschaft wurde ebenfalls durch die Aussagen des
  Präsidenten und des Außenministers der USA transportiert und durch
  einige Bemerkungen der Moderatoren verstärkt: Die Konfliktpartei
  USA/NATO kämpfe für Frieden und Menschenrechte in Syrien, die
  Konfliktpartei Syrien/ Russland hingegen wolle gewaltsam eigene
  Interessen durchsetzen.
\item
  \textbf{Der Feind begeht mit Absicht Grausamkeiten, bei uns ist es
  Versehen:} Diese Botschaft wurde transportiert, indem suggeriert
  wurde, die Konfliktpartei Syrien/ Russland habe absichtlich den
  Hilfskonvoi angegriffen. Im Interview der Sendung \emph{Echo der Zeit}
  wurde zudem behauptet, beim US-Angriff auf syrische Truppen habe es
  sich um ein Versehen gehandelt.
\item
  \textbf{Der Feind verwendet unerlaubte Waffen:} Diese Botschaft wurde
  in den untersuchten Beiträgen nicht transportiert. Dies im Gegensatz
  etwa zu Beiträgen über vermutete oder angebliche Einsätze von Giftgas
  durch die syrische Regierung.
\item
  \textbf{Unsere Verluste sind gering, die des Gegners aber enorm:}
  Diese Botschaft, die eher bei einer direkten kriegerischen Begegnung
  zur Anwendung kommt, wurde ebenfalls nicht transportiert. Im Fokus der
  Beiträge standen vielmehr die hohen humanitären Verluste in Syrien.
\item
  \textbf{Unsere Sache wird von Künstlern und Intellektuellen
  unterstützt:} Diese Botschaft wurde in den untersuchten Beiträgen
  nicht direkt transportiert. Allerdings wurde versucht, die Aussagen
  von humanitären Organisationen (UNO) mittels Falsch­übersetzungen und
  anderen Techniken so zu verändern, dass sie die Konfliktpartei
  Syrien/ Russland einseitig belasten.
\item
  \textbf{Unsere Mission ist heilig:} Diese Botschaft wurde insofern
  transportiert, als dass suggeriert und auch gesagt wurde, einzig die
  Konfliktpartei USA/NATO setze sich für Frieden und Menschenrechte in
  Syrien ein (siehe Botschaft 4)
\item
  \textbf{Wer unsere Berichterstattung in Zweifel zieht, ist ein
  Verräter:} Diese Botschaft wurde in den untersuchten Beiträgen nicht
  transportiert -- mit Ausnahme des Interviewgasts in der Sendung
  \emph{Echo der Zeit}, der gegenteilige Narrative als ``syrische und
  russische Propaganda'' bezeichnete.
\end{enumerate}

Wie bereits bei den Manipulationstechniken, so fielen auch alle durch
das Schweizer Radio und Fernsehen transportierten Propagandabotschaften
zugunsten der Konfliktpartei USA/NATO und zulasten der Konfliktpartei
Syrien/ Russland aus.

\href{https://swprs.files.wordpress.com/2016/10/srf-propagandabotschaften-diagramm.png}{\includegraphics{https://swprs.files.wordpress.com/2016/10/srf-propagandabotschaften-diagramm.png?w=736\&h=416}}

\emph{Vom Schweizer Radio und Fernsehen transportierte
Propagandabotschaften}

\hypertarget{4-schlussfolgerungen}{%
\subsubsection{4. Schlussfolgerungen}\label{4-schlussfolgerungen}}

In der vorliegenden Analyse wurde erstmals systematisch die Verwendung
von Propaganda- und Manipulationstechniken in der geopolitischen
Berichterstattung des \emph{Schweizer Radio und Fernsehens} untersucht.
Dabei zeigte sich, dass solche Techniken in allen untersuchten Beiträgen
auf redaktioneller, sprachlicher und audiovisueller Ebene verwendet
wurden, und dass diese Techniken stets zugunsten der Konfliktpartei
USA/NATO ausfielen. Insgesamt muss daher von einer einseitigen,
selektiv-unkritischen und wenig objektiven Berichterstattung durch das
\emph{Schweizer Radio und Fernsehen} gesprochen werden.

Bisherige Programmanalysen des \emph{Schweizer Radio und Fernsehens}
wurden insbesondere von offiziellen Aufsichtsgremien wie der
\emph{\href{http://www.ubi.admin.ch/de/}{Unabhängigen Beschwerdeinstanz
(UBI)}} durchgeführt, sowie im Rahmen von jährlichen Beurteilungen der
landesweiten Medienqualität durch das
\emph{\href{http://www.foeg.uzh.ch/de/jahrbuch.html}{Forschungs­institut
für Öffentlichkeit und Gesellschaft (FOEG)}} der Universität Zürich.
Diese Ansätze sind indes nicht auf das Erkennen und Evaluieren von
Propaganda- und Manipulationstechniken ausgelegt und verwenden mithin
kein dafür geeignetes theoretisches Instrumentarium.

Welches sind mögliche Gründe für die einseitige und manipulative
Berichterstattung durch das \emph{Schweizer Radio und Fernsehen}? Hier
ist zunächst auf die \href{https://swprs.org/die-nzz-studie/}{bereits
veröffentlichte Untersuchung} zur geopolitischen Berichterstattung der
\emph{Neuen Zürcher Zeitung} zu verweisen, in der die politische,
ökonomische und militärische Abhängigkeit der Schweiz von der
Konfliktpartei USA/NATO und ihren Mitgliedsländern dargestellt wurde.
Diese umfassende Abhängigkeit könnte eine kritische und objektive
Berichterstattung der Konfliktpartei USA/NATO gegenüber durchaus
erschweren, zumal die Schweiz in den 1990er Jahren selbst eine
\href{https://swprs.org/nato-partnerschaft-medien/}{strategische
Partnerschaft} mit der NATO einging.

Die Ombudsstelle des \emph{Schweizer Radio und Fernsehens} betont denn
auch explizit, dass Beiträge zu internationalen Konflikten
\href{https://swprs.org/srf-ombudsstelle-im-faktencheck/}{»weder neutral
noch ausgewogen«} sein müssen, sondern lediglich »sachgerecht« -- was
erklären dürfte, weshalb entsprechende Programmbeschwerden zumeist
abgewiesen werden (siehe:
\href{https://swprs.org/srf-ombudsstelle-im-faktencheck/}{Die
SRF-Ombudsstelle im Faktencheck}).

Zu bedenken ist ferner, dass die Schweiz kein isolierter Medienraum ist.
Dies hat zur Folge, dass die (geopolitische) Berichterstattung in der
Schweiz auch in den umliegenden Ländern wahrgenommen wird -- jedenfalls
dann, wenn sie von der andernorts üblichen Berichterstattung abweichen
oder dieser gar widersprechen sollte. Dieser Effekt könnte den Druck auf
Schweizer Medien, bei geopolitischen Themen NATO-konform zu berichten,
zusätzlich erhöhen. Zwei Beispiele mögen dies illustrieren.

Carla del Ponte, die ehemalige Chef­anklägerin des Internationalen
Strafgerichtshofs und Mitglied der UNO-Kommission zur Menschenrechtslage
in Syrien, machte im Februar 2016 auf dem französischsprachigen Sender
des \emph{Schweizer Fernsehens} die
\href{http://www.rts.ch/info/monde/7480285-carla-del-ponte-l-intervention-de-la-russie-en-syrie-etait-une-bonne-chose-.html}{Aussage},
die russische Militärintervention in Syrien sei eine »gute Sache«, da
radikale Milizen zurückgedrängt würden. Diese Aussage, die dem medialen
Narrativ der Konfliktpartei USA/NATO deutlich zuwiderlief,
\href{http://deutsche-wirtschafts-nachrichten.de/2016/02/10/merkel-und-nato-im-abseits-un-kommissarin-lobt-russlands-einsatz-in-syrien/}{verbreitete}
sich insbesondere in deutschen Internetmedien und führte sogar zu einer
\href{http://www.rationalgalerie.de/schmock/tagesschau-carla-del-ponte-gilt-nicht.html}{Programmbeschwerde}
gegen die ARD (wegen Nachrichten­unter­drückung).

Auch als ein Schweizer Journalist während des Jugo­slawien­kriegs einen
Artikel zu nach­weislichen Kriegslügen der westlichen Allianz
veröffentlichte, intervenierten umgehend bekannte Medienhäuser aus
München und Berlin bei seinem Verleger. Der betreffende Journalist
erhielt in der Folge ein vorläufiges Schreibverbot und sah sich gar mit
seiner möglichen Absetzung
\href{https://swprs.org/das-gewuenschte-narrativ/}{konfrontiert}.

Überdies bestätigt die vorliegende Untersuchung die hohe Abhängigkeit
des \emph{Schweizer Radio und Fernsehens} von den
\href{https://swprs.org/der-propaganda-multiplikator/}{globalen
Nachrichtenagenturen} \emph{Associated Press} in New York,
\emph{Reuters} in London und \emph{AFP} in Paris, von denen das SRF die
Informationen für seine internationalen Berichte
\href{https://swprs.org/die-schleusenwaerter/}{hauptsächlich bezieht}.
Im Falle des Angriffes auf den Hilfskonvoi
\href{https://www.youtube.com/watch?v=RdN_siggEn0}{stammten} die
wesentlichen Bilder und Videos vor allem von der amerikanischen
\emph{AP} und der britischen \emph{Reuters}, die sich teilweise auf
anonyme ``Kriegsbeobachter''
\href{http://www.reuters.com/article/us-mideast-crisis-syria-idUSKCN11P146}{beriefen}.
Auch einige Formulierungen des \emph{SRF} stammten fast wörtlich aus
Agenturmeldungen.

Dies ist insofern problematisch, als dass diese Agenturen -- entgegen
den
\href{http://www.srf.ch/sendungen/hallosrf/warum-berichtet-srf-nicht-ueber-den-friedensmarsch-in-der-ukraine}{Darstellungen}
des \emph{Schweizer Fernsehens} und des
\emph{\href{https://swprs.org/schweizer-presserat-propaganda/}{Schweizer
Presserates}} -- über geopolitische Konflikte und Kriege im Allgemeinen
nicht unabhängig berichten können, sondern ihrerseits von der
Konfliktpartei USA/NATO unter Druck gesetzt und mit manipuliertem
Material beliefert werden, wie führende Mitarbeiter dieser Agenturen
\href{https://swprs.org/der-propaganda-multiplikator/}{bestätigt} haben.
Dessen ungeachtet übernimmt das SRF -- wie die meisten Medien -- das
Material dieser Agenturen zumeist unkritisch und ohne Kennzeichnung
(siehe
\href{https://swprs.org/der-propaganda-multiplikator/}{Vertiefungsstudie}).

Hinzu kommen personelle Aspekte: So ist etwa der Moderator der Sendung
\emph{10 vor 10} gleichzeitig ein
\href{http://www.americanswiss.org/news/arthur-honegger-spotlight/}{»Young
Leader«} der *American Swiss Foundation,~* während sich der Auslandschef
und stv. Chefredakteur des \emph{Schweizer Radios} wiederholt gegen die
Schweizer Neutralität und für einen Beitritt zur NATO
\href{https://swprs.org/2017/03/01/der-korrespondent/}{ausgesprochen}
hat. Alle diese Faktoren dürften dazu beitragen, dass das SRF, wie in
dieser Untersuchung nachgewiesen, tendenziell die Sichtweise der
Konfliktpartei USA/NATO transportiert.

Schließlich zeigt die vorliegende Untersuchung aber auch, dass der
zunehmend populäre Begriff der »Lügenmedien« nicht haltbar ist.
Einerseits ist er viel zu pauschal, da Medien im Normalfall vermeiden,
selbst zu lügen (die unkritische Verbreitung von Lügen Dritter
ausgenommen). Andererseits greift der Begriff deutlich zu kurz, da Lügen
bzw. falsche Behauptungen nur einen kleinen Teil des umfangreichen
Instrumentariums an medialen Manipulationstechniken ausmachen. Insofern
wäre wohl eher von »Manipulations­medien« zu sprechen, die jedoch
ihrerseits als Vehikel der (geo-)po­li­tischen Manipulation zu sehen
sind. Zumal klassische Medien, wie es der langjährige AP-Journalist
Herbert Altschull
\href{https://swprs.org/der-propaganda-multiplikator/}{formuliert} hat,
»in allen Pressesystemen Instrumente der politischen und
wirtschaft­lichen Macht« sind. Daran ist zu denken, auch wenn man die
Tagesschau des \emph{Schweizer Fernsehens} sieht.

\begin{center}\rule{0.5\linewidth}{\linethickness}\end{center}

\hypertarget{vergleich-manipulationstechniken-im-zdf}{%
\subsubsection{Vergleich: Manipulationstechniken im
ZDF}\label{vergleich-manipulationstechniken-im-zdf}}

Um einen Vergleichswert für die Analyse des \emph{Schweizer Radio und
Fernsehens} zu erhalten, wurden zusätzlich die gleichentags
ausgestrahlten Nachrichten­sendungen
\emph{\href{https://swprs.org/zdf-heute-journal/}{ZDF heute}} um 19 Uhr
(ZH) und \emph{\href{https://swprs.org/zdf-heute-journal/}{ZDF
heute-journal}} um 21.45 Uhr (ZHJ) untersucht (siehe
\href{https://swprs.org/zdf-heute-journal/}{Mediathek}). Die wichtigsten
Ergebnisse werden im Folgenden in kompakter Form dargestellt.

\includegraphics{https://swprs.files.wordpress.com/2016/10/zdf-sendungen.png?w=600\&h=173}

*Die Sendungen ZDF heute und ZDF heute-journal.\\
*

\textbf{\{1\} Redaktionelle Techniken}

\begin{itemize}
\tightlist
\item
  Das ZDF widmete dem Angriff auf den Hilfskonvoi ebenfalls das
  \textbf{Hauptthema} (ca. 30\% der jeweiligen Sendezeit). Auch das ZDF
  berichtete indes nicht über die
  \href{http://www.thealeppoproject.com/aleppo-conflict-timeline-2013/}{dreimonatige
  Belagerung} Aleppos im Sommer 2013 durch die Rebellen.
\item
  Auch im \emph{ZDF} kam nur die \textbf{Konfliktpartei USA/NATO} zu
  Wort; dies in Form von insgesamt vier Einspielungen des US-Präsidenten
  \emph{(ZH 04:42, 05:44; ZHJ 04:37, 05:50).}
\item
  Ebenfalls wurden -- abgesehen von der UNO -- nur \textbf{Drittquellen}
  aus dem Umfeld der Konfliktpartei USA/NATO verwendet
  (Filmeinspielungen der \emph{White Helmets} und \emph{Aleppo24}).
  Beide Quellen wurden jedoch nicht mit Namen benannt oder gar politisch
  verortet, sodass das Publikum über ihre Zugehörigkeit im Unklaren
  blieb \emph{(ZH 02:00, 02:16; ZHJ 00:57, 01:19).}
\item
  Auch das ZDF erwähnte nicht, dass die Rebellen den »erniedrigenden«
  Hilfstransport zuvor
  \href{https://web.archive.org/web/20170202013724/http://ogn.news/306/}{öffentlich
  abgelehnt hatten} (\textbf{Kontext}). Erwähnt wurde jedoch die
  Möglichkeit, dass Rebellen die humanitäre Situation für Propaganda
  verwenden könnten \emph{(ZH 03:07; ZHJ 02:37).}
\end{itemize}

\textbf{\{2\} Sprachliche Techniken}

\begin{itemize}
\tightlist
\item
  Der ZDF-Korrespondent \textbf{unterstellte} der syrischen Regierung,
  sie sei \emph{``von Anfang an gegen den Hilfstransport''} gewesen, da
  sie eine \emph{``Aushungerungs­strategie''} verfolge \emph{(ZH 03:07)}
  -- obschon die Regierung den Konvoi selbst bewilligte und dessen
  Abfertigung
  \href{http://blauerbote.com/2016/09/24/medien-verschweigen-attackierter-hilfskonvoi-startete-in-gebiet-der-syrischen-regierung/}{überwachte}.
\item
  Das ZDF verwendete im Allgemeinen eine neutrale \textbf{Wortwahl},
  allerdings auch einige tendenziöse Begriffe. In der Anmoderation des
  \emph{ZDF heute} war von Syrien als ``Bürger­kriegs­land'' die Rede
  \emph{(ZH 0:54),} womit die ausländischen Kämpfer und Interessen
  \emph{(Regime Change)} ausgeblendet werden. Die Rebellen wurden denn
  auch einmal verharmlosend ``Aufständische'' genannt \emph{(ZHJ
  01:39).} Die syrische Regierung wurde einmal abwertend als
  ``Assad-Regime'' bezeichnet \emph{(ZHJ 07:25).}
\item
  Auch das ZDF \textbf{suggerierte} wiederholt, der Angriff auf den
  Hilfskonvoi sei für das Ende des Friedens­prozesses verantwortlich
  \emph{(z.B. ZHJ 0:12, 0:30, 0:57, 03:33).} Immerhin erwähnte das ZDF
  aber auch die syrische Darstellung, wonach die Waffenruhe bereits
  zuvor mehrfach von den Rebellen gebrochen wurde \emph{(ZH 02:25; ZHJ
  03:00)}
\end{itemize}

\textbf{\{3\} Audiovisuelle Techniken}

Auch beim ZDF wurde eine \textbf{manipulative Video­bearbeitung}
identifiziert, die mit einer \textbf{manipulativen Falsch­übersetzung}
und \textbf{manipulativen Suggestionen} kombiniert wurde:

\emph{(ZHJ 05:02) Einspielung UN-General­sekretär. \textbf{Sprecher}:
``Zuvor war der scheidende UN-General­sekretär mit Syrien und seinen
Unterstützern scharf ins Gericht gegangen.''}

\emph{(05:08) Einspielung Rede Ban Ki-moon. \textbf{Sprecher übersetzt}:
``In unserer heutigen Welt kostet der Konflikt in Syrien die meisten
Menschenleben und bringt die größte Instabilität. {[}Schnitt 1{]} Die
syrische Regierung wirft weiter Fassbomben und foltert systematisch
tausende Gefangene zu Tode. Syriens mächtige Unterstützer, die seine
Kriegsmaschine am Laufen halten, haben Blut an ihren Händen.''
{[}Schnitt 2{]}}

\emph{(05:27) Einspielung von Aufnahmen der Vertreter Russlands und
Syriens im Publikum. \textbf{Sprecher}: ``Ohne Russland und Syrien zu
nennen, bezeichnete er den Angriff auf den UN-Hilfskonvoi gestern in
Syrien als eine widerwärtige Tat, die nicht ungesühnt bleiben dürfe.''}

\emph{(05:36) Einspielung Rede Ban Ki-moon. \textbf{Sprecher übersetzt}:
``Die Menschen, die da versucht haben lebensrettende Hilfe zu liefern,
sind Helden. Die, die sie bombardiert haben, Feiglinge. Sie dafür zur
Verantwortung zu ziehen, ist wichtig und notwendig.''}

In dieser Sequenz finden sich folgende Manipulationen:

\begin{itemize}
\tightlist
\item
  Im ersten Abschnitt sagt der Sprecher: \emph{``Zuvor war der
  scheidende UN-Generalsekretär mit Syrien und seinen Unterstützern
  scharf ins Gericht gegangen.''} Dies ist richtig, unterschlägt jedoch,
  dass der UN-Generalsekretär in
  \href{https://gadebate.un.org/en/71/secretary-general-united-nations}{seiner
  Rede} explizit mit sämtlichen Kriegsparteien und ihren jeweiligen
  Unterstützern ``scharf ins Gericht'' ging.
\item
  Im zweiten Abschnitt wurden deshalb folgende Worte des
  Generalsekretärs herausgeschnitten (Schnitt 1): ``\emph{There is no
  military solution. Many groups have killed many innocents.''} Auf
  Deutsch: ``Es gibt keine militärische Lösung. Viele Gruppen haben
  viele Unschuldige umgebracht.'' Durch Weglassen dieser Worte wird der
  Eindruck erzeugt, der UN-Generalsekretär habe \emph{ausschließlich}
  die syrische Regierung beschuldigt. Der Videoschnitt war für die
  Zuschauer indes nicht zu erkennen, da gleich­zeitig eine Saal-Aufnahme
  eingespielt wurde.
\item
  Sodann wurde das Verb ``foltern'' vom Sprecher ergänzt mit den Worten
  ``zu Tode'', die der UN-Generalsekretär im Original jedoch nicht
  gesagt hatte. Dadurch wurde die Aussage vom Sprecher zugespitzt.
\item
  Sodann wurde der dritte Satz so umformuliert, dass aus einer
  allgemeinen Kritik des UN-General­sekretärs eine einseitige
  Anschuldigung Syriens und seiner Verbündeten wurde. Im Original
  lautete der Satz: \emph{``Powerful patrons that keep feeding the war
  machine also have blood on their hands.''} Zu Deutsch: ``Mächtige
  Unterstützer, die die Kriegsmaschinerie weiter füttern, haben auch
  Blut an ihren Händen.'' Der Sprecher machte daraus jedoch:
  \emph{``Syriens {[}!{]} mächtige Unterstützer, die seine {[}!{]}
  Kriegsmaschine {[}!{]} am Laufen halten, haben Blut an ihren
  Händen.''}
\item
  Sodann erfolgt ein weiterer Schnitt (Schnitt 2) und die UN-Vertreter
  Syriens und Russlands werden eingeblendet. Gleichzeitig sagt der
  Sprecher: \emph{``Ohne Russland und Syrien zu nennen, bezeichnete er
  den Angriff auf den UN-Hilfskonvoi gestern in Syrien als eine
  widerwärtige Tat, die nicht ungesühnt bleiben dürfe.''} Dadurch wird
  sprachlich und visuell suggeriert, aus Sicht des UN-General­sekretärs
  seien vermutlich ``Russland und Syrien'' für den Angriff auf den
  UN-Hilfskonvoi verantwortlich.
\item
  Dieser Effekt wird noch verstärkt, indem direkt anschließend ein
  zweiter Ausschnitt aus der Rede des Generalsekretärs eingespielt wird,
  der sich spezifisch auf den Angriff bezieht. Im Original kam diese
  Sequenz jedoch erst später in der Rede, und bezog sich nicht
  spezifisch auf die Konfliktpartei Syrien/ Russland.
\end{itemize}

Im \emph{ZDF heute-journal} wurde zudem gleich zu Beginn des Beitrags
eine \textbf{emotionali­sierende Hinter­grund­musik} eingespielt, als
Bilder des zerstörten Hilfskonvois gezeigt wurden. Gleich­zeitig sprach
der Moderator mit einer betont langsamen und schweren Stimme
(\textbf{Intonation)}. Durch solche akkustischen Effekte kann die
Schock- und ggf. Propagandawirkung eines Ereignisses gesteigert werden:

\emph{(0:12) Einspielung Bilder des zerstörten Hilfskonvois. Einspielung
Hintergrundmusik. \textbf{Moderator}: ``Mindestens (Pause) 20 Menschen
(Pause) sind in diesem Angriff gestorben. Und auch (Pause) die
zerbrechliche Chance (Pause) für einen Friedensprozess (Pause) in
Syrien.''}

\textbf{\{4\} Weitere Techniken}

Schließlich wurde auch im Beitrag des ZDF eine starke
\textbf{Idealisierung der Konfliktpartei USA/NATO} festgestellt, die
jene des Schweizer Fernsehens sogar noch übertraf. Der US-Präsident habe
die ``Aura eines Stars'' \emph{(ZHJ 04:08; im Hintergrund Jubel zu
hören)} und er setze sich für Frieden, Menschenrechte und die
Bedürftigen ein \emph{(ZHJ 04:37, 05:50)} sowie für Freiheit und
Demokratie \emph{(ZHJ 06:26).} Dabei wurden auch Videoschnitt­techniken
eingesetzt (Technik \{3b\}), sodass ein Applaus, der eigentlich dem
UN-Generalsekretär galt, unmittelbar im Anschluss an einen Ausschnitt
aus der Rede des US-Präsidenten zu hören war \emph{(ZHJ 04:45).}

\hypertarget{fazit}{%
\paragraph{\texorpdfstring{\textbf{Fazit}}{Fazit}}\label{fazit}}

Manipulationstechniken zugunsten der Konfliktpartei Syrien/ Russland
wurden auch im \emph{ZDF} nicht festgestellt. Insgesamt muss somit auch
beim ZDF von einem deutlichen Propagandaeffekt zugunsten der
Konfliktpartei USA/NATO gesprochen werden. Insbesondere in den
Kategorien manipulative Bearbeitung von Filmmaterial, manipulative
Übersetzungen, manipulative Hinter­grund­musik sowie Idealisierung der
Konfliktpartei USA/NATO wurde die Propagandawirkung des \emph{Schweizer
Radio und Fernsehens} noch übertroffen.

\href{https://swprs.files.wordpress.com/2016/10/zdf-manipulationstechniken.png}{\includegraphics{https://swprs.files.wordpress.com/2016/10/zdf-manipulationstechniken.png?w=736\&h=409}}

\emph{Vom ZDF verwendete Manipulationstechniken}

\begin{center}\rule{0.5\linewidth}{\linethickness}\end{center}

\hypertarget{studie-als-pdf-herunterladen}{%
\paragraph{\texorpdfstring{\href{https://swprs.files.wordpress.com/2017/11/srf-propaganda-analyse-2016-tp.pdf}{Studie
als PDF
herunterladen}}{Studie als PDF herunterladen}}\label{studie-als-pdf-herunterladen}}

\hypertarget{swiss-policy-research}{%
\subsubsection{Swiss Policy Research}\label{swiss-policy-research}}

\begin{itemize}
\tightlist
\item
  \href{https://swprs.org/kontakt/}{Kontakt}
\item
  \href{https://swprs.org/uebersicht/}{Übersicht}
\item
  \href{https://swprs.org/donationen/}{Donationen}
\item
  \href{https://swprs.org/disclaimer/}{Disclaimer}
\end{itemize}

\hypertarget{english}{%
\subsubsection{English}\label{english}}

\begin{itemize}
\tightlist
\item
  \href{https://swprs.org/contact/}{About Us / Contact}
\item
  \href{https://swprs.org/media-navigator/}{The Media Navigator}
\item
  \href{https://swprs.org/the-american-empire-and-its-media/}{The CFR
  and the Media}
\item
  \href{https://swprs.org/donations/}{Donations}
\end{itemize}

\hypertarget{follow-by-email}{%
\subsubsection{Follow by email}\label{follow-by-email}}

Follow

\href{https://wordpress.com/?ref=footer_custom_com}{WordPress.com}.

\protect\hyperlink{}{Up ↑}

Post to

\protect\hyperlink{}{Cancel}

\includegraphics{https://pixel.wp.com/b.gif?v=noscript}
