\protect\hyperlink{content}{Skip to content}

\href{https://swprs.org/}{}

\protect\hyperlink{search-container}{Search}

Search for:

\href{https://swprs.org/}{\includegraphics{https://swprs.files.wordpress.com/2020/05/swiss-policy-research-logo-300.png}}

\href{https://swprs.org/}{Swiss Policy Research}

Geopolitics and Media

Menu

\begin{itemize}
\tightlist
\item
  \href{https://swprs.org}{Start}
\item
  \href{https://swprs.org/srf-propaganda-analyse/}{Studien}

  \begin{itemize}
  \tightlist
  \item
    \href{https://swprs.org/srf-propaganda-analyse/}{SRF / ZDF}
  \item
    \href{https://swprs.org/die-nzz-studie/}{NZZ-Studie}
  \item
    \href{https://swprs.org/der-propaganda-multiplikator/}{Agenturen}
  \item
    \href{https://swprs.org/die-propaganda-matrix/}{Medienmatrix}
  \end{itemize}
\item
  \href{https://swprs.org/medien-navigator/}{Analysen}

  \begin{itemize}
  \tightlist
  \item
    \href{https://swprs.org/medien-navigator/}{Navigator}
  \item
    \href{https://swprs.org/der-propaganda-schluessel/}{Techniken}
  \item
    \href{https://swprs.org/propaganda-in-der-wikipedia/}{Wikipedia}
  \item
    \href{https://swprs.org/logik-imperialer-kriege/}{Kriege}
  \end{itemize}
\item
  \href{https://swprs.org/netzwerk-medien-schweiz/}{Netzwerke}

  \begin{itemize}
  \tightlist
  \item
    \href{https://swprs.org/netzwerk-medien-schweiz/}{Schweiz}
  \item
    \href{https://swprs.org/netzwerk-medien-deutschland/}{Deutschland}
  \item
    \href{https://swprs.org/medien-in-oesterreich/}{Österreich}
  \item
    \href{https://swprs.org/das-american-empire-und-seine-medien/}{USA}
  \end{itemize}
\item
  \href{https://swprs.org/bericht-eines-journalisten/}{Fokus I}

  \begin{itemize}
  \tightlist
  \item
    \href{https://swprs.org/bericht-eines-journalisten/}{Journalistenbericht}
  \item
    \href{https://swprs.org/russische-propaganda/}{Russische Propaganda}
  \item
    \href{https://swprs.org/die-israel-lobby-fakten-und-mythen/}{Die
    »Israel-Lobby«}
  \item
    \href{https://swprs.org/geopolitik-und-paedokriminalitaet/}{Pädokriminalität}
  \end{itemize}
\item
  \href{https://swprs.org/migration-und-medien/}{Fokus II}

  \begin{itemize}
  \tightlist
  \item
    \href{https://swprs.org/covid-19-hinweis-ii/}{Coronavirus}
  \item
    \href{https://swprs.org/die-integrity-initiative/}{Integrity
    Initiative}
  \item
    \href{https://swprs.org/migration-und-medien/}{Migration \& Medien}
  \item
    \href{https://swprs.org/der-fall-magnitsky/}{Magnitsky Act}
  \end{itemize}
\item
  \href{https://swprs.org/kontakt/}{Projekt}

  \begin{itemize}
  \tightlist
  \item
    \href{https://swprs.org/kontakt/}{Kontakt}
  \item
    \href{https://swprs.org/uebersicht/}{Seitenübersicht}
  \item
    \href{https://swprs.org/medienspiegel/}{Medienspiegel}
  \item
    \href{https://swprs.org/donationen/}{Donationen}
  \end{itemize}
\item
  \href{https://swprs.org/contact/}{English}
\end{itemize}

\protect\hyperlink{}{Open Search}

\hypertarget{die-srf-ombudsstelle-im-faktencheck}{%
\section{Die SRF-Ombudsstelle
im~Faktencheck}\label{die-srf-ombudsstelle-im-faktencheck}}

\includegraphics{https://swprs.files.wordpress.com/2017/03/srf-ombudsstelle.png?w=736}

Die Ombudsstelle des \emph{Schweizer Radio und Fernsehens} ist die erste
Anlaufstelle für Programmbeschwerden des Publikums. Sie nimmt damit eine
wichtige Vermittlungs- und Aufsichtsfunktion wahr. Doch wie
unvoreingenommen und objektiv behandelt die Ombudsstelle Beschwerden zu
geopolitischen Themen? Um dies zu überprüfen, wurden während eines
halben Jahres alle Schlussberichte der Ombudsstelle zum Syrienkonflikt
einem Faktencheck unterzogen.

Die Untersuchung zeigt, dass dem SRF durchwegs eine einseitig
US-freundliche Berichterstattung vorgehalten wurde, während die
Ombudsstelle sämtliche Beanstandungen abwies. Dabei stellten sich jedoch
nahezu alle von SRF und Ombudsstelle angeführten Sachargumente als
unhaltbar heraus, sodass insgesamt keine der Begründungen als
stichhaltig gelten kann. Bedenklich ist zudem, dass sich die
Ombudsstelle explizit \emph{gegen} eine neutrale und ausgewogene
Berichterstattung durch das SRF aussprach und »Propaganda«
ausschließlich auf Seiten der Kritiker erkannte.

Bei geopolitischen Themen scheint die Ombudsstelle somit nicht primär
als objektive Aufsichtsinstanz zu fungieren, sondern in erster Linie als
ein meta-redaktioneller Schutzmechanismus (»Klagemauer«) zur Abwehr
oftmals berechtigter Kritik an der Berichterstattung des SRF. Mögliche
Ursachen für diesen Befund werden diskutiert.

\href{https://swprs.files.wordpress.com/2017/08/faktencheck-srf-ombudsstelle-mv.pdf}{Faktencheck
als PDF herunterladen}

(Hinweis: Bei Interesse am Beitrag bitte auf diese Seite verlinken.
Obige Zusammen­fassung und einzelne Auszüge können übernommen werden.
Keine Volltext-Kopie.)

\begin{center}\rule{0.5\linewidth}{\linethickness}\end{center}

\hypertarget{medienaufsicht-im-faktencheck-eine-analyse}{%
\subsection{Medienaufsicht im Faktencheck: Eine
Analyse}\label{medienaufsicht-im-faktencheck-eine-analyse}}

am Beispiel der SRF-Ombudsstelle

\emph{Ein Beitrag von \href{https://swprs.org/}{Swiss Propaganda
Research}}

März 2017

*»Diejenigen, die dem SRF vorwerfen, einseitig der US- und
Nato-Propaganda\\
zu erliegen, betreiben ihrerseits das Geschäft der russischen
Propaganda,\\
die heute mindestens so raffiniert ist wie weiland die sowjetische.«\\
*Aus einem
\href{https://www.srgd.ch/de/aktuelles/news/2016/11/19/rundschau-sondersendung-zum-assad-interview-beanstandet/}{Schlussbericht}
der SRF-Ombudsstelle

Inhaltsübersicht

\begin{enumerate}
\def\labelenumi{\arabic{enumi}.}
\tightlist
\item
  \protect\hyperlink{kapitel1}{Untersuchte Berichte}
\item
  \protect\hyperlink{kapitel2}{Faktencheck}
\item
  \protect\hyperlink{kapitel3}{Schlussfolgerungen}
\end{enumerate}

\hypertarget{1-untersuchte-berichte}{%
\subsubsection{1. Untersuchte Berichte}\label{1-untersuchte-berichte}}

Für den vorliegenden Faktencheck wurden zwischen September 2016 und
Februar 2017 (d.h. während eines halben Jahres) alle Schlussberichte der
Ombudsstelle zum Syrienkonflikt untersucht. Dabei handelte es sich um
insgesamt fünf Berichte zu folgenden Beanstandungen:

A) \protect\hyperlink{A}{28. September 2016}: »Unterstellung von
Giftgasangriffen der syrischen Regierung«

B) \protect\hyperlink{B}{16. November 2016}: »Unterschlagung eines
Berichts von \emph{Amnesty International} zu zivilen Opfern von
US-Luftangriffen in Syrien«

C) \protect\hyperlink{C}{19. November 2016}: »Diffamierung des syrischen
Präsidenten in der \emph{SRF Rundschau«}

D) \protect\hyperlink{D}{4. Januar 2017}: »Allgemein tendenziöse
Berichterstattung zulasten von Syrien und Russland«

E) \protect\hyperlink{E}{13. Februar 2017}: »Unterschlagung von
Augenzeugenberichten aus Aleppo, die der westlichen Darstellung der
Ereignisse widersprechen«

Wie ersichtlich kritisierten alle fünf Beanstandungen eine angeblich
einseitige Berichterstattung des SRF zugunsten der Konfliktpartei
USA/NATO bzw. zulasten der Konfliktpartei Syrien/Russland. Alle fünf
Beanstandungen wurden von der Ombudsstelle abgelehnt, zumeist deutlich.

Im Folgenden werden die entsprechenden Schlussberichte analysiert und
die seitens SRF und Ombudsstelle angeführten Argumente einem Faktencheck
unterzogen.

\hypertarget{2-faktencheck}{%
\subsubsection{2. Faktencheck}\label{2-faktencheck}}

\hypertarget{a-giftgasangriffe}{%
\paragraph{A) Giftgasangriffe}\label{a-giftgasangriffe}}

In dieser
\href{https://www.srgd.ch/de/aktuelles/news/2016/09/28/sendung-info-3-auf-radio-srf-3-uber-waffenruhe-syrien-beanstandet/}{Beanstandung}
wurde kritisiert, dass der diplomatische Korrespondent und
stellvertretende Chefredakteur von Radio SRF \emph{``zum wiederholten
Male innert kurzer Zeit die Syrischen Regierungstruppen des Einsatzes
von Chlorgas bzw. Chemiewaffen''} beschuldigt hat. Dies sei \emph{``eine
frei erfundene Falschmeldung, die jeder Beweise entbehrt''.}

\textbf{Original-Passage aus der Sendung}: ``{[}SRF-Korrespondent{]}:
``Ich würde sogar sagen, es ist noch nicht mal sicher, dass sie {[}die
lokalen Kriegsparteien{]} überhaupt an Bord sind. Die Russen behaupten
zwar, Assad sei bereit, beispielsweise seine Luftschläge in gewissen
Gebieten aufzugeben, nicht mehr weiterhin Giftgasangriffe auf seine
Bevölkerung zu verüben. Ob er das tatsächlich tut, ob er da wirklich
eingelenkt hat, das muss er noch beweisen.''

Das SRF argumentierte, diese Aussage habe sich auf einen UNO-Bericht vom
August 2016 bezogen. Die Ombudsstelle pflichtete bei: \emph{``Sie haben
sicher Recht, dass Medien keine Behauptungen aufstellen sollten, die
unbelegt sind. Aber in diesem Fall ist die Aussage, dass die syrischen
Regierungstruppen Giftgas einsetzten, durch den sehr seriösen
Uno-Bericht belegt.''} Das \emph{Schweizer Radio und Fernsehen} habe
unparteiisch berichtet, die Beanstandung könne deshalb nicht unterstützt
werden.

\textbf{Faktencheck}: Sicherlich hat der SRF-Korrespondent bei seiner
Aussage an den genannten
\href{http://daccess-ods.un.org/access.nsf/GetFile?OpenAgent\&DS=S/2016/738\&Lang=E\&Type=PDF}{UNO-Bericht}
gedacht. Dennoch sind die Aussage des Korrespondenten und die
Argumentation der Ombudsstelle in dreifacher Hinsicht problematisch:

\emph{Erstens} hat das SRF die Täterschaft Assads als Tatsache
dargestellt und nicht angemerkt, dass die Giftgas-Vorwürfe an die
syrische Regierung auf einem -- unter den Konfliktparteien weiterhin
\href{http://www.reuters.com/article/us-mideast-crisis-syria-chemicalweapons-idUSKCN1152M8}{umstrittenen}
-- UNO-Bericht beruhen. Diesen Fehler hat das SRF in seiner
Stellungnahme teilweise eingeräumt: ``\emph{Die Reaktion von Herrn X auf
die Gesprächspassage zeigt uns andererseits, dass es trotz ausführlicher
vorangegangener Berichterstattung zum Giftgas-Vorwurf vermutlich besser
gewesen wäre, bei dessen Wiederholung auch die zugrundeliegenden
Berichte nochmals zu erwähnen. Dass dies im konkreten Fall unterblieben
ist, hat allerdings damit zu tun, dass sich dieses Gespräch im Kern gar
nicht mit diesem Vorwurf befasst hat.''}

\emph{Zweitens} hat Russland entgegen der Aussage des
SRF-Korrespondenten
\href{http://www.reuters.com/article/us-mideast-crisis-syria-chemicalweapons-idUSKCN1152M8}{nicht
behauptet}, \emph{``Assad sei bereit, {[}\ldots{}{]} nicht mehr
weiterhin Giftgasangriffe auf seine Bevölkerung zu verüben''} -- denn
Russland ging gar nie davon aus, dass Assad solche Angriffe verübt hat,
noch hat Assad solche Angriffe
\href{http://www.presstv.ir/Detail/2016/09/01/482705/Syria-UN-report-chemicals}{jemals
eingeräumt}. Bei dieser Aussage handelt es sich somit um eine doppelte
Unterstellung bzw. Falschbehauptung -- selbst dann, wenn Assad
tatsächlich solche Angriffe verübt haben sollte. Verschärft wurde diese
Unterstellung noch durch den Zusatz \emph{»auf seine Bevölkerung«:}
Damit werden willkürliche Giftgas-Angriffe auf die Zivilbevölkerung
suggeriert, während selbst der UNO-Bericht von Vorfällen auf Territorium
der als Terrororganisation eingestuften \emph{Al-Nusra-Front} und
anderer Milizen
\href{http://daccess-ods.un.org/access.nsf/GetFile?OpenAgent\&DS=S/2016/738\&Lang=E\&Type=PDF}{spricht}
(S. 43ff und 76ff).

Und \emph{drittens} ist es entgegen der Darstellung von SRF und
Ombudsstelle durchaus umstritten, ob der UNO-Bericht die Täterschaft der
fraglichen Giftgasangriffe überhaupt belegt. So hat beispielsweise der
amerikanische Investigativ-Journalist Robert Parry
\href{https://consortiumnews.com/2016/09/08/un-team-heard-claims-of-staged-chemical-attacks/}{nachgewiesen},
dass die entscheidenden Passagen des
\href{http://daccess-ods.un.org/access.nsf/GetFile?OpenAgent\&DS=S/2016/738\&Lang=E\&Type=DOC}{UNO-Berichts}
vom 24. August 2016 letztlich auf Behauptungen regierungsfeindlicher
Milizen sowie Helfern aus deren Umfeld basieren (Punkte 54-57, S. 13-14,
sowie Annex IV, S. 43ff, und Annex VIII, S. 76ff). Eine Untersuchung vor
Ort habe nicht stattgefunden (Punkte 26-27, S. 24), und manche
Einschlagstellen seien nachweislich manipuliert worden (Punkt 49, S.
11). Der Bericht halte zudem fest, dass die von der UNO als
terroristische Organisation eingestufte \emph{Al-Nusra-Front} im August
2012 eine syrische Industrieanlage bei Aleppo eroberte und dabei
ihrerseits in den Besitz von 400 Tonnen Chlorgas kam, das sie teilweise
an einen unbekannten Ort abtransportierte (Punkt 40, Seite 10).

Es ist nicht bekannt, ob Redaktion und Ombudsstelle den 98-seitigen
\href{http://daccess-ods.un.org/access.nsf/GetFile?OpenAgent\&DS=S/2016/738\&Lang=E\&Type=PDF}{UNO-Bericht}
selbst gelesen haben. Er wird im Schlussbericht jedenfalls nicht zitiert
oder verlinkt. Auch über die Kritik am Bericht findet man im
Online-Archiv des SRF keine Informationen.

Bereits über die
\href{https://consortiumnews.com/2014/04/07/the-collapsing-syria-sarin-case/}{Chemiewaffen-Angriffe
mit Sarin} im Frühling und Sommer 2013 -- die beinahe zu einer direkten
militärischen Intervention von NATO-Staaten in den Syrienkrieg
\href{https://www.srf.ch/news/international/syrien-einsatz-der-usa-nimmt-erste-huerde}{geführt}
hätten -- berichtete das SRF unzulänglich: Während Anschuldigungen und
Vermutungen westlicher
\href{http://www.srf.ch/news/international/syrien-fuer-frankreich-ist-giftgas-angriff-erwiesen}{Regierungen}
und
\href{http://www.srf.ch/news/international/fuer-die-usa-ist-klar-assad-regime-steht-hinter-giftgas-einsatz}{Geheimdienste}
ausführlich thematisiert und UNO-Berichte
\href{https://www.srf.ch/news/international/syrienkonflikt-welche-partei-besitzt-fixfertiges-nervengas}{suggestiv
präsentiert} wurden, kamen
\href{http://www.nytimes.com/2013/12/29/world/middleeast/new-study-refines-view-of-sarin-attack-in-syria.html}{Gegendarstellungen}
etwa von Forschern des
\href{https://s3.amazonaws.com/s3.documentcloud.org/documents/1006045/possible-implications-of-bad-intelligence.pdf}{\emph{Massachusetts
Institute of Technology (MIT)}} oder des amerikanischen
Investigativ-Journalisten
\href{https://www.lrb.co.uk/v36/n08/seymour-m-hersh/the-red-line-and-the-rat-line}{Seymour
Hersh} nicht zur Sprache. Selbst die
\href{http://www.globalresearch.ca/turkish-police-find-chemical-weapons-in-the-possession-of-al-nusra-terrorists-heading-for-syria/5336917}{Beschlag­nahmung}von
Sarin bei *Al-Nusra-*Kämpfern in der Südtürkei im Mai 2013 wurde vom SRF
nicht erwähnt.

\textbf{Fazit:} Bei der Aussage zum Giftgaseinsatz handelte es sich zwar
nicht wie in der Beanstandung etwas salopp formuliert um eine »frei
erfundene Falschmeldung« seitens des SRF, sondern um eine Behauptung aus
einem UNO-Bericht. Das SRF stellte diese Behauptung jedoch losgelöst vom
UNO-Bericht als Tatsache dar, machte nicht auf die weiterhin mangelhafte
Beweislage aufmerksam, »erfand« bzw. unterstellte ein nicht-vorhandenes
Eingeständnis seitens der Konfliktparteien Syrien und Russland, und
suggerierte ohne Grundlage willkürliche Angriffe auf die
Zivilbevölkerung. Insgesamt kann deshalb nicht von einer
»unparteiischen« Berichterstattung durch das SRF gesprochen werden,
weshalb auch die Beurteilung der Ombudsstelle nicht haltbar ist.

\hypertarget{b-bericht-von-amnesty-international}{%
\paragraph{B) Bericht von Amnesty
International}\label{b-bericht-von-amnesty-international}}

In dieser
\href{https://www.srgd.ch/de/aktuelles/news/2016/11/17/tagesschau-beitrag-uber-ein-memorandum-von-amnesty-international-beanstandet/}{Beanstandung}
wurde kritisiert, dass die SRF-Tagesschau nicht über einen
\href{https://www.amnesty.ch/de/laender/naher-osten-nordafrika/syrien/dok/2016/zivile-opfer-luftangriffe-us-koalition-syrien}{Bericht}
von \emph{Amnesty International} berichtet hat, wonach bei
US-Luftangriffen in Syrien mindestens 300 Zivilisten ums Leben kamen und
diesbezüglich eine Untersuchung einzuleiten sei.

Das SRF argumentierte, die \emph{Tagesschau} sei an jenem Tag bereits
mit anderen Themen besetzt gewesen, die für das Schweizer Publikum
wichtiger waren als die Meldung von \emph{Amnesty International.} Dabei
handelte es sich insbesondere um Entscheide des Schweizer Bundesrates.

Die Ombudsstelle pflichtete dieser Argumentation bei. Hinzu komme, dass
die Tagesschau \emph{``grundsätzlich über Amnesty-Medienmitteilungen
eher nicht berichtet, es sei denn, es handle sich um gut belegte,
gravierende Anschuldigungen großen Ausmaßes.''} So habe die Tagesschau
zwar ausführlich über einen *Amnesty-*Bericht zu Foltervorwürfen an die
\href{https://www.amnesty.org/en/documents/mde24/4508/2016/en/}{syrische
Regierung} berichtet, nicht jedoch über *Amnesty-*Berichte zu
Gräueltaten des
\href{https://www.amnesty.org/en/latest/news/2014/09/gruesome-evidence-ethnic-cleansing-northern-iraq-islamic-state-moves-wipe-out-minorities/}{IS},
zu Vertreibungen durch
\href{https://www.amnesty.org/en/documents/mde24/2503/2015/en/}{kurdische
Kämpfer}, zu zivilen Opfern
\href{https://www.amnesty.org/en/documents/mde24/3113/2015/en/}{russischer
Luftangriffe}, sowie zu Kriegsverbrechen von
\href{https://www.amnesty.org/en/latest/news/2016/05/syria-armed-opposition-groups-committing-war-crimes-in-aleppo-city/}{Rebellen}
in Aleppo. Weiter: »*Wäre die „Tagesschau`` einseitig US-freundlich,
dann hätte sie über die Amnesty-Vorwürfe gegen Russland mit Sicherheit
berichtet. Dies alles führt mich zum Ergebnis, dass ich Ihre
Beanstandung im konkreten Fall nicht unterstützen kann.«\\
*

\textbf{Faktencheck:} Die Frage, ob es in der Tagesschau noch Sendezeit
für den *Amnesty-*Bericht gegeben hätte oder nicht, kann nicht eindeutig
beantwortet werden. Jedenfalls wurde ein knapp
\href{http://www.srf.ch/play/tv/tagesschau/video/tagesschau-vom-26-10-2016-1930?id=73af5ad6-ee40-4182-ab1a-b73af5ae3134}{dreiminütiger
Beitrag} zur beginnenden Offensive auf die irakische Stadt Mosul
gezeigt, in oder nach welchem der *Amnesty-*Bericht durchaus hätte
erwähnt werden können -- zumal es in beiden Fällen um die Kampagne gegen
den IS ging. Das überwiegend
\href{http://www.srf.ch/news/international/wirkliche-befreiung-von-mossul-hat-begonnen}{positive
Bild}, welches die Tagesschau von der Rückeroberung Mosuls durch
Koalitionstruppen zeichnete -- dies im Unterschied zum deutlich
negativen Bild, das von der Rückeroberung Aleppos durch
syrisch-russische Truppen gezeichnet wurde -- wäre dadurch freilich
etwas getrübt worden.

Eine eindeutige Antwort ergibt sich indes, wenn die fünf von der
Ombudsstelle genannten *Amnesty-*Berichte im Kontext der
SRF-Berichterstattung insgesamt betrachtet werden. Hier zeigt sich, dass
das SRF über \emph{alle} Amnesty-Mitteilungen berichtet hat -- nur über
jene zur USA \emph{nicht} (siehe
\href{http://www.srf.ch/news/international/bericht-russische-bomben-treffen-wohngebiete-in-syrien}{hier},
\href{http://www.srf.ch/sendungen/info-3/amnesty-international-macht-syrischen-kurden-schwere-vorwuerfe}{hier},
\href{http://www.srf.ch/news/international/amnesty-bericht-kurden-vertreiben-irakische-araber}{hier},
\href{http://www.srf.ch/news/international/amnesty-wirft-syrischen-rebellen-menschenrechtsverletzungen-vor}{hier}
und
\href{http://www.srf.ch/news/international/amnesty-grauenhafte-beweise-fuer-is-massaker}{hier}).
Die \emph{SRF-Tagesschau} hat hingegen über \emph{keine} der
Mitteilungen explizit berichtet -- nur über jene zur syrischen Regierung
\emph{schon} (siehe
\href{http://www.srf.ch/play/tv/tagesschau/video/18000-starben-in-syrischen-gefaengnissen?id=7302a9f0-3d4b-43fe-a61f-5f77636cbdc6}{hier}).
In beiden Fällen ergibt sich somit eine Einseitigkeit zugunsten der
Konfliktpartei USA/NATO.

Hinzu kommt, dass auch die \emph{SRF-Tagesschau} -- unabhängig von den
*Amnesty-*Mitteilungen -- selbstverständlich mehrfach über
Foltervorwürfe gegen die
\href{http://www.srf.ch/news/international/uno-in-syrien-werden-systematisch-haeftlinge-getoetet}{syrische
Regierung}, zivile Opfer
\href{http://www.srf.ch/news/international/russische-luftangriffe-verschaerfen-das-leid-in-syrien}{russischer
Luftangriffe} sowie Gräueltaten des IS berichtete -- indes kaum je über
Kriegsverbrechen der westlichen Allianz oder der Rebellen und Milizen in
Aleppo.

Selbst über eines der
\href{http://www.dailymail.co.uk/news/article-3697770/US-backed-Nour-al-Din-al-Zenki-behead-boy-accused-al-Quds-spy-Assad.html}{grausamsten
Verbrechen} -- die Enthauptung eines ca. 11-jährigen Buben durch
US-unterstützte Milizen in Aleppo im Juli 2016 -- findet sich im
SRF-Archiv keine Meldung. Dabei stammte aus dem Umfeld der Täter jener
\href{http://21stcenturywire.com/2017/01/28/child-beheading-terrorist-supporter-wins-getty-and-sunday-times-defining-images-award-2016/}{Fotograf},
der einen Monat später das fragwürdige Foto des Buben Omran aufnahm
(siehe Abschnitt C), welches gegen die syrische Regierung verwendet
wurde. Hierüber
\href{http://www.srf.ch/news/international/dieses-kind-hatte-keine-ahnung-was-mit-ihm-geschah}{berichtete}
auch das SRF wieder
\href{http://www.srf.ch/news/international/ikrk-praesident-maurer-das-ist-die-realitaet-in-syrien}{ausführlich}
-- jedoch ohne auf die bedenklichen Hintergründe von Foto und Fotograf
einzugehen.

Sogar über die
\href{http://www.thealeppoproject.com/aleppo-conflict-timeline-2013/}{mehrmonatige
Belagerung und Aushungerung Aleppos} durch die Rebellen im Sommer 2013
findet sich im Archiv des SRF keine Meldung -- dies im Gegensatz zur
Belagerung Aleppos durch die syrische Armee im zweiten Halbjahr 2016,
über die vom SRF ausführlich (und sehr kritisch) berichtet wurde.

Bezeichnend ist überdies, dass das SRF den in der Beanstandung erwähnten
*Amnesty-*Report zu zivilen Opfern von US-Luftangriffen als einzigen
unerwähnt ließ, jedoch zwei Wochen später eine
\href{http://m.srf.ch/news/international/900-is-kaempfer-seit-beginn-der-mossul-offensive-getoetet}{Mitteilung}
des US-Militärs aufschaltete, in der die zivilen Opfer heruntergespielt
und auf die Erfolge der Luftangriffe verwiesen wurde.

Schließlich erwähnte das SRF -- wie schon bei den Giftgasangriffen --
auch bei den Foltervorwürfen nicht, dass es
\href{https://consortiumnews.com/2017/02/11/amnesty-international-stokes-syrian-war/}{erhebliche
Zweifel} an Quellen und Methodik des *Amnesty-*Berichts gibt. Dabei wäre
auch gegenüber *Amnesty-*Berichten durchaus eine
\href{https://consortiumnews.com/2017/02/11/amnesty-international-stokes-syrian-war/}{gewisse
Vorsicht} angebracht, seitdem die britisch-amerikanische Organisation
1990 öffentlich
\href{https://en.wikipedia.org/wiki/Nayirah_\%28testimony\%29}{»verifizierte«},
dass irakische Soldaten kuwaitische Babys aus ihren Brutkästen gerissen
hätten -- eine Geschichte, die von einer amerikanischen PR-Firma
erfunden wurde (sog.
\href{https://de.wikipedia.org/wiki/Brutkastenl\%C3\%BCge}{»Brutkastenlüge«})
und maßgeblich zum US-Angriff auf den Irak beitrug.

\textbf{Fazit:} All dies weist darauf hin, dass das SRF im Allgemeinen
und die Tagesschau im Besonderen entgegen der Darstellung der
Ombudsstelle durchaus »\emph{einseitig US-freundlich«} über
Kriegsverbrechen und Menschenrechtsverletzungen im Syrienkonflikt
berichtet haben, sowohl betreffend der *Amnesty-*Berichte als auch
darüber hinaus. Die Begründung der Ombudsstelle ist deshalb nicht
haltbar.

\hypertarget{c-pruxe4sident-assad}{%
\paragraph{C) Präsident Assad}\label{c-pruxe4sident-assad}}

In dieser
\href{https://www.srgd.ch/de/aktuelles/news/2016/11/19/rundschau-sondersendung-zum-assad-interview-beanstandet/}{Beanstandung}
wurde kritisiert, der syrische Präsident sei in einer
\emph{Rundschau}-Diskussion von einem als Experten geladenen Studiogast
diffamiert worden: \emph{``Gleich in Ihrem ersten Satz bezeichneten Sie
Präsident Assad als ‚pathologischen Lügner`. Eine unverschämte Anmassung
und Arroganz.''} Die fragliche Diskussion folgte gleich im Anschluss an
ein viel beachtetes
\href{http://www.srf.ch/news/international/assad-ich-greife-keine-menschen-an-ich-verteidige-sie}{Interview}
des \emph{Rundschau}-Moderators mit dem syrischen Präsidenten in
Damaskus.

Studiogast, \emph{Rundschau}-Redaktion und Ombudsstelle führten in ihrer
Antwort mehrere Argumente an, um zu belegen, dass der syrische Präsident
sehr wohl ein »pathologischer Lügner« und somit nicht diffamiert worden
sei. Diese werden im Folgenden einzeln betrachtet.

**Faktencheck:\\
**

\begin{enumerate}
\def\labelenumi{\arabic{enumi}.}
\tightlist
\item
  Assad habe im
  \href{https://www.srf.ch/content/download/11266054/127635839/version/2/file/Transkript+des+ganzen+Interviews.pdf}{Interview}
  behauptet, seine Armee habe keine Chemiewaffen eingesetzt (Frage 20).
  Der Einsatz von Chlorgas sei aber durch einen
  \href{http://daccess-ods.un.org/access.nsf/GetFile?OpenAgent\&DS=S/2016/738\&Lang=E\&Type=PDF}{UNO-Bericht}
  vom August 2016 (sowie eine
  \href{http://www.ohchr.org/Documents/HRBodies/HRCouncil/CoISyria/A.HRC.27.60_Eng.pdf}{frühere
  Version} vom Oktober 2014) belegt. \textbf{Check:} Wie der
  amerikanische Investigativ-Journalist Robert Parry
  \href{https://consortiumnews.com/2016/09/08/un-team-heard-claims-of-staged-chemical-attacks/}{nachgewiesen}
  hat (siehe Abschnitt A), basieren die entscheidenden Passagen des
  98-seitigen UNO-Berichts vom 24. August 2016 letztlich auf
  Behauptungen regierungsfeindlicher Milizen sowie Helfern aus deren
  Umfeld (Punkte 54-57, S. 13-14, sowie Annex IV, S. 43ff, und Annex
  VIII, S. 76ff). Eine Untersuchung vor Ort habe nicht stattgefunden
  (Punkte 26-27, S. 24), und manche Einschlagstellen seien nachweislich
  manipuliert worden (Punkt 49, S. 11). Vielleicht setzte Assad dennoch
  Giftgas ein, aber der UNO-Bericht vermag dies nicht stichhaltig zu
  belegen. Das Argument ist somit nicht zutreffend.
\item
  Assad habe im Interview behauptet, die US-Menschenrechtsorganisation
  \emph{Amnesty International} werde von Katar finanziert. Dies sei
  falsch. \textbf{Check:} Der Rundschau-Moderator sprach in Frage 21 des
  \href{https://www.srf.ch/content/download/11266054/127635839/version/2/file/Transkript+des+ganzen+Interviews.pdf}{Interviews}
  von einem *Amnesty-*Bericht sowie von anderen »schrecklichen
  Berichten« zu Foltergefängnissen. Assad bezog sich bezüglich Katar
  offenkundig auf den bekannten
  \emph{\href{https://en.wikipedia.org/wiki/2014_Syrian_detainee_report}{Caesar-Report}}
  zu angeblichen Folteropfern. Dieser wurde tatsächlich von der
  Regierung Katars -- einem der Hauptgegner der syrischen Regierung --
  in Auftrag gegeben (wobei das SRF gemäß Online-Archiv weder über den
  Auftraggeber Katar noch über die
  \href{http://www.counterpunch.org/2016/03/04/the-caesar-photo-fraud-that-undermined-syrian-negotiations/}{gravierenden
  Mängel} des Reports berichtet hat). Das Argument ist damit nicht
  zutreffend.
\item
  Assad habe im Interview behauptet, seine Luftwaffe habe keine
  Krankenhäuser bombardiert, was nicht stimme. \textbf{Check:} Assad
  sagte im Interview, es gebe keine \emph{Strategie} der Regierung,
  Schulen und Spitäler anzugreifen, aber Fehler könnten in jedem Krieg
  passieren und müssten einzeln geprüft werden (Fragen 9, 10 und 14).
  Das Argument ist somit nicht zutreffend, obschon eine solche Strategie
  nicht ausgeschlossen werden kann (siehe aber
  \href{http://www.globalresearch.ca/a-tale-of-two-hospitals-fabricated-bombing-incident-vs-open-terrorist-targeting-of-facilities-in-aleppo/5523687}{hier},
  \href{http://www.globalresearch.ca/the-aleppo-hospital-smokescreen-covering-up-al-qaeda-massacres-in-syria-once-again/5524250}{hier}
  und
  \href{http://www.globalresearch.ca/the-wests-silence-about-the-bombing-of-russias-hospital-in-aleppo-is-shameful/5560807}{hier}).
\item
  Assad habe im Interview gesagt, in den Beständen seiner Streitkräfte
  gebe es keine »Fassbomben«. Dies sei gelogen. \textbf{Check:} Assad
  sagte, dass seine Armee die improvisierten Freifallbomben nicht wie
  westliche Medien als »Fassbomben« \emph{bezeichne} (Frage 20). Das
  Argument ist damit nicht zutreffend.
\item
  Assad hat im Interview behauptet, das bekannte Foto des verletzten
  Buben Omran sei gestellt worden (Frage 12). Dies sei unwahr, denn der
  Junge auf dem Vergleichsfoto habe abstehende Ohren, Omran jedoch
  nicht. \textbf{Check:} Im Schlussbericht wird nicht das bekannte Foto
  von Omran gezeigt, sondern eines, das ihn nur von rechts zeigt. Im
  bekannten Foto von vorne ist jedoch zu sehen, dass Omran durchaus
  leicht abstehende Ohren hat (rechts durch die Haare verdeckt). Zudem
  taucht auch Omrans Schwester in Fotos mehrerer »Rettungsaktionen« auf.
  Die Behauptung Assads wird damit nicht widerlegt, sondern
  \href{http://www.moonofalabama.org/2016/10/assad-says-the-boy-in-the-ambulance-is-fake-we-show-why.html}{eher
  erhärtet}. Auch dieses Argument ist somit nicht zutreffend.
\end{enumerate}

Ferner wurde argumentiert, der Studiogast sei nicht als »Syrien-Experte«
bezeichnet worden, sondern lediglich als »Syrien-Kenner«. Auf der
\href{http://www.srf.ch/news/international/assad-ist-nur-noch-an-der-macht-weil-er-so-brutal-ist}{Internetseite}
des \emph{Schweizer Fernsehens} trägt das Video zum Beitrag jedoch den
Titel »Das sagen Experten zum Assad-Interview« (gemeint sind Studiogast
und SRF-Korrespondent), und im Text wird der Studiogast sogar als
»Nahost-Experte« bezeichnet.

Schließlich wurde noch argumentiert, der Studiogast sei \emph{``bewusst
als pointierte Gegenstimme zur Sichtweise Präsident Assads eingeladen''}
worden. Gemäß öffentlicher
\href{http://www.srf.ch/news/international/assad-ist-nur-noch-an-der-macht-weil-er-so-brutal-ist}{Ankündigung}
der Sendung durch das SRF sollte das Interview indes \emph{»durch
Experten eingeordnet«} werden. Eine Einordnung durch Experten ist aber
etwas anderes als eine »pointierte Gegenstimme«.

Interessanterweise
\href{https://swprs.org/2017/03/01/der-kriegsreporter/}{bezeichnete}
sich der als Syrien-Experte bzw. -Kenner geladene Studiogast zu einem
früheren Zeitpunkt selbst als »Meinungsjournalist«, dem es um »gute
Geschichten« und nicht um »Neutralität« oder »objektive Beobachtung«
gehe.

\textbf{Fazit:} Die Ombudsstelle kam auf Basis obiger Argumente zum
Schluss, dass \emph{``Alle, die noch einigermassen klar sehen, wissen,
dass Präsident Assad notorisch lügt.''} Vielleicht lügt Präsident Assad
wirklich notorisch, aber die im Bericht angeführten Argumente vermögen
dies nicht zu belegen -- sie haben sich allesamt als unzutreffend
herausgestellt. Auch diese Argumentation der Ombudsstelle ist somit
nicht haltbar.

\hypertarget{d-neutrale-berichterstattung}{%
\paragraph{D) Neutrale
Berichterstattung}\label{d-neutrale-berichterstattung}}

In dieser
\href{https://www.srgd.ch/de/aktuelles/news/2017/01/04/radio-nachrichtensendung-heute-morgen-uber-syrienkonflikt-beanstandet/}{Beanstandung}
wurde kritisiert, das SRF berichte generell nicht neutral und ausgewogen
über den Syrienkonflikt, sondern einseitig zugunsten der Konfliktpartei
USA/NATO. Als konkretes Beispiel wurde genannt: \emph{``Heute Morgen
gleich nach 07:00 hiess es im Radio SRF1: ‚Der Machthaber Assad bombt
sich mit Hilfe Russlands einen Sieg gegen die Rebellen`. In diesem
einseitigen Stil läuft das schon lange so. Warum kann das Radio nicht
NEUTRAL von dieser Krise berichten?''}

Das SRF räumte zwar ein, dass die obigen Worte nicht wertfrei seien,
hielt sie jedoch für angebracht und sprach sich generell gegen eine
neutrale Berichterstattung aus:

``Herr X mahnt bei uns ausserdem eine ‚neutrale` Berichterstattung an.
Neutralität jedoch ist kein journalistisches Kriterium, sondern ein
politisches. Dazu kommt: Zwischen welchen Lagern müssten wir uns neutral
positionieren? Was wir hingegen anstreben, ist eine faktengetreue
Schilderung von Konflikten und nachvollziehbar verargumentierte
Einschätzungen.''

Die Ombudsstelle pflichtete dieser Argumentation bei:

``Sie verlangen, dass Radio SRF ausgewogen und neutral berichtet. Dass
man sich mit keiner Konfliktpartei gemein macht und zu allen Abstand
hält, versteht sich eigentlich von selber, denn Journalismus ist
Fremddarstellung von Entwicklungen, Ereignissen und Personen. Aber
neutral berichten hieße ja, dass man über keinerlei Maßstab verfügt.
{[}\ldots{}{]} Wenn eine Partei lügt und die andere die Wahrheit sagt,
dann würde die neutrale Berichterstattung erfordern, dass man beiden
Recht gibt und als Fazit zieht, dass beide die Halbwahrheit sagen. Wenn
eine Partei ein offensichtliches Verbrechen begeht und die andere
keines, dann würde die neutrale Berichterstattung erfordern, dass man
beiden Parteien Verbrechen zutraut und beiden auch friedfertiges
Verhalten. Das ist aber Unsinn: Medien müssen Verbrechen Verbrechen
nennen können. {[}\ldots{}{]} Die Devise muss sein, dass nicht neutral
und ausgewogen, sondern faktengetreu und fair berichtet wird.''

\textbf{Faktencheck}: Es ist richtig, dass das Schweizer Radio- und
Fernsehgesetz (RTVG) sowie die SRG-Konzession -- im Unterschied etwa zum
deutschen
\href{http://www.die-medienanstalten.de/fileadmin/Download/Rechtsgrundlagen/Gesetze_aktuell/19_RfAendStV_medienanstalten_Layout_final.pdf}{Rundfunkstaatsvertrag}(Art.
11) -- keine »Neutralität und Ausgewogenheit« vorschreiben, sondern
lediglich eine
\href{https://www.admin.ch/opc/de/classified-compilation/20001794/index.html}{»Sachgerechtigkeit«}
(Art. 4), die es den Zuschauern erlauben soll, »sich eine eigene Meinung
zu bilden«. Ausgenommen sind inländische Wahlkämpfe, bei denen strengere
Vorschriften gelten. Dies bedeutet indes nicht, dass das SRF nicht auch
über internationale Konflikte neutral und ausgewogen berichten könnte
oder dies wenigstens anstreben könnte. Die Ombudsstelle lehnt dies
jedoch explizit ab: Ein neutraler Journalismus verfüge über keinerlei
Maßstab und dürfe Verbrechen und Lügen nicht benennen.

Allerdings bedeutet
\href{http://www.duden.de/rechtschreibung/neutral}{»neutral«} gemäß
Duden \emph{``keiner der gegnerischen Parteien angehörend, nicht an eine
Partei, Interessengruppe gebunden; unparteiisch''.} Unparteiisch und
damit neutral ist beispielsweise ein Fußballschiedsrichter -- und von
diesem wird eben gerade erwartet, dass er Foulspiel und Tore korrekt
pfeift, egal bei welcher Mannschaft. Entgegen der Argumentation der
Ombudsstelle ist Neutralität somit keineswegs ein Hinderungsgrund,
sondern im Gegenteil eine unabdingbare Voraussetzung für eine
unvoreingenommene Darstellung von Lügen und Verbrechen.

Die Ombudsstelle möchte das Begriffspaar »neutral und ausgewogen«
hingegen durch die Begriffe »faktentreu und fair« ersetzt wissen. Dabei
bedeutet \emph{faktentreu} lediglich, keine falschen Behauptungen
aufzustellen (wobei das SRF auch dieses Kriterium nicht durchgehend
erfüllt, siehe Abschnitt A), und \emph{Fairness} ist bekanntlich
subjektiv: Auch eine einseitige Berichterstattung kann man als »fair«
empfinden -- insbesondere dann, wenn man (wie im Falle des SRF) ohnehin
keine Ausgewogenheit anstrebt. Mit dieser Argumentation wird daher eine
einseitige und tendenziöse Berichterstattung durch das SRF -- wie sie in
dieser Beanstandung kritisiert wurde -- legitimiert.

Die Ombudsstelle versuchte ihre ablehnende Haltung gegenüber einer
neutralen Berichterstattung zusätzlich mit einem Zitat des bekannten
italienischen Dichters Dante Alighieri zu begründen: »\emph{Dante
schrieb in der „Divina Comedia``, die heißesten Plätze in der Hölle
seien jenen vorbehalten, die in einem moralischen Konflikt neutral
bleiben.«} Bei diesem Zitat handelt es sich jedoch um eine
\href{http://quoteinvestigator.com/2015/01/14/hottest/}{Fälschung}, die
ausgerechnet von zwei US-Präsidenten (Theodore Roosevelt und John F.
Kennedy) verwendet wurde, um gegen Neutralität zu argumentieren. Im
Online-Lexikon \emph{Wikipedia} ist der Ausspruch in der
\href{https://en.wikiquote.org/wiki/List_of_misquotations}{Liste der
bekannten Falschzitate} aufgeführt.

Aufschlussreich ist in diesem Zusammenhang, dass der stellvertretende
Chefredakteur des Radio SRF -- der sich hier gegen eine neutrale
Berichterstattung durch das SRF aussprach -- bereits 1990 die Schweizer
Neutralität als ein \emph{``Konzept von gestern''}
\href{http://www.zeit.de/1990/44/ein-konzept-von-gestern}{bezeichnete}
und später als Chefredakteur einer Schweizer Wochenzeitschrift gemäß
eigenen Worten für den Beitritt der Schweiz zur NATO
\href{https://web.archive.org/web/20040722094101/http://www.weltwoche.ch/artikel/?AssetID=400\&CategoryID=60}{eintrat}.

\textbf{Fazit:} Die Argumentation der Ombudsstelle gegen eine neutrale
und ausgewogene Berichterstattung ist nicht haltbar. Generell stellt
sich die Frage, inwiefern eine einseitige und tendenziöse
Berichterstattung -- die die freie Meinungsbildung des Publikums
selbstredend beeinträchtigen muss -- noch als »sachgerecht« im Sinne des
Gesetzes anzusehen ist.

\hypertarget{e-augenzeugenberichte-aus-aleppo}{%
\paragraph{E) Augenzeugenberichte aus
Aleppo}\label{e-augenzeugenberichte-aus-aleppo}}

In dieser
\href{https://www.srgd.ch/de/aktuelles/news/2017/02/13/syrien-berichterstattung-von-radio-und-fernsehen-srf-beanstandet/}{Beanstandung}
wurde kritisiert, dass das SRF nicht über die
\href{http://www.francetvinfo.fr/monde/revolte-en-syrie/deputes-francais-en-syrie-de-damas-a-alep-l-ami-thierry-mariani-vient-voir-la-realite-du-terrain_2003971.html}{Augenzeugenberichte}
einer französischen Parlamentarier-Delegation aus Aleppo berichtet hat.
Diese kam zum Schluss, dass die Berichterstattung in westlichen Medien
nicht der Realität vor Ort entspreche. Insbesondere sei Aleppo viel
weniger stark zerstört worden als von westlichen Medien dargestellt.
Auch das SRF habe mitgeholfen, *``krasse Lügen in die Welt zu
setzen''.\\
*

Das SRF argumentierte hingegen: ``\emph{Im Fall von Radio SRF trifft die
Kritik von Herrn Mariani {[}dem Leiter der Delegation{]} nicht zu.''}
Denn die SRF-Journalisten seien ``\emph{stets zurückhaltend, wenn wir
über das Ausmass der Zerstörung berichten und haben bewusst den Eindruck
vermieden, Aleppo sei bloss noch ein Trümmerfeld.''}

Die Sichtung einiger SRF-Schlagzeilen widerlegt diese Darstellung jedoch
klar (Auflistung nicht abschließend):

\begin{itemize}
\tightlist
\item
  \emph{\href{http://www.srf.ch/news/international/aleppo-stoesst-die-letzten-atemzuege-aus}{Krieg
  in Syrien: «Aleppo stösst die letzten Atemzüge aus»}} «Aleppo ist
  zerstört. Die Stadt ist fast tot und stösst ihre letzten Atemzüge
  aus.»
\item
  *\href{http://www.srf.ch/news/international/vorher-nachher-wie-aleppo-zur-hoelle-wurde}{Vorher
  -- nachher: Wie Aleppo zur Hölle wurde}: ``*Die wichtigste Stadt im
  Norden Syriens ist heute belagert, gewalterschüttert. In vielen Teilen
  eine Trümmerwüste. Ein Grossteil der Bevölkerung ist geflüchtet.''
\item
  \href{http://www.srf.ch/news/international/apokalypse-in-aleppo-niemand-will-verantwortlich-sein}{Apokalypse
  in Aleppo: Niemand will verantwortlich sein}
\item
  \href{http://www.srf.ch/news/international/eines-der-letzten-spitaeler-in-aleppo-zerstoert}{Eines
  der letzten Spitäler in Aleppo zerstört}
\item
  \href{https://www.srf.ch/news/international/wie-nach-dem-zweiten-weltkrieg-in-deutschen-staedten}{Krieg
  in Syrien «Wie nach dem Zweiten Weltkrieg in deutschen Städten»}
\end{itemize}

Diese Schlagzeilen sind schon allein deshalb unzulässig, weil das SRF
hier durchwegs von »Aleppo« statt von »Ost-Aleppo« spricht, wodurch ein
falscher Eindruck der Situation vor Ort vermittelt wurde: Die syrische
Armee griff nämlich nicht Aleppo an, sondern in- und ausländische
Rebellen und Milizen in den östlichen Bezirken der Stadt, während der
\href{https://en.wikipedia.org/wiki/Battle_of_Aleppo_(2012\%E2\%80\%9316)}{Großteil
der Bevölkerung} im von der Regierung kontrollierten West-Aleppo lebte
und nicht revoltierte. Das SRF
\href{http://www.srf.ch/sendungen/heutemorgen/der-besuch-von-aleppos-buergermeister-bei-der-eu}{erfand}
-- vermutlich basierend auf Agenturmeldungen -- sogar einen
(Assad-kritischen) »Bürgermeister von Aleppo«, bei dem es sich in
Wirklichkeit um einen Vertreter der Milizen in Ost-Aleppo
\href{http://www.euronews.com/2016/12/15/tomorrow-is-too-late-mayor-of-east-aleppo-calls-on-the-eu-to-act}{handelte}.

Die Ombudsstelle hinterfragte indessen primär die Glaubwürdigkeit des
französischen Delegationsleiters: Dieser stamme vom \emph{``äussersten
rechten Flügel''} der Gaullisten und sei \emph{``auf vielfältige Weise
mit Russland verbunden''}, so über seine russische Ehefrau. \emph{``Aus
seiner Russophilie leitet sich auch seine Sympathie für den syrischen
Präsidenten Baschar al-Assad ab. Es ist deshalb problematisch,
Äusserungen Marianis als »Beweis« dafür zu nehmen, Radio und Fernsehen
SRF hätten in den letzten Jahren nicht die Wahrheit berichtet.''}

Es handelt sich hierbei um ein klassisches \emph{ad hominem}
Argumentationsmuster, bei dem eine inhaltliche Auseinandersetzung mit
dem Gesagten vermieden wird. Tatsächlich wurde Aleppo jedoch von
mehreren westlichen Journalisten besucht, die zum gleichen Schluss kamen
wie die französische Delegation: Die Berichterstattung der westlichen
Medien -- die ihre Informationen zumeist von
\href{https://swprs.org/der-propaganda-multiplikator/}{Nachrichtenagenturen}
aus in Syrien involvierten NATO-Staaten beziehen -- habe nicht der
Realität vor Ort entsprochen. Beispiele sind etwa der schwedische
Konfliktforscher
\href{https://www.heise.de/tp/features/Was-Sie-ueber-Aleppo-hoeren-ist-bestenfalls-ein-kleiner-Teil-der-Wahrheit-3610881.html}{Jan
Oberg} (``Was Sie über Aleppo hören, ist bestenfalls ein kleiner Teil
der Wahrheit''), die deutsche Journalistin
\href{http://www.n-tv.de/politik/Die-Syrer-wollen-ein-Ende-der-Kaempfe-article19323121.html}{Karin
Leukefeld}, die britische Journalistin
\href{http://21stcenturywire.com/2016/12/15/vanessa-beeley-heated-debate-on-western-medias-coverage-of-the-liberation-of-east-aleppo/}{Vanessa
Beeley}, die kanadische Publizistin
\href{http://21stcenturywire.com/2016/12/17/eva-bartlett-faces-off-with-dilly-hussain-on-syria-news-sources-and-propaganda/}{Eva
Bartlett} oder der deutsche Publizist
\href{http://www.ksta.de/politik/interview-mit-al-nusra-kommandeur--die-amerikaner-stehen-auf-unserer-seite--24802176}{Jürgen
Todenhöfer}. Solche kritischen Augenzeugen kamen im SRF jedoch nicht
oder im Falle von Karin Leukefeld nur
\href{http://www.srf.ch/news/international/hundertausende-leiden-in-aleppo-zwischen-den-fronten}{am
Rande} zu Wort.

Dennoch urteilt die Ombudsstelle: \emph{``Radio und Fernsehen SRF
bemühten sich im Laufe des Bürgerkriegs in Syrien, soweit überhaupt
möglich alle erhältlichen und als seriös einzustufenden Quellen
auszuschöpfen und vielfältig und differenziert zu berichten. Ich muss
daher Ihre Vorwürfe zurückweisen und kann Ihre Beanstandung in keiner
Weise unterstützen.''}

Diese Einschätzung kann -- wie bereits weiter oben dargestellt -- einer
Überprüfung nicht standhalten:

\begin{itemize}
\tightlist
\item
  Das SRF bezog seine Informationen hauptsächlich durch
  \href{https://swprs.org/der-propaganda-multiplikator/}{internationale
  Agenturen} von oppositionsnahen Organisationen wie dem
  \emph{»Syrischen Beobachtungszentrum für Menschenrechte«} in London,
  den westlich finanzierten \emph{Weißhelmen} oder dem oppositionellen
  \emph{Aleppo Media Center}. Dabei wurde deren Parteizugehörigkeit vom
  SRF oftmals \href{https://swprs.org/srf-propaganda-analyse/}{nicht
  deutlich gemacht}.
\item
  Kritische Stimmen und Gegendarstellungen -- sei es zu
  Giftgasangriffen, der Schlacht um Aleppo oder anderen Themen -- kamen
  hingegen kaum zu Wort, insbesondere nicht während der »heißen Phasen«
  und auf den prominenten Sendeplätzen (siehe oben).
\item
  Für »Experteneinschätzungen« wandte sich das SRF bevorzugt an die
  deutsche \emph{Stiftung Wissenschaft und Politik SWP} (siehe
  beispielsweise
  \emph{\href{http://www.srf.ch/news/international/syrien-experte-perthes-es-ist-ein-anderes-land-geworden}{hier},
  \href{https://www.srf.ch/news/international/auch-nach-aleppo-ist-der-krieg-in-syrien-nicht-vorbei}{hier},
  \href{http://www.srf.ch/news/international/syrienkonferenz-der-erste-schritt-in-richtung-frieden}{hier},
  \href{http://www.srf.ch/news/international/der-zeitplan-zum-frieden-in-syrien-ist-sehr-ambitioniert}{hier},
  \href{http://www.srf.ch/news/amp/article/1536306}{hier},
  \href{http://www.srf.ch/sendungen/echo-der-zeit/uno-stoppt-hilfsgueter-lieferungen-nach-syrien}{hier}oder
  \href{http://www.srf.ch/news/international/syrien-die-russen-haben-eine-flugverbotszone-errichtet}{hier}).}
  Diese wird jedoch von der am Syrienkrieg beteiligten deutschen
  Bundesregierung
  \href{https://spiegelkabinett-blog.blogspot.com/2014/05/stiftung-fur-wissenschaft-und-politik.html}{finanziert}
  und hat bereits 2012 zusammen mit einer US-Organisation einen
  \href{http://www.zeit.de/2012/31/Syrien-Bundesregierung}{Workshop}
  organisiert, um die Zeit nach dem Sturz der syrischen Regierung zu
  planen. Dieser Hintergrund wurde vom SRF indes nicht erwähnt.
\item
  »Propaganda« wurde vom SRF nahezu ausschließlich auf der Seite der
  Konfliktpartei Syrien/Russland (sowie des IS) verortet, aber kaum je
  auf der Seite der Konfliktpartei USA/NATO, wie eine Abfrage im
  SRF-Archiv bestätigt (siehe beispielsweise
  \href{http://m.srf.ch/news/international/wie-sich-russland-inszeniert-auch-in-syrien}{hier},
  \href{http://www.srf.ch/play/tv/10vor10/video/fokus-russischer-propaganda-einsatz-in-syrien?id=398bb4e7-b6ad-4df1-bfe4-798b40592d1a}{hier},
  \href{https://www.srf.ch/news/international/wahlen-in-einem-zerrissenen-land}{hier},
  \href{http://www.srf.ch/news/amp/article/1644822}{hier}oder
  \href{http://www.srf.ch/news/international/mittelfristig-sieht-es-schlecht-aus-fuer-assad}{hier}).
\item
  Auf Seiten der westlich unterstützten Rebellen und Milizen kamen im
  SRF selbst schwerste Kriegsverbrechen oder die mehrmonatige Belagerung
  und Aushungerung Aleppos im Jahre 2013 nicht zur Sprache (siehe
  Abschnitt B).
\end{itemize}

Auch über die geopolitischen Hintergründe des meist als vermeintlicher
\href{https://www.srf.ch/news/international/kriegsverbrechen-in-syrien-aerzte-und-pfleger-als-zielscheibe}{»Bürgerkrieg«}
dargestellten Syrienkonflikts erfuhr man beim SRF insgesamt wenig bis
nichts. Zu den weitgehend ausgeblendeten Aspekten gehören
beispielsweise:

\begin{itemize}
\tightlist
\item
  Die langjährigen
  \href{http://www.globalresearch.ca/the-u-s-carried-out-regime-change-in-syria-in-1949-and-tried-again-in-1957-1986-1991-and-2011-today/5576700}{Putsch}--
  und
  \href{http://www.globalresearch.ca/1983-cia-document-reveals-plan-to-destroy-syria-foreshadows-current-crisis/5577785}{Kriegspläne}
  des US-Geheimdienstes CIA in Syrien
\item
  Die
  \href{http://www.globalresearch.ca/we-re-going-to-take-out-7-countries-in-5-years-iraq-syria-lebanon-libya-somalia-sudan-iran/5166}{Pentagon-Liste}
  von 2001 mit den sieben anzugreifenden Ländern, darunter Syrien
\item
  Das
  \href{http://www.globalresearch.ca/decoding-the-current-war-in-syria-the-wikileaks-files/5473909}{Memo}
  des US-Botschafters von 2006 zur »Destabilisierung Syriens«
\item
  Die
  \href{http://www.globalresearch.ca/former-french-foreign-minister-the-war-against-syria-was-planned-two-years-before-the-arab-spring/5339112}{Äußerung}
  des ehemaligen französischen Außenministers Roland Dumas, wonach
  britische Kollegen bereits 2009 in Syrien \emph{»etwas vorbereiteten«}
  und \emph{»eine Invasion Syriens durch Rebellen organisierten«}.
\item
  Die auf \href{https://www.youtube.com/watch?v=aiMwqm2H8TU}{Video}
  dokumentierten, unbekannten
  \href{http://www.globalresearch.ca/daraa-2011-syrias-islamist-insurrection-in-disguise/5460547}{Scharfschützen},
  die zu Beginn der syrischen Bürgerproteste 2011 auf Demonstranten
  \emph{und} Polizisten schossen (eine auch von den Umstürzen in der
  \href{https://www.heise.de/tp/features/Woher-kamen-die-Todesschuesse-3630949.html}{Ukraine}
  ,
  \href{http://www.globalresearch.ca/unknown-snipers-and-western-backed-regime-change/27904}{Ägypten},
  \href{http://www.globalresearch.ca/unknown-snipers-and-western-backed-regime-change/27904}{Libyen}
  und
  \href{https://www.freitag.de/autoren/gela/ben-ali-was-2011-wirklich-geschah}{Tunesien}
  bekannte Eskalationsstrategie).
\item
  Die
  \href{http://www.ecowatch.com/syria-another-pipeline-war-1882180532.html}{konkurrierenden
  Pläne} Irans und Katars für Gas-Pipelines durch Syrien nach Europa,
  sowie die übergeordneten Projekte Russlands
  (\href{https://en.wikipedia.org/wiki/South_Stream}{SouthStream}), der
  EU (\href{https://de.wikipedia.org/wiki/Nabucco-Pipeline}{Nabucco})
  und der Türkei
  (\href{https://de.wikipedia.org/wiki/Transanatolische_Pipeline}{TANAP})
\item
  Die \href{https://www.youtube.com/watch?v=hIMWfJmti7E}{kolportierte
  Drohung} des damaligen französischen Präsidenten Sarkozy von 2008, er
  werde die ehemalige Kolonie Syrien »dem Erdboden gleichmachen«, wenn
  Assad den französischen Bedingungen für Gaslieferverträge nicht
  zustimmen sollte.
\item
  Der seit Jahrzehnten schwelende
  \href{https://en.wikipedia.org/wiki/Israeli_involvement_in_the_Syrian_Civil_War}{Konflikt}
  zwischen Israel und Syrien
\item
  Das 2015 publik gewordene
  \href{http://www.globalresearch.ca/defense-intelligence-agency-create-a-salafist-principality-in-syria-facilitate-rise-of-islamic-state-in-order-to-isolate-the-syrian-regime/5451216}{Memo}
  des US-Militärgeheimdienstes DIA zu den Sponsoren des IS und ihren
  Absichten in Syrien
\item
  Die mehrfach öffentlich verkündete
  \href{https://www.heise.de/tp/features/Erdogan-Aleppo-gehoert-dem-tuerkischen-Volk-3360601.html}{Absicht}
  des türkischen Präsidenten Erdogan, die ehemals osmanischen Städte
  Aleppo und Mosul zu annektieren
\item
  Die dokumentierten
  \href{http://www.globalresearch.ca/logistics-101-where-does-isis-get-its-guns/5454726}{Hauptversorgungsrouten}
  sowohl der »Rebellen« wie auch des IS über NATO-Mitglied Türkei und
  NATO-Partner Jordanien
\item
  Die durch eine
  \href{https://web.archive.org/web/20170218175638/http://www.marsecreview.com/wp-content/uploads/2015/03/PAPER-on-CRUDE-OIL-and-ISIS.pdf}{britische
  Studie} dokumentierte Hauptexportroute für IS-Erdöl über das
  Ceyhan-Terminal von NATO-Mitglied Türkei
\item
  Die von der CIA koordinierten, milliardenschweren
  \href{http://www.balkaninsight.com/en/article/making-a-killing-the-1-2-billion-euros-arms-pipeline-to-middle-east-07-26-2016}{Waffenlieferungen}
  aus Osteuropa und
  \href{http://www.globalresearch.ca/cia-ops-finally-revealed-what-the-us-ambassador-in-benghazi-was-really-doing/5483957}{Libyen}
  via Türkei, Jordanien und Saudi-Arabien nach Syrien
\end{itemize}

\textbf{Fazit:} Von einer »vielfältigen und differenzierten«
Syrien-Berichterstattung unter »Ausschöpfung aller seriösen Quellen«
durch das \emph{Schweizer Radio und Fernsehen} kann beim besten Willen
nicht gesprochen werden. Auch diese Beurteilung durch die Ombudsstelle
erweist sich somit als unhaltbar.

\hypertarget{3-schlussfolgerungen}{%
\subsubsection{3. Schlussfolgerungen}\label{3-schlussfolgerungen}}

In der vorliegenden Untersuchung wurden erstmals während eines halben
Jahres alle Schlussberichte der SRF-Ombudsstelle zum Syrienkonflikt
einem Faktencheck unterzogen. Während dem SRF durchwegs eine einseitig
US-freundliche Berichterstattung vorgehalten wurde, wies die
Ombudsstelle sämtliche Beanstandungen ab. Dabei stellten sich jedoch
nahezu alle von SRF und Ombudsstelle angeführten Sachargumente als
unhaltbar heraus, sodass insgesamt keine der Begründungen als
stichhaltig gelten kann.

Insgesamt ist den Kritikern somit zuzustimmen: Das SRF berichtete über
den Syrienkonflikt tatsächlich einseitig zugunsten der Konfliktpartei
USA/NATO. Dies äußerte sich neben einer tendenziösen Sprache, unbelegten
und falschen Behauptungen sowie unausgewogenen Drittquellen insbesondere
in einer einseitigen Themengewichtung bis hin zur vollständigen
Ausblendung von »unpassenden« Ereignissen und Sichtweisen. (Siehe auch:
\href{https://swprs.org/srf-propaganda-analyse/}{Die SRF-Studie})

Dennoch
\href{https://www.srgd.ch/de/aktuelles/news/2017/02/13/syrien-berichterstattung-von-radio-und-fernsehen-srf-beanstandet/}{identifizierte}
die Ombudsstelle »Propaganda« ausschließlich auf der Gegenseite und
\href{https://www.srgd.ch/de/aktuelles/news/2016/11/19/rundschau-sondersendung-zum-assad-interview-beanstandet/}{warf}
Kritikern der SRF-Berichterstattung vor, ihrerseits das »Geschäft der
russischen Propaganda« zu betreiben. Gleichzeitig sprach sich die
Ombudsstelle jedoch explizit \emph{gegen} eine neutrale und ausgewogene
Berichterstattung durch das \emph{Schweizer Radio und Fernsehen} aus.

Wie können diese Befunde erklärt werden? Einerseits fällt auf, dass die
Ombudsstelle die Berichterstattung des SRF oftmals anhand von
Informationen aus dieser Berichterstattung selbst sowie aus anderen
\href{https://swprs.org/medien-navigator/}{NATO-affinen} Medien zu
beurteilen scheint, und weniger anhand von Primärquellen oder
Gegendarstellungen. Dadurch kann es zu Zirkelschlüssen kommen, bei denen
das SRF sich letztlich selbst bestätigt.

Andererseits könnte auch die Proximität der SRG zur Schweizer
Bundesregierung eine Rolle spielen: Diese
\href{https://swprs.org/2017/03/01/srg-idee-suisse/}{definiert} nicht
nur die Sendekonzession und mehrere Verwaltungsratsmitglieder der SRG,
sondern ebenso alle Mitglieder der obersten Programmaufsicht
(\href{https://www.ubi.admin.ch/}{UBI}), die der Ombudsstelle
übergeordnet ist (und deren Präsident der aktuelle Leiter der
Ombudsstelle zuvor war, während der vormalige Leiter zuvor Sprecher der
Bundesregierung war).

Die Schweizer Regierung ist ihrerseits mit der Konfliktpartei USA/NATO
eine strategische Militärpartnerschaft eingegangen (die sogenannte
\emph{\href{https://www.eda.admin.ch/eda/de/home/aussenpolitik/internationale-organisationen/nato-partnerschaftfuerdenfrieden.html}{»Partnership
for Peace«}}) und hat sich zudem an den wirtschaftlichen und politischen
\href{https://www.nzz.ch/bundesrat_beschliesst_sanktionen_gegen_syrien-1.10621855}{Sanktionen}
gegen Syrien (und Russland) beteiligt. Eine kritische Berichterstattung
durch das SRF würde diese unter neutralitätspolitischen Aspekten nicht
unproblematische Politik womöglich infrage stellen und die
Bundesregierung dadurch innen- und außenpolitisch unter Druck setzen.
Dies dürfte wenig opportun sein.

Bei geopolitischen Themen scheint die Ombudsstelle somit nicht primär
als objektive Aufsichtsinstanz zu fungieren, sondern in erster Linie als
ein meta-redaktioneller Schutz­mechanismus
(\href{http://www.werbewoche.ch/medien/2016-03-29/srg-ombudsstelle-sieht-sich-als-klagemauer}{»Klagemauer«})
zur Abwehr oftmals berechtigter Kritik an der SRF-Bericht­erstattung
seitens der Zuschauer und Zuhörer. Von diesen dürften sich indes nicht
wenige unter \emph{Service Public} etwas anderes vorstellen.

\begin{center}\rule{0.5\linewidth}{\linethickness}\end{center}

\hypertarget{swiss-policy-research}{%
\subsubsection{Swiss Policy Research}\label{swiss-policy-research}}

\begin{itemize}
\tightlist
\item
  \href{https://swprs.org/kontakt/}{Kontakt}
\item
  \href{https://swprs.org/uebersicht/}{Übersicht}
\item
  \href{https://swprs.org/donationen/}{Donationen}
\item
  \href{https://swprs.org/disclaimer/}{Disclaimer}
\end{itemize}

\hypertarget{english}{%
\subsubsection{English}\label{english}}

\begin{itemize}
\tightlist
\item
  \href{https://swprs.org/contact/}{About Us / Contact}
\item
  \href{https://swprs.org/media-navigator/}{The Media Navigator}
\item
  \href{https://swprs.org/the-american-empire-and-its-media/}{The CFR
  and the Media}
\item
  \href{https://swprs.org/donations/}{Donations}
\end{itemize}

\hypertarget{follow-by-email}{%
\subsubsection{Follow by email}\label{follow-by-email}}

Follow

\href{https://wordpress.com/?ref=footer_custom_com}{WordPress.com}.

\protect\hyperlink{}{Up ↑}

Post to

\protect\hyperlink{}{Cancel}

\includegraphics{https://pixel.wp.com/b.gif?v=noscript}
