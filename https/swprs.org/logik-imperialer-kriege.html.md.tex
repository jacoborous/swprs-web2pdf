\protect\hyperlink{content}{Skip to content}

\href{https://swprs.org/}{}

\protect\hyperlink{search-container}{Search}

Search for:

\href{https://swprs.org/}{\includegraphics{https://swprs.files.wordpress.com/2020/05/swiss-policy-research-logo-300.png}}

\href{https://swprs.org/}{Swiss Policy Research}

Geopolitics and Media

Menu

\begin{itemize}
\tightlist
\item
  \href{https://swprs.org}{Start}
\item
  \href{https://swprs.org/srf-propaganda-analyse/}{Studien}

  \begin{itemize}
  \tightlist
  \item
    \href{https://swprs.org/srf-propaganda-analyse/}{SRF / ZDF}
  \item
    \href{https://swprs.org/die-nzz-studie/}{NZZ-Studie}
  \item
    \href{https://swprs.org/der-propaganda-multiplikator/}{Agenturen}
  \item
    \href{https://swprs.org/die-propaganda-matrix/}{Medienmatrix}
  \end{itemize}
\item
  \href{https://swprs.org/medien-navigator/}{Analysen}

  \begin{itemize}
  \tightlist
  \item
    \href{https://swprs.org/medien-navigator/}{Navigator}
  \item
    \href{https://swprs.org/der-propaganda-schluessel/}{Techniken}
  \item
    \href{https://swprs.org/propaganda-in-der-wikipedia/}{Wikipedia}
  \item
    \href{https://swprs.org/logik-imperialer-kriege/}{Kriege}
  \end{itemize}
\item
  \href{https://swprs.org/netzwerk-medien-schweiz/}{Netzwerke}

  \begin{itemize}
  \tightlist
  \item
    \href{https://swprs.org/netzwerk-medien-schweiz/}{Schweiz}
  \item
    \href{https://swprs.org/netzwerk-medien-deutschland/}{Deutschland}
  \item
    \href{https://swprs.org/medien-in-oesterreich/}{Österreich}
  \item
    \href{https://swprs.org/das-american-empire-und-seine-medien/}{USA}
  \end{itemize}
\item
  \href{https://swprs.org/bericht-eines-journalisten/}{Fokus I}

  \begin{itemize}
  \tightlist
  \item
    \href{https://swprs.org/bericht-eines-journalisten/}{Journalistenbericht}
  \item
    \href{https://swprs.org/russische-propaganda/}{Russische Propaganda}
  \item
    \href{https://swprs.org/die-israel-lobby-fakten-und-mythen/}{Die
    »Israel-Lobby«}
  \item
    \href{https://swprs.org/geopolitik-und-paedokriminalitaet/}{Pädokriminalität}
  \end{itemize}
\item
  \href{https://swprs.org/migration-und-medien/}{Fokus II}

  \begin{itemize}
  \tightlist
  \item
    \href{https://swprs.org/covid-19-hinweis-ii/}{Coronavirus}
  \item
    \href{https://swprs.org/die-integrity-initiative/}{Integrity
    Initiative}
  \item
    \href{https://swprs.org/migration-und-medien/}{Migration \& Medien}
  \item
    \href{https://swprs.org/der-fall-magnitsky/}{Magnitsky Act}
  \end{itemize}
\item
  \href{https://swprs.org/kontakt/}{Projekt}

  \begin{itemize}
  \tightlist
  \item
    \href{https://swprs.org/kontakt/}{Kontakt}
  \item
    \href{https://swprs.org/uebersicht/}{Seitenübersicht}
  \item
    \href{https://swprs.org/medienspiegel/}{Medienspiegel}
  \item
    \href{https://swprs.org/donationen/}{Donationen}
  \end{itemize}
\item
  \href{https://swprs.org/contact/}{English}
\end{itemize}

\protect\hyperlink{}{Open Search}

\hypertarget{die-logik-imperialer-kriege}{%
\section{Die Logik
imperialer~Kriege}\label{die-logik-imperialer-kriege}}

\textbf{Publiziert}: Mai 2018\\
\textbf{Languages}:
\href{https://swprs.org/us-foreign-policy/}{English},
\href{https://swprs.files.wordpress.com/2019/12/logic-of-us-foreign-policy-spanish.pdf}{Spanish},
\href{https://swprs.org/the-logic-of-us-foreign-policy-arabic/}{Arabic},
\href{https://swprs.org/us-foreign-policy-hebrew/}{Hebrew},
\href{https://swprs.files.wordpress.com/2019/12/logic-of-us-foreign-policy-persian.pdf}{Persian}*\\
*

Wie lassen sich die amerikanischen Kriege der letzten Jahrzehnte
rational erklären? Die folgende Analyse zeigt anhand des Modells der
Professoren David Sylvan und Stephen Majeski, dass diese Kriege auf
einer eigenen, genuin imperialen Handlungslogik basieren. Eine besondere
Rolle kommt dabei dem traditionellen Mediensystem zu.

\href{https://swprs.files.wordpress.com/2018/05/logik-imperialer-kriege-spr.png}{\includegraphics{https://swprs.files.wordpress.com/2018/05/logik-imperialer-kriege-spr.png?w=736\&h=512}}~\href{https://swprs.files.wordpress.com/2018/05/logik-imperialer-kriege-spr.png}{\emph{Vergrößern}
🔎}

Aufgrund ihrer ökonomischen und militärischen Vormachtstellung nehmen
die USA seit dem Zweiten Weltkrieg und insbesondere seit 1990 die Rolle
eines modernen \textbf{Imperiums} ein. Hieraus ergibt sich für ihre
Außenpolitik eine eigene, genuin imperiale Handlungslogik (siehe obige
Abbildung).

Die zentrale Unterscheidung (\textbf{Nr. 1}) aus Sicht eines Imperiums
ist dabei jene in \textbf{Klientel- und Nicht-Klientelstaaten}. Der
Begriff des
\href{https://en.wikipedia.org/wiki/Client_state}{Klientelstaates}
stammt aus der Zeit des Römischen Reiches und bezeichnet Staaten, die
sich grundsätzlich selbst verwalten, ihre Außen- und Sicherheitspolitik
aber am Imperium ausrichten und ihre Regierungsnachfolge mit diesem
abstimmen.

Bei \textbf{bestehenden Klientelstaaten} (linke Seite des Diagramms)
geht es aus imperialer Sicht entweder um die Routine-Verwaltung
(\textbf{B} -- bspw. Schweiz und Österreich), eine militärische oder
nicht-militärische (z.B. ökonomische) Unterstützung (\textbf{D bis I} --
bspw. Kolumbien und Pakistan), oder aber um den Versuch, inakzeptable
Klientelregierungen demokratisch oder militärisch zu ersetzen
(\textbf{A} -- bspw. Griechenland 1967, Chile 1973, ev. auch Deutschland
2005 und Türkei 2016). In gewissen Fällen kann sich eine
Klientelregierung trotz imperialer Unterstützung nicht mehr an der Macht
halten und muss fallen­gelassen bzw. der Klientelstaat aufgegeben werden
(\textbf{C, F, G} -- bspw. Südvietnam 1975 oder Iran 1979).

Bei \textbf{Nicht-Klientelstaaten} (rechte Seite des Diagramms) ergibt
sich eine andere Ausgangslage. Gerät eine Region neu in den
Einflussbereich des Imperiums, so wird es zunächst versuchen, die
entsprechenden Staaten auf friedliche Weise als Klientelstaaten zu
erwerben (\textbf{J}). Dies war beispielsweise der Fall in Osteuropa und
dem Baltikum nach 1990.

\href{https://swprs.files.wordpress.com/2018/05/nato_map_final.png}{\includegraphics{https://swprs.files.wordpress.com/2018/05/nato_map_final.png?w=650\&h=416}}

*Die Ost-Erweiterung der NATO
(\href{https://www.cfr.org/backgrounder/north-atlantic-treaty-organization-nato}{CFR/Nato})\\
*

Weigert sich ein Staat hingegen, Klientelstaat zu werden, so gerät er
früher oder später zum \textbf{Feindstaat}, da er den Hegemonialanspruch
des Imperiums \emph{allein durch seine Unabhängigkeit und
Eigenständigkeit} in Frage stellt und damit die innere und äußere
Stabilität des Imperiums bedroht. Denn ein Imperium, das seinen
Hegemonialanspruch nicht mehr durchsetzen kann, zerfällt. Auf diese
Weise geraten die meisten Imperien in einen beinahe unvermeidlichen
Expansionszwang, dem sich selbst grundsätzlich friedliche Staaten nicht
entziehen können.

Bei Feindstaaten muss das Imperium zunächst entscheiden, ob eine
\textbf{militärische Aktion} erfolgsversprechend ist oder nicht
(\textbf{Nr. 11}). Falls nicht, wird das Imperium möglicherweise
Verhandlungen aufnehmen und je nach Erfolgsaussicht entweder den
Feindstatus beenden (\textbf{K}) oder aber \textbf{Sanktionen} verhängen
bzw. einen (zivilen) \textbf{Regimewechsel} anstreben (\textbf{L}).

Typische Beispiele hierfür sind derzeit etwa der Iran, Nordkorea,
Russland und zunehmend China. Nicht zufällig sind dies meist Staaten,
die Nuklearwaffen besitzen oder anstreben, denn nur damit lässt sich die
entscheidende Weiche Nr. 11 nachhaltig von militärischen auf
nicht-militärische Szenarien umlegen. Wichtig ist zudem die
Verfügbarkeit von essentiellen Rohstoffen wie Öl und Gas, da sich
ansonsten die eigene Unabhängigkeit längerfristig nicht aufrecht
erhalten lässt.

Bei den \textbf{Rohstoffen} geht es folglich nicht primär darum, dass
das Imperium diese unmittelbar besitzen möchte -- selbst Feindstaaten
wie die frühere UdSSR, Russland, Iran, Libyen oder Venezuela haben ihre
Rohstoffe stets an oder in das Imperium verkauft -- sondern darum, dass
Rohstoffe den Feind­staa­ten Unabhängigkeit und Einfluss verleihen, was
aus imperialer Sicht eine Bedrohung darstellt.

Beurteilt das Imperium eine militärische Aktion hingegen als
erfolgsversprechend, so stellt sich als nächstes die Frage, ob der
Feindstaat bzw. seine Regierung \textbf{internationale Legitimität}
besitzt oder nicht (\textbf{Nr. 13}). Im ersten Fall wird das Imperium
eine verdeckte feindliche Intervention vorbereiten, im zweiten Fall ist
eine offene feindliche Intervention möglich. Dabei kann die
autokratische Regierungsform vieler Feindstaaten genutzt werden, um
ihnen die internationale Legitimität abzusprechen.

\href{https://swprs.files.wordpress.com/2018/05/nato-partnerships.png}{\includegraphics{https://swprs.files.wordpress.com/2018/05/nato-partnerships.png?w=650\&h=459}}

*Libyen und Syrien/Libanon waren die letzten Mittelmeerländer, die nicht
Mitglied der NATO-Mittelmeer-Partnerschaft (rot) waren und stattdessen
eine eigene Regionalpolitik verfolgen wollten.
(\href{https://www.nato.int/cps/ua/natohq/topics_81136.htm}{Nato})\\
*

Zu den \textbf{verdeckten feindlichen Interventionen} zählen
insbesondere der Staatsstreich (\textbf{M} -- z.B. Iran 1953, Ägypten
1956) sowie die verdeckte Unterstützung von Rebellen (\textbf{N} -- z.B.
Afghanistan 1979ff) oder Exilgruppen (\textbf{O} -- z.B. Kuba 1961ff).
Es sind dies klassische Geheimdienstoperationen.

Bei den \textbf{offenen feindlichen} \textbf{Interventionen} wird
zunächst geprüft, ob sich der Feindstaat bereits in einem Konflikt
befindet, ob lokale Aufständische vorhanden sind, und ob eigene
Bodentruppen erforderlich sind. Je nach Szenario kommt es in der Folge
zu asymmetrischen (Luft-)Angriffen (\textbf{Q} -- z.B. Serbien 1999), zu
einer Unterstützung von Rebellen (\textbf{R} -- z.B. Syrien 2011ff), zu
einer gezielten Invasion (\textbf{S} -- z.B. Irak 2003), oder zu einem
umfassenden Krieg (\textbf{P} -- z.B. Deutschland 1941-45, Korea
1950-51).

Die imperiale Handlungslogik ist grundsätzlich \textbf{unabhängig von
der jeweils amtierenden US-Regierung}. Verschiedene Regierungen können
jedoch zu unterschiedlichen Einschätzungen gelangen bezüglich der
Erfolgsaussicht militärischer Aktionen (Nr. \textbf{11}) und
diplomatischer Verhandlungen (Nr. \textbf{12}), der Vorteile offener
\emph{versus} verdeckter Operationen (Nr. \textbf{13}), der Akzeptanz
und Bedeutung bestehender Klientelregierungen (Nr. \textbf{2}), sowie
der politischen Unterstützung für militärische Eingriffe (Szenario
\textbf{E}).

Aus der dargestellten Logik ergeben sich zugleich die wichtigsten
geopolitischen Funktionen \textbf{imperial orientierter Medien}: Es sind
dies insbesondere das Delegitimieren von Feindstaaten bzw. deren
Regierungen (Nr. \textbf{13}), das Unterstützen offener und das
Ausblenden verdeckter feindlicher Operationen (Nr. \textbf{14 bis 18}),
das Rechtfertigen von Sanktionen und Regimewechseln (Szenario
\textbf{L}), sowie die Mithilfe bei der imperialen Führung bzw.
Absetzung von Klientelregierungen (Szenario \textbf{A}).

Durch das umfangreiche Medienangebot im Internet wird die einheitliche
mediale Darstellung solcher Interventionen indes zunehmend erschwert. Es
ist dies eine neue Entwicklung, deren Auswirkungen auf die imperiale
Politik noch nicht absehbar sind.

*US-General Wesley Clark: »Sieben Länder in fünf Jahren«
(\href{https://www.democracynow.org/2007/3/2/gen_wesley_clark_weighs_presidential_bid}{DN
2007}).\\
Clark war NATO-Oberbefehlshaber zur Zeit des Kosovo-Krieges.\\
*

\hypertarget{traditionelle-erkluxe4rungen}{%
\paragraph{Traditionelle
Erklärungen}\label{traditionelle-erkluxe4rungen}}

Die \emph{Logik Imperialer Kriege} von Sylvan und Majeski bietet einen
konsistenten Erklärungsansatz für die amerikanischen Interventionen der
letzten Jahrzehnte. Die üblichen Begründungen -- von Befürwortern wie
von Gegnern dieser Kriege -- sind indes zumeist als Vorwände,
Rationalisierungen oder allenfalls Teilaspekte zu sehen, wie die
folgende Übersicht zeigt.

\begin{enumerate}
\def\labelenumi{\arabic{enumi}.}
\tightlist
\item
  \textbf{Verteidigung von Demokratie und Menschenrechten:} Diese
  klassische Begründung ist wenig überzeugend, da je nach Bedarf
  demokratische Regierungen gestürzt (\textbf{A, M, N}), Autokraten
  unterstützt (\textbf{E} und \textbf{I}), sowie Menschen- und
  Völkerrecht verletzt bzw. Verletzungen toleriert werden.
\item
  \textbf{Bekämpfung von Terrorismus:} Paramilitärische Gruppierungen --
  inklusive islamistischer Organisationen -- werden seit Jahrzehnten für
  die Beseitigung gegnerischer Regime
  \href{https://de.wikipedia.org/wiki/Operation_Cyclone}{eingesetzt}
  (\textbf{N} und \textbf{R}).
\item
  \textbf{Spezifische Bedrohungen oder Aggressionen:} Die meisten dieser
  Szenarien stellten sich im Nachhinein als unzutreffend heraus
  (\textbf{Nr. 13}; bspw. Tonkin-, Brutkasten- und WMD-Behauptungen).
\item
  \textbf{Rohstoffe (insb. Öl und Gas):} Selbst Feindstaaten möchten
  ihre Rohstoffe im Allgemeinen in den Westen verkaufen, werden indes
  mittels Sanktionen oder Krieg daran gehindert. Denn durch die
  Rohstoffe erlangen diese Länder einen Grad an Unabhängigkeit und
  Einfluss, der sie aus imperialer Sicht zur Bedrohung werden lässt
  (\textbf{Nr. 1,} bspw. Russland, Iran, Libyen).

  \begin{enumerate}
  \def\labelenumii{\arabic{enumii}.}
  \tightlist
  \item
    \textbf{Ging es im Irakkrieg um das Erdöl?} Kaum. Der Irak lieferte
    sein Erdöl bereits zuvor hauptsächlich nach Europa; der irakische
    Erdölsektor wurde nach dem Krieg zudem
    \href{https://theconversation.com/iraq-what-happened-to-the-oil-after-the-war-62188}{nicht}
    privatisiert, und die Förderlizenzen wurden auch an Konzerne in
    Frankreich, Russland und China
    \href{https://www.reuters.com/article/us-iraq-oil-contracts/oil-companies-temper-iraqs-dreams-of-production-expansion-idUSKCN1GQ1ID}{vergeben}.
  \item
    \textbf{Ging es im Syrienkrieg um Erdgas-Pipelines?} Nein (siehe
    \href{https://truthout.org/articles/the-war-against-the-assad-regime-is-not-a-pipeline-war/}{hier}
    und
    \href{https://www.middleeasteye.net/big-story/pipelineistan-conspiracy-war-syria-has-never-been-about-gas}{hier}).
    Die Umsturz- und Kriegspläne gegen Syrien bestanden seit
    \href{https://www.globalresearch.ca/syrian-regime-change-a-70-year-project/5636433}{Jahrzehnten}
    und sollten im Zuge des sogenannten »Arabischen Frühlings« umgesetzt
    werden (vgl. \href{https://youtu.be/flaqLAp0Yp4?t=1674}{Kommentar}
    des syrischen Präsidenten).
  \item
    \textbf{Ging es im Afghanistankrieg um eine Erdgas-Pipeline?}
    \href{https://slate.com/culture/2001/12/is-the-afghan-war-about-an-oil-pipeline.html}{Nein}.
    Die Taliban waren und sind an der
    \href{https://en.wikipedia.org/wiki/Turkmenistan\%E2\%80\%93Afghanistan\%E2\%80\%93Pakistan\%E2\%80\%93India_Pipeline}{TAPI-Pipeline}
    interessiert, akzeptierten jedoch die militärischen und politischen
    Forderungen der USA nicht, etwa die Bildung einer neuen
    pro-westlichen Einheitsregierung.
  \item
    \textbf{Ging es im Libyenkrieg um die Erdölreserven?} Nein. Libyen
    war bereits unter Gaddafi einer der wichtigsten Lieferanten Europas;
    die Versorgungssicherheit hat seither deutlich
    \href{http://www.businessinsider.com/r-how-unstable-is-libyas-oil-production-2018-3}{abgenommen}.
    Libyen verfolgte jedoch (aufgrund seines Ölreichtums) eine
    \href{https://globalresearch.ca/libya-a-war-on-africa/26474}{eigenständige}
    Afrika-Politik, die mit den Plänen der USA (und Frankreichs)
    kollidierte.
  \item
    \textbf{Ging es beim versuchten Regimewechsel in Venezuela ums
    Erdöl?} \href{https://swprs.org/venezuela-erdoel/}{Nein}. Die USA
    waren bereits zuvor Hauptabnehmer des venezuelanischen Erdöls.
    Venezuela ist jedoch Partner von Russland und China und unsterstützt
    weitere US-kritische Staaten in Lateinamerika.
  \item
    \textbf{Ging es beim iranischen Regimewechsel 1953 um die
    Verstaatlichung des Erdöls?} Nein. Die USA versuchten im
    britisch-iranischen Ölstreit zu vermitteln und drängten die Briten
    zu einem
    \href{https://www.heise.de/tp/features/Die-Mossadegh-Legende-4593236.html?seite=all}{Kompromiss}.
    Erst als der iranische Premier Mossadegh mit der kommunistischen
    Tudeh-Partei kooperierte und das Land gegenüber der Sowjetunion
    öffnete, intervenierte die CIA. Das iranische Erdöl blieb jedoch
    auch nach dem Putsch
    \href{https://www.nationalreview.com/2015/07/what-really-happened-shahs-iran/}{verstaatlicht}.
  \item
    \textbf{Könnten erneuerbare Energien die Rohstoff-Problematik
    lösen?} Kaum, denn erneuerbare Energien, Speichertechnologien und
    High-Tech-Elektronik benötigen
    \href{https://de.wikipedia.org/wiki/Metalle_der_Seltenen_Erden}{Seltenerdmetalle},
    die derzeit zu 97\% von China gefördert werden, und
    Konfliktmineralien wie
    \href{https://de.wikipedia.org/wiki/Coltan}{Coltan} aus dem Kongo.
  \end{enumerate}
\item
  \textbf{Der »Petro-Dollar«:} Diese
  \href{https://foreignpolicy.com/2009/10/07/debunking-the-dumping-the-dollar-conspiracy/}{These}
  entstand im Zuge des Irakkriegs. Die Bedeutung des US-Dollars ergibt
  sich jedoch nicht aus dem Erdöl, sondern aus der Wirtschaftsmacht der
  USA; dass insb. kleinere Staaten für ihre Rohstoffexporte den stabilen
  Dollar nutzen, ist naheliegend. Feindstaaten müssen indes oft auf
  andere Währungen
  \href{https://www.nachdenkseiten.de/?p=44020}{ausweichen}, um
  Sanktionen zu umgehen (\textbf{L,} bspw. Iran).
\item
  \textbf{Kapitalismus:} Lenin bezeichnete 1917 den
  \href{https://de.wikipedia.org/wiki/Der_Imperialismus_als_h\%C3\%B6chstes_Stadium_des_Kapitalismus}{\emph{»Imperialismus
  als höchstes Stadium des Kapitalismus«}}, da sich kapitalistische
  Staaten Absatzmärkte für ihre Überproduktion erobern müssten. Selbst
  Feindstaaten möchten indes mit dem Westen Handel treiben, werden
  jedoch durch Sanktionen oder Krieg daran gehindert. Zudem führten
  bereits vorkapitalistische Staaten wie Rom und Spanien und auch
  anti-kapitalistische Staaten wie die Sowjetunion imperiale Kriege.
\item
  \textbf{Staatsverschuldung:} Die Staatsverschuldung ist ebenfalls kein
  Grund für die US-Kriege, da die USA ihr Geld durch die FED selbst
  \href{https://www.investopedia.com/articles/investing/081415/understanding-how-federal-reserve-creates-money.asp}{schöpfen}
  und die Kriege ihrerseits zur »Verschuldung« beitragen.
\item
  \textbf{Rüstungsindustrie:} US-Präsident Eisenhower warnte 1961 vor
  dem Einfluss des
  \href{https://de.wikipedia.org/wiki/Milit\%C3\%A4risch-industrieller_Komplex}{»militärisch-industriellen
  Komplexes«}. Jener zählt sicherlich zu den Hauptprofiteuren der
  Kriege, dies aber auch in Staaten wie Russland, China, Schweden oder
  der Schweiz. Zudem erfolgen die US-Kriege nicht willkürlich, sondern
  nach imperialen Gesichtspunkten. Schließlich führte auch das Römische
  Reich seine Kriege nicht bloß, um möglichst viele Waffen zu
  produzieren.
\item
  \textbf{»Israel-Lobby«:} Dieser Aspekt wurde im bekannten
  \href{https://en.wikipedia.org/wiki/The_Israel_Lobby_and_U.S._Foreign_Policy}{Buch}
  der Professoren Walt und Mearsheimer hervorgehoben. Die israelische
  Regierung und pro-israelische Organisationen wie AIPAC
  \href{https://fpif.org/dont_blame_the_iraq_debacle_on_the_israel_lobby/}{lobbyierten}
  für den Irakkrieg von 2003 sowie für einen Krieg gegen den Iran. Als
  Hegemonialmacht müssen die USA indes von Ostasien über Zentralafrika
  bis nach Südamerika intervenieren, und selbst die Kriege im Nahen
  Osten folgen einer übergeordneten imperialen Logik. (Mehr:
  \href{https://swprs.org/die-israel-lobby-fakten-und-mythen/}{Die
  »Israel-Lobby«})
\item
  \textbf{Die »Neo-Konservativen«}: Eine weitere These besagt, für die
  US-Kriege seien die sogenannten
  \href{https://en.wikipedia.org/wiki/Neoconservatism}{»Neo-Konserva­tiven«}
  verantwortlich. Diese These wird unter anderem durch die Kriege der
  liberalen Clinton- und Obama-Administrationen widerlegt.
\end{enumerate}

\emph{»Wir haben nun etwa fünf bis zehn Jahre Zeit, um mit diesen
sowjetischen Klientelregimen im Nahen Osten aufzuräumen -- Syrien, Irak,
Iran -- bevor uns die nächste Großmacht herausfordern wird.«}\\
Pentagon-Strategiechef Paul Wolfowitz zu General Wesley Clark in 1991
(\href{https://youtu.be/TY2DKzastu8?t=3m6s}{FORA})

\hypertarget{literatur}{%
\paragraph{Literatur}\label{literatur}}

Sylvan, David \& Majeski, Stephen
(\href{http://www.us-foreign-policy-perspective.org/}{2009}):
\href{http://www.us-foreign-policy-perspective.org/}{U.S. Foreign Policy
in Perspective: Clients, Enemies and Empire}. Routledge, London.

Blum, William
(\href{https://www.zedbooks.net/shop/book/killing-hope/}{2014}): US
Military and CIA Interventions Since World War II -- Updated Edition.
ZED Books, London.

Brzezinski, Zbigniew
(\href{https://archive.org/details/TheGrandChessboardAmericanPrimacyAndItsGeostrategicImperatives1997ZbigniewBrzezinski}{1998}):
The Grand Chessboard: American Primacy And Its Geostrategic Imperatives.
Basic Books, New York.

Haass, Richard (\href{https://www.cfr.org/book/world-disarray}{2017}): A
World in Disarray: American Foreign Policy and the Crisis of the Old
Order. Penguin Press, London.

Kagan, Robert
(\href{http://carnegieendowment.org/1998/06/01/benevolent-empire-pub-275}{1998}):
The Benevolent Empire. Foreign Policy Magazine.

Kissinger, Henry
(\href{https://www.penguinrandomhouse.com/books/316669/world-order-by-henry-kissinger/9780143127710}{2015}):
World Order. Penguin Books, London.

\hypertarget{siehe-auch}{%
\paragraph{Siehe auch}\label{siehe-auch}}

\begin{itemize}
\tightlist
\item
  \href{https://swprs.org/propaganda-im-jugoslawienkrieg/}{Propaganda im
  Jugoslawienkrieg}
\item
  \href{https://swprs.org/ruanda-was-geschah-wirklich/}{Ruanda: Was
  geschah 1994 wirklich?}
\item
  \href{https://swprs.org/jemenkrieg-und-medien/}{Der Jemenkrieg und die
  Medien}
\end{itemize}

\begin{center}\rule{0.5\linewidth}{\linethickness}\end{center}

Beitrag teilen auf:
\href{https://twitter.com/intent/tweet?url=https://swprs.org/logik-imperialer-kriege/}{Twitter}
/
\href{https://www.facebook.com/share.php?u=https://swprs.org/logik-imperialer-kriege/}{Facebook}\\
Publiziert: Mai 2018

\hypertarget{swiss-policy-research}{%
\subsubsection{Swiss Policy Research}\label{swiss-policy-research}}

\begin{itemize}
\tightlist
\item
  \href{https://swprs.org/kontakt/}{Kontakt}
\item
  \href{https://swprs.org/uebersicht/}{Übersicht}
\item
  \href{https://swprs.org/donationen/}{Donationen}
\item
  \href{https://swprs.org/disclaimer/}{Disclaimer}
\end{itemize}

\hypertarget{english}{%
\subsubsection{English}\label{english}}

\begin{itemize}
\tightlist
\item
  \href{https://swprs.org/contact/}{About Us / Contact}
\item
  \href{https://swprs.org/media-navigator/}{The Media Navigator}
\item
  \href{https://swprs.org/the-american-empire-and-its-media/}{The CFR
  and the Media}
\item
  \href{https://swprs.org/donations/}{Donations}
\end{itemize}

\hypertarget{follow-by-email}{%
\subsubsection{Follow by email}\label{follow-by-email}}

Follow

\href{https://wordpress.com/?ref=footer_custom_com}{WordPress.com}.

\protect\hyperlink{}{Up ↑}

Post to

\protect\hyperlink{}{Cancel}

\includegraphics{https://pixel.wp.com/b.gif?v=noscript}
