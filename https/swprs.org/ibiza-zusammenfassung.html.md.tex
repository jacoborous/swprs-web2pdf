\protect\hyperlink{content}{Skip to content}

\href{https://swprs.org/}{}

\protect\hyperlink{search-container}{Search}

Search for:

\href{https://swprs.org/}{\includegraphics{https://swprs.files.wordpress.com/2020/05/swiss-policy-research-logo-300.png}}

\href{https://swprs.org/}{Swiss Policy Research}

Geopolitics and Media

Menu

\begin{itemize}
\tightlist
\item
  \href{https://swprs.org}{Start}
\item
  \href{https://swprs.org/srf-propaganda-analyse/}{Studien}

  \begin{itemize}
  \tightlist
  \item
    \href{https://swprs.org/srf-propaganda-analyse/}{SRF / ZDF}
  \item
    \href{https://swprs.org/die-nzz-studie/}{NZZ-Studie}
  \item
    \href{https://swprs.org/der-propaganda-multiplikator/}{Agenturen}
  \item
    \href{https://swprs.org/die-propaganda-matrix/}{Medienmatrix}
  \end{itemize}
\item
  \href{https://swprs.org/medien-navigator/}{Analysen}

  \begin{itemize}
  \tightlist
  \item
    \href{https://swprs.org/medien-navigator/}{Navigator}
  \item
    \href{https://swprs.org/der-propaganda-schluessel/}{Techniken}
  \item
    \href{https://swprs.org/propaganda-in-der-wikipedia/}{Wikipedia}
  \item
    \href{https://swprs.org/logik-imperialer-kriege/}{Kriege}
  \end{itemize}
\item
  \href{https://swprs.org/netzwerk-medien-schweiz/}{Netzwerke}

  \begin{itemize}
  \tightlist
  \item
    \href{https://swprs.org/netzwerk-medien-schweiz/}{Schweiz}
  \item
    \href{https://swprs.org/netzwerk-medien-deutschland/}{Deutschland}
  \item
    \href{https://swprs.org/medien-in-oesterreich/}{Österreich}
  \item
    \href{https://swprs.org/das-american-empire-und-seine-medien/}{USA}
  \end{itemize}
\item
  \href{https://swprs.org/bericht-eines-journalisten/}{Fokus I}

  \begin{itemize}
  \tightlist
  \item
    \href{https://swprs.org/bericht-eines-journalisten/}{Journalistenbericht}
  \item
    \href{https://swprs.org/russische-propaganda/}{Russische Propaganda}
  \item
    \href{https://swprs.org/die-israel-lobby-fakten-und-mythen/}{Die
    »Israel-Lobby«}
  \item
    \href{https://swprs.org/geopolitik-und-paedokriminalitaet/}{Pädokriminalität}
  \end{itemize}
\item
  \href{https://swprs.org/migration-und-medien/}{Fokus II}

  \begin{itemize}
  \tightlist
  \item
    \href{https://swprs.org/covid-19-hinweis-ii/}{Coronavirus}
  \item
    \href{https://swprs.org/die-integrity-initiative/}{Integrity
    Initiative}
  \item
    \href{https://swprs.org/migration-und-medien/}{Migration \& Medien}
  \item
    \href{https://swprs.org/der-fall-magnitsky/}{Magnitsky Act}
  \end{itemize}
\item
  \href{https://swprs.org/kontakt/}{Projekt}

  \begin{itemize}
  \tightlist
  \item
    \href{https://swprs.org/kontakt/}{Kontakt}
  \item
    \href{https://swprs.org/uebersicht/}{Seitenübersicht}
  \item
    \href{https://swprs.org/medienspiegel/}{Medienspiegel}
  \item
    \href{https://swprs.org/donationen/}{Donationen}
  \end{itemize}
\item
  \href{https://swprs.org/contact/}{English}
\end{itemize}

\protect\hyperlink{}{Open Search}

\hypertarget{ibiza-ein-geostrategischer-coup}{%
\section{Ibiza: Ein geostrategischer
Coup}\label{ibiza-ein-geostrategischer-coup}}

Publiziert: Mai 2019; Aktualisiert: Dezember 2019

\textbf{Die Ibiza-Affäre: Ein Coup gegen eine Moskau-freundliche und
EU-kritische Regierungspartei. Eine Übersicht der geostrategischen und
geheimdienstlichen Hintergründe.}

\includegraphics{https://swprs.files.wordpress.com/2019/05/fpoe-einiges-russland.jpg?w=500\&h=300}

\emph{»Eine erfolgreiche Operation eines Nachrichten­dienstes, mit dem
Resultat, dass eine Regierung eines Nachbar­staates der Schweiz nicht
mehr existiert und eine Koalition zerbrochen ist.«}\\
Jean-Philippe Gaudin, Chef des Schweizer Nachrichtendienstes, am
\href{https://www.nzz.ch/schweiz/das-sind-die-neun-groessten-gefahren-fuer-die-schweiz-ld.1483571}{24.
Mai 2019}

\hypertarget{achse-wien-moskau}{%
\subparagraph{\texorpdfstring{\textbf{Achse
Wien-Moskau}}{Achse Wien-Moskau}}\label{achse-wien-moskau}}

Mit \textbf{Gudenus und Strache} wurden die beiden
\href{https://diepresse.com/home/innenpolitik/5136136/FPOe-schliesst-FuenfJahresVertrag-mit-KremlPartei}{Hauptarchitekten}
der Achse Wien-Moskau entfernt. Gudenus war 2014 als Wahlbeobachter auf
der Krim und erteilte der russischen »Annexion« damit einen
»Persilschein«
(\href{https://www.nzz.ch/international/die-meistgenannten-figuren-in-oesterreichs-drama-um-straches-ibizagate-sind-zwei-politiker-im-rausch-zwei-phantome-und-ein-politisch-versierter-unternehmer-ld.1482831}{NZZ}).
Im Dezember 2016 fädelte er das Abkommen zwischen Straches FPÖ und
Putins Partei »Einiges Russland« ein. Ein halbes Jahr später wurden die
beiden durch eine angebliche russische Oligarchennichte in die Falle
gelockt und schließlich zu Fall gebracht.

Strache
\href{https://kurier.at/politik/inland/fpoe-chef-heinz-christian-strache-fordert-einschraenkung-der-menschenrechte-zur-terrorbekaempfung-und-will-aus-dem-nato-partnerschaftsprogramm-aussteigen/268.854.488}{forderte}
seit 2017 wiederholt den \textbf{Austritt Österreichs aus der
NATO-Partnerschaft} sowie den
\href{https://derstandard.at/2000053241970/OeVP-begruesst-Straches-Kurswechsel-zu-einer-EU-Armee}{Austritt}
Europas aus der NATO. »Die Alternative ist, unter amerikanischem
geostra­te­gischem Kommando, das welche Interessen auch immer hat, dort
einzuzahlen und die US-Militärindustrie zu finanzieren.«, so Strache
2017.

Österreich war denn auch eines der wenigen EU-Länder, das nach dem
unaufgeklärten \textbf{Skripal-Zwischenfall} im März 2018 keine
russischen Diplomaten des Landes verwies. Im November 2018
identifizierte der britische Geheimdienst jedoch einen Oberst der
österreichischen Armee, der während bis zu 25 Jahren für Russland
spionierte. Der Oberst wurde daraufhin verhaftet.

Ein \textbf{NATO-Geheimdienstoffizier}
\href{https://www.buzzfeednews.com/article/mitchprothero/austria-russia-colonel-spy}{sagte}
daraufhin, die österreichische Regierung sei »ein Problem für alle«:
»Österreich ist ein Problem für alle. Die aktuelle Regierung hat tiefe
ideologische und ökonomische Beziehungen zum Putin-Regime. Aber man kann
nach Skripal nicht mehr solides EU-Mitglied und gleichzeitig ein enger
Freund von Putin sein.«

FPÖ-Spitzenkandidat \textbf{Vilimsky}, der sich 2014 noch gegen die
Russland­sanktionen
\href{https://www.fpoe.eu/vilimsky-strafsanktionen-gegen-russland-sind-verfassungswidrig-und-umgehend-einzustellen/}{aussprach}
(»umgehend einstellen«), kündigte nach Ibiza an, das 2016 vereinbarte
Koop­e­ra­tions­­abkommen zwischen der FPÖ und Einiges Russland
\href{https://www.oe24.at/oesterreich/politik/euwahl2019/Elefantenrunde-Ibiza-ist-fuer-Politik-tragisch/381787765}{auslaufen}
zu lassen. Er betonte, keine »Verstrickungen nach Russland« zu haben,
und verwies auf seine guten Kontakte zum Pentagon.

Die \emph{Salzburger Nachrichten}
\href{https://www.sn.at/politik/innenpolitik/das-ende-der-russischen-fraktion-in-der-fpoe-70471840}{sprachen}
denn auch vom \textbf{»Ende der ``russischen'' Fraktion in der FPÖ«},
und der \emph{Kurier}
\href{https://kurier.at/politik/inland/die-fpoe-in-der-russland-falle/400647842}{titelte}
noch im Oktober 2019 »Die FPÖ in der Russland-Falle«.

Bemerkenswert ist zudem, dass das österreichische
Verteidigungsministerium am 21. Mai 2019, wenige Stunden nach dem
Rücktritt der FPÖ-Minister, die Durchführung der seit langem geplanten
russisch-europäischen \textbf{Valdai-Diskussions­runde} in Wien zum
Thema »Multipolare Diplomatie« ganz kurzfristig
\href{https://www.nachdenkseiten.de/?p=51996}{absagte} und den Saal
sperrte.

Nach dem Rücktritt der FPÖ-Minister aus der Regierung blieb die
parteifreie, aber von der FPÖ nominierte Außenministerin \textbf{Karin
Kneissl} zunächst im Amt. Dies wurde von deutschen Medien wie
\href{https://www.spiegel.de/politik/ausland/oesterreich-aussenministerin-karin-kneissl-verweigert-ruecktritt-a-1268669.html}{Spiegel}
und
\href{https://www.bild.de/politik/ausland/politik-ausland/oesterreich-warum-darf-putins-lieblingsministerin-kneissl-bleiben-62076358.bild.html}{BILD}
heftig kritisiert, da Kneissl ebenfalls als Russland-freundlich gilt und
Präsident Putin im August 2018 zu ihrer Hochzeit einlud (»Putins
Lieblingsministerin«). Erst durch den vollständigen Sturz der Regierung
wurde auch Kneissl und damit alle Russland-freundlichen und
EU-kritischen Minister aus der österreichischen Regierung entfernt.

Das \textbf{russische Außenministerium} verwies seinerseits auf die
geopolitische Dimension der Ibiza-Affäre: »Aber die Hauptfrage bleibt
offen: Wer hat so eine grobe Einmischung in das innenpolitische Leben
Österreichs begangen und dabei die ‚unabhängige` deutsche Presse als
Instrument eingesetzt?«, fragte Sprecherin Sacharowa gemäß
\href{https://de.sputniknews.com/politik/20190529325090595-skandal-russin-strache-video-lawrow/}{\emph{Sputnik}}.
Russland werde »die OSZE und weitere Organisationen über die
Informationsoperation der deutschen Medien informieren«.

Der \textbf{österreichische Geheimdienst BVT} wurde bereits seit 2018
aufgrund der »Russland-Nähe« der FPÖ von allen europäischen
Geheimdiensten
\href{https://derstandard.at/2000101031061/Russland-Naehe-der-FPOe-sorgt-fuer-Isolation-des-BVT-von}{isoliert}.
Dies nach einer Razzia durch den FPÖ-Innenminister Kickl im Februar
2018. Kickl wurde im Zuge der Ibiza-Affäre von Kanzler Kurz seinerseits
zum Rücktritt gedrängt.

\textbf{Kanzler Sebastian Kurz} genießt als
\href{https://www.ecfr.eu/council/}{Mitglied} des \emph{European Council
on Foreign Relations (ECFR)} den Rückhalt der transatlantischen Elite.
Sein Sturz war vermutlich nicht geplant, und tatsächlich wurde Kurz im
September bereits wieder zum Kanzler gewählt wird.

Interessant ist die Kommentierung der Ibiza-Affäre durch
\textbf{angelsächsische Medien}: Das amerikanische \emph{Foreign Policy
Magazine} titelte
\href{https://foreignpolicy.com/2019/05/23/europe-is-ripe-for-a-return-to-establishment-politics/}{»Europe
Is Ripe for a Return to Establish­ment Politics«}, der britische
\emph{Economist}
\href{https://www.economist.com/europe/2019/05/25/why-cosying-up-to-populists-rarely-ends-well-for-moderates}{»Why
cosying up to populists rarely ends well for moderates«}.

Sein erstes ausführliches
\href{https://www.youtube.com/watch?v=5BD6vYa2c9w}{Interview} nach der
Affäre gab Heinz-Christian Strache im August 2019 demonstrativ dem
russischen TV-Sender RT Deutsch. Darin betonte er erneut seine
Russland-freundliche Haltung, die ihm womöglich zum Verhängnis wurde.

\hypertarget{geheimdienst-operation}{%
\subparagraph{\texorpdfstring{\textbf{Geheimdienst-Operation}}{Geheimdienst-Operation}}\label{geheimdienst-operation}}

\textbf{Jean-Philippe Gaudin}, der Chef des Schweizer
Nachrichtendienstes NDB,
\href{https://www.nzz.ch/schweiz/das-sind-die-neun-groessten-gefahren-fuer-die-schweiz-ld.1483571}{sprach}
bereits im Mai von einer »erfolgreichen Operation eines
Nachrichten­dienstes, mit dem Resultat, dass eine Regierung eines
Nachbar­staates der Schweiz nicht mehr existiert und eine Koalition
zerbrochen ist.«

Der österreichische Interims-Innenminister \textbf{Wolfgang Peschorn}
schloss in einem
\href{https://www.youtube.com/watch?v=Boyi80CyQYk}{Interview} im August
2019 einen ausländischen Geheimdienst als Drahtzieher der Operation
explizit nicht aus.

Tatsächlich hatte die \textbf{private Sicherheitsgruppe}, die die
Ibiza-Operation unmittelbar durchführte, langjährige und enge
\href{http://www.eu-infothek.com/ibiza-hintermaenner-ermittelten-fuer-lka-bk-und-finanzpolizei/}{Kontakte}
zu Behörden und Geheimdiensten. Auch die präparierte Ibiza-Finca wurde
offenbar durch diese Kontakte
\href{http://www.eu-infothek.com/ibiza-connection-die-unfassbare-einzigartige-ganze-geschichte/}{vermittelt}.
Zudem wurde die Sicherheitsgruppe in früheren Strafverfahren durch den
österreichischen Geheimdienst BVT
\href{http://www.eu-infothek.com/der-standard-geheimdienst-hatte-straches-fallensteller-schon-an-der-angel/}{gedeckt}.

Im Dezember 2019 wurde zudem bekannt, dass auch die angebliche
\textbf{Oligarchennichte} von einem Geheimdienst offenbar in Osteuropa
gedeckt wird, für den sie als Lockvogel
\href{https://web.archive.org/web/20191205134916/https://www.oe24.at/oesterreich/politik/Wer-schuetzt-den-Ibiza-Lockvogel/408323776}{arbeite}.
Auch der Chef der Sicherheitsgruppe scheint weiterhin geschützt zu
werden.

Einige
\href{https://www.oe24.at/oesterreich/politik/Verdacht-War-BVT-in-Ibiza-Video-involviert/389533553}{Indizien}
sprechen dafür, dass Mitarbeiter des österreichischen \textbf{BVT}
selbst in die Ibiza-Aktion involviert waren, wie es der ehemalige
österreichische Innenminister Kickl
\href{https://www.oe24.at/oesterreich/politik/Kickl-ueber-Ibiza-Video-Verbindung-zu-BVT/381763805}{vermutete},
der im Februar 2018 eine Razzia beim BVT durchführte.

Vom \textbf{deutschen BND} ist bekannt, dass er in der Vergangenheit
österreichische Politiker und Ministerien systematisch
\href{https://www.zeit.de/2018/26/bnd-spionage-oesterreich-daten}{überwachte}.
Zudem dürften die deutschen Dienste kaum übersehen haben, wenn ihre
eigenen Medien, »Komiker« und »Künstler­gruppen« kurz vor der EU-Wahl
einen solchen Coup vorbereiten. Ehemalige BND-Chefs
\href{https://www.n-tv.de/mediathek/videos/politik/Offenkundig-wird-hier-versucht-Wahlen-zu-manipulieren-article21035713.html}{zeigten}
sich in den Medien gleichwohl überrascht und
\href{https://archive.is/20190522200701/https://www.cicero.de/aussenpolitik/heinz-christian-strache-video-johannes-gudenus-ibiza-israel-mossad}{verwiesen}
stattdessen auf den Mossad, allerdings ohne dies zu belegen.

Im Dezember 2019 wurde zudem
\href{https://www.youtube.com/watch?v=ApetEsHeXv0}{bekannt}, dass sich
der BND unmittelbar in die Ibiza-Ermittlungen eingeschaltet habe. Diese
lieferten bislang allerdings kaum Erkenntnisse.

OE24
\href{https://www.oe24.at/oesterreich/politik/Wer-steckt-hinter-dem-Strache-Video/380559980}{berichtete}
bereits am 18. Mai:~ »Experten sehen die Möglichkeit, dass es sich um
eine Inszenierung eines westlichen Geheimdienstes handelt. Wie
Geheimdienstler () berichten, stehen dabei die Russland-Kontakte der FPÖ
im Mittelpunkt. Man sorgte sich, wie die Zusammen­arbeit künftig
funktionieren werde. Vor allem die USA seien verärgert, dass sich
Österreich zunehmend auf die Seite Moskaus stelle, und befürchten, dass
Informationen Richtung Russland abfließen könnten.«

Es ist bekannt, dass die private Sicherheitsgruppe bereits zuvor
prominente oder vermögende Personen
\href{http://www.eu-infothek.com/ibiza-gate-schwerer-junge-j-h-liess-ueber-spezial-auskunftei-hunderte-wichtige-personen-vollkommen-ausspionieren/}{ausspioniert}
und teilweise erpresst hatte. Die entscheidende Frage ist indes, ob
diese Gruppe von Dritten für politische oder geopolitische Ziele
verwendet wurde.

\hypertarget{das-zentrum-fuxfcr-politische-schuxf6nheit}{%
\subparagraph{**Das »Zentrum für Politische
Schönheit«}\label{das-zentrum-fuxfcr-politische-schuxf6nheit}}

**

Im Rahmen der Ibiza-Affäre wurde das Berliner »Zentrum für Politische
Schönheit« (ZPS) als Vertragspartei für den Videokauf genannt. Das ZPS
gilt gemeinhein als eine »linke« Künstler- und Aktionsgruppe, doch diese
Einschätzung ist fragwürdig, denn tatsächlich ist das ZPS eher als
»künstlerisches« Instrument von Staatsschutz und NATO einzustufen.

Gegründet 2008,
\href{https://web.archive.org/web/20150621135759/http://www.gehvoran.com/2011/02/kriegsverbrechen-in-libyen-wo-bleibt-die-welt/}{setzte}
es sich ab 2011 sowohl für die NATO-Intervention gegen Libyen als auch
für den Krieg \href{http://www.barth-engelbart.de/?p=68977}{gegen}
Syrien ein. Seither unterstützt es mit Aktionen die
Mittel­meer­migration (oder nimmt deren Gegner ins Visier), wie sie vom
ehemaligen EU-Wett­bewerbs­kommissar, WTO-Generaldirektor,
Goldman-Sachs-Präsidenten und UNO-Migrations­beauftragen Peter
Sutherland
\href{https://de.wikipedia.org/wiki/Peter_Sutherland}{gefordert} wurde.

Der Kapitalismus wird derweil gegen »linke Kritiker«
\href{https://www.welt.de/kultur/article152433895/Die-Politik-ist-auf-Abschottung-konzentriert.html}{verteidigt}.
2015 erhielt das Zentrum eine
\href{https://www.youtube.com/watch?v=6Ro4_Wv4BFE}{Auszeichnung} einer
regierungs­nahen Berliner Stiftung mit Staatsschutz-Kontakten. 2017
\href{https://www.ksta.de/kultur/zentrum-fuer-politische-schoenheit-flugblaetter-gegen-diktatoren-27863928}{rief}
es erneut zum Sturz US-kritischer Regierungen auf -- natürlich nur
künstlerisch.

Ein deutscher Autor
\href{http://www.barth-engelbart.de/?p=68977}{spricht} denn auch vom
»Atlantik-Zentrum für Politische Schönheit«.

\hypertarget{die-integrity-initiative}{%
\subparagraph{\texorpdfstring{\textbf{Die »Integrity
Initiative«}}{Die »Integrity Initiative«}}\label{die-integrity-initiative}}

In den 2018 veröffentlichten Dokumenten der
\href{https://swprs.org/die-integrity-initiative/}{»Integrity
Initiative«} des britischen militärischen Geheimdienstes ist auch
Österreich
\href{https://www.pdf-archive.com/2018/11/02/xcountry/xcountry.pdf}{aufgeführt},
und zwar mit dem Zieldatum des Oktober 2017: dem Datum der
österreichischen Parlamentswahlen, in deren Vorfeld das Ibiza-Video
entstand.

Als Kontaktpersonen für Österreich sind
\href{https://www.pdf-archive.com/2018/11/02/xcountry/xcountry.pdf}{angegeben}
eine (britische) Mitarbeiterin der Berliner Stiftung Wissenschaft und
Politik (SWP) mit Fachgebiet Osteuropa/Russland, sowie ein ehemaliger
britischer Geheimdienstmitarbeiter. Die SWP wird von der deutschen
Bundes­regierung finanziert und organisierte 2012 zusammen mit einem
US-Institut eine Serie von
\href{http://www.zeit.de/2012/31/Syrien-Bundesregierung}{Workshops} zur
Planung der Zeit nach einem Regimewechsel in Syrien.

Die »Integrity Initiative«
\href{https://swprs.org/die-integrity-initiative/}{versuchte} bereits in
mehreren Ländern, Russland-freundliche Politiker und Spitzenbeamte zu
verhindern oder zu Fall zu bringen. Für die Schweiz ist als Zieldatum
der Oktober 2019 angeben: das Datum der Eidgenössischen Wahlen.

Die Initiative wird hauptsächlich vom britischen und amerikanischen
Außen­ministerium und von der NATO finanziert. Traditionelle Medien
haben bis heute nicht über die
\href{https://swprs.org/die-integrity-initiative/}{Aufdeckung} der
Geheimdienst-Initiative und ihres internationalen Netzwerks im Dezember
2018 berichtet.

\hypertarget{attentatspluxe4ne-gegen-strache-und-salvini}{%
\subparagraph{\texorpdfstring{\textbf{Attentatspläne gegen Strache und
Salvini}}{Attentatspläne gegen Strache und Salvini}}\label{attentatspluxe4ne-gegen-strache-und-salvini}}

Der ehemalige österreichische Vize-Kanzler Strache und der italienische
Vize-Premier Salvini vertreten beide eine EU-kritische und
Russland-freundliche Position. Beide schlossen beispiels­weise ein
\href{https://www.faz.net/aktuell/politik/ausland/lega-nord-chef-salvini-will-hilfe-aus-moskau-14913862.html}{Kooperations­abkommen}
mit der Putin-Partei Einiges Russland und beide möchten die
Russland-Sanktionen beenden.

Gegen beide gab es eine Video- bzw.
\href{https://www.buzzfeednews.com/article/albertonardelli/salvini-russia-oil-deal-secret-recording}{Audio-Falle}
im Zusammenhang mit Russland, und gegen
\href{https://de.euronews.com/2019/07/17/salvini-ukrainer-haben-ein-attentat-auf-mich-geplant}{beide}
wurden inzwischen
\href{https://diepresse.com/home/innenpolitik/5649832/Anschlag-auf-Strache-geplant}{Attentatspläne}
bekannt. Salvini wurde von den Medien zuletzt zudem mit dem ehemaligen
Premierminister Aldo Moro
\href{https://www.tpi.it/2019/08/05/salvini-moro-mare-fusaro/}{verglichen},
der in den 1970er-Jahren eine Annäherung an Moskau suchte, von Kissinger
gewarnt und schließlich entführt und ermordet wurde.

\hypertarget{fazit}{%
\subparagraph{\texorpdfstring{\textbf{Fazit}}{Fazit}}\label{fazit}}

Der privaten Sicherheitsgruppe, die die Ibiza-Aktion unmittelbar
durchführte, scheint es primär um den finanziellen Erlös des
Videoverkaufs gegangen zu sein, der sich im Sommer 2017 in Österreich
noch nicht realisieren ließ, womöglich auch wegen der gleichzeitig
auffliegenden Silberstein-Affäre. Es bleibt indes die Frage, ob diese
Gruppe aus politischen oder geopolitischen Gründen von Dritten
beauftragt, benutzt, unterstützt oder gedeckt wurde. Wie dargelegt,
deutet einiges darauf hin.

\hypertarget{medienberichte-auf-basis-dieser-recherchen}{%
\subparagraph{**Medienberichte auf Basis dieser
Recherchen}\label{medienberichte-auf-basis-dieser-recherchen}}

**

\begin{itemize}
\tightlist
\item
  Rubikon:
  \href{https://www.rubikon.news/artikel/regime-change-in-osterreich}{Regime
  Change in Österreich}
\item
  KenFM:
  \href{https://kenfm.de/tagesdosis-3-6-2019-oesterreich-im-ibiza-fieber-paaasst-scho/}{Österreich
  im Ibiza-Fieber}
\end{itemize}

\hypertarget{bildergallerie}{%
\subparagraph{\texorpdfstring{\textbf{Bildergallerie}}{Bildergallerie}}\label{bildergallerie}}

\href{https://swprs.org/ibiza-zusammenfassung/fpoe-einiges-russland/}{}

\includegraphics{https://swprs.files.wordpress.com/2019/05/fpoe-einiges-russland.jpg?w=241\&h=241\&crop=1}

Kooperationsabkommen zwischen der FPÖ und Einiges Russland. In der Mitte
HC Strache, zweiter von rechts Johann Gudenus (Dezember 2016).

\href{https://swprs.org/ibiza-zusammenfassung/kneissl-putin/}{}

\includegraphics{https://swprs.files.wordpress.com/2019/05/kneissl-putin.jpg?w=241\&h=241\&crop=1}

Ministerin Kneissl lud Putin zu ihrer Hochzeit ein (August 2018)

\href{https://swprs.org/ibiza-zusammenfassung/valdai/}{}

\includegraphics{https://swprs.files.wordpress.com/2019/05/valdai.jpg?w=241\&h=241\&crop=1}

Putin und Ex-Kanzler Schüssel an der russisch-europäischen
Valdai-Diskussionsrunde von 2014.

\href{https://swprs.org/ibiza-zusammenfassung/kickl-gridling-bvt/}{}

\includegraphics{https://swprs.files.wordpress.com/2019/05/kickl-gridling-bvt.jpg?w=241\&h=241\&crop=1}

Innenminister Kickl und BVT-Chef Gridling (Mai 2018)

\href{https://swprs.org/ibiza-zusammenfassung/kurz-kunasek/}{}

\includegraphics{https://swprs.files.wordpress.com/2019/06/kurz-kunasek.png?w=241\&h=241\&crop=1}

Kanzler Kurz und FPÖ-Verteidigungsminister Kunasek zur Enttarnung eines
russischen Spions.

\href{https://swprs.org/ibiza-zusammenfassung/elefantinibiza/}{}

\includegraphics{https://swprs.files.wordpress.com/2019/05/elefantinibiza.jpg?w=241\&h=241\&crop=1}

Der Elefant in der Finca (Leserbild)

\begin{center}\rule{0.5\linewidth}{\linethickness}\end{center}

Publiziert: Mai 2019; Aktualisiert: Dezember 2019

\textbf{Hinweis}: Für alle Beteiligten und Genannten gilt die
Unschuldsvermutung.

\hypertarget{swiss-policy-research}{%
\subsubsection{Swiss Policy Research}\label{swiss-policy-research}}

\begin{itemize}
\tightlist
\item
  \href{https://swprs.org/kontakt/}{Kontakt}
\item
  \href{https://swprs.org/uebersicht/}{Übersicht}
\item
  \href{https://swprs.org/donationen/}{Donationen}
\item
  \href{https://swprs.org/disclaimer/}{Disclaimer}
\end{itemize}

\hypertarget{english}{%
\subsubsection{English}\label{english}}

\begin{itemize}
\tightlist
\item
  \href{https://swprs.org/contact/}{About Us / Contact}
\item
  \href{https://swprs.org/media-navigator/}{The Media Navigator}
\item
  \href{https://swprs.org/the-american-empire-and-its-media/}{The CFR
  and the Media}
\item
  \href{https://swprs.org/donations/}{Donations}
\end{itemize}

\hypertarget{follow-by-email}{%
\subsubsection{Follow by email}\label{follow-by-email}}

Follow

\href{https://wordpress.com/?ref=footer_custom_com}{WordPress.com}.

\protect\hyperlink{}{Up ↑}

Post to

\protect\hyperlink{}{Cancel}

\includegraphics{https://pixel.wp.com/b.gif?v=noscript}
