\protect\hyperlink{content}{Skip to content}

\href{https://swprs.org/}{}

\protect\hyperlink{search-container}{Search}

Search for:

\href{https://swprs.org/}{\includegraphics{https://swprs.files.wordpress.com/2020/05/swiss-policy-research-logo-300.png}}

\href{https://swprs.org/}{Swiss Policy Research}

Geopolitics and Media

Menu

\begin{itemize}
\tightlist
\item
  \href{https://swprs.org}{Start}
\item
  \href{https://swprs.org/srf-propaganda-analyse/}{Studien}

  \begin{itemize}
  \tightlist
  \item
    \href{https://swprs.org/srf-propaganda-analyse/}{SRF / ZDF}
  \item
    \href{https://swprs.org/die-nzz-studie/}{NZZ-Studie}
  \item
    \href{https://swprs.org/der-propaganda-multiplikator/}{Agenturen}
  \item
    \href{https://swprs.org/die-propaganda-matrix/}{Medienmatrix}
  \end{itemize}
\item
  \href{https://swprs.org/medien-navigator/}{Analysen}

  \begin{itemize}
  \tightlist
  \item
    \href{https://swprs.org/medien-navigator/}{Navigator}
  \item
    \href{https://swprs.org/der-propaganda-schluessel/}{Techniken}
  \item
    \href{https://swprs.org/propaganda-in-der-wikipedia/}{Wikipedia}
  \item
    \href{https://swprs.org/logik-imperialer-kriege/}{Kriege}
  \end{itemize}
\item
  \href{https://swprs.org/netzwerk-medien-schweiz/}{Netzwerke}

  \begin{itemize}
  \tightlist
  \item
    \href{https://swprs.org/netzwerk-medien-schweiz/}{Schweiz}
  \item
    \href{https://swprs.org/netzwerk-medien-deutschland/}{Deutschland}
  \item
    \href{https://swprs.org/medien-in-oesterreich/}{Österreich}
  \item
    \href{https://swprs.org/das-american-empire-und-seine-medien/}{USA}
  \end{itemize}
\item
  \href{https://swprs.org/bericht-eines-journalisten/}{Fokus I}

  \begin{itemize}
  \tightlist
  \item
    \href{https://swprs.org/bericht-eines-journalisten/}{Journalistenbericht}
  \item
    \href{https://swprs.org/russische-propaganda/}{Russische Propaganda}
  \item
    \href{https://swprs.org/die-israel-lobby-fakten-und-mythen/}{Die
    »Israel-Lobby«}
  \item
    \href{https://swprs.org/geopolitik-und-paedokriminalitaet/}{Pädokriminalität}
  \end{itemize}
\item
  \href{https://swprs.org/migration-und-medien/}{Fokus II}

  \begin{itemize}
  \tightlist
  \item
    \href{https://swprs.org/covid-19-hinweis-ii/}{Coronavirus}
  \item
    \href{https://swprs.org/die-integrity-initiative/}{Integrity
    Initiative}
  \item
    \href{https://swprs.org/migration-und-medien/}{Migration \& Medien}
  \item
    \href{https://swprs.org/der-fall-magnitsky/}{Magnitsky Act}
  \end{itemize}
\item
  \href{https://swprs.org/kontakt/}{Projekt}

  \begin{itemize}
  \tightlist
  \item
    \href{https://swprs.org/kontakt/}{Kontakt}
  \item
    \href{https://swprs.org/uebersicht/}{Seitenübersicht}
  \item
    \href{https://swprs.org/medienspiegel/}{Medienspiegel}
  \item
    \href{https://swprs.org/donationen/}{Donationen}
  \end{itemize}
\item
  \href{https://swprs.org/contact/}{English}
\end{itemize}

\protect\hyperlink{}{Open Search}

\hypertarget{professor-sucharit-bhakdi-avalik-kiri-angela-merkelile}{%
\section{Professor Sucharit Bhakdi avalik kiri
Angela~Merkelile}\label{professor-sucharit-bhakdi-avalik-kiri-angela-merkelile}}

\includegraphics{https://swprs.files.wordpress.com/2020/03/bakhdi-letter-header.png?w=736\&h=297}

\textbf{Keeled}:
\href{https://swprs.org/offener-brief-von-professor-sucharit-bhakdi-an-bundeskanzlerin-dr-angela-merkel/}{DE},
\href{https://swprs.org/open-letter-from-professor-sucharit-bhakdi-to-german-chancellor-dr-angela-merkel/}{EN};
\href{https://swprs.org/professor-sucharit-bhakdi-avalik-kiri-angela-merkelile/}{EE},
\href{http://piensachile.com/2020/03/carta-abierta-a-angela-merkel/}{ES},
\href{https://swprs.org/\%d0\%be\%d1\%82\%d0\%ba\%d1\%80\%d1\%8b\%d1\%82\%d0\%be\%d0\%b5-\%d0\%bf\%d0\%b8\%d1\%81\%d1\%8c\%d0\%bc\%d0\%be-\%d0\%bf\%d1\%80\%d0\%be\%d1\%84\%d0\%b5\%d1\%81\%d1\%81\%d0\%be\%d1\%80\%d0\%b0-\%d1\%81\%d1\%83\%d1\%87\%d0\%b0\%d1\%80\%d0\%b8\%d1\%82\%d0\%b0/}{RU},
\href{https://swprs.org/prof-dr-sucharit-bhakdiden-basbakan-dr-angela-merkele-acik-mektup/}{TR}

Mainzi Johannes Gutenbergi ülikooli meditsiinilise mikrobioloogia
professori emeriitprofessori dr Sucharit Bhakdi avalik kiri Saksamaa
liidukantslerile dr Angela Merkelile. Professor Bhakdi nõuab Covid-19-le
reageerimise kiiret ümberhindamist ja küsib kantslerilt viis üliolulist
küsimust. Kirja kuupäev on 26. märts. See on mitteametlik tõlge; vaata
saksakeelset algdokumenti
\href{https://swprs.org/offener-brief-von-professor-sucharit-bhakdi-an-bundeskanzlerin-dr-angela-merkel/}{PDF-vormingus}.

\hypertarget{avalik-kiri}{%
\paragraph{Avalik kiri}\label{avalik-kiri}}

Austatud Kantsler!

Johannes Gutenbergi Mainzi ülikooli emeriitprofessori ja pikaajalise
samas asuva Meditsiinilise Mikrobioloogia ja Hügieeni Instituudi juhina
tunnen ma kohustust kriitiliselt tõstatada küsimus jätkuvate
pikaajaliste avaliku elu piirangute jätkamise osas, mida me praegu oma
vastutusel COVID-19 viiruse levimise piiramiseks teostame.

Minu ülesanne pole viirushaiguse ohtude vähendamine ega poliitilise
sõnumi levitamine. Siiski tunnen, et pean andma teadusliku panuse
praeguse olukorra korrektsesse klassifitseerimisse, et seni teadaolevate
faktide perspektiive hinnata ja esitada ka küsimusi, mis ähvardavad
ülekuumenenud debattides kaduma minna.

Minu mure peamiseks põhjuseks on Euroopas ja Saksamaal praktiseeritavad
drastilised isoleerimismeetmed, millel võivad olla ettenägematud
sotsiaalmajanduslikud tagajärjed.

Minu soov on arutleda kriitiliselt ja ettenägelikult avaliku elu
piiramist ning sellest tulenevate pikaajaliste mõjude plusse ja
miinused.

Siin on viis küsimust, millele on tänaseni ebapiisavalt vastatud, kuid
mis on tasakaalustatud analüüsi jaoks hädavajalikud.

Seoses sellega palun teil kiirendatud korras võtta seisukoht ja pöörduda
föderaalvalitsuse poole strateegiate väljatöötamiseks, mis tõhusalt
kaitseksid kõrge riskiga rühmi ilma avaliku elutegevuse kärpimiseta,
vältimaks ühiskonna veelgi intensiivsemat lõhestumist.

Suurima austusega

\textbf{Em-prof Med-dr Sucharit Bhakdi}

\hypertarget{1-statistika}{%
\subparagraph{\texorpdfstring{\textbf{1.
Statistika}}{1. Statistika}}\label{1-statistika}}

Robert Kochi poolt rajatud infektoloogia põhjal eristatakse
traditsiooniliselt infektsioone ja haigusi. Haigus nõuab kliinilist
ilmingut. {[}1{]} Seetõttu tuleks statistikasse lisada uute juhtudena
vaid need patsiendid, kellel on sellised sümptomid nagu palavik või
köha. Teisisõnu tähendab see, et uus nakatumine -- mõõdetuna COVID-19
testiga -- ei tähenda tingimata, et meil on tegemist äsja haigestunud
patsiendiga, kes vajab haiglakohta. Praegu aga arvestatakse, et viis
protsenti kõigist nakatunud inimestest haigestub raskelt ja vajavad
hingamisaparaati. Sellel baseeruvad prognoosid näitavad, et
tervishoiusüsteem võib tugevalt ülekoormatud saada.

\textbf{Minu küsimus:} Kas prognoosides on eristatud sümptomiteta
nakatunud ja reaalselt haigestunud patsiente -- s.o inimesi, kellel
tekivad sümptomid?

\hypertarget{2-oht}{%
\subparagraph{\texorpdfstring{\textbf{2. Oht}}{2. Oht}}\label{2-oht}}

Meedias on pikemat aega räägitud koroonaviirusest. Kui peaks selguma, et
COVID-19 viirusele pole vaja omistada märkimisväärselt suuremat ohtlikku
potentsiaali kui juba ringlevatele koroonaviirustele, oleksid kõik
vastumeetmed ilmselgelt ülearused. Rahvusvaheliselt tunnustatud
erialaajakirjas ``International Journal of Antimicrobial Agents'' ilmub
peagi uurimustöö, mis käsitleb täpselt seda küsimust. Uuringu esialgseid
tulemusi võib näha juba praegu ja nendest võib järeldada, et uus viirus
EI erine oma ohtlikkuse poolest traditsioonilistest koroonaviirustest.
Autorid väljendavad seda oma töö pealkirjas ``SARS-CoV-2: Hirm versus
Andmed'' {[}3{]}.

\textbf{Minu küsimus}: Kui suur on praegune täituvus intensiivravi
osakondades COVID-19 diagnoositud patsientidega võrreldes teiste
koroonaviirusnakkustega ja mil määral võetakse neid andmeid Saksa
liiduvalitsuse edasistes otsustusprotsessides arvesse? Lisaks: kas
ülaltoodud uuringut on senistes plaanides arvesse võetud? Muidugi kehtib
siin järgnev: „diagnoositud`` tähendab, et viirus mängib olulist rolli
patsiendi seisundis ja varasemad haigused nii suurt osakaalu ei oma.

\hypertarget{3-levik}{%
\subparagraph{\texorpdfstring{\textbf{3.
Levik}}{3. Levik}}\label{3-levik}}

\emph{Süddeutsche Zeitung}i raporti kohaselt ei tea isegi siinkohal
viidatud Robert Kochi Instituut täpselt, kui palju COVID-19 testitakse.
Tõsiasi on aga see, et Saksamaal testimismahu kasvades täheldati hiljuti
haigusjuhtude arvu kiiret kasvu. {[}4{]} Seetõttu arvatakse, et viirus
on terve elanikkonna hulgas juba märkamatult levinud. Sellel oleks kaks
tagajärge: esiteks tähendaks see, et ametlik suremus -- 26. märtsil 2020
oli umbes 37 300 nakkuse tagajärjel 206 surmajuhtumit ehk 0,55 protsenti
{[}5{]} -- on seatud liiga kõrgeks; ja teiseks, et terves elanikkonnas
haiguse levimist pole enam võimalik vältida.

\textbf{Minu küsimus}: kas viiruse tegeliku leviku kinnitamiseks on juba
tehtud sõeluuring tervest elanikkonnast või on see kavandamisel?

\hypertarget{4-suremus}{%
\subparagraph{\texorpdfstring{\textbf{4.
Suremus}}{4. Suremus}}\label{4-suremus}}

Meedias kajastatakse praegu eriti intensiivselt hirmu surmajuhtumite
suurenemise ees Saksamaal (praegu 0,55 protsenti). Paljud inimesed
muretsevad, et kui õigeaegselt midagi ette ei võeta, võivad tulemused
tõusta nagu Itaalias (10 protsenti) ja Hispaanias (7 protsenti).

Samal ajal külvatakse maailmas subjektiivset hinnangut, teatades
viirusega seotud surmajuhtumitest kohe, kui surnul esines viirus --
arvestamata muid tegureid. See rikub infektoloogia põhinõuet: diagnoosi
saab panna ainult siis, kui on kindlaks tehtud, et konkreetne tegur
moodustab olulise osa haigusest või surmast. Teadusmeditsiiniliste
seltside töörühm kirjutab oma juhistes selgesõnaliselt: „Lisaks surma
põhjusele tuleb täpsustada ka põhjuslik ahel, millele vastav peamine
haigus asub surmatunnistusel kolmandal kohal. Mõnikord peab ka neljanda
osa põhjusliku ahela välja tooma.'' {[}6{]}

Siiani pole ametlikku teavet selle kohta, kas vähemalt tagantjärele on
tehtud kriitilised meditsiinitoimikute analüüsid, et teha kindlaks, kui
palju surmajuhtumeid viiruse tõttu tegelikult on.

\textbf{Minu küsimus:} kas Saksamaa on lihtsalt kaasa läinud üldise
suundumusega COVID-19 kahtluste suhtes? Ja kas ta kavatseb seda
kategoriseeringut jätkata sama subjektiivselt nagu seni? Kuidas siis
tuleks vahet teha koroonaga seotud surmade ja viiruse juhusliku
esinemise vahel surmahetkel?

\hypertarget{5-vuxf5rreldavus}{%
\subparagraph{\texorpdfstring{\textbf{5.
Võrreldavus}}{5. Võrreldavus}}\label{5-vuxf5rreldavus}}

Itaalia hirmutavat olukorda on korduvalt kasutatud
võrdlusstsenaariumina. Kuid viiruse tegelik roll selles riigis on mitmel
põhjusel täiesti ebaselge -- mitte ainult seetõttu, et siin kehtivad
punktid 3 ja 4, vaid ka seetõttu, et on olemas erandlikud välised
tegurid, mis muudavad need piirkonnad eriti haavatavaks.

Sinna juurde kuulub suurenenud õhusaaste Põhja-Itaalias. WHO hinnangul
põhjustas selline olukord ainuüksi Itaalia 13 suuremas linnas rohkem kui
8000 täiendavat surmajuhtumit aastas isegi ilma viiruseta. {[}7{]} Aja
jooksul pole olukord märkimisväärselt muutunud. Lõpuks on ka tõestatud,
et õhusaaste tõstab väga tugevalt viiruslikesse kopsuhaigustesse
haigestumise riski nii noorte kui ka vanemate inimeste hulgas {[}9.{]}

Lisaks elab 27,4 protsenti selle riigi kõige ohustatumast elanikkonnast
koos noortega, Hispaanias isegi 33,5 protsenti. Võrdluseks on see
Saksamaal vaid seitse protsenti {[}10{]}.

Lisaks on Berliin Sachen TÜ Tööstuspiirkonna tervishoiukorralduse
osakonna juhataja prof Reinhard Busse intensiivraviosakonnad varustatud
oluliselt paremini kui Itaalias -- kordajaga 2,5 {[}11{]}.

\textbf{Minu küsimus}: Milliseid jõupingutusi tehakse nende
elementaarsete erinevuste selgitamiseks inimestele, et sellised
stsenaariumid nagu Itaalias või Hispaanias pole siin realistlikud?

\hypertarget{viited}{%
\subparagraph{\texorpdfstring{\textbf{Viited:}}{Viited:}}\label{viited}}

{[}1{]} Fachwörterbuch Infektionsschutz und Infektionsepidemiologie.
\href{https://www.rki.de/DE/Content/Service/Publikationen/Fachwoerterbuch_Infektionsschutz.html}{Fachwörter
-- Definitionen -- Interpretationen}. Robert Koch-Institut, Berlin 2015.
(abgerufen am 26.3.2020)

{[}2{]} Killerby et al., Human Coronavirus Circulation in the United
States 2014--2017. J Clin Virol. 2018, 101, 52-56

{[}3{]} Roussel et al. SARS-CoV-2: Fear Versus Data. Int. J. Antimicrob.
Agents 2020, 105947

{[}4{]} Charisius, H.
\href{https://www.sueddeutsche.de/gesundheit/covid-19-coronavirus-testverfahren-1.4855487}{Covid-19:
Wie gut testet Deutschland?} Süddeutsche Zeitung. (abgerufen am
27.3.2020)

{[}5{]} Johns Hopkins University,
\href{https://coronavirus.jhu.edu/map.html}{Coronavirus Resource
Center}. 2020. (abgerufen am 26.3.2020)

{[}6{]} S1-Leitlinie 054-001,
\href{https://www.awmf.org/uploads/tx_szleitlinien/054-002l_S1_Regeln-zur-Durchfuehrung-der-aerztlichen-Leichenschau_2018-02_01.pdf}{Regeln
zur Durchführung der ärztlichen Leichenschau}. AWMF Online (abgerufen am
26.3.2020)

{[}7{]} Martuzzi et al. Health Impact of PM10 and Ozone in 13 Italian
Cities. World Health Organization Regional Office for Europe. WHOLIS
number E88700 2006

{[}8{]} European Environment Agency,
\href{https://www.eea.europa.eu/themes/air/country-fact-sheets/2019-country-fact-sheets}{Air
Pollution Country Fact Sheets 2019}, (abgerufen am 26.3.2020)

{[}9{]} Croft et al. The Association between Respiratory Infection and
Air Pollution in the Setting of Air Quality Policy and Economic Change.
Ann. Am. Thorac. Soc. 2019, 16, 321--330.

{[}10{]} United Nations, Department of Economic and Social Affairs,
Population Division. Living Arrange­ments of Older Persons: A Report on
an Expanded International Dataset (ST/ESA/SER.A/407). 2017

{[}11{]} Deutsches Ärzteblatt,
\href{https://www.aerzteblatt.de/nachrichten/111029/Ueberlastung-deutscher-Krankenhaeuser-durch-COVID-19-laut-Experten-unwahrscheinlich}{Überlastung
deutscher Krankenhäuser durch COVID-19 laut Experten unwahrscheinlich},
(abgerufen am 26.3.2020)

\begin{center}\rule{0.5\linewidth}{\linethickness}\end{center}

\textbf{Professor Sucharit Bhakdi selgitas oma avatud kirja:}

\begin{center}\rule{0.5\linewidth}{\linethickness}\end{center}

Share this letter on:
\href{https://twitter.com/intent/tweet?url=https://swprs.org/professor-sucharit-bhakdi-avalik-kiri-angela-merkelile/}{Twitter}
/
\href{https://www.facebook.com/share.php?u=https://swprs.org/professor-sucharit-bhakdi-avalik-kiri-angela-merkelile/}{Facebook}

Back to main article:
\href{https://swprs.org/a-swiss-doctor-on-covid-19/}{A Swiss Doctor on
Covid-19}

\hypertarget{swiss-policy-research}{%
\subsubsection{Swiss Policy Research}\label{swiss-policy-research}}

\begin{itemize}
\tightlist
\item
  \href{https://swprs.org/kontakt/}{Kontakt}
\item
  \href{https://swprs.org/uebersicht/}{Übersicht}
\item
  \href{https://swprs.org/donationen/}{Donationen}
\item
  \href{https://swprs.org/disclaimer/}{Disclaimer}
\end{itemize}

\hypertarget{english}{%
\subsubsection{English}\label{english}}

\begin{itemize}
\tightlist
\item
  \href{https://swprs.org/contact/}{About Us / Contact}
\item
  \href{https://swprs.org/media-navigator/}{The Media Navigator}
\item
  \href{https://swprs.org/the-american-empire-and-its-media/}{The CFR
  and the Media}
\item
  \href{https://swprs.org/donations/}{Donations}
\end{itemize}

\hypertarget{follow-by-email}{%
\subsubsection{Follow by email}\label{follow-by-email}}

Follow

\href{https://wordpress.com/?ref=footer_custom_com}{WordPress.com}.

\protect\hyperlink{}{Up ↑}

Post to

\protect\hyperlink{}{Cancel}

\includegraphics{https://pixel.wp.com/b.gif?v=noscript}
