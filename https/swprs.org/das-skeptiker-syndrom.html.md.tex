\protect\hyperlink{content}{Skip to content}

\href{https://swprs.org/}{}

\protect\hyperlink{search-container}{Search}

Search for:

\href{https://swprs.org/}{\includegraphics{https://swprs.files.wordpress.com/2020/05/swiss-policy-research-logo-300.png}}

\href{https://swprs.org/}{Swiss Policy Research}

Geopolitics and Media

Menu

\begin{itemize}
\tightlist
\item
  \href{https://swprs.org}{Start}
\item
  \href{https://swprs.org/srf-propaganda-analyse/}{Studien}

  \begin{itemize}
  \tightlist
  \item
    \href{https://swprs.org/srf-propaganda-analyse/}{SRF / ZDF}
  \item
    \href{https://swprs.org/die-nzz-studie/}{NZZ-Studie}
  \item
    \href{https://swprs.org/der-propaganda-multiplikator/}{Agenturen}
  \item
    \href{https://swprs.org/die-propaganda-matrix/}{Medienmatrix}
  \end{itemize}
\item
  \href{https://swprs.org/medien-navigator/}{Analysen}

  \begin{itemize}
  \tightlist
  \item
    \href{https://swprs.org/medien-navigator/}{Navigator}
  \item
    \href{https://swprs.org/der-propaganda-schluessel/}{Techniken}
  \item
    \href{https://swprs.org/propaganda-in-der-wikipedia/}{Wikipedia}
  \item
    \href{https://swprs.org/logik-imperialer-kriege/}{Kriege}
  \end{itemize}
\item
  \href{https://swprs.org/netzwerk-medien-schweiz/}{Netzwerke}

  \begin{itemize}
  \tightlist
  \item
    \href{https://swprs.org/netzwerk-medien-schweiz/}{Schweiz}
  \item
    \href{https://swprs.org/netzwerk-medien-deutschland/}{Deutschland}
  \item
    \href{https://swprs.org/medien-in-oesterreich/}{Österreich}
  \item
    \href{https://swprs.org/das-american-empire-und-seine-medien/}{USA}
  \end{itemize}
\item
  \href{https://swprs.org/bericht-eines-journalisten/}{Fokus I}

  \begin{itemize}
  \tightlist
  \item
    \href{https://swprs.org/bericht-eines-journalisten/}{Journalistenbericht}
  \item
    \href{https://swprs.org/russische-propaganda/}{Russische Propaganda}
  \item
    \href{https://swprs.org/die-israel-lobby-fakten-und-mythen/}{Die
    »Israel-Lobby«}
  \item
    \href{https://swprs.org/geopolitik-und-paedokriminalitaet/}{Pädokriminalität}
  \end{itemize}
\item
  \href{https://swprs.org/migration-und-medien/}{Fokus II}

  \begin{itemize}
  \tightlist
  \item
    \href{https://swprs.org/covid-19-hinweis-ii/}{Coronavirus}
  \item
    \href{https://swprs.org/die-integrity-initiative/}{Integrity
    Initiative}
  \item
    \href{https://swprs.org/migration-und-medien/}{Migration \& Medien}
  \item
    \href{https://swprs.org/der-fall-magnitsky/}{Magnitsky Act}
  \end{itemize}
\item
  \href{https://swprs.org/kontakt/}{Projekt}

  \begin{itemize}
  \tightlist
  \item
    \href{https://swprs.org/kontakt/}{Kontakt}
  \item
    \href{https://swprs.org/uebersicht/}{Seitenübersicht}
  \item
    \href{https://swprs.org/medienspiegel/}{Medienspiegel}
  \item
    \href{https://swprs.org/donationen/}{Donationen}
  \end{itemize}
\item
  \href{https://swprs.org/contact/}{English}
\end{itemize}

\protect\hyperlink{}{Open Search}

\hypertarget{das-skeptiker-syndrom}{%
\section{Das Skeptiker-Syndrom}\label{das-skeptiker-syndrom}}

Dr. Edgar Wunder

1998 / 2019

\begin{center}\rule{0.5\linewidth}{\linethickness}\end{center}

»Das Skeptiker-Syndrom« ist ein von Dr. Edgar Wunder 1998 erstmals
veröffentlichter Aufsatz, in dem er die prägenden Mentalitätsmuster der
sogenannten »Skeptiker« aus soziologischer Sicht heraus­ar­beitet. Dr.
Wunder war Mitgründer und langjähriges Vorstandsmitglied der deutschen
Skeptiker-Orga­ni­sation GWUP, zählt inzwischen jedoch zu ihren
bekanntesten Kritikern. Die Neu­ver­öffent­lichung seiner
aufschluss­reichen Untersuchung soll dazu dienen, diese heutzutage
vorwiegend im Internet -- etwa auf Wikipedia -- agierende Gruppierung
und ihre Weltanschauung besser einordnen zu können.

\href{https://swprs.files.wordpress.com/2019/09/das-skeptiker-syndrom.pdf}{PDF-Version}

\begin{center}\rule{0.5\linewidth}{\linethickness}\end{center}

\hypertarget{vorbemerkung}{%
\paragraph{Vorbemerkung}\label{vorbemerkung}}

Ich bin eines von 19 Gründungsmitgliedern der im Oktober 1987
gegründeten „Skeptiker``-Organisation „Gesellschaft zur
wissenschaftlichen Untersuchung von Parawissenschaften e.V.`` (GWUP),
war von 1992 bis Dezember 1998 deren „Fachbereichsleiter`` für das Thema
Astrologie, von 1996 bis Juli 1998 Mitglied des Verwaltungsrats der
GWUP, von 1994 bis Dezember 1998 Mitglied der Redaktion der von der GWUP
herausgegebenen Zeitschrift Skeptiker und -- last not least -- von
September 1996 bis Dezember 1998 der verantwortliche Redaktionsleiter
des Skeptiker.

Vor diesem Hintergrund kenne ich die GWUP aus der Innenperspektive so
gut wie sicher nur sehr wenige andere. Laut Satzung ist es das
selbsterklärte Ziel der GWUP, „behauptete paranormale Phänomene ohne
Voreingenommenheit mit wissenschaftlichen Methoden zu untersuchen, sowie
solche Untersuchungen zu fördern und über deren Ergebnisse zu
berichten``, „kritisches Denken zu fördern``, eine entsprechende
„Aufklärung der Öffentlichkeit`` zu betreiben und „mit gleichgesinnten
Personen, Organisationen und Institutionen zusammenzuarbeiten``. Nach
Aussagen des ehem. GWUP-Vorstandsmitglieds Rainer Rosenzweig in einem
Editorial der Zeitschrift Skeptiker (Heft 4/97) bedeute dies, „eine
echte neutrale Mitte einzunehmen``, d.h. „Urteile, positive wie
negative, erst nach einer sorgfältigen Überprüfung, und dann mit der
gebotenen Umsicht zu treffen.``

Lobenswerte Ziele, aber meine Erfahrungen mit vielen Mitgliedern der
GWUP sind leider andere.

Es gibt innerhalb der GWUP eine ganze Reihe von Mitgliedern, die ohne
hinreichende fachliche Kenntnis der jeweiligen Materie eine Art
Weltanschauungskampf gegen alles führen wollen, was sie mit dem Begriff
„paranormal`` assoziieren, die dabei auch (bewusst oder unbewusst) eine
selektiv-einseitige Darstellung der Fakten und Argumente sowie zuweilen
auch emotional-unsachliche rhetorische Taktiken in Kauf nehmen, während
sie an wissenschaftlichen Untersuchungen zu Parawissenschaften höchstens
insofern interessiert sind, als deren Ergebnisse „Kanonenfutter`` für
öffentliche Kampagnen liefern könnten.

Mitte der 90er Jahre wurde mir in meiner Position als einer der
führenden GWUP-Funktionäre zunehmend bewusst, dass die diesbezügliche
Diskrepanz zwischen dem Anspruch (bzw. teils auch dem Selbstbild) und
der Wirklichkeit der GWUP derart massiv war, dass es nicht mehr als
bedauerliche Pathologie auf der individuellen Ebene einzelner Mitglieder
abgetan werden konnte. Vielmehr handelte es sich ganz offenbar um ein
strukturelles Merkmal der ``Skeptiker''-Bewegung, wie auch ein Vergleich
mit ähnlichen Organisationen in anderen Ländern ergab.

Als Soziologe beschloss ich, meine Stellung als GWUP-Funktionär dazu zu
nutzen, um durch eine systematische Untersuchung der internen
Kommunikation in ``Skeptiker''-Kreisen ein Merkmals-Set typischer
Mentalitätsmuster herauszuarbeiten, das die real existierende
``Skeptiker''-Bewegung zu prägen scheint, unabhängig davon, ob es mit
der Selbstdarstellung der Bewegung nach Außen hin konvergiert oder
nicht.

Dazu wurden in der Zeit von Februar 1997 bis März 1998 sorgfältig und
planmäßig mündliche wie schriftliche Äußerungen von GWUP-Mitgliedern
gesammelt, dokumentiert und kategorisiert, die situativ unter
Bedingungen erfolgten, bei denen sich die Betreffenden ``unter sich''
glaubten und die Gefahr gering schien, dass sie in den Modus der
Selbstdarstellung nach Außen oder der Rezitation idealisierter
Selbstbilder verfielen (z.B. interne Vorstandssitzungen, E-Mails,
Privatgespräche etc.), selbstverständlich ohne dass die beobachteten
Akteure von diesem Projekt wussten.

Das Ergebnis der Studie war ein als polythetisches Set konzipierter
Merkmalskatalog, den ich das ``Skeptiker-Syndrom'' nannte. Im April 1998
schrieb ich auf dieser Grundlage den ersten Entwurf des vorliegenden
Artikels, der bis auf weiteres noch unter Verschluss blieb und später
noch durch einige weitere aktuellere Beispiele ergänzt wurde. Im Juni
1998 trug ich diesen Merkmalskatalog und die meisten der nachfolgend
dargestellten Ergebnisse und Überlegungen zur ``Skeptiker''-Bewegung bei
einem zweistündigen Kolloquiumsvortrag am Freiburger Institut für
Grenzgebiete der Psychologie und Psychohygiene (IGPP) erstmals
ausführlich öffentlich vor.

Es war nun die spannende Frage, ob die Bewegung in der Lage sein würde,
zu ertragen, dass einer ihrer Funktionäre ein derart
kritisch-schonungsloses Bild der eigenen Gruppe zeichnete und er dies --
vor allem -- in seinem Einfluss- und Zuständigkeitsbereich (also
insbesondere der Vereinszeitschrift Skeptiker) mit sehr konkreten
Maßnahmen verband, die auf eine Überwindung jenes in
``Skeptiker''-Gruppen sozial kultivierten mentalen Sets zielten. Sie war
es nicht.

Im August 1998 wurde ich vom GWUP-Vorstand ohne vorherige Konsultation
aus dem GWUP-Verwaltungsrat ``entfernt'' und wenige Monate später auch
als ``Skeptiker''-Redaktionsleiter und aus allen anderen Funktionen
entlassen. Daraufhin veröffentlichte ich im Februar 1999 den schon lange
vorher erarbeiteten Text ``Das Skeptiker-Syndrom'' nun auch im Internet
(ergänzt durch einen im Januar 1999 zusätzlich verfassten längeren
Anhang, in dem ich meine persönliche biographische Geschichte in und mit
der ``Skeptiker''-Bewegung schilderte).

Obwohl zu diesem Zeitpunkt schon lange klar war, dass die
``Skeptiker''-Bewegung in dieser Hinsicht aus strukturellen Gründen
nicht reformierbar ist und ich deshalb austreten würde, schob ich die
Austrittserklärung noch einige Zeit auf. Denn ich war daran
interessiert, wie die bevorstehende allgemeine Mitgliederversammlung im
Mai 1999 zu der Frage stehen würde, inwieweit solche GWUP-kritischen
Äußerungen zumindest als ``einfaches Mitglied'' erlaubt seien.

Wie von mir aufgrund der bisherigen Erfahrungen nicht anders erwartet,
beschloss die Mitgliederversammlung in der Tat mit großer Mehrheit, mich
umgehend auch als einfaches Mitglied aus der GWUP auszuschließen. Ich
bin meinem Untersuchungsobjekt, der GWUP, dafür dankbar, dass so mein
seit August 1997 laufendes quasi-experimentelles Programm, welche Normen
und Sanktions­mechanismen die ``Skeptiker''-Bewegung steuern, in vollem
Umfang umgesetzt werden konnte.

\hypertarget{skeptiker--begriffsprobleme-und-die-folgen}{%
\paragraph{Skeptiker? -- Begriffsprobleme und die
Folgen}\label{skeptiker--begriffsprobleme-und-die-folgen}}

GWUP-Mitglieder nennen sich in der Regel „Skeptiker`` und fühlen sich
als Teil einer weltweiten „Skeptiker-Bewegung``, die sich den „Kampf
gegen das Paranormale und Pseudowissenschaften`` auf die Fahnen
geschrieben hat („battle against the paranormal and pseudoscience`` --
so die weltweit führende amerikanische „Skeptiker``-Organisation CSICOP
in einer Presseerklärung zum „2. Welt-Skeptiker-Kongress`` in Heidelberg
im Juli 1998).

Die Probleme beginnen damit, dass mit dem Begriff „Skeptiker``
(mindestens) zwei verschiedene semantische Dimensionen verbunden sind,
die sowohl von der Öffentlichkeit, aber vor allem auch innerhalb von
sog. „Skeptiker``-Organisationen immer wieder durcheinander gebracht
werden.

Die beiden Dimensionen sind in der dargestellten Graphik verdeutlicht:
Es gibt einerseits die Dimension „belief -- unbelief`` (z.B.
hinsichtlich des „Paranormalen``, was immer das sein mag), andererseits
die Dimension „dogmatism -- open mindedness / critical thinking``. Dabei
ist zu betonen, dass mit „unbelief`` keineswegs nur „non-belief``
gemeint ist, sondern der „unbelief``, verstanden als „disbelief``,
durchaus selbst ein belief-system darstellt.

\includegraphics{https://swprs.files.wordpress.com/2019/03/grafik1-e1551987680960.gif?w=400\&h=350}

Eine in „Skeptiker``-Organisationen weit verbreitete und folgenreiche
Kurzsichtigkeit besteht nun darin, gar nicht wahrzunehmen, dass diese
beiden Dimensionen nicht identisch sind, dass also „critical thinking``
keineswegs zwingend mit „unbelief`` zusammenfallen muss, genau so wenig
wie „dogmatism`` zwingend mit „belief``. Vielmehr können sich empirisch
Individuen in allen vier Quadranten der Graphik aufhalten. Im Diskurs
unter Mitgliedern von „Skeptiker``-Organisationen ist es aber üblich,
den Begriff „Skeptiker`` sowohl in der Bedeutung „kritisch denkende
Menschen`` als auch in der Bedeutung „nicht an Paranormales glaubende
Personen`` zu verwenden, beides wird also gleichgesetzt.

Zur Verdeutlichung mag eine Umfrage unter der Leserschaft des Skeptical
Inquirer dienen, die der CSICOP-Vorsitzende Paul Kurtz im Frühjahr 1998
durchführen ließ: Dort waren auf die Frage „Which of the following would
you say best describes your point of view?`` fünf Antwortalternativen
vorgegeben (in Klammern die Ergebnisse der Umfrage): „Strong skeptic``
(77,5 \%), „Mild skeptic`` (16,2 \%), „Neutral`` (2,4 \%), „Mild
believer`` (1,0 \%), „Strong believer`` (0,4 \%).

Dem kann wohl entnommen werden, dass erstens für Kurtz der Begriff
„skeptic`` das Gegenteil von „believer`` meint, er also für „unbelief``
steht (oder zumindest, dass Kurtz ein derartiges Kategorienschema in den
Köpfen der Leser des Skeptical Inquirer vermutet), zweitens, dass für
Kurtz die Position eines „skeptic`` nicht „neutral`` ist. Drittens, dass
sich zumindest unter CSICOP-Anhängern empirisch nur eine verschwindende
Minderheit als „neutral`` versteht.

Würde man im Kontext dieser Umfrage „skeptic`` im Sinne von „open
mindedness / critical thinking`` verstehen, wären Begriffe wie „mild
skeptic`` oder „neutral`` ziemlich sinnlos bzw. schwer verständlich.
Ganz offensichtlich ist mit „skeptic`` hier ein „unbeliever`` bezüglich
des „Paranormalen`` gemeint. (Zahlreiche weitere Textstellen aus
„Skeptiker``-Publikationen ließen sich anführen, in denen der
Skeptiker-Begriff ganz augenscheinlich in dieser Bedeutung verwendet
wird.)

Andererseits gibt es zum Beispiel folgendes Verständnis des Begriffs,
das „Skeptiker``-Organisationen nicht selten in ihren öffentlichen
Selbstdarstellungen anführen: „Ein Skeptiker in unserem Verständnis
nimmt so wenig wie möglich als gegeben hin, sondern ist bereit, jede
Aussage zu hinterfragen und zu prüfen. Insbesondere ist er auch bereit,
die eigene Meinung einer kritischen Prüfung zu unterziehen. Mit dieser
Einstellung steht der Skeptizismus im Gegensatz zum Dogmatismus.
Skeptizismus heißt also nicht, andere Meinungen blind abzulehnen oder
gar von vornherein die Existenz von paranormalen Phänomenen zu
leugnen``.

(Dieser Satz ist aus der offiziellen GWUP-Vorstellungsbroschüre
entnommen und wurde von mir selbst 1996 für die GWUP verfasst -- in
einem normativen Sinne, wie es in „Skeptiker``-Organisationen eigentlich
sein sollte, nicht unbedingt als Beschreibung eines realen Zustandes.)

Die Frage ist nun, im Sinne welcher der beiden Begriffsverständnisse die
real existierende „Skeptiker``-​Bewegung zusammengesetzt ist. Anders
formuliert: Bildet die „belief / unbelief``-​Dimension die
Demarkationslinie für die Mitgliedschaft jener Bewegungen, oder ist es
die „dogmatism / open mindedness-critical thinking``-Dimension? Bezogen
auf die Graphik: Welche der in der Abbildung dargestellten beiden Mengen
entspricht der realen Zusammensetzung z.B. der GWUP?

Da zumindest mir kein einziges Mitglied der GWUP bekannt ist, das man im
rechten oberen Quadranten ansiedeln könnte, jedoch eine ganze Reihe von
Mitgliedern, die wohl unzweifelhaft im linken unteren Quadranten
einzuordnen sind (und die intern teilweise nicht einmal davor
zurückschrecken, ihre eigene Position selbst als „ideologisch`` zu
bezeichnen!), kann meines Erachtens kein ernsthafter Zweifel daran
bestehen, dass die real existierende GWUP der in der Graphik unten
dargestellten Menge entspricht.

Dies hat Konsequenzen. Denn es bedeutet, dass die Kohäsion der Gruppe
gefährdet und sie vom Auseinanderfallen bedroht ist für den Fall, dass
eine ernsthafte, ergebnisoffene, gleichberechtigte und kollegiale
Diskussion mit Personen im rechten oberen Sektor stattfinden soll, denn
die Personen im linken unteren Quadranten befürchten dann eine „Aufgabe
des skeptischen (unbelief-)Profils`` oder gar eine Infragestellung der
Gruppenidentität.

Gleiches kann passieren, wenn Gruppenmitglieder im rechten unteren
Sektor Personen im linken unteren Sektor offen kritisieren und/oder
hervorheben, dass gewisse Ergebnisse empirischer Untersuchungen dem
„unbelief``-Überzeugungssystem zu widersprechen scheinen und deshalb
eine ernsthafte, offene, informierte wissenschaftliche
Auseinandersetzung anmahnen.

Das Resultat sind erhebliche Spannungen und Konflikte in der Gruppe, die
zwangsläufig den Vorstand einer derartigen Vereinigung beschäftigen
müssen, da unschwer zu erkennen ist, dass eine wie auch immer geartete
Infragestellung oder gar Verschiebung der Gruppengrenzen in der Graphik
zu schwerwiegenden Verwerfungen, ja Austrittswellen führen könnte.

Der Vorstand wird also im wesentlichen den Status quo der Gruppe in der
Graphik zu erhalten versuchen und jene, die in seinen Augen diesen
Status quo gefährden könnten, mit Sanktionen bedrohen und notfalls mit
Gewalt entsprechende Maßnahmen und „Säuberungen`` durchführen. Was sich
im Jahr 1998 innerhalb der GWUP abgespielt hat (und oben angedeutet
wurde), ist in dieser Hinsicht geradezu ein Lehrbuchbeispiel für eine
derartige Dynamik.

\hypertarget{das-skeptiker-syndrom-als-idealtypisches-polythetisches-set}{%
\paragraph{Das Skeptiker-Syndrom als idealtypisches polythetisches
Set}\label{das-skeptiker-syndrom-als-idealtypisches-polythetisches-set}}

Um zu verstehen, warum hier rasch ein die Stabilität der Gruppe
insgesamt gefährdendes Bedrohungspotential gesehen wird, müssen einige
typische Merkmalscharakteristiken aufgezählt werden, die insbesondere
die Personen im linken unteren Quadranten der Graphik kennzeichnen. Es
handelt sich um ein idealtypisches polythetisches Set, das ich als
„Skeptiker-Syndrom`` bezeichne.

Damit ist gemeint, dass das Syndrom in Bezug auf ein konkretes
Individuum bereits dann als gegeben angesehen werden muss, wenn einige
der nachfolgend genannten Merkmale erfüllt sind (es müssen nicht alle in
jedem Einzelfall zutreffen).

Gleichzeitig handelt es sich um ein emergentes Phänomen, d.h. es
entsteht etwas Neues, wenn viele der Merkmale in ihrer spezifischen
Kombination und inneren Relationierung zusammenkommen: die Mentalität
des idealtypischen ``Skeptikers'' als soziokulturelle Realität, die
gerade in der sozialen Vergemeinschaftung (in einer
``Gesinnungsgemeinschaft'') immer wieder neu erzeugt, bekräftigt und
stabilisiert wird.

Insofern haben wir es vorwiegend mit einem sozialen Phänomen zu tun,
nicht bloß mit Einstellungen einzelner isolierter Individuen. Die
``Skeptiker''-Bewegung ist jener sozialer Ort, an dem dieses spezifische
Set von Mentalitätsmustern tradiert und reproduziert wird.

Was sind nun die einzelnen Elemente des Merkmals-Sets dieses Syndroms?

\textbf{(1)} Jene „Skeptiker`` sehen das primäre oder sogar einzige Ziel
der Gruppe in Lobby- und Öffentlichkeitsarbeit mit dem Ziel, gewisse
„paranormale`` Vorstellungen in der Bevölkerung zurückzudrängen oder den
aktiven Vertretern solcher Überzeugungen „das Handwerk zu legen``. Es
geht insofern um Mission und Advokatentum, bei dem \textbf{(2)} die
Durchführung von eigenen wissenschaftlichen Untersuchungen als relativ
überflüssig erachtet wird, da ja eh klar sei, „daß alles Quatsch ist``.

(Da die Kenntnis relevanter Fakten und wissenschaftlicher Untersuchungen
zum jeweiligen Thema unter jenen Personen meist nicht allzu groß ist,
beschränkt sich dann die „Öffentlichkeitsarbeit`` inhaltlich nicht
selten auf die Popularisierung des Namens der eigenen Organisation in
Verbindung mit bloßen Meinungen oder bei anderen ausgeborgten Fakten.)

\textbf{(3)} Die eigene Gruppe wird nicht als „wissenschaftliche
(Forschungs-)Gemeinschaft`` verstanden, sondern als soziale Bewegung,
als „verschworene (Gesinnungs-)Gemeinschaft`` mit letztlich politischen
Zielen, nämlich der eigenen Vorstellung von „Rationalismus`` in der
gesamten Gesellschaft zum Durchbruch zu verhelfen. Man müsse sich
hinsichtlich des Vorgehens und anderer Fragen deshalb auch bei
politischen Parteien ein Vorbild nehmen, nicht etwa bei
wissenschaftlichen Gesellschaften.

\textbf{(4)} Im Rahmen einer solchen Auffassung befindet sich die eigene
Gruppe in einer steten Kampfsituation, bei der interne
Meinungsverschiedenheiten nur als hinderlich empfunden werden und
Geschlossenheit zumindest nach außen hin erwartet wird. Entsprechender
Konformitätsdruck wird in der „ingroup`` ausgeübt. Eine sich in einer
solchen Kampfsituation sehende Gruppe hat naturgemäß auch für
demokratische Abstimmungen und Verfahren in den eigenen Reihen wenig
übrig, da sie -- ähnlich wie bei einer Armee -für das eigentliche Ziel,
dem geschlossenen Wirken nach Außen, nur als Behinderung und
kontraproduktiv empfunden werden.

Als eingetragener Verein ist die GWUP zwar (etwa im Unterschied zum
amerikanischen CSICOP) formal demokratisch strukturiert, doch die
Realität sieht so aus, dass z.B. nach Auffassung eines
GWUP-Vorstandsmitglieds Mitgliederversammlungen nur dazu dienen sollen,
„um gemeinsam Kraft zu schöpfen`` und in Harmonie das
„Zusammengehörigkeitsgefühl`` zu stärken. Kontroverse Diskussionen,
Debatten oder gar Abstimmungen werden diesem Zweck nur als
zuwiderlaufend und folglich nach Möglichkeit zu verhindernd angesehen.

So gab es in meiner Erinnerung in der gesamten 12jährigen Geschichte der
GWUP bei Vorstandswahlen auch noch niemals zwei konkurrierende
Kandidaten um ein Vorstandsamt, und die entsprechenden Personen wurden
stets vom Vorstand selbst vorgeschlagen. Real praktizierte
Vereinsdemokratie sei, so mir gegenüber ein
GWUP-Vorstandsmitglied„unnötiger Luxus``, auf den man getrost verzichten
könne, da die Aufgaben der GWUP andere seien.

Dem Skeptiker-Syndrom unterliegende Personen sind nicht zuletzt auch
\textbf{(5)} an der häufigen Verwendung des Wortes „wir`` (anstelle von
„ich``) zu erkennen: Es geht ständig darum, dass „wir`` gegen „die``
antreten und zusammenhalten müssten; und wenn „wir`` untereinander
kontrovers diskutierten, würden sich „die`` nur ins Fäustchen lachen
usw. (ingroup-outgroup-Polarisierung). Deshalb müssten Kontroversen
innerhalb der Gruppe auch so schnell wie möglich beendet werden.

Während \textbf{(6)} nach Außen hin Angriff und Kritik groß geschrieben
wird, herrscht im Innern eine bereits dysfunktionale
Konfliktvermeidungsstrategie beinahe um jeden Preis, zumindest was die
Führungsgremien betrifft.

\textbf{(7)} „Outgroup``-Positionen hingegen dürfe man keinesfalls ein
Forum geben, weder in Publikationen noch bei Tagungen, denn dies wäre ja
„Werbung`` für den Gegner, der ja schon „genug Möglichkeiten hätte``,
man dürfe ihn so „nicht weiter aufwerten``.

Dass groupthink-Symptome unter solchen Bedingungen gut gedeihen, braucht
nicht weiter betont zu werden. Ich habe innerhalb der GWUP
Gremiensitzungen erlebt, bei denen sich alle Teilnehmer gegenseitig
versicherten, dass eine bestimmte Studie „Unsinn`` und „widerlegt`` sei,
ohne dass auch nur ein einziger Teilnehmer jene Studie gelesen hätte,
relevante Argumente oder eine „Widerlegung`` hätte anführen können.

\textbf{(8)} Sogar entdeckte, teils peinliche nachweisliche Fehler und
Falschbehauptungen von einzelnen Mitgliedern werden organisationsintern
kaum kritisiert (und schon gar nicht öffentlich!), sondern werden
geduldet, solange sie hinsichtlich ihrer Zielrichtung den eigenen
Überzeugungen nicht zuwider laufen. „Hauptsache dagegen!`` scheint für
viele die Devise zu sein.

So war es beispielsweise möglich, dass ein früheres GWUP-Mitglied
jahrelang Gauquelins These eines „Mars-Effekts`` mit nachweislich
falschen Argumenten heftig attackierte. Sogar als diese Person (aus
anderen Gründen) nicht mehr GWUP-Mitglied war, sah sich außer mir keiner
zu einer kritischen Aufarbeitung genötigt. In vielen anderen Beispielen
haben mir Mitglieder unter vier Augen gesagt, dass sie diese und jene
Behauptungen anderer Mitglieder für nachweislich falsch hielten, sie
aber nicht offen kritisieren wollten, „um der skeptischen Bewegung nicht
zu schaden``.

\textbf{(9)} Empfundene Gefahren- und Bedrohungspotentiale spielen eine
große Rolle für jene „Skeptiker`` und ihre Motivation. „Defending the
rational world from a rising tide of nonsense`` (Paul Kurtz) sei eine
für den zukünftigen Bestand der Gesellschaft und der Menschheit
überlebenswichtige Aufgabe, die alle Anstrengungen erfordere.

In diesem Zusammenhang werden auch \textbf{(10)} die gesellschaftliche
Bedeutung und die Einflussmöglichkeiten der eigenen Gruppe, also der
„Skeptiker``-Organisation, von vielen Mitgliedern maßlos überschätzt.
Man sieht sich als einmalige und unverzichtbare Elite, von deren Agieren
die weitere Entwicklung der Gesellschaft maßgeblich abhänge.

\textbf{(11)} Dies geht einher mit einer immer wieder geäußerten starken
emotional-persönlichen Betroffenheit („mir lief es heiß und kalt den
Rücken herunter``), wenn jene „Skeptiker`` z.B. in der Bekanntschaft mit
diversen „esoterischen`` Vorstellungen und Praktiken konfrontiert
werden. Es sei \textbf{(12)} eine große emotionale Befriedigung,
wenigstens einmal im Jahr als Teilnehmer einer GWUP-Konferenz „unter
sich`` zu sein, und sich abseits einer vom Irrationalismus geplagten
Welt gegenseitig bestärken zu können -- weshalb kontroverse Diskussionen
mit „Nicht-Skeptikern`` auf einer solchen Tagung als „störend``
empfunden und strikt abgelehnt werden. Als besonders
gemeinschaftsstiftend wird in diesem Zusammenhang offenbar \textbf{(13)}
auch das gemeinsame „Sich-empören-über \ldots{}`` empfunden.

Überhaupt sei (\textbf{14}) nur sinnvoll, sich mit solchen
parawissenschaftlichen Vorstellungen zu beschäftigen, von denen eine
ernsthafte Gefahr für Mensch und Gesellschaft ausgehe, alles andere sei
unwesentlich. Nur in den allerseltensten Fällen wird dabei (\textbf{15})
die „Gefahr`` (genauer: die Chancen-Risiko-Relation) anhand empirischer
Studien oder Abschätzungen belegt, sondern es wird mit Einzelfällen
(deren Repräsentativität fraglich ist), subjektiven Erfahrungen,
Horrorszenarien und Betroffenheitsgefühlen argumentiert -- im Prinzip
nur spiegelbildlich zu sog. „Esoterikern``, die mit ähnlichen Argumenten
uns vom heilsbringenden Nutzen ihrer jeweiligen Systeme überzeugen
wollen.

(\textbf{16}) Fragt man jene „Skeptiker``, warum sie sich überhaupt mit
solchen Themen beschäftigen, erhält man nicht etwa die Antwort, weil sie
diese oder jene Frage interessant fänden, sondern es werden bedrohliche
Gefahren ausgemalt, weshalb man gegen gewisse Vorstellungen angehen
müsse.

Ansonsten geht es (\textbf{17}) jenen „Skeptikern`` aber nur darum, ob
eine bestimmte Behauptung „stimmt`` oder nicht. Stimmt sie nicht -- und
das glaubt man ohnehin schon zu wissen -- wird sie oft vorschnell auch
als „gefährlich`` eingestuft. Denn der Hinweis auf die „Gefährlichkeit``
wird benötigt, um letztlich politisches Handeln zu rechtfertigen, an dem
man primär interessiert ist.

Dass (\textbf{18}) die Fragen nach dem Wahrheitsgehalt und der
Chancen-Risiko-Relation relativ unabhängig voneinander sind und sich
nicht einfach aufeinander reduzieren lassen, wird kaum gesehen, genau so
wenig (\textbf{19}), dass etwa die Fragen nach der Psychologie und
Soziologie derartiger „paranormaler`` Überzeugungssysteme von zentralem
Interesse und empirisch untersuchenswert wären. Jedenfalls wird dies
nicht als Angelegenheit der GWUP angesehen. Diese Ignoranz und
einseitige Fixierung auf die Frage nach dem Wahrheitsgehalt ist
selbstverständlich auch deshalb naiv, da sich ohne Klärung der
psychosozialen Hintergründe wohl niemals eine effektive
„Aufklärungsarbeit`` wird leisten lassen.

Ohnehin gehen aber (\textbf{20}) jene Personen kaum von (für sie
offenen) Fragen, sondern vielmehr von (für sie feststehenden) Antworten
aus.

(\textbf{21}) Die Anhänger von „paranormalen`` Überzeugungen -- oder
überhaupt Andersdenkende -- werden pathologisiert. Ihnen wird ein Mangel
an kognitiven Fähigkeiten („Spinner``, „Dummköpfe``, „geisteskrank``
usw.) oder kriminelle Absichten unterstellt („Betrüger``, „Scharlatane``
usw.).

Damit einher geht (\textbf{22}) nicht selten Repressionsbereitschaft,
der Ruf nach den Gerichten, nach dem Staat, nach aggressiven Kampagnen,
um z.B. zu erreichen, dass bestimmte Personen etwa in Volkshochschulen
nicht mehr eingeladen werden u.a.m.

Auffällig ist auch, dass viele derartige „Skeptiker`` nach außen hin,
öffentlich, mit solchen Pathologisierungen ihrer „Gegner`` eher
vorsichtig sind, da sie erkannt haben, dass dies kontraproduktiv sein
kann; gruppenintern nehmen sie aber kein Blatt vor den Mund („intern
muss man das offen sagen dürfen``), woran (\textbf{23}) erkennbar ist,
dass ihre öffentlichen Erklärungen taktischen Charakter haben, aber
nicht ihren tatsächlichen Überzeugungen entsprechen.

Es ist ein Kennzeichen vorurteilsbehafteter Personen, dass sie
(\textbf{24}) an die inhärente Inferiorität einer bestimmten Gruppe
glauben bzw. dass Menschen bereits nur aufgrund ihrer
Gruppenzugehörigkeit schon negativ beurteilt werden. Es war für mich
frappierend, wie schnell eine ganze Reihe von GWUP-Mitgliedern bereits
(zuweilen drastische!) Urteile über (ihnen ansonsten unbekannte)
Personen oder gar über die (ihnen erst recht unbekannte)
wissenschaftliche Qualität von deren Arbeit fällten, sobald nur deren
Zugehörigkeit zu einer bestimmten real existierenden Gruppe bekannt oder
auch nur behauptet (!) wurde -- oder sobald die betreffende Person von
einem anderen GWUP-Mitglied kurzum mit einem bestimmten „Label``
versehen wurde (besonders beliebt ist: „Esoteriker``).

(\textbf{25}) Die im Diskurs gewählten Begriffe sind für jene
„Skeptiker`` ebenfalls typisch: Es handelt es sich um von vornherein
wertende bis diffamierende Begriffe (z.B. „Aberglaube``, „Humbug``,
„Pseudowissenschaft``, „Scharlatane``, „Sekten``, „PSI-Exponenten`` --
als Bezeichnung für Parapsychologen -- u.a.m.), nicht um weitgehend
deskriptiv-analytische Begriffe (z.B. „Parawissenschaft``, „Anomalien``,
„außergewöhnliche menschliche Erfahrungen`` u.a.).

Auch (\textbf{26}) die Zuschreibung des Begriffs „paranormal`` zu
bestimmten behaupteten Phänomenen hat hier oft bereits diffamierenden
Charakter, da der Begriff für jene Personen negativ besetzt ist und
manchmal fast synonym mit „unsinnig`` verwendet wird. Typischerweise
wird (\textbf{27}) von solchen „Skeptikern`` der Begriff
„Parawissenschaft``, sofern er verwendet wird, in der Bedeutung mit dem
Begriff „Pseudowissenschaft`` weitgehend gleichgesetzt und hier nicht
weiter differenziert.

(\textbf{28}) Damit einher geht auch mangelnde
Differenzierungsbereitschaft zwischen verschiedenen
parawissenschaftlichen Disziplinen: Es wird oft pauschal alles in einen
Topf geworfen und undifferenziert von einem „Glauben an das
Paranormale`` gesprochen (den es zu bekämpfen gelte), so als ob wir es
hier mit einem irgendwie einheitlichen Überzeugungssystem zu tun hätten
-- eine Vorstellung, die längst empirisch widerlegt ist.

(\textbf{29}) Ebenso wird bei der Wahrnehmung des gesellschaftlichen
Konfliktfelds um Parawissenschaften unzureichend differenziert: Es
herrscht stereotypes „Lagerdenken`` vor, wobei eine häufige Einteilung
die in „Wölfe`` (=„Para-Vertreter``), „Schafe`` (= die zu „schützende``
Bevölkerung) und „Hüter`` (=die organisierten „Skeptiker``) ist.

(\textbf{30}) Wer solche simplizistischen Stereotype in Frage stellt und
einen „lagerübergreifenden`` Dialog fordert, dem wird vorgeworfen, er
„setzte sich zwischen alle Stühle``, sei nur noch bedingt
vertrauenswürdig, zumindest aber „naiv``.

(\textbf{31}) Die Dämonisierungen der „anderen Seite`` gehen zudem
einher mit der Bereitschaft, sehr schnell von einer einzigen Person auf
z.B. „alle Parapsychologen`` zu generalisieren. Dies überrascht nicht,
denn in der Sozialpsychologie ist es ein typisches Merkmal dogmatischen
Denkens bzw. von „closed-mindedness``, dass Wahrnehmungen, Vorstellungen
und Urteile, die positiv bewertete Objekte betreffen, wesentlich genauer
und komplexer ausfallen als solche, die negativ bewertete Objekte
betreffen.

(\textbf{32}) Jene „Skeptiker`` haben kaum -- in der Regel gar keine --
persönlichen freundschaftlichen Kontakte zu führenden
„Parawissenschaftlern`` oder „Esoterikern``, die ja trotz inhaltlicher
Meinungsverschiedenheiten theoretisch ohne weiteres möglich wären, ja
geradezu auf der Hand liegen würden, wenn ein fairer offener Dialog
gesucht werden würde.

An solchen Kontakten haben derartige „Skeptiker`` auch gar kein
Interesse, sie nehmen (abgesehen von manchen Esoterik-Messen als
Kuriosum am eigenen Wohnort) auch (\textbf{33}) an keinen
Veranstaltungen des „anderen Lagers`` teil, da sie sich dadurch keinen
Informationsgewinn versprechen, sondern höchstens Verärgerung über „den
ganzen Unsinn``.

(\textbf{34}) Gleichzeitig lesen diese „Skeptiker`` auch keine
Publikationen aus dem parawissenschaftlichen Bereich (z.B. Zeitschrift
für Parapsychologie und Grenzgebiete der Psychologie, Journal of
Scientific Exploration), genau so wenig aus dem esoterischen Bereich
(z.B. Esotera, Magazin 2000). Nach einer von mir 1997 durchgeführten
Umfrage unter Skeptiker-Beziehern lesen ca. 90 \% der GWUP-Mitglieder
keine einzige derartige Zeitschrift. Sie sind entsprechend schlecht
informiert, und zwar sowohl über aktuelle Entwicklungen in der
„Esoterik-Szene``, als auch -- und dies ist bedeutsamer -- über diverse
Untersuchungen (bzw. auch den Diskussionsstand allgemein), wie sie immer
wieder z.B. in den genannten „parawissenschaftlichen`` Zeitschriften
publiziert werden.

Entsprechend der genannten Umfrage bildet (\textbf{35}) -- abgesehen vom
eigenen „Hausblatt`` Skeptiker und anderer „skeptischer`` Literatur --
die reguläre Informationsquelle zu Parawissenschaften für die meisten
GWUP-Mitglieder vielmehr die allgemeine Tages- und Wochenpresse sowie
populärwissenschaftliche Magazine. (Zwar werden auch wissenschaftliche
Fachzeitschriften der jeweils eigenen Disziplin -- z.B. Chemie,
Biologie, Physik usw. -- gelesen, diese enthalten aber bekanntlich kaum
irgendwelche Artikel zu parawissenschaftlichen Themen.)

Dies gilt nach meinen Erfahrungen auch für die überwiegende Zahl der
Personen mit Führungspositionen innerhalb der GWUP, z.B. für Vorstände
oder Mitglieder des Wissenschaftsrats der GWUP. (Der sog.
GWUP-„Wissenschaftsrat`` steht allerdings im wesentlichen ohnehin nur
auf dem Papier und ist quasi inaktiv, dient vielmehr nur als
akademisches „Aushängeschild``.)

(\textbf{36}) Es fehlt folglich in der Regel an grundlegendem
Faktenwissen, was überhaupt tatsächlich von parawissenschaftlicher Seite
behauptet wird und was nicht. Die Urteile rekurrieren vielmehr auf
diverse teils irreführende Stereotype, die in den Medien gängig sind.

Nach meinen Erfahrungen hat z.B. ein ganz erheblicher Anteil der
GWUP-Mitgliedschaft keine Ahnung, was etwa der Unterschied zwischen
„Tierkreiszeichen`` und „Sternbildern`` ist, was der Ausdruck „Begegnung
der dritten Art`` wirklich korrekt bedeutet oder welche verschiedenen
„parapsychologischen`` Einrichtungen in Deutschland existieren oder wie
sie institutionalisiert sind -- was viele nicht hindert, sich lautstark
zu Astrologie, Ufologie, Parapsychologie oder anderen Themen zu Wort zu
melden, großteils mit entsprechend unqualifizierten Verlautbarungen.

In diesem Kontext versteht sich (\textbf{37}) wohl auch das häufige
pauschale Berufen auf bekannte „Entlarver`` (insb. James Randi und seine
1-Million-Dollar-Wette) als Autoritäten und Vorbilder, anstatt konkrete
Argumente anzuführen.

Überhaupt ist es (\textbf{38}) beliebt, zu erklären, man ``wette'', dass
dieser und jener Effekt sich (in unbestimmter Zeit!) als Artefakt
herausstellen werde: dies ermöglicht es, hohe subjektive Sicherheit zu
demonstrierten, ohne sich mit der Materie näher beschäftigen zu müssen.

(\textbf{39}) Eigene Untersuchungstätigkeit zu Parawissenschaften tritt
in der Regel gar nicht auf, denn es sei ja ohnehin schon klar, daß alles
„Quatsch`` ist, was solle man denn noch untersuchen?

(\textbf{40}) Wenn überhaupt „Untersuchungen`` vorgenommen werden, dann
nur, um einer breiten Öffentlichkeit zu demonstrieren, was man ohnehin
schon für gesichert hält (der Ausdruck „Demonstrationen`` wäre also
angemessener), jedoch nicht, um Fragen nachzuspüren, die man noch für
offen erachtet und bei denen man ernsthaften Forschungsbedarf sieht. Im
letzteren Fall bestünde -- da die finanziellen Mittel begrenzt sind --
ein Konkurrenzverhältnis zur Öffentlichkeitsarbeit, die innerhalb der
GWUP ohne jeden Zweifel das absolute Primat genießt.

Da es innerhalb der Parawissenschaften nichts mehr ernsthaft zu
untersuchen gebe, seien entsprechende Untersuchungen Zeit- und
Geldverschwendung; die Mittel sollten besser für eine Intensivierung der
Öffentlichkeitsarbeit verwendet werden. Wenn ich die Überzeugung habe,
dass ein bestimmter Effekt nicht existiert, warum sollte ich viel Zeit
und Geld aufwenden, um diesen angeblichen Effekt zu untersuchen? Lieber
die Öffentlichkeit von meiner Meinung überzeugen. Aber das ist keine
Wissenschaft, es ist letztlich eine religiös-missionarische Haltung.

Ein Mitglied des Wissenschaftsrats (!) der GWUP (heute Leiter der
GWUP-Geschäftsstelle) sagte mir gar auf meine Anregung hin, zu
GWUP-Tagungen externe Referenten zu Präsentationen neuerer empirischer
Untersuchungen einzuladen (extern, da es GWUP-intern kaum derartige
Referenten gibt), dass empirische Untersuchungen doch ohnehin langweilig
seien, das sei „immer das gleiche``, was solle man da schon Neues
erwarten, von derartigen Präsentationen halte er nichts.

Wenn überhaupt irgendetwas untersucht wird, dann sind es (\textbf{41})
relativ leicht zu entkräftende und ohnehin schon sehr fragwürdige Fälle
(z.B. offensichtliche Scharlatanerie im Esoterik-Bereich), während um
die „härteren Nüsse`` (z.B. diverse parapsychologische Laborexperimente)
ein großer Bogen gemacht wird. Einer wissenschaftlichen Haltung wäre es
angemessen, sich den besten Argumenten der (so empfundenen)
„Gegenseite`` kritisch zuzuwenden, nicht ersatzweise den schwächsten.

(\textbf{42}) Unternimmt jemand im „anderen Lager`` wissenschaftliche
Untersuchungen zu Parawissenschaften, wird dies als Ärgernis empfunden,
das man gerne verhindern würde, wenn man es könnte, sofern der
betreffende Forscher öffentliche Mittel zur Finanzierung seiner Studie
erhält.

(\textbf{43}) Es gibt keine positive Einstellung, für wissenschaftliche
Untersuchungen von Parawissenschaften Geld auszugeben. Bedenkt man, dass
dies auf einen ganz erheblichen Teil der Mitglieder der GWUP zutrifft,
kann der Name „Gesellschaft zur wissenschaftlichen Untersuchung von
Parawissenschaften`` eigentlich nur noch als ein Etikettenschwindel
aufgefasst werden.

Man fragt sich, welche Funktion die Gruppe überhaupt für viele
Mitglieder der GWUP hat. Unzählige Male habe ich als verantwortlicher
Redaktionsleiter des Skeptiker aus der Leserschaft und aus der
Mitgliedschaft der GWUP Anfragen und Aussagen folgenden Sinngehalts
bekommen: (\textbf{44}) „Daß Parawissenschaften Quatsch sind, weiß ich
ohnehin. Die GWUP brauche ich vor allem deshalb, um gut begründen zu
können, warum es Quatsch ist``.

Eine wissenschaftliche Haltung verbirgt sich dahinter freilich nicht. Es
geht für viele Mitglieder der GWUP offensichtlich darum, in der Gruppe
soziale Sicherheit für ihre schon fest bestehenden Überzeugungen und
Vorurteile zu gewinnen, sie sozial durch eine Gruppe bekräftigt zu
bekommen, die als autoritativ empfunden wird, sowie Argumentationshilfen
für entsprechende Diskussionen im eigenen sozialen Umfeld zu erhalten.

(\textbf{45}) Ein weiteres Merkmal des Skeptiker-Syndroms scheint mir
ein besonderes Vorsichhertragen, ja sogar Stolz auf den
„Skeptiker``-Begriff zu sein. Die Frage „Wer sind die Skeptiker?``
beantworten solche Personen häufig kurzum mit „Wir sind es`` -- und
führen damit eine dritte Bedeutung des „Skeptiker``-Begriffs ein, indem
sie ihn (\textbf{46}) schlicht als Bezeichnung für die „ingroup``
verwenden.

Man muss sich genau vergegenwärtigen, was letztlich dadurch geschieht,
indem (\textbf{47}) die drei „Skeptiker``-Bedeutungen unreflektiert
gleichgesetzt werden: „kritisch denkende Menschen`` = „nicht an
Paranormales Glaubende`` = „ingroup``. Die Mitglieder der eigenen Gruppe
(„Skeptiker``) werden dadurch nicht nur klammheimlich per definitionem
zu kritisch denkenden Menschen („Skeptikern``) erklärt, sondern auch
deren inhaltliche Position („Skeptiker`` als „unbeliever``) festgelegt.

Wird von außen Kritik an „Skeptikern`` (ingroup) geübt, lautet
(\textbf{48}) die Erwiderung, dass „Skeptiker`` ja „in Wirklichkeit``
nichts weiter als „kritisch denkende Menschen`` bedeute und insofern die
Kritik an den „Skeptikern`` (nun wieder „ingroup``) ungerechtfertigt
sei.

Umgekehrt kann jemand (\textbf{49}) rasch zur „outgroup``
(„Nicht-Skeptiker``) erklärt werden, indem ihm „Glaube an Paranormales``
(=„Nicht-Skeptiker``) unterstellt wird, ohne dass eine Prüfung
hinsichtlich der verbleibenden „Skeptiker``-Dimension des kritischen
Denkens noch vorgenommen zu werden bräuchte.

Sensibilisiert auf die unterschiedlichen Bedeutungen des
„Skeptiker``-Begriffs habe ich in der GWUP derart häufig solche durch
Kontextwechsel erschlichenen Argumentationsmuster erlebt, dass ich für
die Zukunft plane, durch eine umfassende Analyse von Texten führender
Repräsentanten von „Skeptiker``-Organisationen detailliert aufzuzeigen,
wie jene Personen je nach Kontext den „Skeptiker``-Begriff in
unterschiedlicher Weise verwenden und wie sich dies auf ihre
Schlussfolgerungen auswirkt. Ich habe übrigens keinen Zweifel daran,
dass dies unreflektiert geschieht.

Einen wie auch immer gearteten apriorischen Grund für die Annahme, dass
„Skeptiker`` im ersten Sinne auch automatisch „Skeptiker`` im zweiten
Sinne seien (oder umgekehrt) oder gar zwangsläufig mit „Skeptikern`` im
dritten Sinne identisch sind, sehe ich nicht, vielmehr zahlreiche Belege
dafür, dass dies nicht der Fall ist.

Die Abgrenzung des Gegenstandsbereichs, zu dem die GWUP aktiv sein
sollte, ist ein Thema für sich. Syndrom-Skeptiker tendieren dazu,
(\textbf{50}) die Grenzen sehr weit und auch auf Religions- und
Weltanschauungsfragen auszudehnen. Dies ist nur konsequent, wenn man das
Agieren gegen Parawissenschaften als Weltanschauungskampf begreift, wie
dies jene „Skeptiker`` oft tun. Dann braucht auch keine Rücksicht mehr
darauf genommen zu werden, welche Fragen einem
empirisch-wissenschaftlichen Zugriff eigentlich noch zugänglich sind und
welche nicht. In Extremfällen kann sich dieser Kampf sogar pauschal auf
„alles Schlechte in der Welt`` beziehen.

Während manche selbsterklärte „Skeptiker`` offen fordern, dass auch in
Religions- und Weltanschauungsfragen die GWUP klar und kämpferisch
Position beziehen sollte, erkennen andere, dass dies zumindest taktisch
unklug wäre, da es die Glaubwürdigkeit der Organisation beeinträchtigen
und vermutlich gruppeninterne Spannungen hervorrufen würde (denn die
GWUP ist in weltanschaulicher Hinsicht nicht völlig homogen, wenn auch
atheistisch-naturalistisch-szientistische Positionen klar dominieren).

Folglich wird (\textbf{51}) aus taktischen (!) Gründen die Behandlung
von Religions- und Weltanschauungsfragen vermieden und hier eine
„Arbeitsteilung`` mit anderen Organisationen (in der Regel organisierten
Atheisten) angestrebt oder empfohlen. Der Geschäftsführer der GWUP
vertritt z.B.eine solche Haltung, nicht anders auch der
CSICOP-Vorsitzende Paul Kurtz.

(\textbf{52}) Die Möglichkeit bzw. Wahrscheinlichkeit, dass sich doch
noch eines der als „paranormal`` abgelehnten Phänomene irgendwann als
existent erweisen könnte, wird -- falls diese Frage überhaupt ernsthaft
gestellt wird -- als gegen Null gehend, vernachlässigbar gering bzw. als
rein hypothetisch jenseits aller ernsthaften Erwägungen angesehen.

Da vielen Mitgliedern der GWUP aus diversen öffentlichen Kontroversen
klar geworden ist, dass man bei einem allzu deutlichen Zeigen jener
subjektiven quasi absoluten Sicherheiten in einem dogmatischen Licht
erscheinen würde, haben sich derartige „Skeptiker`` vielfach angewöhnt,
im Sinne einer Rhetorik zwar stets ihre „grundsätzliche Offenheit`` zu
betonen, dem aber kaum ernsthafte Erwägungen folgen zu lassen.

Ein typisches Beispiel ist etwa eine in GWUP-Aktuell 1/98 abgedruckte
Antwort des GWUP-Geschäftsführers Amardeo Sarma zur Frage, ob er es für
möglich halte, dass sich bisher als „paranormal`` eingestufte Thesen
einmal als wahr erweisen könnten: „Ich wäre bei entsprechender Lage der
Dinge bereit, einen solchen grundlegenden Paradigmenwechsel \ldots{}
mitzumachen. Dass diese Lage aber eintritt würde mich mehr überraschen,
als zu erfahren, dass Karl Marx nie gelebt hat und eine Erfindung von
Thomas Gottschalk ist.``

Der letzte Satz unterstreicht einerseits die absolute Sicherheit von
Sarma, andererseits erfüllt er die Funktion, (\textbf{53}) entsprechende
Thesen ins Lächerliche zu ziehen.

Je sicherer wir uns in unserem Urteil sind, umso schwerer fällt es uns
natürlich, neue Daten fair zu beurteilen. Und genau dies ist das Problem
jener „Skeptiker``. Hinzu kommt ihre schon angesprochene weitgehende
Unkenntnis relevanter Literatur, weshalb sie bei entsprechender „Lage
der Dinge`` sicher unter den Letzten wären, die einen solchen
„Paradigmenwechsel`` erkennen und vollziehen würden, mit Sicherheit erst
deutlich nach der allgemeinen scientific community selbst.

Dies ist aber eine fragwürdige Situation für eine Gesellschaft zur
„wissenschaftlichen Untersuchung von Parawissenschaften``, von der man
eigentlich erwarten sollte, dass ihr Herz ganz dicht am jeweils
aktuellen Forschungs- und Erkenntnishorizont schlägt und sie auch in der
Vermittlung dessen sowohl gegenüber der scientific community als auch
gegenüber der Öffentlichkeit eine Vorreiterrolle einnimmt.

Dass dem aufgrund mangelnder Kenntnisse nicht so ist, hat mir gegenüber
eines der GWUP-Vorstandsmitglieder in einem persönlichen Gespräch auch
ganz offen zugegeben -- und mit dem Argument verteidigt, dass es ja gar
nicht die Aufgabe der GWUP sei, über den aktuellen Forschungsstand zu
informieren, sondern nur über die Bedingungen, unter denen man einen
solchen „Paradigmenwechsel`` ggf. akzeptieren könne.

Inwieweit Personen zu solchen Meta-Urteilen besonders qualifiziert sind,
die kaum Verbindung zum jeweiligen Forschungsprozess und dessen
spezifischen Problemen haben, sei dahingestellt.

Die Frage, inwiefern typische „Skeptiker-Organisationen`` in der Lage
wären, ihrem „unbelief``-System widersprechende Erkenntnisse zu
rezipieren, lässt -- abgesehen vom mangelnden oder bestenfalls sehr
selektiven Fluss relevanter Informationen in jenen Organisationen sowie
der weitgehend fehlenden kontroversen Diskussionskultur auf
wissenschaftlichem Niveau -- auch noch in anderer Hinsicht Zweifel
aufkeimen: Denn für eine ganze Reihe von jenen „Skeptikern`` heiligt
(\textbf{54}) bis zu einem gewissen Grad der Zweck die Mittel im Sinne
ihres „Kampfes gegen das Paranormale``.

Mir haben wiederholt verschiedene Mitglieder der GWUP versichert, dass
sie auch unsachliche Argumente (Anspielen auf Emotionen, Zynismus u.a.)
für legitim halten, um gegen das „Paranormale`` anzutreten. Dies kann
bis zum bewussten Verschweigen eventuell „störender`` Informationen
gehen.

Anlässlich einer von der GWUP geplanten Tagung, zu der auf Anregung von
Rudolf Henke und mir auch „Pro``-Vertreter (so ein in der GWUP üblicher
terminus technicus, der übrigens schon impliziert, dass die GWUP immer
„contra`` ist) als Referenten eingeladen werden sollten, um einen
sachlichen und konstruktiven Dialog zu führen, meinte mir gegenüber
beispielsweise der Geschäftsführer der GWUP, Amardeo Sarma, man solle
einen bestimmten Referenten lieber nicht einladen, da die von ihm
präsentierte Studie (die Sarma zu diesem Zeitpunkt noch gar nicht
bekannt war!) möglicherweise so gut und so fehlerfrei erscheinen könnte,
dass den „Skeptikern`` der GWUP keine Argumente mehr dagegen einfallen
könnten.

Genauso forderte Sarma, dass Pro-Contra-Dialoge im Skeptiker (die von
mir eingeführt worden waren und von ihm und anderen GWUP-Mitgliedern mit
großem Misstrauen gesehen wurden, da sie „das skeptische Profil
gefährden`` würden) von vornherein so angelegt sein müssten, dass die
„skeptische Seite`` am längeren Hebel sitze, das Schlusswort habe und
als Gewinner dastehe.

So teilte mir Sarma in einer E-Mail mit: „Kontroverse Diskussionen sind
dann und nur dann zulässig, wenn es im Interesse des skeptischen Lesers
ist oder der Überzeugung von noch-nicht-skeptischen Lesern dient. In
jedem Fall ist sicherzustellen, dass \ldots{} ein Fazit immer aus
skeptischer Sicht gezogen werden muss. Es soll verhindert werden, auch
in jedem Einzelfall, dass Zweifel über die Position des Skeptikers
auftritt``.

In welchem Sinne „skeptisch`` hier gemeint ist, braucht nicht weiter
betont zu werden und geht auch durch den Kontext der genannten Intention
„Überzeugen`` (natürlich bezüglich inhaltlicher Positionen) klar hervor.

Für Sarma ist die Zielgruppe für die Zeitschrift Skeptiker
ausschließlich „die skeptisch eingestellte Person im Sinne der GWUP bzw.
Personen, bei denen wir glauben, dass wir sie entsprechend überzeugen
können``. Definitiv nicht zur Zielgruppe gehörten laut Sarma Personen,
„bei denen eher nicht davon auszugehen ist, dass sie ins `skeptische
Lager' wechseln werden``. Solchen Personen seien „keine Zugeständnisse``
zu machen, „das heißt konkret, dass wir Aussagen von z.B.
Parapsychologen nicht unwidersprochen lassen`` dürfen. Die Leser dürften
nämlich nicht „über die Zielrichtung der Zeitschrift verwirrt werden``,
stets und in jedem Einzelfall sei zu beachten, „dass der Leser nicht in
Unklarheit darüber gelassen wird, was die Position im Sinne der GWUP
ist`` usw.

Man dürfe zudem nicht dem „Mythos vom mündigen Leser`` aufsitzen, so ein
anderes Vorstandsmitglied. Die Redaktion habe also stets dafür zu
sorgen, dass nur die „richtigen`` Meinungen und Informationen „im Sinne
der GWUP`` in der Zeitschrift erscheinen bzw. falls überhaupt
abweichende Meinungen auftauchten, dann nur und in vorgegebener Weise
kommentiert.

Man könnte sich fragen, ob sich hinter einer solchen Auffassung nicht
eine gehörige Portion von Misstrauen gegenüber dem „selbstreinigenden``
Prozess der Wissenschaft als solchem sowie eine bemerkenswerte
Geringschätzung der eigenen Leserschaft hinsichtlich deren kritischer
Denkfähigkeit verbirgt.

Eine solche Haltung könnte man (\textbf{55}) als Cui-bono-Denken
bezeichnen, welches einen weiteren typischen Bestandteil des
Skeptiker-Syndroms darstellt: Entscheidungskriterium für das eigene
Handeln ist letztlich immer die Frage „Wem nützt es?``. Nicht akzeptiert
wird die aus einer wissenschaftlichen Perspektive angebrachte Norm, dass
z.B. auf Tagungen oder in Publikationen schlicht derjenige zu Wort
kommt, der etwas Relevantes zu sagen hat und seine Position in einem
kritischen Diskurs mit sachlich-fundierten Argumenten verteidigen kann
-- und nicht der, der die „richtige`` Meinung hat, zu den „richtigen``
Ergebnissen kommt oder der „richtigen`` Gruppe angehört.

In öffentlichen Verlautbarungen präsentieren sich jene „Skeptiker``
freilich ganz anders. So führt z.B. Sarma in einem Artikel im Skeptiker
4/96 aus: „Die Zuhörer sind in der Lage, sich selbst eine Meinung zu
bilden; deshalb sollte man die Fakten für sich sprechen lassen \ldots{}
Das Ziel der GWUP ist es nicht, recht zu haben oder zu bekommen, sondern
gemeinsam möglichst nahe an die Wahrheit heranzukommen.`` Die Diskrepanz
zu den oben angeführten intern vertretenen Positionen von Sarma ist
offensichtlich. Cui bono-Denken ist zwar ein zentrales Merkmal des
Syndroms, jedoch eines, das aus guten Gründen nur in der internen
Kommunikation mit vermuteten „Gleichgesinnten`` offen zutage tritt.

Sarma hatte mit solchen und ähnlichen mir gegenüber intern erhobenen
Forderungen aber zumindest aus einer funktionalen Perspektive sicher
nicht unrecht, denn würde anderes gelten, stünde nach meiner
Einschätzung die GWUP in der Tat vor einer existenzgefährdenden
Zerreißprobe, weil der Großteil der Mitglieder dann „Profil`` und
„Identität`` der GWUP grundlegend gefährdet sähen. In letzter Konsequenz
würden die ``Skeptiker''-Organisationen zerfallen, denn sie leben von
dieser ``kommunikativen Schließung'', ohne die ihre Ideologeme genauso
zerbröseln würden wie ihre soziale Basis.

Und für Personen, die die Prioritäten anders setzen und im Zweifelsfall
wissenschaftliche Seriosität den Selbstbestätigungs-, Selbsterhaltungs-
und ideologischen Positionierungsbedürfnissen der Gruppen vorziehen,
gilt (so Sarma im September 2002 in Prag auf einem internationalen
Koordinationstreffen von ``Skeptiker''-Funktionären): ``It is fine to
have such persons outside a skeptical organization and they sometimes
correctly point out flawed reasoning amongst skeptics. It is within a
skeptical group that they pose a real danger, because this position
undermines the identification of skeptics with their skeptical group.''

Dem kann ich nur zustimmen.

\begin{center}\rule{0.5\linewidth}{\linethickness}\end{center}

Erstveröffentlichung: 1998\\
Diese Veröffentlichung: 2019\\
Quelle:
\href{https://web.archive.org/web/20170905223349/http://www.skeptizismus.de/index.html}{Skeptizismus.de}
(Archiv)\\
© Dr. Edgar Wunder

Schlagwörter: Skeptiker, GWUP, Psiram, Wikipedia

\hypertarget{swiss-policy-research}{%
\subsubsection{Swiss Policy Research}\label{swiss-policy-research}}

\begin{itemize}
\tightlist
\item
  \href{https://swprs.org/kontakt/}{Kontakt}
\item
  \href{https://swprs.org/uebersicht/}{Übersicht}
\item
  \href{https://swprs.org/donationen/}{Donationen}
\item
  \href{https://swprs.org/disclaimer/}{Disclaimer}
\end{itemize}

\hypertarget{english}{%
\subsubsection{English}\label{english}}

\begin{itemize}
\tightlist
\item
  \href{https://swprs.org/contact/}{About Us / Contact}
\item
  \href{https://swprs.org/media-navigator/}{The Media Navigator}
\item
  \href{https://swprs.org/the-american-empire-and-its-media/}{The CFR
  and the Media}
\item
  \href{https://swprs.org/donations/}{Donations}
\end{itemize}

\hypertarget{follow-by-email}{%
\subsubsection{Follow by email}\label{follow-by-email}}

Follow

\href{https://wordpress.com/?ref=footer_custom_com}{WordPress.com}.

\protect\hyperlink{}{Up ↑}

Post to

\protect\hyperlink{}{Cancel}

\includegraphics{https://pixel.wp.com/b.gif?v=noscript}
