\protect\hyperlink{content}{Skip to content}

\href{https://swprs.org/}{}

\protect\hyperlink{search-container}{Search}

Search for:

\href{https://swprs.org/}{\includegraphics{https://swprs.files.wordpress.com/2020/05/swiss-policy-research-logo-300.png}}

\href{https://swprs.org/}{Swiss Policy Research}

Geopolitics and Media

Menu

\begin{itemize}
\tightlist
\item
  \href{https://swprs.org}{Start}
\item
  \href{https://swprs.org/srf-propaganda-analyse/}{Studien}

  \begin{itemize}
  \tightlist
  \item
    \href{https://swprs.org/srf-propaganda-analyse/}{SRF / ZDF}
  \item
    \href{https://swprs.org/die-nzz-studie/}{NZZ-Studie}
  \item
    \href{https://swprs.org/der-propaganda-multiplikator/}{Agenturen}
  \item
    \href{https://swprs.org/die-propaganda-matrix/}{Medienmatrix}
  \end{itemize}
\item
  \href{https://swprs.org/medien-navigator/}{Analysen}

  \begin{itemize}
  \tightlist
  \item
    \href{https://swprs.org/medien-navigator/}{Navigator}
  \item
    \href{https://swprs.org/der-propaganda-schluessel/}{Techniken}
  \item
    \href{https://swprs.org/propaganda-in-der-wikipedia/}{Wikipedia}
  \item
    \href{https://swprs.org/logik-imperialer-kriege/}{Kriege}
  \end{itemize}
\item
  \href{https://swprs.org/netzwerk-medien-schweiz/}{Netzwerke}

  \begin{itemize}
  \tightlist
  \item
    \href{https://swprs.org/netzwerk-medien-schweiz/}{Schweiz}
  \item
    \href{https://swprs.org/netzwerk-medien-deutschland/}{Deutschland}
  \item
    \href{https://swprs.org/medien-in-oesterreich/}{Österreich}
  \item
    \href{https://swprs.org/das-american-empire-und-seine-medien/}{USA}
  \end{itemize}
\item
  \href{https://swprs.org/bericht-eines-journalisten/}{Fokus I}

  \begin{itemize}
  \tightlist
  \item
    \href{https://swprs.org/bericht-eines-journalisten/}{Journalistenbericht}
  \item
    \href{https://swprs.org/russische-propaganda/}{Russische Propaganda}
  \item
    \href{https://swprs.org/die-israel-lobby-fakten-und-mythen/}{Die
    »Israel-Lobby«}
  \item
    \href{https://swprs.org/geopolitik-und-paedokriminalitaet/}{Pädokriminalität}
  \end{itemize}
\item
  \href{https://swprs.org/migration-und-medien/}{Fokus II}

  \begin{itemize}
  \tightlist
  \item
    \href{https://swprs.org/covid-19-hinweis-ii/}{Coronavirus}
  \item
    \href{https://swprs.org/die-integrity-initiative/}{Integrity
    Initiative}
  \item
    \href{https://swprs.org/migration-und-medien/}{Migration \& Medien}
  \item
    \href{https://swprs.org/der-fall-magnitsky/}{Magnitsky Act}
  \end{itemize}
\item
  \href{https://swprs.org/kontakt/}{Projekt}

  \begin{itemize}
  \tightlist
  \item
    \href{https://swprs.org/kontakt/}{Kontakt}
  \item
    \href{https://swprs.org/uebersicht/}{Seitenübersicht}
  \item
    \href{https://swprs.org/medienspiegel/}{Medienspiegel}
  \item
    \href{https://swprs.org/donationen/}{Donationen}
  \end{itemize}
\item
  \href{https://swprs.org/contact/}{English}
\end{itemize}

\protect\hyperlink{}{Open Search}

\hypertarget{die-schleusenwuxe4rter}{%
\section{Die Schleusenwärter}\label{die-schleusenwuxe4rter}}

Publiziert: April 2019

\textbf{Beanstandung eines SRF-Zuschauers}: Warum berichtete SRF (sowie
fast alle anderen Medien) nicht über den mehrtägigen Generalstreik in
Indien von immerhin 200 Millionen Menschen?

\textbf{Antwort des stv. Chefredakteurs:} »Tatsächlich trifft es zu,
dass wir diesen Generalstreik nicht thematisiert haben. Auch die
allermeisten anderen relevanten Medien taten das nicht. Weder in der
Schweiz noch anderswo in Europa. Nach Abklärungen unserer
Nachrichten­redaktion war der Streik in den grossen Nachrichtenagenturen
-- Reuters, AFP, DPA oder AP -- kein oder so gut wie kein Thema. Unsere
Nachrichten­redaktion stützt sich massgeblich auf diese Agenturen, da
wir als Schweizer Medium mit sehr beschränkten Mitteln in vielen Fällen
ausserstande sind, die gigantische Fülle an internationaler Information
selber zu sichten und zu verarbeiten. Wir brauchen die Agenturen und
andere internationale Medien als Schleusenwärter. Den indischen Streik
aufgegriffen haben unter den ausländischen Medien praktisch
ausschliesslich ein paar wenige mit prononciert linkem Kurs («Neues
Deutschland» zum Beispiel) -- was selbstverständlich nicht heisst, dass
es falsch war, darüber zu berichten. Angesichts eines riesigen Landes
wie Indien mit einer Milliarden­bevölkerung können wir stets nur einen
winzigen Teil der aktuellen Ereignisse aufgreifen in unserer
Berichterstattung. ()«

\textbf{Ergänzung des stv. Redaktionsleiters:} »Aus Fernsehsicht ist
hinzuzufügen, dass der zweitägige Streik in Indien auf keiner Agenda
(News-Agentur oder Video-Agentur) als Top-Event angekündigt war. SRF hat
über den internationalen Video-Austausch, auf den SRF als kleine
TV-Anstalt in der aktuellen Ausland­bericht­erstattung angewiesen ist,
keine bewegten Bilder erhalten. ()«

\textbf{Anmerkung SPR:} Ob in Syrien, in Venezuela oder in Indien: »Was
die Agentur nicht meldet, findet nicht statt.« (Wilke 2000) Und die drei
Weltagenturen befinden sich in den westlichen Metropolen London, New
York und Paris (plus Berlin). Würde sich das traditionell blockfreie
Indien zu stark an Russland und China annähern, so stünde so ein
Generalstreik womöglich weiter oben auf der westlichen Agentur- und
Medienagenda -- ohne dass der einzelne Journalist wissen muss, warum.

Wohlgemerkt: Das SRF verfügt seit mehreren Jahren über einen festen
Korrespondenten in Indien, und dieser hat den Generalstreik von 200
Millionen Menschen sicherlich mitbekommen. Doch mangels Agentur­meldung
wurde das Thema von der Redaktion gar nicht erst eingeplant.

\textbf{Quelle:}
\href{https://www.srgd.ch/de/aktuelles/news/2019/02/19/nichtberichterstattung-uber-generalstreik-indien-beanstandet/}{Nichtberichterstattung
über Generalstreik in Indien beanstandet} (SRG, 02/2019)

\hypertarget{siehe-auch}{%
\paragraph{Siehe auch}\label{siehe-auch}}

\begin{itemize}
\tightlist
\item
  \href{https://swprs.org/der-propaganda-multiplikator/}{Der
  Propaganda-Multiplikator}
\item
  \href{https://swprs.org/srf-propaganda-analyse/}{SRF: Die
  Propaganda-Analyse}
\item
  \href{https://swprs.org/srf-ombudsstelle-im-faktencheck/}{Medienaufsicht
  im Faktencheck}
\end{itemize}

\begin{center}\rule{0.5\linewidth}{\linethickness}\end{center}

Publiziert: April 2019

\hypertarget{swiss-policy-research}{%
\subsubsection{Swiss Policy Research}\label{swiss-policy-research}}

\begin{itemize}
\tightlist
\item
  \href{https://swprs.org/kontakt/}{Kontakt}
\item
  \href{https://swprs.org/uebersicht/}{Übersicht}
\item
  \href{https://swprs.org/donationen/}{Donationen}
\item
  \href{https://swprs.org/disclaimer/}{Disclaimer}
\end{itemize}

\hypertarget{english}{%
\subsubsection{English}\label{english}}

\begin{itemize}
\tightlist
\item
  \href{https://swprs.org/contact/}{About Us / Contact}
\item
  \href{https://swprs.org/media-navigator/}{The Media Navigator}
\item
  \href{https://swprs.org/the-american-empire-and-its-media/}{The CFR
  and the Media}
\item
  \href{https://swprs.org/donations/}{Donations}
\end{itemize}

\hypertarget{follow-by-email}{%
\subsubsection{Follow by email}\label{follow-by-email}}

Follow

\href{https://wordpress.com/?ref=footer_custom_com}{WordPress.com}.

\protect\hyperlink{}{Up ↑}

\includegraphics{https://pixel.wp.com/b.gif?v=noscript}
