\protect\hyperlink{content}{Skip to content}

\href{https://swprs.org/}{}

\protect\hyperlink{search-container}{Search}

Search for:

\href{https://swprs.org/}{\includegraphics{https://swprs.files.wordpress.com/2020/05/swiss-policy-research-logo-300.png}}

\href{https://swprs.org/}{Swiss Policy Research}

Geopolitics and Media

Menu

\begin{itemize}
\tightlist
\item
  \href{https://swprs.org}{Start}
\item
  \href{https://swprs.org/srf-propaganda-analyse/}{Studien}

  \begin{itemize}
  \tightlist
  \item
    \href{https://swprs.org/srf-propaganda-analyse/}{SRF / ZDF}
  \item
    \href{https://swprs.org/die-nzz-studie/}{NZZ-Studie}
  \item
    \href{https://swprs.org/der-propaganda-multiplikator/}{Agenturen}
  \item
    \href{https://swprs.org/die-propaganda-matrix/}{Medienmatrix}
  \end{itemize}
\item
  \href{https://swprs.org/medien-navigator/}{Analysen}

  \begin{itemize}
  \tightlist
  \item
    \href{https://swprs.org/medien-navigator/}{Navigator}
  \item
    \href{https://swprs.org/der-propaganda-schluessel/}{Techniken}
  \item
    \href{https://swprs.org/propaganda-in-der-wikipedia/}{Wikipedia}
  \item
    \href{https://swprs.org/logik-imperialer-kriege/}{Kriege}
  \end{itemize}
\item
  \href{https://swprs.org/netzwerk-medien-schweiz/}{Netzwerke}

  \begin{itemize}
  \tightlist
  \item
    \href{https://swprs.org/netzwerk-medien-schweiz/}{Schweiz}
  \item
    \href{https://swprs.org/netzwerk-medien-deutschland/}{Deutschland}
  \item
    \href{https://swprs.org/medien-in-oesterreich/}{Österreich}
  \item
    \href{https://swprs.org/das-american-empire-und-seine-medien/}{USA}
  \end{itemize}
\item
  \href{https://swprs.org/bericht-eines-journalisten/}{Fokus I}

  \begin{itemize}
  \tightlist
  \item
    \href{https://swprs.org/bericht-eines-journalisten/}{Journalistenbericht}
  \item
    \href{https://swprs.org/russische-propaganda/}{Russische Propaganda}
  \item
    \href{https://swprs.org/die-israel-lobby-fakten-und-mythen/}{Die
    »Israel-Lobby«}
  \item
    \href{https://swprs.org/geopolitik-und-paedokriminalitaet/}{Pädokriminalität}
  \end{itemize}
\item
  \href{https://swprs.org/migration-und-medien/}{Fokus II}

  \begin{itemize}
  \tightlist
  \item
    \href{https://swprs.org/covid-19-hinweis-ii/}{Coronavirus}
  \item
    \href{https://swprs.org/die-integrity-initiative/}{Integrity
    Initiative}
  \item
    \href{https://swprs.org/migration-und-medien/}{Migration \& Medien}
  \item
    \href{https://swprs.org/der-fall-magnitsky/}{Magnitsky Act}
  \end{itemize}
\item
  \href{https://swprs.org/kontakt/}{Projekt}

  \begin{itemize}
  \tightlist
  \item
    \href{https://swprs.org/kontakt/}{Kontakt}
  \item
    \href{https://swprs.org/uebersicht/}{Seitenübersicht}
  \item
    \href{https://swprs.org/medienspiegel/}{Medienspiegel}
  \item
    \href{https://swprs.org/donationen/}{Donationen}
  \end{itemize}
\item
  \href{https://swprs.org/contact/}{English}
\end{itemize}

\protect\hyperlink{}{Open Search}

\hypertarget{covid-19-cinjenice}{%
\section{Covid-19 cinjenice}\label{covid-19-cinjenice}}

\textbf{Aktualizovano}: May 18, 2020; \textbf{Share on}:
\href{https://twitter.com/intent/tweet?url=https://swprs.org/a-swiss-doctor-on-covid-19/}{Twitter}
/
\href{https://www.facebook.com/share.php?u=https://swprs.org/a-swiss-doctor-on-covid-19/}{Facebook}\\
\textbf{Jezici}: \href{https://swprs.org/fakta-o-covid-19/}{CZ},
\href{https://swprs.org/covid-19-hinweis-ii/}{DE},
\href{https://swprs.org/a-swiss-doctor-on-covid-19/}{EN},
\href{https://swprs.org/hechos-sobre-covid-19/}{ES},
\href{https://swprs.org/faktoja-covid-19sta/}{FI},
\href{https://swprs.org/coronavirus-un-medecin-suisse-parle/}{FR},
\href{https://swprs.org/facts-about-covid19-greek/}{GR},
\href{https://swprs.org/covid-19-cinjenice/}{HBS},
\href{https://yanivhamo.com/facts-about-covid-19-hebrew/}{HE},
\href{https://swprs.org/egy-svajci-orvos-a-covid-19-rol/}{HU},
\href{https://swprs.org/un-medico-svizzero-su-covid-19/}{IT},
\href{https://swprs.org/covid19-facts-japanese/}{JP},
\href{https://swprs.org/covid19-korean/}{KO},
\href{https://www.globalinfo.nl/Achtergrond/een-kritische-kijk-op-het-coronabeleid-transparantie-in-tijden-van-crisis}{NL},
\href{https://midtifleisen.wordpress.com/2020/03/14/en-sveitsisk-lege-om-covid-19/}{NO},
\href{https://swprs.org/szwajcarski-lekarz-o-covid-19/}{PL},
\href{https://swprs.org/fatos-sobre-covid-19/}{PT},
\href{https://swprs.org/informatii-despre-covid-19/}{RO},
\href{https://swprs.org/\%d0\%bd\%d0\%b0-\%d0\%ba\%d0\%be\%d0\%b2\%d0\%b8\%d0\%b4-19/}{RU},
\href{https://swprs.org/fakta-om-covid-19/}{SE},
\href{http://www.ninamvseeno.org/pregled-clanka.aspx?naslov=pomembne-informacije-o-novem-koronavirusu-covid-19\&id=148}{SI},
\href{https://alatyr.sk/covid-19_swiss_propaganda_research.htm}{SK},
\href{https://swprs.org/isvicreli-bir-doktordan-kovid-19-uezerine/}{TR}

Informacije i cinjenice o Covid-19 od strane strucnjaka analizirane i
uredjene, koje nasim citaocima omogucuju realisticnu procenu rizika.

\textbf{„Jedina mogucnost, boriti se protiv kuge, je iskrenost`` (Albert
Camus, 1947)}

\hypertarget{pregled}{%
\paragraph{Pregled}\label{pregled}}

\begin{enumerate}
\def\labelenumi{\arabic{enumi}.}
\tightlist
\item
  Prema podacima najbolje ispitanih zemalja kao sto su Juzna Koreja,
  Island, Nemačka i Danska smrtnost od bolesti Covid-19 je, ukupno
  uzevši, u
  \href{https://swprs.org/studies-on-covid-19-lethality/}{promilima},
  sto znaci i do ~dvadeset puta manje nego sto je to provobitno
  pretpostavila Svetska Zdravstvena Organizacija.
\item
  Jedna studija u naučnom časopisu \emph{Nature Medicine} je dosla do
  \href{https://www.nature.com/articles/s41591-020-0822-7}{sličnih
  rezultata} čak i za kineski grad Wuhan. U početku znatno više
  vrednosti za Wuhan su nastale jer jako mnogo osoba sa blagim
  simptomima, odnosno bez simptoma a sa virusom, nije~ bilo
  registrovano.
\item
  50\% do 80\% pozitivno testiranih osoba ostaje
  \href{https://www.bmj.com/content/369/bmj.m1375}{bez simptoma}. Čak i
  među osobama između 70 i 79 godina starosti oko 60\% ostaje
  \href{https://www.niid.go.jp/niid/en/2019-ncov-e/9407-covid-dp-fe-01.html}{bez
  simptoma}, a kod velikog broja ostalih se pojavljuju samo
  \href{https://swprs.org/studies-on-covid-19-lethality/\#hospitalizations}{blagi
  simptomi}.
\item
  Prosecna starost umrlih kod najvise zemalja (ukljucujuci i
  \href{https://www.epicentro.iss.it/coronavirus/sars-cov-2-decessi-italia}{Italiju})
  iznosi preko 80 godina i samo ca.
  \href{https://www.bloomberg.com/news/articles/2020-03-18/99-of-those-who-died-from-virus-had-other-illness-italy-says}{1\%}
  umrlih nije imalo prethodno nikakvih ozbiljnih oboljenja. Profil
  smrtnosti odgovara tako u sustini
  \href{https://www.vienna.at/analyse-zeigt-covid-19-opferkurve-entspricht-normaler-mortalitaet/6581246}{normanoj
  smrtnosti}. Do 60\% svih Covid-19 smrtnih slucajeva se desilo u
  posebno ugrozenim
  \href{https://ltccovid.org/2020/04/12/mortality-associated-with-covid-19-outbreaks-in-care-homes-early-international-evidence/}{starackim
  domovima}.
\item
  Mnogi izveštaji medija, po kojima i mlade i zdrave osobe umiru od
  Covid-19 ispostavili su se kao netačni. Ili te osobe
  \href{https://archive.is/20200329015127/https:/www.theguardian.com/world/2020/mar/27/chloe-middleton-death-21-year-old-not-recorded-nhs-covid-19-related}{nisu}
  umrle zbog Covid-19, ili su ranije bile teško
  \href{https://www.msn.com/de-ch/news/other/spanischer-nachwuchs-trainer-stirbt-an-corona/ar-BB11gT64}{bolesne}
  (npr. od leukemije, koja nije bila~ prepoznata)
\item
  Normalna \href{https://www.euromomo.eu/index.html}{ukupna dnevna
  smrtnost} iznosi u USA ca. 8000, u Nemackoj oko 2600, u Italiji oko
  1800 i u Svajcarskoj oko 200 osoba dnevno.
  \href{https://www.statnews.com/2018/09/26/cdc-us-flu-deaths-winter/}{Smrtnost
  kao posledica gripa} iznosi u USA i do 80.000, u Nemackoj i Italiji do
  25.000 i u Svajcarskoj do 2500 osoba u vreme jednog zimskog perioda.
\item
  Povećana smrtnost, kao na primer u severnoj Italiji, može da
  proizilazi iz dodatnih faktora rizika kao sto su jako velika
  \href{https://www.heise.de/tp/features/Feinstaubpartikel-als-Viren-Vehikel-4687454.html}{zagađenost
  vazduha} i zaraženost
  \href{https://www.apotheke-adhoc.de/nachrichten/detail/coronavirus/erhoehen-legionellen-die-todesrate-einer-corona-infektion/}{legionellama},
  kao i zbog~ kolapsa
  \href{https://www.sueddeutsche.de/politik/coronavirus-pflegekraefte-ausland-1.4866124}{sistema
  nege starijih i bolesnih} ljudi zbog~ masovne panike i zabrane
  kretanja.
\item
  U zemljama kao sto su Italija i Spanija i delimično Velika Britanija i
  SAD su već i ranije epidemije gripa
  \href{https://off-guardian.org/2020/04/02/coronavirus-fact-check-1-flu-doesnt-overwhelm-our-hospitals/}{dovodile}
  su do preopterećenosti zdravstvenog sistema. Uz to, trenutno skoro
  \href{https://www.reuters.com/article/us-health-coronavirus-spain-morgue-idUSKBN21B1PP}{15\%}
  lekara i pomoćnog osoblja -- i bez simptoma -- mora u karantin.
\item
  Jedno važno pitanje je sledeće : da li osobe umiru samo \emph{sa}
  virusom, ili \emph{zbog} virusa. Obdukcije
  \href{https://www.abendblatt.de/hamburg/article228828787/rechtsmedizin-pueschel-hamburg-corona-virus-infektion-covid-19-coronavirus-krise-patienten-krankenhaeuser-kliniken-infektionsrate-krankheit-pandemie-test-lungenkrankheit-sars-cov-epidemie-sars-cov-2.html}{pokazuju}
  da su u mnogim slučajevima prethodno postojeća oboljenja bila
  odlučujuća, ali oficijelni podaci to
  \href{https://swprs.org/rki-relativiert-corona-todesfaelle/}{uglavnom
  ne reflektuju.}
\item
  Da bi se ocenila opasnost bolesti nije stoga ključan često navođeni
  broj pozitivno testiranih i umrlih, nego broj onih koji su zaista
  umrli od upale pluća, i to iznenada obolelih ili umrlih od upale
  pluća.
\item
  Često pokazivane eksponencijalne krivulje „Korona-pozitivnih``
  \href{https://multipolar-magazin.de/artikel/coronavirus-regierung-ignoriert-daten}{vode~
  pogrešnim zaključcima}, jer i broj testova takođe eksponencijalno
  raste. Odnos pozitivnih nalaza prema ukupnom broju testova ostaje u
  najvećem broju zemalja konstantan, i iznosi
  \href{https://swprs.org/rate-of-positive-covid19-tests/}{5\% do 25\%},
  ili raste sporo.
\item
  Zemlje bez zabrane izlaska i zabrane kontakta, kao npr. Japan, Juzna
  Koreja i
  \href{https://www.telegraph.co.uk/news/2020/04/03/coronavirus-swedish-experiment-could-prove-britain-wrong/}{Švedska}
  nisu do sada
  \href{https://www.japantimes.co.jp/news/2020/03/20/national/coronavirus-explosion-expected-japan/\#.Xo8poW5uJv0}{doživele}
  lošiji razvoj situacije nego ostala zemlje. To govori protiv nekog
  posebnog efekta ovako ekstremnih mera.
\item
  Po lekarima za plućne bolesti je tretman invazivnog vestackog disanja
  Covid-19 pacijenata cesto
  \href{https://www.faz.net/aktuell/gesellschaft/gesundheit/coronavirus/beatmung-beim-coronavirus-lungenfacharzt-im-gespraech-16714565.html}{kontraproduktiv}an
  i steti plucima dodatno. Invazivno veštacko disanje kod Covid-19 se
  upraznjava posebno \href{https://archive.is/KX5IQ}{radi straha} od
  širenja virusa putem aerosola.
\item
  Nasuprot prethodnim pretpostavkama, Svetska Zdravstvena Organizacija
  je utvrdila da veliko širenje virusa putem ``aerosola'', malih
  kapljica u vazduhu,
  \href{https://www.who.int/news-room/commentaries/detail/modes-of-transmission-of-virus-causing-covid-19-implications-for-ipc-precaution-recommendations}{nije
  dokazano}. U
  \href{https://www.zeit.de/zustimmung?url=https\%3A\%2F\%2Fwww.zeit.de\%2Fwissen\%2Fgesundheit\%2F2020-04\%2Fhendrik-streeck-covid-19-heinsberg-symptome-infektionsschutz-massnahmen-studie\%2Fkomplettansicht}{početnoj
  studiji} jednog nemačkog virologa nisu utvđene infekcije putem
  aerosola.
\item
  Mnoge klinike u Nemačkoj i Švajcarskoj su trenutno
  \href{https://www.spiegel.de/wirtschaft/unternehmen/trotz-corona-pandemie-warum-kliniken-jetzt-kurzarbeit-anmelden-a-3dc61bc9-fb12-4298-8022-bb4c2be39d7d}{bez
  dovoljno posla}, te su najavile
  \href{https://www.20min.ch/schweiz/news/story/Spitaeler-28949526}{skraćeno
  radno vreme}, ili cak morale da objave bankrot. Operacije i terapije
  su otkazane. Čak i pacijenti kojima je potrebna hitna operacija radi
  straha ostaju \href{https://www.youtube.com/watch?v=KucqjgM0P1E}{kod
  kuće}.
\item
  Više medija je bilo
  \href{https://nypost.com/2020/04/01/cbs-admits-to-using-footage-from-italy-in-report-about-nyc/}{otkriveno}
  u pokusaju da dramatizuju situaciju u klinikama, delimično čak sa
  obmanjujućim slikama i video-snimcima.
  \href{https://www.deutschlandfunk.de/die-nachrichten-kulturnachrichten.2792.de.html}{Mnogi
  mediji} generalno
  \href{https://swprs.org/das-bag-im-corona-fieber/}{ne proveravaju} čak
  ni sumnjive oficijelne podatke i vesti.
\item
  Testovi za viruse koji su u sirom u upotrebi su
  \href{https://www.ncbi.nlm.nih.gov/pubmed/32219885}{podložni
  greskama}: ranija istraživanja su pokazala, da i normalni
  Korona-virusi
  \href{https://www.ncbi.nlm.nih.gov/pmc/articles/PMC2095096/}{mogu da
  daju} pogrešan pozitivni rezultat. Aktuelno korišćeni test
  \href{https://www.youtube.com/watch?v=p_AyuhbnPOI}{nije klinički
  potvrđen.} Prihvaćen je usled manjka vremena i pritiska da se nešto
  učini.
\item
  Brojni međunarodno renomirani
  \href{https://off-guardian.org/2020/03/24/12-experts-questioning-the-coronavirus-panic/}{eksperti}
  iz oblasti virologije, imunologije i epidemiologije smatraju da su
  uvedene mere sumnjive i
  \href{https://off-guardian.org/2020/03/28/10-more-experts-criticising-the-coronavirus-panic/}{kontraproduktivne},
  i predlažu brzu
  \href{https://off-guardian.org/2020/04/08/watch-perspectives-on-the-pandemic-2/}{prirodnu
  imunizaciju} populacije.
\item
  Broj ljudi koji su zbog uvedenih mera pogođeni nezaposlenošću,
  izloženi psihičkim mukama i kućnom nasilju
  \href{https://www.reuters.com/article/us-health-coronavirus-usa-layoffs/us-weekly-jobless-claims-seen-at-record-high-again-idUSKBN21K0FX}{eksplodira}
  u SAD i širom sveta. Mnogi eksperti tvrde da će uvedene mere izazvati
  više žrtava nego virus.
\item
  Bivši agent američke obaveštajne agencije NSA, Edward Snowden,
  \href{https://www.youtube.com/watch?v=-pcQFTzck_c}{upozorava} da će
  Korona-kriza biti iskorišćena u cilju~stalne izgradnje instrumenata
  svetske masovne kontrole. Renomirani virolog Pablo Goldschmidt
  \href{https://www.rubikon.news/artikel/der-corona-totalitarismus}{govori}
  o jednom „globalnom medijskom teroru`` i „totalitarnim merama``.
  Britanski infektolog John Oxford
  \href{https://novuscomms.com/2020/03/31/a-view-from-the-hvivo-open-orphan-orph-laboratory-professor-john-oxford/}{govori}
  o jednoj „medijskoj epidemiji``.
\end{enumerate}

\hypertarget{takodje-pogledati}{%
\subparagraph{\texorpdfstring{\textbf{Takodje
pogledati:}}{Takodje pogledati:}}\label{takodje-pogledati}}

\begin{itemize}
\tightlist
\item
  \href{https://swprs.org/offener-brief-von-professor-sucharit-bhakdi-an-bundeskanzlerin-dr-angela-merkel/}{Otvoreno
  pismo profesora Bhakdi-ja}
\item
  \href{https://www.euromomo.eu/}{Evropski monitoring mortaliteta}
\item
  Analiza: \href{https://swprs.org/corona-medien-propaganda/}{Corona,
  mediji, propaganda}
\end{itemize}

\begin{center}\rule{0.5\linewidth}{\linethickness}\end{center}

\hypertarget{viux161e-aux17euriranja-na-engleskom-ili-njemaux10dkom-jeziku}{%
\paragraph{\texorpdfstring{Više ažuriranja na
\href{https://swprs.org/a-swiss-doctor-on-covid-19/}{engleskom} ili
\href{https://swprs.org/covid-19-hinweis-ii/}{njemačkom}
jeziku.}{Više ažuriranja na engleskom ili njemačkom jeziku.}}\label{viux161e-aux17euriranja-na-engleskom-ili-njemaux10dkom-jeziku}}

\hypertarget{25-april-2020}{%
\paragraph{25. April 2020.}\label{25-april-2020}}

\hypertarget{medicinske-vesti}{%
\subparagraph{\texorpdfstring{\textbf{Medicinske
vesti}}{Medicinske vesti}}\label{medicinske-vesti}}

\begin{itemize}
\tightlist
\item
  Profesor Detlef Krüger, direktni prethodnik poznatog nemackog
  virologa, Cristiana Drostena, na Charite-klinici u Berlinu, objasnjava
  u
  \href{https://de.sputniknews.com/interviews/20200425326953541-corona-gefahr-virologe/}{jednom
  novom intervjuu}, da je Covid-19, „u mnogo cemu uporediv sa gripom`` i
  „ne opasniji od nekih varijanti virusa gripa``. „Od strane politicara
  otkrivena zastita nosa i usta`` je po profesoru Krüger-u samo
  „akcionizam``, odnosno potencijalni „rasprsivac mikroba``. U isto
  vreme on upozorava na masovnu „kolateralnu stetu`` radi uvedenih mera.
\item
  Svajcarski patolozi dolaze do zakljucka, da puno umrlih pozitivno
  testiranih
  \href{https://www.welt.de/wissenschaft/article207417811/Corona-Tote-In-den-wenigsten-Faellen-eine-Lungenentzuendung.html}{nije
  obolelo od upale pluca}, vec se radilo o smetnjama u razmeni kiseonika
  u plucima. To bi objasnilo, zbog cega vestacko disanje kod Covid-19
  pacijenata cesto ostaje bez dejstva, te zasto pacijenti sa postojecim
  kardiovaskularnim problemima pripadaju rizicnoj grupi. Sve obducirane
  osobe sui male visoki krvni pritisak, veliki deo je bio preterano
  gojazan i dve trecine je imalo ostecene srcane krvne sudove.
\item
  \href{https://www.epicentro.iss.it/coronavirus/bollettino/Bollettino-sorveglianza-integrata-COVID-19_16-aprile-2020.pdf\#page=13}{Najnoviji
  podaci iz Italije} pokazuju (s.12/13), da je od oko 17.000 pozitivno
  testiranih lekara i pomocnog osoblja, 60 umrlo. Odatle proizilazi
  letalitet Covid-19 kod onih osoba ispod 50 godina starosti ispod
  0,1\%, kod onih od 50-60 godina 0,27\%, kod 60-70 godina 1,4\% i kod
  70-80 godina 12,6\%. Cak i ove vrednosti mogu biti previsoke, posto se
  i ovde radi o smrtnim slucajevima sa ali ne i sigurno usled
  Corona-virusa, te da skoro 80\% ljudi ostaje bez simptoma i neki od
  njih i nisu bili testirani. Sve u svemu, ove vrednosti se slazu sa
  onima iz na primer Juzne Koreje i iz njih proizilazi smrtnost obicne
  populacije u opsegu gripa.
\item
  Sef italijanske Civilne zastite je
  \href{https://www.theguardian.com/world/2020/apr/16/italian-police-broaden-care-home-coronavirus-milan}{sredinom
  aprila izjavio}, da je 1800 osoba u Lombardiji umrlo u starackim
  domovima i da u mnogim slucajevima jos nije utvrdjen uzrok smrti. Vec
  pre toga je bilo poznato, da je sistem zbrinjavanja starih ljudi i u
  sledu toga i sistem zbrinjavanja bolesnih
  \href{https://swprs.org/covid19-bericht-aus-italien/}{kolabirao} usled
  straha od virusa i uvedenog Lockdown-a.
\item
  \href{https://covid-19.sciensano.be/sites/default/files/Covid19/Meest\%20recente\%20update.pdf}{Najnoviji
  podaci iz Belgije} pokazuju, da se i tamo preko 50\% svih dodatnih
  smrtnih slucajeva desilo u starackim domovima, koji radi Lockdown-a
  nisu bili bolje zasticeni. Kod 6\% ovih smrtnih slucajeva je bio
  Covid-19 „potvrdjen``, kod 94\% je to bilo pretpostavljeno. Oko 70\%
  pozitivno testiranih (osoblje i stanovnici) nije pokazivalo simptome u
  trenutku testiranja.
\item
  Nemacki ekspert za vakcinaciju, Dr. Siegwart Bigl
  ~\href{https://www.pressreader.com/germany/dresdner-neueste-nachrichten/20200423/281496458428447}{smatra,
  da je zastita protiv Corone „preterana``}. Ne postoji pandemija (sa
  posebno mnogo smrtnih slucajeva), Lockdown je nepotreban i pogresan.
  Jedno poredjenje sa influencom ima smisla.
\item
  Nekadasnji svedski i evropski sef epidemiolog, profesor Johan
  Giesecke, je dao austrijskom listu Addendum
  \href{https://www.addendum.org/coronavirus/interview-johan-giesecke/}{veoma
  otvoreni intervju}. Profesor Giesecke kaze, da je 75 do 90\% epidemije
  „nevidljivo``, jer kod toliko osoba nema nikakvih, ili ima samo blagih
  simptoma. Lockdown je stoga „besmislen`` i steti drustvu. Osnov
  svedske strategije je bio taj, da „ljudi nisu glupi``. Giesecke racuna
  sa stopom smrtnosti izmedju 0,1 i 0,2\%, slicno influenci. Italija i
  New York su bili lose pripremljeni i nisu zastitili svoje rizicne
  grupe.
\item
  Britanski ~Guardian
  \href{https://www.theguardian.com/environment/2020/apr/20/air-pollution-may-be-key-contributor-to-covid-19-deaths-study?utm_medium}{citira
  nove studije}, po kojima je zagadjenje vazduha kljucni faktor kod
  Covid-19 smrtnih slucajeva. Tako je 80\% smrtnih slucajeva u cetiri
  drzave usledilo u oblastima sa najvecom zagadjenoscu (medju njima
  Lombardija i Madrid)
\item
  Nemacki list ZEIT tematizuje
  \href{https://www.zeit.de/zustimmung?url=https\%3A\%2F\%2Fwww.zeit.de\%2F2020\%2F18\%2Fkliniken-coronavirus-intensivbetten-patienten-behandlung-notaufnahme}{prazne
  nemacke klinike}, koje su na nekim odeljenjima i do 70\% nepopunjene.
  Cak i pregledi na onkologiji su otkazivani, kao i ne akutne, ali od
  zivotne vaznosti bitne transplantacije organa, da bi se obezbedilo
  dovoljno mesta za Covid-19 pacijente, koji su dosada u najvecem broju
  slucajeva izostali.
\item
  U Nemackoj je uvedena obaveza nosenja zastitnih maski u javnom prevozu
  i prodavnicama. Predsednik Svetskog udruzenja lekara, Frank
  Montgomery, je to
  \href{https://www.aerztezeitung.de/Politik/Montgomery-haelt-Maskenpflicht-fuer-falsch-408844.html}{kritikovao}
  kao pogresno, a predvidjeno koriscenje salova i marama kao „smesno``.
  Studije pokazuju zaista, da koriscenje maski u svakodnevnici od strane
  zdravih i osoba bez simptoma nema nekog merljivog efekta, zbog cega
  svajcarski infektolog, Dr. Vernazza govori o jednom
  \href{https://infekt.ch/2020/04/atemschutzmasken-fuer-alle-medienhype-oder-unverzichtbar/}{„medijskom
  hype-u``.} Drugi kriticari govore o jednom simbolu
  \href{https://multipolar-magazin.de/artikel/maskenpflicht-gesellschaftliches-klima}{„prisilne,
  javno vidljive poslusnosti``.}
\item
  Jedna studija WHO nije nasla
  \href{https://www.heise.de/tp/features/COVID-19-WHO-Studie-findet-kaum-Belege-fuer-die-Wirksamkeit-von-Eindaemmungsmassnahmen-4706446.html}{gotovo
  nikakve naucne dokaze.} za delotvornost mera, kao sto su socijalna
  distanca, ogranicenje putovanja ili zabrana izlaska.
\item
  Jedna nemacka laboratorija je
  \href{http://www.labor-augsburg-mvz.de/de/aktuelles/coronavirus}{objavila
  pocetkom aprila}, da testovi za Corona-virus, po novoj preporuci WHO i
  onda vaze za pozitivne, ako je specificna krajnja sekvenca virusa
  Covid-19 negativna, a samo opsta krajnja sekvenca raznih Corona virusa
  pozitivna. Znaci to moze dovesti do toga, da i drugi Corona virusi
  (virusi prehlade) izazovu pozitivni rezultat. Uz to je laboratorija
  objasnila, da Covid-19 antitela cesto tek dve do tri nedelje posle
  pocetka simptoma mogu biti evidentirani. To se mora uzeti u obzir,
  kako se stvarni broj osoba, koje su vec imune na Covid-19, ne bi
  potcenio.
\item
  Kako u
  \href{https://www.20min.ch/schweiz/news/story/-rzte-und-Politiker-fordern-Corona-Impfzwang-20853917}{Svajcarskoj},
  tako i u
  \href{https://www.faz.net/agenturmeldungen/dpa/soeder-waere-fuer-deutschlandweite-impfpflicht-gegen-corona-16738369.html}{Nemackoj}
  su pojedini politicari zahtevali „obaveznu vakcinaciju protiv
  Corone``. Pri tom je vakcinacija, na primer protiv „svinjskog gripa``
  2009/2010 dovela u mnogo slucajeva do
  \href{https://www.ibtimes.co.uk/brain-damaged-uk-victims-swine-flu-vaccine-get-60-million-compensation-1438572}{teskih
  neuroloskih ostecenja,} posebno kod dece, kao i do zahteva za
  obestecenje u milionskim iznosima.
\item
  Lekar iz Kalifornije, Dr. dan Erickson izvestava u
  \href{https://www.turnto23.com/news/coronavirus/video-interview-with-dr-dan-erickson-and-dr-artin-massihi-taken-down-from-youtube}{jednoj
  konferenciji za stampu} o svojim zapazanjima o Covid-19. U kaliforniji
  i drugim saveznim drzavama su bolnice i odeljenja intenzivne nege do
  sada ostali pretezno prazni. Dr. Erickson izvestava o lekarima iz vise
  saveznih drzava, koji su „pod pritiskom`` izdavali smrtne listove sa
  uzrokom Covid-19, iako to nije bilo njihovo misljenje. Dr. Erickson
  argumentuje, da zabrane izlaska i strah slabe imuni sistem i zdravlje
  ljudi. Vec dolazi na videlo jasno povecanje „sekundarnih efekata`` kao
  sto su alkoholizam, depresije, samoubistva kao i zlostavljanje dece i
  bracnih partnera. Dr. Erickson preporucuje da se samo bolesne osobe
  stave u karantin, a ne i zdrave, odnosno celo drustvo. Zastita maskom
  ima smisla samo u akutnim situacijama kao u bolnici, ali ne i u
  svakodnevnici.
\item
  Profesor Christoph Kuhbandner:
  \href{https://www.heise.de/tp/features/Von-der-fehlenden-wissenschaftlichen-Begruendung-der-Corona-Massnahmen-4709563.html?seite=all}{O
  nedostajucim naucnim obrazlozenjima uvedenih mera:} „Objavljeni
  brojevi o novim infekcijama dramaticno precenjuju stvarno sirenje
  Corona virusa. Posmatrano razantno povecanje novih infekcija se svodi
  iskljucivo na cinjenicu, da je broj testova sa vremenom razantno
  rastao (vidi grafiku). Dakle nije postojalo nikada, barem po
  objavljenim brojkama, neko eksponencijalno sirenje Corona virusa.
  Objavljeni brojevi o novim infekcijama skrivaju cinjenicu, da je isti
  jos od pocetka -- sredine marta u padu.
\end{itemize}

\includegraphics{https://swprs.files.wordpress.com/2020/04/zunahme-infektionen-tests-tag.png?w=550\&h=404}

\hypertarget{svedska-predstavljanje-u-medijima-versus-realnost}{%
\subparagraph{\texorpdfstring{\textbf{Svedska: Predstavljanje u medijima
versus
realnost}}{Svedska: Predstavljanje u medijima versus realnost}}\label{svedska-predstavljanje-u-medijima-versus-realnost}}

Neki citaoci su bili iznenadjeni smanjenjem broja smrtnih slucajeva u
Svedskoj, jar u najvecem broju medija se pokazuje jedna strmo rastuca
krivulja. O cemu se radi? Najveci broj medija pokazuje kumulirane
brojeve po datumu objavljivanja, dok svedske vlasti objavljuju znatno
jasnije dnevne brojeve po datumu smrti.

Svedske vlasti naglasavaju stalno, da novi objavljeni slucajevi nisu svi
u poslednjih 24 sata umrli, ali mnogi mediji to ignorisu. (vidi
grafiku). Najnoviji brojevi u svedskoj mogu kao i u svim zemljama jos
narasti, ali u trendu opadanja to u sustini nista ne menja.

Uz to treba reci, da i ovi brojevi predstavljaju smrtne slucajeve sa ali
ne i obavezno usled Corona virusa. Prosecna starost je i u Svedskoj
preko 80 godina, oko 50\% smrtnih slucajeva se desilo u starackim
domovima, efekat po normalno stanovnistvo je bio minimalan, iako Svedska
raspolaze sa jednim od najnizih
\href{https://link.springer.com/article/10.1007/s00134-012-2627-8}{kapaciteta
intenzivne medicine} u Evropi.

Doduse je i svedska vlada u toku Corona-krize dobila nova
\href{https://www.tagesschau.de/faktenfinder/ausland/corona-kursaenderung-schweden-103.html}{ovlascenja
u vanrednom stanju} i mogla bi ucestvovati u kasnijim programima
pracenja kontakata.

\includegraphics{https://swprs.files.wordpress.com/2020/04/sweden-corona-media-vs-reality.png?w=736\&h=338}

\hypertarget{situacija-u-velikoj-britaniji}{%
\subparagraph{\texorpdfstring{\textbf{Situacija u Velikoj
Britaniji}}{Situacija u Velikoj Britaniji}}\label{situacija-u-velikoj-britaniji}}

Broj smrtnih slucajeva se u Velikoj Britaniji zadnjih nedelja jako
povecao, ali se i pored toga nalazi u razini
\href{http://inproportion2.talkigy.com/}{najtezih talasa gripa} u
zadnjih 50 godina (vidi grafiku). I u Velikoj Britaniji otpada
\href{https://ltccovid.org/2020/04/12/mortality-associated-with-covid-19-outbreaks-in-care-homes-early-international-evidence/}{50\%}
dodatnih smrtnih slucajeva na staracke domove, koji ne profitiraju od
jednog opsteg Lockdown-a.

\href{https://ltccovid.org/2020/04/12/mortality-associated-with-covid-19-outbreaks-in-care-homes-early-international-evidence/}{Kod
skoro 50\%} dodatnih smrtnih slucajeva se ne radi o pretpostavljenim
Covid-19 slucajevima, i
\href{https://www.ft.com/content/67e6a4ee-3d05-43bc-ba03-e239799fa6ab}{do
25\%} dodatnih smrtnih slucajeva se desava kod kuce. I u velikoj
Britaniji se postavlja pitanje, da li Lockdown vise koristi ili steti.

Frasor Nelson, izdavac britanskog Spectator-a,
\href{https://www.telegraph.co.uk/politics/2020/04/09/boris-worried-lockdown-has-gone-far-can-end/}{izvestava,}
da vladina odeljenja srednjerocno racunaju na do 150.000 dodatnih
smrtnih slucajeva radi Lockdown-a, znatno vise, nego sto ce to ,
pretpostavlja se, da uzrokuje Covid-19. Od pre kratkog vremena je poznat
slucaj jedne 17-godisnje ucenice i pevacice,
\href{https://sports.yahoo.com/coronavirus-bethany-palmer-teenager-death-suicide-152707750.html}{koja
se ubila} radi Lockdown-a.

Upadljivo je da je u Engleskoj znatno povisena
\href{https://www.euromomo.eu/}{smrtnost} kod osoba od 15 do 64 godina
starosti. Uzrok moze lezati u cestim kardiovaskularnim bolestima, ili
biti uslovljen efektima Lockdown-a.

\href{http://inproportion2.talkigy.com/}{Projekat InProportion} je
objavio brojne nove grafike, koje pokazuju aktuelnu smrtnost u
poredjenju sa ranijim talasima gripa i drugim uzrocima smrti. Dalje Web
stranice, koje se kriticki bave uvedenim merama jesu
\href{https://lockdownsceptics.org/}{Lockdown Skeptics} i
\href{https://www.ukcolumn.org/}{UK Column.}

\includegraphics{https://swprs.files.wordpress.com/2020/04/inproportion2_chart5.png?w=736\&h=363}

\hypertarget{svajcarska-preterana-smrtnost-znatno-ispod-jakih-epidemija-gripa}{%
\subparagraph{\texorpdfstring{\textbf{Svajcarska: Preterana smrtnost
znatno ispod jakih epidemija
gripa}}{Svajcarska: Preterana smrtnost znatno ispod jakih epidemija gripa}}\label{svajcarska-preterana-smrtnost-znatno-ispod-jakih-epidemija-gripa}}

\begin{itemize}
\tightlist
\item
  Jedna seroloska studija Univerziteta u Zenevi
  \href{https://www.hug-ge.ch/medias/communique-presse/seroprevalence-covid-19-premiere-estimation}{dolazi
  do zakljucka}, da je u zenevskom kantonu najmanje sest puta vise osoba
  imalo kontakt sa Covid-19, nego sto se to pretpostavljalo. Time se
  smanjuje letalitet Covid-19 i u Svajcarskoj na razinu promila, dok
  oficijelni izvori jos uvek govore o 5\%.
\item
  I u najteze pogodjenom kantonu Tessin je
  \href{https://www.bluewin.ch/de/news/schweiz/sp-chef-levrat-will-die-reichen-schropfen-383977.html}{polovina}
  dodatnih smrtnih slucajeva usledila u starackim domovima, koji od
  opsteg Lockdown-a ne profitiraju.
\item
  U Svajcarskoj je vec 1,85 miliona, odnosno preko jedne trecine
  zaposlenih ljudi,
  \href{https://www.bluewin.ch/de/news/schweiz/sp-chef-levrat-will-die-reichen-schropfen-383977.html}{prijavljeno}
  na skraceno radno vreme. Ekonomski troskovi su od strane ETH Zürich za
  period od marta do juna 2020. procenjeni na 32 milijardi.
\item
  ETH Zürich je svoju studiju o stopi reprodukcije
  \href{https://www.nau.ch/politik/regional/coronavirus-eth-forscherin-passt-studie-an-und-stutzt-lockdown-65695817}{preinacio}
  i sad „podrzava`` Lockdown Savezne skupstine. U osnovi to ne menja
  rezultat studije: Stopa reprodukcije je pala jos pre Lockdown-a na
  stabilnu vrednost 1, jednostavne svakodnevne mere i mere higijene su
  za to bile dovoljne i time je Lockdown za suzbijanje epidemije bio
  \href{https://infekt.ch/2020/04/sind-wir-tatsaechlich-im-blindflug/}{nepotreban.}
\item
  \textbf{Infosperber}: Corona:
  \href{https://www.infosperber.ch/Artikel/Medien/Corona-NZZ-deckt-das-Nachplappern-anderer-Medien-auf}{NZZ
  otkriva jednostavno ponavljanje ostalih medija}. „Veliki mediji
  precutkuju, da rade sa netransparentnim podacima``
\item
  \textbf{Ktipp}:
  \href{https://www.ktipp.ch/artikel/artikeldetail/bund-fast-alle-zahlen-ohne-gewaehr/}{Savez:
  skoro svi brojevi bez garancije}. „Ove godine je u prvih 14 nedelja
  umrlo manje ljudi ispod 65 godina, nego u poslednjih pet godina. Kod
  onih preko 65 godina je taj broj takodje bio ~srazmerno nizak.
\end{itemize}

Sledeca grafika pokazuje, da je ukupna smrtnost u Svajcarskoj u prvom
kvartalu 2020. bila u normali i do sredine aprila jos uvek oko 2.000
osoba ispod epidemije gripa 2015. 50\% smrtnih slucajeva se desilo u
\href{https://www.nzz.ch/zuerich/coronavirus-zuerich-aendert-nun-das-testregime-in-heimenauch-viele-aeltere-covid-19-infizierte-entwickeln-keine-symptome-zuerich-aendert-nun-das-testregime-in-heimen-ld.1552089}{starackim
domovima}, koji nisu profitirali od Lockdown-a.

Ukupno se 75\% dodatnih smrtnih slucajeva desilo
\href{https://www.tagesspiegel.de/wissen/woran-sterben-corona-patienten-wirklich-ein-schweizer-forscher-macht-hoffnung-im-kampf-gegen-covid-19/25750666.html}{kod
kuce}, dok su bolnice i intenzivna odeljenja i dalje
\href{https://swprs.files.wordpress.com/2020/04/intensivbettenbelegung-schweiz-2020-04-14.png}{znatno
neoptereceni}, a mnoge operacije su otkazane. I u Svajcarskoj se
postavlja bitno pitanje, ~da nije mozda Lockdown mnogo vise egzistencija
i zivota kostao, nego sto ih je mogao spasiti.

\includegraphics{https://swprs.files.wordpress.com/2020/04/schweiz-todesfaelle-2010-2020.png?w=736\&h=357}

\hypertarget{politicke-vesti}{%
\subparagraph{\texorpdfstring{\textbf{Politicke
vesti}}{Politicke vesti}}\label{politicke-vesti}}

\begin{itemize}
\tightlist
\item
  Web strana \href{https://kollateral.news/}{kollateral.news} jedne
  nemacke pravnice sakuplja izvestaje o „Lockdown-patnji`` i o stvarnoj
  situaciji u nemackim bolnicama.
\item
  Nemacki lekari opste prakse su objavili
  \href{https://aerzteinnenvorort.de/der-appell}{apel upucen politici i
  nauci}, u kome zahtevaju „jedno odgovornije delovanje u
  Corona-krizi``.
\item
  Jedan minhenski lokalni radio, koji je u martu intervjuisao lekare sa
  kriticnim stavom, je posle zalbi
  \href{https://norberthaering.de/medienversagen/radiomuenchen-blm-meinungsvielfalt/}{obavesten}
  od strane kontrole medija, da ce „ovakve problematicne emisije u
  buduce morati da budu izostavljene``.
\item
  \textbf{Video}: U australijskoj saveznoj drzavi Queensland je
  policijski helikopter sa sistemom za nocno trazenje otkrio tri
  mladica, koji su preko noci pili pivo na krovu jedne kuce i time
  krsili „Corona-pravila``. Preko megafona su bili obavesteni, da je
  kuca od strane policije opkoljena i da moraju da podju ka izlazu.
  Ljudi su
  \href{https://www.dailystar.co.uk/news/world-news/police-helicopter-uses-night-vision-21899640}{kaznjeni}
  sa po 1000\$.
\item
  Jedan svajcarski lekar, Corona-kriticar, koji je od strane specijalne
  svajcarske policije uhapsen i sproveden na psihijatriju (vidi update
  od 15. aprila), je u medjuvremenu na slobodi. Jedno
  \href{https://uncut-news.ch/wp-content/uploads/2020/04/Wer-löste-den-Fehlalarm-aus.pdf}{istrazivanje
  lista Weltwoche} je pokazalo, da je lekar uhapsen na osnovu netacnih
  optuzbi: nije postojala nikakva pretnja vlastima, niti se radilo o
  posedovanju oruzja. Time je hapsenje bilo, kako izgleda, politicki ili
  lokalno-politicki motivisano.
\item
  Kako u
  \href{https://www.sn.at/panorama/oesterreich/arzt-droht-berufsverbot-wegen-kritik-an-corona-massnahmen-86594140}{Austriji},
  tako i u
  \href{https://magyarhang.org/belfold/2020/04/16/etikai-vizsgalat-indul-az-orvos-ellen-aki-szerint-nincs-jarvany-es-az-idosek-csak-a-felelemtol-halnak-meg/}{Madjarskoj}
  preti lekarima koji se kriticki oglasavaju prema uvedenim merama,
  zabrana prakse.
\item
  U Nigeriji je policija, pri sprovodjenju mera zabrane izlaska, po
  zvanicnim podacima
  \href{https://www.bbc.com/news/world-africa-52317196}{ubila vise
  ljudi} nego sam virus.
\item
  U Izraelu moze od sredine marta unutrasnja tajna antiteroristicka
  policija Shin Bet u saradnji sa policijom da
  \href{https://www.jewishpress.com/news/the-courts/state-to-high-court-even-more-shin-bet-involvement-in-fighting-the-coronavirus/2020/04/14/}{prati
  i nadgleda} stanovnistvo preko mobilnih telefona, kako bi u kontekstu
  Covid-19 pratila kontakte i bila u stanju da ljude stavi u kucni
  pritvor. Ove mere su isprva uvedene bez odluke parlamenta i treba da
  vaze do kraja aprila.
\item
  \textbf{OffGuardian}:
  \href{https://off-guardian.org/2020/04/23/the-seven-step-path-from-pandemic-to-totalitarianism/}{The
  Seven Step Path from Pandemic to Totalitarianism} (nemacki
  \href{https://uncut-news.ch/wp-content/uploads/2020/04/In-sieben-Schritte-von-der-Pandemie-zum-Totalitarismus.pdf}{prevod})
\item
  \textbf{UK Column}:
  \href{https://www.ukcolumn.org/article/who-controls-british-government-response-covid19-part-one}{Who
  controls the British Government response to Covid--19?}
\end{itemize}

\hypertarget{21-april-2020}{%
\paragraph{21. april 2020.}\label{21-april-2020}}

\hypertarget{medicinske-vesti-1}{%
\subparagraph{\texorpdfstring{\textbf{Medicinske
vesti}}{Medicinske vesti}}\label{medicinske-vesti-1}}

\begin{itemize}
\tightlist
\item
  Profesor medicine na Stanford-u John Ioannidis objasnjava u jednom
  \href{https://www.youtube.com/watch?v=cwPqmLoZA4s}{novom jednocasovnom
  intervjuu} vise studija o Covid-19. Letalitet od Covid-19 se po
  profesoru Ioannidis-u nalazi „u oblasti sezonalnog gripa``. Za osobe
  ispod 65 godina je rizik od smrti cak i u „hotspots`` uporediv sa
  svakodnevnom voznjom autom na posao, za zdrave osobe ispod 65 godina
  je rizik od smrti „potpuno zanemarljiv``. Jedino u New York-u je rizik
  od smrti za osobe ispod 65 godina u oblasti jednog profesionalnog
  vozaca kamiona.
\item
  Profesor Carl Heneghan, direktor centra za medicinu baziranu na
  cinjenicama Univerziteta u Oxford-u,
  \href{https://news.yahoo.com/lockdown-damage-outweighs-coronavirus-warning-121940675.html}{upozorava
  u jednom novom prilogu}, da bi steta usled Lockdown-a mogla biti veca
  nego od virusa. Vrhunac epidemije je u najvecem broju ~zemalja
  dosegnut vec pre uvodjenja mera.
\item
  Jedna
  \href{http://publichealth.lacounty.gov/phcommon/public/media/mediapubhpdetail.cfm?prid=2328}{nova
  seroloska studija} u jednoj cetvrti Los Angeles-a dolazi do zakljucka,
  da je 28 do 55 puta vise osoba, nego sto je to dosada pretpostavljano,
  vec imala Covid-19 (bez jakih simptoma), cime se opasnost bolesti
  odgovarajuci tome smanjuje.
\item
  U gradu Chelsea kod Boston-a je
  \href{https://archive.is/20200418222442/https:/www.bostonglobe.com/2020/04/17/business/nearly-third-200-blood-samples-taken-chelsea-show-exposure-coronavirus/}{trecina
  od 200 davalaca krvi} imala antitela protiv Covid-19. Polovina njih je
  izjavila, da je u proteklom mesecu imala neki simptom prehlade. U
  jednom domu za beskucnike kod Bostona je nesto vise od trecine
  pozitivno testirano, pri cemu
  \href{https://www.wsbtv.com/news/trending/coronavirus-cdc-reviewing-stunning-universal-testing-results-boston-homeless-shelter/ZADQ45HCAZEVJAZA3OTCUR7M6M/}{niko
  nije imao simptome.}
\item
  Skotska javlja, da polovina (dodatnih) kreveta intenzivne nege
  \href{https://www.heraldscotland.com/news/18377095.coronavirus-scotland-half-icu-beds-empty/}{stoji
  prazna}. U medjuvremenu stagnira prijem novih pacijenata.
\item
  Odeljenje za prijem hitnih slucajeva gradske bolnice u Bergamu je
  pocetkom ove nedelje po prvi put u poslednjih 45 dana
  \href{https://orf.at/stories/3162642/}{opet potpuno prazno}.
\item
  U medjuvremenu se tretira opet vise ljudi sa drugim bolestima nego
  Covid-19 pacijenata.
\item
  Izvestaj strucnog casopisa Lancet je dosao jos pocetkom aprila do
  \href{https://www.thelancet.com/journals/lanchi/article/PIIS2352-4642(20)30095-X/fulltext}{zakljucka},
  da zatvaranje skola radi suzbijanja Corona-virusa nije imalo nikakav
  ili samo minimalan efekat.
\item
  Jedno devetogodisnje francusko dete sa Corona-infekcijom je imalo
  kontakt sa 172 osobe, a da pri tome
  \href{https://www.n-tv.de/panorama/172-Kontaktpersonen-von-Corona-verschont-article21727469.html}{nijednu
  osobu to dete nije zarazilo}. To potvrdjuje ranije rezultate, da se
  Corona infekcija (za razliku od influence) prakticno od strane dece ne
  prenosi.
\item
  Nemacki profesor mikrobiologije u penziji, Sucharit Bhakdi, dao je
  \href{https://kenfm.de/kenfm-am-set-gespraech-mit-prof-dr-sucharit-bhakdi-zu-covid-19/}{novi
  jednocasovni intervju} o Covid-19. Profesor Bhakdi pored ostalog
  smatra, da je vecina medija „potpuno neodgovorno`` delovala.
\item
  Nemacka Inicijativa za etiku nege
  \href{http://pflegeethik-initiative.de/2020/04/15/corona-krise-falsche-prioritaeten-gesetzt-und-ethische-prinzipien-verletzt/}{kritikuje
  pausalne zabrane posecivanja i mucne tretmane na intenzivnoj nezi}
  pacijenata iz starackih domova: „Jos pre Corone je svaki dan umiralo
  oko 900 starih ljudi koji su bili negovani u nemackim starackim
  domovima, bez da su morali kratko pre toga da budu dovedeni u bolnicu.
  Kod njih bi bio smislen, ako uopste ikakav, samo jedan palijativni
  tretman. Na osnovu svega sto znamo o Coroni, ne postoji ni jedan
  jedini logicni razlog da zastitu od zaraze i dalje vise vrednujemo od
  osnovnih gradjanskih prava. Ukinite zabranu posecivanja! Ona je
  necovecna i nepotrebna!``
\item
  Najstarija zena u svajcarskom kantonu St. Gallen je umrla prosle
  nedelje u 109.-oj godini starosti. Ona je prezivela „spanski grip``
  1918. godine, nije bila inficirana Coronom i „za svoje godine se
  odlicno snalazila``. „Izolacija radi Corone`` joj je
  „\href{https://swprs.files.wordpress.com/2020/04/tagblatt-109.jpg}{jako
  naskodila}``: „Bez svakodnevnih poseta clanova porodice je jednostavno
  uvenula.
\item
  Svajcarski kardiolog Dr. Nils Kucher izvestava, da se u Svajcarskpj
  oko 75\% dodatnih smrtnih slucajeva desava ne u bolnicama,
  \href{https://www.tagesspiegel.de/wissen/woran-sterben-corona-patienten-wirklich-ein-schweizer-forscher-macht-hoffnung-im-kampf-gegen-covid-19/25750666.html}{vec
  kod kuce}. To razjasnjava sigurno, zasto su svajcarske bolnice i
  intenzivna odeljenja
  \href{https://swprs.files.wordpress.com/2020/04/intensivbettenbelegung-schweiz-2020-04-14.png}{pretezno
  prazni.} Osim toga je vec poznato, da se oko 50\% svih dodatnih
  smrtnih slucajeva
  \href{https://www.nzz.ch/zuerich/coronavirus-zuerich-aendert-nun-das-testregime-in-heimenauch-viele-aeltere-covid-19-infizierte-entwickeln-keine-symptome-zuerich-aendert-nun-das-testregime-in-heimen-ld.1552089}{desava}
  u starackim domovima i domovima za negu ljudi. Dr. Kucher
  pretpostavlja, da deo tih osoba umire od iznenadne plucne embolije. To
  je moguce. Opet se postavlja pitanje, koju ulogu kod tih dodatnih
  smrtnih slucajeva ima Lockdown.
\item
  Italijanski Zdravstveni zavod ISS
  \href{https://www.iss.it/en/rapporti-covid-19/-/asset_publisher/btw1J82wtYzH/content/id/5334891}{upozorava},
  da Covid-19 pacijenti iz oblasti Sredozemnog mora, koji cesto imaju
  jednu genetsku posebnost metabolizma pod nazivom favizam, ne treba da
  budu tretirani lekovima protiv malarije kao sto je chloroquine, jer
  kod favizma mogu da dovedu do smrti. To je jedan dodatni pokazatelj,
  da jedna pogresna, ili suvise agresivna medikacija, moze bolest
  \href{https://www.sciencedaily.com/releases/2020/02/200206110703.htm}{dodatno
  da pogorsa}.
\item
  Rubikon:
  \href{https://www.rubikon.news/artikel/120-expertenstimmen-zu-corona}{120
  eksperata o Coroni}. Sirom sveta kritikuju renomirani naucnici,
  lekari, pravnici i drugi eksperti desavanja u vezi Corone.
\end{itemize}

\hypertarget{rangiranje-jedne-pandemije}{%
\subparagraph{\texorpdfstring{\textbf{Rangiranje jedne
pandemije}}{Rangiranje jedne pandemije}}\label{rangiranje-jedne-pandemije}}

Zdravstvene vlasti USA su 2007. godine definisale
\href{https://www.cidrap.umn.edu/news-perspective/2007/02/hhs-ties-pandemic-mitigation-advice-severity}{5
stepeni} za pandemiju gripa kao i odgovarajuce mere. Pet kategorija se
razlikuju na osnovu posmatranog letaliteta (CFR) pandemije, od prve
kategorije (manje od 0,1\%) do pete kategorije (vise od 2\%). Aktuelna
Corona-pandemija bi po tom kljucu verovatno spadala u drugu kategoriju
(0,1\% do 0,5\%). Za tu kategoriju ja tada bila predvidjena samo
„dobrovoljna izolacija bolesnih osoba`` kao glavna mera.

WHO je 2009. izbacila letalitet iz definicije za jednu pandemiju, tako
da je od tada moguce u principu svaki svetski talas gripa proglasiti za
pandemiju, kao sto se to
\href{https://www.forbes.com/consent/?toURL=https://www.forbes.com/2010/02/05/world-health-organization-swine-flu-pandemic-opinions-contributors-michael-fumento.html}{prvi
put desilo} 2009/2010. sa jako blagim „svinjskim gripom``, za koji je
vladama prodato vakcina u vrednosti od 18 milijardi dolara.

Dokumentacija TrustWHO („Veruj kome?``), koja je tematizovala sumnjivu
ulogu WHO u okviru „svinjskog gripa`` je
\href{https://www.youtube.com/watch?v=VjQGyqVN5RM}{izbrisana sa VIMEO}

\hypertarget{primarius-pietro-vernazza-jednostavne-mere-su-dovoljne}{%
\subparagraph{\texorpdfstring{\textbf{Primarius Pietro Vernazza:
jednostavne mere su
dovoljne}}{Primarius Pietro Vernazza: jednostavne mere su dovoljne}}\label{primarius-pietro-vernazza-jednostavne-mere-su-dovoljne}}

Svajcarski lekar infektolog, Pietro Vernazza, pokazuje na osnovu
rezultata Robrt-Koch-Instituta i ETH Zürich
\href{https://infekt.ch/2020/04/sind-wir-tatsaechlich-im-blindflug/}{u
svom najnovijem prilogu}, da je Covid-19 epidemija vec pre uvodjenja
mera bila pod kontrolom:

„Ovi rezultati su eksplozivni: ocigledno pokazuju oba rada manje-vise
identicno: jednostavne mere, uzdrzavanje od velikih manifestacija i
uvodjenje mera higijene imaju jak ucinak. Narod je u stanju da sprovede
te savete i te mere mogu da skoro u potpunosti ~zaustave epidemiju. U
svakom slucaju su te mere dovoljne, da nas zdravstveni sistem tako
zastite, da bolnice ne budu preopterecene.

\includegraphics{https://swprs.files.wordpress.com/2020/04/ch-reproduktionszahl-eth-infekt.png?w=650\&h=379}

\hypertarget{svajcarska-kumulirana-ukupna-smrtnost-u-normali}{%
\subparagraph{\texorpdfstring{\textbf{Svajcarska: kumulirana ukupna
smrtnost u
normali}}{Svajcarska: kumulirana ukupna smrtnost u normali}}\label{svajcarska-kumulirana-ukupna-smrtnost-u-normali}}

U Svajcarskoj se
\href{https://swprs.files.wordpress.com/2020/04/ch-sterblichkeit-kumuliert-q1-2020.pdf}{kumulirana
ukupna smrtnost} u prvom kvartalu (do 5. Aprila) nalazila na srednjem
ocekivanom nivou i sa preko 1500 osoba ispod gornjeg ocekivanog nivoa.
Ukupna smrtnost je bila do pocetka aprila uz to preko 2000 osoba ispod
vrednosti iz godine teske sezone gripa 2015. (vidi grafike)

\includegraphics{https://swprs.files.wordpress.com/2020/04/schweiz-todesfaelle-2010-2020.png?w=700\&h=339}

\hypertarget{svedska-epidemija-bez-lockdown-a-na-kraju}{%
\subparagraph{\texorpdfstring{\textbf{Svedska: Epidemija bez Lockdown-a
na
kraju}}{Svedska: Epidemija bez Lockdown-a na kraju}}\label{svedska-epidemija-bez-lockdown-a-na-kraju}}

Najnoviji brojevi o pacijentima i smrtnim slucajevima pokazuju, da se
epidemija u Svedskoj blizi kraju. I u Svedskoj je povisena smrtnost
nastala u starackim domovima, koji nisu bili dovoljno zasticeni, kao sto
\href{https://www.washingtontimes.com/news/2020/apr/15/sweden-coronavirus-rates-easing-despite-loose-rule/}{objasnjava}
epidemiolog.

Svedski narod profitira sad u poredjenju sa ostalim zemljama, uz to od
jednog vrlo visokog prirodnog imuniteta protiv Covid-19 virusa, koji ga
moze bolje stititi od jednog moguceg „drugog talasa``.

Moze se poci od toga, da Covid-19 u svedskoj ukupnoj smrtnosti 2020.
nece biti vidljiv. Svedski primer pokazuje, da su „Lockdown-i`` bili
medicinski nepotrebni ili cak kontraproduktivni kao i drustveno i
ekonomski
katastrofalni.\href{https://swprs.files.wordpress.com/2020/04/sweden-deaths-day-2.png}{\includegraphics{https://swprs.files.wordpress.com/2020/04/sweden-deaths-day-2.png?w=736\&h=293}}

Pozitivno testirani smrtni slucajevi~
(\href{https://en.wikipedia.org/wiki/2020_coronavirus_pandemic_in_Sweden\#Charts7be6f9f87457ed9aa}{FOHM/Wikipedia})

\hypertarget{anegdote-vs-cinjenice}{%
\subparagraph{\texorpdfstring{\textbf{Anegdote vs.
cinjenice}}{Anegdote vs. cinjenice}}\label{anegdote-vs-cinjenice}}

Usled nedostajucih naucnih cinjenica, poneki mediji koriste zastrasujuce
anegdote, da bi odrzali strah u narodu. Jedan tipicni primer su navodno
od Covid-19 umrla „zdrava deca``, kod kojih se kasnije najcesce
ispostavlja, da
\href{https://www.dailymail.co.uk/news/article-8193487/Coroner-refuses-rule-COVID-19-cause-death-six-week-old-Connecticut-baby.html}{ipak
nisu} umrla od Covid-19 ili su bila vec
\href{https://www.msn.com/de-ch/news/other/spanischer-nachwuchs-trainer-stirbt-an-corona/ar-BB11gT64}{ranije
tesko bolesna}.

Austrijski mediji su pre kratkog vremena izvestavali o
\href{https://www.rainews.it/tgr/tagesschau/articoli/2020/04/tag-Coronavirus-Lungeschaden-Forschung-Uniklinik-Innsbruck-6708e11e-28dc-4843-a760-e7f926ace61c.html}{roniocima},
koji su 6 nedelja posle Covid-19 plucnih bolesti imali snizenu
kondiciju. U jednom odeljku se govori o „ireverzibilnim posledicama``, u
sledecem se objasnjava kako je sve to „nejasno``. Ne napominje se, da
ronioci posle upale pluca generalno moraju da
\href{https://www.gesundheitsfrage.net/g/frage/tauchen-lungenentzuendung}{pauziraju}
6 do 12 meseci.

Cesto su navodjeni neuroloski efekti, kao sto je temporarni gubitak cula
mirisa ili ukusa. I ovde se najcesce izostavlja cinjenica, da je to isto
tako \href{https://www.ncbi.nlm.nih.gov/pubmed/25294743}{tipicni efekat}
virusa gripa i prehlade, i da kod Covid-19 ima
\href{https://www.ncbi.nlm.nih.gov/pubmed/23948436}{pre blazi oblik}.

U drugim izvestajima se isticu posledice po razne organe, kao sto su
bubrezi, jetra ili mozak, bez napomene, da se kod mnogih pacijenata
radilo o jako starim osobama i sa
\href{https://www.epicentro.iss.it/coronavirus/sars-cov-2-decessi-italia}{teskim
hronicnim oboljenjima.}

\hypertarget{politicke-vesti-1}{%
\subparagraph{\texorpdfstring{\textbf{Politicke
vesti}}{Politicke vesti}}\label{politicke-vesti-1}}

\begin{itemize}
\tightlist
\item
  WOZ:
  \href{https://www.woz.ch/2016/grundrechte/wenn-die-angst-regiert}{Kada
  strah vlada}. „Sa dronama, App-ovima i zabranom demonstracija: u toku
  Corona-krize su osnovne slobode potkopane. Ako ne budemo pazili,
  ostace takve i posle Lockdown-a -- ipak ovakva ekstremna situacija
  nudi osnove za nadu``
\item
  Multipolar:
  \href{https://multipolar-magazin.de/artikel/die-massnahmen-wirken}{Koji
  cilj se ovde sledi?} „Vlada samu sebe hvali, siri parole da se izdrzi
  i u isto vreme koci objavljivanje osnovnih podataka, na osnovu kojih
  bi bilo moguce pouzdano proceniti opasnost od virusa. Brzo i odlucno
  deluju vlasti pri tom u instaliranju sumnjivih instrumenata, kao
  „Corona-Apps`` za kolektivno merenje pulsa i pracenje kontakata.``
\item
  Profesor Christian Piska, ekspert za javno pravo i Legal Tech u Becu:
  „Austrija se promenila. Jako, iako vecina izgleda da to jednostavno
  prihvata. Postepeno ozivljavanje ekonomije, ostavimo po strani -- ~mi
  najednom zivimo u ukolnostima svojstvenim za jednu policijsku drzavu
  sa ogranicenjima nasih osnovnih ljudskih prava, koje bi najbolje
  pristajale diktatorskim rezimima. To je Pandorina kutija, koja --
  jednom otvorena, eventualno
  \href{https://kurier.at/meinung/das-juristische-totschlagargument-vom-menschenleben/400814570}{nikad
  vise nece moci da bude zatvorena.}
\item
  Vise od 300 naucnika iz 26 zemalja upozoravaju na
  \href{https://www.golem.de/news/corona-app-300-wissenschaftler-warnen-vor-zentraler-datenspeicherung-2004-147973.html}{besprimernu
  kontrolu drustva} putem Corona-Apps, koji nisu usaglaseni sa zastitom
  licnih podataka. Vise naucnika i univerziteta je u medjuvremeno
  istupilo iz evropskog projekta za pracenje kontakata PEPP-PT, radi
  nedovoljne transparentnosti. Pre nekog vremena je objavljeno, da je u
  projekat involvirana i svajcarska firma AGT, koja je za arapske drzave
  izgradila sisteme za masovnu opservaciju i kontrolu.
\item
  U Izraelu je demonstriralo oko 5000 ljudi (sa odstojanjem od 2m)
  \href{https://edition.cnn.com/2020/04/20/middleeast/israel-protest-social-distancing-intl/index.html}{protiv
  uvedenih mera vlade Netanjahu-a}: „Oni govore o eksponencijalnom
  sirenju slucajeva Corone, ali jedino sto eksponencijalno raste je broj
  ljudi, koji ustaju da brane nasu zemlju i nasu demokratiju``
\item
  Irski novinar Jason O'Toole, koji zivi u Madridu,
  \href{https://www.rt.com/op-ed/486350-spain-tough-rules-covid-19-lockdown/}{opisuje
  situaciju u Spaniji}: „sa vojskom na ulicama je tesko ne govoriti o
  opsadnom stanju. George Orwell-ov Big Brother je ovde ziv i spanska
  policija nadgleda svakog uz pomoc kamera ili drona. () Samo u prvih
  cetiri nedelje je osudjeno 650.000 ljudi na novcanu kaznu i 5.568 je
  uhapseno. Ja sam bio sokiran, kada sam video video-snimak, na kome
  jedan policajac uz brutalnu silu hapsi jednog mladog psihicki bolesnog
  coveka, koji je ocigledno upravo krenuo sa hlebom kuci.
\item
  OffGuardian:
  \href{https://off-guardian.org/2020/04/18/the-disturbing-developments-in-uk-policing/}{Uznemirujuci
  razvoj kod britanske policije.}
\item
  USA novinarka Whitney Webb pise u novom clanku pod naslovom
  \href{https://www.thelastamericanvagabond.com/top-news/techno-tyranny-how-us-national-security-state-using-coronavirus-fulfill-orwellian-vision/}{„Tehno-tiranija:
  kako sigurnosni aparat USA koristi Corona-virus, kako bi ostvario
  orvelovsku viziju.``}: „Protekle godine je jedna USA vladina komisija
  zahtevala uvodjenje vestackom inteligencijom upravljanog sistema za
  masovnu kontrolu, koji daleko premasuje one koji se u bilo kojoj
  drugoj zemlji koriste, kako bi ~se osigurala americka hegemonija u
  domenu vestacke inteligencije. Sada se pod plastom borbe protiv
  Corona-virusa mnoge „prepreke``, koje su do sada bile navodjene kao
  razlog za sprecavanje uvodjenja tog sistema, brzo odstranjuju.``
\item
  U jednom
  \href{https://www.thelastamericanvagabond.com/top-news/all-roads-lead-dark-winter/}{clanku
  pocetkom aprila} se Whitney Webb pozabavila centralnom ulogom „Centra
  za sigurnost zdravlja`` John Hopkins Univerziteta u aktuelnoj
  pandemiji, kao i umesanosti istog u ranijim pandemijama i simulacijama
  bioloskog oruzja, te njegovim vezama sa vojskom USA i drzavnim
  sigurnosnim aparatom.
\item
  „Istina o Fauci-ju``: U jednom
  \href{https://childrenshealthdefense.org/news/the-truth-about-fauci-featuring-dr-judy-mikovits/}{novom
  intervjuu} govori americka doktorka virologije, Judy Mikovits, o
  svojim negativnim profesionalnim iskustvima sa Dr. Anthony Fauci-jem,
  koji trenutno bitno utice na kreiranje mera americke vlade u borbi
  protiv Covid-19.
\item
  Dobrovoljne organizacije upozoravaju, da ce neuporedivo vise ljudi
  \href{https://www.welt.de/wirtschaft/article207092745/Corona-Pandemie-Rezession-beschert-der-Welt-die-noch-groessere-Katastrophe.html}{umreti
  usled ekonomskih posledica uvedenih mera,} nego od samog virusa.
  Prognoze govore o 35 do 65 miliona ljudi, koji ce zbog globalne
  recesije pasti u apsolutno siromastvo. A mnogima od njih preti smrt od
  gladi.
\item
  U Nemackoj se
  \href{https://www.boeckler.de/pdf/p_wsi_pb_38_2020.pdf}{rancuna} u
  2020. na 2,35 miliona zaposlenih, koji rade skraceno radno vreme, to
  je vise nego duplo vise u odnosu na krizu iz 2008./2009.
\end{itemize}

\hypertarget{18-april-2020}{%
\paragraph{18. april 2020.}\label{18-april-2020}}

\hypertarget{medicinske-vesti-2}{%
\subparagraph{\texorpdfstring{\textbf{Medicinske
vesti}}{Medicinske vesti}}\label{medicinske-vesti-2}}

\begin{itemize}
\tightlist
\item
  Jedna
  \href{https://www.medrxiv.org/content/10.1101/2020.04.14.20062463v1}{nova
  seroloska studija} sa Stanford Univerziteta sprovedena u oblasti Santa
  Clara u Kaliforniji, otkrila je antitela kod 50 do 85 puta vise osoba,
  nego sto je to do sada bilo pretpostavljano, iz cega proizilazi, da je
  letalitet Covid-19 izmedju 0,12\% i 0,2\%, ili cak ispod toga (sto
  znaci u razini jedne jake influence)
\item
  Centar za medicinu baziranu na cinjenicama (CEBM) Univerziteta Oxford,
  na osnovu jedne
  \href{https://www.cebm.net/covid-19/global-covid-19-case-fatality-rates/}{nove
  analize} polazi od toga, da letalitet Covid-19 (IFR) iznosi izmedju
  0,1\% i 0,36\% (to znaci u opsegu jednog jakog gripa). Kod ljudi preko
  70 godina starosti bez teskih prethodnih oboljenja bi letalitet
  iznosio ispod 1\%. Kod onih preko 80 godina je letalitet izmadju 3\% i
  15\%, zavisno od toga, da li su smrtni slucajevi dosada usledili sa
  ili usled bolesti. Letalitet kod dece se nalazi -- za razliku od gripa
  -- prakticno na nuli. U vezi visoke smrtnosti u severnoj Italiji,
  ukazuje ova grupa istrazivaca izmedju ostalog na
  \href{https://www.ansa.it/english/news/science_tecnology/2019/11/19/italy-top-in-eu-in-antibiotic-resistance_369e0123-0107-445e-8c17-f11932c9d27c.html}{najvisu
  stopu rezistencije na antibiotike u Evropi}, ~koja vazi za Italiju.
  Podaci italijanskih vlasti pokazuju zaista, da ~je oko 80\% umrlih
  tretirano antibioticima, sto ukazuje na bakterioloske superinfekcije.
\item
  Finski epidemiolog, profesor Mikko Paunio sa Univerziteta u Helsinkiju
  je u jednoj
  \href{https://lockdownsceptics.org/wp-content/uploads/2020/04/How-the-World-got-Fooled-by-COVID-ed-2c.pdf}{analizi}
  vise internacionalnih studija dosao do zakljucka, da letalitet
  Covid-19 (IFR) iznosi 0,1\% ili manje (sto znaci u opsegu sezonalnog
  gripa). Utisak jednog povisenog letaliteta je nastao, jer se virus
  jako brzo rasirio, posebno u domacinstvima sa vise generacija u
  Italiji i Spaniji. „Lockdown``-mere su bile svuda prekasno uvedene i
  nisu nicemu doprinele, odnosno su bile cak kontraproduktivne.
\item
  Ukupna smrtnost u Italiji se u prvom kvartalu 2020. (do 08.aprila),
  uprkus Covid-19, nalazila
  \href{https://latina.biz/confronto-decessi-italia-1-trimestre-2019-e-2020-con-covid19/}{ispod
  onih u prethodne tri godine}. Jedan od razloga za to bi mogao biti,
  blaga sezona gripa radi blage zime, koja je bila delimicno
  „kompenzovana`` bolescu Covid-19.
  (\href{http://demo.istat.it/bilmens2019gen/index.html}{izvor
  podataka})
\item
  Ukupna smrtnost u Svajcarskoj se nalazila u prvom kvartalu 2020. (do
  05. Aprila), na osnovu podataka Saveznog zavoda za statistiku, uprkos
  Covid-19 na
  \href{http://demo.istat.it/bilmens2019gen/index.html}{srednjem
  normalnom nivou}. I ovde bi blaga zima mogla da bude jedan od bitnih
  razloga.
\item
  U Svajcarskoj su, po jednom istrazivanju od 14. aprila, ne samo
  bolnice imale jako mali broj pacijenata,
  \href{https://swprs.files.wordpress.com/2020/03/42ad8-intensivbettenbelegung2bschweiz2b-2b14-4-20202bpdf.png}{vec
  i odeljenja intenzivne nege}. Ovde se dakle i dalje postavlja pitanje,
  gde i od cega su pozitivno testirani Svajcarci zaista umrli.
\item
  Predsednik Nemackog Bolnickog Udruzenja
  \href{https://www.bz-berlin.de/deutschland/kliniken-verband-schlaegt-alarm-wegen-corona-regeln}{dize
  alarm}: Vise od 50\% svih planiranih operacija je otkazano, u tom
  „zastoju`` se radi o hiljadama. Uz to se tretira 30\% do 40\% manje
  pacijenata sa infarktom i izlivom krvi u mozak, posto oni iz straha od
  Corone ne smeju u klinike. Sirom zemlje je trenutno 150.000 slobodnih
  bolnickih kreveta i 10.000 slobodnih kreveta intenzivne nege.. U
  Berlinu je samo 68 kreveta na intenzivnoj nezi zauzeto
  Corona-pacijentima, klinika za hitne slucajeve sa 1000 kreveta nije u
  upotrebi.
\item
  Novi podaci RKI pokazuju, da je reproduktivni broj Covid-19 vec pre
  Lockdown
  \href{https://www.rki.de/DE/Content/Infekt/EpidBull/Archiv/2020/Ausgaben/17_20_SARS-CoV2_vorab.pdf}{pao}
  ispod kriticne vrednosti od jedan. Opste higijenske mere su bile dakle
  dovoljne, da se spreci eksponencionalno sirenje. Vec ranije je to
  pokazano od strane ETH Zürich.
\item
  U Kanadi je \href{https://orf.at/stories/3162365/}{31 osoba umrla u
  jednom starackom domu}, „posto je skoro kompletno osoblje, zaduzeno za
  negu ljudi, iz straha od Corona -virusa bezeci napustilo ustanovu.
  Zdravstvene vlasti su nasle ljude u domu u Dorval-u kod Montreal-a tek
  nekoliko dana kasnije --~ i mnoge prezivele koji su dehidrirali,
  izgladneli i pali u letargiju.
\item
  Jedan skotski lekar, koji se brine i o starackim domovima,
  \href{https://drmalcolmkendrick.org/2020/04/17/care-homes-and-covid19/}{pise}:
  „Sta je bila strategija vlade u vezi ~~sa starackim domovima?
  Dosadasnje akcije su stanje mnogo, mnogo pogorsale``
\item
  Na jednom francuskom nosacu aviona je bio
  \href{https://www.ouest-france.fr/sante/virus/coronavirus/coronavirus-au-moins-940-marins-positifs-sur-le-charles-de-gaulle-et-son-escorte-6810816}{pozitivno
  testiran} 1081 vojnik. Do sada je 50\% ostalo bez simptoma, a oko 50\%
  je bilo sa blagim simptomima. 24 vojnika je na bolnickom lecenju, od
  kojih jedan na intenzivnom odeljenju (nije utvrdjeno, da li sa
  prethodnim oboljenjima).
\item
  Nemacki virolog, Christian Drosten, smatra da je moguce, da su neki
  ljudi putem kontakta sa normalnim (Corona-) virusima prehlade,
  \href{https://www.watson.de/!324026684}{stekli} takozvani pozadinski
  imunitet prema novom Corona-virusu.
\item
  Hamburski lekar sudske medicine Klaus Püschel, koji je vec pregledao
  umrle pozitivno testirane, objasnjava u jednom novom prilogu:
  \href{https://www.abendblatt.de/hamburg/article228908865/hamburg-corona-virus-uke-infektion-covid-19-pueschel-coronavirus-krise-patienten-impfstoff-immunitaet-krankenhaeuser-kontaktverbot-kliniken-infektionsrate-krankheit-pandemie-test-lungenkrankheit-sars-cov-epidemie-sars-cov-2.html}{„Brojevi
  ne opravdavaju strah od Corone``}. Njegova saznanja: „Corona je jedna
  srazmerno bezazleno oboljenje. Mi se moramo pozabaviti time, da je
  Corona jedna normalna infekcija i moramo nauciti da zivimo sa tim i to
  bez karantina``. Umrli pacijenti koje je on ispitao, su svi imali tako
  teska oboljenja, da bi svi oni, ma kako to grubo zvucalo, u toku ove
  godine, umrli. Püschel je jos jasniji: „Vreme virologa je proslo. Mi
  treba sada da pitamo nekog drugog, sta je ispravno u Corona-krizi, na
  primer lekare intenzivne medicine.``
\item
  Jedan
  \href{https://emedicine.medscape.com/article/227820-overview}{pregled
  na Medscape} pokazuje, da se oboljenja prouzrokovana Corona-virusima
  krajem aprila tipicno povlace -- sa ili bez Lockdown.
\item
  Infosperber:
  \href{https://www.infosperber.ch/Artikel/Gesundheit/Weniger-Corona-Falle-Einfach-weniger-testen}{„Manje
  slucajeva Corone? Jednostavno testirajmo manje!``} Svakodnevno
  javljanje „novih slucajeva`` ne govori mnogo o tome, na kom stepenu se
  epidemija nalazi. Neodgovorno je stvarati strah sa krivuljom
  akumuliranih smrtnih slucajeva.
\item
  OffGuardian:
  \href{https://off-guardian.org/2020/04/17/8-more-experts-questioning-the-coronavirus-panic/}{Jos
  osam eksperata, koji su kriticki okrenuti prema panici usled Corone}.
  (na engleskom)
\end{itemize}

\hypertarget{vestacko-disanje-kod-covid-19}{%
\subparagraph{\texorpdfstring{\textbf{Vestacko disanje kod
Covid-19}}{Vestacko disanje kod Covid-19}}\label{vestacko-disanje-kod-covid-19}}

Dalji strucnjaci u Evropi i USA su se javili po pitanju tretmana
kriticnih Covid-19 pacijenata i hitno odvracaju od primene invazivnog
vestackog disanja (intubacijom). Kod Covid-19 pacijenata ne dolazi do
akutnog otkazivanja pluca (ARDS) vec se radi o nedostatku kiseonika.

\begin{itemize}
\tightlist
\item
  WELT:
  \href{https://www.welt.de/vermischtes/article207221877/Corona-Pandemie-Sterberate-bei-Beatmungspatienten-gibt-Raetsel-auf.html}{Nejasnoce
  u stopi smrtnosti kod pacijenata sa vestackim disanjem}
\item
  AP: \href{https://apnews.com/8ccd325c2be9bf454c2128dcb7bd616d}{Some
  doctors moving away from ventilators for virus patients}
\item
  Video: \href{https://www.youtube.com/watch?v=QPlEUAVjxV8}{Dr. Thomas
  Voshaar o tretmanu vestackim disanjem kod Covid-19 pacijenata}
\item
  Video: \href{https://www.youtube.com/watch?v=NmRlvX3VrAQ}{New York
  intensive care doctor on Covid19 as a possible diffusion hypoxemia}
\item
  \href{https://link.springer.com/article/10.1007/s00134-020-06033-2}{COVID-19
  pneumonia: different respiratory treatments for different phenotypes?}
\end{itemize}

\hypertarget{politicke-vesti-2}{%
\subparagraph{\texorpdfstring{\textbf{Politicke
vesti}}{Politicke vesti}}\label{politicke-vesti-2}}

\begin{itemize}
\tightlist
\item
  Video: \href{https://www.youtube.com/watch?v=ZphK_CMUbKg}{Policijska
  sila u okviru „Corona-Lockdown`` u celom svetu}
\item
  Video\href{https://www.youtube.com/watch?v=SO2JMkKtq40}{: „Svajcarsko
  Vece treba u zatvor. Jedna polemika``}
\item
  Video: \href{https://www.youtube.com/watch?v=eU6IdglI-wc}{„Svajcarskim
  lekarima je stavljena brnjica, Savezno Vece je u svadji``}~ Intervju
  sa Dr.med.Stephan Rietiker-om, pokretacem i uredjivacem
  \href{https://www.insidecorona.ch/}{InsideCorona.ch}
\item
  Jedan svajcarski gradjanin je poslao
  \href{https://faktenb-covid-19-massnahmen.jimdofree.com/}{hitni zahtev
  Saveznom Ustavnom Sudu i Saveznom Vecu} sa ciljem momentalnog ukidanja
  uvedenih mera.
\item
  Nemacki ekonom Norbert Haering
  \href{https://norberthaering.de/}{objasnjava u vise priloga,} kako se
  Corona-kriza koristi za vec dugo planirano uvodjenje instrumenata
  nadzora i kontrole sirom sveta u oblastima putovanja, platnih sistema,
  pracenja kontakata i biometrije.
\item
  Giorgio Agamben, italijanski filozof,
  \href{https://www.nzz.ch/feuilleton/coronavirus-giorgio-agamben-zum-zusammenbruch-der-demokratie-ld.1551896}{o
  uvedenim merama}: „Jedna zemlja, cak jedna kultura implodira upravo, i
  nikoga nije briga, izgleda. Sta se pred nasim ocima desava u zemljama,
  koje za sebe tvrde da su civilizovane?``
\item
  Italijanski advokati
  \href{https://www.tvprato.it/2020/04/la-camera-civile-degli-avvocati-pratesi-chiede-lannullamento-del-dpcm-del-10-aprile-e-illegittimo/}{ulazu
  zalbu} protiv uvedenih mera vlade.
\item
  Nemacki profesor ekonomije, Stefan Homburg, u WELT:
  \href{https://www.msn.com/de-de/nachrichten/coronavirus/warum-deutschlands-lockdown-falsch-ist-–-und-schweden-vieles-besser-macht/ar-BB12E6km}{„Zasto
  je nemacki Lockdown pogresan -- a Svedska mnogo toga bolje radi``}:
  „Ukupno gledano su zemlje kao Svedska, Juzna Koreja ili Taiwan, time
  sto nisu uvele Lockdown, pametnije reagovale. Njihovi virolozi su
  vodili narod i politicare kroz krizu sigurnom rukom i nisu im ulivali
  nesigurnost stalnim promenama kursa. Corona-virus je uspesno obuzdan
  bez ugrozenih gradjanskih prava ili radnih mesta. Nemacka treba da se
  ugleda na tu politiku.`` (Vidi takodje:
  \href{https://www.youtube.com/watch?v=Vy-VuSRoNPQ}{Video intervju sa
  prof. Homburg-om}).
\end{itemize}

\hypertarget{16-april-2020}{%
\paragraph{16. april 2020.}\label{16-april-2020}}

\begin{itemize}
\tightlist
\item
  Londonski Times \href{https://archive.is/2eKCW}{izvestava}, da 50\%
  trenutne preterane smrtnosti nije prouzrokovano Corona-virusom, vec
  efektima Lockdown-a, opste panike i drustvenog kolapsa. Tu se radi o
  oko 3000 ljudi
  \href{https://www.ons.gov.uk/peoplepopulationandcommunity/birthsdeathsandmarriages/deaths/bulletins/deathsregisteredweeklyinenglandandwalesprovisional/weekending3april2020}{nedeljno}.
  Taj broj bi mogao da bude cak i znatno veci, jer britanska definicija
  Corona-smrtnih slucajeva ukljucuje smrtne slucajeve sa (umesto usled),
  kao i slucajeve, gde se samo sumnja na prisustvo virusa. Pri tome oko
  50\% Corona-smrtnih slucajeva se
  \href{https://ltccovid.org/2020/04/12/mortality-associated-with-covid-19-outbreaks-in-care-homes-early-international-evidence/}{odnosi}
  na staracke domove, koji usled opsteg Lockdown-a nisu bolje zasticeni.
\item
  U Danskoj postoji u medjuvremenu
  \href{https://jyllands-posten.dk/debat/breve/ECE12074246/vi-skulle-aldrig-have-trykket-paa-stopknappen/}{kajanje}
  radi uvedenih mera: „Trebalo je da pritisnemo „stop``. Danski sistem
  zdravstva je imao situaciju pod kontrolom. Totalni Lockdown je bio
  korak predaleko.``, argumentuje profesor Jens Otto Lunde Jorgensen iz
  Aarhus Univerzitetske bolnice.. U Danskoj vec pocinje skolska nastava.
\item
  Profesor David Katz sa Yale-Univerziteta, koji je jos u pocetku
  upozoravao na stetne posledice uvedenih mera, dao je
  \href{https://www.youtube.com/watch?v=VK0Wtjh3HVA}{jednocasovni
  intervjuu} povodom aktuelne sitiacije.
\item
  Nemacki virolog Hendrik Streek objasnjava, da do sada slucajevi
  \href{https://today.rtl.lu/news/science-and-environment/a/1498185.html}{kontaminacije
  na predmetima} u supermarketima, restoranima ili frizerskim salonima
  nisu bili utvrdjeni.
\item
  Novi podaci o antitelima iz italijanskog mesta Robbia u Lombardiji
  pokazuju, da je oko
  \href{https://www.tgcom24.mediaset.it/cronaca/a-robbio-pv-il-22-ha-o-ha-avuto-il-coronavirus-ok-del-sindaco-ai-test-per-tutti_17285128-202002a.shtml}{deset
  puta vise ljudi} bilo kontaminirano virusom, nego sto se prvobitno
  pretpostavljalo, jer su pokazivali samo blage ili nikakve simptome.
  Stopa imunizacije iznosi oko 22\%.
\item
  Novi podaci iz kantona Zürich pokazuju, da se oko 50\% svih smrtnih
  slucajeva desilo u starckim domovima ili domovima za negu; i pored
  toga je oko 40\% svih pozitivno testiranih osoba bilo bez simptoma.
  Prosecna starost pozitivno testiranih umrlih iznosi 84 godine.
\item
  Svajcarski lekar infektolog, Pietro Vernazza izjasnjava se o
  strategiji
  „\href{https://infekt.ch/2020/04/exitstrategie-lockdown/}{zivota sa
  virusom}`` i preporucuje izmedju ostalog jednu individualnu
  optimizaciju u zastiti ugrozenih osoba. I imunitet ostalih gradjana
  doprinosi zastiti ugrozenih ljudi.
\item
  Novi britanski Website \href{https://lockdownsceptics.org/}{Lockdown
  Skeptics} kriticki izvestava o Covid-19, uvedenim merama i
  izvestavanju u medijima.
\item
  Austrijska drustvena
  „\href{https://www.initiative-corona.info/}{Inicijativa za
  Corona-informacije bazirane na cinjenicama}`` nudi jedan pregled
  studija i analiza o Corona- virusima.
\item
  Dokumentacija: \href{https://www.youtube.com/watch?v=dYlia_fQOLk}{„WHO
  -- u rukama lobija``} (ARTE, 2017)
\end{itemize}

\hypertarget{15-april-2020}{%
\paragraph{15. april 2020.}\label{15-april-2020}}

\hypertarget{vesti-iz-medicine}{%
\subparagraph{\texorpdfstring{\textbf{Vesti iz
medicine}}{Vesti iz medicine}}\label{vesti-iz-medicine}}

\begin{itemize}
\tightlist
\item
  Profesor Alexander Kekule, jedan od vodecih nemackih mikrobiologa i
  epidemiologa, u jednom intervjuu za britanski Telegraf zahteva
  \href{https://www.telegraph.co.uk/news/2020/04/11/german-scientist-predicted-european-epidemic-calls-end-lockdown/}{kraj
  zabrane izlaska}, posto ta mera pravi vise stete nego virus. Kod ljudi
  ispod 50 godina su teski razvoji bolesti ~ili smrtni slucajevi „skoro
  iskljuceni``. Trebalo bi da normalna populacija brzo stvori imunitet,
  dok ljude iz rizicne grupe treba zastititi. Ne moze se ~cekati na
  vakcinu, sto ce trajati najmanje sest do dvanaest meseci, nego se mora
  naci nacin, kako ziveti sa virusom.
\item
  Nemacka mreza za medicinu baziranu na evidentnim cinjenicama
  izvestava, da je letalitet jedne jake sezonske influence (gripa) kao
  na primer 2017/2018, od strane Robert-Koch-Instituta
  \href{https://www.ebm-netzwerk.de/en/publications/covid-19}{procenjen}
  na 0,4\% -- 0,5\%, a ne kao sto se to ranije pretpostavljalo samo na
  0,1\%. To bi znacilo, da letalitet od Covid-19 lezi cak ispod
  letaliteta ~jednog jakog sezonalnog gripa, samo deluje u jednom kracem
  vremenskom periodu.
\item
  Luksemburski Tageblatt
  \href{https://swprs.files.wordpress.com/2020/04/volksblatt_schweden_corona_20200414_18.pdf}{izvestava},
  da svedska opustena strategija, kako izgleda, funkcionise. Uprkos
  minimalnim merama i glasne internacionalne kritike, izgleda da se
  situacija znacajno smiruje. Ogromna poljska bolnica, koja je
  sagradjena kod Stockholm-a, ostaje i dalje zatvorena, jer za nju ne
  postoji potreba. Broj pacijenata na intenzivnoj nezi je na stalnom
  niskom nivou ili se cak smanjuje. „Postoji puno slobodnih mesta na
  odeljenjima za intenzivnu negu u svim Stockholmskim bolnicama.
  Priblizavamo se ispravljanju bolesnicke krivulje``, javlja primarius
  Karolinska klinike. Do sada je bilo oko 900 smrtnih slucajeva sa
  Covid-19.
\item
  Direktno poredjenje izmedju Velike Britanije (sa zabranom izlaska i
  ostalim merama) i Svedske (prakticno bez njih) pokazuje, da obe zemlje
  u odnosu na populaciju, pokazuju
  \href{http://www.theblogmire.com/a-comparison-of-lockdown-uk-with-non-lockdown-sweden/}{skoro
  isti} odnos izmedju broja zarazenih i umrlih.
\item
  Jedno saopstenje u New England Journal of Medicine
  \href{https://www.nejm.org/doi/full/10.1056/NEJMc2009316}{kaze}, da je
  88\% pozitivno testiranih trudnih zena u jednom ispitivanju bilo bez
  simptoma -- jedan jako visoki procenat, koji se slaze sa ranijim
  istrazivanjima iz Kine i Islanda.
\item
  Profesor Dan Yanim, direktor istrazivacke laboratorije na Univerzitetu
  u Tel Aviv-u,
  \href{https://www.ynet.co.il/articles/0,7340,L-5714371,00.html}{objasnjava
  u jednom intervjuu}, da novi Corona-virus za veliki deo populacije
  nije jako opasan i da bi cilj morao biti, svaranje jednog brzog
  imuniteta. Steta usled uvedenih mera je ogromna i sa tim novcem bi
  bilo bolje sagraditi jednu novu kliniku.
\item
  Predsednik izraelskog Nacionalnog Istrazivackog Saveta, profesor Isaac
  Ben-Israel,
  \href{http://www.israelnationalnews.com/News/News.aspx/278658}{argumentuje},
  da je epidemija Corone u najvecem broju zemalja posle 8 nedelja
  prosla, bez obzira koje su mere bile uvedene. On preporucuje stoga
  brzo ukidanje uvedenih mera.
\item
  Britanski profesor statistike, David Spiegelhalter, ~pokazuje, da
  rizik od smrti usled Covid-19 odgovara riziku
  \href{https://medium.com/wintoncentre/how-much-normal-risk-does-covid-represent-4539118e1196}{normalne
  smrtnosti} i da je samo u starosnoj dobi izmedju 70 i 80 godina vidno
  povisen (uporedi donju grafiku u clanku).
\item
  Profesor Karin Moelling, penzionisana direktorka Instituta za
  virologiju Univerzitata u Zürich-u, koja je
  \href{https://www.rubikon.news/artikel/die-stimme-der-vernunft}{vec u
  pocetku kritikovala preterane mere}, naglasava u jednom
  \href{https://www.youtube.com/watch?v=4rl2sqLcDoQ}{novom intervjuu}
  ulogu lokalnih faktora kao sto su zagadjenost vazduha i gustina
  naseljenosti.
\item
  Britanski Guardian je
  \href{https://www.theguardian.com/world/2015/aug/14/air-pollution-in-china-is-killing-4000-people-every-day-a-new-study-finds}{ukazao}
  2015. na to, da ekstremno zagadjenje vazduha u kineskim gradovima
  dnevno ubija 4000 ljudi. To je vise od ukupnog broja umrlih od
  Covid-19, koje je Kina objavila.
\item
  Nemacki virolog Hendrik Streek se
  \href{https://www.tagesspiegel.de/wissen/virologe-streeck-zur-coronavirus-studie-die-veroeffentlichung-zu-heinsberg-war-nicht-leichtfertig/25735672.html}{brani}
  od napada na njegov istrazivacki rad. Streek je utvrdio da je
  letalitet kod Covid-19 (u odnosu na broj zarazenih)~ oko 0,37\%, a
  smrtnost (racunajuci u odnosu na celokupnu populaciju) oko 0,06\%, sto
  odgovara jednom jakom sezonalnom gripu.
\item
  Jedan svajcarski biofizicar je po prvi put analizirao pozitivnu stopu
  zarazenosti u Svajcarskoj i nju
  \href{https://mail.protonmail.com/\#SARSCoV2}{graficki prikazao}. Iz
  grafike se vidi, da pozitivna stopa rasta osciluje izmedju 10\% i
  20\%, i da Lockdown nema nekog bitnog uticaja. Svajcarske vlasti i
  mediji nisu do danas ovo eksplicitno prikazali.
\item
  Jedan svajcarski istrazivac je analizirao najnoviji Covid-19 izvestaj
  Saveznog Zavoda za zravstvo i ponovo dolazi do
  \href{https://covid-19-fakten.blogspot.com/2020/04/der-bag-situationsbericht-vom-1442020.html}{jako
  kriticne ocene}: „BAG-izvestaj je nepodesan za politiku i donosenje
  kompetentnih odluka, ponovo je jako nespecifican, sa propustima i
  neubedljiv.
\item
  Svajcarski lekar i Infektolog, Dr. Pietro Vernazza, objasnjava
  ~\href{https://infekt.ch/2020/04/hinterlaesst-coronavirus-eine-immunitaet/}{u
  jednom novom prilogu}, ~da se kod navodno nepostojeceg stvaranja
  imuniteta, radi o retkim, pojedinacnim slucajevima, te da pri
  pazljivijem posmatranju to ne predstavlja nikakav problem, ali da se
  to od strane nekih medija prezentuje u vidu zastrasujucih poruka.
\item
  Iz Francuske dolazi
  \href{https://www.midilibre.fr/2020/04/09/coronavirus-ces-suicides-de-malades-ou-de-personnes-tenaillees-par-langoisse,8839373.php}{sve
  vise vesti o samoubistvima} iz straha od Corona-virusa, odnosno iz
  straha da neko drugi nije eventualno mogao da bude zarazen.
\item
  Novi francuski sajt \href{https://covidinfos.net/}{Covid Infos} se
  kriticki bavi sa temom Covid-19 ~i medijskim izvestavanjem.
\end{itemize}

\hypertarget{usa-i-velika-britanija}{%
\subparagraph{\texorpdfstring{\textbf{USA i Velika
Britanija}}{USA i Velika Britanija}}\label{usa-i-velika-britanija}}

\begin{itemize}
\tightlist
\item
  Na americkom ratnom brodu, Theodore Roosevelt, je 600 mornara je
  pozitivno testirano na Covid-19.
\item
  Prvi mornar je u medjuvremenu
  \href{https://www.theguardian.com/world/2020/apr/14/sailor-dies-from-covid-19-and-600-test-positive-after-outbreak-on-uss-theodore-roosevelt-guam}{usled
  ili sa Covid-19 preminuo}. Ovaj ratni brod ce biti vazan kao predmet
  istrazivanja dejstva zaraze na zdrave ljude ispod 65 godina starosti.
\item
  Britanski profesor patologije u penziji, Dr. John Lee, argumentuje,
  potrebna je
  \href{https://www.spectator.co.uk/article/to-understand-covid-we-need-evidence-scepticism-and-vigorous-debate}{robusna
  rasprava bazirana na cinjenicama}, kako bi se izbegle „velike
  greske``. Mnoge cifre kojima se sluze vlade i mediji nisu bile
  pouzdane.
\item
  U Velikoj Britaniji je trenutno
  \href{https://www.hsj.co.uk/acute-care/nhs-hospitals-have-four-times-more-empty-beds-than-normal/7027392.article}{40\%
  bolnickih kreveta prazno}, to je cetiri puta vise nego inace. Razlog
  tome je u jako snizenom prijemu normalnih pacijenata. Sto se tice
  kreveta za intenzivnu negu, ciji je broj povecan, ukupno je~ 78\%
  zauzeto, u nekim regionima i vise. Uz to je 10\% pomocnog ososblja u
  karantinu.
\item
  Privremeni Corona-centri americke vojske kod New York-a su
  \href{https://nypost.com/2020/04/09/usns-comfort-and-javits-center-mostly-empty-amid-coronavirus/}{pretezno
  prazni}. „Predvidjanje potrebnog bolnickog lecenja je sedam puta
  \href{https://www.nytimes.com/2020/04/10/nyregion/new-york-coronavirus-hospitals.html}{precenjeno}``.
\item
  Jedna USA studija
  \href{https://www.medrxiv.org/content/10.1101/2020.04.01.20050542v2}{dolazi
  do zakljucka}, da se Corona-virus vec mnogo vise rasirio, nego sto se
  to pretpostavljalo, ali da kod vecine ljudi ne izaziva nikakve ili
  samo blage simptome, tako da bi letalitet mogao da iznosi samo 0,1\%,
  sto odgovara jednom sezonskom gripu. Usled lakseg prenosa se broj
  obolelih, na primer u New York-u, desio u
  \href{https://archive.is/7w2XE}{kracem vremenskom roku, nego obicno}.
\item
  Primarius, lekar za bolesti pluca i intenzivnu medicinu Estern
  Virginia Medical School objasnjava u jednom novom dokumentu o tretmanu
  Covid-19 pacijenata.: „Covid-19 ne prouzrokuje tipicno otkazivanje
  pluca\ldots{}Ova bolest mora drugacije da se leci i verovatno je, da
  se radi primene vestackog disanja, tj. usled ostecivanja pluca,
  situacija pogorsava.
\item
  U USA je tvrdio jedan guverner, da je jedno malo dete u svetu
  najmladja zrtva Covid-19. Poznanici porodice su nasuprot izjavili, da
  se dete u jednom tragicnom slucaju
  \href{https://www.washingtonexaminer.com/news/candace-owens-accuses-connecticut-governor-of-lying-about-coronavirus-death-calls-for-resignation}{ugusilo}
  i naknadno u bolnici bilo pozitivno testirano. Nadlezni lekar sudske
  medicine
  \href{https://www.dailymail.co.uk/news/article-8193487/Coroner-refuses-rule-COVID-19-cause-death-six-week-old-Connecticut-baby.html}{nije
  utvrdio Covid-19 kao uzrok smrti.}
\item
  Lekarka iz US drzave Montana objasnjava u
  \href{https://www.youtube.com/watch?v=V0lIWZpiRU0}{jednom predavanju},
  kako se smrtni listovi kod slucajeva, za koje se Covid-19 kao uzrok
  smrti samo pretpostavlja, manipulisu.
\end{itemize}

\hypertarget{domovi-za-negustaracki-domovi}{%
\subparagraph{\texorpdfstring{\textbf{Domovi za negu/staracki
domovi}}{Domovi za negu/staracki domovi}}\label{domovi-za-negustaracki-domovi}}

\begin{itemize}
\tightlist
\item
  Jedna analiza podataka iz pet evropskih zemalja pokazuje, da
  stanovnici starackih domova dosada predstavljaju
  \href{https://ltccovid.org/2020/04/12/mortality-associated-with-covid-19-outbreaks-in-care-homes-early-international-evidence/}{42\%
  do 57\% svih Covid-19 smrtnih slucajeva}. U isto vreme pokazuju tri
  studije iz USA, da do 50\% svih pozitivno testiranih stanovnika
  ovakvih domova u trenutku testiranja (jos) ne pokazuje nikakve
  simptome. Iz toga proizilaze dva zakljucka: s jedne strane izgleda da
  je opasnost Corona-virusa skoncentrisana na -- kao sto se
  pretpostavljalo -- jednu malu, jako ranjivu grupu ljudi, koju treba
  jos vise cuvati. Sa druge strane moze da se pretpostavi, da jedan deo
  tih ljudi ne umire od, ili ne samo od Corona-virusa, nego radi
  ekstremnog stresa koji je sa tim povezan. U Nemackoj i Italiji je bilo
  izvestavano o stanovnicima domova, koji su bez simptoma odjednom
  umrli.
\item
  Jedan nemacki medicinski radnik palijativnog smera argumentuje
  \href{https://www.deutschlandfunk.de/palliativmediziner-zu-covid-19-behandlungen-sehr-falsche.694.de.html?dram:article_id=474488}{u
  jednom intervjuu}, da se u tretmanu Covid-19 pacijenata postavljaju
  jako pogresni prioriteti i povredjuju svi eticki principi. Postoji
  jedno jako jednostrano usmerenje ka intenzivom tretmanu, iako odnos
  izmedju koristi i stete cesto nije u redu. Cesto se od jednog
  pacijenta koji zahteva najvisi stepen nege i koji je u proslosti
  najvecim delom bio palijativno tretiran, pravi jedan pacijent za
  intenzivno odeljenje i pri tom podvrgava jednom mucnom, ali cesto
  bezizglednom tretmanu (sa vestackim disanjem). Ispred svega bi zelja
  doticnog pacijenta morala da ima prioritet.
\end{itemize}

\includegraphics{https://swprs.files.wordpress.com/2020/04/c19-nursing-homes.jpg?w=700\&h=218}

\hypertarget{politicke-teme}{%
\subparagraph{\texorpdfstring{\textbf{Politicke
teme}}{Politicke teme}}\label{politicke-teme}}

\begin{itemize}
\tightlist
\item
  U Nemackoj je jedna \href{http://beatebahner.de/}{pravnica medicinskog
  smera}, koja je pri ustavnom sudu ulozila zalbu protiv uvedenih mera u
  vezi Corone i pozvala na demonstracije, uhapsena i ~dva dana drzana na
  odeljenju jedne
  \href{https://www.rnz.de/nachrichten/heidelberg_artikel,-nach-aufruf-zu-corona-demo-heidelberger-anwaeltin-aus-psychiatrischer-einrichtung-entlassen-150-unte-_arid,508747.html}{zatvorske
  psihijatrije.} Jedan dalji advokat pita u svom
  \href{https://www.nachrichtenspiegel.de/2020/04/14/brief-an-die-bundesrechtsanwaltskammer-in-causa-bahmer/}{otvorenom
  pismu} upucenom nemackoj Saveznoj advokatskoj komori: „Pravnicu radi
  protesta u psihijatriju? Jesmo li opet dosli do toga u Nemackoj?``
\item
  U Svajcarskoj je jedan lekar, Corona-kriticar, radi navodne pretnje
  vlastima, od strane specijalne policije bio
  \href{https://www.srf.ch/news/regional/aargau-solothurn/festnahme-von-corona-kritiker-verschwoerung-oder-normale-intervention-der-aargauer-behoerden}{uhapsen
  i sproveden u psihijatriju}. Njegova porodica je u medjuvremenu
  objasnila, da nikakve pretnje nije bilo. Lekar je uz to objasnio, da
  mu na saslusanju pretnja vlastima nije prebacivana. Policija je
  pravdala akciju specijalne jedinice, da se poslo od moguceg
  posedovanja oruzja -- pri tom se radilo o obicnom svajcarskom
  sanitetskom pistolju bez municije. Prebacivanje lekara u psihijatriju
  je pravdano navodno time, sto nije bio u stanju da ostane u normalnom
  zatvoru -- i to je moguce uzeti kao paravan. Po onome sto se dosada
  zna, moguce je poci od jedne politicki motivisane psihijatrizacije.
  Slicne onima, koje su u Svajcarskoj do cak 1980-tih godina bile
  upraznjavane i imale desetine godina dugu i mracnu tradiciju.
\item
  Na ovaj svajcarski slucaj, koji podseca na praktike nekadasnje
  Sovjetske drzave, internacionalnu paznju je skrenula americka polanica
  u Kongresu, Cynthia McKinney.
\item
  Italija koristi
  \href{https://www.ansa.it/english/news/2020/04/06/coronavirus-italy-activates-satellite-to-monitor-nation-3_f2ffb30c-d550-42f5-82fc-ec1f82c5c625.html}{podatke
  evropskih satelita}, kako bi observirala gradjane.
\item
  Britanska policija je
  \href{https://mail.protonmail.com/\#CoronaPolice}{provalila vrata
  jednog privatnog stana}, da bi izvrsila kontrolu.
\end{itemize}

\hypertarget{12-april-2020}{%
\paragraph{12. april 2020.}\label{12-april-2020}}

\hypertarget{nove-studije}{%
\subparagraph{\texorpdfstring{\textbf{Nove
studije}}{Nove studije}}\label{nove-studije}}

\begin{itemize}
\tightlist
\item
  Profesor medicine na Stanford-u, John Ioannidis, dolazi u
  \href{https://www.medrxiv.org/content/10.1101/2020.04.05.20054361v1}{jednoj
  novoj studiji} do zakljucka, da smrtni rizik usled Covid-19 kod ljudi
  mladjih od 65 godina, cak i u globalnim „Hotspot`` -- oblastima,
  odgovara riziku smrti usled jedne automobilske nesrece kod ljudi koji
  svakodnevno idu autom na posao.
\item
  Nemacki virolog Hendrick Streek dolazi u jednoj
  \href{https://www.land.nrw/sites/default/files/asset/document/zwischenergebnis_covid19_case_study_gangelt_0.pdf}{seroloskoj
  studiji} do privremenog zakljucka, da je letalitet kod Covid-19~ oko
  0,37\%, a smrtnost (racunajuci u odnosu na celokupnu populaciju) oko
  0,06\%. Ove vrednosti su oko deset puta nize od onih navedenih od
  strane WHO i oko pet puta nize od navedenih od strane John Hopkins
  Univerziteta. Ova studija je doduse vec
  \href{https://www.sueddeutsche.de/wissen/heinsberg-studie-herdenimmunitaet-kritik-1.4873480}{kritikovana}
  od strane drugih virologa.
\item
  Jedna danska studija sa 1500 davaoca krvi dolazi do zakljucka, da je
  letalitet od Covid-19
  \href{https://www.dr.dk/nyheder/indland/doedelighed-skal-formentlig-taelles-i-promiller-danske-blodproever-kaster-nyt-lys}{samo
  1,6 promila}, to znaci preko 20 puta manja nego sto je to WHO
  prvobitno pretpostavila i na taj nacin se nalazi na nivou jedne jake
  (pandemske) influence. Istovremeno je Danska
  \href{https://www.thelocal.dk/20200406/denmark-to-reopen-schools-and-kindergartens-next-week}{donela
  odluku} o otvaranju skola i obdanista sledece nedelje.
\item
  Jedna seroloska studija u US drzavi Colorado je dosla do
  \href{https://reason.com/2020/04/08/mass-antibody-testing-in-this-rural-colorado-county-sheds-light-on-covid-19s-prevalence-and-lethality/}{privremenog
  rezultata,} da je letalitet kod Covid-19 5 do 20 puta bila precenjena~
  i da lezi u opsegu izmedju jednog normalnog i pandemskog gripa.
\item
  Jedno istrazivanje Medicinskog Univerziteta u Becu, dolazi do
  \href{https://www.vienna.at/analyse-zeigt-covid-19-opferkurve-entspricht-normaler-mortalitaet/6581246}{zakljucka},
  da starost i profil rizika kod umrlih od Covid-19 odgovara normalnoj
  smrtnosti.
\item
  Jedna studija u Journal of Medicin Virology dolazi do
  \href{https://www.ncbi.nlm.nih.gov/pubmed/32219885}{zakljucka}, da je
  test za Corona-virus, koji se primenjuje u mnogim zemljama,
  nestabilan: uz vec poznati problem pogresnih pozitivnih rezultata
  postoji jedna moguce visoka stopa pogresnih negativnih rezultata, to
  znaci, test ne reaguje cak ni kod osoba sa simptomima, dok kod drugih
  pacijenata jednom pokazuje infekciju~ i posle opet ne. To otezava
  mogucnost razlikovanja od nekih drugih, gripu slicnih, bolesti.
\item
  Jedan svajcarski biofizicar je po prvi put analizirao pozitivnu stopu
  zarazenosti u USA, Nemackoj i Svajcarskoj i nju
  \href{https://mail.protonmail.com/\#SARSCoV2}{graficki prikazao}. Iz
  grafike se vidi, da pozitivna stopa rasta u ovim zemljama samo lagano
  a ne eksponencijalno raste.
\item
  Istrazivaci iz USA dolaze do zakljucka,da lokalno zagadjenje vazduha
  \href{https://www.heise.de/tp/features/Luftverschmutzung-erhoeht-Covid-19-Sterberisiko-4699306.html}{jako
  povecava} rizik od smrti usled Covid-19. To potvrdjuju ranija
  istrazivanja iz Italije i Kine.
\item
  WHO je krajem marta dosla do
  \href{https://www.who.int/news-room/commentaries/detail/modes-of-transmission-of-virus-causing-covid-19-implications-for-ipc-precaution-recommendations}{zakljucka},
  da se Covid-19, nasuprot ranijim pretpostavkama, ne prenosi putem
  aerosola („kroz vazduh``). Prenos infekcije se pre svega vrsi putem
  direktnog kontakta ili infekcijom putem „kapljica`` (kasljanjem i
  kijanjem).
\item
  Nemacko-americki epidemiolog, profesor Knut Wittkowski, u jednom
  \href{https://www.youtube.com/watch?v=ARTf4bpiXuI}{intervjuu}
  pretpostavlja, da epidemija Covid-19 u mnogim zemljama vec slabi, ~ili
  je cak „vec prosla``. Zabrane izlaska su dosle prekasno i bile
  kontraproduktivne.
\end{itemize}

\hypertarget{evropski-monitoring-mortaliteta}{%
\subparagraph{\texorpdfstring{\textbf{Evropski monitoring
mortaliteta}}{Evropski monitoring mortaliteta}}\label{evropski-monitoring-mortaliteta}}

\href{https://www.euromomo.eu/outputs/zscore_country_total.html}{Evropski
monitoring mortaliteta} pokazuje u medjuvremenu u vise evropskih zemalja
jasnu prognoziranu prteranu smrtnost u starosnoj grupi preko 65 godina.
U drugim zemljama, kao sto su Nemacka i Austrija, je smrtnost i u toj
starosnoj grupi u normali (ili cak ispod nje).

Otvoreno je pitanje, da li se delimicno povisena smrtnost iskljucivo
svodi na Corona-virus, ili takodje i na drasticne mere (izolacija,
stres, otkazane operacije, itd.) i da li ce smrtnost i u godisnjoj slici
biti povisena.

Za starosne grupe mlađe od 65 godina, u Engleskoj je postojao samo
(predviđeni) porast smrtnosti koji nadilazi prethodne talase gripa.
Srednja dob test-pozitivnog pokojnika je 80 u Italiji, 83 u Njemačkoj i
84 u Švicarskoj.

\hypertarget{svajcarska}{%
\subparagraph{\texorpdfstring{\textbf{Svajcarska}}{Svajcarska}}\label{svajcarska}}

\begin{itemize}
\tightlist
\item
  Po
  \href{https://www.bag.admin.ch/bag/de/home/krankheiten/ausbrueche-epidemien-pandemien/aktuelle-ausbrueche-epidemien/novel-cov/situation-schweiz-und-international.html}{najnovijem
  izvestaju} BAG-a je prosecna starost pozitivno testiranih i umrlih sad
  vec 84 godine.
\item
  Jedna studija ETH iz Zürich-a,
  \href{https://www.tagesanzeiger.ch/ansteckungsraten-flachten-bereits-vor-dem-lockdown-ab-809893127675}{dolazi
  do zakljucka}, da je stopa zaraze vec mnogo dana pre zabrane izlaska
  pala na vrednost ispod 1, verovatno radi opstih higijenskih i
  svakodnevnih mera. Ukoliko je rezultat korektan, postavlja se pitane
  opravdanosti ovih zabrana
  \href{https://bsse.ethz.ch/cevo/research/sars-cov-2/real-time-monitoring-in-switzerland.html}{(ETH
  -- studija})
\item
  Svajcarski magazin Infosperber kritijuje informativnu politiku vlasti
  i medija:
  \href{https://www.infosperber.ch/Artikel/Gesundheit/Corona-Statt-zu-informieren-fuhren-Behorden-eine-PR-Kampagne}{„Umesto
  da informisu, vlasti vode jednu PR-kampanju``}. Sa obmanjujucim
  brojevima i grafikama siri se jedan,dobrim delom neopravdani strah.
\item
  Takodje i svajcarski casopis zastite potrosaca Ktipp kritikuje
  politiku informisanja i medijskog izvestavanja:
  \href{https://www.ktipp.ch/artikel/artikeldetail/behoerden-informieren-irrefuehrend/}{„Vlasti
  izvestavaju manipulativno``}
\item
  Jedan svajcarski istrazivac je analizirao najnoviji izvastaj saveznog
  zavoda za zdravstvo o Covid-19 i dolazi do
  \href{https://covid-19-fakten.blogspot.com/2020/04/die-analyse-des-aktuellen.html}{jako
  kriticnog zakljucka}: Izvestaj je ~„naucno neujednacen, tendencijalno
  sugestivan i obmanjuje (ili barem zbunjuje)``. Uvedene mere su u
  svetlu cinjenica „neodgovorne i sire strah``.
\item
  Svajcarski lekari govore u jednom
  \href{https://www.rontalpraxis.ch/aktuelles}{otvorenom pismu upucenom
  svajcarskom ministru zdravlja} o „nepodudaranju izmedju pretecih
  scenarija, koje pre svega ~raspiruju meidiji i nase realne
  situacije``. U obicnoj populaciji ima malo slucajeva Covid-19 i
  najcesce su blagog intenziteta, a nasuprot tome je sve je vise
  strahova i panicnih reakcija u narodu pri cemu ~veliki broj pacijenata
  radi straha ne dolazi ~i na vazne lekarske preglede i ispitivanja. „I
  sve to u vezi sa jednim virusom, cija se opasnost ~po nasem opazanju u
  centralnoj svajcarskoj, nalazi samo u medijima i nasim glavama.``
\item
  Radi jako smanjenog broja pacijenata su mnoge klinike u
  \href{https://www.20min.ch/schweiz/news/story/Spitaeler-28949526}{Svajcarskoj}
  i
  \href{https://www.spiegel.de/wirtschaft/unternehmen/trotz-corona-pandemie-warum-kliniken-jetzt-kurzarbeit-anmelden-a-3dc61bc9-fb12-4298-8022-bb4c2be39d7d}{Nemackoj}
  morale da u medjuvremenu najave skraceno radno vreme. Broj pacijenata
  je smanjen i do 80\%.
\item
  Dr.Daniel Jeanmonod, svajcarski profesor fiziologije i neurohirurgije
  u penziji, savetuje u jednoj
  analizi\href{https://off-guardian.org/2020/04/07/think-deep-do-good-science-and-do-not-panic/}{:
  „Misli trezveno, radi na naucnoj osnovi, i ne panici!``}
\item
  Svajcarski medicinar, profesor Dr. Paul Robert Vogt je napravio prilog
  o Covid-19, koji je izazvao veliku paznju. On
  \href{https://www.mittellaendische.ch/2020/04/07/covid-19-eine-zwischenbilanz-oder-eine-analyse-der-moral-der-medizinischen-fakten-sowie-der-aktuellen-und-zukünftigen-politischen-entscheidungen/}{kritikuje}
  senzacionalisticku stampu, ali i upozorava, da se ne radi o „normalnom
  gripu``. U nekim tackama ovaj lekar gresi: letalitet i prosek
  starosti, kao i razlika, izmedju sa i usled Corona-virusa su ipak od
  sustinske vaznosti, zastitne maske i aparati za vestacko disanje u
  mnogim slucajevima nepodobni (vidi dole), a zabrane izlaska sumnjiva i
  moguce kontraproduktivna mera.
\end{itemize}

\hypertarget{nemacka-i-austrija}{%
\subparagraph{\texorpdfstring{\textbf{Nemacka i
Austrija}}{Nemacka i Austrija}}\label{nemacka-i-austrija}}

\begin{itemize}
\tightlist
\item
  Nemacki zdravstveni eksperti
  \href{https://www.tagesschau.de/investigativ/ndr-wdr/corona-experten-thesenpapier-101.html}{kritikuju}
  kriznu politiku savezne vlade. Oni govore o dugorocnim posledicama po
  narod, koje u jednoj meri prouzrokuje Shutdown. „Podaci koje
  objavljuje RKI nemaju dovoljnu tezinu``.
\item
  Nemacki savez~patologa
  \href{https://www.pathologie-dgp.de/die-dgp/aktuelles/meldung/pressemitteilung-an-corona-verstorbene-sollten-obduziert-werden/}{zahteva
  u u jednom saopstenju}, da Corona-slucajevi moraju da budu obducirani,
  (kako bi se utvrdio stvarni razlog smrti) i tako eksplicitno
  protivreci „preporukama Robert-Koch-Instituta``, koji se izjasnio
  protiv obdukcija, navodno iz razloga prevelike opasnosti.
\item
  Dr. Martin Sprenger je dao
  \href{https://mailchi.mp/addendum/fles-home-office-260342}{ostavku} na
  svoje clanstvo u Savetu za Coronu austrijskog zdravstvenog
  ministarstva, da bi „ponovo dosao do svoje gradjanske i naucne slobode
  misljenja``.
\item
  Dr. Sprenger je prethodno pored ostalog kritikovao, da vlada
  nedovoljno razlikuje razlicite stepene rizika Corona-virusa kod
  razlicitih starosnih grupa i da donosi
  \href{https://www.addendum.org/coronavirus/interview-sprenger/}{pausalne
  odluke}: „Mi moramo paziti, da gubitak zdravih zivotnih godina usled
  nedovoljnog zbrinjavanja drugih akutnih i hronicnih bolesti ne bude 10
  puta veci od gubitka zdravih zivotnih godina uzrokovanog od strane
  Covid-19.`` Corona -virus je opasan pre svega za starije, ljude u
  godinama``, navodi Sprengler.
\item
  U jednom nemackom starackom domu je je jedan stanar od 84 godina
  starosti bio pozitivno testiran na Covid-19, radi cega je cela
  institucija bila stavljena u karantin i izvrseno je masovno
  testiranje. Posle se ispostavilo, da je prvi test bio
  \href{https://www.schwerin.de/news/4a3e5560-78c9-11ea-b543-1967de695b51/}{pogresan}.
\end{itemize}

\hypertarget{skandinavija}{%
\subparagraph{\texorpdfstring{\textbf{Skandinavija}}{Skandinavija}}\label{skandinavija}}

\begin{itemize}
\tightlist
\item
  Norvesko udruzenje lekara pise u otvorenom pismu ministru zdravstva,
  kako postoji bojazan, da bi ~uvedene mere mogle biti
  \href{https://www.abcnyheter.no/helse-og-livsstil/helse/2020/04/06/195667780/nesten-halvparten-av-sengene-pa-oslo-universitetssykehus-star-tomme}{opasnije
  od virusa}, posto se normalni pacijenti vise ne pregledaju niti lece.
\item
  Jedan svedski publicista
  \href{https://www.spectator.co.uk/article/no-lockdown-please-w-re-swedish}{objasnjava
  u britanskom Spectator-}u: „Nije Svedska ta koja vrsi eksperiment, vec
  to vrse ostale zemlje.``
\item
  Profesor Ansgar Lohse, direktor na Hamburskoj Univerzitetskoj Klinici,
  \href{https://www.abendblatt.de/hamburg/article228880917/uke-professor-shutdown-lohse-deutschland-hamburg-corona-virus-infektion-covid-19-impfstoff-coronavirus-krise-patienten-immunitaet-krankenhaeuser-kontaktverbot-kliniken-infektionsrate.html}{objasnjava
  u jednom intervjuu}: „Svedske mere su po meni najracionalnije u celom
  svetu. Naravno da se postavlja pitanje, da li je to moguce izdrzati
  psiholoski. U pocetku moraju Svedjani racunati sa znatno vise smrtnih
  slucajeva, koji se onda srednje -- do dugorocno znacajno smanjuju.
  Preracunace se posle godinu dana -- ukoliko Svedjani to izdraze. Strah
  od virusa primorava politicare nazalost na poteze, koji nisu obavezno
  razumni. Politika je takodje gonjena slikama iz medija.
\item
  Po svedskom epidemiologu Anders-u Tegnell-u je Stockholm sto se tice
  Covid-19, dostigao „plato``.
  (\href{https://www.thelocal.se/20200310/timeline-how-the-coronavirus-has-developed-in-sweden}{Dodatne
  vesti o Svedskoj})
\end{itemize}

\hypertarget{usa-i-azija}{%
\subparagraph{\texorpdfstring{\textbf{USA i
Azija}}{USA i Azija}}\label{usa-i-azija}}

\begin{itemize}
\tightlist
\item
  U USA preporucuju vlasti takodje, da se svi pozitivno testirani smrtni
  slucajevi, cak i slucajevi za koje se samo sumnja, pa makar bili i bez
  pozitivnog rezultata,
  \href{https://nypost.com/2020/04/07/feds-classify-all-coronavirus-patient-deaths-as-covid-19-deaths/?link=TD_mansionglobal_new_mansion_global.11147f181987fd93}{registruju}
  kao „Corona-smrtni slucajevi``. Jedan americki lekar i drzavni senator
  iz Minessote je
  \href{https://nypost.com/2020/04/07/feds-classify-all-coronavirus-patient-deaths-as-covid-19-deaths/?link=TD_mansionglobal_new_mansion_global.11147f181987fd93}{izjavio},
  da je to identicno manipulaciji. Pored toga postoje financijske
  pogodnosti za bolnice, ukoliko pacijente deklarisu kao
  Covid-pacijente. (Malo
  \href{https://swprs.files.wordpress.com/2020/04/cv-2019-2020.jpg}{humora}
  u ovoj tematici)
\item
  Jedna poljska bolnica za Covid-19 pacijente kod Seatle u saveznoj
  drzavi Washington je vec posle tri dana bila
  \href{https://www.yahoo.com/news/armys-seattle-field-hospital-closes-165646379.html}{zatvorena},
  bez da su neki pacijenti bili primljeni. To podseca na brzo izgradjene
  bolnice kod Wuhan-a, koje su isto tako bile samo minimalno opterecene
  ili su cak ostale prazne, te posle kratkog vremena bile demontirane.
\item
  Mnogobrojni mediji izvestavaju o navodnim masovnim
  „Corona-grobnicama`` na ostrvu Hart Island kod New York-a. Te vesti su
  na dvostruki nacin obmanjujuce: prvo je ostrvo Hart Island vec dugo
  vremena jedno od najpoznatijih
  \href{https://en.wikipedia.org/wiki/Hart_Island_(Bronx)\#Cemetery}{grobalja
  za sirotinju} u Sjedinjenim Drzavama, i drugo, da je gradonacelnik New
  York-a
  \href{https://www.independent.co.uk/news/world/americas/new-york-coronavirus-cases-burials-bodies-covid-19-hart-island-a9459956.html}{izjavio},
  da nikakve masovne grobnice nisu planirane, vec da na Hart Island-u
  treba da budu sahranjeni neidentifikovani umrli (znaci oni bez
  rodjaka).
\item
  Jedan od vodecih indijskih epidemiologa objasnjava,
  \href{https://www.business-standard.com/article/current-affairs/we-cannot-run-away-to-the-moon-need-to-develop-herd-immunity-dr-muliyil-120040601232_1.html}{„Mi
  ne mozemo da pobegnemo na mesec``} i predlaze brzi razvoj jednog
  prirodnog imuniteta u populaciji.
\end{itemize}

\hypertarget{severna-italija}{%
\subparagraph{\texorpdfstring{\textbf{Severna
Italija}}{Severna Italija}}\label{severna-italija}}

Tacno je da da su u Lombardiji u mesecima neposredno pre izbijanja
Covid-19 sprovedene ~dve opsezne vakcinacije protiv
\href{https://www.bergamonews.it/2019/10/21/vaccinazione-antinfluenzale-a-bergamo-ordinate-185-000-dosi-di-vaccino/332164/}{influence}
i
\href{https://www.bsnews.it/2020/01/18/meningite-vaccinate-34mila-persone-tra-brescia-e-bergamo/}{meningokoka},
posebno u kasnijim centrima zaraze Bergamu i Bresci. Teoretski je
moguce, da ~ovakve vakcinacije mogu da dodju u uzajamno dejstvo sa
Corona-infekcijama, ali ovakav ~uzajamni uticaj nije medicinski
potvrdjen.

Isto tako je tacno, da je u severnoj Italiji postojala visoka
kontaminacija
\href{https://www.spiegel.de/panorama/justiz/asbest-prozess-in-italien-nun-sind-alle-krank-a-666421.html}{azbestom},
koja povecava rizik za kasnija kancerogena plucna oboljenja. Ni ovde
nije moguce poci od direktne veze sa Covid-19.

Generalno je ipak tacno, da je negativni uticaj velikog
\href{https://www.heise.de/tp/features/Feinstaubpartikel-als-Viren-Vehikel-4687454.html}{zagadjenja
vazduha} i nekih drugih faktora na zdravlje ljudi u severnoj Italiji vec
dugo prisutan, ~te da je stoga populacija
\href{https://www.srf.ch/news/international/massive-schadstoffbelastung-nirgendwo-erkranken-so-viele-wegen-smog-wie-in-norditalien}{posebno
osetljiva} na oboljenja disajnih organa.

\includegraphics{https://swprs.files.wordpress.com/2020/03/italy-smog.png?w=500\&h=281}

\hypertarget{nacelnik-bolnice-lekar-pietro-vernazza}{%
\subparagraph{\texorpdfstring{\textbf{Nacelnik bolnice, lekar Pietro
Vernazza}}{Nacelnik bolnice, lekar Pietro Vernazza}}\label{nacelnik-bolnice-lekar-pietro-vernazza}}

Svajcarski lekar, infektolog, Pietro Vernazza, objavio je cetiri nova
clanka u vezi studija vezanih za Covid-19

\begin{itemize}
\tightlist
\item
  U
  \href{https://infekt.ch/2020/04/schulen-schliessen-hilfreich-oder-nicht/}{prvom
  clanku} se radi o tome, da efekat zatvaranja skola sa medicinskog
  stanovista nije evidentan (niti je ikad bio), posto deca uglavnom
  ~niti obolevaju ozbiljno od virusa, niti~ prenose virus (za razliku od
  influence)
\item
  U
  \href{https://infekt.ch/2020/04/atemschutzmasken-fuer-alle-medienhype-oder-unverzichtbar/}{drugom
  clanku} se radi o tome, da zastitne maske nemaju nikakav dokazani
  ucinak, sa jednim izuzetkom: obolele osobe sa simptomima (to znaci
  posebno kasalj) mogu time da smanje sirenje virusa. Inace su maske pre
  simbol, tj. jedan „medijski hype``
\item
  U
  \href{https://infekt.ch/2020/03/immunschwaeche-und-schwangerschaft-kein-covid-19-risikofaktor/}{trecem
  clanku} se radi o rizicnim grupama Covid-19. U njih spadaju po
  dosadasnjim saznanjima ljudi sa povisenim krvnim pritiskom --
  pretpostavlja se da Covid-19 virus koristi celijske receptore, koji su
  zaduzeni i za regulaciju krvnog pritiska. U rizicnu grupu, sto je
  iznenadjujuce, ne spadaju~ ljudi sa oslabljenim imunim sistemom kao i
  trudnice (koje prirodno imaju jedan snizeni imuni sistem). Rizik kod
  Covid-19 je nasuprot tome cesto jedna preterana reakcija imnunog
  sistema.
\item
  U
  \href{https://infekt.ch/2020/04/corona-testen-testen-und-kein-ende/}{cetvrtom
  clanku} se radi o masovnom testiranju. Zakljucak profesora Vernazza-e:
  „Ko ima simptome nekog oboljenja disajnih organa, ostaje kod kuce. To
  vazi tacno tako i za grip. Neko testiranje ne donosi nikakvu dodatnu
  korist.
\end{itemize}

\hypertarget{intenzivna--vs-palliativna-medicina}{%
\subparagraph{\texorpdfstring{\textbf{Intenzivna- vs. Palliativna
medicina}}{Intenzivna- vs. Palliativna medicina}}\label{intenzivna--vs-palliativna-medicina}}

Jedan nemacki lekar paliativnog smera objasnjava
\href{https://www.ruhr24.de/ruhrgebiet/coronavirus-behandlung-intensivstation-nrw-lungenentzuendung-matthias-thoens-witten-zr-13645038.html}{u
jednom intervjuu}, da Covid-19 „nije nikakva bolest intenzivne
medicine``, posto se kod tesko pogodjenih ljudi tipicno radi o ljudima u
visokoj starosti sa vecim brojem raznih oboljenja. Ako ti ljudi dobiju
upalu pluca, bivaju „kao i uvek do sada paliativno tretirani
(zbrinjavanje pred smrt)``. Sa dodatnom Covid-19 dijagnozom od njih se
pravi jedan slucaj za odeljenje intenzivne nege, ali i pored toga
naravno pacijenta nije moguce spasiti.

Aktuelnu delatnost mnogih koji odlucuju, ocenjuje lekar kao „panicni
modus``. Trenutno su u Nemackoj kreveti na intenzivnoj nezi jos
relativno prazni. Aparati za vestacko disanje su slobodni. Iz ekonomskih
razloga bi mogli direktori bolnica da dodju na ideju, da prime starije
osobe. „Imacemo za 14 dana situaciju, sa punim odeljenjima i starcima sa
mnogo bolesti koje je nemoguce spasiti. A kad su oni na aparatima,
postavlja se pitanje, ko ce onda da ich iskljuci. Pa ~to je onda
ubistvo.`` Preti nam jedna eticka katastrofa`` radi gramzivosti za
novcem, smatra medicinar.

\hypertarget{vestacko-disanje-kod-covid-19-1}{%
\subparagraph{\texorpdfstring{\textbf{Vestacko disanje kod
Covid-19}}{Vestacko disanje kod Covid-19}}\label{vestacko-disanje-kod-covid-19-1}}

Sirom sveta postoji juris na aparate za disanje za Covid-19 pacijente.
Ova stranica je bila jedna od prvih koja ja skrenula paznju, da je
invazivna intubacija u puno slucajeva kontraproduktivna i da dodatno
steti pacijentu.

Invazivno vestacko disanje je prvobitno bilo preporucivano, jer se radi
niskog nivoa kiseonika pogresno zakljucivalo da se radi o akutnom
otkazivanju pluca, i zato sto je postojao strah, da bi se kod jedne
blaze neinvazivne varijante virus mogao prosiriti putem aerosola.

U medjuvremenu su se javili za rec mnogi vodeci pneumolozi i lekari
intenzivne medicine iz USA i Evrope, koji odvracaju od jednog invazivnog
vestackog disanja i preporucuju blaze metode, tj. terapiju kiseonikom,
koja je vec uspesno bila primenjivana u Juznoj Koreji.

\begin{itemize}
\tightlist
\item
  \href{https://time.com/5818547/ventilators-coronavirus/}{Why Some
  Doctors Are Now Moving Away From Ventilator Treatments} (TIME)
\item
  \href{https://www.spectator.co.uk/article/Ventilators-aren-t-a-panacea-for-a-pandemic-like-coronavirus}{Ventilators
  aren't a panacea for a pandemic like coronavirus} (Dr. Matt Strauss)
\item
  \href{https://www.statnews.com/2020/04/08/doctors-say-ventilators-overused-for-covid-19/}{With
  ventilators running out, doctors say the machines are overused for
  Covid-19} (SN)
\item
  \href{https://www.atsjournals.org/doi/pdf/10.1164/rccm.202003-0817LE}{Covid-19
  Does Not Lead to a ``Typical'' Acute Respiratory Distress Syndrome}
  (ATSJ)
\item
  \href{https://www.medscape.com/viewarticle/928156}{Do COVID-19
  Ventilator Protocols Need a Second Look?} (Medscape)
\item
  DE: \href{https://archive.is/KX5IQ}{„Previse cesto se sprovodi
  intubacija``} (Dr. Thomas Voshaar, FAZ)
\item
  DE:
  \href{https://www.doccheck.com/de/detail/articles/26271-covid-19-beatmung-und-dann}{COVID-19:
  Beatmung -- und dann?} (DocCheck)
\end{itemize}

\hypertarget{razvoj-politicke-situacije}{%
\subparagraph{\texorpdfstring{\textbf{Razvoj politicke
situacije}}{Razvoj politicke situacije}}\label{razvoj-politicke-situacije}}

\begin{itemize}
\tightlist
\item
  Bivsi NSA-agent, Edward Snowden, upozorava u jednom novom intervjuu,
  da vlade koriste Corona-virus, kako bi sagradile
  \href{https://www.vice.com/en_us/article/bvge5q/snowden-warns-governments-are-using-coronavirus-to-build-the-architecture-of-oppression}{„arhitekturu
  potlacivanja``}.
\item
  Apple i Google su najavili, da ce ~u saradnji sa nacionalnim vlastima,
  u svoje mobilne sisteme ugraditi takozvane
  \href{https://www.bloomberg.com/news/articles/2020-04-10/apple-google-bring-covid-19-contact-tracing-to-3-billion-people}{„Contact
  Tracing``}, sa kojima je moguce nadgledati kontakte unutar naroda.
\item
  Nemacki pravnik za ustavno pravo, Uwe Volkmann,
  \href{https://www.youtube.com/watch?v=DvzrGLvzllU}{izjavio je u ARD
  Monitor}-u, da ne poznaje ni jednog od svojih kolega, koji za uvedene
  mere radi Corone smatra, da su u skladu sa ustavom.
\item
  Italijanska vlada je obrazovala „Tasc Force``, ciji je zadatak, da
  „netacne vesti`` o Covid-19
  \href{https://www.faz.net/aktuell/feuilleton/medien/corona-in-italien-das-virus-und-die-wahrheit-16714529.html}{„odstrani``}
  iz interneta. Pravo slobode misljenja „nije time uskraceno``
\item
  Francuska je dozvoljeni istrazni zatvor produzila i kontrolu od strane
  jednog sudije
  \href{https://www.lefigaro.fr/politique/coronavirus-le-conseil-d-etat-sur-la-ligne-de-crete-des-libertes-publiques-20200406}{ukinula}.
  Zalbe od strane advokatskih udruzenja se odbijaju.
\item
  Danska je uvela pocetkom aprila
  \href{https://www.fr.de/politik/coronavirus-sars-cov-2-daenemark-notfalls-militaer-13598503.html}{„besprimerno
  ostre vanredne zakone``}: „Zdravstvene vlasti mogu od ovog trenutka da
  narede sprovodjenje testiranja, vakcinacije i medicinskog tretmana
  putem sile i za sprovodjenje tih naredbi angazuju pored policije,
  takodje i vojsku kao i privatne sluzbe.
\item
  Policija u nemackoj pokrajini Nordrhein-Westfalen
  \href{https://rp-online.de/nrw/panorama/nrw-polizei-testet-drohnen-bei-einsaetzen-wegen-corona-massnahmen_aid-50006143}{testira
  drone} u potraznji za zabranjenim grupisanjima gradjana.
\item
  Nemacka pokrajina Sachsen zeli da one, koji ne postuju pravilo
  karantina, zatvori na
  \href{https://www.welt.de/politik/deutschland/article207198029/Coronavirus-Sachsen-will-Quarantaene-Verweigerer-in-Psychiatrien-sperren.html}{psihijatrijsko
  odeljenje}.
\item
  Jedan svajcarski lekar je bio uhapsen i
  \href{https://www.blick.ch/news/schweiz/mittelland/in-baden-ag-polizei-in-vollmontur-im-einsatz-id15841510.html}{odveden
  na psihijatriju}, jer je kritikovao Corona-mere i navodno pretio
  vlastima.
\item
  U Nemackoj je jedna pravnica za medicinsko pravo ulozila zalbu radi
  nepostovanja ustava u vezi sa uvedenim merama i objavila otvoreno
  pismo, u kome upozorava na opasnost od sunovrata u policijsku drzavu i
  izmedju ostalog poziva na prijavljivanje organizovanih demonstracija
  protiv ovakvog stanja. Na to su javno tuzilastvo i policija pokrenuli
  \href{https://www.morgenweb.de/mannheimer-morgen_artikel,-coronavirus-aufruf-zu-straftaten-ermittlungen-gegen-heidelberger-rechtsanwaeltin-_arid,1627078.html}{istragu}
  protiv pravnice radi „poziva na kaznjivo delo``; internet stranica
  pravnice je neko vreme bila deaktivirana. U medjuvremenu je njena
  zalba odbijena.
\item
  I u Austriji je u medjuvremenu vise advokata
  \href{https://wien.orf.at/stories/3043172/}{ulozilo} zalbu kod
  ustavnog suda protiv uvedenih mera. Osnovna prava ~su ovim merama
  povredjena, argumentuju pravnici.
\item
  Gradonacelnik Los Angeles-a je
  \href{https://townhall.com/tipsheet/bethbaumann/2020/04/04/la-mayor-garcetti-says-snitches-get-rewards-for-ratting-out-their-neighbors-n2566348}{obecao
  nagradu} za cinkarose ~(snitches), koji prijave svoje susede vlastima,
  ukoliko se ovi ne pridrzavaju zabrane izlaska.
\item
  U USA je radi Lockdoowns vec preko 16 miliona ljudi
  \href{https://www.nytimes.com/2020/04/09/us/coronavirus-us-news.html}{bez
  posla}, to odgovara oko 10\% radno sposobnog stanovnistva. Po
  internacionalnoj agenturi ILO je trenutno 80\% od 3,3 milijarde radnog
  sveta pogodjeno uvedenim merama. 1,25 milijardi radnog sveta moze biti
  \href{https://www.ilo.org/global/about-the-ilo/newsroom/news/WCMS_740893/lang--en/index.htm}{pogodjeno}
  drasticnim ili katastrofalnim posledicama.
\end{itemize}

\hypertarget{7-april-2020}{%
\paragraph{7. April 2020.}\label{7-april-2020}}

\href{https://multipolar-magazin.de/artikel/coronavirus-regierung-ignoriert-daten}{Najnovij
podaci iz jednog specijalnog izvestaja} nemackog Robert-Koch-Instituta
pokazuju da takozvana pozitivna rata (znaci broj pozitivnih rezultata u
odnosu na broj testova) znatno sporije raste nego sto je to u medijima
pokazivala ~eksponencijalna funkcija i krajem marta je iznosila oko
10\%, sto je ~tipicna vrednost za viruse iz Corona-grupe. Znaci „ne moze
biti ~govora o jednom opasno brzom sirenju virusa`` ~naglasava magazin
Multipolar.

\textbf{Profesor Klaus Püschel, sef zavoda za sudsku medicinu u Hamburgu
objasnjava:}

\href{https://www.pressreader.com/germany/hamburger-morgenpost/20200403/281487868456736}{Covid-19}:
„Ovaj virus utice na nas zivot~ na potpuno preteran nacin. I to ne stoji
ni u kakvom odnosu prema opasnosti koja nam od njega preti. Astronomska
ekonomska steta koja sada nastaje isto tako nije primerena toj
opasnosti. Ja sam ubedjen, da smrtnost od Corone nece biti evidentna ni
kao vrh u godisnjoj smrtnosti.

Tako na primer u Hamburgu jos nije ni jedan pacijent od virusa umro, a
da vec nije imao neku bolest: „Svi koje smo do sada pregledali su imali
rak, neku hronicnu bolest pluca, bili su izraziti pusaci, imali ozbiljne
probleme sa jetrom, patili od dijabetisa ili imali neku bolest
kardiovaskularnog sistema.`` Tu je virus bio takoreci ona poslednja kap
gde se casa prelila. „Covid-19 je samo u izuzetnim slucajevima
smrtonosna bolest, i u najvecem broju slucajeva jedna pretezno bezazlena
virusna infekcija.``

\href{https://www.abendblatt.de/hamburg/article228828787/rechtsmedizin-pueschel-hamburg-corona-virus-infektion-covid-19-coronavirus-krise-patienten-krankenhaeuser-kliniken-infektionsrate-krankheit-pandemie-test-lungenkrankheit-sars-cov-epidemie-sars-cov-2.html}{Uz
to objasnjava Dr.Püschel}: „U ne malom broju slucajeva smo utvrdili da
aktuelna infekcija Corona-virusom sa smrtnim ishodom uopste nista nema,
jer su prisutni drugi uzroci, na primer izliv krvi u mozak ili
infarkt.`` Corona je po sebi jedno ne posebno opsno virusno oboljenje,
izjavljuje medicinar. On pledira za jednu statistiku koja je bazirana na
konkretnim nalazima. „Sve pretpostavke o smrtnim slucajevima, koje nisu
proverene od strane kompetentnih, samo seju strah.``

Slobodni Hanse-grad Hamburg je pre nekog vremena poceo da, nasuprot
uputstvima Robert-Koch-Instituta, razlikuje izmedju smrtnih slucajeva
„sa`` i „usled`` Corona-virusa, sto je dovelo do
\href{https://www.t-online.de/nachrichten/deutschland/id_87636856/coronavirus-hamburg-will-nur-echte-covid-19-tote-zaehlen.html}{pada}
smrtnosti Covid-19 oboljenja.

Nemacki virolog Hendrik Streek vodi trenutno studiju sa ciljem
utvgrdjivanja prosirenosti i nacina prenosa uzrocnika Covid-19. U
jedenom
\href{https://www.zeit.de/zustimmung?url=https\%3A\%2F\%2Fwww.zeit.de\%2Fwissen\%2Fgesundheit\%2F2020-04\%2Fhendrik-streeck-covid-19-heinsberg-symptome-infektionsschutz-massnahmen-studie\%2Fkomplettansicht}{itervjuu}
on objasnjava:

„Tacnije sam proucio 31 smrtni slucaj od 40 umrlih u okrugu Heinsberg --
i nisam bio iznenadjen, da su ti ljudi umrli. Jedan umrli je imao 100
godina, u tom slucaju je i jedna obicna kijavica mogla da izazove smrt.
Nasuprot dosadasnjim pretpostavkama, on nije utvrdio mogucnost
prenosenja zaraze preko kvake od vrata i sl.

U Svajcarskoj se javljaju prve bolnice koje
\href{https://www.engadinerpost.ch/2020/4/04/Engadiner-Spitaeler-haben-freie-Kapazitaeten}{najavljuju
sraceno radno vreme}:

„Na svim odeljenjima osoblje ima premalo posla i u prvom koraku je
prekovremeni rad kompenzovan. Sad je najavljeno skraceno radno vreme.
Finansijske posledice su velike.`` Da podsetimo: Jedna studija
~ETC-Zürich, bazirana na nerealisticnim pretpostavkama je za 02. april
\href{https://www.toponline.ch/news/coronavirus/detail/news/studie-bestaetigt-engpass-bei-spitalbetten-steht-kurz-bevor-00131333/}{prognozirala}
~prve probleme preopterecenosti u svajcarskim klinikama. Do toga do sada
nigde nije doslo.

U Svajcarskoj je pocetkom 2017. vladao posebno izrazeni talas gripa.
Tada je u prvih 6 nedelja godine bilo oko
\href{https://www.srf.ch/news/schweiz/todesursachen-statistik-woran-die-meisten-schweizerinnen-und-schweizer-sterben}{1500
dodatnih smrtnih slucajeva} u populaciji preko 65 godina. U Svajcarskoj
umire uobicajeno oko
\href{https://www.nzz.ch/lungenentzuendung-1.4550285}{1300 osoba}
godisnje iz razloga upale pluca, od kojih je 95\% starije od 65 godina.
Za poredjenje: trenutni ukupni \href{https://www.corona-data.ch/}{broj
umrlih ~u Svajcarskoj} ~sa (ne usled) Covid-19 iznosi 803.

Direktor jednog nemackog laboratorijuma za zivotnu okolinu pretpostavlja
da su stanovnici severnoitalijanske pokrajine Lombardije
\href{https://m.apotheke-adhoc.de/nachrichten/detail/coronavirus/erhoehen-legionellen-die-todesrate-einer-corona-infektion/}{posebno
podlozni virusnim infekcijama} usled notorno visokog opterecenja
legionelama: „Ukoliko su pluca, kao u trenutnoj situaciji, usled neke
vrusne infekcije oslabljena, onda bakterije imaju lak zadatak, mogu
negativno da uticu na tok bolesti kao i da prouzrokuju komplikacije.`` U
proslosti je u Lombardiji dolazilo do izbijanja brojnih upala pluca,
koje su bile prouzrokovane klima-uredjajima koji su bili~
zarazeni~legionelama.

Na osnovu podataka iz Kine, sirom sveta su definsani medicinski
protokoli, koji za pozitivno testirane pacijente u teskom stanju
predvidjaju jedno brzo invazivno vestacko disanje putem intubacije.
Protokoli polaze od toga da jedno neinvazivno ~olaksavanje disanja preko
maske, koje se mnogo lakse podnosi, nije dovoljno, te da pre svega
postoji opasnost od sirenja „opasnog virusa`` putem aerosola. Jos u
martu su
\href{https://www.doccheck.com/de/detail/articles/26271-covid-19-beatmung-und-dann}{nemacki
medicinari skrenuli paznju} da intubacija moze dovesti do dodatnih
ostecenja pluca i da ukupno gledano ima lose izglede na uspeh. U
medjuvremenu su se javili USA-lekari sa stanovistem da intubacija
\href{https://www.youtube.com/watch?v=k9GYTc53r2o}{vise steti nego
koristi}. Cesto pacijenti ne pate od akutnog prestanka rada pluca, vec
od neke vrste visinske bolesti, koja se vestackim disanjem pri visokom
pritisku jos pogorsava. Jos u februaru
\href{https://www.upi.com/Top_News/World-News/2020/02/14/Oxygen-therapy-working-for-coronavirus-patient-Seoul-says/6651581696794/}{javljaju
juznokorejski lekari}, da pacijenti u kriticnom stanju dobro reaguju ~na
terapiju kiseonikom bez aparata za vestacko disanje. Gore navedeni
americki lekar upozorava da je potrebno jos jednom ~razmisliti o u
upotrebi ararata za disanje, da se ne bi izazivala dodatna steta.

Oficijelni US-model za Covid-19 je dosada
\href{https://mail.protonmail.com/\#COVID19\%20Projections\%20(https://covid19.healthdata.org/projections}{preterao}
u predvidjanju neophodnih mera:

\begin{itemize}
\tightlist
\item
  Sto se tice potrebe bolnickog lecenja: 8 puta
\item
  Sto se tice pacijenata kojim je potrebna intenzivna nega: 6 puta
\item
  Sto se tice potrebnih aparata za vestacko disanje: 40 puta
\end{itemize}

Poznati americki statisticar Nate Silver objasnjava, zbog cega je
podatak o broju Corona-zarazenih
\href{https://fivethirtyeight.com/features/coronavirus-case-counts-are-meaningless/}{„besmislen``},
sve dok se ne sazna vise o~ broju i nacinu testiranja.

Jedan \href{https://vimeo.com/403175258}{prilog ~ARD Monitora} o
preteravanjima u vezi „svinjskog gripa`` iz 2009 pokazuje zaprepascujuce
paralele sa danasnjom situacijom. Zakljucak ARD-priloga glasi: „Stvarna
pandemija je ustvari strah ood nje``

\hypertarget{dalje-vesti}{%
\subparagraph{\texorpdfstring{\textbf{Dalje
vesti}}{Dalje vesti}}\label{dalje-vesti}}

\href{https://www.wodarg.com/}{Internet strana Dr. Wolfgang-a Wodarg-a},
jednog od prvih i internacionalno najpoznatijeg kriticara panike
Covid-19, je bila danas u toku nekoliko sati iskljucena od strane
nemackog providera Jimdo i tek posle jakih protesta opet aktivirana.

Nije poznato, da li je to iskljucenje bila posledica uopstenih protesta
ili radi neko politicke naredbe.

Jos pre toga je e-mail adresa profesora u penziji, doktora Sucharit-a
Bhakdi-ja, koji je napisao
\href{https://swprs.org/offener-brief-von-professor-sucharit-bhakdi-an-bundeskanzlerin-dr-angela-merkel/}{otvoreno
pismo upuceno kancelarki Angeli Merkel}, bila deaktivirana i isto tako
tek nakon protesta, aktivirana.

Danski parlament je 02.aprila doneo
\href{https://newsvoice.se/2020/04/danmark-forbjuder-corona-policy/}{novi
zakon} koji zabranjuje publiciiranje informacija o Covid-19, koje nisu u
skladu sa vladinim, omogucava brisanje internet strana kao i kaznjavanje
ili hapsenje njihovih autora. Neki komentatori su se posle toga smesta
povukli.

Nemacki publicista i novinar Harald Wiesendanger pise u jednom clanku,
da je
\href{https://www.nachrichten-fabrik.de/news/harald-wiesendanger-ueber-die-massenmedien-waehrend-der-corona-krise-ich-schaeme-mich---meines-berufsstands-152103}{novinarski
zanat u trenutnoj krizi potpuno zakazao.}: „Kako jedan poziv koji kao
jedna nezavisna, kriticna i bez predrasuda Cetvrta sila treba da
kontrolise mocnike, isto tako munjevito i jednodusno moze da podlegne
kolektivnoj histeriji ~kao i njegova publika i pristane ~na propagandu
vlasti, nasminkano izvestavanje i religiozno obozavanje eksperata Svete
Krave Nauke: to mi je neshvatljivo, odvratno, dosta mi je toga,
distanciram se u sramoti od te nedostojne predstave.``

Trenutno je
\href{https://www.sciencealert.com/one-third-of-the-world-s-population-are-now-restricted-in-where-they-can-go}{trecina
covecanstva} ~pod zabranom izlaska, to je vise ljudi nego sto je bilo
zivih u vreme Drugog svetskog rata.

U SAD se broj onih koji traze novcanu naknadu za nezaposlene brzo popeo
na
\href{https://www.reuters.com/article/us-health-coronavirus-usa-layoffs/us-weekly-jobless-claims-seen-at-record-high-again-idUSKBN21K0FX}{6
miliona.} , sto od velike krize 1929. godine nije zapamceno.

Preko 100 organizacija za borbu za ljudska i gradjanska
prava~\href{https://www.dailymail.co.uk/news/article-8181381/World-sleepwalking-surveillance-state-rights-groups-warn.html}{upozorava}
da covecanstvo koraca kroz Corona-krizu ~u drzavni sistem totalitarne
kontrole.~ Na Twitter-u su se etablirale Hastag \#covid19 i \#covid1984.

Americki geostrateg Henry Kissinger pise u „Wall Street
Journal``,\href{https://www.wsj.com/articles/the-coronavirus-pandemic-will-forever-alter-the-world-order-11585953005}{``Pandemija
Corone ce promeniti svetski poredak zauvek.``}~Sjedinjene Drzave moraju
da „zastite`` svoje gradjane i da istovremeno „planiraju jednu novu
epohu``.

\hypertarget{5-april-2020}{%
\paragraph{5. april 2020.}\label{5-april-2020}}

U jednom poucnom
\href{https://www.youtube.com/watch?v=lGC5sGdz4kg}{40-minutnom
intervjuu} objasnjava iternacionalno renomirani epidemiolog, profesor
Knut Wittkowski iz New York-a, da su sve mere preduzete protiv Covid-19
kontraproduktivne. Umesto „socijalne distance``, zatvaranja skola,
zabrane izlaska, zastite maskama, masovnog testiranje i vakcinisanja,
morao bi zivot da~ ide nesmetano dalje i da se tako brzo stvori imunitet
kod ljudi. Po svim dosadasnjim saznanjima Covid-19 nije opasniji od
ranijih epidemija gripa.

„British Medical Journal`` (BMJ)
\href{https://www.bmj.com/content/369/bmj.m1375}{izvestava} da po
najnovijim podacima iz Kine 78\% pozitivno testiranih osoba nije imalo
nikakvih simptoma. Jedan oxfordski epidemiolog kaze: „Ovi rezultati su
jako, jako vazni. () Onda se moramo pitati: zasto onda, do djavola, te
zabrane izlaska?``

Dr. Andreas Sönnichsen, direktor Odeljenja za opstu i porodicnu medicinu
Medicinskog fakulteta u Becu
\href{https://www.diepresse.com/5794224/was-machen-wir-da-auf-den-intensivstationen-eigentlich}{smatra
do sada preduzete mere „ludackim``.} Cela drzava je paralizovana, samo
dabi se oni malobrojni zastitili koji bi mogli biti pogodjeni.

Svedska vlada je
\href{https://www.telegraph.co.uk/news/2020/04/03/coronavirus-swedish-experiment-could-prove-britain-wrong/}{prva
u svetu najavila}, da ce ubuduce razlikovati smrtne slucajeve „usled`` i
smrtne slucajeva „sa`` Corona-virusom.

Hamburski Zdravstveni zavod od sada ispituje smrtne slucajeve na
\href{https://www.t-online.de/nachrichten/deutschland/id_87636856/coronavirus-hamburg-will-nur-echte-covid-19-tote-zaehlen.html}{odeljenju
za sudsku medicinu}, kako bi se brojali samo „pravi`` smrtni slucajevi
od Corona-virusa. Time se broj smrtnih slucajeva smanjio u odnosu na
podatke od „Rbert-Koch-Instituta`` vec za 50\%

Nemacki ~list „Ärzteblatt`` je izvestavao jos 2018. o
\href{https://www.aerzteblatt.de/nachrichten/97750/Vielzahl-an-Lungenentzuendungen-beunruhigen-Behoerden-in-Norditalien}{velikom
broju upale pluca} u severnoj Italiji, koji je uznemirio vlasti. Kao
razlog je pretpostavljena zagadjena pijaca voda.

Nemacki „Farmaceutski List``
\href{https://www.pharmazeutische-zeitung.de/atemstillstand-koennte-auch-zentrale-ursache-haben-116664/}{ukazuje}
na to, da u aktuelnoj situaciji „pacijenti cesto tesko obolevaju, cak i
umiru bez izrazenih simptoma na disajnim organima.`` Neurolozi
pretpostavljaju da Corona-virusi mogu da ostete i nervne celije. Jedno
drugo objasnjenje bi bilo, da da ti pacijenti, kojima je cesto potrebna
posebna nega, umiru usled jakog stresa.

Na osnovu \href{https://swprs.org/covid-19.hinweis-ii/}{najnovijih
podataka iz Svajcarske} su najcesci simptomi pozitivno testiranih
pacijenata u bolnicama povisena temperatura, kasalj i problemi kod
disanja. Kod 43\% tj. ca. 900 osoba je prisutna upala pluca. Takodje je
i u ovim slucajevima nejasno, da li je ona izazvana Corona-virusom, ili
od strane nekih drugih uzrocnika. Prosecna starost umrlih i pozitivno
testiranih je 83 godine, a raspon ide do 101 godine.

Britanski novinar Peter Hitchens u clanku
\href{https://www.firstthings.com/web-exclusives/2020/04/we-love-big-brother}{„We
love Big Brother``} opisuje kako i ljudi sa do sada kriticnim stavom i
pored nepostojecih medicinskih dokaza podlezu zarazi strahom. U jednom
intervjuu on objasnjava da je s obzirom na ugrozena osnovna gradjanska
prava, kritika jedna
\href{https://www.spiked-online.com/podcast-episode/in-this-lockdown-dissent-is-a-moral-duty/}{moralna
obaveza}.

Nemacki istoricar Rene Schlott pise o
\href{https://www.spiegel.de/consent-a-?targetUrl=https\%3A\%2F\%2Fwww.spiegel.de\%2Fpolitik\%2Fdeutschland\%2Fcorona-krise-und-buergerrechte-rendezvous-mit-dem-polizeistaat-a-68611322-f4d4-453f-aba5-5ec5a49ae329\&ref=https\%3A\%2F\%2Fswprs.org\%2Fcovid-19.hinweis-ii\%2F}{„Rendesvouz
mit dem Polizeistaat``} („Susret sa policijskom drzavom``): Kupiti
knjigu, sedeti na klupi u parku, sresti se sa prijateljima, sve to je
sada zabranjeno, kontrolisano i izlozeno denunciranju. Demokratski
osiguraci izgleda da su pregoreli. Gde i kako treba to treba da sa
zavrsi?``

U nemackoj se sire tuzbe od strane mnogih advokatskih kancelarija protiv
uvedenih mera i uredbi. Jedna pravnica sa tezistem na medicinskom pravu
pise u jednom
\href{http://beatebahner.de/lib.medien/aktualisierte\%20Pressemitteilung.pdf}{obavestenju:}
„Mere koje su uvele savezna vlada i republicke vlade su u eklatantnoj
suprotnosti sa ustavom i povredjuju mnostvo osnovnih prava gradjana
Nemacke u do sada nezabelezenom obimu. To vazi za sve Corona-uredbe svih
16 republika. Narocito je da te odredbe nisu u skladu sa Zakonom o
zastiti od infekcija, koji je tek pre nekoliko dana na brzinu izmenjen.
Jer postojeci podaci i statistike pokazuju da Corona-infekcija kod 95\%
populacije prolazi bezazleno (ili je verovetno vec prosla) i tako ne
predstavlja veliku opasnost za zajednicu.``

\href{https://swprs.org/offener-brief-von-professor-sucharit-bhakdi-an-bundeskanzlerin-dr-angela-merkel/}{Otvoreno
pismo} profesora Sucharit-a Bhakdi-ja upuceno kancelarki Angeli Merkel
je do sada dostupno na nemackom, engleskom, francuskom, spanskom,
ruskom, turskom, holandskom i estonskom; slede dalji prevodi.

U jednom intervjuu upozorava Edward Snowden da je Covid-19 opasan, ali
trenutan, dok su unistavanja ljudskih prava smrtonosna i permanentna.

\hypertarget{3-april-2020}{%
\paragraph{3. april 2020.}\label{3-april-2020}}

USA: \href{https://www.youtube.com/watch?v=5pIMD1enwd4}{video snimci
gradjanskih novinara} pokazuju da je u bolnicama, za koje su US-mediji
javljali da se nalaze u „ratnom stanju``, u stvarnosti prilicno mirno.

Austrija: I u Austriji su „Corona-smrtni slucajevi „jako liberalno``
definisani,
\href{https://www.heute.at/s/osterreich-bei-corona-todesstatistik-sehr-liberal-48665863}{kao
sto javljaju mediji}: „Da li se neko, ko je inficiran virusom, ali je
umro od neceg drugog, ubraja u „Corona-mrtve``? Da, kazu Rudi Anschober
i Bernhard Benka, clanovi Corona-Task-Force u Ministarstvu zdravstva.
„Trenutno postoji jasno pravilo: umrli sa virusom ili umrli radi
virusa``, svi oni ulaze u statistiku. Ne pravi se razlika, od cega je
pacijent zaista umro.

Prosto receno, neki 90-godisnji pacijent koji je slomio kuk i nekoliko
sati pre smrti biva zarazen Corona-virusom, ubraja se u „Corona-mrtve``.
Samo jedan primer da navedemo.

Nemacka: Nemacki „Robert-Koch-Institut`` odvraca od autopsije pozitivno
testiranih i umrlih, jer je opasnost od infekcije preko aerosola navodno
\href{https://www.youtube.com/watch?v=gSn_YaOYYcY}{prevelika.} Na taj
nacin je u velikom broju slucajeva nemoguce utvrditi pravi uzrok smrti.
Jedan doktor patolog
\href{https://www.youtube.com/watch?v=gSn_YaOYYcY}{komentira} na sledeci
nacin (Pismo je odstampano ispod video-snimka): „Budala je onaj, koji tu
nesto lose misli! Do sada se ~za patologe podrazumevalo da uz
odgovarajuce mere opreza vrse autopsije i kod infektivnih bolesti kao
sto su HIV/AIDS, hepatitis, tuberkuloza, PRION-bolesti itd. Zaista je
cudno da kod ove zaraze, koja po celom globusu odnosi~ hiljade ljudi i
paralise ekonomiju mnogih zemalja, postoje tako malobrojni rezultati
obdukcije (6 pacijenata iz Kine). Kako sa strane naucnog, tako i
policijskog gledista, trebao bi da postoji javni interes za rezultate
autopsija.. Da li se neko plasi da sazna prave razloge umrlih --
pozitivno testiranih? Da li bi se mozda ~broj „Corona-mrtvih`` onda
istopio kao sneg na prolecnom suncu?``

Italija: ruski personal je primetio
\href{https://de.sputniknews.com/panorama/20200402326767475-fachpersonal-todesfaelle-lombardei-zeitung/}{cudne
smrtne slucajeve} u starackim domovima u Lombardiji: „Tako je
registrovano mnogo slucajeva, kod kojih su navodno inficirani virusom
jednostavno zaspali, ali se nisu vise probudili. Kod umrlih do tada nisu
bili primeceni nikakvi ozboiljniji simptomi bolesti. (). Po kasnijoj
izjavli direktora ~starackog doma ~za RIA Novosti, nije jasno da li su
preminuli zaista bili inficirani virusom, jer niko u domu nije bio
testiran. U ustanovama, u kojima rade ruski medicinski timovi,
dezinfikuju se hodnici, sobe i prostorije za obedovanje.

O slicnim slucajevima je
\href{https://web.archive.org/web/20200330082928/https:/www.sueddeutsche.de/panorama/coronavirus-news-deutschland-wolfsburg-laschet-1.4828033}{izvestavano}
i iz Nemacke:~ Pacijenti bez simptoma umiru i vaze onda za
„Corona-mrtve``. Ovde se postavlja opet pitanje: Ko umire od virusa, a
ko radi ekstremnih mera koje su uvedene?

Osoblje za negu: Novine „Die Süddeutsche Zeitung``
\href{https://web.archive.org/web/20200330082928/https:/www.sueddeutsche.de/panorama/coronavirus-news-deutschland-wolfsburg-laschet-1.4828033}{izvestavaju}:
„U celoj Evropi ugrozena je nega starih ljudi u njihovim kucama, jer
osoblje ne moze do njih -- ili u bekstvu napusta doticnu zemlju u pravcu
domovine.

Ostalo: Profesor medicine na Stanford-u, Dr. Jay Bhattacharya je dao
\href{https://www.youtube.com/watch?v=-UO3Wd5urg0}{polucasovni
intervju}, u kome postavlja pitanje o „conventional wisdom`` u vezi
Covid-19. Dosadasnje mere su donesene na osnovu jako nesigurnih i
delimicno sumnjivih podataka.

\hypertarget{2-april-2020-i}{%
\paragraph{2. April 2020 (I)}\label{2-april-2020-i}}

\hypertarget{nemacka}{%
\subparagraph{\texorpdfstring{\textbf{Nemacka}}{Nemacka}}\label{nemacka}}

Kako navodi najnoviji
\href{https://influenza.rki.de/Wochenberichte/2019_2020/2020-13.pdf}{izvestaj}
nemackog „Robert-Koch-Instituta`` broj akutnih oboljenja disajnih organa
sirom zemlje je jako opao, „u svim starosnim grupama``.

Do 20. Marta (12. kalendarska nedelja) je ukupni broj stacionarno
lecenih akutnih slucajeva oboljenja disajnih organa jasno opao. U
starosnoj grupi od preko 80 godina se broj cak prepolovio u odnosu na
prethodnu nedelju.

U 73 kontrolisane bolnice je 7\% svih slucajeva oboljenja disajnih
organa dobilo dijagnozu Covid-19. U starosnoj grupi 35-59 godina je bilo
16\%, a u grupi 60-79 godina bilo je 13\% onih koji su dobili dijagnozu
Covid-19.

Ovi podaci odgovaraju onima iz drugih zemalja kao i tipicnoj rasirenosti
virusa iz Corona-grupe (5\% -15\%).

Jedan
\href{https://www.zeit.de/zustimmung?url=https\%3A\%2F\%2Fwww.zeit.de\%2Fwissen\%2F2020-04\%2Fkrankenhaeuser-kapazitaeten-coronavirus-patienten-deutschland\%2Fseite-2}{clanak
~casopisa ~„Zeit``} se bavi pitanjem pacijenata na intenzivnoj nezi u
Nemackoj:

„U ovo vreme svakodnevno i sa briznjom posmatraju politicari, strucnjaci
i gradjani eksponencijalno rastuci broj ljudi koji su inficirani.

To ipak nije odlucujuci podatak, ukoliko zelimo da ocenimo koliko tesko
ova Corona-kriza Nemacku pogadja, odnosno koliko tesko ce je pogoditi.
Jer taj broj je pre svega deformisan jednim stalnim porastom uradjenih
testova.

Da bi se ocenilo opterecenje zdravstvenog sistema je nasuprot tome vazan
pre svega broj onih, koji su tako tesko oboleli, da im je potrebno
vestacko disanje. Dok god za njih postoji dovoljno mesta za vestacko
disanje, mnogi od njih mogu biti spaseni. Tek ukoliko broj tih kreveta
postane nedovoljan, preti situacija iz Italije.

DIVI registar ~pokazuje sad, da je stanje na nemackim odeljenjima
intenzivne nege dosada opusteno. „Mi se jos uvek nalazimo u jednoj
„udobnoj`` situaciji`` izjavljuje Grabenhenrich. Broj tesko obolelih ne
raste ni iz daleka tako brzo kao broj inficiranih, a cak i onda bi bilo
moguce na raspolaganje staviti jako mnogo izuzetno dobro opremljenih
kreveta za intenzivnu negu``

\hypertarget{svajcarska-1}{%
\subparagraph{\texorpdfstring{\textbf{Svajcarska}}{Svajcarska}}\label{svajcarska-1}}

Svajcarski savezni zdravstveni zavod javlja da je do sada sprovedeno oko
139.330 Covid-19 testova sa oko 15\% pozitivnih rezultata (PDF). I ovaj
broj takodje odgovara broju u ostalim zemljama, ~predstavlja tipicnu
vrednost i koliko je to moguce sagledati, izgleda da da se ne povecava.

Eksponencionalno raste samo broj uradjenih testova, ali ne i broj
„inficirani``, obolelih ili cak umrlih.

31. marta je objavljena
\href{https://www.bfs.admin.ch/bfs/de/home/statistiken/gesundheit/gesundheitszustand/sterblichkeit-todesursachen.html}{nova
nedeljna statistika smrtnosti}, koja za 12. Kalendarsku nedelju (do 22.
Marta) po prvi put prognozira porast ukupne smrtnosti u starosnoj grupi
65+ (vidi grafiku). Konkretno bi ukupna smrtnost trebala da poraste za
200 smrtnih slucajeva nedeljno.

Ovaj porast je „izraz sadasnje pandemije``. Ovde se javlja sledeci
problem: do 22. Marta bilo je u Svajcarskoj ukupno
\href{https://de.wikipedia.org/wiki/COVID-19-Pandemie_in_der_Schweiz\#Todesf\%C3\%A4lle}{106
smrtnih slucajeva} pozitivno testiranih. Jedan porast od 200 smrtnih
slucajeva nedeljno bi znacio, da jedan veliki deo dodatne smrtnosti nije
prouzrokovan virusom, vec „merama protiv istog``.

Jedno drugo objasnjenje bi bilo, da je 200 smrtnih slucajeva pozitivno
testiranih od sledece
(\href{https://de.wikipedia.org/wiki/COVID-19-Pandemie_in_der_Schweiz\#Todesf\%C3\%A4lle}{13.})
nedelje vec uracunato. To bi znacilo, da su svi smrtni slucajevi
pozitivno testiranih uzeti kao dodatni smrtni slucajevi. S obzirom na
starost i profil bolesti kao i
\href{https://swprs.org/rki-relativiert-corona-todesfaelle/}{internacionalno
iskustvo} bi to bila sumnjiva pretpostavka.

U izvestaju se zaista i navodi: „Ove prve procene su jako nesigurne,
tako da je nemoguce objaviti tacne podatke``

Ukoliko bi se ispostavilo, da jedan veliki deo smrtnih slucajeva
pozitivno testiranih (prosecna starost: 83 godine) nisu dodatni smrtni
slucajevi, znacilo bi da se ukupni mortalitet ili nije povecao, ili da
se povecao pre svega radi drasticnih uvedenih mera, sto je
\href{https://swprs.org/offener-brief-von-professor-sucharit-bhakdi-an-bundeskanzlerin-dr-angela-merkel/}{bojazan}
nekih eksperata.

Svajcarwski „Tages-Anzeiger`` prikazuje aktuelnu ukupnu smrtnost u
poredjenju sa ranijim godinama (vidi grafiku). Ovo ilustruje da je
sadasnja smrtnost, cak i ako je zaista povisena, jos uvek ispod onih iz
prethodnoh godina u vreme jakih epidemija gripa.

\hypertarget{usa}{%
\subparagraph{\texorpdfstring{\textbf{USA}}{USA}}\label{usa}}

Jedan
\href{https://swprs.org/rate-of-positive-covid19-tests/}{svajcarski
biofizicar} je vizualizovao okolnosti, iz kojih se vidi da u USA (kao i
ostatku sveta) ne raste eksponencijalno broj „inficiranih``, vec broj
testova. ~Broj pozitivno testiranih u relaciji prema broju testiranih
ostaje konstantan ili raste vrlo lagano, sto govori protiv jedne
eksponencijalne virusne epidemije.

~

\includegraphics{https://swprs.files.wordpress.com/2020/04/ud-data-2-fs.png?w=736}

\hypertarget{ostalo}{%
\subparagraph{\texorpdfstring{\textbf{Ostalo}}{Ostalo}}\label{ostalo}}

Testovi namenjeni Velikoj Britaniji morali su da budu
\href{https://www.telegraph.co.uk/news/2020/03/30/uks-attempt-ramp-coronavirus-testing-hindered-key-components/}{vraceni},
jer su vec sadrzavali komponente nekih Corona-virusa.

Jedna studija britanskog „Imperial College``, koja je predvidela stotine
hiljada dodatnih smrtnih slucajeva, ali ni u jednom strucnom listu nije
bila objavljena ili proverena, bila je
\href{https://judithcurry.com/2020/04/01/imperial-college-uk-covid-19-numbers-dont-seem-to-add-up/}{bazirana
na jako nerealisticnim pretpostavkama}, kao sto se sad ispostavlja.

BBC se pita: \href{https://www.bbc.com/news/health-51979654}{„Da li su
smrtni slucajevi bili prouzrokovani Corona-virusom?``} , i odgovara:
„Virus bi mogao da bude glavni uzrok, jedan dodatni faktor, ili
jednostavno, samo tu``. Tako je jedan 18-godisnj muskarac predstavljen
kao „najmladja zrtva Corone``, posto je jedan~ dan pre smrti bio
pozitivno testiran. Bolnica je ipak posle saopstila, da je mladic umro
od jedne teske vec ranije prisutne bolesti.

Evropske zdravstvene vlasti ECDC su objavile
\href{https://www.ecdc.europa.eu/sites/default/files/documents/COVID-19-safe-handling-of-bodies-or-persons-dying-from-COVID19.pdf}{jako
stroga pravila} u ophodjenu sa lesevima pozitivno testiranih, odnosno
kod onih za koje se to pretpostavlja. S obzirom na jako nisku smrtnost,
izgledaju ova pravila vrlo sumnjivo; uz to ona znacajno opterecuju
zdravstvo i pogrebne ustanove, a pri tom su istovremeno sa jakim
medijskoim dejstvom.

„Bayerischer Rundfunk`` je obavio jedan
\href{https://www.br.de/nachrichten/wissen/bhakdis-brief-an-die-kanzlerin-was-ist-dran-an-seinen-fragen,RutYDhd}{kriticni
komentar} na otvoreno pismo profesora Sucharit-a Bhakdi-ja upuceno
kancelarki Angeli Merkel.

U dokumentarnom filmu Arte-kanala
\href{https://www.youtube.com/watch?v=1--c2SBYlMY}{„Profiteri strahom``}
iz 2009. je pokazano, kako je, u najvecoj meri privatno finansirana WHO,
jednu blagu epidemiju gripa (takozvani „svinjski`` grip) uzdigla na nivo
pandemije najviseg stepena i kako su kao posledica toga delimicno opasne
vakcine prodate vladama u vrednosti vise milijardi dolara. Neki od
ondasnjih protagonista su i u danasnjoj situaciji
\href{https://www.nature.com/articles/news.2009.424}{prisutni}.

Nekadasnji sudija britanskog „Supreme Court`` suda, Jonathan Sumption,
objasnjava ~u jednom
\href{https://www.spectator.co.uk/article/former-supreme-court-justice-this-is-what-a-police-state-is-like-}{intervjuu
za BBC} u vezi sa merama u Britaniji: „Ovako izgleda jedna policijska
drzava``.

\hypertarget{2-april-2020-ii}{%
\paragraph{2. april 2020 (II)}\label{2-april-2020-ii}}

Jos 2018 je naslovio britanski „Guardian``:
„\href{https://www.theguardian.com/society/2018/dec/09/steep-rise-lung-related-illness-hospitals-nhs}{Pollution
and flu bring steep rise in lung-related illnesses}„. Shortage of
specialists adds to worries that surge in respiratory diseases is
putting pressure on A\&Es.

U medjuvremenu se zale cak i
\href{https://pflege-prisma.de/2020/03/31/sterbezahlen-in-pflegeheimen/}{predstavnici
starackih domova i domova za negu} na restriktivne mere i neprimereno
izvestavanje medija o Covid-19: „I pre Corone se cesto desavalo da u
zimskim mesecima u relativno kratkom periodu umre puno ljudi u nasim
domovima, bez da su televizijski timovi stajali ~ispred vrata i
pokazivali osobe u zastitnim odelima, koje se herojski izlazu zarazi.``

Podaci iz severnoitalijanskog grada Trevisa (kod Venecije) pokazuju da
je ukupni mortalitet u gradskim bolnicama do kraja marta, uprkos 108
umrlih i pozitivno testiranh,
\href{https://swprs.files.wordpress.com/2020/04/reppublica-treviso.jpg}{otprilike
jednak} onom iz prethodnih godina. To je jedan daljni pokazatelj, da je
trenutno poviseni mortalitet u nekim mestima pre mogao biti vezan za
trece faktore kao sto su panika i kolaps, nego za Corona-virus.

Profesor Martin Haditsch, lekar, mikrobiolog, virolog i epidemiolog
\href{https://www.youtube.com/watch?v=PtzHH8DhgZM}{ostro kritikuje
uvedene mere} One su „potpuno bez osnova`` i gaze „osnove etike``.

Profesor John Oxford sa „Queen Mary`` univerziteta u Londonu, vodeci
svetski virolog i specijalista za grip i influencu, dolazi do
\href{https://novuscomms.com/2020/03/31/a-view-from-the-hvivo-open-orphan-orph-laboratory-professor-john-oxford/}{sledece
procene} u vezi sa Covid-19: „Licno bih rekao da je najbolji savet,
manje provoditi vremena uz televizijske vesti, koje su
senzacionalisticke i ne jako dobre. Ja licno smatram da je ovaj Covid-19
ustvari jedna jaca epidemija zimskog gripa. U ovom slucaju smo prosle
godine imali 8000 smrtnih slucajeva kod rizicnih grupa, znaci preko 65\%
ljudi sa bolestima srca itd. Ja ne verujem, da ce aktuelni Covid-19 te
brojke da nadmasi. Mi patimo od jedene medijske epidemije!``

\hypertarget{1-april-2020}{%
\paragraph{1. April 2020}\label{1-april-2020}}

\textbf{O situaciji u Italiji}

Italijanski lekari su izvestavali, da su
\href{https://www.scmp.com/news/china/society/article/3076334/coronavirus-strange-pneumonia-seen-lombardy-november-leading}{teske
upale pluca} u severnoj Italiji bile evidentirane jos krajam prosle
godine. Genetske analize pokazuju nasuprot tome, da se Covid-19 pojavio
u Italiji ocigledno tek u januaru. „Teske upale pluca, koje su u
novembru i decembru u Italiji dijagnozirane, moraju imati nekog drugog
uzrocnika``, kaze
\href{https://www.nzz.ch/wissenschaft/coronavirus-der-stammbaum-verraet-woher-es-kommt-ld.1548271}{NZZ}.
Time se jos jednom postavlja pitanje, koju ulogu u situaciji u Italiji
Covid-19 virus zaista ima, a koju ulogu igraju drugi cinioci.

30. marta je ovde skrenuta paznja na listu preminulih lekara „za vreme
Corona-krize``, od kojih su mnogi vec dugo vremena bili penzionisani i
stari i do 90 godina, te sa krizom nisu bili u vezi. Danas su sa te
liste
\href{https://portale.fnomceo.it/elenco-dei-medici-caduti-nel-corso-dellepidemia-di-covid-19/}{izbrisane
sve godine rodjenja} (pogledajte ipak poslednju
\href{https://web.archive.org/web/20200328152430/https://portale.fnomceo.it/elenco-dei-medici-caduti-nel-corso-dellepidemia-di-covid-19/}{verziju
iz arhiva}). Vrlo cudna praktika.

Osim toga smo dobili sledece saopstenje jednog posmatraca iz Italije,
koji navodi daljnje aspekte u vezi sa dramaticnom situacijom u Italiji,
koji mozda dosezu i dalje od jednog virusa.

„U poslednjim nedeljama osoblje iz istocne Evrope, koje se 24 casa
dnevno i sedam dana u nedelji u Italiji brinulo o onima kojima je nega
bila neophodna, bezeci je napustilo zemlju. I to ne malo radi stvaranja
panike kao i zabrane kretanja i zatvaranja granica, koje su bile
najavljene od strane „Vlasti vanrednog stanja``. Radi toga su stare
osobe, kojima je potrebna nega, kao i invalidi, ponekad i bez rodjaka,
ostavljani ~bespomocni i bez osoblja koje ih neguje.

Mnogi od njih posle nekoliko dana dospevaju u ionako vec godinama
propterecene bolnice, izmedju ostalog, zato sto su dehidrirali. Nazalost
je bolnicama falio personal, koji je bio zatvoren u kucama, jer je morao
da pazi na svoju decu, posto su skole i obdanista bili zatvoreni. To je
u sledu dovelo do potpunog sloma sistema negovanja starijih osoba i
invalida i haoticnog stanja upravo u oblastima, u kojima su uvedene
dodatne i jos ostrije „mere``.

Urgentna situacija koja je nastala radi panike, dovela je do temporarno
previse smrtnih slucajeva medju onima kojima je bila potrebna nega i
dodatno, medju mladjim bolnickim pacijentima. Te zrtve su posluzile
odgovornima i medijima da ljude jos vise uspanice, tako sto su na primer
javljali „novih 475 zrtava``, „Umrle iznosi armija iz bolnica``, i to
podupirali slikama naslaganih kovcega i vojnih kamiona.

Pri tome je to bila posledica straha grobarskih preduzeca od
„virusa-ubice``, radi koga su odbili da izvrse svoju duznost. Osim toga,
bilo je previse smrtnih slucajeva odjednom,a sa druge strane donet je
zakon od strane vlasti, po kojemu su lesevi, koji su u sebi imali
Corona-virus, morali da budu kremirani. U Italiji je do tog datuma
sprovodjen mali broj kremacija. Stoga je bilo i jako malo malih
krematorijuma, koji su su vrlo brzo dosli do svojih granica. Umri su
morali biti prebaceni u razne crkve.

Ovaj razvoj se odvijao prakticno u svim zemljama na isti nacin. Kvalitet
zdravstvenog sistema ima ipak odlucujuci uticaj na posledice. Stoga ima
manje problema u Nemackoj, Austriji ili Svacarskoj nego u Italiji,
Spaniji ili USA. Kao sto se u oficijelnim podacima moze videti, ne
postoji neko znacajnije povecanje mortaliteta. Samo jedno brdasce, koje
potice od ove tragedije.``

\hypertarget{klinike--u-usa-nemackoj-i-svajcarskoj}{%
\subparagraph{\texorpdfstring{\textbf{Klinike ~u USA, Nemackoj i
Svajcarskoj}}{Klinike ~u USA, Nemackoj i Svajcarskoj}}\label{klinike--u-usa-nemackoj-i-svajcarskoj}}

Americka televizija CBS je bila
\href{https://nypost.com/2020/04/01/cbs-admits-to-using-footage-from-italy-in-report-about-nyc/}{otkrivena}
kako u ~prilogu o aktuelnoj situaciji u New York-u koristi snimke iz
jednog italijanskog odeljenje intenzivne nege, bez da je to naglasila.
Mnogobrojni \href{https://www.youtube.com/watch?v=5pIMD1enwd4}{snimci
gradjanskih novinara} pokazuju da je u bolnicama na zapadnoj i istocnoj
obali trenutno mirno, nasuprot informacijama u medijima, koji su bolnice
poredili sa ratnim zonama. (war zones). Takodje su ~i „hladnjace za
leseve``, upadljivo prikazivane u medijima, nekoriscene i prazne.

Registar nemackih odeljenja intenzivne nege takodje ne pokazuje,
\href{https://www.intensivregister.de/\#/umgezogen}{nikakvo povecano}
opterecenje, nasuprot izvestajima u medijima. Gradjanski novinari
izvestavaju o napustenim centrima za prijem Covid-19 bolesnika.
Zaposleni u jednoj minhenskoj klinici objasnjava, da se „vec nedeljama
ceka na talas``, ali ~„broj pacijenata ne raste``. Izjave politicara
nisu u saglasnosti sa stvarnom situacijom, „mit o virusu-ubici`` „ne
moze biti potvrdjen``.

I u svajcarskim klinikama do sada nije prepoznatljiva povecana zauzetost
kreveta. Posetilac Kantonske Bolnice Luzern izvestava, da se tamo „manje
toga desava, nego u normalnim vremenima``. Celi spratovi su rezervisani
za Covid-19, ali osoblje „i dalje ceka na pacijente``.Takodje i bolnice
u Bern-u, Basel-u, Zug-u i Zürich-u su prazne. Cak ni u Tessin-u
\href{https://www.nzz.ch/schweiz/tessin-verlegt-erste-corona-patienten-in-deutschschweizer-spitaeler-ld.1549417}{nisu
odeljenja intenzivne nege opterecena}, ali i pored toga se prebacuju
pacijenti u nemacko-svajcarske oblasti. Sa medicinskog stanaovista to
nema smisla.

\hypertarget{ostale-medicinske-vesti}{%
\subparagraph{\texorpdfstring{\textbf{Ostale medicinske
vesti}}{Ostale medicinske vesti}}\label{ostale-medicinske-vesti}}

Dr. Ansgar Lohse, infektolog i direktor Univerzitetske klinike
Hamburg-Eppendorf,
\href{https://www.mopo.de/hamburg/uke-infektiologe-fordert-es-muessen-sich-mehr-menschen-mit-corona-infizieren-36483636}{zahteva
brzo ukidanje zabrane izlaska i socijalnog kontakta}. Jos vise ljudi
mora da se zarazi Corona-virusom. Obdanista i skole moraju da se otvore,
kako bi deca i roditelji putem zaraze stekli imunitet. Produzavanje
striktnih mera ce dovesti do ekonomske krize, koja ce isto tako odneti
ljudske zivote, izlaze medicinski radnik.

15\% pozitivno testiranih u Spaniji su lekari ili medicinsko osoblje.
Kod njih se u najvise slucajeva ne javljaju simptomi, ali moraju i pored
toga u karantin, sto dovodi dodatnog kolapsa spanskog zdravstvenog
sistema.

Dr. John Lee, profesor patologije u penziji, u clanku britanskog „The
Spectator`` tematira
\href{https://www.spectator.co.uk/article/how-to-understand-and-report-figures-for-covid-19-deaths-}{jako
obmanjujucu definiciju} „Corona-smrtnih slucajeva``.

\href{https://swprs.files.wordpress.com/2020/04/die-lage-in-norwegen.pdf}{Najnoviji
podaci iz Norveske}, analizirani od strane jedne naucnice iz oblasti
toksikologije, takodje pokazuju da stopa pozitivno testiranih ne raste
-- kako bis ocekivalo kod jedne epidemije -- vec osciluje u normalnom
opsegu za Corona-viruse izmedju 2\% i 10\%. Prosecna starost umrlih
iznosi 84 godine, uzroci smrti nisu javno objavljeni, preterana smrtnost
ne postoji.

Svedska, koja jos uvek opstaje bez radikalnih mera i ne javlja vecu
smrtnost ljudi (slicno azijskim zemljama kao sto su Japan ili Juzna
Koreja), biva, sto je vredno napomenuti,
\href{https://www.theguardian.com/world/2020/mar/30/catastrophe-sweden-coronavirus-stoicism-lockdown-europe}{izlozena
pritisku} internacionalnih medija, kako bi promenila strategiju.

Podaci iz savezne drzave New York pokazuju, da bi stopa pozitivno
testiranih kojima je ptrebno bolnicko lecenje mogla biti i
\href{https://www.nytimes.com/2020/03/27/nyregion/new-rochelle-coronavirus.html}{preko
dvadeset puta manja}, nego sto je to bilo prvobitno pretpostavljano.

Jedan
\href{https://www.doccheck.com/de/detail/articles/26271-covid-19-beatmung-und-dann}{prilog
na portalu „DocCheck}`` tematizuje problematiku primene vestackog
disanja kod pozitivno testiranih pacijenata. Kod pozitivno testiranih
pacijenata je odvracano od mogucnosti jedne jednostavne metode sa
maskom. S jedne strane se pretpostavlja da bi takva metoda mogla da bude
preslaba; s druge strane postoji bojazan, da se Corona-virus moze
rasiriti preko aerosola. Stoga pozitivno testirani pacijenti na
intenzivnoj nezi bivaju~ odmah tretirani intubacijom. Ali intubacija ima
lose izglede na uspeh i dovodi cesto do dodatnih ostecenja na plucima
(takozvana ventilatorom indukovana ostecenja). Kao kod pitanja
medikacije, postavlja se pitanje, da li bi neka blaza terapija imala
vise ~medicinskog smisla.

\hypertarget{ostale-vesti}{%
\subparagraph{\texorpdfstring{\textbf{Ostale
vesti}}{Ostale vesti}}\label{ostale-vesti}}

Jedan nemacki republicki ministar
\href{https://de.nachrichten.yahoo.com/strobl-bürger-verstöße-gegen-corona-regeln-polizei-melden-095746341.html?guccounter=1\&guce_referrer=aHR0cHM6Ly9zd3Bycy5vcmcvY292aWQtMTkuaGlud2Vpcy1paS8\&guce_referrer_sig=AQAAAJVaaBhiC-cLtQEICEBkNWa4gI877p2SHKrJ-LrsojdMmHDRqmD1o16GD-gMGYWBcqEw1kotQgyNfK1u3rSR8nJWkkNBDaamOjNzpLUFAcU6MGRrpgxxWDyHz70vxqcw1sFwMu_EI0MbuHgRdzRFCEhV1JlnM1WsSdJVqi6abywn}{poziva}
stanovnike „da budu budni i policiji jave eventualno nepridrzavanje
prema uvedenim merama protiv Corona-epidemije``.
„\href{https://www.br.de/nachrichten/bayern/buerger-melden-eifrig-verstoesse-gegen-corona-regeln,RuGXp1h}{Revnosno
prijavljen}o`` bese na primer „zabranjeno stvaranje grupica, deca na
igralistima ili proslave``. Takodje su setaci u Allgäu-u bili
prijavljeni.

Nemacki pravnici dizu alarm radi
„\href{https://www.focus.de/politik/deutschland/corona-regelungen-der-regierung-medizin-darf-nicht-gefaehrlicher-sein-als-die-krankheit_id_11827625.html}{teskih
povreda osnovnih gradjanskih prava}``. Ekspert za ustavno pravo Hans
Michael Heinig upozorava, da se demokratska pravna drzava u najkracem
roku moze preobratiti u jednu fasistoidnu-histericnu drzavu „higijene``.
Profesor Christoph Möllers sa berlinskog Humboldt Univerziteta
objasnjava, da zakon o zastiti od infekcija „ne moze biti osnova za
ovako dalekosezna ogranicenja prava na slobodu gradjana``. Po
nekadasnjem predsedniku nemackog Ustavnog suda, Hans Jürgen Papier-u,
vanredne mere ne opravdavaju ponistavanje prava na slobodu u korist
vlasti u jednoj drzavi masovne kontrole.

U vise zemalja su startovane online-peticije za ukidanje zabrane izlaska
i ponovno uspostavljanje osnovnih prava gradjana. Istovremeno se mnoze
slucajevi brisanja kritickih video priloga, cak i lekarskih. U Berlinu
je jedna najavljena manifestacija, na kojoj je deljen nemacki ustav,
\href{https://www.heise.de/tp/features/Wenn-Demonstranten-zu-Gefaehrdern-erklaert-werden-4692869.html}{raspustena
od strane policije.}

\hypertarget{31-mart-2020-i}{%
\paragraph{31. mart 2020 (I)}\label{31-mart-2020-i}}

Dr. Richard Capek i drugi istrazivaci su
\href{https://coronadaten.wordpress.com/}{vec pokazali}, da broj
pozitivno testiranih u odnosu na broj uradjenih testova u svim ispitanim
zemljama ostaje konstantan, sto govori protiv jednog eksponencijalnog
sirenja („epidemije``) virusa i ukazuje samo na eksponencijalni porast
testiranja.

Zavisno od zemlje se procenat pozitivno testiranih nalazi izmedju 5 i
15\%, sto odgovara uobicajenoj rasirenosti raznih Corona-virusa.
Interesantno je da se ove konstantne brojke od strane vlasti i medija ne
prezentuju
(\href{https://multipolar-magazin.de/artikel/coronavirus-irrefuhrung-fallzahlen}{ili
se cak uklanjaju}). Umesto toga se pokazuju eksponencijalne, ali nebitne
i obmanjujuce krivulje izvan konteksta.

To naravno ne odgovara profesionalnim medicinskim standardima, sto
pokazuje i tradicionalni
\href{https://influenza.rki.de/Saisonberichte/2017.pdf}{Izvestaj
influence/gripa} Robert-Koch-Instituta (S.30, grafika dole). Ovde su
pored broja pozitivnih nalaza (desno), prikazani ~kako broj proba (levo,
sivi „stubovi``), tako i pozitivna stopa rasta (levo, plava krivulja).

Iz toga se vidi, da pozitivna stopa u vreme sezone gripa o d 0-10\%
naglo raste do 80\% i posle nekoliko nedelja se spusta na normalni nivo.
U poredjenju sa tim, testovi kod Covid-19 pokazuju jednu konstantni
pozitivnu stopu u normalnom opsegu (vidi dole).

~

\includegraphics{https://swprs.files.wordpress.com/2020/03/rki-influenza-report-2017.png?w=650\&h=530}

Konstantna pozitivna stopa na primeru USA (dr. Richard Capek) vazi
analogno za sve ostale zemlje, za koje trenutno postoje podaci o broju
proba.

~

\includegraphics{https://swprs.files.wordpress.com/2020/03/infizierte-pro-test2603.jpg?w=600\&h=325}

\hypertarget{31-mart-2020-ii}{%
\paragraph{31. mart 2020 (II)}\label{31-mart-2020-ii}}

\href{https://off-guardian.org/2020/03/30/covid19-yet-to-impact-europes-overall-mortality/}{Graficki
prikaz evropskih podataka (monitoring)} pokazuje jasno, da je ukupna
smrtnost u celoj Evropi, bez obzira na uvedene mere, do 25. Marta na
normalnom nivou, ili ispod njega, a cesto i znatno ispod vrednosti iz
prethodnih godina. Jedino je u Italiji (65+) ukupna smrtnost bila
povisena (verovatno iz vise razloga), ali se i pored toga nalazila ispod
nivoa iz prethodnih (zimskih) perioda epidemija gripa.

Predsednik nemackog Robert-Koch-Instituta potvrdio je na konferenciji za
stampu, da prethodna oboljenja i pravi razlog smrti za definiciju
takozvanih „Corona-smrtnih slucajeva``
\href{https://swprs.org/rki-relativiert-corona-todesfaelle/}{ne igraju
~, nikakvu ulogu}. Sa medicinskog stanovista je ovakva definicija jasno
obmanjujuca. Ona ima ocigledni i opsti efekat stvaranje straha u drustvu
i politici.

U Italiji dolazi u medjuvremenu do
\href{https://www.tagesspiegel.de/politik/die-verlangsamung-ist-da-in-italien-zeichnet-sich-die-wende-in-der-coronakrise-ab/25698124.html}{smirivanja
situacije}. Ukoliko je to do sada moguce sagledati, kod temporarnog
povecanja smrtnosti (65+) se radilo o lokalnim efektima, koji su cesto
imali veze sa masovnom panikom i kolapsom zdravstvenog sistema. Jedan
italijanski politicar se pita, „kako je moguce, da se Covid-pacijenti iz
Brescia-e transportuju u Nemacku, ako u obliznjoj Veneciji, ili Veroni,
dve trecine kreveta za intenzivnu negu stoji prazno``.

Profesor medicine na Stanford-u, John Ioanisdis u
\href{https://onlinelibrary.wiley.com/doi/full/10.1111/eci.13222}{jednom
clanku} u European Journal of Clinical Investigation ~kritikuje „stetu
od preteranih informacija i mera koje su nisu uvedene na osnovu
baziranih cinjenica. „Cak su i strucni casopisi objavljivali neozbiljne
tvrdnje``

Jedna kineska studija, koja je pocetkom marta bila objavljena u Chinese
Journal of Epidemiology ~i dokazala nepouzdanost testova za Covid-virus
(ca. 50\% pogresno pozitivnih rezultata kod asimptomaticnih slucajeva)
je u medjuvremenu povucena. Glavni autor studije, dekan jednog
medicinskog fakulteta, nije zeleo da kaze razloge za to povlacenje i
spomenuo je samo jednu
\href{https://choice.npr.org/index.html?origin=https://www.npr.org/sections/health-shots/2020/03/26/822084429/in-defense-of-coronavirus-testing-strategy-administration-cited-retracted-study}{„osetljivu
situaciju``}. Neovisno o toj studiji je cinjenica o nepouzdanosti tzv.
PCR-testa vec dugo poznata: 2003 je u jednom kanadskom domu bila
„dokazana`` ~masovna infekcija SARS-Corona-virusima, za koju se
\href{https://www.ncbi.nlm.nih.gov/pmc/articles/PMC2095096/}{ispostavilo}
da je izazvana uobicajenim Corona-virusima (koji za osobe iz rizicne
grupe takodje modu da budu smrtonosni).

Autori nemackog Risk Menagement Networks RiskNET ~govore
\href{https://www.risknet.de/themen/risknews/covid-19-und-der-blindflug/}{u
jednoj analizi} o Covid-19 o „letu naslepo`` i o problematicnoj etici
plasiranja podataka.

Spanski intervju sa renomiranim argentinsko-francuskim virologom Pablo
Goldschmidt-om
\href{https://www.rubikon.news/artikel/der-corona-totalitarismus}{preveden
je na nemacki}. Goldschmidt smatra da su uvedene mere medicinski
kontraproduktivne i primecuje da bi sad trebalo citati Hann-u Arendt,
kako bi se razumeli izvori totalitarizma.

Madjarski predsednik vlade, ~Viktor Orban, je, kao i drugi premijeri i
predsednici, u okviru jednog „zakona o vanrednom stanju`` u
\href{https://www.krone.at/2127086}{velikoj meri ogranicio} madjarski
parlament i moze u sustini da vlada putem dekreta.

\hypertarget{14-marta-2020}{%
\paragraph{14. Marta 2020}\label{14-marta-2020}}

Prema
\href{https://www.epicentro.iss.it/coronavirus/sars-cov-2-decessi-italia}{podacima}
italijanskog nacionalnog zdravstvenog instituta prosecna starost
preminulih pozitivno testiranih u Italiji je trenutno ca. 81 godina.
10\% umrlih je bilo starosti preko 90 godina. 90\% umrlih je bilo
starosti preko 70 godina.

80\% umrlih je imalo dve ili vise hronicnih bolesti. 50\% umrlih je
imalo tri ili vise hronicnih bolesti. U njih ubrajamo pre svega
kardiovaskularna oboljenja, dijabetis, probleme respiratornih organa i
kancer.

Kod manje od 1\% umrlih se radilo o zdravim osobama bez hronocnih
oboljenja. Samo 30\% umrlih su zenskog pola.

Italijanski zdravstveni institut
\href{https://www.youtube.com/watch?v=0M4kbPDHGR0\&feature=youtu.be\&t=210}{razlikuje}
~pri tom umrle od virusa od onih umrlih sa virusom. U mnogo slucajeva
nije jasno da li je osoba umrla radi virusa, radi neke od prisutnih
hronocnih oboljenja ili radi eventualne kombinacije ove dve mogucnosti.

Kod dva umrla italijanska pacijenta mladja od 40 godina (obe osobe 39
godina starosti) radilo se o bolesniku od kancera kao i pacijentu sa
dijabetisom i dodatnim komplikacijama. I ovde je razlog smrti nejasan,
dakle da li radi virusa ili usled drugih oboljenja.

Preopterecenost klinika proizilazi iz uopstene navale pacijenata kao i
povecanog broja pacijenata kojima je trebala posebna ili intenzivna
nega. Pri tom se radai o stabilizovanju funkcije disanja kao i u slucaju
teskih slucajeva o anti-virusnim terapijama.

(Update: Nacionalni institut zdravstva je u medjuvremenu izdao
\href{https://www.epicentro.iss.it/coronavirus/bollettino/Report-COVID-20}{statisticki
izvestaj} o pozitivno testiranim~ kao i umrlim pacijentima, koji
potvrdjuje gore navedene podatke)

\textbf{Potrebno je obratiti paznju na sledece aspekte:}

Severna Italija ima jednu od najstarijih evropskih populacija~ kao i
\href{https://www.srf.ch/news/international/massive-schadstoffbelastung-nirgendwo-erkranken-so-viele-wegen-smog-wie-in-norditalien}{najvece
zagadjenje vazduha}, sto je vec u proslosti dovodilo do povecanog broja
oboljenja disajnih organa i time uslovljenih smrtnih slucajeva. Ovo je
neophodno uzeti kao dotatni faktor rizika.

Juzna Koreja je na primer dozivela mnogo blazi razvoj epidemije nego
Italija i vrhunac vec ostavila iza sebe. U Juznoj Koreji je bilo samo 70
smrtnih slucajeva sa pozitivnim testom. Pogodjeni su bili, kao u Italiji
pre svega pacijenti sa povecanim rizikom.

Kod svajcarskihh pozitivno testiranih smrtnih slucajeva se radilo
takodje o pacijentima sa povecanim rizikom i sa drugim oboljenjima.
Prosecna~ starosti je bila preko 80 godina, a tacan razlog smrti, dakle
radi virusa ili radi vec prisutne bolesti jos nije poznat.

Studije pokazuju takodje da testovi za viruse koji se sirom sveta
koriste u nekim slucajevima pokazuju pogresan ~pozitivni rezultat, sto
znaci da te osobe nisu obolele od novog Corona-virusa, nego vrlo moguce
od nekog dosadasnjeg Corona-virusa koji su neki od uzrocnika godisnje (i
aktuelne) prehlade i gripa. (1)

Za cenu opasnosti ove bolesti stoga nije odlucujuci cesto navodjeni broj
pozitivno testiranih i umrlih, vec broj zaista i iznenada od upale pluca
obolelih ili umrlih.

Kod uobicajene populacije u skolskom i zrelom uzrastu je na osnovu svih
dosadasnjih saznanja moguce poci od blagog ili umerenog toka ove~
bolesti (Covid-19). Starije osobe kao i one sa vec postojecim hronocnim
oboljenjima moraju biti posebno zasticene. Medicinski kapaciteti se
moraju optimalno pripremiti.

\textbf{Strucna literatura}

(1) Patrick et al.,
\href{https://www.ncbi.nlm.nih.gov/pmc/articles/PMC2095096/}{An Outbreak
of Human Coronavirus OC43 Infection and Serological Cross-reactivity
with SARS Coronavirus}, JIDMM, 2006.

(2) Grasselli et al.,
\href{https://jamanetwork.com/journals/jama/fullarticle/2763188}{Critical
Care Utilization for the COVID-19 Outbreak in Lombardy}, JAMA, März
2020.

(3) WHO,
\href{https://www.who.int/docs/default-source/coronaviruse/who-china-joint-mission-on-covid-19-final-report.pdf}{Report
of the WHO-China Joint Mission on Coronavirus Disease 2019}, Februar
2020.

\textbf{Parametri}

Vazni parametri su: Broj umrlih od gripa u jednoj godini, koji u
Svajcarskoj iznosi do 1000, Italiji oko 8000 i u Nemackoj oko 10000
(ponkad cifra dostize i 25000); Normalna smrtnost koja u Italiji iznosi
do 2000 na dan; Prosecni broj upale pluca koji u Italiji iznosi 120000
godisnje.

Aktuelna (ukupna) smrtnost u Italiji se nalazi na uobicajenoj normali,
cak i ispod nje. Povecana smrtnost bi trebala da bude uocljiva na
\href{https://www.euromomo.eu/index.html}{„evropskom monitoru``}.

\includegraphics{https://swprs.files.wordpress.com/2020/03/italy-smog.png?w=550\&h=309}

\begin{center}\rule{0.5\linewidth}{\linethickness}\end{center}

\hypertarget{swiss-policy-research}{%
\subsubsection{Swiss Policy Research}\label{swiss-policy-research}}

\begin{itemize}
\tightlist
\item
  \href{https://swprs.org/kontakt/}{Kontakt}
\item
  \href{https://swprs.org/uebersicht/}{Übersicht}
\item
  \href{https://swprs.org/donationen/}{Donationen}
\item
  \href{https://swprs.org/disclaimer/}{Disclaimer}
\end{itemize}

\hypertarget{english}{%
\subsubsection{English}\label{english}}

\begin{itemize}
\tightlist
\item
  \href{https://swprs.org/contact/}{About Us / Contact}
\item
  \href{https://swprs.org/media-navigator/}{The Media Navigator}
\item
  \href{https://swprs.org/the-american-empire-and-its-media/}{The CFR
  and the Media}
\item
  \href{https://swprs.org/donations/}{Donations}
\end{itemize}

\hypertarget{follow-by-email}{%
\subsubsection{Follow by email}\label{follow-by-email}}

Follow

\href{https://wordpress.com/?ref=footer_custom_com}{WordPress.com}.

\protect\hyperlink{}{Up ↑}

Post to

\protect\hyperlink{}{Cancel}

\includegraphics{https://pixel.wp.com/b.gif?v=noscript}
