\protect\hyperlink{content}{Skip to content}

\href{https://swprs.org/}{}

\protect\hyperlink{search-container}{Search}

Search for:

\href{https://swprs.org/}{\includegraphics{https://swprs.files.wordpress.com/2020/05/swiss-policy-research-logo-300.png}}

\href{https://swprs.org/}{Swiss Policy Research}

Geopolitics and Media

Menu

\begin{itemize}
\tightlist
\item
  \href{https://swprs.org}{Start}
\item
  \href{https://swprs.org/srf-propaganda-analyse/}{Studien}

  \begin{itemize}
  \tightlist
  \item
    \href{https://swprs.org/srf-propaganda-analyse/}{SRF / ZDF}
  \item
    \href{https://swprs.org/die-nzz-studie/}{NZZ-Studie}
  \item
    \href{https://swprs.org/der-propaganda-multiplikator/}{Agenturen}
  \item
    \href{https://swprs.org/die-propaganda-matrix/}{Medienmatrix}
  \end{itemize}
\item
  \href{https://swprs.org/medien-navigator/}{Analysen}

  \begin{itemize}
  \tightlist
  \item
    \href{https://swprs.org/medien-navigator/}{Navigator}
  \item
    \href{https://swprs.org/der-propaganda-schluessel/}{Techniken}
  \item
    \href{https://swprs.org/propaganda-in-der-wikipedia/}{Wikipedia}
  \item
    \href{https://swprs.org/logik-imperialer-kriege/}{Kriege}
  \end{itemize}
\item
  \href{https://swprs.org/netzwerk-medien-schweiz/}{Netzwerke}

  \begin{itemize}
  \tightlist
  \item
    \href{https://swprs.org/netzwerk-medien-schweiz/}{Schweiz}
  \item
    \href{https://swprs.org/netzwerk-medien-deutschland/}{Deutschland}
  \item
    \href{https://swprs.org/medien-in-oesterreich/}{Österreich}
  \item
    \href{https://swprs.org/das-american-empire-und-seine-medien/}{USA}
  \end{itemize}
\item
  \href{https://swprs.org/bericht-eines-journalisten/}{Fokus I}

  \begin{itemize}
  \tightlist
  \item
    \href{https://swprs.org/bericht-eines-journalisten/}{Journalistenbericht}
  \item
    \href{https://swprs.org/russische-propaganda/}{Russische Propaganda}
  \item
    \href{https://swprs.org/die-israel-lobby-fakten-und-mythen/}{Die
    »Israel-Lobby«}
  \item
    \href{https://swprs.org/geopolitik-und-paedokriminalitaet/}{Pädokriminalität}
  \end{itemize}
\item
  \href{https://swprs.org/migration-und-medien/}{Fokus II}

  \begin{itemize}
  \tightlist
  \item
    \href{https://swprs.org/covid-19-hinweis-ii/}{Coronavirus}
  \item
    \href{https://swprs.org/die-integrity-initiative/}{Integrity
    Initiative}
  \item
    \href{https://swprs.org/migration-und-medien/}{Migration \& Medien}
  \item
    \href{https://swprs.org/der-fall-magnitsky/}{Magnitsky Act}
  \end{itemize}
\item
  \href{https://swprs.org/kontakt/}{Projekt}

  \begin{itemize}
  \tightlist
  \item
    \href{https://swprs.org/kontakt/}{Kontakt}
  \item
    \href{https://swprs.org/uebersicht/}{Seitenübersicht}
  \item
    \href{https://swprs.org/medienspiegel/}{Medienspiegel}
  \item
    \href{https://swprs.org/donationen/}{Donationen}
  \end{itemize}
\item
  \href{https://swprs.org/contact/}{English}
\end{itemize}

\protect\hyperlink{}{Open Search}

\hypertarget{prof-dr-sucharit-bhakdiden-baux15fbakan-dr-angela-merkele-auxe7ux131k-mektup}{%
\section{Prof. Dr. Sucharit Bhakdi'den Başbakan Dr. Angela Merkel'e
Açık~Mektup}\label{prof-dr-sucharit-bhakdiden-baux15fbakan-dr-angela-merkele-auxe7ux131k-mektup}}

\includegraphics{https://swprs.files.wordpress.com/2020/03/bakhdi-letter-header.png?w=736\&h=297}

\textbf{Diller}:
\href{https://swprs.org/offener-brief-von-professor-sucharit-bhakdi-an-bundeskanzlerin-dr-angela-merkel/}{DE},
\href{https://swprs.org/open-letter-from-professor-sucharit-bhakdi-to-german-chancellor-dr-angela-merkel/}{EN},
\href{https://swprs.org/professor-sucharit-bhakdi-avalik-kiri-angela-merkelile/}{EE},
\href{http://piensachile.com/2020/03/carta-abierta-a-angela-merkel/}{ES},
\href{https://swprs.org/covid-19-lettre-ouverte-du-professeur-sucharit-bhakdi-a-la-chanceliere-allemande-dre-angela-merkel/}{FR},
\href{https://swprs.org/professor-bhakdi-open-letter-greek/}{GR},
\href{https://yanivhamo.com/open-letter-from-professor-sucharit-bhakdi-to-german-chancellor-dr-angela-merkel-hebrew/}{HE},
\href{https://swprs.org/lettera-aperta-del-professor-sucharit-bhakdi-al-cancelliere-tedesco-dr-angela-merkel/}{IT},
\href{https://swprs.org/open-brief-van-professor-sucharit-bhakdi-aan-de-duitse-bondskanselier-dr-angela-merkel/}{NL},
\href{https://swprs.org/carta-aberta-do-professor-sucharit-bhakdi-a-chanceler-alema-dra-angela-merkel/}{PT},
\href{https://swprs.org/\%d0\%be\%d1\%82\%d0\%ba\%d1\%80\%d1\%8b\%d1\%82\%d0\%be\%d0\%b5-\%d0\%bf\%d0\%b8\%d1\%81\%d1\%8c\%d0\%bc\%d0\%be-\%d0\%bf\%d1\%80\%d0\%be\%d1\%84\%d0\%b5\%d1\%81\%d1\%81\%d0\%be\%d1\%80\%d0\%b0-\%d1\%81\%d1\%83\%d1\%87\%d0\%b0\%d1\%80\%d0\%b8\%d1\%82\%d0\%b0/}{RU},
\href{https://alatyr.sk/open-letter-from-professor_sk.htm}{SK},
\href{https://swprs.org/prof-dr-sucharit-bhakdiden-basbakan-dr-angela-merkele-acik-mektup/}{TR}\\
\textbf{Bu mektubu paylaş}:
\href{https://twitter.com/intent/tweet?url=https://swprs.org/prof-dr-sucharit-bhakdiden-basbakan-dr-angela-merkele-acik-mektup/}{Twitter}
/
\href{https://www.facebook.com/share.php?u=https://swprs.org/prof-dr-sucharit-bhakdiden-basbakan-dr-angela-merkele-acik-mektup/}{Facebook}

Açık mektup Tıbbi Mikrobiyoloji Profesörü Sucharit Bhakdi'den ,,Johannes
Gutenberg Üniversitesi Mainz den emekli,, Almanya Başbakanı Angela
Merkel'e . Profesör Bhakdi Covid-19'a karşı tepkilerin acilen yeniden
değerlendirilmesi çağrısında bulundu. Ve ona 5 çok önemli soru sordu.
Mektup 26 Mart 2020 tarihli. İşte Mektubun Türkçesi.
(\href{https://swprs.org/offener-brief-von-professor-sucharit-bhakdi-an-bundeskanzlerin-dr-angela-merkel/}{PDF
olarak Almanca orijinal mektup.})

\hypertarget{auxe7ux131k-mektup}{%
\subsubsection{Açık Mektup}\label{auxe7ux131k-mektup}}

Sayın Başbakan,

Johannes Gutenberg Üniversitesinden emekli, uzun yıllar tıbbı
mikrobiyoloji enstitüsünde müdürlük yapmış biri olarak covid-19
virüsünün yayılımını azaltmak için şu anda halk yaşamı üzerinde yapılan
geniş kapsamlı kısıtlamaları ciddi derecede sorgulama zorunluluğu
hissediyorum.

Niyetim kesinlikle ne politik mesaj vermek ne de virüsün tehlikelerini
hafife almaktır. Ancak mevcut verileri ve gerçekleri göz önüne sermeye
bilimsel katkıda bulunmanın vazifem olduğunu hissediyorum. Ve ayrıca bu
hararetli tartışmada kaybolma tehlikesiyle karşı karşıya olan sorular
sormak istiyorum.

Endişemin altında yatan neden tamamıyla şu an büyük ölçüde hali hazırda
Almanya'da uygulanan ve Avrupa'nın büyük bir bölümünde hayata geçirilen
köklü sınırlama önlemlerinin öngörülemeyen sosyo-ekonomik sonuçlarıdır.

Dileğim gerekli öngörüyle halk yaşamını kısıtlamanın avantajlarını,
dezavantajlarını ve uzun vadede bunu sonuçlarının etkilerini ciddi
olarak tartışmaktır.

Bu maksatla bugüne kadar yeterli yanıtlar alamadığım ancak dengeli bir
analiz için vazgeçilmez olan 5 soruyla karşı karşıyayım.

Sizden çabuk yorum yapmanızı ve aynı zamanda da Federal Hükumete ülke
çapında etkili bir şekilde risk gruplarını halkın yaşamını kısıtlamadan
koruyacak stratejiler geliştirmeleri çağrısında bulunmanızı isteyeceğim.

Saygılarımla.

\textbf{Prof. em. Dr. med. Sucharit Bhakdi}

\hypertarget{1-istatistikler}{%
\subparagraph{\texorpdfstring{\textbf{1.
İstatistikler}}{1. İstatistikler}}\label{1-istatistikler}}

Robert Koch tarafından kurulan, Enfeksiyolojide, hastalık ve enfeksiyon
arasındaki geleneksel ayrım yapılır. Bir hastalık için klinik belirtiler
gerekir. Bu yüzden sadece ateş öksürük gibi semptomları olan hastalar
yeni vakalar olarak istatistiğe dahil edilir.\\
Diğer bir deyişle, covid-19 testiyle belirlenen yeni bir enfeksiyon
demek tam olarak hastanede yatağa ihtiyaç duyan yeni bir hasta demek
değildir. Ancak şu an farz edilen tüm enfekte olmuş insanların \% 5 inin
ciddi şekilde hasta olmaları ve solunum desteğine ihtiyaç duymalarıdır.
Bu tahmine dayalı öngörüler sağlık sisteminin aşırı yükleneceğine
yöneliktir.

\textbf{Sorum}: Bu öngörüler semptom geliştirmeyen enfekte olan hastalar
ve gerçekten hasta olanlar arasında ayrım yapıyor mu?

\hypertarget{2-tehlikelilik}{%
\subparagraph{\texorpdfstring{\textbf{2.
Tehlikelilik}}{2. Tehlikelilik}}\label{2-tehlikelilik}}

Uzun zamandır bir dizi korona virüsü medya tarafından fark edilmeden
zaten ortalıkta dolaşmakta. Şayet covid-19 virüsünün mevcut diğer korona
virüslerinden daha tehlikeli olmadığı ortaya çıksaydı, bütün karşı
önlemler belli ki gereksiz olurdu.

Uluslarası tanınan International Antimicrobial Agents dergisi çok
yakında tam da bu meseleye dikkat çeken bir yazı yayımlayacak.
Çalışmanın ön sonuçları hali hazırda bugün görülebilir ve tehlike
anlamında yeni korona virüsün eskisinden farklı olmadığı sonucuna
varılabilir. Yazarlar, bunu ``SARS-CoV-2: Korku Verilere Karşı''
başlıklı yazılarında ifade ediyorlar..

\textbf{Sorum}: Diğer koronavirüs enfeksiyonlarına kıyasla covid-19
tanılı hastaların yoğun bakım ünitelerinde sebep oldukları mevcut
yoğunluk nasıldır? Ve bu veriler hükumet tarafından daha fazla karar
vermekte ne ölçüde hesaba katılıyor? Ayrıca, yukarıdaki çalışma bugüne
kadarki planlamalarda hesaba katıldı mı? Burada da elbette
``tanılanmış'' demek: Kişinin hastalığında virüsün önemli bir payının
olması demektir, önceki hastalıklarının değil.

\hypertarget{3-yayux131lmasux131}{%
\subparagraph{\texorpdfstring{\textbf{3.
Yayılması}}{3. Yayılması}}\label{3-yayux131lmasux131}}

Süddeutsche Zeitung'daki bir rapora göre, çok alıntı yapılan Robert Koch
enstitüsü bile covid-19 için ne kadar test edildiğini tam olarak
bilmiyor. Ancak bununla birlikte test hacmi arttıkça vaka sayısında da
hızlı bir artış olduğu bir gerçektir.

Bu nedenle virüsün sağlıklı popülasyonda zaten fark edilmeden
yayıldığından şüphe etmek mantıklıdır.\\
Bunun iki sonucu olur. Birincisi; resmi ölüm oranı çok yüksektir -26
Mart 2020 de örneğin 37.300 enfeksiyondan 206 ölüm meydana gelmiş (yani
\%0.55 oranında). İkincisi; virüsün sağlıklı popülasyonda yayılmasını
önlemenin mümkün olmadığı anlamına gelir.

\textbf{Sorum}: Virüsün gerçek yayılımını doğrulamak için halihazırda
``sağlıklı genel nüfusun'' rastgele örnekleme usulü bir araştırması var
mı veya yakın gelecekte planlanıyor mu?

\hypertarget{4-uxf6luxfcm-oranux131}{%
\subparagraph{\texorpdfstring{\textbf{4. Ölüm
Oranı}}{4. Ölüm Oranı}}\label{4-uxf6luxfcm-oranux131}}

Almanya'da ölüm oranındaki artış korkusu (şu anda \%0.55) medyanın yoğun
ilgisine maruz kalıyor. Birçok kişi, zamanında harekete geçilmezse
İtalya (\%10) ve İspanya'da (\%7) olduğu gibi hızla artacağı konusunda
endişe duyuyor.

Aynı zamanda, ölüm sırasında bu virüsün mevcut olduğu tespit edilir
edilmez, ölümdeki diğer faktörler gözetilmeksizin ölümler ``virüs
kaynaklı ölümler'' şeklinde bildirilerek dünya genelinde hata
yapılıyor.\\
Bu, enfeksiyolojinin temel prensibini ihlal eder. Sadece bir etken
hastalıkta ya da ölümde önemli bir rol oynadığı kesinse tanılanabilir.\\
Almanya Bilimsel Tıp Dernekleri Birliği açıkça kendi kurallarında
yazmaktadır.;\\
Ölüm nedenine ek olarak Ölüm belgesinde üçüncü sırada yer alan ilgili
hastalık ile nedensel bir zincir belirtilmelidir. Zaman zaman dört
bağlantılı nedensel zincir de belirtilmelidir.

Şu anda virüsün gerçekte kaç ölüme neden olduğunu belirlemek için en
azından geçmişe bakıldığında, tıbbi kaynakların daha kritik
analizlerinin yapılıp yapılmadığına dair resmi bir bilgi
bulunmamaktadır.

\textbf{Sorum}: Almanya covid-19 genel şüphe trendini takip mi etti? Bu
kategorizeleştirmeyi diğer ülkelerde olduğu gibi eleştirel olmayan bir
şekilde sürdürmek niyetinde mi? Öyleyse gerçek korona ile ilişkili
ölümler ve ölüm anında kazara virüs varlığı arasında nasıl bir ayrım
yapılır?

\hypertarget{5-karux15fux131laux15ftux131rux131labilirlik}{%
\subparagraph{\texorpdfstring{\textbf{5.
Karşılaştırılabilirlik}}{5. Karşılaştırılabilirlik}}\label{5-karux15fux131laux15ftux131rux131labilirlik}}

İtalya'daki dehşet verici durum tekrar tekrar referans senaryosu olarak
kullanılıyor. Ancak o ülkede virüsün gerçek rolü birçok nedenden dolayı
tamamen belirsizdir, sadece yukarıdaki 3. ve 4. noktaların burada
geçerli olması nedeniyle değil, aynı zamanda bu bölgeleri özellikle
savunmasız hale getiren istisnai dış faktörler olduğu içindir.

Bu faktörlerden biri, İtalya'nın kuzeyinde artan hava kirliliğidir.
Dünya Sağlık Örgütü tahminlerine göre bu durum, virüs olmadan bile
2006'da sadece İtalya'nın en büyük 13 şehrinde yılda 8000'den fazla ek
ölüme yol açtı. Durum o zamandan beri önemli ölçüde değişmedi. Son
olarak hava kirliliğinin çok genç ve yaşlılarda viral akciğer
hastalıkları riskini büyük ölçüde arttırdığı da gösterilmiştir.

Ayrıca bu ülkede (İtalya'da) riskli kesimde olan nüfusun \%27.4 `ü,
İspanya`da da \%35.5 `i gençlerle yaşıyor. Almanya'da ise bu oran
yalnızca \%7. Buna ek olarak TU Berlin'deki Sağlık Bakım Yönetimi
Başkanı Prof. Dr Reinhard Busse'ye göre Almanya yoğun bakım üniteleri
açısından İtalya'dan çok daha iyi donanımlıdır, yaklaşık 2.5 kat daha
fazla.

\textbf{Sorum}: Nüfusu bu temel farklılıklardan haberdar etmek ve
insanların İtalya ve İspanya'daki senaryoların burada gerçekçi
olmadığını anlamalarını sağlamak için ne tür çabalar harcanıyor?

\hypertarget{referanslar}{%
\subparagraph{\texorpdfstring{\textbf{Referanslar:}}{Referanslar:}}\label{referanslar}}

{[}1{]} Fachwörterbuch Infektionsschutz und Infektionsepidemiologie.
\href{https://www.rki.de/DE/Content/Service/Publikationen/Fachwoerterbuch_Infektionsschutz.html}{Fachwörter
-- Definitionen -- Interpretationen}. Robert Koch-Institut, Berlin 2015.
(abgerufen am 26.3.2020)

{[}2{]} Killerby et al., Human Coronavirus Circulation in the United
States 2014--2017. J Clin Virol. 2018, 101, 52-56

{[}3{]} Roussel et al. SARS-CoV-2: Fear Versus Data. Int. J. Antimicrob.
Agents 2020, 105947

{[}4{]} Charisius, H.
\href{https://www.sueddeutsche.de/gesundheit/covid-19-coronavirus-testverfahren-1.4855487}{Covid-19:
Wie gut testet Deutschland?} Süddeutsche Zeitung. (abgerufen am
27.3.2020)

{[}5{]} Johns Hopkins University,
\href{https://coronavirus.jhu.edu/map.html}{Coronavirus Resource
Center}. 2020. (abgerufen am 26.3.2020)

{[}6{]} S1-Leitlinie 054-001,
\href{https://www.awmf.org/uploads/tx_szleitlinien/054-002l_S1_Regeln-zur-Durchfuehrung-der-aerztlichen-Leichenschau_2018-02_01.pdf}{Regeln
zur Durchführung der ärztlichen Leichenschau}. AWMF Online (abgerufen am
26.3.2020)

{[}7{]} Martuzzi et al. Health Impact of PM10 and Ozone in 13 Italian
Cities. World Health Organization Regional Office for Europe. WHOLIS
number E88700 2006

{[}8{]} European Environment Agency,
\href{https://www.eea.europa.eu/themes/air/country-fact-sheets/2019-country-fact-sheets}{Air
Pollution Country Fact Sheets 2019}, (abgerufen am 26.3.2020)

{[}9{]} Croft et al. The Association between Respiratory Infection and
Air Pollution in the Setting of Air Quality Policy and Economic Change.
Ann. Am. Thorac. Soc. 2019, 16, 321--330.

{[}10{]} United Nations, Department of Economic and Social Affairs,
Population Division. Living Arrange­ments of Older Persons: A Report on
an Expanded International Dataset (ST/ESA/SER.A/407). 2017

{[}11{]} Deutsches Ärzteblatt,
\href{https://www.aerzteblatt.de/nachrichten/111029/Ueberlastung-deutscher-Krankenhaeuser-durch-COVID-19-laut-Experten-unwahrscheinlich}{Überlastung
deutscher Krankenhäuser durch COVID-19 laut Experten unwahrscheinlich},
(abgerufen am 26.3.2020)

\hypertarget{ana-makaleye-geri-duxf6n-isviuxe7reli-bir-doktordan-kovid-19-uxfczerine}{%
\paragraph{\texorpdfstring{Ana makaleye geri dön:
\href{https://swprs.org/isvicreli-bir-doktordan-kovid-19-uezerine/}{İsviçreli
Bir Doktordan Kovid-19
Üzerine}}{Ana makaleye geri dön: İsviçreli Bir Doktordan Kovid-19 Üzerine}}\label{ana-makaleye-geri-duxf6n-isviuxe7reli-bir-doktordan-kovid-19-uxfczerine}}

\begin{center}\rule{0.5\linewidth}{\linethickness}\end{center}

\hypertarget{swiss-policy-research}{%
\subsubsection{Swiss Policy Research}\label{swiss-policy-research}}

\begin{itemize}
\tightlist
\item
  \href{https://swprs.org/kontakt/}{Kontakt}
\item
  \href{https://swprs.org/uebersicht/}{Übersicht}
\item
  \href{https://swprs.org/donationen/}{Donationen}
\item
  \href{https://swprs.org/disclaimer/}{Disclaimer}
\end{itemize}

\hypertarget{english}{%
\subsubsection{English}\label{english}}

\begin{itemize}
\tightlist
\item
  \href{https://swprs.org/contact/}{About Us / Contact}
\item
  \href{https://swprs.org/media-navigator/}{The Media Navigator}
\item
  \href{https://swprs.org/the-american-empire-and-its-media/}{The CFR
  and the Media}
\item
  \href{https://swprs.org/donations/}{Donations}
\end{itemize}

\hypertarget{follow-by-email}{%
\subsubsection{Follow by email}\label{follow-by-email}}

Follow

\href{https://wordpress.com/?ref=footer_custom_com}{WordPress.com}.

\protect\hyperlink{}{Up ↑}

Post to

\protect\hyperlink{}{Cancel}

\includegraphics{https://pixel.wp.com/b.gif?v=noscript}
