\protect\hyperlink{content}{Skip to content}

\href{https://swprs.org/}{}

\protect\hyperlink{search-container}{Search}

Search for:

\href{https://swprs.org/}{\includegraphics{https://swprs.files.wordpress.com/2020/05/swiss-policy-research-logo-300.png}}

\href{https://swprs.org/}{Swiss Policy Research}

Geopolitics and Media

Menu

\begin{itemize}
\tightlist
\item
  \href{https://swprs.org}{Start}
\item
  \href{https://swprs.org/srf-propaganda-analyse/}{Studien}

  \begin{itemize}
  \tightlist
  \item
    \href{https://swprs.org/srf-propaganda-analyse/}{SRF / ZDF}
  \item
    \href{https://swprs.org/die-nzz-studie/}{NZZ-Studie}
  \item
    \href{https://swprs.org/der-propaganda-multiplikator/}{Agenturen}
  \item
    \href{https://swprs.org/die-propaganda-matrix/}{Medienmatrix}
  \end{itemize}
\item
  \href{https://swprs.org/medien-navigator/}{Analysen}

  \begin{itemize}
  \tightlist
  \item
    \href{https://swprs.org/medien-navigator/}{Navigator}
  \item
    \href{https://swprs.org/der-propaganda-schluessel/}{Techniken}
  \item
    \href{https://swprs.org/propaganda-in-der-wikipedia/}{Wikipedia}
  \item
    \href{https://swprs.org/logik-imperialer-kriege/}{Kriege}
  \end{itemize}
\item
  \href{https://swprs.org/netzwerk-medien-schweiz/}{Netzwerke}

  \begin{itemize}
  \tightlist
  \item
    \href{https://swprs.org/netzwerk-medien-schweiz/}{Schweiz}
  \item
    \href{https://swprs.org/netzwerk-medien-deutschland/}{Deutschland}
  \item
    \href{https://swprs.org/medien-in-oesterreich/}{Österreich}
  \item
    \href{https://swprs.org/das-american-empire-und-seine-medien/}{USA}
  \end{itemize}
\item
  \href{https://swprs.org/bericht-eines-journalisten/}{Fokus I}

  \begin{itemize}
  \tightlist
  \item
    \href{https://swprs.org/bericht-eines-journalisten/}{Journalistenbericht}
  \item
    \href{https://swprs.org/russische-propaganda/}{Russische Propaganda}
  \item
    \href{https://swprs.org/die-israel-lobby-fakten-und-mythen/}{Die
    »Israel-Lobby«}
  \item
    \href{https://swprs.org/geopolitik-und-paedokriminalitaet/}{Pädokriminalität}
  \end{itemize}
\item
  \href{https://swprs.org/migration-und-medien/}{Fokus II}

  \begin{itemize}
  \tightlist
  \item
    \href{https://swprs.org/covid-19-hinweis-ii/}{Coronavirus}
  \item
    \href{https://swprs.org/die-integrity-initiative/}{Integrity
    Initiative}
  \item
    \href{https://swprs.org/migration-und-medien/}{Migration \& Medien}
  \item
    \href{https://swprs.org/der-fall-magnitsky/}{Magnitsky Act}
  \end{itemize}
\item
  \href{https://swprs.org/kontakt/}{Projekt}

  \begin{itemize}
  \tightlist
  \item
    \href{https://swprs.org/kontakt/}{Kontakt}
  \item
    \href{https://swprs.org/uebersicht/}{Seitenübersicht}
  \item
    \href{https://swprs.org/medienspiegel/}{Medienspiegel}
  \item
    \href{https://swprs.org/donationen/}{Donationen}
  \end{itemize}
\item
  \href{https://swprs.org/contact/}{English}
\end{itemize}

\protect\hyperlink{}{Open Search}

\hypertarget{urteil-im-wikipedia-prozess}{%
\section{Urteil im
Wikipedia-Prozess}\label{urteil-im-wikipedia-prozess}}

\textbf{Publiziert}: 21. Februar 2019; \textbf{Aktualisiert}: Februar
2020; \textbf{Sprachen}: DE,
\href{https://swprs.org/ruling-wikipedia-trial/}{EN}*\\
*

\textbf{Landgericht Hamburg: Urteil in einem der bedeutendsten modernen
Medienprozesse.}

Rechercheure der Wiener \href{https://gruppe42.com/}{\emph{Gruppe42}}
berichteten 2018 über einen der einflussreichsten manipulativ agierenden
Wikipedia-Autoren und nannten dabei erstmals seinen echten Namen,
wogegen der Autor eine einstweilige Verfügung mit Strafandrohung von bis
zu €250.000 erwirkte.

Das Landgericht Hamburg
\href{https://kenfm.de/tagesdosis-26-2-2019-wikipedia-manipulationen-feliks-darf-nach-gerichtsurteil-wieder-mit-klarnamen-genannt-werden/}{entschied}
nun in einem wegweisenden Urteil, dass die Namensnennung aufgrund des
überwiegenden öffentlichen Interesses rechtmäßig war.

Beim fraglichen Wikipedia-Autor mit dem Decknamen »Feliks« handelt es
sich um einen ehemaligen Funktionär des
\href{https://de.wikipedia.org/wiki/Antideutsche}{pro-israelischen
Flügels} der \emph{Linkspartei} sowie um ein ehemaliges
\href{https://de.wikipedia.org/wiki/Sar-El}{Auslands­­mitglied} der
israelischen Armee mit Spezialabzeichen der US-Armee und weiterer
Streitkräfte.

Der Autor bearbeitete insgesamt mehrere tausend Wikipedia-Artikel und
denunzierte dabei zahlreiche Personen, darunter insbesondere Politiker,
Publizisten und Forscher, die sich kritisch zu transatlantischen oder
israelischen Positionen geäußert hatten.

Der selbstgewählte Deckname »Feliks« bezieht sich auf den Gründer und
ersten Direktor des sowjetischen Geheimdienstes Tscheka/GPU, Feliks
Dserschinski, unter dessen Leitung bis zu einhunderttausend politische
Gegner exekutiert wurden.

Das Hamburger Urteil dürfte einen Präzedenzfall darstellen und
erhebliche Signalwirkung haben. Derzeit prüfen offenbar mehrere
Betroffene rechtliche Maßnahmen gegen »Feliks« und weitere denunziativ
oder manipulativ agierende Wikipedia-Autoren.

Die Bedeutung des Hamburger Wikipedia-Urteils könnte jene des
\href{http://www.spiegel.de/kultur/tv/verfassungsgericht-klage-gegen-zdf-staatsvertrag-a-960571.html}{ZDF-Urteils}
des Bundes­­ver­fassungs­­gerichts von 2014 noch übertreffen, zumal in
Anbetracht der internationalen Relevanz der deutsch­sprachigen Wikipedia
(ca. eine Milliarde Aufrufe durch 100 Millionen Geräte
\href{https://swprs.org/propaganda-in-der-wikipedia/}{pro Monat}).

Bereits im August 2018 wurde der Wikipedia durch das Landgericht Berlin
das sogenannte »Laienprivileg«
\href{https://www.heise.de/newsticker/meldung/Urteil-gegen-Wikipedia-Keine-rufschaedigende-Kritik-ohne-Recherche-4209610.html}{aberkannt}.
Dies bedeutet, dass Wikipedia-Autoren rufschädigende Behauptungen aus
Medienbeiträgen oder anderen Quellen nicht mehr ungeprüft übernehmen
dürfen.

Die Trägerstiftung \emph{Wikimedia} sowie traditionelle Medien äußerten
sich bisher nicht über die (geo-)politische Manipulation der Wikipedia
und die damit zusammen­hängenden Prozesse.

\textbf{Nachtrag (Februar 2020):} Im Februar 2020 wurde das Urteil vom
OLG Hamburg
\href{https://kanzleikompa.de/2020/02/18/olg-hamburg-deanonymisierung-von-autoren-politischer-beitraege-zulaessig/}{bestätigt}.

\textbf{Siehe auch}:
\href{https://swprs.org/weiteres-urteil-im-fall-wikipedia/}{Bezirksgericht
Münchwilen: Weiteres Wikipedia-Urteil} (SPR, Februar 2019)

\hypertarget{medienberichte-zum-urteil}{%
\paragraph{Medienberichte zum Urteil}\label{medienberichte-zum-urteil}}

\begin{itemize}
\tightlist
\item
  \href{https://www.youtube.com/watch?v=Xr0CDRL4vKk}{Interviews zum
  Prozess (vor Urteilsverkündung)} (MGTV, 16. Februar 2019)
\item
  \href{https://kenfm.de/tagesdosis-26-2-2019-wikipedia-manipulationen-feliks-darf-nach-gerichtsurteil-wieder-mit-klarnamen-genannt-werden/}{»Feliks
  darf nach Gerichtsurteil wieder mit Klarnamen genannt werden«} (KenFM,
  26. Feb. 2019)
\item
  \href{https://derstandard.at/2000098702546/Verschwoerungstheoretiker-duerfen-Wikipedia-Autor-outen}{»Verschwörungstheoretiker
  dürfen Wikipedia-Autor outen«} (Standard, 28. Feb.~ 2019)
\item
  \href{https://conservo.wordpress.com/2019/03/01/sensationell-die-enttarnung-eines-denunzianten-das-ende-des-maskenballs-fuer-anonyme-wikipedia-desinformanten/}{»Ende
  des Maskenballs für anonyme Wikipedia-Desinformanten«} (Conservo, 1.
  März 2019)
\item
  \href{https://deutsch.rt.com/inland/85235-landgericht-hamburg-namensnennung-von-wikipedia/}{»LG
  Hamburg: Namensnennung von Wikipedia-Autor gesetzeskonform«} (RTD, 4.
  März 2019)
\item
  \href{https://de.wikipedia.org/wiki/Wikipedia:Kurier/Ausgabe_3_2019\#Die_Gesellschaft_will_wissen,_wer_wir_sind}{»Die
  Gesellschaft will wissen, wer wir sind«} (Wikipedia Kurier, 6. März
  2019)
\item
  \href{https://www.jungewelt.de/artikel/350538.aus-dem-rahmen-gefallen.html}{»Wikipedia:
  Aus dem Rahmen gefallen«} (Junge Welt, 7. März 2019)
\item
  \href{https://kanzleikompa.de/2020/02/18/olg-hamburg-deanonymisierung-von-autoren-politischer-beitraege-zulaessig/}{»Deanonymisierung
  von Autoren politischer Beiträge zulässig«} (Kompa, Februar 2020)
\end{itemize}

\hypertarget{weitere-themen}{%
\paragraph{Weitere Themen}\label{weitere-themen}}

\begin{itemize}
\tightlist
\item
  \href{https://swprs.org/propaganda-in-der-wikipedia/}{Propaganda in
  der Wikipedia}
\item
  \href{https://swprs.org/wikipedia-missbrauch-massnahmen/}{WP:
  Maßnahmen bei Missbrauch}
\item
  \href{https://swprs.org/wikipedia-manipulation-autoren/}{WP:
  Ideologisch agierende Autoren}
\end{itemize}

\begin{center}\rule{0.5\linewidth}{\linethickness}\end{center}

Beitrag teilen auf:
\href{https://twitter.com/intent/tweet?url=https://swprs.org/der-wikipedia-prozess/}{Twitter}
/
\href{https://www.facebook.com/share.php?u=https://swprs.org/der-wikipedia-prozess/}{Facebook}

\hypertarget{swiss-policy-research}{%
\subsubsection{Swiss Policy Research}\label{swiss-policy-research}}

\begin{itemize}
\tightlist
\item
  \href{https://swprs.org/kontakt/}{Kontakt}
\item
  \href{https://swprs.org/uebersicht/}{Übersicht}
\item
  \href{https://swprs.org/donationen/}{Donationen}
\item
  \href{https://swprs.org/disclaimer/}{Disclaimer}
\end{itemize}

\hypertarget{english}{%
\subsubsection{English}\label{english}}

\begin{itemize}
\tightlist
\item
  \href{https://swprs.org/contact/}{About Us / Contact}
\item
  \href{https://swprs.org/media-navigator/}{The Media Navigator}
\item
  \href{https://swprs.org/the-american-empire-and-its-media/}{The CFR
  and the Media}
\item
  \href{https://swprs.org/donations/}{Donations}
\end{itemize}

\hypertarget{follow-by-email}{%
\subsubsection{Follow by email}\label{follow-by-email}}

Follow

\href{https://wordpress.com/?ref=footer_custom_com}{WordPress.com}.

\protect\hyperlink{}{Up ↑}

\includegraphics{https://pixel.wp.com/b.gif?v=noscript}
