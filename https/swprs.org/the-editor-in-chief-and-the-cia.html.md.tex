\protect\hyperlink{content}{Skip to content}

\href{https://swprs.org/}{}

\protect\hyperlink{search-container}{Search}

Search for:

\href{https://swprs.org/}{\includegraphics{https://swprs.files.wordpress.com/2020/05/swiss-policy-research-logo-300.png}}

\href{https://swprs.org/}{Swiss Policy Research}

Geopolitics and Media

Menu

\begin{itemize}
\tightlist
\item
  \href{https://swprs.org}{Start}
\item
  \href{https://swprs.org/srf-propaganda-analyse/}{Studien}

  \begin{itemize}
  \tightlist
  \item
    \href{https://swprs.org/srf-propaganda-analyse/}{SRF / ZDF}
  \item
    \href{https://swprs.org/die-nzz-studie/}{NZZ-Studie}
  \item
    \href{https://swprs.org/der-propaganda-multiplikator/}{Agenturen}
  \item
    \href{https://swprs.org/die-propaganda-matrix/}{Medienmatrix}
  \end{itemize}
\item
  \href{https://swprs.org/medien-navigator/}{Analysen}

  \begin{itemize}
  \tightlist
  \item
    \href{https://swprs.org/medien-navigator/}{Navigator}
  \item
    \href{https://swprs.org/der-propaganda-schluessel/}{Techniken}
  \item
    \href{https://swprs.org/propaganda-in-der-wikipedia/}{Wikipedia}
  \item
    \href{https://swprs.org/logik-imperialer-kriege/}{Kriege}
  \end{itemize}
\item
  \href{https://swprs.org/netzwerk-medien-schweiz/}{Netzwerke}

  \begin{itemize}
  \tightlist
  \item
    \href{https://swprs.org/netzwerk-medien-schweiz/}{Schweiz}
  \item
    \href{https://swprs.org/netzwerk-medien-deutschland/}{Deutschland}
  \item
    \href{https://swprs.org/medien-in-oesterreich/}{Österreich}
  \item
    \href{https://swprs.org/das-american-empire-und-seine-medien/}{USA}
  \end{itemize}
\item
  \href{https://swprs.org/bericht-eines-journalisten/}{Fokus I}

  \begin{itemize}
  \tightlist
  \item
    \href{https://swprs.org/bericht-eines-journalisten/}{Journalistenbericht}
  \item
    \href{https://swprs.org/russische-propaganda/}{Russische Propaganda}
  \item
    \href{https://swprs.org/die-israel-lobby-fakten-und-mythen/}{Die
    »Israel-Lobby«}
  \item
    \href{https://swprs.org/geopolitik-und-paedokriminalitaet/}{Pädokriminalität}
  \end{itemize}
\item
  \href{https://swprs.org/migration-und-medien/}{Fokus II}

  \begin{itemize}
  \tightlist
  \item
    \href{https://swprs.org/covid-19-hinweis-ii/}{Coronavirus}
  \item
    \href{https://swprs.org/die-integrity-initiative/}{Integrity
    Initiative}
  \item
    \href{https://swprs.org/migration-und-medien/}{Migration \& Medien}
  \item
    \href{https://swprs.org/der-fall-magnitsky/}{Magnitsky Act}
  \end{itemize}
\item
  \href{https://swprs.org/kontakt/}{Projekt}

  \begin{itemize}
  \tightlist
  \item
    \href{https://swprs.org/kontakt/}{Kontakt}
  \item
    \href{https://swprs.org/uebersicht/}{Seitenübersicht}
  \item
    \href{https://swprs.org/medienspiegel/}{Medienspiegel}
  \item
    \href{https://swprs.org/donationen/}{Donationen}
  \end{itemize}
\item
  \href{https://swprs.org/contact/}{English}
\end{itemize}

\protect\hyperlink{}{Open Search}

\hypertarget{the-editor-in-chief-and-the-cia}{%
\section{The Editor-in-Chief and
the~CIA}\label{the-editor-in-chief-and-the-cia}}

Published: October 2019\\
Other languages:
\href{https://swprs.org/der-chefredakteur-und-die-cia/}{German}

The clandestine cooperation between Western intelligence services and
the media has been
\href{http://carlbernstein.com/magazine_cia_and_media.php}{known} for
decades and is well
\href{https://swprs.org/video-the-cia-and-the-media/}{documented}. The
following case shows just how closely and comprehensively even leading
European journalists have been cooperating with secret services such as
the CIA.

From 1961 to 1989, former Austrian journalist
\href{https://en.wikipedia.org/wiki/Otto_Schulmeister}{Otto
Schulmeister} was editor-in-chief and publisher of
\href{https://en.wikipedia.org/wiki/Die_Presse}{\emph{Die Presse}}, one
of the most renowned daily newspapers in Austria. In 2009, his former
CIA dossier was made public --- with remarkable details about the
decades-long covert collaboration.

In an exclusive article, Austrian news magazine \emph{Profil}
\href{https://www.profil.at/home/ex-presse-chef-dienste-cia-otto-schulmeister-geheimdienst-239634}{revealed}
the following (translated excerpts):

«The Schulmeister dossier testifies to a problematic, essentially
illegal relationship between a journalist, who believed he was
fulfilling a mission, and the CIA: Schulmeister (code name GRCAMERA)
made editorial statements based on the wishes of the CIA, suppressed
stories if they harmed the US position, urged his editors to contact the
representatives of the US government stationed in Vienna for dinner
meetings and revealed information from background talks with Austrian
politicians and ambassadors of Eastern bloc countries.

The recruitment began with the usual inquiries through third parties and
the search for biographical vulnerabilities that could be used when
needed against the person concerned.

From then on, CIA directives on how to assess this or that political
situation went directly to the chief editor's office. If there was a
rush and there was no time for a personal meeting, the documents would
be sent to Schulmeister by courier.

On October 29, 1962, the CIA reported: «Material handed out. An
editorial has appeared according to our instructions.»

On December 28, 1962, it said: «The gentlemen's night has paid off. The
political line of \emph{Die Presse} could hardly be better from our
point of view. () We can place articles anytime. According to
instructions from headquarters, this was requested concerning the Cuba
crisis. () I met GRCAMERA the same day and handed him material that the
head office wants to see published. The story appeared in the Sunday
edition in the form of an editorial signed by him on the front page. ()
GRCAMERA has requested support for his son to receive a Quaker
Scholarship, () it does not look like he will be taken, but this is
still a useful way to strengthen the bond.»

On December 17, 1962, the CIA reported that it was worried about a story
that the \emph{Presse} correspondent in Washington wanted to publish
concerning secret US support for the Austrian army. «The story is
correct, () but should not become public at this time, to avoid an
uproar over Austria's neutrality. () I convinced GRCAMERA that this
would only benefit the Soviets. GRCAMERA agrees not to print the story
() GRCAMERA is pleased to be the first journalist to be received by the
new US Ambassador to Vienna. () It's a scoop for him.»

On September 19, 1963, the CIA wanted to stop the coverage of an
embarrassing espionage affair in the Interior Ministry: «GRCAMERA ()
promised not to publish anything about this case.»

On April 3, 1964, praise again for the content of the newspaper: «it
leaves nothing to be desired.» Many times Schulmeister was even ahead of
the instructions from headquarters. Only the US correspondent of
\emph{Die Presse} causes minor irritations. «That does not mean
Schulmeister is our agent. () But we can lead him just as if he were our
agent ().»

On January 19, 1965, Schulmeister was handed material about the Congo
crisis. «GRCAMERA said he did not need to be convinced of American
interests in the Congo, but () the New York Times recently released a
report on a massacre of rebels there, which was also picked up in
Austria. This puts his newspaper in the uncomfortable situation of
publishing stories that are contrary to the Times version. () GRCAMERA
said we should get the New York Times in line.»

In the course of 1965, the CIA was able to gain another confidential
contact in \emph{Die Presse}, but Schulmeister did not know that, as
reported in the October 12, 1965 report.

In the following years, numerous CIA documents about the war in Vietnam
and other hot spots in US foreign policy went over the desk of the chief
editor in \emph{Die Presse}. Schulmeister often asked for appropriate
analyzes on his own initiative. In 1968, Schulmeister was invited to the
US as part of a Red Carpet Program.

When a policy of détente between the blocks began to emerge in the early
1970s, Schulmeister's relationship with the CIA began to crumble. ()
Schulmeister apparently behaved like a «fleeing bird». The CIA had
already taken sight of a new, supposedly «less evasive» confidant called
IDENTITY in the files. According to the description -- he was a domestic
policy editor, had studied in the US, had a weekend house in Lower
Austria -- they had the future chief editor of \emph{Die Presse}, Thomas
Chorherr, in mind.

Thomas Chorherr told \emph{Profil}: «I've certainly had a lot to do with
secretaries from the US embassy. But that the CIA was involved, I did
not know. I did not suspect it either, and I have a clear conscience.»
He could not believe that Schulmeister had been in contact with the CIA.
«I should have noticed that,» Chorherr said.

According to the official history on the \emph{Die Presse} homepage,
Schulmeister made the newspaper into a »bastion of independent thinking«
in the 1960s.

The Schulmeister files also contain references to CIA confidants in
other well-known Austrian media, including national broadcaster ORF.
Former ORF director Gerd Bacher said he had never received an offer by
the CIA, but
\href{https://www.diepresse.com/471594/otto-schulmeister-in-den-akten-der-cia}{added}:
«If they had asked me, I would have joined in.»

\textbf{See also}:
\href{https://swprs.org/video-the-cia-and-the-media/}{The CIA and the
Media}. A restored 1985 Channel 4 documentary.

\begin{center}\rule{0.5\linewidth}{\linethickness}\end{center}

Authored by the \href{https://swprs.org/contact/}{Swiss Propaganda
Research Group}. Translated by Terje Maloy. The quotes from the CIA file
might have slight variations from the original text due to being
retranslated. This article is Creative Commons 4.0 for non-commercial
purposes.

Original source of published CIA file: Incriminating Otto Schulmeister
of \emph{Die Presse}. The CIA Name File: A Select Edition. In: Journal
for Intelligence, Propaganda and Security Studies 6/1 (2012), 179--207.
(\href{http://www.acipss.org/wp-content/uploads/JIPSS_V6N1_extract.pdf}{Abstract})

\hypertarget{swiss-policy-research}{%
\subsubsection{Swiss Policy Research}\label{swiss-policy-research}}

\begin{itemize}
\tightlist
\item
  \href{https://swprs.org/kontakt/}{Kontakt}
\item
  \href{https://swprs.org/uebersicht/}{Übersicht}
\item
  \href{https://swprs.org/donationen/}{Donationen}
\item
  \href{https://swprs.org/disclaimer/}{Disclaimer}
\end{itemize}

\hypertarget{english}{%
\subsubsection{English}\label{english}}

\begin{itemize}
\tightlist
\item
  \href{https://swprs.org/contact/}{About Us / Contact}
\item
  \href{https://swprs.org/media-navigator/}{The Media Navigator}
\item
  \href{https://swprs.org/the-american-empire-and-its-media/}{The CFR
  and the Media}
\item
  \href{https://swprs.org/donations/}{Donations}
\end{itemize}

\hypertarget{follow-by-email}{%
\subsubsection{Follow by email}\label{follow-by-email}}

Follow

\href{https://wordpress.com/?ref=footer_custom_com}{WordPress.com}.

\protect\hyperlink{}{Up ↑}

\includegraphics{https://pixel.wp.com/b.gif?v=noscript}
