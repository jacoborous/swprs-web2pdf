\protect\hyperlink{content}{Skip to content}

\href{https://swprs.org/}{}

\protect\hyperlink{search-container}{Search}

Search for:

\href{https://swprs.org/}{\includegraphics{https://swprs.files.wordpress.com/2020/05/swiss-policy-research-logo-300.png}}

\href{https://swprs.org/}{Swiss Policy Research}

Geopolitics and Media

Menu

\begin{itemize}
\tightlist
\item
  \href{https://swprs.org}{Start}
\item
  \href{https://swprs.org/srf-propaganda-analyse/}{Studien}

  \begin{itemize}
  \tightlist
  \item
    \href{https://swprs.org/srf-propaganda-analyse/}{SRF / ZDF}
  \item
    \href{https://swprs.org/die-nzz-studie/}{NZZ-Studie}
  \item
    \href{https://swprs.org/der-propaganda-multiplikator/}{Agenturen}
  \item
    \href{https://swprs.org/die-propaganda-matrix/}{Medienmatrix}
  \end{itemize}
\item
  \href{https://swprs.org/medien-navigator/}{Analysen}

  \begin{itemize}
  \tightlist
  \item
    \href{https://swprs.org/medien-navigator/}{Navigator}
  \item
    \href{https://swprs.org/der-propaganda-schluessel/}{Techniken}
  \item
    \href{https://swprs.org/propaganda-in-der-wikipedia/}{Wikipedia}
  \item
    \href{https://swprs.org/logik-imperialer-kriege/}{Kriege}
  \end{itemize}
\item
  \href{https://swprs.org/netzwerk-medien-schweiz/}{Netzwerke}

  \begin{itemize}
  \tightlist
  \item
    \href{https://swprs.org/netzwerk-medien-schweiz/}{Schweiz}
  \item
    \href{https://swprs.org/netzwerk-medien-deutschland/}{Deutschland}
  \item
    \href{https://swprs.org/medien-in-oesterreich/}{Österreich}
  \item
    \href{https://swprs.org/das-american-empire-und-seine-medien/}{USA}
  \end{itemize}
\item
  \href{https://swprs.org/bericht-eines-journalisten/}{Fokus I}

  \begin{itemize}
  \tightlist
  \item
    \href{https://swprs.org/bericht-eines-journalisten/}{Journalistenbericht}
  \item
    \href{https://swprs.org/russische-propaganda/}{Russische Propaganda}
  \item
    \href{https://swprs.org/die-israel-lobby-fakten-und-mythen/}{Die
    »Israel-Lobby«}
  \item
    \href{https://swprs.org/geopolitik-und-paedokriminalitaet/}{Pädokriminalität}
  \end{itemize}
\item
  \href{https://swprs.org/migration-und-medien/}{Fokus II}

  \begin{itemize}
  \tightlist
  \item
    \href{https://swprs.org/covid-19-hinweis-ii/}{Coronavirus}
  \item
    \href{https://swprs.org/die-integrity-initiative/}{Integrity
    Initiative}
  \item
    \href{https://swprs.org/migration-und-medien/}{Migration \& Medien}
  \item
    \href{https://swprs.org/der-fall-magnitsky/}{Magnitsky Act}
  \end{itemize}
\item
  \href{https://swprs.org/kontakt/}{Projekt}

  \begin{itemize}
  \tightlist
  \item
    \href{https://swprs.org/kontakt/}{Kontakt}
  \item
    \href{https://swprs.org/uebersicht/}{Seitenübersicht}
  \item
    \href{https://swprs.org/medienspiegel/}{Medienspiegel}
  \item
    \href{https://swprs.org/donationen/}{Donationen}
  \end{itemize}
\item
  \href{https://swprs.org/contact/}{English}
\end{itemize}

\protect\hyperlink{}{Open Search}

\hypertarget{faktoja-covid-19sta-suomeksi}{%
\section{Faktoja Covid-19:sta
(suomeksi)}\label{faktoja-covid-19sta-suomeksi}}

\textbf{Päivitetty}: 18. toukokuuta 2020; \textbf{Jaa}:
\href{https://twitter.com/intent/tweet?url=https://swprs.org/faktoja-covid-19sta/}{Twitter}
/
\href{https://www.facebook.com/share.php?u=https://swprs.org/faktoja-covid-19sta/}{Facebook}\\
\textbf{Kielet}: \href{https://swprs.org/fakta-o-covid-19/}{CZ},
\href{https://swprs.org/covid-19-hinweis-ii/}{DE},
\href{https://swprs.org/a-swiss-doctor-on-covid-19/}{EN},
\href{https://swprs.org/coronavirus-un-medecin-suisse-parle/}{FR},
\href{https://swprs.org/hechos-sobre-covid-19/}{ES},
\href{https://swprs.org/faktoja-covid-19sta/}{FI},
\href{https://swprs.org/covid-19-cinjenice/}{HBS},
\href{https://yanivhamo.com/facts-about-covid-19-hebrew/}{HE},
\href{https://swprs.org/egy-svajci-orvos-a-covid-19-rol/}{HU},
\href{https://swprs.org/un-medico-svizzero-su-covid-19/}{IT},
\href{https://swprs.org/covid19-facts-japanese/}{JP},
\href{https://swprs.org/covid19-korean/}{KO},
\href{https://www.globalinfo.nl/Achtergrond/een-kritische-kijk-op-het-coronabeleid-transparantie-in-tijden-van-crisis}{NL},
\href{https://midtifleisen.wordpress.com/2020/04/15/fakta-om-covid-19/}{NO},
\href{https://swprs.org/szwajcarski-lekarz-o-covid-19/}{PL},
\href{https://swprs.org/fatos-sobre-covid-19/}{PT},
\href{https://swprs.org/informatii-despre-covid-19/}{RO},
\href{https://swprs.org/\%d0\%bd\%d0\%b0-\%d0\%ba\%d0\%be\%d0\%b2\%d0\%b8\%d0\%b4-19/}{RU},
\href{https://swprs.org/fakta-om-covid-19/}{SE},
\href{http://www.ninamvseeno.org/pregled-clanka.aspx?naslov=pomembne-informacije-o-novem-koronavirusu-covid-19\&id=148}{SI},
\href{https://alatyr.sk/covid-19_swiss_propaganda_research.htm}{SK},
\href{https://swprs.org/isvicreli-bir-doktordan-kovid-19-uezerine/}{TR}

Alan asiantuntijoiden koostamat ja täysin lähdeviite merkityt faktat
Covid-19:sta, jotka auttavat meidän lukijoita tekemään realistisen
riskinarvioinnin. (Säännölliset päivitykset alla)

\textbf{``Ainoa keino torjua ruttoa on rehellisyys.'' Albert Camus,
rutto (1947)}

\hypertarget{yleiskatsaus}, joka
  on samoissa rajoissa vakavan
  \href{https://www.ebm-netzwerk.de/en/publications/covid-19}{influenssan}
  kanssa ja noin kaksikymmentä kertaa matalampi kuin WHO alun perin
  \href{https://www.businessinsider.com/coronavirus-death-rate-by-age-countries-2020-3?r=US\&IR=T}{oletti}.
\item
  Jopa globaaleissa ``hotspoteissa'', koulu- ja työikäisen väestön
  kuolemanriski on samoissa lukemissa kuin riski kuolla
  \href{https://www.medrxiv.org/content/10.1101/2020.04.05.20054361v1}{päivittäisellä
  ajomatkalla työn ja kodin välillä}. Riski alun perin yliarvioitiin,
  koska monia ihmisiä, joilla oli vain lieviä oireita tai joilla ei
  ollut oireita, ei otettu huomioon.
\item
  80\% henkilöistä, jotka on testattu positiiviseksi,
  \href{https://www.bmj.com/content/369/bmj.m1375}{pysyy oireettomana}.
  Myös 70-79-vuotiaista jopa
  \href{https://www.niid.go.jp/niid/en/2019-ncov-e/9407-covid-dp-fe-01.html}{noin
  60\%} pysyvät oireettomina. Yli 95 prosentilla kaikista henkilöistä on
  enintään
  \href{https://swprs.org/studies-on-covid-19-lethality/\#hospitalizations}{lieviä
  oireita}.
\item
  Jopa kolmannella kaikista henkilöistä on jo tietty
  \href{https://www.medrxiv.org/content/10.1101/2020.04.17.20061440v1}{immuniteetti}
  taustalla Covid-19 suhteen johtuen aikaisemmista koronaviruksista (ts.
  tavallisista kylmä viruksista).
\item
  Kuolleiden mediaani- tai keski-ikä on useimmissa maissa (myös
  \href{https://www.epicentro.iss.it/coronavirus/sars-cov-2-decessi-italia}{Italiassa})
  yli 80 vuotta, ja vain noin
  \href{https://www.bloomberg.com/news/articles/2020-03-18/99-of-those-who-died-from-virus-had-other-illness-italy-says}{yhdellä
  prosentilla}kuolleista ei ollut aiempia vakavia sairauksia. Kuolemien
  ikä- ja riskiprofiili vastaa siis olennaisesti
  \href{https://www.vienna.at/analyse-zeigt-covid-19-opferkurve-entspricht-normaler-mortalitaet/6581246}{normaalia
  kuolleisuutta}.
\item
  Suurimmassa osassa maita 50 -- 70 \% kaikista extra kuolemista on
  tapahtunut
  \href{https://ltccovid.org/2020/04/12/mortality-associated-with-covid-19-outbreaks-in-care-homes-early-international-evidence/}{hoitokodeissa},
  jotka eivät hyödy yleisestä lockdownista. Lisäksi monissa tapauksissa
  \href{https://www.hsj.co.uk/commissioning/thousands-of-extra-deaths-outside-hospital-not-attributed-to-covid-19/7027459.article}{ei
  ole selvää}ovatko nämä ihmiset todella kuolleet Covid-19 vai
  \href{https://www.nytimes.com/2020/04/16/world/canada/montreal-nursing-homes-coronavirus.html}{äärimmäisen
  stressin}, pelon ja yksinäisyyden takia.
\item
  Jopa 50\% kaikista lisäkuolemista on saattanut johtua
  \href{https://www.thetimes.co.uk/edition/news/coronavirus-record-weekly-death-toll-as-fearful-patients-avoid-hospitals-bm73s2tw3}{Covid-19}
  sijaan
  \href{https://www.telegraph.co.uk/global-health/science-and-disease/two-new-waves-deaths-break-nhs-new-analysis-warns/}{lockdownin,
  paniikin ja pelon} vaikutuksista. Esimerkiksi sydänkohtausten ja
  aivohalvausten hoito
  \href{https://www.nytimes.com/2020/04/06/well/live/coronavirus-doctors-hospitals-emergency-care-heart-attack-stroke.html}{on
  laskenut} jopa 60\%, koska monet potilaat eivät enää uskalla mennä
  sairaalaan.
\item
  Jopa ns. ``Covid19 -kuolemissa''
  \href{https://spectator.us/understand-report-figures-covid-deaths/}{ei
  usein ole selvää}, kuolevatko ihmiset koronaviruksen takia vai sen
  kanssa (ts. taustalla oleviin sairauksiin) tai laskettiinko ne
  \href{https://www.youtube.com/watch?v=V0lIWZpiRU0}{``oletettuihin
  tapauksiin''} jolloin koronaa ei testata lainkaan. Viralliset luvut
  \href{https://www.hsj.co.uk/coronavirus/systematic-reviews-to-discover-true-cause-of-outbreak-deaths/7027491.article}{eivät
  yleensä heijasta} tätä erottelua lainkaan.
\item
  ~Monet mediassa raportoidut nuorten ja terveiden ihmisten Covid-19
  kuolemat ovat osoittautuneet vääriksi; monet näistä nuorista joko
  \href{https://www.dailymail.co.uk/news/article-8193487/Coroner-refuses-rule-COVID-19-cause-death-six-week-old-Connecticut-baby.html}{eivät
  kuolleet} Covid-19 takia, he olivat jo
  \href{https://sports.yahoo.com/spanish-football-coach-francisco-garcia-163153573.html}{vakavasti
  sairaita} (esim. diagnosoimaton leukemia) tai he olivat itse asiassa
  \href{https://www.tagesanzeiger.ch/bund-muss-in-seiner-todesfallstatistik-fehler-korrigieren-584308129723}{109-vuotta
  vanhoja eikä 9-vuotiaita}.
\item
  Normaali kokonaiskuolleisuus päivässä on USA:ssa on n.
  \href{https://www.cdc.gov/mmwr/volumes/68/wr/mm6826a5.htm}{8000
  ihmistä}, Saksassa n. 2600 ihmistä ja Italiassa n. 1800 ihmistä.
  Influenssakuolleisuus kausittain on USA:ssa jopa
  \href{https://www.statnews.com/2018/09/26/cdc-us-flu-deaths-winter/}{80
  000} ja jopa
  \href{https://www.sciencedirect.com/science/article/pii/S1201971219303285}{25
  000} Saksassa sekä Italiassa. Useissa maissa Covid19 kuolemat ovat
  pysyneet \href{https://www.euromomo.eu/graphs-and-maps}{vahvojen
  flunssakausien alapuolella}.
\item
  Kuolleisuuden alueelliseen kasvuun ovat voineet vaikuttaa muut
  riskitekijät, kuten
  \href{https://www.theguardian.com/environment/2020/apr/20/air-pollution-may-be-key-contributor-to-covid-19-deaths-study?utm_mediumhttps://www.theguardian.com/environment/2020/apr/20/air-pollution-may-be-key-contributor-to-covid-19-deaths-study?utm_medium}{korkea
  ilmansaaste} ja
  \href{https://www.ansa.it/english/news/science_tecnology/2019/11/19/italy-top-in-eu-in-antibiotic-resistance_369e0123-0107-445e-8c17-f11932c9d27c.html}{mikrobikontaminaatio},
  samoin kuin
  \href{https://swprs.org/covid-19-a-report-from-italy/}{vanhusten ja
  sairaiden hoidon romahtaminen} infektioiden, massan paniikin ja
  lockdownin vuoksi.
  \href{https://www.ecdc.europa.eu/sites/default/files/documents/COVID-19-safe-handling-of-bodies-or-persons-dying-from-COVID19.pdf}{Erityissäädökset}
  tavoissa hoitaa kuolleiden järjestelyt ovat toisinaan johtaneet
  pullonkauloihin hautajais- ja tuhkauspalveluissa.
\item
  Italian ja Espanjan kaltaisissa maissa, sekä jossain määrin
  Isossa-Britanniassa ja Yhdysvalloissa, sairaaloiden ylikuormitus
  vahvojen flunssa-aaltojen aikana
  \href{https://off-guardian.org/2020/04/02/coronavirus-fact-check-1-flu-doesnt-overwhelm-our-hospitals/}{ei
  ole epätavallista}. Lisäksi, jopa 15 \% lääkäreistä ja
  terveydenhoitajista
  \href{https://www.reuters.com/article/us-health-coronavirus-spain-morgue-idUSKBN21B1PP}{on
  laitettu karanteeniin}, vaikka heillä ei olisikaan oireita.
\item
  Usein esitetyt eksponentiaalikäyrät ``korona-tapauksista'' ovat
  \href{https://fivethirtyeight.com/features/coronavirus-case-counts-are-meaningless/}{harhaanjohtavia},
  koska myös testien määrä on kasvanut eksponentiaalisesti. Useimmissa
  maissa, positiivisten testien suhde kokonaistesteihin on pysynyt
  \href{https://swprs.org/rate-of-positive-covid19-tests/}{vakiona
  välillä 5 -- 25\%} tai kasvaa melko hitaasti. Monissa maissa
  leviämisen huippu saavutettiin jo hyvissä ajoin
  \href{https://www.dailymail.co.uk/news/article-8235979/UKs-coronavirus-crisis-peaked-lockdown-Expert-argues-draconian-measures-unnecessary.html}{ennen
  lockdownia}.
\item
  Maat, joissa ei ole ulkonaliikkumis- ja kontakti-kieltoja, kuten
  \href{https://www.japantimes.co.jp/news/2020/03/20/national/coronavirus-explosion-expected-japan/}{Japani},
  \href{https://www.businessinsider.com/south-korea-coronavirus-testing-death-rate-2020-3?op=1\&r=US\&IR=T}{Etelä-Korea}
  ja \href{https://www.youtube.com/watch?v=bfN2JWifLCY}{Ruotsi},
  \href{https://www.washingtontimes.com/news/2020/apr/15/sweden-coronavirus-rates-easing-despite-loose-rule/}{eivät
  ole kokeneet} negatiivisempia tapahtumia kuin muut maat. WHO jopa
  \href{https://nypost.com/2020/04/29/who-lauds-sweden-as-model-for-resisting-coronavirus-lockdown/}{kehui}
  Ruotsia, ja nyt se hyötyy korkeammasta immuniteetista verrattuna
  lockdown-maihin.
\item
  Pelko hengityslaitteiden puutteesta oli
  \href{https://apnews.com/8ccd325c2be9bf454c2128dcb7bd616d}{perusteeton}.
  Keuhkoasiantuntijoiden mukaan Covid19-potilaiden invasiivinen tuuletus
  (intubaatio), jotaa tehdään osittain viruksen leviämisen
  \href{https://www.dailymail.co.uk/news/article-8262351/Nurse-New-York-claims-city-killing-COVID-19-patients-putting-ventilators.html}{pelon
  takia}, on usein
  \href{https://www.medscape.com/viewarticle/928156}{haitallista} ja
  aiheuttaa vaurioita keuhkoihin.
\item
  Vastoin alkuperäisiä oletuksia, useat tutkimukset ovat osoittaneet,
  että
  \href{https://www.who.int/news-room/commentaries/detail/modes-of-transmission-of-virus-causing-covid-19-implications-for-ipc-precaution-recommendations}{ei
  ole todisteita} viruksen leviämisestä aerosolien (eli ilmassa olevien
  hiukkasten) tai
  \href{https://www.telegraph.co.uk/news/2020/04/02/no-proof-coronavirus-can-spread-shopping-says-leading-german/}{saastuneiden
  pintojen} (kuten ovenkahvojen, älypuhelinten tai hiustenkuivaajien)
  kautta.
\item
  Ei ole myöskään
  \href{https://www.researchgate.net/publication/340570735_Masks_Don't_Work_A_review_of_science_relevant_to_COVID-19_social_policy}{tieteellistä
  näyttöä} kasvonaamarien tehokkuudesta terveillä tai oireettomilla
  henkilöillä. Päinvastoin, asiantuntijat varoittavat, että tällaiset
  naamarit häiritsevät normaalia hengitystä ja saattavat tulla
  \href{https://de.sputniknews.com/interviews/20200425326953541-corona-gefahr-virologe/}{``bakteereita
  kantaviksi''}. Johtavat lääkärit kutsuivat niitä ''mediahypeksi'' ja
  \href{https://www.aerztezeitung.de/Politik/Montgomery-haelt-Maskenpflicht-fuer-falsch-408844.html}{``naurettaviksi''}.
\item
  Monet klinikat Euroopassa ja Yhdysvalloissa ovat olleet
  \href{https://www.hsj.co.uk/acute-care/nhs-hospitals-have-four-times-more-empty-beds-than-normal/7027392.article}{vahvasti
  vajaakäytössä} tai melkein tyhjiä Covid19-huipun aikana, ja joissakin
  tapauksissa heidän on pitänyt
  \href{https://eu.usatoday.com/story/news/health/2020/04/02/coronavirus-pandemic-jobs-us-health-care-workers-furloughed-laid-off/5102320002/}{lähettää
  henkilökunta kotiin}. Lukuisia leikkauksia ja hoitoja
  \href{https://www.sfchronicle.com/bayarea/article/Stanford-hospital-system-to-cut-pay-20-furlough-15227591.php}{on
  peruttu}, mukaan lukien jotkin elinsiirrot ja syöpäseulonnat.
\item
  Useat mediat ovat
  \href{https://nypost.com/2020/04/01/cbs-admits-to-using-footage-from-italy-in-report-about-nyc/}{jääneet
  kiinni}tilanteiden dramatisoinnista sairaaloissa, joskus jopa
  manipuloivilla kuvilla ja videoilla. Yleisesti, monien medioiden
  \href{https://onlinelibrary.wiley.com/doi/full/10.1111/eci.13222}{epäammattimainen
  raportointi} on maksimoinut väestön pelon ja paniikin.
\item
  Kansainvälisesti käytetyt virustestit ovat
  \href{https://www.ncbi.nlm.nih.gov/pubmed/32219885}{alttiita
  virheille} ja voivat antaa väärän positiivisen tai väärän negatiivisen
  tuloksen. Lisäksi virallista virustestiä
  \href{https://www.youtube.com/watch?v=p_AyuhbnPOI}{ei validoitu
  kliinisesti} aikapaineen takia, ja se saattaa joskus reagoida muihin
  koronaviruksiin.
\item
  Lukuisat kansainvälisesti tunnetut virologian, immunologian ja
  epidemiologian
  \href{https://off-guardian.org/2020/03/24/12-experts-questioning-the-coronavirus-panic/}{asiantuntijat}
  pitävät toteutettuja toimenpiteitä
  \href{https://off-guardian.org/2020/03/28/10-more-experts-criticising-the-coronavirus-panic/}{haitallisina}
  ja suosittelevat väestön nopeaa
  \href{https://off-guardian.org/2020/04/17/8-more-experts-questioning-the-coronavirus-panic/}{luonnollista
  immunisointia} ja riskiryhmien suojaamista. Riskit lapsille ovat
  \href{https://www.thelancet.com/journals/lanchi/article/PIIS2352-4642(20)30095-X/fulltext}{käytännössä
  nolla} eikä koulujen sulkeminen ollut koskaan lääketieteellisesti
  perusteltua.
\item
  Useat lääketieteen asiantuntijat ovat kuvanneet koronavirusten
  vastaisia rokotteita
  \href{https://www.youtube.com/watch?v=vrL9QKGQrWk}{tarpeettomiksi} tai
  jopa
  \href{https://www.nature.com/articles/d41586-020-00751-9}{vaarallisiksi}.
  Itse asiassa, esimerkiksi vuoden 2009
  \href{https://www.forbes.com/2010/02/05/world-health-organization-swine-flu-pandemic-opinions-contributors-michael-fumento.html\#2bfd64ff48e8}{ns.
  sikainfluenssarokote} johti toisinaan
  \href{https://www.ibtimes.co.uk/brain-damaged-uk-victims-swine-flu-vaccine-get-60-million-compensation-1438572}{vakaviin
  neurologisiin vahinkoihin} ja miljoonien oikeusjuttuihin.
\item
  Toimenpiteiden seurauksena työttömyydestä,
  \href{https://eu.indystar.com/story/news/health/2020/04/03/coronavirus-indiana-how-get-help-mental-health-addiction/5104357002/}{psykologisista
  ongelmista} ja perheväkivallasta kärsivien ihmisten määrä on
  \href{https://www.reuters.com/article/us-health-coronavirus-usa-layoffs/us-weekly-jobless-claims-seen-at-record-high-again-idUSKBN21K0FX}{kasvanut
  todella nopeasti koko maailmassa}. Useat asiantuntijat uskovat, että
  toimenpiteet saattavat vaatia
  \href{https://www.nytimes.com/2020/03/20/opinion/coronavirus-pandemic-social-distancing.html}{enemmän
  ihmishenkiä} kuin virus itse. YK:n mukaan miljoonat ihmiset ympäri
  maailmaa voivat joutua täydelliseen köyhyyteen ja nälänhätään.
\item
  NSA:n ilmiantaja Edward Snowden varoitti, että koronakriisiä käytetään
  maailmanlaajuisen seurannan
  \href{https://www.youtube.com/watch?v=-pcQFTzck_c}{massiiviseen ja
  pysyvään laajentamiseen}. Tunnettu virologi Pablo Goldschmidt
  \href{https://www.rubikon.news/artikel/der-corona-totalitarismus}{puhui}
  ''globaalista mediaterrorista'' ja ''totalitaarisista
  toimenpiteistä''. Johtava brittiläinen virologian professori John
  Oxford
  \href{https://novuscomms.com/2020/03/31/a-view-from-the-hvivo-open-orphan-orph-laboratory-professor-john-oxford/}{puhui}''mediaepidemiasta''.
\item
  Yli 500 tutkijaa
  \href{https://www.esat.kuleuven.be/cosic/sites/contact-tracing-joint-statement/}{on
  varoittanut} ``ennennäkemättömästä yhteiskunnan seurannasta''
  ongelmallisten sovellusten avulla ``kontaktien jäljittämistä'' varten.
  Joissakin maissa,
  \href{https://www.jewishpress.com/news/the-courts/state-to-high-court-even-more-shin-bet-involvement-in-fighting-the-coronavirus/2020/04/14/}{salainen
  palvelu} suorittaa jo tällaista ``kontaktien jäljittämistä''. Monissa
  paikoissa maailmassa, jo nyt väestöä
  \href{https://off-guardian.org/2020/04/25/50-headlines-darker-more-of-the-new-normal/}{tarkkaillaan
  droneilla} ja väestö on kohdannut poliisin liiallista voimankäyttöä.
\end{enumerate}

\hypertarget{katso-myuxf6s}{%
\paragraph{Katso myös}\label{katso-myuxf6s}}

\begin{itemize}
\tightlist
\item
  \href{https://swprs.org/open-letter-from-professor-sucharit-bhakdi-to-german-chancellor-dr-angela-merkel/}{Professori
  Bhakdin avoin kirje}
\item
  \href{https://swprs.org/koronasta-mediasta-ja-propagandasta/}{Koronasta,
  mediasta ja propagandasta}
\item
  \href{https://swprs.org/a-swiss-doctor-on-covid-19/}{Lisätietoja
  englanniksi}
\end{itemize}

\hypertarget{ruotsi-media-vs-todellisuus}{%
\paragraph{\texorpdfstring{\textbf{Ruotsi: Media vs.
todellisuus}}{Ruotsi: Media vs. todellisuus}}\label{ruotsi-media-vs-todellisuus}}

Entinen Ruotsin ja Euroopan pääepidemiologi professori Johan Giesecke
antoi itävaltalaiselle Addendum-lehdelle
\href{https://www.addendum.org/coronavirus/interview-johan-giesecke/}{haastattelun}.
Professori Giesecke toteaa, että 75-90\% epidemiasta on ``näkymätöntä'',
koska monilla ihmisillä ei esiinny oireita tai ne ovat lähes
olemattomia. Siksi sulkeminen olisi ``turhaa'' ja vahingoittaisi
yhteiskuntaa. Ruotsin strategian lähtökohtana oli, että ''ihmiset eivät
ole tyhmiä''. Giesecke arvioi kuolleisuuden olevan 0,1 -- 0,2\%,
samanlainen kuin influenssalla. Professori Giesecken mukaan, Italia ja
New York ovat olleet erittäin heikosti valmistautuneita virukseen
eivätkä ole suojelleet riskiryhmiä.

Jotkut lukijat olivat yllättyneitä kuolemien vähentymisestä Ruotsissa,
koska useimmat tiedotusvälineet kuvaavat käyvän nousevan jyrkästi. Mikä
on syy tähän? Useimmat tiedotusvälineet esittävät kumulatiiviset luvut
raportointipäivän mukaan, kun taas Ruotsin viranomaiset julkaisevat
huomattavasti merkityksellisemmät luvut, eli päivämäärät
kuolemantapahtumien mukaan.

Ruotsin viranomaiset korostavat aina, että kaikki uudet ilmoitetut
tapaukset eivät ole kuolleet viimeisen 24 tunnin aikana, mutta monet
tiedotusvälineet jättävät tämän huomioimatta (ks. alla oleva kaavio).
Vaikka viimeisimmät Ruotsin luvut saattavat edelleen nousta jonkin
verran, kuten kaikissa maissa, tämä ei muuta yleisesti laskevaa trendiä.

Lisäksi nämä luvut edustavat kuolemia koronaviruksen kanssa eikä
välttämättä siitä johtuen. Ruotsissa keskimääräinen kuolinikä on myös
yli 80 vuotta ja noin 50\% kuolemista tapahtui haavoittuvissa
hoitokodeissa, kun taas vaikutus väestöön on pysynyt vähäisenä, vaikka
\href{https://link.springer.com/article/10.1007/s00134-012-2627-8}{Ruotsissa
on yksi Euroopan alhaisimmista tehohoidon kyvyistä}.

Ruotsin hallitukselle on kuitenkin annettu myös
\href{https://www.tagesschau.de/faktenfinder/ausland/corona-kursaenderung-schweden-103.html}{uusia
hätä valtuuksia} koronan vuoksi, ja se voi silti osallistua myöhempiin
tartuntojen jäljitys ohjelmiin.

\includegraphics{https://swprs.files.wordpress.com/2020/04/sweden-corona-media-vs-reality.png?w=736\&h=338}

\hypertarget{covid-19-ja-media}{%
\paragraph{Covid-19 ja media}\label{covid-19-ja-media}}

Moni ihminen on järkyttynyt useiden medioiden tavasta raportoida
Covid-19:sta spekulatiivisesti ja usein pelkoa lietsovasti. Kyse ei
selvästikään ole ``tavallisesta raportoinnista'', vaan klassisesta ja
massiivisesta propagandasta, jota käytetään tyypillisesti
\href{https://swprs.org/propaganda-in-the-war-on-yugoslavia/}{laittomien
hyökkäyssotien} tai
\href{https://www.motherjones.com/politics/2013/01/terror-factory-fbi-trevor-aaronson-book/}{väitetyn
terrorismin} yhteydessä.

Swiss Propaganda Research on kuvannut medioiden taustalla olevia
rakenteita, jotka vastaavat tällaisen propagandan levittämisestä
aiemmissa infografiikoissa
\href{https://swprs.org/the-american-empire-and-its-media/}{Yhdysvaltojen},
\href{https://swprs.org/netzwerk-medien-deutschland/}{Saksan} ja
\href{https://swprs.org/netzwerk-medien-schweiz/}{Sveitsin} osalta. Jopa
oletettu ''avoin'' Internet-sivusto Wikipedia on olennainen osa tätä
geopoliittista mediarakennetta.

Eri medioiden poliittista kantaa ja suhdetta valtaan on analysoitu Swiss
Propaganda Researchin rakentamalla
\href{https://swprs.org/media-navigator/}{Media Navigaattorilla}. Media
Navigaattorin käytöstä voi olla apua myös arvioitaessa nykyistä
Covid19-raportointia eri uutismedioiden välillä.

Jos televisiossa näytetään esimerkiksi sotilaiden kuvia suojapuvuissa,
jotka desinfioivat kokonaisia katuja, se ei todista koronaviruksen
vaaraa, vaan -- kuten professori Giesecke
\href{https://www.addendum.org/coronavirus/interview-johan-giesecke/}{hyväntahtoisesti
totesi} -- se osoittaa hyödytöntä ``poliittista aktivismia''. Tai kuten
muut sanovat: propagandaa.

\hypertarget{lisuxe4tietoja-englanniksi}{%
\paragraph{\texorpdfstring{\href{https://swprs.org/a-swiss-doctor-on-covid-19/}{Lisätietoja
englanniksi}}{Lisätietoja englanniksi}}\label{lisuxe4tietoja-englanniksi}}

\begin{center}\rule{0.5\linewidth}{\linethickness}\end{center}

Share this on:
\href{https://twitter.com/intent/tweet?url=https://swprs.org/faktoja-covid-19sta/}{Twitter}
/
\href{https://www.facebook.com/share.php?u=https://swprs.org/faktoja-covid-19sta/}{Facebook}

\hypertarget{swiss-policy-research}{%
\subsubsection{Swiss Policy Research}\label{swiss-policy-research}}

\begin{itemize}
\tightlist
\item
  \href{https://swprs.org/kontakt/}{Kontakt}
\item
  \href{https://swprs.org/uebersicht/}{Übersicht}
\item
  \href{https://swprs.org/donationen/}{Donationen}
\item
  \href{https://swprs.org/disclaimer/}{Disclaimer}
\end{itemize}

\hypertarget{english}{%
\subsubsection{English}\label{english}}

\begin{itemize}
\tightlist
\item
  \href{https://swprs.org/contact/}{About Us / Contact}
\item
  \href{https://swprs.org/media-navigator/}{The Media Navigator}
\item
  \href{https://swprs.org/the-american-empire-and-its-media/}{The CFR
  and the Media}
\item
  \href{https://swprs.org/donations/}{Donations}
\end{itemize}

\hypertarget{follow-by-email}{%
\subsubsection{Follow by email}\label{follow-by-email}}

Follow

\href{https://wordpress.com/?ref=footer_custom_com}{WordPress.com}.

\protect\hyperlink{}{Up ↑}

Post to

\protect\hyperlink{}{Cancel}

\includegraphics{https://pixel.wp.com/b.gif?v=noscript}
