\protect\hyperlink{content}{Skip to content}

\href{https://swprs.org/}{}

\protect\hyperlink{search-container}{Search}

Search for:

\href{https://swprs.org/}{\includegraphics{https://swprs.files.wordpress.com/2020/05/swiss-policy-research-logo-300.png}}

\href{https://swprs.org/}{Swiss Policy Research}

Geopolitics and Media

Menu

\begin{itemize}
\tightlist
\item
  \href{https://swprs.org}{Start}
\item
  \href{https://swprs.org/srf-propaganda-analyse/}{Studien}

  \begin{itemize}
  \tightlist
  \item
    \href{https://swprs.org/srf-propaganda-analyse/}{SRF / ZDF}
  \item
    \href{https://swprs.org/die-nzz-studie/}{NZZ-Studie}
  \item
    \href{https://swprs.org/der-propaganda-multiplikator/}{Agenturen}
  \item
    \href{https://swprs.org/die-propaganda-matrix/}{Medienmatrix}
  \end{itemize}
\item
  \href{https://swprs.org/medien-navigator/}{Analysen}

  \begin{itemize}
  \tightlist
  \item
    \href{https://swprs.org/medien-navigator/}{Navigator}
  \item
    \href{https://swprs.org/der-propaganda-schluessel/}{Techniken}
  \item
    \href{https://swprs.org/propaganda-in-der-wikipedia/}{Wikipedia}
  \item
    \href{https://swprs.org/logik-imperialer-kriege/}{Kriege}
  \end{itemize}
\item
  \href{https://swprs.org/netzwerk-medien-schweiz/}{Netzwerke}

  \begin{itemize}
  \tightlist
  \item
    \href{https://swprs.org/netzwerk-medien-schweiz/}{Schweiz}
  \item
    \href{https://swprs.org/netzwerk-medien-deutschland/}{Deutschland}
  \item
    \href{https://swprs.org/medien-in-oesterreich/}{Österreich}
  \item
    \href{https://swprs.org/das-american-empire-und-seine-medien/}{USA}
  \end{itemize}
\item
  \href{https://swprs.org/bericht-eines-journalisten/}{Fokus I}

  \begin{itemize}
  \tightlist
  \item
    \href{https://swprs.org/bericht-eines-journalisten/}{Journalistenbericht}
  \item
    \href{https://swprs.org/russische-propaganda/}{Russische Propaganda}
  \item
    \href{https://swprs.org/die-israel-lobby-fakten-und-mythen/}{Die
    »Israel-Lobby«}
  \item
    \href{https://swprs.org/geopolitik-und-paedokriminalitaet/}{Pädokriminalität}
  \end{itemize}
\item
  \href{https://swprs.org/migration-und-medien/}{Fokus II}

  \begin{itemize}
  \tightlist
  \item
    \href{https://swprs.org/covid-19-hinweis-ii/}{Coronavirus}
  \item
    \href{https://swprs.org/die-integrity-initiative/}{Integrity
    Initiative}
  \item
    \href{https://swprs.org/migration-und-medien/}{Migration \& Medien}
  \item
    \href{https://swprs.org/der-fall-magnitsky/}{Magnitsky Act}
  \end{itemize}
\item
  \href{https://swprs.org/kontakt/}{Projekt}

  \begin{itemize}
  \tightlist
  \item
    \href{https://swprs.org/kontakt/}{Kontakt}
  \item
    \href{https://swprs.org/uebersicht/}{Seitenübersicht}
  \item
    \href{https://swprs.org/medienspiegel/}{Medienspiegel}
  \item
    \href{https://swprs.org/donationen/}{Donationen}
  \end{itemize}
\item
  \href{https://swprs.org/contact/}{English}
\end{itemize}

\protect\hyperlink{}{Open Search}

\hypertarget{migration-und-medien}{%
\section{Migration und Medien}\label{migration-und-medien}}

Worum geht es bei der Migration nach Europa, und warum wird sie von den
etablierten Medien zumeist
\href{https://www.otto-brenner-stiftung.de/wissenschaftsportal/informationsseiten-zu-studien/studien-2017/die-fluechtlingskrise-in-den-medien/}{begrüßt},
während ihre
\href{https://www.heise.de/tp/features/Massenwanderungen-haben-sowohl-in-den-Herkunftslaendern-als-auch-den-Ziellaendern-der-Migranten-4205760.html?seite=all}{Ursachen
und Folgen} kaum kritisch hinterfragt werden?

Es geht um Geoökonomie: Ab circa 2025 wird die Bevölkerung in den
reichen Ländern erstmals seit der Industriellen Revolution
\href{https://econimica.blogspot.com/2019/07/2019-un-population-prospects-highlights.html}{insgesamt
schrumpfen}, wodurch das auf anhaltendes Wachstum angewiesene
Wirtschafts­system und die zugehörigen Finanzmärkte innerhalb kurzer
Zeit kollabieren würden.

Um dies zu verhindern und zugleich mit Blick auf die Globalisierung die
traditionellen, kleinräumig-national­­staatlichen Strukturen zu
überwinden, wurden Mechanismen zur raschen Aufnahme von mehreren
Millionen Migranten insbesondere aus dem afrikanischen und arabischen
Raum geschaffen.

Tatsächlich war der ehemalige UNO-Mi­gra­tions­beauftragte Peter
Sutherland zuvor Ge­ne­ral­di­rek­tor der Welt­handels­organisation,
Wett­be­werbs­kommissar der EU sowie Präsident von Goldman Sachs
Inter­na­tional. Bereits 2012
\href{https://de.wikipedia.org/wiki/Peter_Sutherland}{empfahl er der EU}
mit Blick auf Wachstum und Globalisierung, »ihr Möglichstes« zu tun, um
die »ethnische Homogenität« der EU-Staaten durch Migration zu
»unterminieren«.

Medial wird dieses geoökonomisch motivierte,
\href{https://www.youtube.com/watch?v=y9rVVYU-cS0\&t=1559}{»historisch
einzigartige Experiment«} der gesell­schaft­li­chen Transformation durch
ein humanitäres Narrativ flankiert, während ablehnende Politiker, die
sich am »gemeinen Volk« statt an der globalen Strategie orientieren, zu
»Populisten« wurden.

Die etablierten Medien wandelten sich mithin nicht von »rechts« nach
»links«, sondern sie blieben eliten- und wachstumsorientiert. Früher
gegen Sozial­progressive, heute gegen National­konservative:
Po­li­tische Gegner und Gehilfen tauschten die Rollen, doch das
übergeordnete Ziel blieb das­sel­be.

Die Migrationspolitik sollte ursprünglich ergänzt werden durch
internationale Frei­handels­zonen wie TTIP und TPP. Auf diese Weise
würde ein integrierter Milliardenmarkt mit über 50\% der weltweiten
Wirtschafts­leistung entstehen, der es im 21. Jahrhundert mit China
aufnehmen könnte.

Die unerwartete Wahl des migrations- und freihandelskritischen
US-Präsidenten Trump sowie der uner­wartete Ausgang der
Brexit-Abstimmung waren hingegen schwere Rückschläge für diese
Strategie.

\href{https://swprs.files.wordpress.com/2019/07/rich-poor-population-development.png}{}

\includegraphics{https://swprs.files.wordpress.com/2019/07/rich-poor-population-development.png?w=180\&h=180\&crop=1}

Bevölkerungsentwicklung in reichen und armen Ländern (Econimica/UNO)

\href{https://swprs.files.wordpress.com/2019/11/40246456-15730981712775445.png}{}

\includegraphics{https://swprs.files.wordpress.com/2019/11/40246456-15730981712775445.png?w=180\&h=180\&crop=1}

Entwicklung U65 in reichen Lände (grün)

\href{https://swprs.files.wordpress.com/2018/11/europe-population.png}{}

\includegraphics{https://swprs.files.wordpress.com/2018/11/europe-population.png?w=180\&h=180\&crop=1}

Demographie Europas (Econimica)

\href{https://swprs.files.wordpress.com/2018/11/population-germany.png}{}

\includegraphics{https://swprs.files.wordpress.com/2018/11/population-germany.png?w=180\&h=180\&crop=1}

Demographie Deutschlands (GEFIRA)

\href{https://swprs.files.wordpress.com/2018/11/gdp-2050-economist.png}{}

\includegraphics{https://swprs.files.wordpress.com/2018/11/gdp-2050-economist.png?w=180\&h=180\&crop=1}

Wirtschaftsleistung 2014 und 2050 (Economist)

\href{https://swprs.files.wordpress.com/2018/11/ttip-tpp.png}{}

\includegraphics{https://swprs.files.wordpress.com/2018/11/ttip-tpp.png?w=180\&h=180\&crop=1}

Geplante Freihandelszonen TTIP und TPP (Guardian)

\href{https://swprs.files.wordpress.com/2018/11/migration.png}{}

\includegraphics{https://swprs.files.wordpress.com/2018/11/migration.png?w=180\&h=180\&crop=1}

Migration nach Europa (EU 2015)

\href{https://swprs.files.wordpress.com/2018/11/migration-routen.png}{}

\includegraphics{https://swprs.files.wordpress.com/2018/11/migration-routen.png?w=180\&h=180\&crop=1}

Herkunftsländer und Migrationsrouten (EU 2016)

Geoökonomie: Demographie, Freihandel und Migration*\\
*

\hypertarget{worum-es-bei-der-migration-nach-europa-nicht-geht}{%
\paragraph{\texorpdfstring{Worum es bei der Migration nach Europa
\emph{nicht}
geht}{Worum es bei der Migration nach Europa nicht geht}}\label{worum-es-bei-der-migration-nach-europa-nicht-geht}}

\begin{enumerate}
\def\labelenumi{\arabic{enumi}.}
\tightlist
\item
  Es geht nicht um ein \textbf{ungewolltes Phänomen}, denn Migration
  wird gefordert und gefördert.
\item
  Es geht nicht um \textbf{Humanitarismus}, denn sonst würden die
  Ursachen -- insbesondere Kriege und wirtschaftliche Ungleichheit --
  bekämpft. Doch das Gegenteil ist der Fall.
\item
  Es geht nicht um \textbf{Facharbeiter}, denn diese müssten gezielt
  akquiriert werden -- was im Falle von Entwicklungsländern jedoch
  ethisch fragwürdig wäre.
\item
  Es geht auch nicht um \textbf{unqualifizierte Arbeitskräfte}, denn
  erstens verfügt Europa bereits selbst über mehrere Millionen
  Arbeitslose (inkl. Jugendarbeitslose), und zweitens entfallen durch
  Automatisierung in den kommenden Jahrzehnten weitere Millionen
  Arbeitsplätze.
\item
  Es geht nicht um die Anwendung der sog.
  \textbf{\href{https://de.wikipedia.org/wiki/Weapons_of_Mass_Migration}{»Migrationswaffe«}}
  gegen Deutschland oder Europa, denn die Migration wird von den
  europäischen Eliten selbst gefordert, und sie betrifft insbesondere
  auch die Siegerstaaten des Zweiten Weltkriegs wie England, Frankreich
  und die USA. Gleichwohl ist es zutreffend, dass durch die
  Migrationspolitik die national­staatlichen Strukturen mittelfristig in
  größere geoökonomische Entitäten überführt werden können (vgl.
  Sutherland-Zitat oben).
\end{enumerate}

Der Engpass des modernen, hochproduktiven Wirtschaftssystems sind nicht
die fehlenden Arbeits­kräfte, sondern die fehlenden Konsumenten.
Schrumpfende Nationalstaaten haben in der globali­sierten Welt des 21.
Jahrhunderts insbesondere mit Blick auf China einen besonders schweren
Stand.

Demographisch stagnierende oder schrumpfende Staaten wie Deutschland und
Japan konnten ihr Wirtschaftswachstum seit ca. 2000 nur noch durch hohe
Exportüberschüsse erhalten. Dieses Export­wachstum war wiederum
\href{https://www.zeit.de/wirtschaft/2017-02/china-deutschland-handel-exporte-import-statistisches-bundesamt}{primär
aufgrund} der wirtschaftlichen Entwicklung Chinas möglich.

Beispielsweise sank die
\href{https://www.welt.de/politik/deutschland/article198942039/Selbst-bei-Nullzuwanderung-wuerde-der-Migrationsanteil-steigen.html}{Zahl
der Herkunftsdeutschen} in der BRD zwischen 2005 und 2018 von 66,4
Millionen auf 60,8 Millionen. 60 Millionen Einwohner hatte Deutschland
zuletzt um 1900.

Im selben Kontext ist die moderne Tiefzinspolitik zu sehen: Auf diese
Weise soll das fehlende demographische und reale Wachstum
\href{https://econimica.blogspot.com/2018/12/a-debt-based-system-cant-succeed.html}{kompensiert}
und Aktien- und Immobilienmärkte stabilisiert werden. Dies wiederum
\href{https://www.businessinsider.de/warum-banken-das-bargeld-wirklich-abschaffen-wollen-2016-10}{erfordert
mittelfristig} den Wechsel in ein digitales Geldsystem, um die Ent­nahme
von Bargeld aus dem Bankensystem bei potentiell negativen Zinsen zu
verhindern.

Doch wovon leben die Menschen, wenn sie als Arbeitskräfte nicht mehr
benötigt werden, wohl aber als Konsumenten? Von Sozialhilfe bzw. einem
Grundeinkommen, das durch Verschuldung oder Geld­schö­p­fung generiert
wird -- ein Konzept, mit dem sich auch das \emph{World Economic Forum}
\href{https://theconversation.com/how-basic-income-can-solve-one-of-the-digital-economys-biggest-problems-53081}{bereits
befasst}.

Eine Alternative würde die Abkehr vom wachstumsorientierten
Wirtschaftsmodell erfordern und so­mit die größte Umwälzung seit der
Industriellen Revolution bedeuten. Allerdings wäre selbst damit die
geostrategische und ökonomische Herausforderung durch das »Reich der
Mitte« noch nicht gelöst.

Kritiker einer wachstumsorientierten Migration wie etwa der langjährige
Direktor der UNO-Division für Be­völ­ke­rungs­entwicklung, Joseph
Chamie, sprechen hingegen von einer
\href{https://www.theglobalist.com/is-population-growth-a-ponzi-scheme/}{»Ponzi-Demographie«}
mit unabsehbaren Folgen für die Gesellschaften der Herkunfts- und
Zielländer sowie für die Umwelt.

\hypertarget{siehe-auch}{%
\paragraph{Siehe auch}\label{siehe-auch}}

\begin{itemize}
\tightlist
\item
  KAS:
  \href{https://web.archive.org/web/20181123235853/https://www.kas.de/veranstaltungsberichte/detail/-/content/-das-herz-der-demokratie-}{Angela
  Merkel fordert Bereitschaft zur Abgabe von Souveränität}.
  Parlamentarismus im Spannungsverhältnis von Globalisierung und
  nationaler Souveränität (2018, Archiv)
\item
  Pelle Neroth Taylor:
  \href{https://www.bitchute.com/video/BmjhwEPedFzI/}{Sweden: Dying to
  Be Multicultural} (Doku, 60 Minuten, 2017)
\item
  Friederike Beck:
  \href{https://www.amazon.de/Die-geheime-Migrationsagenda-superreichen-Stiftungen-ebook/dp/B01KN7SHMC}{Die
  geheime Migrationsagenda} (Monographie, 300 Seiten, 2016)
\item
  Frans Timmermans:
  \href{https://www.youtube.com/watch?v=dIE9Ztn56Ig}{Vielfalt ist das
  Schicksal der Menschheit} (EU, Vortrag, 2015)
\item
  Julian Assange: \href{https://www.youtube.com/watch?v=sMwMboGWBSg}{The
  EU refugee crisis and the strategic depopulation of Syria} (2015)
\item
  Peter Sutherland:
  \href{https://www.bbc.com/news/uk-politics-18519395}{EU should
  `undermine national homogeneity'} (BBC, 2012)
\end{itemize}

\hypertarget{zusatz-der-kampf-der-kulturen}{%
\paragraph{Zusatz: Der »Kampf der
Kulturen«}\label{zusatz-der-kampf-der-kulturen}}

Einige Medien und Organisationen nutzen die teilweise gravierenden
Komplikationen der modernen Massen­migration insbesondere aus
islamischen Län­dern zusätzlich zur Promotion einer eigenen
geo­po­li­tischen und psychologischen Strategie: dem
\href{https://de.wikipedia.org/wiki/Kampf_der_Kulturen}{»Kampf der
Kulturen«}.

Diese Medien und Organisationen berichten einerseits schonungslos und im
Allgemeinen korrekt über die realen Probleme im Zusammenhang mit
vorwiegend muslimischen Migranten, versuchen diese Probleme jedoch
ande­rer­seits auf den Islam und islamische Länder als Ganzes zu
generalisieren.

Eine zentrale Rolle spielen in diesem Zusammenhang insbesondere das von
pro-israelischen US-Mil­liar­­dä­ren finan­zierte
\href{https://www.alternet.org/2015/11/one-americas-most-dangerous-think-tanks-spreading-islamophobic-hate-across-atlantic/}{»Gatestone
Institute«} sowie dessen internationale Ableger und Partner, da­run­ter
auch einige reichweitenstarke Plattformen und Publizisten im
deutschsprachigen Raum.

Das Problem mit dieser Darstellung ist einerseits, dass islamische
Länder wie Afghanistan, Iran oder Irak bis zu den -- oftmals westlich
initiierten --
\href{https://www.voltairenet.org/article165889.html}{Kriegen} und
\href{https://www.theguardian.com/world/2016/jun/10/ayatollah-khomeini-jimmy-carter-administration-iran-revolution}{Regime­wechseln}
in den 70er- und 80er-Jahren durchaus gesellschaftlich liberal waren --
gerade auch gegenüber Frauen --, und dass anderer­seits Saudi-Arabien
als Zentrum und Finanzier des radikalen politischen Islams bekanntlich
einer der wich­­tig­s­­ten westlichen und israelischen Verbündeten im
Nahen Osten ist.

\href{https://swprs.org/migration-und-medien/kabul-1972/}{}

\includegraphics{https://swprs.files.wordpress.com/2019/08/kabul-1972.jpg?w=241\&h=241\&crop=1}

Kabul 1972

\href{https://swprs.org/migration-und-medien/baghdad-university-1970/}{}

\includegraphics{https://swprs.files.wordpress.com/2019/08/baghdad-university-1970.jpg?w=241\&h=241\&crop=1}

Baghdad 1972

\href{https://swprs.org/migration-und-medien/iran-1970/}{}

\includegraphics{https://swprs.files.wordpress.com/2019/08/iran-1970.jpg?w=241\&h=241\&crop=1}

Teheran 1971

Kabul, Baghdad und Teheran in den 1970er Jahren

\begin{center}\rule{0.5\linewidth}{\linethickness}\end{center}

Beitrag teilen auf:
\href{https://twitter.com/intent/tweet?url=https://swprs.org/migration-und-medien/}{Twitter}
/
\href{https://www.facebook.com/share.php?u=https://swprs.org/migration-und-medien/}{Facebook}

November 2018; Aktualisiert: November 2019*\\
*

\hypertarget{swiss-policy-research}{%
\subsubsection{Swiss Policy Research}\label{swiss-policy-research}}

\begin{itemize}
\tightlist
\item
  \href{https://swprs.org/kontakt/}{Kontakt}
\item
  \href{https://swprs.org/uebersicht/}{Übersicht}
\item
  \href{https://swprs.org/donationen/}{Donationen}
\item
  \href{https://swprs.org/disclaimer/}{Disclaimer}
\end{itemize}

\hypertarget{english}{%
\subsubsection{English}\label{english}}

\begin{itemize}
\tightlist
\item
  \href{https://swprs.org/contact/}{About Us / Contact}
\item
  \href{https://swprs.org/media-navigator/}{The Media Navigator}
\item
  \href{https://swprs.org/the-american-empire-and-its-media/}{The CFR
  and the Media}
\item
  \href{https://swprs.org/donations/}{Donations}
\end{itemize}

\hypertarget{follow-by-email}{%
\subsubsection{Follow by email}\label{follow-by-email}}

Follow

\href{https://wordpress.com/?ref=footer_custom_com}{WordPress.com}.

\protect\hyperlink{}{Up ↑}

Post to

\protect\hyperlink{}{Cancel}

\includegraphics{https://pixel.wp.com/b.gif?v=noscript}
