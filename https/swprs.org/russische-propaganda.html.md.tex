\protect\hyperlink{content}{Skip to content}

\href{https://swprs.org/}{}

\protect\hyperlink{search-container}{Search}

Search for:

\href{https://swprs.org/}{\includegraphics{https://swprs.files.wordpress.com/2020/05/swiss-policy-research-logo-300.png}}

\href{https://swprs.org/}{Swiss Policy Research}

Geopolitics and Media

Menu

\begin{itemize}
\tightlist
\item
  \href{https://swprs.org}{Start}
\item
  \href{https://swprs.org/srf-propaganda-analyse/}{Studien}

  \begin{itemize}
  \tightlist
  \item
    \href{https://swprs.org/srf-propaganda-analyse/}{SRF / ZDF}
  \item
    \href{https://swprs.org/die-nzz-studie/}{NZZ-Studie}
  \item
    \href{https://swprs.org/der-propaganda-multiplikator/}{Agenturen}
  \item
    \href{https://swprs.org/die-propaganda-matrix/}{Medienmatrix}
  \end{itemize}
\item
  \href{https://swprs.org/medien-navigator/}{Analysen}

  \begin{itemize}
  \tightlist
  \item
    \href{https://swprs.org/medien-navigator/}{Navigator}
  \item
    \href{https://swprs.org/der-propaganda-schluessel/}{Techniken}
  \item
    \href{https://swprs.org/propaganda-in-der-wikipedia/}{Wikipedia}
  \item
    \href{https://swprs.org/logik-imperialer-kriege/}{Kriege}
  \end{itemize}
\item
  \href{https://swprs.org/netzwerk-medien-schweiz/}{Netzwerke}

  \begin{itemize}
  \tightlist
  \item
    \href{https://swprs.org/netzwerk-medien-schweiz/}{Schweiz}
  \item
    \href{https://swprs.org/netzwerk-medien-deutschland/}{Deutschland}
  \item
    \href{https://swprs.org/medien-in-oesterreich/}{Österreich}
  \item
    \href{https://swprs.org/das-american-empire-und-seine-medien/}{USA}
  \end{itemize}
\item
  \href{https://swprs.org/bericht-eines-journalisten/}{Fokus I}

  \begin{itemize}
  \tightlist
  \item
    \href{https://swprs.org/bericht-eines-journalisten/}{Journalistenbericht}
  \item
    \href{https://swprs.org/russische-propaganda/}{Russische Propaganda}
  \item
    \href{https://swprs.org/die-israel-lobby-fakten-und-mythen/}{Die
    »Israel-Lobby«}
  \item
    \href{https://swprs.org/geopolitik-und-paedokriminalitaet/}{Pädokriminalität}
  \end{itemize}
\item
  \href{https://swprs.org/migration-und-medien/}{Fokus II}

  \begin{itemize}
  \tightlist
  \item
    \href{https://swprs.org/covid-19-hinweis-ii/}{Coronavirus}
  \item
    \href{https://swprs.org/die-integrity-initiative/}{Integrity
    Initiative}
  \item
    \href{https://swprs.org/migration-und-medien/}{Migration \& Medien}
  \item
    \href{https://swprs.org/der-fall-magnitsky/}{Magnitsky Act}
  \end{itemize}
\item
  \href{https://swprs.org/kontakt/}{Projekt}

  \begin{itemize}
  \tightlist
  \item
    \href{https://swprs.org/kontakt/}{Kontakt}
  \item
    \href{https://swprs.org/uebersicht/}{Seitenübersicht}
  \item
    \href{https://swprs.org/medienspiegel/}{Medienspiegel}
  \item
    \href{https://swprs.org/donationen/}{Donationen}
  \end{itemize}
\item
  \href{https://swprs.org/contact/}{English}
\end{itemize}

\protect\hyperlink{}{Open Search}

\hypertarget{russische-propaganda}{%
\section{Russische Propaganda}\label{russische-propaganda}}

\includegraphics{https://swprs.files.wordpress.com/2018/11/kreml.png?w=736}

Wie funktioniert russische Propaganda, und was macht sie so
wirkungsvoll?

Das Grundprinzip ist universell und fokusiert auf macht­re­le­vante
Dis­so­n­an­zen und Dis­kre­panzen in den Ziel­ländern, kombiniert mit
der positiven Dar­stel­lung der eigenen Ambitionen.

Zu den wichtigsten
\href{https://swprs.org/der-propaganda-schluessel/}{Techniken} zählen
dabei nicht »Fa­kes« -- die widerlegt werden können -- son­dern eine
geschickte Wahl von Themen, As­pek­ten und Inter­view­part­nern. Da man
selbst keinen neutralen Jour­na­lis­mus verspricht, wird er vom Publikum
auch nicht erwartet.

Am wirkungsvollsten ist dieser Ansatz, wenn eine Einseitigkeit oder
Unvollständigkeit der westlichen Berichterstattung nachgewiesen und
dadurch das Vertrauen ins westliche Mediensystem insgesamt erschüttert
werden kann -- siehe Slogans und Formate wie \emph{Question More} oder
\emph{Der Fehlende Part}.

Progressiv wirkend, ist das übergeordnete strategische Ziel dennoch die
mediale Unter­stüt­zung der russischen Außenpolitik -- im Frieden wie im
Krieg. Am besten lässt sich dies bei wechselhaften diplomatischen
Beziehungen beobachten, etwa zu Ländern wie Frankreich, Israel oder der
Türkei.

Zu bedenken ist ferner, dass russische Auslandsmedien von einer
Regierung finanziert werden, deren Einnahmen wesentlich auf dem Export
von fossilen Rohstoffen und militärischen Gütern beruhen.

Für westliche Kritiker aus den unterschiedlichsten politischen
Richtungen ist ein solches Programm -- ob zur Information oder als
Plattform -- gleichwohl at­trak­tiv. Westliche Medien geraten hingegen
in ein Dilemma: Sollen sie die russisch geförderte Kritik aufnehmen,
ignorieren, oder bekämpfen?

Weiterhin unbelegt ist hingegen eine angeblich russische
Wahlbeeinflussung durch
\href{https://www.thenation.com/article/russiagate-elections-interference/}{Facebook}
oder
\href{https://consortiumnews.com/2018/08/13/too-big-to-fail-russia-gate-one-year-after-vips-showed-a-leak-not-a-hack/}{Computerhacks}.
2018 wurde eine vom US-Kongress beauftragte Beratungsfirma zudem
\href{https://thegrayzone.com/2018/12/25/senate-report-on-russian-interference-was-written-by-disinformation-warriors-behind-alabama-false-flag-operation/}{erwischt},
wie sie selbst ein »russisches Botnet« vortäuschte, um eine
US-Senatswahl zu manipulieren.

Eigentliche
\href{https://www.wilsoncenter.org/blog-post/operation-denver-kgb-and-stasi-disinformation-regarding-aids}{Desinformation}
kommt russischerseits insbesondere dann zum Einsatz, wenn die
Fakten­lage schwierig zu überprüfen ist. Zu ihrer Verbreitung werden
dabei oftmals Drittparteien genutzt.

\textbf{Westliche Projekte zu russischer Propaganda:}
\href{https://euvsdisinfo.eu/}{EU versus Dis­info} (EU),
\href{https://disinfoportal.org/}{DisinfoPortal} (USA),
\href{https://faktenfinder.tagesschau.de/ausland/index.html}{FaktenFinder}
(Deutschland), \href{https://www.bellingcat.com/}{BellingCat} (England),
\href{https://www.stopfake.org/en/news/}{StopFake} (Ukraine)

\textbf{Siehe auch}:
\href{https://www.t-online.de/nachrichten/deutschland/id_84584050/mitten-in-berlin-russlands-heimliche-medienzentrale-in-europa.html}{Russlands
heimliche Medienzentrale in Europa} (T-Online, November 2018)

\hypertarget{selbsttest-russische-propaganda}{%
\subsubsection{Selbsttest: Russische
Propaganda}\label{selbsttest-russische-propaganda}}

Zahlreiche Menschen sind Opfer transatlantischer Propaganda. Manche sind
aber auch Opfer russischer Propaganda. Gehören Sie dazu? Machen Sie den
Selbsttest und finden Sie es heraus.

\begin{enumerate}
\def\labelenumi{\arabic{enumi}.}
\tightlist
\item
  Russland intervenierte in Syrien, weil \textbf{A)} Terroristen
  bekämpft werden mussten, \textbf{B)} das Christen­tum verteidigt
  werden musste, \textbf{C)} Syrien als Klientelstaat erhalten bleiben
  sollte.
\item
  Russland baut Nord Stream II, weil \textbf{A)} Erdgas ökologischer ist
  als Kohle, \textbf{B)} Putin in der Nordsee gerne tauchen geht,
  \textbf{C)} die Pipeline Polen umgeht sowie Einnahmen und Einfluss
  bringt.
\item
  Zu den russischen Hauptexporten zählen \textbf{A)} Friedenstauben,
  \textbf{B)} Erdbeeren, \textbf{C)} Waffen und Erdöl.
\item
  In Russland \textbf{A)} ist die Welt noch in Ordnung, \textbf{B)} sind
  die Frauen noch in Ordnung, \textbf{C)} ist Alkoho­lis­mus und
  häusliche Gewalt ein ernstes Problem.
\item
  Der russische Spitzensport \textbf{A)} ist sauber, \textbf{B)} im
  Westen wird auch gedopt, \textbf{C)} betrieb systema­tisches
  Staatsdoping unter Mitwirkung des Geheimdienstes.
\item
  Staatsnahe russische Hacker sind \textbf{A)} eine Erfindung der NATO,
  \textbf{B)} im Unterschied zu den CIA-Hackern von noblen Motiven
  geleitet, \textbf{C)} eine reale Bedrohung.
\item
  RT-Moderatoren \textbf{A)} können sagen was sie wollen, \textbf{B)}
  dürfen sagen was sie wollen, \textbf{C)} dürfen sagen was sie wollen,
  solange es nicht der russischen Außenpolitik widerspricht.
\item
  Wenn ich RT schaue, dann weil \textbf{A)} die Nachrichten objektiver
  sind, \textbf{B)} die Moderatoren sympathischer sind, \textbf{C)} ich
  an der russischen Sicht der Dinge interessiert bin.
\item
  Putin integrierte die Krim, weil \textbf{A)} die Krim-Bewohner das so
  wollten, \textbf{B)} die Krim von ukrai­ni­schen Faschisten bedroht
  wurde, \textbf{C)} die Krim ein wichtiger russischer Militärstützpunkt
  ist.
\item
  Die »Grünen Männchen« auf der Krim waren \textbf{A)} welche grünen
  Männchen?, \textbf{B)} freiwillige Aktivisten, \textbf{C)} russische
  Spezialeinheiten ohne Abzeichen.
\item
  Die UdSSR im Zweiten Weltkrieg \textbf{A)} wollte nur Frieden,
  \textbf{B)} führte Krieg, aber nur zur Verteidigung, \textbf{C)}
  marschierte in Finnland, dem Baltikum, Polen, Rumänien und China ein,
  noch bevor sie selbst von Deutschland angegriffen wurde, und besetzte
  am Ende weite Teile Europas.
\item
  Zusatzfrage: Dieser Selbsttest ist \textbf{A)} eine Frechheit,
  \textbf{B)} nicht ernst gemeint, \textbf{C)} wichtig.
\end{enumerate}

\textbf{Auswertung}: Wenn Sie bei irgendeiner der zwölf Fragen
\emph{nicht} mit C) geantwortet haben, sind Sie vermutlich Opfer
russischer Propaganda und sollten russische Medien kritischer
konsumieren.

\begin{center}\rule{0.5\linewidth}{\linethickness}\end{center}

Publiziert: November 2018; Aktualisiert: November 2019

\hypertarget{swiss-policy-research}{%
\subsubsection{Swiss Policy Research}\label{swiss-policy-research}}

\begin{itemize}
\tightlist
\item
  \href{https://swprs.org/kontakt/}{Kontakt}
\item
  \href{https://swprs.org/uebersicht/}{Übersicht}
\item
  \href{https://swprs.org/donationen/}{Donationen}
\item
  \href{https://swprs.org/disclaimer/}{Disclaimer}
\end{itemize}

\hypertarget{english}{%
\subsubsection{English}\label{english}}

\begin{itemize}
\tightlist
\item
  \href{https://swprs.org/contact/}{About Us / Contact}
\item
  \href{https://swprs.org/media-navigator/}{The Media Navigator}
\item
  \href{https://swprs.org/the-american-empire-and-its-media/}{The CFR
  and the Media}
\item
  \href{https://swprs.org/donations/}{Donations}
\end{itemize}

\hypertarget{follow-by-email}{%
\subsubsection{Follow by email}\label{follow-by-email}}

Follow

\href{https://wordpress.com/?ref=footer_custom_com}{WordPress.com}.

\protect\hyperlink{}{Up ↑}

Post to

\protect\hyperlink{}{Cancel}

\includegraphics{https://pixel.wp.com/b.gif?v=noscript}
