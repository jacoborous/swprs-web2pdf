\protect\hyperlink{content}{Skip to content}

\href{https://swprs.org/}{}

\protect\hyperlink{search-container}{Search}

Search for:

\href{https://swprs.org/}{\includegraphics{https://swprs.files.wordpress.com/2020/05/swiss-policy-research-logo-300.png}}

\href{https://swprs.org/}{Swiss Policy Research}

Geopolitics and Media

Menu

\begin{itemize}
\tightlist
\item
  \href{https://swprs.org}{Start}
\item
  \href{https://swprs.org/srf-propaganda-analyse/}{Studien}

  \begin{itemize}
  \tightlist
  \item
    \href{https://swprs.org/srf-propaganda-analyse/}{SRF / ZDF}
  \item
    \href{https://swprs.org/die-nzz-studie/}{NZZ-Studie}
  \item
    \href{https://swprs.org/der-propaganda-multiplikator/}{Agenturen}
  \item
    \href{https://swprs.org/die-propaganda-matrix/}{Medienmatrix}
  \end{itemize}
\item
  \href{https://swprs.org/medien-navigator/}{Analysen}

  \begin{itemize}
  \tightlist
  \item
    \href{https://swprs.org/medien-navigator/}{Navigator}
  \item
    \href{https://swprs.org/der-propaganda-schluessel/}{Techniken}
  \item
    \href{https://swprs.org/propaganda-in-der-wikipedia/}{Wikipedia}
  \item
    \href{https://swprs.org/logik-imperialer-kriege/}{Kriege}
  \end{itemize}
\item
  \href{https://swprs.org/netzwerk-medien-schweiz/}{Netzwerke}

  \begin{itemize}
  \tightlist
  \item
    \href{https://swprs.org/netzwerk-medien-schweiz/}{Schweiz}
  \item
    \href{https://swprs.org/netzwerk-medien-deutschland/}{Deutschland}
  \item
    \href{https://swprs.org/medien-in-oesterreich/}{Österreich}
  \item
    \href{https://swprs.org/das-american-empire-und-seine-medien/}{USA}
  \end{itemize}
\item
  \href{https://swprs.org/bericht-eines-journalisten/}{Fokus I}

  \begin{itemize}
  \tightlist
  \item
    \href{https://swprs.org/bericht-eines-journalisten/}{Journalistenbericht}
  \item
    \href{https://swprs.org/russische-propaganda/}{Russische Propaganda}
  \item
    \href{https://swprs.org/die-israel-lobby-fakten-und-mythen/}{Die
    »Israel-Lobby«}
  \item
    \href{https://swprs.org/geopolitik-und-paedokriminalitaet/}{Pädokriminalität}
  \end{itemize}
\item
  \href{https://swprs.org/migration-und-medien/}{Fokus II}

  \begin{itemize}
  \tightlist
  \item
    \href{https://swprs.org/covid-19-hinweis-ii/}{Coronavirus}
  \item
    \href{https://swprs.org/die-integrity-initiative/}{Integrity
    Initiative}
  \item
    \href{https://swprs.org/migration-und-medien/}{Migration \& Medien}
  \item
    \href{https://swprs.org/der-fall-magnitsky/}{Magnitsky Act}
  \end{itemize}
\item
  \href{https://swprs.org/kontakt/}{Projekt}

  \begin{itemize}
  \tightlist
  \item
    \href{https://swprs.org/kontakt/}{Kontakt}
  \item
    \href{https://swprs.org/uebersicht/}{Seitenübersicht}
  \item
    \href{https://swprs.org/medienspiegel/}{Medienspiegel}
  \item
    \href{https://swprs.org/donationen/}{Donationen}
  \end{itemize}
\item
  \href{https://swprs.org/contact/}{English}
\end{itemize}

\protect\hyperlink{}{Open Search}

\hypertarget{covid-19-bericht-aus-italien}{%
\section{Covid-19: Bericht
aus~Italien}\label{covid-19-bericht-aus-italien}}

\textbf{Publiziert}: 31. März 2020\\
\textbf{Hauptartikel}:
\href{https://swprs.org/covid-19-hinweis-ii/}{Fakten zu Covid-19}

Bericht eines Beobachters aus Italien vom 31. März 2020 zur Situation in
Norditalien:

„In den letzten Wochen haben die meisten osteuropäischen Pflegekräfte,
die im 24 Stunden Dienst 7 Tage die Woche in der Betreuung von
Pflegebedürftigen in Italien arbeiteten, fluchtartig das Land verlassen.
Dies nicht zuletzt wegen der Panikmache und den von den
„Notstandsregierungen`` angedrohten Ausgangssperren und
Grenzschließungen. Deshalb wurden alte pflegebedürftige Personen und
Behinderte, teilweise ohne Verwandte, von ihren Betreuern hilflos
zurückgelassen.

Viele von diesen verlassenen Menschen landeten dann nach einigen Tagen
in den seit Jahren permanent überlasteten Krankenhäusern, weil sie unter
anderem dehydriert waren. Leider fehlte den Spitälern jetzt auch noch
das Personal, welches eingesperrt in den Wohnungen auf die Kinder
aufpassen mussten, weil Schulen und Kindergärten geschlossen worden
waren. Dies führte dann in der Folge zum vollkommenen Zusammenbruch der
Behinderten- und Altenpflege gerade in den Gebieten, wo weitere noch
härtere „Maßnahmen`` angeordnet wurden und zu chaotischen Verhältnissen.

Der Pflegenotstand, der durch die Panik entstand, führte temporär zu
vielen Todesopfern unter den Pflegebedürftigen und zunehmend auch unter
jüngeren Patienten der Krankenhäuser. Diese Todesopfer dienten dann den
Verantwortlichen und den Medien dazu, die Leute in noch mehr Panik zu
versetzen, indem sie zum Beispiel meldeten „weitere 475 Todesopfer``,
„Die Toten werden von der Armee aus den Krankenhäusern geholt``,
untermalt mit Bildern von aufgereihten Särgen und Armeelastwagen.

Das war jedoch die Folge der Angst der Bestattungsunternehmer vor dem
„Killervirus``, die deshalb ihre Dienste verwehrten. Außerdem waren es
zum einen zu viele Todesfälle auf einmal und zum anderen wurde von der
Regierung ein Gesetz erlassen, dass die Leichen, die den Coronavirus
trugen eingeäschert werden mussten. In Italien wurden bis zu diesem
Datum nur wenige Feuerbestattungen vollzogen. Deshalb gab es nur wenige
kleine Krematorien, die sehr schnell an Ihre Grenzen stießen. Die
Verstorbenen mussten deshalb in verschiedenen Kirchen aufgebahrt werden.

Diese Entwicklung lief im Prinzip in allen Ländern gleich ab. Die
Qualität des Gesund­heits­systems hat jedoch einen erheblichen Einfluss
auf die Auswirkungen. Deshalb gibt es in Deutschland, Österreich oder
der Schweiz weniger Probleme als in Italien, Spanien oder den USA. Wie
man aber in den offiziellen Zahlen sehen kann, gibt es keine
nennenswerte Erhöhung der Mortalitätsrate. Nur einen kleiner Berg, der
von dieser Tragödie stammt.``

\hypertarget{siehe-auch}{%
\paragraph{Siehe auch}\label{siehe-auch}}

\begin{itemize}
\tightlist
\item
  \href{https://ltccovid.org/2020/04/12/mortality-associated-with-covid-19-outbreaks-in-care-homes-early-international-evidence/}{Mortality
  associated with COVID-19 outbreaks in care homes: early international
  evidence}
\item
  \href{https://orf.at/stories/3162365/}{31 Tote in kanadischem
  Altersheim nach Flucht der Pflegekräfte} (ORF)
\item
  \href{https://www.sueddeutsche.de/politik/coronavirus-pflegekraefte-ausland-1.4866124}{Coronavirus
  Die Flucht der Pflegekräfte} (Süddeutsche Zeitung)
\end{itemize}

\hypertarget{zuruxfcck-zum-hauptartikel-fakten-zu-covid-19}{%
\subparagraph{\texorpdfstring{\href{https://swprs.org/covid-19-hinweis-ii/}{Zurück
zum Hauptartikel: Fakten zu
Covid-19}}{Zurück zum Hauptartikel: Fakten zu Covid-19}}\label{zuruxfcck-zum-hauptartikel-fakten-zu-covid-19}}

\begin{center}\rule{0.5\linewidth}{\linethickness}\end{center}

\hypertarget{swiss-policy-research}{%
\subsubsection{Swiss Policy Research}\label{swiss-policy-research}}

\begin{itemize}
\tightlist
\item
  \href{https://swprs.org/kontakt/}{Kontakt}
\item
  \href{https://swprs.org/uebersicht/}{Übersicht}
\item
  \href{https://swprs.org/donationen/}{Donationen}
\item
  \href{https://swprs.org/disclaimer/}{Disclaimer}
\end{itemize}

\hypertarget{english}{%
\subsubsection{English}\label{english}}

\begin{itemize}
\tightlist
\item
  \href{https://swprs.org/contact/}{About Us / Contact}
\item
  \href{https://swprs.org/media-navigator/}{The Media Navigator}
\item
  \href{https://swprs.org/the-american-empire-and-its-media/}{The CFR
  and the Media}
\item
  \href{https://swprs.org/donations/}{Donations}
\end{itemize}

\hypertarget{follow-by-email}{%
\subsubsection{Follow by email}\label{follow-by-email}}

Follow

\href{https://wordpress.com/?ref=footer_custom_com}{WordPress.com}.

\protect\hyperlink{}{Up ↑}

\includegraphics{https://pixel.wp.com/b.gif?v=noscript}
