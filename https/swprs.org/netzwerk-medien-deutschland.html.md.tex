\protect\hyperlink{content}{Skip to content}

\href{https://swprs.org/}{}

\protect\hyperlink{search-container}{Search}

Search for:

\href{https://swprs.org/}{\includegraphics{https://swprs.files.wordpress.com/2020/05/swiss-policy-research-logo-300.png}}

\href{https://swprs.org/}{Swiss Policy Research}

Geopolitics and Media

Menu

\begin{itemize}
\tightlist
\item
  \href{https://swprs.org}{Start}
\item
  \href{https://swprs.org/srf-propaganda-analyse/}{Studien}

  \begin{itemize}
  \tightlist
  \item
    \href{https://swprs.org/srf-propaganda-analyse/}{SRF / ZDF}
  \item
    \href{https://swprs.org/die-nzz-studie/}{NZZ-Studie}
  \item
    \href{https://swprs.org/der-propaganda-multiplikator/}{Agenturen}
  \item
    \href{https://swprs.org/die-propaganda-matrix/}{Medienmatrix}
  \end{itemize}
\item
  \href{https://swprs.org/medien-navigator/}{Analysen}

  \begin{itemize}
  \tightlist
  \item
    \href{https://swprs.org/medien-navigator/}{Navigator}
  \item
    \href{https://swprs.org/der-propaganda-schluessel/}{Techniken}
  \item
    \href{https://swprs.org/propaganda-in-der-wikipedia/}{Wikipedia}
  \item
    \href{https://swprs.org/logik-imperialer-kriege/}{Kriege}
  \end{itemize}
\item
  \href{https://swprs.org/netzwerk-medien-schweiz/}{Netzwerke}

  \begin{itemize}
  \tightlist
  \item
    \href{https://swprs.org/netzwerk-medien-schweiz/}{Schweiz}
  \item
    \href{https://swprs.org/netzwerk-medien-deutschland/}{Deutschland}
  \item
    \href{https://swprs.org/medien-in-oesterreich/}{Österreich}
  \item
    \href{https://swprs.org/das-american-empire-und-seine-medien/}{USA}
  \end{itemize}
\item
  \href{https://swprs.org/bericht-eines-journalisten/}{Fokus I}

  \begin{itemize}
  \tightlist
  \item
    \href{https://swprs.org/bericht-eines-journalisten/}{Journalistenbericht}
  \item
    \href{https://swprs.org/russische-propaganda/}{Russische Propaganda}
  \item
    \href{https://swprs.org/die-israel-lobby-fakten-und-mythen/}{Die
    »Israel-Lobby«}
  \item
    \href{https://swprs.org/geopolitik-und-paedokriminalitaet/}{Pädokriminalität}
  \end{itemize}
\item
  \href{https://swprs.org/migration-und-medien/}{Fokus II}

  \begin{itemize}
  \tightlist
  \item
    \href{https://swprs.org/covid-19-hinweis-ii/}{Coronavirus}
  \item
    \href{https://swprs.org/die-integrity-initiative/}{Integrity
    Initiative}
  \item
    \href{https://swprs.org/migration-und-medien/}{Migration \& Medien}
  \item
    \href{https://swprs.org/der-fall-magnitsky/}{Magnitsky Act}
  \end{itemize}
\item
  \href{https://swprs.org/kontakt/}{Projekt}

  \begin{itemize}
  \tightlist
  \item
    \href{https://swprs.org/kontakt/}{Kontakt}
  \item
    \href{https://swprs.org/uebersicht/}{Seitenübersicht}
  \item
    \href{https://swprs.org/medienspiegel/}{Medienspiegel}
  \item
    \href{https://swprs.org/donationen/}{Donationen}
  \end{itemize}
\item
  \href{https://swprs.org/contact/}{English}
\end{itemize}

\protect\hyperlink{}{Open Search}

\hypertarget{medien-in-deutschland}{%
\section{Medien in Deutschland}\label{medien-in-deutschland}}

Deutsche Medien und Journalisten sind aus historischen Gründen besonders
eng in transatlantische Netzwerke eingebunden. Die folgende Infografik
gibt einen Überblick über die wichtigsten Akteure und Verbindungen.

\href{https://swprs.files.wordpress.com/2017/08/netzwerk-medien-deutschland-spr-mt.png}{\includegraphics{https://swprs.files.wordpress.com/2017/08/netzwerk-medien-deutschland-spr-mt.png?w=736}}~\href{https://swprs.files.wordpress.com/2017/08/netzwerk-medien-deutschland-spr-mt.png}{Vergrößern🔎}
(Lizenz:
\href{https://creativecommons.org/licenses/by-nc-nd/4.0/deed.de}{CC-BY-NC-ND})

\hypertarget{quellen-zur-infografik}{%
\paragraph{Quellen zur Infografik}\label{quellen-zur-infografik}}

\begin{enumerate}
\def\labelenumi{\arabic{enumi}.}
\tightlist
\item
  Jahresberichte der Atlantik-Brücke von
  \href{https://www.atlantik-bruecke.org/unsere-arbeit/publikationen/jahresberichte/}{2006
  bis 2016}***

  \begin{center}\rule{0.5\linewidth}{\linethickness}\end{center}
\item
  Teilnehmerlisten der Bilderberg-Konferenzen von
  \href{https://swprs.files.wordpress.com/2016/07/bilderberg_teilnehmer_1954-2014.pdf}{1954-2014}
  und von
  \href{https://www.bilderbergmeetings.org/meetings/meetings-overview/index.html}{2015-2017}
\item
  Aktuelle
  \href{https://swprs.files.wordpress.com/2017/07/trilateral-commission-members-2017.pdf}{Mitgliederliste}
  der Trilateralen Kommission (Mitglieder
  \href{https://swprs.files.wordpress.com/2017/07/trilateral-commission-members-1985.pdf}{1985},
  \href{https://swprs.files.wordpress.com/2017/07/trilateral-commission-members-1995.pdf}{1995},
  \href{https://swprs.files.wordpress.com/2017/07/trilateral-commission-members-2010.pdf}{2010})
\item
  Der
  \href{https://swprs.files.wordpress.com/2018/02/a-message-to-the-people-of-the-united-states-of-america.pdf}{Offene
  Brief} der Atlantik-Brücke zum Irak-Krieg von 2003
\item
  Wikipedia-Artikel zur
  \href{https://de.wikipedia.org/wiki/Atlantik-Br\%C3\%BCcke}{Atlantik-Brücke},
  ihren
  \href{https://de.wikipedia.org/wiki/Liste_von_Mitgliedern_der_Atlantik-Br\%C3\%BCcke}{Mitgliedern}
  und den
  \href{https://web.archive.org/web/20170627193020/https://de.wikipedia.org/wiki/Liste_von_Young_Leaders_der_Atlantik-Br\%C3\%BCcke}{Young
  Leaders}
\item
  Der
  \href{https://de.wikipedia.org/wiki/Vernon_A._Walters_Award}{\emph{Vernon
  A. Walters Award}} der Atlantik-Brücke
\item
  \href{http://spiegelkabinett-blog.blogspot.com/2013/03/journalisten-der-atlantikbrucke-in.html}{Liste
  von Journalisten}, die an Veranstaltungen der Atlantik-Brücke
  teilgenommen haben
\item
  Dr. Uwe Krüger (2013):
  \href{http://www.halem-verlag.de/meinungsmacht-und-elite-journalismus/}{Meinungsmacht.
  Der Einfluss von Eliten auf Leitmedien und Alpha-Journalisten -- eine
  kritische Netzwerkanalyse}. Herbert von Halem Verlag, Köln.
\end{enumerate}

\hypertarget{hinweis-zur-interpretation}{%
\paragraph{Hinweis zur
Interpretation}\label{hinweis-zur-interpretation}}

Die Infografik stellt weder ein »Organigramm« noch eine »Konspiration«
dar, sondern ein öffentlich dokumentiertes, politisch-publizistisches
Netzwerk. Die oberste Ebene (CFR, NSC, NATO) definiert die
transatlantische Geostrategie, die von den aufgeführten Medien im
Allgemeinen abgebildet wird.

Medienforscher Noam Chomsky erklärte dies in einem
\href{https://chomsky.info/199710__/}{Aufsatz von 1997} wie folgt: »Der
entscheidende Punkt ist: Diese Journalisten wären nicht dort, wenn sie
nicht schon lange bewiesen hätten, dass ihnen niemand sagen muss, was
sie zu schreiben haben -- da sie ohnehin das „Richtige`` schreiben
werden. () Mit anderen Worten: Diese Journalisten durchliefen einen
Sozialisierungsprozess.«

\hypertarget{zusuxe4tzliche-informationen}{%
\paragraph{Zusätzliche
Informationen}\label{zusuxe4tzliche-informationen}}

\begin{itemize}
\tightlist
\item
  Die
  \href{https://bildblog.de/89290/axel-springer-gibt-sich-neue-alte-grundsaetze/}{Unternehmensgrundsätze}
  des Axel Springer Verlags (Herausgeber u.a. von \emph{Bild} und
  \emph{Welt}) stipulieren eine ``Unterstützung des trans­atlantischen
  Bündnisses'' (alt) bzw. ``die Solidarität in der freiheitlichen
  Werte­gemeinschaft mit den Vereinigten Staaten von Amerika'' (neu).
\item
  Die ZDF-Kabarett-Sendung »Die Anstalt« brachte 2014 einen
  vielbeachteten\href{https://swprs.org/zdf-anstalt-transatlantische-medien-netzwerke/}{Beitrag
  zur transatlantischen Vernetzung} deutscher Elite-Journalisten.
\item
  Weitere transatlantische Gremien mit Involvierung von Journalisten
  sind beispielsweise das
  \href{https://de.wikipedia.org/wiki/Aspen_Institute}{Aspen Institute},
  die
  \href{https://de.wikipedia.org/wiki/Atlantische_Initiative}{Atlantische
  Initiative}, und der
  \href{https://de.wikipedia.org/wiki/European_Council_on_Foreign_Relations}{European
  Council on Foreign Relations}.
\item
  Die Journalismus-Zeitschrift \emph{Message}
  \href{https://www.lobbycontrol.de/download/Message_Bilderberg.pdf}{interviewte}
  2007 den ehemaligen ZEIT-Chefredakteur Theo Sommer zu seinen
  Teilnahmen an der Bilderberg-Konferenz.
\item
  Ein
  \href{https://www.heise.de/tp/features/Jan-Fleischhauer-die-Atlantik-Bruecke-und-die-CIA-3838580.html?seite=all}{Telepolis-Artikel}
  von 2017 thematisierte die Kontakte zwischen Atlantik-Brücke und CIA.
\item
  Im August 2019 wurde der US-Finanzinvestor KKR
  \href{https://www.reuters.com/article/us-axel-sprngr-m-a-idUSKCN1VG1DK}{zum
  größten Aktionär} bei Axel Springer.
\end{itemize}

\hypertarget{aktualisierungen}{%
\paragraph{Aktualisierungen}\label{aktualisierungen}}

\begin{itemize}
\tightlist
\item
  An der
  \href{https://www.bilderbergmeetings.org/meetings/meeting-2019/participants-2019}{Bilderberg-Konferenz
  2019} nahm Axel-Springer-CEO Mathias Döpfner teil
\item
  An der
  \href{https://www.bilderbergmeetings.org/meetings/meeting-2018/participants-2018}{Bilderberg-Konferenz
  2018} nahm neben Axel-Springer-CEO Mathias Döpfner unter anderem auch
  Bruno Patino teil, der Chief Content Officer von Arte France TV.
\item
  Im
  \href{https://www.atlantik-bruecke.org/unsere-arbeit/publikationen/jahresberichte/}{Jahresbericht
  2018/19} der Atlantik-Brücke sind neu aufgeführt: Julia von Cube
  (Moderatorin WDR) und Sarah Kelly (Moderatorin Deutsche Welle).
\item
  Im
  \href{https://www.atlantik-bruecke.org/unsere-arbeit/publikationen/jahresberichte/}{Jahresbericht
  2017/18} der Atlantik-Brücke sind neu aufgeführt: Julian Reichelt
  (Chefredakteur der \emph{Bild,} Nachfolger von Kai Diekmann), Juliane
  Schäuble (Ressortleiterin Politik beim \emph{Tagesspiegel,} Tochter
  von Wolfgang Schäuble), und Sarah Tacke (Redakteurin beim ZDF).
\item
  Im
  \href{https://www.atlantik-bruecke.org/unsere-arbeit/publikationen/jahresberichte/}{Jahresbericht
  2016/17} der Atlantik-Brücke sind neu aufgeführt: Ines Pohl
  (Chefredakteurin des staatlichen Auslandrundfunks \emph{Deutsche
  Welle,} zuvor Chefredakteurin der Tageszeitung \emph{taz)} sowie
  Martin Klingst (Politischer Korrespondent der Chefredaktion, \emph{Die
  Zeit}).
\item
  Im Februar 2018 wurde Gabor Steingart (Herausgeber
  \emph{Handelsblatt})
  \href{https://de.wikipedia.org/wiki/Gabor_Steingart}{entlassen},
  Thomas Ebeling (Vorstandsvorsitzender der ProSiebenSat.1 Media) musste
  \href{https://de.wikipedia.org/wiki/Thomas_Ebeling_(Manager)}{zurücktreten}.
\end{itemize}

\hypertarget{vertiefende-analysen}{%
\paragraph{Vertiefende Analysen}\label{vertiefende-analysen}}

\begin{itemize}
\tightlist
\item
  Zur Rolle der drei
  \href{https://swprs.org/der-propaganda-multiplikator/}{globalen
  Nachrichtenagenturen}
\item
  Zur Rolle des \href{https://swprs.org/die-propaganda-matrix/}{US
  Council on Foreign Relations (CFR)}
\item
  Zur Rolle des \href{https://swprs.org/atlantic-council/}{Atlantic
  Council of the United States}
\item
  Zur \href{https://swprs.org/propaganda-in-der-wikipedia/}{Organisation
  und Manipulation der Wikipedia}
\item
  Zur \href{https://swprs.org/medien-navigator/}{(geo-)politischen
  Positionierung der Medien}
\item
  Zur Berichterstattung über
  \href{https://swprs.org/migration-und-medien/}{Migration nach Europa}
\end{itemize}

\hypertarget{zitate}{%
\paragraph{Zitate}\label{zitate}}

\begin{itemize}
\tightlist
\item
  »Man muß eine Elite schaffen, die ganz auf Amerika eingestellt ist.
  Diese Elite darf andererseits nicht so beschaffen sein, daß sie im
  deutschen Volk selber kein Vertrauen mehr genießt und als bestochen
  gilt.« \textbf{Max Horkheimer}, Insitut für Sozialforschung (IfS):
  Memorandum on the elimination of German chauvinism. Vorstudie für das
  US State Department, 1942. (Zitiert in Albrecht et al.:
  \href{https://www.campus.de/buecher-campus-verlag/wissenschaft/geschichte/die_intellektuelle_gruendung_der_bundesrepublik-3146.html}{Die
  intellektuelle Gründung der Bundesrepublik}. Campus, 2007, S. 121)
\item
  »Da muss man natürlich auch fragen: Wer macht Nachrichten? Und aus
  welcher politischen Richtung kommen diejenigen, die die Nachrichten
  machen? Diese sind eben oftmals aus einer anderen politischen Herkunft
  zu erklären als das, was in der breiten Masse der Bevölkerung
  vorhanden ist. Ich glaube, diese Disparität kann man nicht völlig
  aufheben.« \textbf{Ernst-Jörg von Studnitz}, deutscher Botschafter in
  Moskau von 1995 bis 2002. (RTD,
  \href{https://deutsch.rt.com/europa/86144-in-konfrontation-zwischen-amerika-und/}{Interview},
  2019, Min. 30)
\item
  »Es steht zwar deutsch-amerikanische Freundschaft drauf, aber
  letztlich ist die Atlantik-Brücke eher ein Trans­missions­mecha­nismus
  für amerikanische Ideen nach Europa.« \textbf{Prof. Max Otte},
  Mitglied der Atlantik-Brücke und des American Council on Germany. (OF,
  \href{https://www.youtube.com/watch?v=9Ouns9KhQFQ}{Interview}, Min.
  10)
\end{itemize}

\hypertarget{weitere-themen}{%
\paragraph{Weitere Themen}\label{weitere-themen}}

\begin{itemize}
\tightlist
\item
  \href{https://swprs.org/das-american-empire-und-seine-medien/}{Medien
  in den USA}
\item
  \href{https://swprs.org/netzwerk-medien-schweiz/}{Medien in der
  Schweiz}
\item
  \href{https://swprs.org/medien-in-oesterreich/}{Medien in Österreich}
\end{itemize}

\begin{center}\rule{0.5\linewidth}{\linethickness}\end{center}

Beitrag teilen auf:
\href{https://twitter.com/intent/tweet?url=https://swprs.org/netzwerk-medien-deutschland/}{Twitter}
/
\href{https://www.facebook.com/share.php?u=https://swprs.org/netzwerk-medien-deutschland/}{Facebook}\\
Publiziert: Mai 2017; Aktualisiert: Mai 2020

\hypertarget{swiss-policy-research}{%
\subsubsection{Swiss Policy Research}\label{swiss-policy-research}}

\begin{itemize}
\tightlist
\item
  \href{https://swprs.org/kontakt/}{Kontakt}
\item
  \href{https://swprs.org/uebersicht/}{Übersicht}
\item
  \href{https://swprs.org/donationen/}{Donationen}
\item
  \href{https://swprs.org/disclaimer/}{Disclaimer}
\end{itemize}

\hypertarget{english}{%
\subsubsection{English}\label{english}}

\begin{itemize}
\tightlist
\item
  \href{https://swprs.org/contact/}{About Us / Contact}
\item
  \href{https://swprs.org/media-navigator/}{The Media Navigator}
\item
  \href{https://swprs.org/the-american-empire-and-its-media/}{The CFR
  and the Media}
\item
  \href{https://swprs.org/donations/}{Donations}
\end{itemize}

\hypertarget{follow-by-email}{%
\subsubsection{Follow by email}\label{follow-by-email}}

Follow

\href{https://wordpress.com/?ref=footer_custom_com}{WordPress.com}.

\protect\hyperlink{}{Up ↑}

Post to

\protect\hyperlink{}{Cancel}

\includegraphics{https://pixel.wp.com/b.gif?v=noscript}
