\protect\hyperlink{content}{Skip to content}

\href{https://swprs.org/}{}

\protect\hyperlink{search-container}{Search}

Search for:

\href{https://swprs.org/}{\includegraphics{https://swprs.files.wordpress.com/2020/05/swiss-policy-research-logo-300.png}}

\href{https://swprs.org/}{Swiss Policy Research}

Geopolitics and Media

Menu

\begin{itemize}
\tightlist
\item
  \href{https://swprs.org}{Start}
\item
  \href{https://swprs.org/srf-propaganda-analyse/}{Studien}

  \begin{itemize}
  \tightlist
  \item
    \href{https://swprs.org/srf-propaganda-analyse/}{SRF / ZDF}
  \item
    \href{https://swprs.org/die-nzz-studie/}{NZZ-Studie}
  \item
    \href{https://swprs.org/der-propaganda-multiplikator/}{Agenturen}
  \item
    \href{https://swprs.org/die-propaganda-matrix/}{Medienmatrix}
  \end{itemize}
\item
  \href{https://swprs.org/medien-navigator/}{Analysen}

  \begin{itemize}
  \tightlist
  \item
    \href{https://swprs.org/medien-navigator/}{Navigator}
  \item
    \href{https://swprs.org/der-propaganda-schluessel/}{Techniken}
  \item
    \href{https://swprs.org/propaganda-in-der-wikipedia/}{Wikipedia}
  \item
    \href{https://swprs.org/logik-imperialer-kriege/}{Kriege}
  \end{itemize}
\item
  \href{https://swprs.org/netzwerk-medien-schweiz/}{Netzwerke}

  \begin{itemize}
  \tightlist
  \item
    \href{https://swprs.org/netzwerk-medien-schweiz/}{Schweiz}
  \item
    \href{https://swprs.org/netzwerk-medien-deutschland/}{Deutschland}
  \item
    \href{https://swprs.org/medien-in-oesterreich/}{Österreich}
  \item
    \href{https://swprs.org/das-american-empire-und-seine-medien/}{USA}
  \end{itemize}
\item
  \href{https://swprs.org/bericht-eines-journalisten/}{Fokus I}

  \begin{itemize}
  \tightlist
  \item
    \href{https://swprs.org/bericht-eines-journalisten/}{Journalistenbericht}
  \item
    \href{https://swprs.org/russische-propaganda/}{Russische Propaganda}
  \item
    \href{https://swprs.org/die-israel-lobby-fakten-und-mythen/}{Die
    »Israel-Lobby«}
  \item
    \href{https://swprs.org/geopolitik-und-paedokriminalitaet/}{Pädokriminalität}
  \end{itemize}
\item
  \href{https://swprs.org/migration-und-medien/}{Fokus II}

  \begin{itemize}
  \tightlist
  \item
    \href{https://swprs.org/covid-19-hinweis-ii/}{Coronavirus}
  \item
    \href{https://swprs.org/die-integrity-initiative/}{Integrity
    Initiative}
  \item
    \href{https://swprs.org/migration-und-medien/}{Migration \& Medien}
  \item
    \href{https://swprs.org/der-fall-magnitsky/}{Magnitsky Act}
  \end{itemize}
\item
  \href{https://swprs.org/kontakt/}{Projekt}

  \begin{itemize}
  \tightlist
  \item
    \href{https://swprs.org/kontakt/}{Kontakt}
  \item
    \href{https://swprs.org/uebersicht/}{Seitenübersicht}
  \item
    \href{https://swprs.org/medienspiegel/}{Medienspiegel}
  \item
    \href{https://swprs.org/donationen/}{Donationen}
  \end{itemize}
\item
  \href{https://swprs.org/contact/}{English}
\end{itemize}

\protect\hyperlink{}{Open Search}

\hypertarget{propaganda-in-the-war-on-yugoslavia}{%
\section{Propaganda in the War
on~Yugoslavia}\label{propaganda-in-the-war-on-yugoslavia}}

\textbf{Published}: December 2019 (upd.)\\
\textbf{Languages}:
\href{https://swprs.org/propaganda-im-jugoslawienkrieg/}{DE},
\href{https://swprs.org/propaganda-in-the-war-on-yugoslavia/}{EN};
\href{https://mondo.ba/Info/Region/a923284/Trnopolje-Markale-Srebrenica-Racak-Podvale-oo-kojima-bruji-svijet.html}{BS},
\href{https://prima.iprima.cz/zpravodajstvi/svycari-o-valce-v-byvale-jugoslavii-dezinformacich-a-fake-news-medii-boju-se-ucastnila}{CZ},
\href{http://www.politika.rs/sr/clanak/444725/Na-Zapadu-se-urusava-mit-o-zlim-Srbima}{SR}

From a geopolitical perspective, the war on Yugoslavia in the 1990s was
about restructuring South­east Europe after the end of the Cold War. To
this end, the US even deployed the
\href{https://www.theguardian.com/world/2002/apr/22/warcrimes.comment}{combatants}
with which it had previously fought the USSR in Afghanistan and which it
would later call ``Al Qaeda''.

The political and media propaganda regarding the war on Yugoslavia has
been well researched by now. Interestingly, however, many media outlets
and commentators are still trying to uphold the official narrative of
the time, in contrast to the later war in Iraq, for example.

There may be various reasons for this. On the one hand, the propaganda
in question dates back to the early days of the Internet and is
therefore generally less well known to the public. On the other hand,
the implications, notably for Europe, are particularly far-reaching in
this case.

From today's perspective, it is a rather trivial statement that most
Western media outlets supported NATO's war on Yugoslavia, but at the
time even critics believed in a media ``failure'', especially because
the
\href{https://swprs.org/the-american-empire-and-its-media/}{influence}
of foreign policy groups on media reporting was not yet broadly known.

The following sections provide an overview of propaganda in the war on
Yugoslavia as well as references to further literature and
documentation. Please note that the analysis does not call into question
regional aspects of the conflict or any actual war crimes on any side of
the conflict.

\hypertarget{1-the-serbian-death-camp-1992}{%
\paragraph{1. The Serbian ``Death Camp''
(1992)}\label{1-the-serbian-death-camp-1992}}

One of the most notorious cases of propaganda concerns the alleged
Serbian ``death camp'' of
\href{https://en.wikipedia.org/wiki/Trnopolje_camp}{Trnopolje} in
Bosnia. The story began in August 1992, when three British journalists
visited a refugee camp whose inmates stressed that they were being
treated very well (see video below).

The journalists, however, went inside a fenced-in storage area right
next to the refugee camp and filmed the men on the outside through a
barbed wire fence, making it appear as if the men were imprisoned, which
in fact they were not (see site map below). The journalists then asked a
man emaciated from illness or war-related malnutrition to take off his
T-shirt.

The resulting photograph -- carefully cut to size -- landed on the front
pages of most Western media as ``proof'' of Serbian ``death camps'',
which in turn served as justification for NATO's subsequent
\href{https://en.wikipedia.org/wiki/NATO_intervention_in_Bosnia_and_Herzegovina}{intervention}
in Bosnia, starting with a no-fly zone.

The Trnopolje death camp deception was
\href{https://swprs.files.wordpress.com/2019/12/the-picture-that-fooled-the-world_thomas-deichmann_1997.pdf}{exposed}
by a German journalist in 1997. A British magazine that republished his
article got sued by the three British journalists for libel and
eventually lost the case because it couldn't prove their intent.

The head of an American PR agency that had spread the false death camp
reports later
\href{https://www.sourcewatch.org/index.php/James_Harff}{explained}:
``We are professionals. We had a job to do and we did it. We are not
paid to be moral.''

\textbf{See also}:
\href{https://swprs.files.wordpress.com/2019/12/the-picture-that-fooled-the-world_thomas-deichmann_1997.pdf}{The
original article exposing the ``death camp'' deception} (Thomas
Deichmann, 1997)

Full documentary:
\href{https://www.youtube.com/watch?v=xtQ-PJLIpcE}{Yugoslavia: The
Pictures that Fooled the World} (2000)\\
The German captions explain how the men being filmed were standing
\emph{outside} of the barbed wire fence and what type of questions the
British journalists were asking them.

\href{https://swprs.files.wordpress.com/2019/12/trnopolje_tv_screenshot.jpg}{}

\includegraphics{https://swprs.files.wordpress.com/2019/12/trnopolje_tv_screenshot.jpg?w=241\&h=241\&crop=1}

TV screenshot

\href{https://swprs.files.wordpress.com/2019/12/trnopolje_collage.jpg}{}

\includegraphics{https://swprs.files.wordpress.com/2019/12/trnopolje_collage.jpg?w=241\&h=241\&crop=1}

Front pages about Trnopolje

\href{https://swprs.files.wordpress.com/2019/12/trnopolje-site-map-1992.png}{}

\includegraphics{https://swprs.files.wordpress.com/2019/12/trnopolje-site-map-1992.png?w=241\&h=241\&crop=1}

Site map of Trnopolje camp (fenced off storage area at the bottom)

TV screenshot, press headlines and site map of the Trnopolje camp

\hypertarget{2-the-sarajevo-marketplace-massacres-1992-1995}{%
\paragraph{2. The Sarajevo Marketplace Massacres
(1992-1995)}\label{2-the-sarajevo-marketplace-massacres-1992-1995}}

Another well-known case of propaganda concerns the so-called marketplace
massacres during the
\href{https://en.wikipedia.org/wiki/Siege_of_Sarajevo}{four-year siege}
of Sarajevo, in particular the so-called bread line massacre of May 1992
and the two so-called
\href{https://en.wikipedia.org/wiki/Markale_massacres}{Markale
massacres} of February 1994 and August 1995.

These incidents allegedly took place by mortar fire from outside of the
city and often happened shortly before important political consultations
at the UN or EU. They ultimately led to a direct military
\href{https://en.wikipedia.org/wiki/Operation_Deliberate_Force}{intervention}
by NATO -- the first in its history -- and thus to a turnaround in the
Bosnian war.

In the cases mentioned above as well as some others, investigations by
officers of the UN protection mission came to the
\href{https://swprs.files.wordpress.com/2019/12/anatomy-of-a-massacre_david-binder_foreign-policy_1994.pdf}{conclusion}
that these incidents may have been carried out by the Bosnian side
itself, perhaps to influence Western public opinion (so-called false
flag attacks).

The relevant UN reports, however, were
\href{https://swprs.files.wordpress.com/2019/12/dpa_un-report-sarajevo_1996.pdf}{kept
secret}. Instead, American media --- notably CNN --- and the US
government usually claimed without delay that the respective attack had
probably been carried out by the Serbian side (see video below).

Canadian General Lewis MacKenzie, commander of the UN forces in
Sarajevo,
\href{https://archive.org/details/peacekeeperroadt0000mack/page/194}{wrote}
about the 1992 incident: ``Our people told us there were a number of
things that didn't fit. The street had been blocked off just before the
incident. Once the crowd was let in and had lined up, the media appeared
but kept their distance. The attack took place, and the media were
immediately on the scene.''

About the 1994 incident, a BBC journalist
\href{http://news.bbc.co.uk/2/hi/europe/3459965.stm}{noted} with
surprise how ``television crews were on the scene, filming within
seconds of the blast'', while UN officers and even doctors were
\href{https://swprs.files.wordpress.com/2019/12/dpa_un-report-sarajevo_1996.pdf}{prevented}
from entering the site, and all of the alleged 197 victims were carried
away within 25 minutes. Others pointed out that the market was in fact
closed at the time of the incident (see video below).

Regarding the 1995 incident, the London Sunday Times later
\href{https://swprs.files.wordpress.com/2019/12/sunday-times_serbs-not-guilty-of-massacre_1995.pdf}{revealed}
that British and French UN ammunition experts had concluded the Serbian
side was ``not guilty'', but they were then ``overruled by a senior
American officer'', and NATO air strikes began within less than 48
hours.

US professor Yossef Bodansky, the longtime director of the US
Congressional Task Force on Terrorism and Unconventional Warfare, later
\href{https://swprs.files.wordpress.com/2019/12/bodansky_offensive-in-the-balkans_1995.pdf\#page=40}{described}
these incidents as ``expertly-staged spectacle of gore'' that included
the use of ``corpses of Bosnian troops recently killed in action''.

In the following you will find the most important articles from that
time by journalists who were able to study the unpublished UN reports or
talk to officials involved in writing them.

\begin{itemize}
\tightlist
\item
  \href{https://www.independent.co.uk/news/muslims-slaughter-their-own-people-bosnia-bread-queue-massacre-was-propaganda-ploy-un-told-1541801.html}{Bosnia
  bread queue massacre was propaganda ploy, UN told}~ (Independent,
  1992)
\item
  \href{https://swprs.files.wordpress.com/2019/12/dateline-yugoslavia-the-partisan-press_peter-brock_foreign-policy_1994.pdf}{Dateline
  Yugoslavia: The Partisan Press} (Peter Brock, Foreign Policy, 1994,
  archived)
\item
  \href{https://swprs.files.wordpress.com/2019/12/anatomy-of-a-massacre_david-binder_foreign-policy_1994.pdf}{Anatomy
  of a Massacre} (David Binder, Foreign Policy, 1994, about Markale I,
  archived)
\item
  \href{https://swprs.files.wordpress.com/2019/12/bosnias-bombers_david-binder_the-nation-1995.pdf}{Bosnia's
  Bombers} (David Binder, The Nation, 1995, about Markale II, archived)
\item
  \href{https://swprs.files.wordpress.com/2019/12/sunday-times_serbs-not-guilty-of-massacre_1995.pdf}{``Serbs
  not guilty of massacre''} (Hugh McManners, The Sunday Times, 1995,
  archived)
\item
  \href{https://swprs.files.wordpress.com/2019/12/dpa_un-report-sarajevo_1996.pdf}{Senior
  official admits to secret U.N. report on Sarajevo massacre} (DPA,
  1996, archived)
\item
  2004: \href{http://news.bbc.co.uk/2/hi/europe/3459965.stm}{Sarajevo
  massacre remembered} (BBC; see quote by General Michael Rose)
\end{itemize}

Twenty years later, the Bosnian Markale massacres of 1994/95 were
\href{https://swprs.files.wordpress.com/2019/12/sarajevo-1995-damscus-2013-mass-attack-deceptions_world-tribune.pdf}{recalled}
when poison gas attacks during the Syrian war turned out to be
questionable and the results of UN and OPCW investigations were again
\href{https://www.dailymail.co.uk/news/article-7793253/PETER-HITCHENS-reveals-evidence-watchdog-suppressed-report-casting-doubt-Assad-gas-attack.html}{suppressed}
to justify military strikes by NATO countries.

The 1994 Markale incident\\
Source: BBC,
\href{https://www.youtube.com/playlist?list=PLJvRFxihL4d03IzmoxyhU1C-kn27lxVvB}{The
Death of Yugoslavia}, 1995

\hypertarget{3-the-genocide-of-srebrenica-1995}{%
\paragraph{3. The ``Genocide of Srebrenica''
(1995)}\label{3-the-genocide-of-srebrenica-1995}}

The ``Genocide of Srebrenica'' in July 1995 is regarded as the sad
climax of the Bosnian war. According to Western accounts originally
based on a
\href{https://www.nytimes.com/1995/08/11/world/us-seeks-to-prove-mass-killings.html}{report}
by the US government, more than 8000 Bosnian civilians may have been
killed.

But according to Phillip Corwin, the highest-ranking UN civilian
official in Bosnia during the war, the actual evidence points to a more
complex situation and a somewhat different context. Corwin
\href{https://swprs.files.wordpress.com/2019/12/the-srebrenica-massacre_edward-herman_2011.pdf\#page=7}{called}
the official Western account of events in Srebrenica a ``distortion''.

The late political scientist Edward S. Herman and the former CIA officer
Robert Baer, who was operating in Yugoslavia during the war, even
\href{https://www.globalresearch.ca/the-politics-of-the-srebrenica-massacre/660}{spoke}
of a ``fraud'' in this regard.

For further details, please refer to the following articles and
documentaries:

\begin{itemize}
\tightlist
\item
  \href{https://www.globalresearch.ca/the-politics-of-the-srebrenica-massacre/660}{The
  Politics of the Srebrenica Massacre} (Edward S. Herman, Global
  Research, 2005)
\item
  \href{https://www.counterpunch.org/2005/10/12/srebrenica-revisited/}{Srebrenica
  Revisited} (Diana Johnstone, Counterpunch Magazine, 2005)
\item
  \href{https://www.youtube.com/watch?v=FvqHWS_4AuM}{Srebrenica: A Town
  Betrayed} (Norwegian documentary, 60 minutes, 2010)
\item
  \href{https://swprs.files.wordpress.com/2019/12/the-srebrenica-massacre_edward-herman_2011.pdf}{The
  Srebrenica Massacre: Evidence, Context, Politics} (Edward S. Herman,
  ed., 2011)
\item
  Video: \href{https://www.youtube.com/watch?v=KmkV7zHiHoQ}{Evacuation
  of Srebrenica refugees} (AP, original footage of July 12, 1995)
\end{itemize}

In general, even events with very high reported victim numbers must
sometimes be critically examined. This was shown, for example, by the
\href{https://www.france24.com/en/20091220-twenty-years-later-timisoara-affair-exposes-media-credulity}{``Timisoara
Massacre''} of 1989 with allegedly 4630 dead, which later turned out to
be a psychological operation to launch the Romanian revolution.

Srebrenica: A Town Betrayed (Norwegian documentary, 60m, 2010,
\href{https://en.wikipedia.org/wiki/A_Town_Betrayed}{Wikipedia})

\hypertarget{4-kosovo-operation-horseshoe-raux10dak-massacre-and-more-1999}{%
\paragraph{4. Kosovo: ``Operation Horseshoe'', ``Račak massacre'', and
more
(1999)}\label{4-kosovo-operation-horseshoe-raux10dak-massacre-and-more-1999}}

After the separation of Slovenia, Croatia and Bosnia from Yugoslavia,
the US and NATO started another war in 1999 against the remainder of
Yugoslavia to additionally separate the province of Kosovo from Serbia.
This war again had to be justified by propaganda and disinformation.

In particular, politicians and the media discussed alleged expulsion
plans, concentration camps and massacres, which later turned out to be
fabricated or questionable. Examples include the made-up
\href{https://en.wikipedia.org/wiki/Operation_Horseshoe}{``Operation
Horseshoe''} (to expulse Albanians) and the alleged
\href{https://en.wikipedia.org/wiki/Ra\%C4\%8Dak_massacre}{``Račak
massacre''}.

In the case of Račak, Finnish forensic experts later
\href{https://swprs.files.wordpress.com/2019/12/racak-massacre_peter-worthington_toronto-sun_2001.pdf}{concluded}
that the bodies of KLA figthers killed in action had been moved,
redressed, and presented as civilian victims of an execution.

After the war, the head of an American public relations agency that had
spread such dubious stories about the situation in Kosovo
\href{https://www.hintergrund.de/globales/kriege/operation-balkan-werbung-fuer-krieg-und-tod/}{stated}
in an interview: ``To be honest, when NATO finally attacked in 1999, we
opened a bottle of champagne.''

For further details, the German documentary ``It Began with a Lie'' from
2001 will be shown below (English subtitles available). The documentary
shows how Western governments deliberately published false information
in order to legitimize the war.

\textbf{See also:}

\begin{itemize}
\tightlist
\item
  \href{https://swprs.files.wordpress.com/2019/12/meet-mister-massacre_ames-taibbi_the-exile_2000.pdf}{Meet
  Mr. Massacre} (Mark Ames and Matt Taibbi, The Exile, 2000, archived)
\item
  \href{https://swprs.files.wordpress.com/2019/12/racak-massacre_peter-worthington_toronto-sun_2001.pdf}{The
  hoax that started a war} (Peter Worthington, Toronto Sun, 2001,
  archived)
\item
  \href{https://swprs.files.wordpress.com/2019/12/kosovo_operation-horseshoe-fake_sunday-times_2000.pdf}{Serbian
  ethnic cleansing scare was a fake, says general} (Sunday Times, 2000,
  archived)
\item
  \href{https://www.globalresearch.ca/correspondence-between-german-politicians-reveals-the-hidden-agenda-behind-kosovo-s-independence/8304}{German
  politician Willy Wimmer on the US strategy in the Balkans} (GRC, 2008)
\end{itemize}

Kosovo War: It Began with a Lie (German documentary, 2001)

\hypertarget{additional-references}{%
\paragraph{Additional References}\label{additional-references}}

\begin{itemize}
\tightlist
\item
  Larry E. Craig (1997):
  \href{https://web.archive.org/web/20110206110107/http://rpc.senate.gov/releases/1997/iran.htm}{US
  Senate report discussing US strategy in the Balkans}
\item
  Diana Johnstone (2003):
  \href{https://www.amazon.com/Fools-Crusade-Yugoslavia-Delusions-Illusions/dp/0745319505}{Fools'
  Crusade: Yugoslavia, NATO and Western Delusions}
\item
  Peter Brock (2006):
  \href{https://www.amazon.com/Media-Cleansing-Reporting-Peter-Brock-ebook-dp-B005VGNROO/dp/B005VGNROO}{Media
  Cleansing: Dirty Reporting. Journalism and Tragedy in Yugoslavia}
\item
  Edward Herman and David Peterson (2007):
  \href{https://monthlyreview.org/2007/10/01/the-dismantling-of-yugoslavia/}{The
  Dismantling of Yugoslavia}
\item
  \textbf{Video}: Pierre Gallois (2009):
  \href{https://archive.org/details/french-general-pierre-marie-gallois-on-yugoslavia-war-2009}{French
  General on Western strategy in the Balkans}
\end{itemize}

\hypertarget{see-also}{%
\paragraph{See also}\label{see-also}}

\begin{itemize}
\tightlist
\item
  \href{https://swprs.org/us-foreign-policy/}{The Logic of US Foreign
  Policy}
\item
  \href{https://swprs.org/rwanda-what-did-really-happen-in-1994/}{Rwanda:
  What did really happen?}
\item
  \href{https://swprs.org/the-syria-deception/}{The Syria Deception}
\end{itemize}

\begin{center}\rule{0.5\linewidth}{\linethickness}\end{center}

Published: December 2019 (updated)\\
Share this study on:
\href{https://twitter.com/intent/tweet?url=https://swprs.org/propaganda-in-the-war-on-yugoslavia/}{Twitter}
/
\href{https://www.facebook.com/share.php?u=https://swprs.org/propaganda-in-the-war-on-yugoslavia/}{Facebook}

\hypertarget{swiss-policy-research}{%
\subsubsection{Swiss Policy Research}\label{swiss-policy-research}}

\begin{itemize}
\tightlist
\item
  \href{https://swprs.org/kontakt/}{Kontakt}
\item
  \href{https://swprs.org/uebersicht/}{Übersicht}
\item
  \href{https://swprs.org/donationen/}{Donationen}
\item
  \href{https://swprs.org/disclaimer/}{Disclaimer}
\end{itemize}

\hypertarget{english}{%
\subsubsection{English}\label{english}}

\begin{itemize}
\tightlist
\item
  \href{https://swprs.org/contact/}{About Us / Contact}
\item
  \href{https://swprs.org/media-navigator/}{The Media Navigator}
\item
  \href{https://swprs.org/the-american-empire-and-its-media/}{The CFR
  and the Media}
\item
  \href{https://swprs.org/donations/}{Donations}
\end{itemize}

\hypertarget{follow-by-email}{%
\subsubsection{Follow by email}\label{follow-by-email}}

Follow

\href{https://wordpress.com/?ref=footer_custom_com}{WordPress.com}.

\protect\hyperlink{}{Up ↑}

Post to

\protect\hyperlink{}{Cancel}

\includegraphics{https://pixel.wp.com/b.gif?v=noscript}
