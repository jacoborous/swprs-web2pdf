\protect\hyperlink{content}{Skip to content}

\href{https://swprs.org/}{}

\protect\hyperlink{search-container}{Search}

Search for:

\href{https://swprs.org/}{\includegraphics{https://swprs.files.wordpress.com/2020/05/swiss-policy-research-logo-300.png}}

\href{https://swprs.org/}{Swiss Policy Research}

Geopolitics and Media

Menu

\begin{itemize}
\tightlist
\item
  \href{https://swprs.org}{Start}
\item
  \href{https://swprs.org/srf-propaganda-analyse/}{Studien}

  \begin{itemize}
  \tightlist
  \item
    \href{https://swprs.org/srf-propaganda-analyse/}{SRF / ZDF}
  \item
    \href{https://swprs.org/die-nzz-studie/}{NZZ-Studie}
  \item
    \href{https://swprs.org/der-propaganda-multiplikator/}{Agenturen}
  \item
    \href{https://swprs.org/die-propaganda-matrix/}{Medienmatrix}
  \end{itemize}
\item
  \href{https://swprs.org/medien-navigator/}{Analysen}

  \begin{itemize}
  \tightlist
  \item
    \href{https://swprs.org/medien-navigator/}{Navigator}
  \item
    \href{https://swprs.org/der-propaganda-schluessel/}{Techniken}
  \item
    \href{https://swprs.org/propaganda-in-der-wikipedia/}{Wikipedia}
  \item
    \href{https://swprs.org/logik-imperialer-kriege/}{Kriege}
  \end{itemize}
\item
  \href{https://swprs.org/netzwerk-medien-schweiz/}{Netzwerke}

  \begin{itemize}
  \tightlist
  \item
    \href{https://swprs.org/netzwerk-medien-schweiz/}{Schweiz}
  \item
    \href{https://swprs.org/netzwerk-medien-deutschland/}{Deutschland}
  \item
    \href{https://swprs.org/medien-in-oesterreich/}{Österreich}
  \item
    \href{https://swprs.org/das-american-empire-und-seine-medien/}{USA}
  \end{itemize}
\item
  \href{https://swprs.org/bericht-eines-journalisten/}{Fokus I}

  \begin{itemize}
  \tightlist
  \item
    \href{https://swprs.org/bericht-eines-journalisten/}{Journalistenbericht}
  \item
    \href{https://swprs.org/russische-propaganda/}{Russische Propaganda}
  \item
    \href{https://swprs.org/die-israel-lobby-fakten-und-mythen/}{Die
    »Israel-Lobby«}
  \item
    \href{https://swprs.org/geopolitik-und-paedokriminalitaet/}{Pädokriminalität}
  \end{itemize}
\item
  \href{https://swprs.org/migration-und-medien/}{Fokus II}

  \begin{itemize}
  \tightlist
  \item
    \href{https://swprs.org/covid-19-hinweis-ii/}{Coronavirus}
  \item
    \href{https://swprs.org/die-integrity-initiative/}{Integrity
    Initiative}
  \item
    \href{https://swprs.org/migration-und-medien/}{Migration \& Medien}
  \item
    \href{https://swprs.org/der-fall-magnitsky/}{Magnitsky Act}
  \end{itemize}
\item
  \href{https://swprs.org/kontakt/}{Projekt}

  \begin{itemize}
  \tightlist
  \item
    \href{https://swprs.org/kontakt/}{Kontakt}
  \item
    \href{https://swprs.org/uebersicht/}{Seitenübersicht}
  \item
    \href{https://swprs.org/medienspiegel/}{Medienspiegel}
  \item
    \href{https://swprs.org/donationen/}{Donationen}
  \end{itemize}
\item
  \href{https://swprs.org/contact/}{English}
\end{itemize}

\protect\hyperlink{}{Open Search}

\hypertarget{weiteres-wikipedia-urteil}{%
\section{Weiteres Wikipedia-Urteil}\label{weiteres-wikipedia-urteil}}

26. Februar 2019*\\
*

\textbf{BG Münchwilen: Schweizer Tierschützer gewinnt weiteren Prozess
gegen Wikimedia.}

Ein Schweizer Tierschützer
\href{https://www.vgt.ch/news/190224-wikipedia.htm}{gewann} Mitte
Februar bereits den dritten Prozess gegen Wikimedia aufgrund von
ehrverletzenden Behauptungen auf seiner
\href{https://de.wikipedia.org/wiki/Erwin_Kessler}{Wikipedia-Seite}.
Wikimedia muss die fraglichen Behauptungen nun durch eine sogenannte
\href{https://de.wikipedia.org/wiki/Wikipedia:Office_Action}{\emph{»Office
Action«}} definitiv entfernen.

Aus dem publizierten
\href{https://www.vgt.ch/doc/hetzkampagne/190214-wikipedia-urteil.pdf}{Gerichtsurteil}
geht hervor, dass mehrere falsche und irreführende Aussagen aus dem
Artikel entfernt werden müssen. Der Tierschützer wurde in der Wikipedia
unter anderem aufgrund seiner Schächtkritik als Rassist und Antisemit
dargestellt.

Klagen gegen Wikimedia müssen im Allgemeinen nicht bei einem US-Gericht
eingereicht werden, sondern können am Wohn- bzw. Firmensitz des
Betroffenen unter Berufung auf das Internationale Privatrecht
\href{https://www.watson.ch/schweiz/digital/989282505-tierschuetzer-kessler-knoepft-sich-facebook-und-wikipedia-vor-und-gewinnt}{erhoben}
werden. Die Kosten hierfür sind, im Erfolgsfall, von Wikimedia zu
tragen.

Seit dem \href{https://swprs.org/der-wikipedia-prozess/}{Präzedenzfall
»Feliks«} haben Betroffene zudem die Möglichkeit, juristisch unmittelbar
gegen denunziativ agierende Wikipedia-Autoren vorzugehen. Im Falle des
Tierschützers handelt es sich dabei um einen Ostschweizer mit dem
Decknamen
\href{https://de.wikipedia.org/w/index.php?title=Spezial:Beitr\%C3\%A4ge/Rocky187\&offset=\&limit=500\&target=Rocky187}{»Rocky187«},
der sich auf Wikipedia seit 2010 hauptsächlich um historische,
politische und israelische Themen gekümmert hat.

Bereits im August 2018 wurde Wikipedia durch das Landgericht Berlin das
sogenannte »Laienprivileg«
\href{https://www.heise.de/newsticker/meldung/Urteil-gegen-Wikipedia-Keine-rufschaedigende-Kritik-ohne-Recherche-4209610.html}{aberkannt}.
Dies bedeutet, dass Wikipedia-Autoren rufschädigende Behauptungen aus
Medienbeiträgen oder anderen Quellen nicht mehr ungeprüft übernehmen
dürfen.

\emph{Das Urteil des Bezirksgerichts Münchwilen kann noch vor das
Obergericht weitergezogen werden.}

\hypertarget{siehe-auch}{%
\paragraph{Siehe auch:}\label{siehe-auch}}

\begin{itemize}
\tightlist
\item
  \href{https://swprs.org/propaganda-in-der-wikipedia/}{Propaganda in
  der Wikipedia}
\item
  \href{https://swprs.org/wikipedia-missbrauch-massnahmen/}{WP:
  Maßnahmen bei Missbrauch}
\item
  \href{https://swprs.org/wikipedia-manipulation-autoren/}{WP:
  Ideologisch agierende Autoren}
\end{itemize}

\begin{center}\rule{0.5\linewidth}{\linethickness}\end{center}

Beitrag teilen auf:
\href{https://twitter.com/intent/tweet?url=https://swprs.org/weiteres-urteil-im-fall-wikipedia/}{Twitter}
/
\href{https://www.facebook.com/share.php?u=https://swprs.org/weiteres-urteil-im-fall-wikipedia/}{Facebook}

\hypertarget{swiss-policy-research}{%
\subsubsection{Swiss Policy Research}\label{swiss-policy-research}}

\begin{itemize}
\tightlist
\item
  \href{https://swprs.org/kontakt/}{Kontakt}
\item
  \href{https://swprs.org/uebersicht/}{Übersicht}
\item
  \href{https://swprs.org/donationen/}{Donationen}
\item
  \href{https://swprs.org/disclaimer/}{Disclaimer}
\end{itemize}

\hypertarget{english}{%
\subsubsection{English}\label{english}}

\begin{itemize}
\tightlist
\item
  \href{https://swprs.org/contact/}{About Us / Contact}
\item
  \href{https://swprs.org/media-navigator/}{The Media Navigator}
\item
  \href{https://swprs.org/the-american-empire-and-its-media/}{The CFR
  and the Media}
\item
  \href{https://swprs.org/donations/}{Donations}
\end{itemize}

\hypertarget{follow-by-email}{%
\subsubsection{Follow by email}\label{follow-by-email}}

Follow

\href{https://wordpress.com/?ref=footer_custom_com}{WordPress.com}.

\protect\hyperlink{}{Up ↑}

\includegraphics{https://pixel.wp.com/b.gif?v=noscript}
