\protect\hyperlink{content}{Skip to content}

\href{https://swprs.org/}{}

\protect\hyperlink{search-container}{Search}

Search for:

\href{https://swprs.org/}{\includegraphics{https://swprs.files.wordpress.com/2020/05/swiss-policy-research-logo-300.png}}

\href{https://swprs.org/}{Swiss Policy Research}

Geopolitics and Media

Menu

\begin{itemize}
\tightlist
\item
  \href{https://swprs.org}{Start}
\item
  \href{https://swprs.org/srf-propaganda-analyse/}{Studien}

  \begin{itemize}
  \tightlist
  \item
    \href{https://swprs.org/srf-propaganda-analyse/}{SRF / ZDF}
  \item
    \href{https://swprs.org/die-nzz-studie/}{NZZ-Studie}
  \item
    \href{https://swprs.org/der-propaganda-multiplikator/}{Agenturen}
  \item
    \href{https://swprs.org/die-propaganda-matrix/}{Medienmatrix}
  \end{itemize}
\item
  \href{https://swprs.org/medien-navigator/}{Analysen}

  \begin{itemize}
  \tightlist
  \item
    \href{https://swprs.org/medien-navigator/}{Navigator}
  \item
    \href{https://swprs.org/der-propaganda-schluessel/}{Techniken}
  \item
    \href{https://swprs.org/propaganda-in-der-wikipedia/}{Wikipedia}
  \item
    \href{https://swprs.org/logik-imperialer-kriege/}{Kriege}
  \end{itemize}
\item
  \href{https://swprs.org/netzwerk-medien-schweiz/}{Netzwerke}

  \begin{itemize}
  \tightlist
  \item
    \href{https://swprs.org/netzwerk-medien-schweiz/}{Schweiz}
  \item
    \href{https://swprs.org/netzwerk-medien-deutschland/}{Deutschland}
  \item
    \href{https://swprs.org/medien-in-oesterreich/}{Österreich}
  \item
    \href{https://swprs.org/das-american-empire-und-seine-medien/}{USA}
  \end{itemize}
\item
  \href{https://swprs.org/bericht-eines-journalisten/}{Fokus I}

  \begin{itemize}
  \tightlist
  \item
    \href{https://swprs.org/bericht-eines-journalisten/}{Journalistenbericht}
  \item
    \href{https://swprs.org/russische-propaganda/}{Russische Propaganda}
  \item
    \href{https://swprs.org/die-israel-lobby-fakten-und-mythen/}{Die
    »Israel-Lobby«}
  \item
    \href{https://swprs.org/geopolitik-und-paedokriminalitaet/}{Pädokriminalität}
  \end{itemize}
\item
  \href{https://swprs.org/migration-und-medien/}{Fokus II}

  \begin{itemize}
  \tightlist
  \item
    \href{https://swprs.org/covid-19-hinweis-ii/}{Coronavirus}
  \item
    \href{https://swprs.org/die-integrity-initiative/}{Integrity
    Initiative}
  \item
    \href{https://swprs.org/migration-und-medien/}{Migration \& Medien}
  \item
    \href{https://swprs.org/der-fall-magnitsky/}{Magnitsky Act}
  \end{itemize}
\item
  \href{https://swprs.org/kontakt/}{Projekt}

  \begin{itemize}
  \tightlist
  \item
    \href{https://swprs.org/kontakt/}{Kontakt}
  \item
    \href{https://swprs.org/uebersicht/}{Seitenübersicht}
  \item
    \href{https://swprs.org/medienspiegel/}{Medienspiegel}
  \item
    \href{https://swprs.org/donationen/}{Donationen}
  \end{itemize}
\item
  \href{https://swprs.org/contact/}{English}
\end{itemize}

\protect\hyperlink{}{Open Search}

\hypertarget{ux3b3ux3b5ux3b3ux3bfux3bdux3ccux3c4ux3b1-ux3b3ux3cdux3c1ux3c9-ux3b1ux3c0ux3cc-ux3c4ux3bfux3bd-covid-19}{%
\section{Γεγονότα γύρω από
τον~Covid-19}\label{ux3b3ux3b5ux3b3ux3bfux3bdux3ccux3c4ux3b1-ux3b3ux3cdux3c1ux3c9-ux3b1ux3c0ux3cc-ux3c4ux3bfux3bd-covid-19}}

\textbf{Επικαιροποιημένο}: 18. Μάιος 2020; \textbf{Δημοσιεύτηκε}: 14
Μάρτη 2020\\
\textbf{Γλώσσες}: \href{https://swprs.org/fakta-o-covid-19/}{CZ},
\href{https://swprs.org/covid-19-hinweis-ii/}{DE},
\href{https://swprs.org/a-swiss-doctor-on-covid-19/}{EN},
\href{https://swprs.org/coronavirus-un-medecin-suisse-parle/}{FR},
\href{https://swprs.org/hechos-sobre-covid-19/}{ES},
\href{https://swprs.org/faktoja-covid-19sta/}{FI},
\href{https://swprs.org/facts-about-covid19-greek/}{GR},
\href{https://swprs.org/covid-19-cinjenice/}{HBS},
\href{https://yanivhamo.com/facts-about-covid-19-hebrew/}{HE},
\href{https://swprs.org/egy-svajci-orvos-a-covid-19-rol/}{HU},
\href{https://swprs.org/un-medico-svizzero-su-covid-19/}{IT},
\href{https://swprs.org/covid19-facts-japanese/}{JP},
\href{https://swprs.org/covid19-korean/}{KO},
\href{https://www.globalinfo.nl/Achtergrond/een-kritische-kijk-op-het-coronabeleid-transparantie-in-tijden-van-crisis}{NL},
\href{https://midtifleisen.wordpress.com/2020/03/14/en-sveitsisk-lege-om-covid-19/}{NO},
\href{https://swprs.org/szwajcarski-lekarz-o-covid-19/}{PL},
\href{https://swprs.org/fatos-sobre-covid-19/}{PT},
\href{https://swprs.org/informatii-despre-covid-19/}{RO},
\href{https://swprs.org/\%d0\%bd\%d0\%b0-\%d0\%ba\%d0\%be\%d0\%b2\%d0\%b8\%d0\%b4-19/}{RU},
\href{https://swprs.org/fakta-om-covid-19/}{SE},
\href{http://www.ninamvseeno.org/pregled-clanka.aspx?naslov=pomembne-informacije-o-novem-koronavirusu-covid-19\&id=148}{SI},
\href{https://alatyr.sk/covid-19_swiss_propaganda_research.htm}{SK},
\href{https://swprs.org/isvicreli-bir-doktordan-kovid-19-uezerine/}{TR}\\
\textbf{Μοιραστείτε το}:
\href{https://twitter.com/intent/tweet?url=https://swprs.org/facts-about-covid19-greek/}{Twitter}
/
\href{https://www.facebook.com/share.php?u=https://swprs.org/facts-about-covid19-greek/}{Facebook}

Πλήρης αναφορά γεγονότων σχετικά με τον Covid-19, που παρέχονται από
ειδικούς στον τομέα, για να βοηθήσουν τους αναγνώστες μας να κάνουν μια
ρεαλιστική αξιολόγηση του κινδύνου.

\textbf{«Το μόνο μέσο για την καταπολέμηση της πανούκλας είναι η
ειλικρίνεια.»\\
Albert Camus, H πανούκλα (1947)}

\hypertarget{ux3c3ux3cdux3bdux3bfux3c8ux3b7}{%
\paragraph{Σύνοψη}\label{ux3c3ux3cdux3bdux3bfux3c8ux3b7}}

\begin{enumerate}
\def\labelenumi{\arabic{enumi}.}
\tightlist
\item
  Σύμφωνα με στοιχεία από τις χώρες και τις περιοχές με τις καλύτερες
  μελέτες, η θνησιμότητα του Covid19 κυμαίνεται μεταξύ
  \href{https://swprs.org/studies-on-covid-19-lethality/}{0,1\% και
  0,4\%,} η οποία είναι της τάξεως μιας σοβαρής
  \href{https://www.ebm-netzwerk.de/en/publications/covid-19}{γρίπης}
  και έως και τριάντα φορές χαμηλότερη από την αρχική υπόθεση του ΠΟΥ.
\item
  Ακόμα και στα παγκόσμια «hotspots», ο κίνδυνος θανάτου για τον γενικό
  πληθυσμό σχολικής και εργασιακής ηλικίας είναι συνήθως της τάξεως της
  \href{https://www.medrxiv.org/content/10.1101/2020.04.05.20054361v1}{καθημερινής
  βόλτας με το αυτοκίνητο} για τη δουλειά. Ο κίνδυνος αρχικά
  υπερεκτιμήθηκε, επειδή πολλοί άνθρωποι με μόνο ήπια ή καθόλου
  συμπτώματα δεν ελήφθησαν υπόψη.
\item
  Έως και το 80\% όλων των θετικών στα τεστ ατόμων
  \href{https://www.bmj.com/content/369/bmj.m1375}{παραμένουν χωρίς
  συμπτώματα}. Ακόμα και μεταξύ 70-79 ετών,
  \href{https://www.niid.go.jp/niid/en/2019-ncov-e/9407-covid-dp-fe-01.html}{περίπου
  το 60\%} παραμένει χωρίς συμπτώματα. Πάνω από το 95\% όλων των ατόμων
  εμφανίζουν το πολύ
  \href{https://swprs.org/studies-on-covid-19-lethality/\#hospitalizations}{ήπια
  συμπτώματα}.
\item
  Έως και το ένα τρίτο όλων των ατόμων έχουν ήδη κάποια
  \href{https://www.medrxiv.org/content/10.1101/2020.04.17.20061440v1}{βασική
  ανοσία} στον Covid19 λόγω επαφής με προηγούμενους κορωναϊούς (δηλαδή
  ιούς κοινού κρυολογήματος).
\item
  Η μέση ή κατά μέσο όρο ηλικία των αποθανόντων στις περισσότερες χώρες
  (συμπεριλαμβανομένης της
  \href{https://www.epicentro.iss.it/coronavirus/sars-cov-2-decessi-italia}{Ιταλίας})
  είναι άνω των 80 ετών και
  \href{https://www.bloomberg.com/news/articles/2020-03-18/99-of-those-who-died-from-virus-had-other-illness-italy-says}{μόνο
  περίπου στο 1\%} των αποθανόντων δεν προϋπήρχαν σοβαρές ασθένειες. Η
  ηλικία και το προφίλ κινδύνου των θανάτων αντιστοιχεί ουσιαστικά στην
  \href{https://www.vienna.at/analyse-zeigt-covid-19-opferkurve-entspricht-normaler-mortalitaet/6581246}{κανονική
  θνησιμότητα}.
\item
  Στις περισσότερες χώρες, το 50\% έως 70\% όλων των επιπλέον θανάτων
  σημειώθηκαν σε
  \href{https://ltccovid.org/2020/04/12/mortality-associated-with-covid-19-outbreaks-in-care-homes-early-international-evidence/}{γηροκομεία},
  τα οποία δεν επωφελούνται από το γενικό κλείδωμα. Επιπλέον, σε πολλές
  περιπτώσεις
  \href{https://www.hsj.co.uk/commissioning/thousands-of-extra-deaths-outside-hospital-not-attributed-to-covid-19/7027459.article}{δεν
  είναι σαφές}, εάν αυτοί οι άνθρωποι πέθαναν πραγματικά από τον Covid19
  ή
  \href{https://www.nytimes.com/2020/04/16/world/canada/montreal-nursing-homes-coronavirus.html}{από
  υπερβολικό άγχος}, φόβο και μοναξιά.
\item
  Έως και το 50\% όλων των επιπλέον θανάτων μπορεί να έχουν προκληθεί
  \href{https://www.thetimes.co.uk/edition/news/coronavirus-record-weekly-death-toll-as-fearful-patients-avoid-hospitals-bm73s2tw3}{όχι
  από τον Covid19}, αλλά από τις συνέπειες του
  \href{https://www.telegraph.co.uk/global-health/science-and-disease/two-new-waves-deaths-break-nhs-new-analysis-warns/}{κλειδώματος,
  του πανικού και του φόβου}. Για παράδειγμα, η θεραπεία καρδιακών
  προσβολών και εγκεφαλικών επεισοδίων
  \href{https://www.nytimes.com/2020/04/06/well/live/coronavirus-doctors-hospitals-emergency-care-heart-attack-stroke.html}{μειώθηκε}
  έως και 60\% επειδή πολλοί ασθενείς δεν τολμούσαν πλέον να πάνε στο
  νοσοκομείο.
\item
  Ακόμη και στους λεγόμενους «θανάτους από Covid19», συχνά
  \href{https://spectator.us/understand-report-figures-covid-deaths/}{δεν
  είναι σαφές} εάν πέθαναν οι άνθρωποι από ή με κορωναϊό (δηλαδή από
  υποκείμενες ασθένειες) ή αν θεωρήθηκαν ως
  \href{https://www.youtube.com/watch?v=V0lIWZpiRU0}{«υποτιθέμενα
  κρούσματα}» χωρίς να έχουν ελεγχθεί καθόλου. Ωστόσο, οι επίσημοι
  αριθμοί συνήθως
  \href{https://www.hsj.co.uk/coronavirus/systematic-reviews-to-discover-true-cause-of-outbreak-deaths/7027491.article}{δεν
  αντικατοπτρίζουν} αυτή τη διάκριση.
\item
  Πολλές αναφορές των μέσων ενημέρωσης για νέους και υγιείς ανθρώπους
  που πέθαναν από το Covid19 αποδείχθηκαν ψευδείς: πολλοί από αυτούς
  τους νέους είτε
  \href{https://www.hsj.co.uk/coronavirus/systematic-reviews-to-discover-true-cause-of-outbreak-deaths/7027491.article}{δεν
  πέθαναν} από το Covid19, είτε ήταν ήδη
  \href{https://sports.yahoo.com/spanish-football-coach-francisco-garcia-163153573.html}{σοβαρά
  άρρωστοι} (π.χ. από μη διαγνωσμένη λευχαιμία) ή στην πραγματικότητα
  ήταν
  \href{https://www.tagesanzeiger.ch/bund-muss-in-seiner-todesfallstatistik-fehler-korrigieren-584308129723}{109
  αντί 9 χρονών}.
\item
  Η φυσιολογική συνολική θνησιμότητα ανά ημέρα είναι περίπου
  \href{https://www.cdc.gov/mmwr/volumes/68/wr/mm6826a5.htm}{8000 άτομα}
  στις ΗΠΑ, περίπου 2600 στη Γερμανία και περίπου 1800 στην Ιταλία. Η
  θνησιμότητα της γρίπης ανά σεζόν
  είναι\href{https://www.statnews.com/2018/09/26/cdc-us-flu-deaths-winter/}{έως
  80.000} στις ΗΠΑ και
  \href{https://www.sciencedirect.com/science/article/pii/S1201971219303285}{έως
  25.000} στη Γερμανία και την Ιταλία. Σε αρκετές χώρες, οι θάνατοι από
  Covid19 \href{https://www.euromomo.eu/graphs-and-maps/}{παρέμειναν πιο
  κάτω} από τις εποχές έντονης γρίπης.
\item
  Οι τοπικέ αυξήσεις στη θνησιμότητα μπορεί να επηρεαστούν από
  πρόσθετους παράγοντες κινδύνου, όπως
  \href{https://www.theguardian.com/environment/2020/apr/20/air-pollution-may-be-key-contributor-to-covid-19-deaths-study?utm_medium}{υψηλά
  επίπεδα ατμοσφαιρικής ρύπανσης} και
  \href{https://www.ansa.it/english/news/science_tecnology/2019/11/19/italy-top-in-eu-in-antibiotic-resistance_369e0123-0107-445e-8c17-f11932c9d27c.html}{μικροβιακής
  μόλυνσης}, καθώς και
  \href{https://swprs.org/covid-19-a-report-from-italy/}{κατάρρευση στο
  σύστημα φροντίδας των ηλικιωμένων και των ασθενών λόγω} λοιμώξεων,
  μαζικού πανικού και κλειδώματος. Οι ειδικοί
  \href{https://www.ecdc.europa.eu/sites/default/files/documents/COVID-19-safe-handling-of-bodies-or-persons-dying-from-COVID19.pdf}{κανονισμοί}
  για την αντιμετώπιση των αποθανόντων είχαν ως αποτέλεσμα μερικές φορές
  να δημιουργηθούν πρόσθετα σημεία συμφόρησης στις υπηρεσίες κηδείας ή
  αποτέφρωσης.
\item
  Σε χώρες όπως η Ιταλία και η Ισπανία, και σε κάποιο βαθμό το Ηνωμένο
  Βασίλειο και οι ΗΠΑ, η υπερφόρτωση των νοσοκομείων λόγω ισχυρών
  κυμάτων γρίπης
  \href{https://off-guardian.org/2020/04/02/coronavirus-fact-check-1-flu-doesnt-overwhelm-our-hospitals/}{δεν
  είναι ασυνήθιστη}. Επιπλέον, έως και το 15\% των γιατρών και των
  εργαζομένων στον τομέα της υγείας
  \href{https://www.reuters.com/article/us-health-coronavirus-spain-morgue-idUSKBN21B1PP}{τέθηκαν
  σε καραντίνα}, ακόμη και αν δεν εμφάνισαν συμπτώματα.
\item
  Οι συχνά εμφανιζόμενες εκθετικές καμπύλες των «περιπτώσεων κορώνα»
  είναι
  \href{https://fivethirtyeight.com/features/coronavirus-case-counts-are-meaningless/}{παραπλανητικές},
  καθώς ο αριθμός των δοκιμών αυξήθηκε επίσης εκθετικά. Στις
  περισσότερες χώρες, ο λόγος των θετικών τεστ προς τα συνολικά τεστ
  (δηλαδή το θετικό ποσοστό) παρέμεινε
  \href{https://swprs.org/rate-of-positive-covid19-tests/}{σταθερός στο
  5\% έως 25\%} ή αυξήθηκε μόνο ελαφρώς. Σε πολλές χώρες, η εξάπλωση
  είχε ήδη κορυφωθεί πολύ
  \href{https://www.youtube.com/watch?v=lGC5sGdz4kg}{πριν από το
  κλείσιμο}.
\item
  Χώρες χωρίς απαγόρευση της κυκλοφορίας και απαγόρευση επαφών, όπως η
  \href{https://www.japantimes.co.jp/news/2020/03/20/national/coronavirus-explosion-expected-japan/}{Ιαπωνία},
  \href{https://www.businessinsider.com/south-korea-coronavirus-testing-death-rate-2020-3?op=1}{η
  Νότια Κορέα} ή \href{https://www.youtube.com/watch?v=bfN2JWifLCY}{η
  Σουηδία},
  \href{https://www.washingtontimes.com/news/2020/apr/15/sweden-coronavirus-rates-easing-despite-loose-rule/}{δεν
  έχουν βιώσει} μια πιο αρνητική πορεία των γεγονότων από άλλες χώρες. Η
  Σουηδία
  \href{https://nypost.com/2020/04/29/who-lauds-sweden-as-model-for-resisting-coronavirus-lockdown/}{επαινέθηκε}
  ακόμη και από τον ΠΟΥ και τώρα επωφελείται από υψηλότερη ανοσία σε
  σύγκριση με τις χώρες που εφάρμοσαν το κλειδώμα.
\item
  Ο φόβος για έλλειψη αναπνευστήρων ήταν
  \href{https://apnews.com/8ccd325c2be9bf454c2128dcb7bd616d}{αδικαιολόγητος}.
  Σύμφωνα με τους πνευμονολόγους, η διασωλήνωση ασθενών με Covid19, ο
  οποίος γίνεται εν μέρει
  \href{https://www.dailymail.co.uk/news/article-8262351/Nurse-New-York-claims-city-killing-COVID-19-patients-putting-ventilators.html}{λόγω
  του φόβου} της εξάπλωσης του ιού, είναι στην πραγματικότητα συχνά
  \href{https://www.medscape.com/viewarticle/928156}{αντιπαραγωγικός}
  και βλαβερός για τους πνεύμονες.
\item
  Σε αντίθεση με τις αρχικές υποθέσεις, διάφορες μελέτες έχουν δείξει
  ότι
  \href{https://www.who.int/news-room/commentaries/detail/modes-of-transmission-of-virus-causing-covid-19-implications-for-ipc-precaution-recommendations}{δεν
  υπάρχουν ενδείξεις} για την εξάπλωση του ιού μέσω αερολυμάτων (δηλαδή
  σωματιδίων που επιπλέουν στον αέρα) ή μέσω
  \href{https://www.telegraph.co.uk/news/2020/04/02/no-proof-coronavirus-can-spread-shopping-says-leading-german/}{μολύνσεων
  από επιχρίσματα} (π.χ. σε λαβές πορτών, smartphone ή στο κομμωτήριο).
\item
  \href{https://www.researchgate.net/publication/340570735_Masks_Don't_Work_A_review_of_science_relevant_to_COVID-19_social_policy}{Δεν
  υπάρχει επίσης επιστημονική ένδειξη} για την αποτελεσματικότητα των
  μασκών προσώπου σε υγιή ή ασυμπτωματικά άτομα. Αντίθετα, οι ειδικοί
  προειδοποιούν ότι τέτοιες μάσκες παρεμποδίζουν την κανονική αναπνοή
  και μπορεί να γίνουν
  \href{https://de.sputniknews.com/interviews/20200425326953541-corona-gefahr-virologe/}{«φορείς
  μικροβίων»}. Κορυφαίοι γιατροί αναφέρθηκαν σε αυτό ως «διαφημιστική
  εκστρατεία» και
  «\href{https://www.aerztezeitung.de/Politik/Montgomery-haelt-Maskenpflicht-fuer-falsch-408844.html}{γελοιότητα}».
\item
  Πολλές κλινικές στην Ευρώπη και τις ΗΠΑ παρέμειναν σε
  \href{https://www.hsj.co.uk/acute-care/nhs-hospitals-have-four-times-more-empty-beds-than-normal/7027392.article}{υπολειτουργεία}
  ή σχεδόν άδειες κατά τη διάρκεια της κορυφωσης του Covid19 και σε
  ορισμένες περιπτώσεις έπρεπε
  \href{https://www.usatoday.com/story/news/health/2020/04/02/coronavirus-pandemic-jobs-us-health-care-workers-furloughed-laid-off/5102320002/}{να
  στείλουν το προσωπικό τους στο σπίτι}. Πολυάριθμες επεμβάσεις και
  θεραπείες
  \href{https://www.sfchronicle.com/bayarea/article/Stanford-hospital-system-to-cut-pay-20-furlough-15227591.php}{ακυρώθηκαν},
  συμπεριλαμβανομένων μερικών μεταμοσχεύσεων οργάνων και
  προσυμπτωματικού ελέγχου καρκίνου.
\item
  Αρκετά μέσα ενημέρωσης πιάστηκαν
  \href{https://nypost.com/2020/04/01/cbs-admits-to-using-footage-from-italy-in-report-about-nyc/}{προσπαθώντας
  να δραματοποιήσουν} την κατάσταση στα νοσοκομεία, μερικές φορές ακόμη
  και με παραποιητικές εικόνες και βίντεο. Σε γενικές γραμμές, οι
  \href{https://onlinelibrary.wiley.com/doi/full/10.1111/eci.13222}{μη
  επαγγελματικές αναφορές} πολλών μέσων μεγιστοποίησαν τον φόβο και τον
  πανικό στον πληθυσμό.
\item
  Τα~ τεστ που χρησιμοποιούνται διεθνώς είναι
  \href{https://www.ncbi.nlm.nih.gov/pubmed/32219885}{επιρρεπή σε
  σφάλματα} και μπορούν να παράγουν ψευδώς θετικά και ψευδώς αρνητικά
  αποτελέσματα. Επιπλέον, τα επίσημα τεστ ιών
  \href{https://www.youtube.com/watch?v=p_AyuhbnPOI}{δεν έχουν
  πιστοποιηθεί κλινικά} λόγω της πίεσης του χρόνου και μερικές φορές
  μπορεί να αντιδρούν σε άλλους κορωναϊούς.
\item
  Πολλοί
  \href{https://off-guardian.org/2020/03/24/12-experts-questioning-the-coronavirus-panic/}{διεθνώς
  γνωστοί εμπειρογνώμονες} στους τομείς της ιολογίας, της ανοσολογίας
  και της επιδημιολογίας θεωρούν ότι τα μέτρα που λαμβάνονται είναι
  \href{https://off-guardian.org/2020/03/28/10-more-experts-criticising-the-coronavirus-panic/}{αντιπαραγωγικά}
  και συνιστούν ταχεία
  \href{https://off-guardian.org/2020/04/17/8-more-experts-questioning-the-coronavirus-panic/}{φυσική
  ανοσοποίηση} του γενικού πληθυσμού και προστασία των ομάδων κινδύνου.
  Οι κίνδυνοι για τα παιδιά είναι
  \href{https://www.thelancet.com/journals/lanchi/article/PIIS2352-4642(20)30095-X/fulltext}{σχεδόν
  μηδενικοί} και το κλείσιμο σχολείων δεν δικαιολογείται ιατρικά.
\item
  Αρκετοί ιατροί εμπειρογνώμονες χαρακτήρισαν τα εμβόλια κατά των
  κορωναϊών \href{https://www.youtube.com/watch?v=vrL9QKGQrWk}{περιττά}
  ή ακόμη και
  \href{https://www.nature.com/articles/d41586-020-00751-9}{επικίνδυνα}.
  Πράγματι, το εμβόλιο κατά της λεγόμενης
  \href{https://www.forbes.com/2010/02/05/world-health-organization-swine-flu-pandemic-opinions-contributors-michael-fumento.html\#658c006c48e8}{γρίπης
  των χοίρων} του 2009, για παράδειγμα, οδήγησε μερικές φορές σε
  \href{https://www.ibtimes.co.uk/brain-damaged-uk-victims-swine-flu-vaccine-get-60-million-compensation-1438572}{σοβαρές
  νευρολογικές βλάβες} και αγωγές εκατομμύριων.
\item
  Ο αριθμός των ατόμων που πάσχουν από ανεργία,
  \href{https://www.indystar.com/story/news/health/2020/04/03/coronavirus-indiana-how-get-help-mental-health-addiction/5104357002/}{ψυχολογικά
  προβλήματα} και ενδοοικογενειακή βία ως αποτέλεσμα των μέτρων έχει
  \href{https://www.reuters.com/article/us-health-coronavirus-usa-layoffs/us-weekly-jobless-claims-seen-at-record-high-again-idUSKBN21K0FX}{φτάσει
  στα ύψη} παγκοσμίως. Αρκετοί ειδικοί πιστεύουν ότι τα μέτρα ενδέχεται
  να
  ~\href{https://www.nytimes.com/2020/03/20/opinion/coronavirus-pandemic-social-distancing.html}{κοστίσουν
  περισσότερες ζωές} από τον ίδιο τον ιό. Σύμφωνα με τα Ηνωμένα Έθνη,
  \href{https://www.theguardian.com/global-development/2020/apr/21/coronavirus-pandemic-will-cause-famine-of-biblical-proportions}{εκατομμύρια
  άνθρωποι} σε όλο τον κόσμο ενδέχεται να πέσουν σε απόλυτη φτώχεια και
  πείνα.
\item
  Ο πληροφοριοδότης της NSA Edward Snowden προειδοποίησε ότι η «κρίση
  του κορώνα» θα χρησιμοποιηθεί για τη
  \href{https://www.youtube.com/watch?v=-pcQFTzck_c}{μαζική και μόνιμη
  επέκταση} της παγκόσμιας επιτήρησης. Ο διάσημος ιολόγος Pablo
  Goldschmidt
  \href{https://www.rubikon.news/artikel/der-corona-totalitarismus}{μίλησε
  για} «παγκόσμια τρομοκρατία των μέσων ενημέρωσης» και «ολοκληρωτικά
  μέτρα». Ο κορυφαίος Βρετανός καθηγητής ιολογίας John Oxford
  \href{https://novuscomms.com/2020/03/31/a-view-from-the-hvivo-open-orphan-orph-laboratory-professor-john-oxford/}{μίλησε
  για} «επιδημία των μέσων ενημέρωσης».
\item
  Περισσότεροι από 500 επιστήμονες
  \href{https://www.esat.kuleuven.be/cosic/sites/contact-tracing-joint-statement/}{έχουν
  προειδοποιήσει για} μια «άνευ προηγουμένου επιτήρηση της κοινωνίας»
  μέσω προβληματικών εφαρμογών για «ανίχνευση επαφών». Σε ορισμένες
  χώρες, τέτοια «ανίχνευση επαφών» πραγματοποιείται ήδη απευθείας
  \href{https://www.jewishpress.com/news/the-courts/state-to-high-court-even-more-shin-bet-involvement-in-fighting-the-coronavirus/2020/04/14/}{από
  τη μυστική υπηρεσία}. Σε πολλά μέρη του κόσμου, ο πληθυσμός
  \href{https://off-guardian.org/2020/04/25/50-headlines-darker-more-of-the-new-normal/}{παρακολουθείται
  ήδη από drones} ή έρχεται αντιμέτωπος με αστυνομική αυθαιρεσία.
\end{enumerate}

\hypertarget{ux3b4ux3b5ux3afux3c4ux3b5-ux3b5ux3c0ux3afux3c3ux3b7ux3c2}{%
\subparagraph{\texorpdfstring{\textbf{Δείτε
επίσης:}}{Δείτε επίσης:}}\label{ux3b4ux3b5ux3afux3c4ux3b5-ux3b5ux3c0ux3afux3c3ux3b7ux3c2}}

\begin{itemize}
\tightlist
\item
  \href{https://swprs.org/professor-bhakdi-open-letter-greek/}{Ανοιχτή
  επιστολή από τον Δρ. Sucharit Bhakdi}
\item
  \href{https://www.euromomo.eu/}{European Mortality Monitoring
  (EuroMomo)}
\item
  \href{https://swprs.org/corona-media-propaganda-greek/}{Περί
  κορωναϊού, ΜΜΕ και προπαγάνδας}
\end{itemize}

\hypertarget{ux3c0ux3b5ux3c1ux3b9ux3c3ux3c3ux3ccux3c4ux3b5ux3c1ux3b5ux3c2-ux3b5ux3bdux3b7ux3bcux3b5ux3c1ux3ceux3c3ux3b5ux3b9ux3c2-ux3c3ux3c4ux3b1-ux3b1ux3b3ux3b3ux3bbux3b9ux3baux3ac}{%
\paragraph{\texorpdfstring{\href{https://swprs.org/a-swiss-doctor-on-covid-19/}{Περισσότερες
ενημερώσεις στα
Αγγλικά.}}{Περισσότερες ενημερώσεις στα Αγγλικά.}}\label{ux3c0ux3b5ux3c1ux3b9ux3c3ux3c3ux3ccux3c4ux3b5ux3c1ux3b5ux3c2-ux3b5ux3bdux3b7ux3bcux3b5ux3c1ux3ceux3c3ux3b5ux3b9ux3c2-ux3c3ux3c4ux3b1-ux3b1ux3b3ux3b3ux3bbux3b9ux3baux3ac}}

\begin{center}\rule{0.5\linewidth}{\linethickness}\end{center}

Μοιραστείτε το:
\href{https://twitter.com/intent/tweet?url=https://swprs.org/facts-about-covid19-greek/}{Twitter}
/
\href{https://www.facebook.com/share.php?u=https://swprs.org/facts-about-covid19-greek/}{Facebook}

\hypertarget{swiss-policy-research}{%
\subsubsection{Swiss Policy Research}\label{swiss-policy-research}}

\begin{itemize}
\tightlist
\item
  \href{https://swprs.org/kontakt/}{Kontakt}
\item
  \href{https://swprs.org/uebersicht/}{Übersicht}
\item
  \href{https://swprs.org/donationen/}{Donationen}
\item
  \href{https://swprs.org/disclaimer/}{Disclaimer}
\end{itemize}

\hypertarget{english}{%
\subsubsection{English}\label{english}}

\begin{itemize}
\tightlist
\item
  \href{https://swprs.org/contact/}{About Us / Contact}
\item
  \href{https://swprs.org/media-navigator/}{The Media Navigator}
\item
  \href{https://swprs.org/the-american-empire-and-its-media/}{The CFR
  and the Media}
\item
  \href{https://swprs.org/donations/}{Donations}
\end{itemize}

\hypertarget{follow-by-email}{%
\subsubsection{Follow by email}\label{follow-by-email}}

Follow

\href{https://wordpress.com/?ref=footer_custom_com}{WordPress.com}.

\protect\hyperlink{}{Up ↑}

\includegraphics{https://pixel.wp.com/b.gif?v=noscript}
