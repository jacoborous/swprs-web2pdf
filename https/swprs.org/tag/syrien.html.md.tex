\protect\hyperlink{content}{Skip to content}

\href{https://swprs.org/}{}

\protect\hyperlink{search-container}{Search}

Search for:

\href{https://swprs.org/}{\includegraphics{https://swprs.files.wordpress.com/2020/05/swiss-policy-research-logo-300.png}}

\href{https://swprs.org/}{Swiss Policy Research}

Geopolitics and Media

Menu

\begin{itemize}
\tightlist
\item
  \href{https://swprs.org}{Start}
\item
  \href{https://swprs.org/srf-propaganda-analyse/}{Studien}

  \begin{itemize}
  \tightlist
  \item
    \href{https://swprs.org/srf-propaganda-analyse/}{SRF / ZDF}
  \item
    \href{https://swprs.org/die-nzz-studie/}{NZZ-Studie}
  \item
    \href{https://swprs.org/der-propaganda-multiplikator/}{Agenturen}
  \item
    \href{https://swprs.org/die-propaganda-matrix/}{Medienmatrix}
  \end{itemize}
\item
  \href{https://swprs.org/medien-navigator/}{Analysen}

  \begin{itemize}
  \tightlist
  \item
    \href{https://swprs.org/medien-navigator/}{Navigator}
  \item
    \href{https://swprs.org/der-propaganda-schluessel/}{Techniken}
  \item
    \href{https://swprs.org/propaganda-in-der-wikipedia/}{Wikipedia}
  \item
    \href{https://swprs.org/logik-imperialer-kriege/}{Kriege}
  \end{itemize}
\item
  \href{https://swprs.org/netzwerk-medien-schweiz/}{Netzwerke}

  \begin{itemize}
  \tightlist
  \item
    \href{https://swprs.org/netzwerk-medien-schweiz/}{Schweiz}
  \item
    \href{https://swprs.org/netzwerk-medien-deutschland/}{Deutschland}
  \item
    \href{https://swprs.org/medien-in-oesterreich/}{Österreich}
  \item
    \href{https://swprs.org/das-american-empire-und-seine-medien/}{USA}
  \end{itemize}
\item
  \href{https://swprs.org/bericht-eines-journalisten/}{Fokus I}

  \begin{itemize}
  \tightlist
  \item
    \href{https://swprs.org/bericht-eines-journalisten/}{Journalistenbericht}
  \item
    \href{https://swprs.org/russische-propaganda/}{Russische Propaganda}
  \item
    \href{https://swprs.org/die-israel-lobby-fakten-und-mythen/}{Die
    »Israel-Lobby«}
  \item
    \href{https://swprs.org/geopolitik-und-paedokriminalitaet/}{Pädokriminalität}
  \end{itemize}
\item
  \href{https://swprs.org/migration-und-medien/}{Fokus II}

  \begin{itemize}
  \tightlist
  \item
    \href{https://swprs.org/covid-19-hinweis-ii/}{Coronavirus}
  \item
    \href{https://swprs.org/die-integrity-initiative/}{Integrity
    Initiative}
  \item
    \href{https://swprs.org/migration-und-medien/}{Migration \& Medien}
  \item
    \href{https://swprs.org/der-fall-magnitsky/}{Magnitsky Act}
  \end{itemize}
\item
  \href{https://swprs.org/kontakt/}{Projekt}

  \begin{itemize}
  \tightlist
  \item
    \href{https://swprs.org/kontakt/}{Kontakt}
  \item
    \href{https://swprs.org/uebersicht/}{Seitenübersicht}
  \item
    \href{https://swprs.org/medienspiegel/}{Medienspiegel}
  \item
    \href{https://swprs.org/donationen/}{Donationen}
  \end{itemize}
\item
  \href{https://swprs.org/contact/}{English}
\end{itemize}

\protect\hyperlink{}{Open Search}

\hypertarget{tag-syrien}{%
\section{Tag: Syrien}\label{tag-syrien}}

\hypertarget{das-gewuxfcnschte-narrativ-ii}{%
\section{\texorpdfstring{\href{https://swprs.org/2017/03/01/das-gewuenschte-narrativ-ii/}{Das
gewünschte
Narrativ~II}}{Das gewünschte Narrativ~II}}\label{das-gewuxfcnschte-narrativ-ii}}

\href{https://swprs.org/2017/03/01/das-gewuenschte-narrativ-ii/}{\includegraphics{https://swprs.files.wordpress.com/2017/03/zeitungen-schweiz.png?w=450}}

Im Dezember 2015 publi­zierte das News­portal \emph{Watson} (AZ Medien)
einen \href{https://www.watson.ch/!148360008}{Artikel} des lang­jährigen
Tages­schau-Kor­res­pon­denten Hel­mut Sche­­ben zum Syrien­krieg.
Scheben stellte den Krieg in einen geo­po­li­tischen Kontext und
kri­ti­sierte die westliche Be­richt­er­stattung als einseitig und
ma­ni­pu­la­tiv.

Der Artikel un­ter­schied sich deutlich von anderen Aus­lands­bei­trägen
auf \emph{Watson}, die meist vom deutsch-transatlantischen
\href{https://www.watson.ch/Corporate/articles/502582965-Spiegel-Online-und-watson-machen-gemeinsame-Sache}{Content
Partner} \emph{Spiegel Online} geliefert werden.

Keine zwei Tage später veröffentlichte \emph{Watson} jedoch einen
aufgebrachten \href{https://www.watson.ch/!491379853}{Rückruf}, in dem
sich das Portal vom Artikel distanzierte und Helmut Scheben wüst
beschimpfte: Man sei auf einen ``Putin-Troll'' herein­ge­fallen, der
wo­möglich in der ``russischen Propaganda-Maschinerie'' mit­wirke. Auch
Leser, die sich positiv zum ur­sprüng­lichen Artikel geäußert hatten,
wurden als »Trolle« verun­glimpft.

Wer oder was hat wohl hinter den Kulissen zu dieser selt­samen Reak­tion
geführt? Jeden­falls wurde den hiesigen Journa­listen damit einmal mehr
in Er­in­nerung gerufen: Wer sich in der Schweiz nicht an das
ge­wünschte Nar­ra­tiv hält, ris­kiert Ruf und Karriere.

\begin{center}\rule{0.5\linewidth}{\linethickness}\end{center}

\href{https://swprs.org/2017/03/01/das-gewuenschte-narrativ-ii/}{**1.
March 2017}

\hypertarget{der-kriegsreporter}{%
\section{\texorpdfstring{\href{https://swprs.org/2017/03/01/der-kriegsreporter/}{Der
Kriegsreporter}}{Der Kriegsreporter}}\label{der-kriegsreporter}}

\href{https://swprs.org/2017/03/01/der-kriegsreporter/}{\includegraphics{https://swprs.files.wordpress.com/2016/11/pelda-syrien.jpg?w=600}}

Wie wird man in den Schweizer Medien zum »Nahost-Experten«? Kurt Pelda
muss es wissen: Von der \emph{Welt­woche} bis zum \emph{Schwei­zer
Fern­se­hen} ist er der Mann, der die Ereig­nisse in Sy­ri­en und Irak
für das Publi­kum
\href{http://www.srf.ch/news/international/assad-ist-nur-noch-an-der-macht-weil-er-so-brutal-ist}{»ein­ord­nen«}
darf.

Pelda
\href{https://www.youtube.com/watch?v=dtV25eIECKY}{be­glei­tete}schon in
den 80er Jahren als junger Journa­list die Mudschahedin im von den USA
\href{https://www.voltairenet.org/article165889.html}{lancier­ten} Krieg
gegen die afgha­nische Regie­rung, die mit Moskau verbün­det war. Nach
Sta­tionen bei der \emph{Financial Times} und der \emph{NZZ} bereist er
heute als freier Journa­list erneut Kriegs­ge­biete -- wie damals meist
nur
\href{https://tageswoche.ch/politik/ein-basler-im-syrischen-kampfgebiet/}{auf
Seiten} der US-unter­stützten Milizen.

Ist diese Ein­seitig­keit ein Pro­blem? Nicht für Pelda, denn er sei
schließ­lich -- so erklärte er in einem
\href{https://www.tageswoche.ch/de/2014_36/international/667493/}{Interview}
-- ein »Mei­nungs­jour­na­list« und »kein objek­ti­ver Be­obach­ter«,
wes­wegen Neutra­li­tät für ihn »keine Option« ist; viel­mehr gehe es
ihm um »gute Ge­schich­ten«, für die die Medien zu zahlen be­reit sind.
Wer in diesen Ge­schich­ten die Guten sind -- und wer
\href{http://www.srf.ch/news/international/assad-ist-nur-noch-an-der-macht-weil-er-so-brutal-ist}{die
Bösen} -- dürf‌te dabei niemanden über­raschen.

Mit diesem Ansatz wurde Pelda 2014 zum
\href{http://www.srf.ch/news/panorama/kurt-pelda-ist-journalist-des-jahres}{»Jour­na­list
des Jahres«} gekürt. Andere Nahost-Ken­ner, denen Objek­ti­vi­tät und
Neutra­lität wich­ti­ger sind als eine »gute Ge­schichte«, kommen in
Schwei­zer Medien indes
\href{https://swprs.org/das-gewuenschte-narrativ-ii/}{kaum noch} zu
Wort. Statt »ein­ge­ordnet« wurde hier -- aus­sor­tiert.

\emph{Foto oben:} Pelda in Syrien.
\emph{(\href{https://tageswoche.ch/politik/ein-basler-im-syrischen-kampfgebiet/}{TW})}

\begin{center}\rule{0.5\linewidth}{\linethickness}\end{center}

\href{https://swprs.org/2017/03/01/der-kriegsreporter/}{**1. March 2017}

\hypertarget{swiss-policy-research}{%
\subsubsection{Swiss Policy Research}\label{swiss-policy-research}}

\begin{itemize}
\tightlist
\item
  \href{https://swprs.org/kontakt/}{Kontakt}
\item
  \href{https://swprs.org/uebersicht/}{Übersicht}
\item
  \href{https://swprs.org/donationen/}{Donationen}
\item
  \href{https://swprs.org/disclaimer/}{Disclaimer}
\end{itemize}

\hypertarget{english}{%
\subsubsection{English}\label{english}}

\begin{itemize}
\tightlist
\item
  \href{https://swprs.org/contact/}{About Us / Contact}
\item
  \href{https://swprs.org/media-navigator/}{The Media Navigator}
\item
  \href{https://swprs.org/the-american-empire-and-its-media/}{The CFR
  and the Media}
\item
  \href{https://swprs.org/donations/}{Donations}
\end{itemize}

\hypertarget{follow-by-email}{%
\subsubsection{Follow by email}\label{follow-by-email}}

Follow

\href{https://wordpress.com/?ref=footer_custom_com}{WordPress.com}.

\protect\hyperlink{}{Up ↑}

\includegraphics{https://pixel.wp.com/b.gif?v=noscript}
