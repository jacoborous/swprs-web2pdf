\protect\hyperlink{content}{Skip to content}

\href{https://swprs.org/}{}

\protect\hyperlink{search-container}{Search}

Search for:

\href{https://swprs.org/}{\includegraphics{https://swprs.files.wordpress.com/2020/05/swiss-policy-research-logo-300.png}}

\href{https://swprs.org/}{Swiss Policy Research}

Geopolitics and Media

Menu

\begin{itemize}
\tightlist
\item
  \href{https://swprs.org}{Start}
\item
  \href{https://swprs.org/srf-propaganda-analyse/}{Studien}

  \begin{itemize}
  \tightlist
  \item
    \href{https://swprs.org/srf-propaganda-analyse/}{SRF / ZDF}
  \item
    \href{https://swprs.org/die-nzz-studie/}{NZZ-Studie}
  \item
    \href{https://swprs.org/der-propaganda-multiplikator/}{Agenturen}
  \item
    \href{https://swprs.org/die-propaganda-matrix/}{Medienmatrix}
  \end{itemize}
\item
  \href{https://swprs.org/medien-navigator/}{Analysen}

  \begin{itemize}
  \tightlist
  \item
    \href{https://swprs.org/medien-navigator/}{Navigator}
  \item
    \href{https://swprs.org/der-propaganda-schluessel/}{Techniken}
  \item
    \href{https://swprs.org/propaganda-in-der-wikipedia/}{Wikipedia}
  \item
    \href{https://swprs.org/logik-imperialer-kriege/}{Kriege}
  \end{itemize}
\item
  \href{https://swprs.org/netzwerk-medien-schweiz/}{Netzwerke}

  \begin{itemize}
  \tightlist
  \item
    \href{https://swprs.org/netzwerk-medien-schweiz/}{Schweiz}
  \item
    \href{https://swprs.org/netzwerk-medien-deutschland/}{Deutschland}
  \item
    \href{https://swprs.org/medien-in-oesterreich/}{Österreich}
  \item
    \href{https://swprs.org/das-american-empire-und-seine-medien/}{USA}
  \end{itemize}
\item
  \href{https://swprs.org/bericht-eines-journalisten/}{Fokus I}

  \begin{itemize}
  \tightlist
  \item
    \href{https://swprs.org/bericht-eines-journalisten/}{Journalistenbericht}
  \item
    \href{https://swprs.org/russische-propaganda/}{Russische Propaganda}
  \item
    \href{https://swprs.org/die-israel-lobby-fakten-und-mythen/}{Die
    »Israel-Lobby«}
  \item
    \href{https://swprs.org/geopolitik-und-paedokriminalitaet/}{Pädokriminalität}
  \end{itemize}
\item
  \href{https://swprs.org/migration-und-medien/}{Fokus II}

  \begin{itemize}
  \tightlist
  \item
    \href{https://swprs.org/covid-19-hinweis-ii/}{Coronavirus}
  \item
    \href{https://swprs.org/die-integrity-initiative/}{Integrity
    Initiative}
  \item
    \href{https://swprs.org/migration-und-medien/}{Migration \& Medien}
  \item
    \href{https://swprs.org/der-fall-magnitsky/}{Magnitsky Act}
  \end{itemize}
\item
  \href{https://swprs.org/kontakt/}{Projekt}

  \begin{itemize}
  \tightlist
  \item
    \href{https://swprs.org/kontakt/}{Kontakt}
  \item
    \href{https://swprs.org/uebersicht/}{Seitenübersicht}
  \item
    \href{https://swprs.org/medienspiegel/}{Medienspiegel}
  \item
    \href{https://swprs.org/donationen/}{Donationen}
  \end{itemize}
\item
  \href{https://swprs.org/contact/}{English}
\end{itemize}

\protect\hyperlink{}{Open Search}

\hypertarget{tag-wikipedia}{%
\section{Tag: Wikipedia}\label{tag-wikipedia}}

\hypertarget{die-angst-vor-den-lesern}{%
\section{\texorpdfstring{\href{https://swprs.org/2017/03/01/leserkommentare/}{Die
Angst vor
den~Lesern}}{Die Angst vor den~Lesern}}\label{die-angst-vor-den-lesern}}

\href{https://swprs.org/2017/03/01/leserkommentare/}{\includegraphics{https://swprs.files.wordpress.com/2016/07/leserkommentare.png?w=600}}

Weil Propaganda von kritischen Lesern immer öfter und schneller entlarvt
wird, sind viele Medien dazu über­ge­gangen, die Kommentar­funktion auf
ihren Inter­net­­seiten stark zu zensieren oder ganz zu
\href{https://www.heise.de/tp/features/Konzentriertes-Gejammer-NZZ-schliesst-Kommentarspalte-3618957.html}{deaktivieren}.
Zuletzt griff selbst die vermeintlich liberale \emph{NZZ} zu dieser
\href{https://www.heise.de/tp/features/Konzentriertes-Gejammer-NZZ-schliesst-Kommentarspalte-3618957.html}{Maßnahme}.

Schließlich versuchten die ertappten Medien, die kri­ti­schen Leser als
Trolle
\href{https://www.nzz.ch/international/putins-internetpiraten-1.18324628}{dar­zu­stellen},
die womöglich von aus­län­dischen Re­gie­rungen fürs Kom­men­tieren
bezahlt würden. Be­lege da­für blie­ben aus, und inhaltlich wurde auf
die Leser­kritik ohnehin nicht ein­ge­gangen.

Doch nicht nur von den Medien, auch im Online-Lexikon \emph{Wikipedia}
werden die Leser an der freien Meinungs­bil­dung
\href{https://swprs.org/propaganda-in-der-wikipedia/}{gehindert}: Hier
sorgt eine kleine Gruppe anonymer »Adminis­tra­toren« dafür, dass bei
geo­po­li­tisch brisanten Themen ab­wei­chende Positionen gelöscht,
Autoren gesperrt und kritische Forscher diffamiert werden (siehe
\href{https://swprs.org/propaganda-in-der-wikipedia/}{Vertiefungsstudie}).

\begin{center}\rule{0.5\linewidth}{\linethickness}\end{center}

\href{https://swprs.org/2017/03/01/leserkommentare/}{**1. March 2017}

\hypertarget{propaganda-in-der-wikipedia}{%
\section{\texorpdfstring{\href{https://swprs.org/2017/03/01/propaganda-in-der-wikipedia/}{Propaganda
in
der~Wikipedia}}{Propaganda in der~Wikipedia}}\label{propaganda-in-der-wikipedia}}

Die Online-Enzyklopädie Wikipedia ist ein integraler Bestandteil des
transatlantischen Medien- und Informationssystems. In der folgenden
Analyse werden zentrale Aspekte ihrer Organisationsstruktur,
Funktionsweise und Manipulation dargestellt.

\href{https://swprs.org/propaganda-in-der-wikipedia/}{\includegraphics{https://swprs.files.wordpress.com/2019/03/wikipedia-2019-s.png?w=736}}

\href{https://swprs.org/propaganda-in-der-wikipedia/}{Zur Analyse →}

\begin{center}\rule{0.5\linewidth}{\linethickness}\end{center}

\href{https://swprs.org/2017/03/01/propaganda-in-der-wikipedia/}{**1.
March 2017}

\hypertarget{leserbriefe}{%
\section{\texorpdfstring{\href{https://swprs.org/2017/03/01/leserbriefe/}{Leserbriefe}}{Leserbriefe}}\label{leserbriefe}}

\href{https://swprs.org/2017/03/01/leserbriefe/}{\includegraphics{https://swprs.files.wordpress.com/2016/10/comments3.png?w=250}}

\emph{»Ich habe 5 Jahre auf einer renom­mier­ten Schweizer
Nachrichten­redaktion ge­ar­bei­tet. Es ist das erste Mal, dass ich eine
so umfassende Arbeit zum Thema sehe. Konnte noch nicht alles lesen, aber
das, was ich gelesen habe, deckt sich mit meiner Erfahrung und
Wahr­nehmung.«}

\emph{»Ich hatte selbst in den 80er Jahren bei der NZZ gearbeitet.
Damals eine Auszeichnung. Heute leider so wie von Ihnen beschrieben.«}

\emph{»Ganz herzliche Gratulation zu eurer sehr infor­ma­tiven Seite. So
etwas hat für die Schweiz noch gefehlt. Freue mich schon auf weitere
Beiträge!«}

\href{https://swprs.org/leserbriefe/}{Zu den Leserbriefen →}

\begin{center}\rule{0.5\linewidth}{\linethickness}\end{center}

\href{https://swprs.org/2017/03/01/leserbriefe/}{**1. March 2017}

\hypertarget{medienspiegel}{%
\section{\texorpdfstring{\href{https://swprs.org/2017/03/01/medienspiegel/}{Medienspiegel}}{Medienspiegel}}\label{medienspiegel}}

\href{https://swprs.org/2017/03/01/medienspiegel/}{\includegraphics{https://swprs.files.wordpress.com/2017/03/newspaper2.png?w=500}}

Als eines der bekanntesten medien­kritischen Forschungs­projekte werden
unsere Arbeiten zunehmend auch von tradi­tio­nellen und neuen Medien
rezipiert.

Im Folgenden findet sich ein fort­laufend aktuali­sierter Medien­spiegel
mit den wich­tig­sten Beiträgen und Übersetzungen.

\href{https://swprs.org/medienspiegel/}{Zum Medienspiegel →}

\begin{center}\rule{0.5\linewidth}{\linethickness}\end{center}

\href{https://swprs.org/2017/03/01/medienspiegel/}{**1. March 2017}

\hypertarget{das-forschungsprojekt}{%
\section{\texorpdfstring{\href{https://swprs.org/2017/03/01/das-forschungsprojekt/}{Das
Forschungsprojekt}}{Das Forschungsprojekt}}\label{das-forschungsprojekt}}

\href{https://swprs.org/2017/03/01/das-forschungsprojekt/}{\includegraphics{https://swprs.files.wordpress.com/2017/03/spr_logo_post.png?w=364}}

Swiss Policy Research (SPR) ist ein Forschungs- und
Infor­ma­tions­projekt zu geo­po­li­tischer Pro­pa­ganda in Schweizer
Medien.

Sämtliche Studien und Bei­träge wurden von einer po­li­tisch und
pu­bli­zis­tisch un­ab­hän­gigen For­schungs­gruppe ohne Beauf­tra­gung
oder Fremd­finan­zierung er­­stellt.

Das Forschungsprojekt wurde 2016 lanciert und zählt inzwischen zu den
bekanntesten Publi­ka­tionen auf diesem Gebiet.

Hier können Sie uns \href{https://swprs.org/kontakt/}{kon­­tak­tieren}.

\begin{center}\rule{0.5\linewidth}{\linethickness}\end{center}

\href{https://swprs.org/2017/03/01/das-forschungsprojekt/}{**1. March
2017}

\hypertarget{swiss-policy-research}{%
\subsubsection{Swiss Policy Research}\label{swiss-policy-research}}

\begin{itemize}
\tightlist
\item
  \href{https://swprs.org/kontakt/}{Kontakt}
\item
  \href{https://swprs.org/uebersicht/}{Übersicht}
\item
  \href{https://swprs.org/donationen/}{Donationen}
\item
  \href{https://swprs.org/disclaimer/}{Disclaimer}
\end{itemize}

\hypertarget{english}{%
\subsubsection{English}\label{english}}

\begin{itemize}
\tightlist
\item
  \href{https://swprs.org/contact/}{About Us / Contact}
\item
  \href{https://swprs.org/media-navigator/}{The Media Navigator}
\item
  \href{https://swprs.org/the-american-empire-and-its-media/}{The CFR
  and the Media}
\item
  \href{https://swprs.org/donations/}{Donations}
\end{itemize}

\hypertarget{follow-by-email}{%
\subsubsection{Follow by email}\label{follow-by-email}}

Follow

\href{https://wordpress.com/?ref=footer_custom_com}{WordPress.com}.

\protect\hyperlink{}{Up ↑}

\includegraphics{https://pixel.wp.com/b.gif?v=noscript}
