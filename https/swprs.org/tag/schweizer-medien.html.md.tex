\protect\hyperlink{content}{Skip to content}

\href{https://swprs.org/}{}

\protect\hyperlink{search-container}{Search}

Search for:

\href{https://swprs.org/}{\includegraphics{https://swprs.files.wordpress.com/2020/05/swiss-policy-research-logo-300.png}}

\href{https://swprs.org/}{Swiss Policy Research}

Geopolitics and Media

Menu

\begin{itemize}
\tightlist
\item
  \href{https://swprs.org}{Start}
\item
  \href{https://swprs.org/srf-propaganda-analyse/}{Studien}

  \begin{itemize}
  \tightlist
  \item
    \href{https://swprs.org/srf-propaganda-analyse/}{SRF / ZDF}
  \item
    \href{https://swprs.org/die-nzz-studie/}{NZZ-Studie}
  \item
    \href{https://swprs.org/der-propaganda-multiplikator/}{Agenturen}
  \item
    \href{https://swprs.org/die-propaganda-matrix/}{Medienmatrix}
  \end{itemize}
\item
  \href{https://swprs.org/medien-navigator/}{Analysen}

  \begin{itemize}
  \tightlist
  \item
    \href{https://swprs.org/medien-navigator/}{Navigator}
  \item
    \href{https://swprs.org/der-propaganda-schluessel/}{Techniken}
  \item
    \href{https://swprs.org/propaganda-in-der-wikipedia/}{Wikipedia}
  \item
    \href{https://swprs.org/logik-imperialer-kriege/}{Kriege}
  \end{itemize}
\item
  \href{https://swprs.org/netzwerk-medien-schweiz/}{Netzwerke}

  \begin{itemize}
  \tightlist
  \item
    \href{https://swprs.org/netzwerk-medien-schweiz/}{Schweiz}
  \item
    \href{https://swprs.org/netzwerk-medien-deutschland/}{Deutschland}
  \item
    \href{https://swprs.org/medien-in-oesterreich/}{Österreich}
  \item
    \href{https://swprs.org/das-american-empire-und-seine-medien/}{USA}
  \end{itemize}
\item
  \href{https://swprs.org/bericht-eines-journalisten/}{Fokus I}

  \begin{itemize}
  \tightlist
  \item
    \href{https://swprs.org/bericht-eines-journalisten/}{Journalistenbericht}
  \item
    \href{https://swprs.org/russische-propaganda/}{Russische Propaganda}
  \item
    \href{https://swprs.org/die-israel-lobby-fakten-und-mythen/}{Die
    »Israel-Lobby«}
  \item
    \href{https://swprs.org/geopolitik-und-paedokriminalitaet/}{Pädokriminalität}
  \end{itemize}
\item
  \href{https://swprs.org/migration-und-medien/}{Fokus II}

  \begin{itemize}
  \tightlist
  \item
    \href{https://swprs.org/covid-19-hinweis-ii/}{Coronavirus}
  \item
    \href{https://swprs.org/die-integrity-initiative/}{Integrity
    Initiative}
  \item
    \href{https://swprs.org/migration-und-medien/}{Migration \& Medien}
  \item
    \href{https://swprs.org/der-fall-magnitsky/}{Magnitsky Act}
  \end{itemize}
\item
  \href{https://swprs.org/kontakt/}{Projekt}

  \begin{itemize}
  \tightlist
  \item
    \href{https://swprs.org/kontakt/}{Kontakt}
  \item
    \href{https://swprs.org/uebersicht/}{Seitenübersicht}
  \item
    \href{https://swprs.org/medienspiegel/}{Medienspiegel}
  \item
    \href{https://swprs.org/donationen/}{Donationen}
  \end{itemize}
\item
  \href{https://swprs.org/contact/}{English}
\end{itemize}

\protect\hyperlink{}{Open Search}

\hypertarget{tag-schweizer-medien}{%
\section{Tag: Schweizer Medien}\label{tag-schweizer-medien}}

\hypertarget{zensur-und-selbstzensur}{%
\section{\texorpdfstring{\href{https://swprs.org/2017/03/01/zensur-in-schweizer-medien/}{Zensur
und
Selbstzensur}}{Zensur und Selbstzensur}}\label{zensur-und-selbstzensur}}

\href{https://swprs.org/2017/03/01/zensur-in-schweizer-medien/}{\includegraphics{https://swprs.files.wordpress.com/2016/07/zensur-selbstzensur-c.png?w=306}}

Zensur und Selbst­zensur bei geo­po­li­tischen Kon­f‌lik­ten sind in der
Schweiz keines­wegs un­be­kannt, wie ein Blick in die Ge­schichte zeigt.

Um das Land keinen un­nöti­gen Ri­si­ken aus­zu­setzen,
\href{http://www.amazon.de/Selbstzensur-schweizerische-Pressepolitik-Zweiten-Weltkrieg/dp/3719304566}{muss­ten}
sich Medien und
\href{https://www.chronos-verlag.ch/node/20528}{Buch­ver­lage} wäh­rend
des

\begin{enumerate}
\def\labelenumi{\arabic{enumi}.}
\tightlist
\item
  und 2. Welt­kriegs und während des
  \href{http://www.swissinfo.ch/ger/das-ende-eines-nationalen-maenner-netzwerks/4205194}{Kal­ten
  Kriegs} an einen po­li­tisch definierten Mei­nungs­korri­dor halten,
  der sich an den welt­wei­ten Kräfte­ver­hält­nissen orientierte.
\end{enumerate}

Durch die Ereignisse von 1990 und 2001 nahm der
\href{https://www.youtube.com/watch?v=a4eGtXFDFJA}{Druck} auf
Drittstaaten und ihre Medien wei­ter zu: »Entweder mit uns, oder gegen
uns.«

Aufgrund der Medien­kon­zen­tration werden in­zwi­schen zudem
\href{https://swprs.files.wordpress.com/2018/03/broschur_jahrbuch_foeg_deutsch_2015.pdf\#page=13}{über
90\%} des Schwei­zer Mark­tes von nur noch fünf Medien­häusern bedient:
Tamedia, Ringier, NZZ Medien und AZ Medien, sowie der SRG (siehe
\href{https://swprs.org/netzwerk-medien-schweiz/}{Info­grafik}).

Eine echte \href{https://swprs.org/medien-navigator/}{Medienvielfalt}
entstand mithin erst durch das Internet -- obschon auch hier bereits
diverse
\href{https://www.heise.de/tp/features/Facebook-Fake-News-und-die-Privatisierung-der-Zensur-3599878.html}{Zensurversuche}
zu beobachten sind.

\begin{center}\rule{0.5\linewidth}{\linethickness}\end{center}

\href{https://swprs.org/2017/03/01/zensur-in-schweizer-medien/}{**1.
March 2017}

\hypertarget{die-partnerschaft-mit-der-nato}{%
\section{\texorpdfstring{\href{https://swprs.org/2017/03/01/schweizer-medien-nato/}{Die
Partnerschaft mit
der~NATO}}{Die Partnerschaft mit der~NATO}}\label{die-partnerschaft-mit-der-nato}}

\href{https://swprs.org/2017/03/01/schweizer-medien-nato/}{\includegraphics{https://swprs.files.wordpress.com/2016/07/nato-logo-3s.png?w=305}}

Die Schweiz ist nicht Mit­glied in der NATO, trat jedoch 1996 der
\emph{\href{http://www.pfp.admin.ch/}{»NATO Partner­ship for Peace«}}
und 1997 dem
\emph{\href{http://www.nato.int/docu/review/2007/issue2/german/art5.html}{Euro-Atlan­tischen
Par­tner­schafts­rat}} bei -- je­weils ohne Volks­ab­stimmung.

Seit­dem kommt das Schweizer Militär im Zuge von NATO-Inter­­ven­­tionen
zum \href{https://www.peace-support.ch/de/}{Einsatz}, so im Kosovo, in
Bosnien und in Afgha­ni­stan (ISAF). Auch der Schweizer
Nach­richten­dienst (NDB) wird inzwischen von einem
\href{https://www.admin.ch/gov/de/start/dokumentation/medienmitteilungen.msg-id-70400.html}{General}
geführt, der durch die NATO ausgebildet wurde.

Würden Schweizer Medien trotz NATO-Part­ner­schaft allzu kritisch über
Interventionen der US-Allianz berichten, so könnte dies als
\href{https://swprs.org/russische-propaganda/}{»feind­li­che
Pro­pa­gan­da«} ge­wer­tet werden -- was po­li­tisch und ökonomisch
wenig opportun wäre.

Auf diese Weise ergibt sich eine weitgehend
\href{https://swprs.org/medien-navigator/}{NATO-kon­forme} Darstellung
von geopolitischen Kon­flik­ten, so in Jugoslawien, Afgha­ni­stan, Irak,
Li­by­en, Syrien, Jemen oder der Ukraine.

An wirtschaftlichen Sanktionen muss sich die Schweiz auf Wunsch der USA
schon seit 1951
\href{https://de.wikipedia.org/wiki/Hotz-Linder-Agreement}{be­tei­li­gen}.
Jour­na­listen, die diese Ver­letzung der Neu­tra­lität damals
kri­ti­sierten,
\href{https://web.archive.org/web/20141206061445/http://buchundnetz.com/online-buch/schnueffelstaat-schweiz-ob/iii-modernisieren-oder-abschaffen/staatsschutz-je-nach-wetterlage/}{er­hielten}
15 Mo­nate Gefäng­nis wegen Landes­verrats.

\begin{center}\rule{0.5\linewidth}{\linethickness}\end{center}

\href{https://swprs.org/2017/03/01/schweizer-medien-nato/}{**1. March
2017}

\hypertarget{das-gewuxfcnschte-narrativ}{%
\section{\texorpdfstring{\href{https://swprs.org/2017/03/01/das-gewuenschte-narrativ/}{Das
gewünschte
Narrativ}}{Das gewünschte Narrativ}}\label{das-gewuxfcnschte-narrativ}}

\href{https://swprs.org/2017/03/01/das-gewuenschte-narrativ/}{\includegraphics{https://swprs.files.wordpress.com/2016/02/medien-narrativ1.png?w=400}}

Bei geopolitischen Konflikten bestehen oftmals vordefinierte mediale
Narrative. Was geschieht, wenn ein Schweizer Jour­na­list davon abweicht
und über die »falschen« Themen be­richtet?

Heute kaum noch vorstellbar, doch mitten im
\href{https://de.wikipedia.org/wiki/Bosnienkrieg}{Bosnien­krieg}
(1992-95) veröffentlichte der damalige Aus­lands­chef der
\emph{Welt­woche} einen Artikel zu Kriegs­lügen in west­lichen Medien.

Daraufhin geschah Folgendes:

\href{https://swprs.org/das-gewuenschte-narrativ\#weiterlesen}{Weiterlesen
→}

\begin{center}\rule{0.5\linewidth}{\linethickness}\end{center}

\href{https://swprs.org/2017/03/01/das-gewuenschte-narrativ/}{**1. March
2017}

\hypertarget{die-konferenz}{%
\section{\texorpdfstring{\href{https://swprs.org/2017/03/01/schweizer-medien-bilderberg-konferenz/}{Die
Konferenz}}{Die Konferenz}}\label{die-konferenz}}

\href{https://swprs.org/2017/03/01/schweizer-medien-bilderberg-konferenz/}{\includegraphics{https://swprs.files.wordpress.com/2016/02/bilderberg_2011.png?w=440}}

Die großen Schweizer Medien­­häuser sind in geo­poli­tische
\href{https://swprs.org/netzwerk-medien-schweiz/}{Netz­werke}
ein­ge­bun­den: So nehmen die wichtigsten Schweizer Verleger und
Chef­redakteure im Turnus an der jähr­lichen
\href{http://www.bilderbergmeetings.org/}{Bilderberg-Konferenz} teil, wo
sie im privaten Rahmen auf die trans­atlan­tische Elite aus
Wirt­schaf‌t, Politik und Militär treffen.

Teilnehmer seit 1991 (siehe
\href{https://swprs.org/netzwerk-medien-schweiz/}{Info­grafik}):

\includegraphics{https://swprs.files.wordpress.com/2017/03/teilnehmer-bilderberg-ch-1.png?w=736}

Auch der journa­lis­tische Nach­wuchs wird ge­för­dert: Sowohl der
\href{http://www.americanswiss.org/news/arthur-honegger-spotlight/}{*10vor10-*​Mode­ra­tor}
des SRF wie auch der
\href{http://www.americanswiss.org/news/niklaus-nuspliger-spotlight/}{*NZZ-*Korres­pon­dent}
für die EU \& NATO wurden von der
\href{http://www.americanswiss.org/}{\emph{Ameri­can Swiss
Foun­da­tion}} zu »Young Leaders« ernannt -- und neh­men in dieser Rolle
an
\href{http://www.americanswiss.org/ambassador-barras-hosts-dinner-for-young-leaders-1/}{exklu­siven
Dinners} mit hoch­rang­igen US-Ver­tre­tern teil.

\emph{Foto:} Bilder­berg-Meeting
\href{https://www.theguardian.com/world/gallery/2011/jun/15/bilderberg-in-pictures}{2011}
in St. Moritz.

\begin{center}\rule{0.5\linewidth}{\linethickness}\end{center}

\href{https://swprs.org/2017/03/01/schweizer-medien-bilderberg-konferenz/}{**1.
March 2017}

\hypertarget{das-transatlantik-netzwerk}{%
\section{\texorpdfstring{\href{https://swprs.org/2017/03/01/das-netzwerk/}{Das
Transatlantik-Netzwerk}}{Das Transatlantik-Netzwerk}}\label{das-transatlantik-netzwerk}}

Wie sind Schweizer Medien in trans­at­lantische Netz­werke
ein­ge­bunden? Welche Personen, Organi­sa­tionen und Kon­fe­ren­zen sind
von Bedeutung? Unsere Info­grafik gibt Auskunft.

\href{https://swprs.org/netzwerk-medien-schweiz}{\includegraphics{https://swprs.files.wordpress.com/2019/10/medien-netzwerk-schweiz-hdz-s.png?w=736}}

\href{https://swprs.org/netzwerk-medien-schweiz}{Zur Infografik →}

\begin{center}\rule{0.5\linewidth}{\linethickness}\end{center}

\href{https://swprs.org/2017/03/01/das-netzwerk/}{**1. March 2017}

\hypertarget{der-kriegsreporter}{%
\section{\texorpdfstring{\href{https://swprs.org/2017/03/01/der-kriegsreporter/}{Der
Kriegsreporter}}{Der Kriegsreporter}}\label{der-kriegsreporter}}

\href{https://swprs.org/2017/03/01/der-kriegsreporter/}{\includegraphics{https://swprs.files.wordpress.com/2016/11/pelda-syrien.jpg?w=600}}

Wie wird man in den Schweizer Medien zum »Nahost-Experten«? Kurt Pelda
muss es wissen: Von der \emph{Welt­woche} bis zum \emph{Schwei­zer
Fern­se­hen} ist er der Mann, der die Ereig­nisse in Sy­ri­en und Irak
für das Publi­kum
\href{http://www.srf.ch/news/international/assad-ist-nur-noch-an-der-macht-weil-er-so-brutal-ist}{»ein­ord­nen«}
darf.

Pelda
\href{https://www.youtube.com/watch?v=dtV25eIECKY}{be­glei­tete}schon in
den 80er Jahren als junger Journa­list die Mudschahedin im von den USA
\href{https://www.voltairenet.org/article165889.html}{lancier­ten} Krieg
gegen die afgha­nische Regie­rung, die mit Moskau verbün­det war. Nach
Sta­tionen bei der \emph{Financial Times} und der \emph{NZZ} bereist er
heute als freier Journa­list erneut Kriegs­ge­biete -- wie damals meist
nur
\href{https://tageswoche.ch/politik/ein-basler-im-syrischen-kampfgebiet/}{auf
Seiten} der US-unter­stützten Milizen.

Ist diese Ein­seitig­keit ein Pro­blem? Nicht für Pelda, denn er sei
schließ­lich -- so erklärte er in einem
\href{https://www.tageswoche.ch/de/2014_36/international/667493/}{Interview}
-- ein »Mei­nungs­jour­na­list« und »kein objek­ti­ver Be­obach­ter«,
wes­wegen Neutra­li­tät für ihn »keine Option« ist; viel­mehr gehe es
ihm um »gute Ge­schich­ten«, für die die Medien zu zahlen be­reit sind.
Wer in diesen Ge­schich­ten die Guten sind -- und wer
\href{http://www.srf.ch/news/international/assad-ist-nur-noch-an-der-macht-weil-er-so-brutal-ist}{die
Bösen} -- dürf‌te dabei niemanden über­raschen.

Mit diesem Ansatz wurde Pelda 2014 zum
\href{http://www.srf.ch/news/panorama/kurt-pelda-ist-journalist-des-jahres}{»Jour­na­list
des Jahres«} gekürt. Andere Nahost-Ken­ner, denen Objek­ti­vi­tät und
Neutra­lität wich­ti­ger sind als eine »gute Ge­schichte«, kommen in
Schwei­zer Medien indes
\href{https://swprs.org/das-gewuenschte-narrativ-ii/}{kaum noch} zu
Wort. Statt »ein­ge­ordnet« wurde hier -- aus­sor­tiert.

\emph{Foto oben:} Pelda in Syrien.
\emph{(\href{https://tageswoche.ch/politik/ein-basler-im-syrischen-kampfgebiet/}{TW})}

\begin{center}\rule{0.5\linewidth}{\linethickness}\end{center}

\href{https://swprs.org/2017/03/01/der-kriegsreporter/}{**1. March 2017}

\hypertarget{die-vertrauensfrage}{%
\section{\texorpdfstring{\href{https://swprs.org/2017/03/01/schweizer-medien-vertrauen/}{Die
Vertrauensfrage}}{Die Vertrauensfrage}}\label{die-vertrauensfrage}}

\href{https://swprs.org/2017/03/01/schweizer-medien-vertrauen/}{\includegraphics{https://swprs.files.wordpress.com/2017/03/foeg-jahrbuch_logo.png?w=500}}

Das Forschungs­institut für Öf­fent­lich­­keit und Gesell­schaft der
Uni­ver­sität Zürich publi­ziert all­jähr­lich das »Jahr­buch Qualität
der Medien«. 2016 ver­mel­dete das In­sti­tut, das Ver­trau­en in die
Schwei­zer Me­dien sei
\href{http://www.foeg.uzh.ch/dam/jcr:7234c6d3-1f09-4d36-b6ab-f14e659d046e/Medienmitteilung_JB_2016_dt.pdf}{»weiter­hin
hoch«} -- so das Er­geb­nis eines Länder­ver­gleichs in
Zu­sam­men­ar­beit mit dem \emph{\emph{Reu­ters Insti­tute.}}

Doch wie hoch ist das Vertrauen in die Schweizer Medien nun wirklich?
Dazu findet man in der Mit­tei­lung des Instituts keine An­ga­ben. Und
auch die
\href{http://www.tagesanzeiger.ch/schweiz/standard/Diese-Menschen-sind-anfaellig-fuer-Populisten/story/23804017}{Zei­tungs­be­richte}
zur Studie er­wäh­nen diese wich­tige Kenn­zahl
\href{http://www.nzz.ch/schweiz/analyse-zum-medienvertrauen-oeffentliche-medien-staerken-auch-die-privaten-ld.128965}{nicht}.
Aus gutem Grund -- denn die Resultate sind er­schüt­ternd.

Demnach
\href{http://media.digitalnewsreport.org/wp-content/uploads/2018/11/Digital-News-Report-2016.pdf\#page=60}{halten}
nur noch 50\% der Schwei­zer Be­völ­ke­rung die Nach­rich­ten für
glaub­würdig. Das Ver­trauen in die Medien­unter­nehmen und in die
Jour­na­listen liegt mit 39\% bzw. 35\% sogar noch tiefer. Mit anderen
Worten: Rund zwei Drittel der Schweizer Be­völ­ke­rung ver­traut den
ei­ge­nen Jour­na­listen nicht mehr*.*

Dennoch glaubt das For­schungs­in­sti­tut -- das u.a. vom Bundes­amt für
Kom­mu­ni­ka­tion finanziert wird -- die Nutzung tra­di­tio­neller und
ins­b. öffent­licher Medien würde das Ver­trauen ins Medien­system
\href{http://www.foeg.uzh.ch/dam/jcr:7234c6d3-1f09-4d36-b6ab-f14e659d046e/Medienmitteilung_JB_2016_dt.pdf}{»för­dern«}.
Die Da­ten zei­gen je­doch nur, dass regel­mäßige Kon­su­menten die­ser
Me­dien we­ni­ger kri­tisch sind -- und ihre An­zahl immer ge­ringer
wird.

\emph{Update:} 2017
\href{http://www.digitalnewsreport.org/survey/2017/switzerland-2017/}{sank}
das Medienvertrauen auf 46\%. Die Werte bzgl. Journalisten und
Unter­neh­men wurden nicht mehr erhoben. Gemäß FÖG war das Vertrauen
\href{http://www.foeg.uzh.ch/dam/jcr:0d0e5a10-27be-4e97-b264-b2cf7de96bbd/Broschur_Jahrbuch_foeg_deutsch_2017_ohne_Sperrvermerk.pdf}{»weiterhin
hoch«}.

\begin{center}\rule{0.5\linewidth}{\linethickness}\end{center}

\href{https://swprs.org/2017/03/01/schweizer-medien-vertrauen/}{**1.
March 2017}

\hypertarget{die-angst-vor-den-lesern}{%
\section{\texorpdfstring{\href{https://swprs.org/2017/03/01/leserkommentare/}{Die
Angst vor
den~Lesern}}{Die Angst vor den~Lesern}}\label{die-angst-vor-den-lesern}}

\href{https://swprs.org/2017/03/01/leserkommentare/}{\includegraphics{https://swprs.files.wordpress.com/2016/07/leserkommentare.png?w=600}}

Weil Propaganda von kritischen Lesern immer öfter und schneller entlarvt
wird, sind viele Medien dazu über­ge­gangen, die Kommentar­funktion auf
ihren Inter­net­­seiten stark zu zensieren oder ganz zu
\href{https://www.heise.de/tp/features/Konzentriertes-Gejammer-NZZ-schliesst-Kommentarspalte-3618957.html}{deaktivieren}.
Zuletzt griff selbst die vermeintlich liberale \emph{NZZ} zu dieser
\href{https://www.heise.de/tp/features/Konzentriertes-Gejammer-NZZ-schliesst-Kommentarspalte-3618957.html}{Maßnahme}.

Schließlich versuchten die ertappten Medien, die kri­ti­schen Leser als
Trolle
\href{https://www.nzz.ch/international/putins-internetpiraten-1.18324628}{dar­zu­stellen},
die womöglich von aus­län­dischen Re­gie­rungen fürs Kom­men­tieren
bezahlt würden. Be­lege da­für blie­ben aus, und inhaltlich wurde auf
die Leser­kritik ohnehin nicht ein­ge­gangen.

Doch nicht nur von den Medien, auch im Online-Lexikon \emph{Wikipedia}
werden die Leser an der freien Meinungs­bil­dung
\href{https://swprs.org/propaganda-in-der-wikipedia/}{gehindert}: Hier
sorgt eine kleine Gruppe anonymer »Adminis­tra­toren« dafür, dass bei
geo­po­li­tisch brisanten Themen ab­wei­chende Positionen gelöscht,
Autoren gesperrt und kritische Forscher diffamiert werden (siehe
\href{https://swprs.org/propaganda-in-der-wikipedia/}{Vertiefungsstudie}).

\begin{center}\rule{0.5\linewidth}{\linethickness}\end{center}

\href{https://swprs.org/2017/03/01/leserkommentare/}{**1. March 2017}

\hypertarget{der-chefredakteur-und-die-cia}{%
\section{\texorpdfstring{\href{https://swprs.org/2017/03/01/chefredakteur-cia/}{Der
Chefredakteur und
die~CIA}}{Der Chefredakteur und die~CIA}}\label{der-chefredakteur-und-die-cia}}

\href{https://swprs.org/2017/03/01/chefredakteur-cia/}{\includegraphics{https://swprs.files.wordpress.com/2016/05/cia-media.png?w=450}}

Die klandestine Zu­sam­men­arbeit west­licher Geheim­dienste mit Medien,
Think Tanks und NGOs ist seit langem
\href{http://carlbernstein.com/magazine_cia_and_media.php}{bekannt} und
vielfach
\href{http://www.amazon.de/Geheimdienst-Politik-Medien-Meinungsmache-Zeitgeschichte/dp/3897068796}{doku­men­tiert}.

Wie eng und um­fas­send bisweilen selbst füh­ren­de deutsch­spra­chige
Jour­na­listen mit den Ge­heim­diens­ten kooperieren, dies zeigt
bei­spiel­haft der Fall von Otto Schul­meister.

Schul­meister war lang­jäh­riger Chef­re­dak­teur der
\href{https://de.wikipedia.org/wiki/Die_Presse}{\emph{Presse}}, einer
der größ­ten und tra­di­tions­reich­sten Tages­­zeitungen Öster­reichs.
2009 wurde sein ehemaliges CIA-Dossier publik -- mit bemerkenswerten
Einzel­heiten zur ver­deckten Kol­la­bo­ration:

\href{https://swprs.org/der-chefredakteur-und-die-cia\#weiterlesen}{Weiterlesen
→}

\begin{center}\rule{0.5\linewidth}{\linethickness}\end{center}

\href{https://swprs.org/2017/03/01/chefredakteur-cia/}{**1. March 2017}

\hypertarget{die-grenzen-der-pressefreiheit}{%
\section{\texorpdfstring{\href{https://swprs.org/2017/03/01/die-grenzen-der-pressefreiheit/}{Die
Grenzen
der~Pressefreiheit}}{Die Grenzen der~Pressefreiheit}}\label{die-grenzen-der-pressefreiheit}}

\href{https://swprs.org/2017/03/01/die-grenzen-der-pressefreiheit/}{\includegraphics{https://swprs.files.wordpress.com/2017/12/reporter_ohne_grenzen_logo_s.png?w=530}}

Der \href{http://pressclub.ch/?lang=en}{Schweizer Presseclub} in Genf
genießt einen ausgezeichneten Ruf: Seit seiner Gründung hat er über
zweitausend Anlässe mit illustren Rednern von Fidel Castro bis Henry
Kissinger und von Jean Ziegler bis Klaus Schwab organisiert.

Doch für Ende November 2017 war ein
\href{http://pressclub.ch/they-dont-care-about-us-white-helmets-true-agenda/?lang=en}{Vortrag}
angekündigt, der sich kritisch mit den in west­li­chen Medien populären
\href{https://www.hintergrund.de/globales/kriege/weisse-helme-ohne-weisse-westen/}{Syrischen
Weiß­helmen} befassen wollte. Daraufhin geschah Folgendes:

\href{https://swprs.org/die-grenzen-der-pressefreiheit/}{Weiterlesen →}

\begin{center}\rule{0.5\linewidth}{\linethickness}\end{center}

\href{https://swprs.org/2017/03/01/die-grenzen-der-pressefreiheit/}{**1.
March 2017}

\hypertarget{anschlag-auf-die-forschungsfreiheit}{%
\section{\texorpdfstring{\href{https://swprs.org/2017/03/01/anschlag-auf-die-forschungsfreiheit/}{Anschlag
auf die
Forschungsfreiheit}}{Anschlag auf die Forschungsfreiheit}}\label{anschlag-auf-die-forschungsfreiheit}}

\href{https://swprs.org/2017/03/01/anschlag-auf-die-forschungsfreiheit/}{\includegraphics{https://swprs.files.wordpress.com/2018/11/ganser.png?w=500}}

So ergeht es US-kritischen Forschern in der Schweiz: Der Historiker Dr.
Daniele Ganser geriet 2006 nach einer öffentlichen Inter­vention der
amerika­nischen Bot­schaf­terin unter Druck und musste seine Forschung
an der ETH Zürich schließlich aufgeben.

Ganser forschte zu ver­deckter Kriegs­führung und
\href{http://ofv.ch/sachbuch/detail/natogeheimarmeen-in-europa/3193/}{ins­ze­nier­tem
Terror} durch die NATO im Kalten Krieg sowie zu den An­schlägen vom 11.
September 2001 (s.
\href{http://archiv.ethlife.ethz.ch/articles/9.11.html}{Artikel im
ETH-Magazin}).

\href{https://swprs.org/anschlag-auf-die-forschungsfreiheit\#weiterlesen}{Weiterlesen
→}

\begin{center}\rule{0.5\linewidth}{\linethickness}\end{center}

\href{https://swprs.org/2017/03/01/anschlag-auf-die-forschungsfreiheit/}{**1.
March 2017}

\hypertarget{medienaufsicht-im-faktencheck}{%
\section{\texorpdfstring{\href{https://swprs.org/2017/03/01/medienaufsicht-faktencheck/}{Medienaufsicht
im
Faktencheck}}{Medienaufsicht im Faktencheck}}\label{medienaufsicht-im-faktencheck}}

\href{https://swprs.org/2017/03/01/medienaufsicht-faktencheck/}{\includegraphics{https://swprs.files.wordpress.com/2017/03/srf-ombudsstelle.png?w=600}}

Die Ombudsstelle des \emph{SRF} ist die erste Anlaufstelle für
Programm­be­schwerden des Publi­kums. Doch wie un­vor­ein­ge­nommen und
objektiv behandelt sie Beschwerden zu geo­po­li­tischen Themen?

Um dies zu über­prüfen, wurden während eines halben Jahres alle
Schluss­be­richte zum Syrien­kon­flikt einem Fakten­check unter­zogen.
Die Resul­tate sind bedenk­lich.

\href{https://swprs.org/srf-ombudsstelle-im-faktencheck/}{Zum
Faktencheck~→}

\begin{center}\rule{0.5\linewidth}{\linethickness}\end{center}

\href{https://swprs.org/2017/03/01/medienaufsicht-faktencheck/}{**1.
March 2017}

\hypertarget{swiss-policy-research}{%
\subsubsection{Swiss Policy Research}\label{swiss-policy-research}}

\begin{itemize}
\tightlist
\item
  \href{https://swprs.org/kontakt/}{Kontakt}
\item
  \href{https://swprs.org/uebersicht/}{Übersicht}
\item
  \href{https://swprs.org/donationen/}{Donationen}
\item
  \href{https://swprs.org/disclaimer/}{Disclaimer}
\end{itemize}

\hypertarget{english}{%
\subsubsection{English}\label{english}}

\begin{itemize}
\tightlist
\item
  \href{https://swprs.org/contact/}{About Us / Contact}
\item
  \href{https://swprs.org/media-navigator/}{The Media Navigator}
\item
  \href{https://swprs.org/the-american-empire-and-its-media/}{The CFR
  and the Media}
\item
  \href{https://swprs.org/donations/}{Donations}
\end{itemize}

\hypertarget{follow-by-email}{%
\subsubsection{Follow by email}\label{follow-by-email}}

Follow

\href{https://wordpress.com/?ref=footer_custom_com}{WordPress.com}.

\protect\hyperlink{}{Up ↑}

Post to

\protect\hyperlink{}{Cancel}

\includegraphics{https://pixel.wp.com/b.gif?v=noscript}
