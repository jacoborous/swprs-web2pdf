\protect\hyperlink{content}{Skip to content}

\href{https://swprs.org/}{}

\protect\hyperlink{search-container}{Search}

Search for:

\href{https://swprs.org/}{\includegraphics{https://swprs.files.wordpress.com/2020/05/swiss-policy-research-logo-300.png}}

\href{https://swprs.org/}{Swiss Policy Research}

Geopolitics and Media

Menu

\begin{itemize}
\tightlist
\item
  \href{https://swprs.org}{Start}
\item
  \href{https://swprs.org/srf-propaganda-analyse/}{Studien}

  \begin{itemize}
  \tightlist
  \item
    \href{https://swprs.org/srf-propaganda-analyse/}{SRF / ZDF}
  \item
    \href{https://swprs.org/die-nzz-studie/}{NZZ-Studie}
  \item
    \href{https://swprs.org/der-propaganda-multiplikator/}{Agenturen}
  \item
    \href{https://swprs.org/die-propaganda-matrix/}{Medienmatrix}
  \end{itemize}
\item
  \href{https://swprs.org/medien-navigator/}{Analysen}

  \begin{itemize}
  \tightlist
  \item
    \href{https://swprs.org/medien-navigator/}{Navigator}
  \item
    \href{https://swprs.org/der-propaganda-schluessel/}{Techniken}
  \item
    \href{https://swprs.org/propaganda-in-der-wikipedia/}{Wikipedia}
  \item
    \href{https://swprs.org/logik-imperialer-kriege/}{Kriege}
  \end{itemize}
\item
  \href{https://swprs.org/netzwerk-medien-schweiz/}{Netzwerke}

  \begin{itemize}
  \tightlist
  \item
    \href{https://swprs.org/netzwerk-medien-schweiz/}{Schweiz}
  \item
    \href{https://swprs.org/netzwerk-medien-deutschland/}{Deutschland}
  \item
    \href{https://swprs.org/medien-in-oesterreich/}{Österreich}
  \item
    \href{https://swprs.org/das-american-empire-und-seine-medien/}{USA}
  \end{itemize}
\item
  \href{https://swprs.org/bericht-eines-journalisten/}{Fokus I}

  \begin{itemize}
  \tightlist
  \item
    \href{https://swprs.org/bericht-eines-journalisten/}{Journalistenbericht}
  \item
    \href{https://swprs.org/russische-propaganda/}{Russische Propaganda}
  \item
    \href{https://swprs.org/die-israel-lobby-fakten-und-mythen/}{Die
    »Israel-Lobby«}
  \item
    \href{https://swprs.org/geopolitik-und-paedokriminalitaet/}{Pädokriminalität}
  \end{itemize}
\item
  \href{https://swprs.org/migration-und-medien/}{Fokus II}

  \begin{itemize}
  \tightlist
  \item
    \href{https://swprs.org/covid-19-hinweis-ii/}{Coronavirus}
  \item
    \href{https://swprs.org/die-integrity-initiative/}{Integrity
    Initiative}
  \item
    \href{https://swprs.org/migration-und-medien/}{Migration \& Medien}
  \item
    \href{https://swprs.org/der-fall-magnitsky/}{Magnitsky Act}
  \end{itemize}
\item
  \href{https://swprs.org/kontakt/}{Projekt}

  \begin{itemize}
  \tightlist
  \item
    \href{https://swprs.org/kontakt/}{Kontakt}
  \item
    \href{https://swprs.org/uebersicht/}{Seitenübersicht}
  \item
    \href{https://swprs.org/medienspiegel/}{Medienspiegel}
  \item
    \href{https://swprs.org/donationen/}{Donationen}
  \end{itemize}
\item
  \href{https://swprs.org/contact/}{English}
\end{itemize}

\protect\hyperlink{}{Open Search}

\hypertarget{tag-markus-spillmann}{%
\section{Tag: Markus Spillmann}\label{tag-markus-spillmann}}

\hypertarget{die-konferenz}{%
\section{\texorpdfstring{\href{https://swprs.org/2017/03/01/schweizer-medien-bilderberg-konferenz/}{Die
Konferenz}}{Die Konferenz}}\label{die-konferenz}}

\href{https://swprs.org/2017/03/01/schweizer-medien-bilderberg-konferenz/}{\includegraphics{https://swprs.files.wordpress.com/2016/02/bilderberg_2011.png?w=440}}

Die großen Schweizer Medien­­häuser sind in geo­poli­tische
\href{https://swprs.org/netzwerk-medien-schweiz/}{Netz­werke}
ein­ge­bun­den: So nehmen die wichtigsten Schweizer Verleger und
Chef­redakteure im Turnus an der jähr­lichen
\href{http://www.bilderbergmeetings.org/}{Bilderberg-Konferenz} teil, wo
sie im privaten Rahmen auf die trans­atlan­tische Elite aus
Wirt­schaf‌t, Politik und Militär treffen.

Teilnehmer seit 1991 (siehe
\href{https://swprs.org/netzwerk-medien-schweiz/}{Info­grafik}):

\includegraphics{https://swprs.files.wordpress.com/2017/03/teilnehmer-bilderberg-ch-1.png?w=736}

Auch der journa­lis­tische Nach­wuchs wird ge­för­dert: Sowohl der
\href{http://www.americanswiss.org/news/arthur-honegger-spotlight/}{*10vor10-*​Mode­ra­tor}
des SRF wie auch der
\href{http://www.americanswiss.org/news/niklaus-nuspliger-spotlight/}{*NZZ-*Korres­pon­dent}
für die EU \& NATO wurden von der
\href{http://www.americanswiss.org/}{\emph{Ameri­can Swiss
Foun­da­tion}} zu »Young Leaders« ernannt -- und neh­men in dieser Rolle
an
\href{http://www.americanswiss.org/ambassador-barras-hosts-dinner-for-young-leaders-1/}{exklu­siven
Dinners} mit hoch­rang­igen US-Ver­tre­tern teil.

\emph{Foto:} Bilder­berg-Meeting
\href{https://www.theguardian.com/world/gallery/2011/jun/15/bilderberg-in-pictures}{2011}
in St. Moritz.

\begin{center}\rule{0.5\linewidth}{\linethickness}\end{center}

\href{https://swprs.org/2017/03/01/schweizer-medien-bilderberg-konferenz/}{**1.
March 2017}

\hypertarget{die-nzz-studie}{%
\section{\texorpdfstring{\href{https://swprs.org/2017/03/01/die-nzz-studie/}{Die
NZZ-Studie}}{Die NZZ-Studie}}\label{die-nzz-studie}}

\href{https://swprs.org/2017/03/01/die-nzz-studie/}{\includegraphics{https://swprs.files.wordpress.com/2017/03/nzz-propaganda-gesamt-small.png?w=500}}

Die \emph{Neue Zürcher Zeitung} ist das Flagg­schiff unter den Schweizer
Tages­zei­tungen. Doch wie objektiv und kritisch berichtet die
\emph{NZZ} über geo­politische Konf‌likte?

Um dies zu be­ant­worten, wurde die Bericht­erstattung der \emph{NZZ}
zur Ukraine-Krise und zum Syrien­krieg während je eines Monats
unter­sucht. Die Ergebnisse sind eindeutig.

\href{https://swprs.org/die-nzz-studie/}{Zur NZZ-Studie →}

\begin{center}\rule{0.5\linewidth}{\linethickness}\end{center}

\href{https://swprs.org/2017/03/01/die-nzz-studie/}{**1. March 2017}

\hypertarget{der-schweizer-presserat}{%
\section{\texorpdfstring{\href{https://swprs.org/2017/03/01/der-schweizer-presserat/}{Der
Schweizer
Presserat}}{Der Schweizer Presserat}}\label{der-schweizer-presserat}}

\href{https://swprs.org/2017/03/01/der-schweizer-presserat/}{\includegraphics{https://swprs.files.wordpress.com/2016/07/presserat-logo.png?w=200}}

Der \href{https://presserat.ch/}{Schweizer Presse­rat} nimmt
Be­schwer­den zu Me­dien­be­rich­ten ent­ge­gen und prüft, ob die
Beiträge seinen
\href{https://presserat.ch/journalistenkodex/richtlinien/}{Richt­linien}
ent­spre­chen.

Aller­dings
\href{https://presserat.ch/der-presserat/presseratsmitglieder/}{besteht}
das Gre­mium selbst aus 15 Jour­na­listen und nur sechs
\emph{Pub­li­kums­ver­tre­tern} -- und auch diese werden von einem
\href{https://presserat.ch/der-presserat/stiftungsratsmitglieder/}{Stif‌­tungs­rat}
er­nannt, der gänz­lich von Medien­orga­ni­sa­tionen
\href{https://presserat.ch/der-presserat/geschaeftsreglement/}{kon­trol­liert}
wird.

Das Resultat ist naheliegend. Im Som­mer 2014 wurde etwa eine
Be­schwerde gegen die
\href{https://swprs.org/die-nzz-studie/}{no­to­risch ein­sei­tige}
Ukraine-Bericht­er­stattung der \emph{NZZ} ein­ge­legt. Ganze zwei Jahre
später kam der Presse­rat zu seinem
\href{https://presserat.ch/complaints/wahrheitspflicht-kommentarfreiheit-unterschlagen-wichtiger-informationen-entstellen-von-tatsachen/}{Verdikt}:
Die Rich­tig­keit der *NZZ- *Dar­stel­lung stehe \emph{»außer Frage«},
denn auf \emph{»amt­liche Ver­laut­ba­rungen und Agen­tur­mel­dungen«}
sei \emph{»Verlass«,} während russische Quel­len weder glaub­haf‌t noch
erforderlich wären; Kom­men­tare müss­ten nicht auf Fak­ten ba­sie­ren,
Ge­gen­mei­nungen ein­zu­holen sei \emph{»un­üb­lich«,} und an den
Aus­füh­rungen der \emph{NZZ} zu \emph{»Kreml- Trollen«} sei
\emph{»nicht zu zwei­feln«}. Be­schwerde ab­ge­lehnt.

Pikant: Einige der be­ur­teil­ten Ar­tikel stam­mten von einem
\href{http://www.nzz.ch/international/europa/beschwerde-beim-presserat-kritik-an-nzz-abgewiesen-ld.104814}{*NZZ-*Redak­teur},
der selbst im Stif‌­tungs­rat des Gremiums sitzt -- und inzwischen wurde
der damalige *NZZ-*Chef gar zu dessen
\href{http://www.nzz.ch/schweiz/medien-selbstregulierung-markus-spillmann-wird-praesident-des-presserats-ld.135619}{Prä­si­denten}
ernannt. Beim Presse­rat nennt man dies
\href{https://de.wikipedia.org/wiki/Schweizer_Presserat}{»Selbst­re­gu­lierung«\ldots{}}

\begin{center}\rule{0.5\linewidth}{\linethickness}\end{center}

\href{https://swprs.org/2017/03/01/der-schweizer-presserat/}{**1. March
2017}

\hypertarget{swiss-policy-research}{%
\subsubsection{Swiss Policy Research}\label{swiss-policy-research}}

\begin{itemize}
\tightlist
\item
  \href{https://swprs.org/kontakt/}{Kontakt}
\item
  \href{https://swprs.org/uebersicht/}{Übersicht}
\item
  \href{https://swprs.org/donationen/}{Donationen}
\item
  \href{https://swprs.org/disclaimer/}{Disclaimer}
\end{itemize}

\hypertarget{english}{%
\subsubsection{English}\label{english}}

\begin{itemize}
\tightlist
\item
  \href{https://swprs.org/contact/}{About Us / Contact}
\item
  \href{https://swprs.org/media-navigator/}{The Media Navigator}
\item
  \href{https://swprs.org/the-american-empire-and-its-media/}{The CFR
  and the Media}
\item
  \href{https://swprs.org/donations/}{Donations}
\end{itemize}

\hypertarget{follow-by-email}{%
\subsubsection{Follow by email}\label{follow-by-email}}

Follow

\href{https://wordpress.com/?ref=footer_custom_com}{WordPress.com}.

\protect\hyperlink{}{Up ↑}

Post to

\protect\hyperlink{}{Cancel}

\includegraphics{https://pixel.wp.com/b.gif?v=noscript}
