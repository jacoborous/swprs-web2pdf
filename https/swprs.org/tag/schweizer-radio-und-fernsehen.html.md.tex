\protect\hyperlink{content}{Skip to content}

\href{https://swprs.org/}{}

\protect\hyperlink{search-container}{Search}

Search for:

\href{https://swprs.org/}{\includegraphics{https://swprs.files.wordpress.com/2020/05/swiss-policy-research-logo-300.png}}

\href{https://swprs.org/}{Swiss Policy Research}

Geopolitics and Media

Menu

\begin{itemize}
\tightlist
\item
  \href{https://swprs.org}{Start}
\item
  \href{https://swprs.org/srf-propaganda-analyse/}{Studien}

  \begin{itemize}
  \tightlist
  \item
    \href{https://swprs.org/srf-propaganda-analyse/}{SRF / ZDF}
  \item
    \href{https://swprs.org/die-nzz-studie/}{NZZ-Studie}
  \item
    \href{https://swprs.org/der-propaganda-multiplikator/}{Agenturen}
  \item
    \href{https://swprs.org/die-propaganda-matrix/}{Medienmatrix}
  \end{itemize}
\item
  \href{https://swprs.org/medien-navigator/}{Analysen}

  \begin{itemize}
  \tightlist
  \item
    \href{https://swprs.org/medien-navigator/}{Navigator}
  \item
    \href{https://swprs.org/der-propaganda-schluessel/}{Techniken}
  \item
    \href{https://swprs.org/propaganda-in-der-wikipedia/}{Wikipedia}
  \item
    \href{https://swprs.org/logik-imperialer-kriege/}{Kriege}
  \end{itemize}
\item
  \href{https://swprs.org/netzwerk-medien-schweiz/}{Netzwerke}

  \begin{itemize}
  \tightlist
  \item
    \href{https://swprs.org/netzwerk-medien-schweiz/}{Schweiz}
  \item
    \href{https://swprs.org/netzwerk-medien-deutschland/}{Deutschland}
  \item
    \href{https://swprs.org/medien-in-oesterreich/}{Österreich}
  \item
    \href{https://swprs.org/das-american-empire-und-seine-medien/}{USA}
  \end{itemize}
\item
  \href{https://swprs.org/bericht-eines-journalisten/}{Fokus I}

  \begin{itemize}
  \tightlist
  \item
    \href{https://swprs.org/bericht-eines-journalisten/}{Journalistenbericht}
  \item
    \href{https://swprs.org/russische-propaganda/}{Russische Propaganda}
  \item
    \href{https://swprs.org/die-israel-lobby-fakten-und-mythen/}{Die
    »Israel-Lobby«}
  \item
    \href{https://swprs.org/geopolitik-und-paedokriminalitaet/}{Pädokriminalität}
  \end{itemize}
\item
  \href{https://swprs.org/migration-und-medien/}{Fokus II}

  \begin{itemize}
  \tightlist
  \item
    \href{https://swprs.org/covid-19-hinweis-ii/}{Coronavirus}
  \item
    \href{https://swprs.org/die-integrity-initiative/}{Integrity
    Initiative}
  \item
    \href{https://swprs.org/migration-und-medien/}{Migration \& Medien}
  \item
    \href{https://swprs.org/der-fall-magnitsky/}{Magnitsky Act}
  \end{itemize}
\item
  \href{https://swprs.org/kontakt/}{Projekt}

  \begin{itemize}
  \tightlist
  \item
    \href{https://swprs.org/kontakt/}{Kontakt}
  \item
    \href{https://swprs.org/uebersicht/}{Seitenübersicht}
  \item
    \href{https://swprs.org/medienspiegel/}{Medienspiegel}
  \item
    \href{https://swprs.org/donationen/}{Donationen}
  \end{itemize}
\item
  \href{https://swprs.org/contact/}{English}
\end{itemize}

\protect\hyperlink{}{Open Search}

\hypertarget{tag-schweizer-radio-und-fernsehen}{%
\section{Tag: Schweizer Radio und
Fernsehen}\label{tag-schweizer-radio-und-fernsehen}}

\hypertarget{srf-die-propaganda-analyse}{%
\section{\texorpdfstring{\href{https://swprs.org/2017/03/01/srf-propaganda-analyse/}{SRF:
Die
Propaganda-Analyse}}{SRF: Die Propaganda-Analyse}}\label{srf-die-propaganda-analyse}}

\href{https://swprs.org/2017/03/01/srf-propaganda-analyse/}{\includegraphics{https://swprs.files.wordpress.com/2016/10/srf-analyse-s.png?w=500}}

Das Schweizer Radio und Fern­se­hen (SRF) leistet mit seinen
Nach­rich­ten- und In­for­ma­tions­sen­dungen einen wich­tigen Bei­trag
zur öffent­lichen Meinungs­bildung in der Schweiz. Doch wie objektiv und
kritisch be­rich­tet das SRF über geo­po­li­tische The­men?

Um dies zu über­prü­fen, wurde erst­mals eine sys­te­ma­tische Ana­lyse
der SRF-​Be­richt­er­stat­tung zu einem geo­po­li­tisch relevanten
Ereig­nis durch­ge­führt.

Die Resul­tate sind alar­mie­rend: In allen unter­such­ten Bei­trä­gen
des SRF wurden Pro­pa­ganda- und Mani­pu­la­tions­tech­niken auf
re­dak­tio­nel­ler, sprach­licher und audio­vi­su­el­ler Ebene
fest­ge­stellt.

\href{https://swprs.org/srf-propaganda-analyse/}{Zur SRF
Propaganda-Analyse →}

\begin{center}\rule{0.5\linewidth}{\linethickness}\end{center}

\href{https://swprs.org/2017/03/01/srf-propaganda-analyse/}{**1. March
2017}

\hypertarget{der-korrespondent}{%
\section{\texorpdfstring{\href{https://swprs.org/2017/03/01/der-korrespondent/}{Der
Korrespondent}}{Der Korrespondent}}\label{der-korrespondent}}

\href{https://swprs.org/2017/03/01/der-korrespondent/}{\includegraphics{https://swprs.files.wordpress.com/2017/03/srf-gsteiger-nato.jpg?w=600}}

Wie wird man Kor­res­pon­dent beim \emph{Schwei­zer Radio und
Fern­sehen}? Fredy Gsteiger muss es wissen: Er ist
\href{http://www.persoenlich.com/medien/fredy-gsteiger-neu-in-der-radio-chefredaktion-232921}{stv.
Chef­redakteur}, Auslands­chef und
\href{http://www.srf.ch/radio-srf-1/radio-srf-1/fredy-gsteiger-unser-mann-in-der-uno}{diplo­ma­tischer
Korres­pon­dent} des \emph{Schwei­zer Radios SRF}. In dieser Funktion
be­richtet er etwa über die UNO, NATO und EU -- und damit z.B. auch über
\href{http://www.srf.ch/news/international/dieser-eu-rueckzieher-ist-peinlich}{Russ­land-Sanktionen}
und die Genfer
\href{http://www.srf.ch/news/international/assad-kommt-mit-giftgaseinsaetzen-vorlaeufig-davon}{Syrien-Ver­hand­lungen}.

Gsteiger begann seine journa­lis­tische Lauf­bahn Ende der 80er Jahre
als Nahost-Redakteur bei der
\href{https://swprs.org/netzwerk-medien-deutschland/}{deutsch-trans­atlan­tischen}
Wochen­zeitung \emph{Die Zeit}. Die Schwei­zer Neutra­lität war für ihn
schon vor dem Ersten Irak­krieg 1991 ein
\emph{\href{http://www.zeit.de/1990/44/ein-konzept-von-gestern}{»Konzept
von gestern«},} wirt­schaft­liche Neutralität ohnehin
\emph{\href{http://www.zeit.de/1990/44/ein-konzept-von-gestern}{»gänz­lich
über­holt«}.} Von 1997 bis 2001 war Gsteiger dann Chef­redakteur bei der
\emph{Welt­woche}. Unter seiner Leitung trat das Blatt »\emph{für den
Bei­tritt der Schweiz zur NATO«} ein, wie er in seinem
\href{https://web.archive.org/web/20040722094101/http://www.weltwoche.ch/artikel/?AssetID=400\&CategoryID=60}{Abschieds­artikel}
schrieb.

Damit kam Gsteiger 2002 zum Schweizer Radio. Be­schwer­den über eine
ein­sei­tige Be­richt­er­stattung wurden von der Ombuds­stelle mehr­fach
\href{https://www.srgd.ch/de/aktuelles/news/2016/09/28/sendung-info-3-auf-radio-srf-3-uber-waffenruhe-syrien-beanstandet/}{abge­lehnt}.
Und so
\href{http://www.swissinfo.ch/ger/kooperation_die-nato-umwirbt-die-schweiz/42225918}{be­tont}
Gsteiger auch heute noch die »\emph{vielen
Koope­ra­tions­möglich­keiten«} mit der NATO;
\href{http://www.srf.ch/news/international/dieser-eu-rueckzieher-ist-peinlich}{be­dauert},
dass die Russ­land-Sanktionen nicht ver­schärft werden; und
\href{http://www.srf.ch/news/international/assad-kommt-mit-giftgaseinsaetzen-vorlaeufig-davon}{weiß
genau}, wer in Syrien der Böse­wicht ist.

\emph{Update:} 2019 erhält das SRF eine neue
\href{https://www.srgd.ch/de/aktuelles/news/2017/12/05/luzia-tschirky-wird-neue-russland-korrespondentin/}{Russland-Korrespondentin}
-- die zuvor beim amerikanischen
\href{https://de.wikipedia.org/wiki/Radio_Free_Europe}{\emph{Radio Free
Europe}} arbeitete. (\emph{Foto oben:} Gsteiger 2014 auf einer
Jour­na­­listen-​Tour der US NATO-Mission.)

\begin{center}\rule{0.5\linewidth}{\linethickness}\end{center}

\href{https://swprs.org/2017/03/01/der-korrespondent/}{**1. March 2017}

\hypertarget{iduxe9e-suisse}{%
\section{\texorpdfstring{\href{https://swprs.org/2017/03/01/srg-idee-suisse/}{Idée
suisse}}{Idée suisse}}\label{iduxe9e-suisse}}

\href{https://swprs.org/2017/03/01/srg-idee-suisse/}{\includegraphics{https://swprs.files.wordpress.com/2016/07/srg-logo-1.png?w=350}}

Groß war der Auf­schrei in den Schweizer Medien, als Polen 2016 ein
\href{http://www.nzz.ch/international/europa/wie-medien-zu-nationalen-kulturinstituten-werden-1.18670792}{neues
Medien­ge­setz} erließ, welches die Er­nennung von Di­rek­toren des
öffent­lichen Rundfunks der Regierung übertrug. Doch wie un­ab­hängig
sind die öffentlichen Medien in der Schweiz?

Die Realität ist er­nüch­ternd: Obschon die \emph{Schwei­ze­rische
Radio- und Fern­seh­ge­sell­schaft (SRG)} gerne betont, dass sie als
\href{https://web.archive.org/web/20190412225655/https://www.srginsider.ch/service-public/2013/10/30/warum-ist-der-ausdruck-staatsfernsehen-oder-oeffentlich-rechtlicher-sender-falsch/}{privater
Verein} orga­ni­siert ist,
\href{https://www.srgd.ch/de/aktuelles/news/2016/11/04/srg-konzession-weiterhin-den-handen-des-bundesrats/}{definiert}
der Bundesrat nicht nur die Sendekonzession, sondern
\href{http://www.srgssr.ch/de/srg/organe/verwaltungsrat/}{ernennt} auch
meh­rere Ver­wal­tungs­rats­mit­glieder sowie
\href{https://www.ubi.admin.ch/}{alle} Mit­glieder der obersten
Pro­gramm­auf­sicht UBI.

Selbst der SRG-Präsi­dent wurde bis 2012 offiziell von der
Landesregierung
\href{https://web.archive.org/web/20150919041519/http://www.srgssr.ch/fileadmin/pdfs/Vereinsgeschichte_SRG.pdf}{be­stimmt}.
Seit­her kommt ein un­durch­sich­tiges Pro­ce­dere zum Ein­satz, bei dem
das Minis­terium vorab über die Kan­di­daten
\href{http://www.tagesanzeiger.ch/schweiz/standard/Neuer-SRGPraesident-verzweifelt-gesucht/story/18371394}{»infor­miert«}
wird. Dabei wurde das An­for­de­rungs­profil sowohl bei der
\href{http://www.aargauerzeitung.ch/schweiz/srg-extrawurst-fuer-roger-de-weck-8808607}{Wahl
des General­di­rektors 2010} wie auch bei der
\href{http://www.nzz.ch/nzzas/nzz-am-sonntag/favorit-fuer-das-srg-praesidium-leuthard-will-cvp-freund-an-srg-spitze-ld.90097}{Wahl
des Prä­si­denten 2016} noch während des Ver­fahrens ange­passt -- und
in beiden Fällen letzt­lich ein
\href{http://www.aargauerzeitung.ch/schweiz/war-roger-de-weck-der-lieblingskandidat-von-moritz-leuenberger-8833796}{»Wunsch­kan­di­dat«}
des am­tie­renden Medien­mi­nis­ters
\href{http://www.nzz.ch/nzzas/nzz-am-sonntag/favorit-fuer-das-srg-praesidium-leuthard-will-cvp-freund-an-srg-spitze-ld.90097}{gewählt}.

\emph{Update:} Auch die neue SRF-Direktorin wurde 2018 in einem
erstaunlich
\href{http://www.kleinreport.ch/news/geheimloge-srg-intransparenz-bei-der-besetzung-der-srf-direktion-91015/}{intransparenten}
Ver­fah­ren bestimmt.

\begin{center}\rule{0.5\linewidth}{\linethickness}\end{center}

\href{https://swprs.org/2017/03/01/srg-idee-suisse/}{**1. March 2017}

\hypertarget{medienaufsicht-im-faktencheck}{%
\section{\texorpdfstring{\href{https://swprs.org/2017/03/01/medienaufsicht-faktencheck/}{Medienaufsicht
im
Faktencheck}}{Medienaufsicht im Faktencheck}}\label{medienaufsicht-im-faktencheck}}

\href{https://swprs.org/2017/03/01/medienaufsicht-faktencheck/}{\includegraphics{https://swprs.files.wordpress.com/2017/03/srf-ombudsstelle.png?w=600}}

Die Ombudsstelle des \emph{SRF} ist die erste Anlaufstelle für
Programm­be­schwerden des Publi­kums. Doch wie un­vor­ein­ge­nommen und
objektiv behandelt sie Beschwerden zu geo­po­li­tischen Themen?

Um dies zu über­prüfen, wurden während eines halben Jahres alle
Schluss­be­richte zum Syrien­kon­flikt einem Fakten­check unter­zogen.
Die Resul­tate sind bedenk­lich.

\href{https://swprs.org/srf-ombudsstelle-im-faktencheck/}{Zum
Faktencheck~→}

\begin{center}\rule{0.5\linewidth}{\linethickness}\end{center}

\href{https://swprs.org/2017/03/01/medienaufsicht-faktencheck/}{**1.
March 2017}

\hypertarget{swiss-policy-research}{%
\subsubsection{Swiss Policy Research}\label{swiss-policy-research}}

\begin{itemize}
\tightlist
\item
  \href{https://swprs.org/kontakt/}{Kontakt}
\item
  \href{https://swprs.org/uebersicht/}{Übersicht}
\item
  \href{https://swprs.org/donationen/}{Donationen}
\item
  \href{https://swprs.org/disclaimer/}{Disclaimer}
\end{itemize}

\hypertarget{english}{%
\subsubsection{English}\label{english}}

\begin{itemize}
\tightlist
\item
  \href{https://swprs.org/contact/}{About Us / Contact}
\item
  \href{https://swprs.org/media-navigator/}{The Media Navigator}
\item
  \href{https://swprs.org/the-american-empire-and-its-media/}{The CFR
  and the Media}
\item
  \href{https://swprs.org/donations/}{Donations}
\end{itemize}

\hypertarget{follow-by-email}{%
\subsubsection{Follow by email}\label{follow-by-email}}

Follow

\href{https://wordpress.com/?ref=footer_custom_com}{WordPress.com}.

\protect\hyperlink{}{Up ↑}

\includegraphics{https://pixel.wp.com/b.gif?v=noscript}
