\protect\hyperlink{content}{Skip to content}

\href{https://swprs.org/}{}

\protect\hyperlink{search-container}{Search}

Search for:

\href{https://swprs.org/}{\includegraphics{https://swprs.files.wordpress.com/2020/05/swiss-policy-research-logo-300.png}}

\href{https://swprs.org/}{Swiss Policy Research}

Geopolitics and Media

Menu

\begin{itemize}
\tightlist
\item
  \href{https://swprs.org}{Start}
\item
  \href{https://swprs.org/srf-propaganda-analyse/}{Studien}

  \begin{itemize}
  \tightlist
  \item
    \href{https://swprs.org/srf-propaganda-analyse/}{SRF / ZDF}
  \item
    \href{https://swprs.org/die-nzz-studie/}{NZZ-Studie}
  \item
    \href{https://swprs.org/der-propaganda-multiplikator/}{Agenturen}
  \item
    \href{https://swprs.org/die-propaganda-matrix/}{Medienmatrix}
  \end{itemize}
\item
  \href{https://swprs.org/medien-navigator/}{Analysen}

  \begin{itemize}
  \tightlist
  \item
    \href{https://swprs.org/medien-navigator/}{Navigator}
  \item
    \href{https://swprs.org/der-propaganda-schluessel/}{Techniken}
  \item
    \href{https://swprs.org/propaganda-in-der-wikipedia/}{Wikipedia}
  \item
    \href{https://swprs.org/logik-imperialer-kriege/}{Kriege}
  \end{itemize}
\item
  \href{https://swprs.org/netzwerk-medien-schweiz/}{Netzwerke}

  \begin{itemize}
  \tightlist
  \item
    \href{https://swprs.org/netzwerk-medien-schweiz/}{Schweiz}
  \item
    \href{https://swprs.org/netzwerk-medien-deutschland/}{Deutschland}
  \item
    \href{https://swprs.org/medien-in-oesterreich/}{Österreich}
  \item
    \href{https://swprs.org/das-american-empire-und-seine-medien/}{USA}
  \end{itemize}
\item
  \href{https://swprs.org/bericht-eines-journalisten/}{Fokus I}

  \begin{itemize}
  \tightlist
  \item
    \href{https://swprs.org/bericht-eines-journalisten/}{Journalistenbericht}
  \item
    \href{https://swprs.org/russische-propaganda/}{Russische Propaganda}
  \item
    \href{https://swprs.org/die-israel-lobby-fakten-und-mythen/}{Die
    »Israel-Lobby«}
  \item
    \href{https://swprs.org/geopolitik-und-paedokriminalitaet/}{Pädokriminalität}
  \end{itemize}
\item
  \href{https://swprs.org/migration-und-medien/}{Fokus II}

  \begin{itemize}
  \tightlist
  \item
    \href{https://swprs.org/covid-19-hinweis-ii/}{Coronavirus}
  \item
    \href{https://swprs.org/die-integrity-initiative/}{Integrity
    Initiative}
  \item
    \href{https://swprs.org/migration-und-medien/}{Migration \& Medien}
  \item
    \href{https://swprs.org/der-fall-magnitsky/}{Magnitsky Act}
  \end{itemize}
\item
  \href{https://swprs.org/kontakt/}{Projekt}

  \begin{itemize}
  \tightlist
  \item
    \href{https://swprs.org/kontakt/}{Kontakt}
  \item
    \href{https://swprs.org/uebersicht/}{Seitenübersicht}
  \item
    \href{https://swprs.org/medienspiegel/}{Medienspiegel}
  \item
    \href{https://swprs.org/donationen/}{Donationen}
  \end{itemize}
\item
  \href{https://swprs.org/contact/}{English}
\end{itemize}

\protect\hyperlink{}{Open Search}

\hypertarget{tag-srf}{%
\section{Tag: SRF}\label{tag-srf}}

\hypertarget{srf-die-propaganda-analyse}{%
\section{\texorpdfstring{\href{https://swprs.org/2017/03/01/srf-propaganda-analyse/}{SRF:
Die
Propaganda-Analyse}}{SRF: Die Propaganda-Analyse}}\label{srf-die-propaganda-analyse}}

\href{https://swprs.org/2017/03/01/srf-propaganda-analyse/}{\includegraphics{https://swprs.files.wordpress.com/2016/10/srf-analyse-s.png?w=500}}

Das Schweizer Radio und Fern­se­hen (SRF) leistet mit seinen
Nach­rich­ten- und In­for­ma­tions­sen­dungen einen wich­tigen Bei­trag
zur öffent­lichen Meinungs­bildung in der Schweiz. Doch wie objektiv und
kritisch be­rich­tet das SRF über geo­po­li­tische The­men?

Um dies zu über­prü­fen, wurde erst­mals eine sys­te­ma­tische Ana­lyse
der SRF-​Be­richt­er­stat­tung zu einem geo­po­li­tisch relevanten
Ereig­nis durch­ge­führt.

Die Resul­tate sind alar­mie­rend: In allen unter­such­ten Bei­trä­gen
des SRF wurden Pro­pa­ganda- und Mani­pu­la­tions­tech­niken auf
re­dak­tio­nel­ler, sprach­licher und audio­vi­su­el­ler Ebene
fest­ge­stellt.

\href{https://swprs.org/srf-propaganda-analyse/}{Zur SRF
Propaganda-Analyse →}

\begin{center}\rule{0.5\linewidth}{\linethickness}\end{center}

\href{https://swprs.org/2017/03/01/srf-propaganda-analyse/}{**1. March
2017}

\hypertarget{propaganda-im-staatsauftrag}{%
\section{\texorpdfstring{\href{https://swprs.org/2017/03/01/propaganda-im-staatsauftrag/}{Propaganda
im
Staatsauftrag?}}{Propaganda im Staatsauftrag?}}\label{propaganda-im-staatsauftrag}}

\href{https://swprs.org/2017/03/01/propaganda-im-staatsauftrag/}{\includegraphics{https://swprs.files.wordpress.com/2016/02/srf-syrien11.png?w=600}}

Von den öf‌fentlichen Rund­funk­an­stalten er­war­tet das Pu­bli­kum
eine aus­ge­wogene Bericht­er­stattung. Doch of‌t ist ge­rade dort der
politische Druck be­sonders hoch, sich an das
\href{https://swprs.org/das-gewuenschte-narrativ/}{trans­at­lan­tische
Narra­tiv} zu halten.

So haben Mitarbeiter der \emph{ARD} gemäß internen Memos Weisung, bei
geo­po­li­tischen Kon­f‌lik­ten
\emph{\href{https://www.heise.de/tp/features/Ukraine-Konflikt-ARD-Programmbeirat-bestaetigt-Publikumskritik-3367400.html}{»west­liche
Posi­tionen zu ver­tei­di­gen«}}, ver­trau­liche
\href{https://www.heise.de/tp/features/Die-vertraulichen-Sprachregelungen-der-ARD-3758887.html}{Sprach­­re­­ge­lungen}
zu be­fol­gen und aus­­schließ­­lich
\href{https://www.oxmoxhh.de/magazin/story-interview/oxmox-exklusiv-interview-mit-volker-braeutigam-friedhelm-klinkhammer/}{konforme
Quellen} zu ver­wen­den.

Beim \emph{ZDF} machte der ehe­ma­lige Chef­re­dakteur publik, dass
Bei­träge zu US-Kriegen poli­tisch
\href{https://www.youtube.com/watch?v=i2423aDq_hE}{be­ein­f‌‌lusst}
werden. Nahost-Kor­res­pon­dent Ulrich Tilgner be­klagte
re­dak­tio­nelle Ein­grif‌fe aufgrund von
\href{http://www.berliner-zeitung.de/korrespondent-ulrich-tilgner-sucht-mehr-distanz-zum-zdf--ich-fuehle-mich-eingeschraenkt--15870684}{»Bünd­nis­rück­sich­ten«},
und der vormalige Leiter des *ZDF-*Studios Bonn be­stä­tig­te
\href{https://propagandaschau.wordpress.com/2016/01/30/wolfgang-herles-es-gibt-in-den-oeffentlich-rechtlichen-anweisungen-von-oben/}{»An­wei­sungen
von oben«} und eine
\href{http://www.rolandtichy.de/daili-es-sentials/meinungsfreiheit-anordnung-zur-anpassung/}{»frei­willige
Gleich­schal­tung«} der Jour­na­lis­ten.

Auch das \emph{SRF} verwendet diverse
\href{https://swprs.org/srf-propaganda-analyse/}{Mani­pu­lations­tech­niken}
zugunsten der Konflikt­partei USA \& NATO und
\href{http://www.srf.ch/sendungen/srfglobal/propagandagruesse-aus-moskau-2}{thematisiert}
Propaganda stets nur auf der Gegenseite. Selbst vor dem Einsatz
sub­tiler
\href{http://www.srf.ch/play/tv/10vor10/video/warum-assad-bleibt?id=a6d267c9-52b3-470b-868e-95bb919a0b96}{Grusel­musik}
in den Nach­rich­ten schreckt das \emph{SRF} nicht zurück, um Gegner der
US-Allianz zu dämo­ni­sieren.

Programmbe­schwer­den sind indes chan­cen­los, denn: Beiträge zu
inter­na­tio­nalen Kon­flik­ten müssten
\emph{\href{https://swprs.org/srf-ombudsstelle-im-faktencheck/}{``weder
neutral noch ausgewogen''}} sein, und \emph{``die­je­ni­gen, die dem SRF
vor­wer­fen, ein­sei­tig der US- und Nato-Pro­pa­gan­da zu er­lie­gen,
be­trei­ben ihrer­seits das Ge­schäf‌t der russischen Pro­pa­ganda''} --
so die erstaun­liche
\href{https://swprs.org/srf-ombudsstelle-im-faktencheck/}{Ar­gu­men­ta­tion}
der Om­buds­stelle.*\\
*

\begin{center}\rule{0.5\linewidth}{\linethickness}\end{center}

\href{https://swprs.org/2017/03/01/propaganda-im-staatsauftrag/}{**1.
March 2017}

\hypertarget{der-korrespondent}{%
\section{\texorpdfstring{\href{https://swprs.org/2017/03/01/der-korrespondent/}{Der
Korrespondent}}{Der Korrespondent}}\label{der-korrespondent}}

\href{https://swprs.org/2017/03/01/der-korrespondent/}{\includegraphics{https://swprs.files.wordpress.com/2017/03/srf-gsteiger-nato.jpg?w=600}}

Wie wird man Kor­res­pon­dent beim \emph{Schwei­zer Radio und
Fern­sehen}? Fredy Gsteiger muss es wissen: Er ist
\href{http://www.persoenlich.com/medien/fredy-gsteiger-neu-in-der-radio-chefredaktion-232921}{stv.
Chef­redakteur}, Auslands­chef und
\href{http://www.srf.ch/radio-srf-1/radio-srf-1/fredy-gsteiger-unser-mann-in-der-uno}{diplo­ma­tischer
Korres­pon­dent} des \emph{Schwei­zer Radios SRF}. In dieser Funktion
be­richtet er etwa über die UNO, NATO und EU -- und damit z.B. auch über
\href{http://www.srf.ch/news/international/dieser-eu-rueckzieher-ist-peinlich}{Russ­land-Sanktionen}
und die Genfer
\href{http://www.srf.ch/news/international/assad-kommt-mit-giftgaseinsaetzen-vorlaeufig-davon}{Syrien-Ver­hand­lungen}.

Gsteiger begann seine journa­lis­tische Lauf­bahn Ende der 80er Jahre
als Nahost-Redakteur bei der
\href{https://swprs.org/netzwerk-medien-deutschland/}{deutsch-trans­atlan­tischen}
Wochen­zeitung \emph{Die Zeit}. Die Schwei­zer Neutra­lität war für ihn
schon vor dem Ersten Irak­krieg 1991 ein
\emph{\href{http://www.zeit.de/1990/44/ein-konzept-von-gestern}{»Konzept
von gestern«},} wirt­schaft­liche Neutralität ohnehin
\emph{\href{http://www.zeit.de/1990/44/ein-konzept-von-gestern}{»gänz­lich
über­holt«}.} Von 1997 bis 2001 war Gsteiger dann Chef­redakteur bei der
\emph{Welt­woche}. Unter seiner Leitung trat das Blatt »\emph{für den
Bei­tritt der Schweiz zur NATO«} ein, wie er in seinem
\href{https://web.archive.org/web/20040722094101/http://www.weltwoche.ch/artikel/?AssetID=400\&CategoryID=60}{Abschieds­artikel}
schrieb.

Damit kam Gsteiger 2002 zum Schweizer Radio. Be­schwer­den über eine
ein­sei­tige Be­richt­er­stattung wurden von der Ombuds­stelle mehr­fach
\href{https://www.srgd.ch/de/aktuelles/news/2016/09/28/sendung-info-3-auf-radio-srf-3-uber-waffenruhe-syrien-beanstandet/}{abge­lehnt}.
Und so
\href{http://www.swissinfo.ch/ger/kooperation_die-nato-umwirbt-die-schweiz/42225918}{be­tont}
Gsteiger auch heute noch die »\emph{vielen
Koope­ra­tions­möglich­keiten«} mit der NATO;
\href{http://www.srf.ch/news/international/dieser-eu-rueckzieher-ist-peinlich}{be­dauert},
dass die Russ­land-Sanktionen nicht ver­schärft werden; und
\href{http://www.srf.ch/news/international/assad-kommt-mit-giftgaseinsaetzen-vorlaeufig-davon}{weiß
genau}, wer in Syrien der Böse­wicht ist.

\emph{Update:} 2019 erhält das SRF eine neue
\href{https://www.srgd.ch/de/aktuelles/news/2017/12/05/luzia-tschirky-wird-neue-russland-korrespondentin/}{Russland-Korrespondentin}
-- die zuvor beim amerikanischen
\href{https://de.wikipedia.org/wiki/Radio_Free_Europe}{\emph{Radio Free
Europe}} arbeitete. (\emph{Foto oben:} Gsteiger 2014 auf einer
Jour­na­­listen-​Tour der US NATO-Mission.)

\begin{center}\rule{0.5\linewidth}{\linethickness}\end{center}

\href{https://swprs.org/2017/03/01/der-korrespondent/}{**1. March 2017}

\hypertarget{iduxe9e-suisse}{%
\section{\texorpdfstring{\href{https://swprs.org/2017/03/01/srg-idee-suisse/}{Idée
suisse}}{Idée suisse}}\label{iduxe9e-suisse}}

\href{https://swprs.org/2017/03/01/srg-idee-suisse/}{\includegraphics{https://swprs.files.wordpress.com/2016/07/srg-logo-1.png?w=350}}

Groß war der Auf­schrei in den Schweizer Medien, als Polen 2016 ein
\href{http://www.nzz.ch/international/europa/wie-medien-zu-nationalen-kulturinstituten-werden-1.18670792}{neues
Medien­ge­setz} erließ, welches die Er­nennung von Di­rek­toren des
öffent­lichen Rundfunks der Regierung übertrug. Doch wie un­ab­hängig
sind die öffentlichen Medien in der Schweiz?

Die Realität ist er­nüch­ternd: Obschon die \emph{Schwei­ze­rische
Radio- und Fern­seh­ge­sell­schaft (SRG)} gerne betont, dass sie als
\href{https://web.archive.org/web/20190412225655/https://www.srginsider.ch/service-public/2013/10/30/warum-ist-der-ausdruck-staatsfernsehen-oder-oeffentlich-rechtlicher-sender-falsch/}{privater
Verein} orga­ni­siert ist,
\href{https://www.srgd.ch/de/aktuelles/news/2016/11/04/srg-konzession-weiterhin-den-handen-des-bundesrats/}{definiert}
der Bundesrat nicht nur die Sendekonzession, sondern
\href{http://www.srgssr.ch/de/srg/organe/verwaltungsrat/}{ernennt} auch
meh­rere Ver­wal­tungs­rats­mit­glieder sowie
\href{https://www.ubi.admin.ch/}{alle} Mit­glieder der obersten
Pro­gramm­auf­sicht UBI.

Selbst der SRG-Präsi­dent wurde bis 2012 offiziell von der
Landesregierung
\href{https://web.archive.org/web/20150919041519/http://www.srgssr.ch/fileadmin/pdfs/Vereinsgeschichte_SRG.pdf}{be­stimmt}.
Seit­her kommt ein un­durch­sich­tiges Pro­ce­dere zum Ein­satz, bei dem
das Minis­terium vorab über die Kan­di­daten
\href{http://www.tagesanzeiger.ch/schweiz/standard/Neuer-SRGPraesident-verzweifelt-gesucht/story/18371394}{»infor­miert«}
wird. Dabei wurde das An­for­de­rungs­profil sowohl bei der
\href{http://www.aargauerzeitung.ch/schweiz/srg-extrawurst-fuer-roger-de-weck-8808607}{Wahl
des General­di­rektors 2010} wie auch bei der
\href{http://www.nzz.ch/nzzas/nzz-am-sonntag/favorit-fuer-das-srg-praesidium-leuthard-will-cvp-freund-an-srg-spitze-ld.90097}{Wahl
des Prä­si­denten 2016} noch während des Ver­fahrens ange­passt -- und
in beiden Fällen letzt­lich ein
\href{http://www.aargauerzeitung.ch/schweiz/war-roger-de-weck-der-lieblingskandidat-von-moritz-leuenberger-8833796}{»Wunsch­kan­di­dat«}
des am­tie­renden Medien­mi­nis­ters
\href{http://www.nzz.ch/nzzas/nzz-am-sonntag/favorit-fuer-das-srg-praesidium-leuthard-will-cvp-freund-an-srg-spitze-ld.90097}{gewählt}.

\emph{Update:} Auch die neue SRF-Direktorin wurde 2018 in einem
erstaunlich
\href{http://www.kleinreport.ch/news/geheimloge-srg-intransparenz-bei-der-besetzung-der-srf-direktion-91015/}{intransparenten}
Ver­fah­ren bestimmt.

\begin{center}\rule{0.5\linewidth}{\linethickness}\end{center}

\href{https://swprs.org/2017/03/01/srg-idee-suisse/}{**1. March 2017}

\hypertarget{die-srf-rundschau-hinterfragt}{%
\section{\texorpdfstring{\href{https://swprs.org/2017/03/01/srf-rundschau/}{Die
SRF-Rundschau
hinterfragt}}{Die SRF-Rundschau hinterfragt}}\label{die-srf-rundschau-hinterfragt}}

\href{https://swprs.org/2017/03/01/srf-rundschau/}{\includegraphics{https://swprs.files.wordpress.com/2018/07/rundschau.png?w=500}}

Die
\emph{\href{https://de.wikipedia.org/wiki/Rundschau_(SRF)}{Rundschau}}
ist das bekannteste Polit­ma­ga­zin des \emph{Schweizer Fernsehens.} Sie
\href{https://www.srf.ch/sendungen/rundschau/50-jahre-rundschau-die-jubilaeumssendung}{möchte}
»die Mächtigen hinterfragen« -- doch geht's um Geo­po­litik, so steht
sie meist selbst auf deren Seite.

Während die
\href{https://www.srf.ch/sendungen/rundschau/subventionierte-piloten-vaeter-am-limit-bombenhoelle-aleppo}{»Bombenhölle«}
Aleppo »fällt«, wird Mossul »befreit« -- von einem Familien­vater, der
gerne »US-Popmusik«
\href{https://www.srf.ch/sendungen/rundschau/buben-beschneidung-michel-bollag-pkb-west-mossul}{hört}.

Vom »Giftgasangriff« bei Ghouta
\href{https://www.srf.ch/sendungen/rundschau/kriminaltouristen-verhuetungsmittel-j-bitzer-giftgaseinsatz}{berichtet}
der »Augen­zeuge« einer »Hilfs­organisation« -- wer diese
\href{http://www.uossm.org/who_we_are}{finanziert}, verrät die
\emph{Rundschau} nicht.

Beim
\href{https://www.srf.ch/sendungen/rundschau/gehorsam-und-ehelos-klamauk-statt-kompromiss-vergessener-krieg}{»Vergessenen
Krieg«} im Jemen werden prompt die saudischen Luftangriffe »vergessen«
-- und deren westliche
\href{https://www.strategic-culture.org/news/2018/06/18/western-media-whitewash-yemen-genocide.html}{Unterstützung}
ebenso.

Putin indes hege
\href{https://www.srf.ch/sendungen/rundschau/gianni-infantino-fatma-samoura-iv-kosovaren-zittern-vor-russen}{»Expansionsgelüste«}
und füh­re einen
\href{https://www.srf.ch/sendungen/rundschau/putins-informationskrieg-milliarden-jongleur-bastos-camorra}{»Informationskrieg«},
seine Angriffe auf den Westen seien bereits »mehrfach be­legt« -- doch
statt Fakten
\href{https://www.srf.ch/sendungen/rundschau/putins-informationskrieg-milliarden-jongleur-bastos-camorra}{folgen}
finstere Sound­effekte.

Schon der Gründer der \emph{Rundschau} und spätere Leiter der
\emph{Tagesschau} nahm an der Konferenz der trans­atlantischen Elite
\href{https://wikileaks.org/plusd/cables/1978ZURICH00660_d.html}{teil}
-- ob man dort lernt, die Mächtigen zu hinterfragen?

\begin{center}\rule{0.5\linewidth}{\linethickness}\end{center}

\href{https://swprs.org/2017/03/01/srf-rundschau/}{**1. March 2017}

\hypertarget{anschlag-auf-die-forschungsfreiheit}{%
\section{\texorpdfstring{\href{https://swprs.org/2017/03/01/anschlag-auf-die-forschungsfreiheit/}{Anschlag
auf die
Forschungsfreiheit}}{Anschlag auf die Forschungsfreiheit}}\label{anschlag-auf-die-forschungsfreiheit}}

\href{https://swprs.org/2017/03/01/anschlag-auf-die-forschungsfreiheit/}{\includegraphics{https://swprs.files.wordpress.com/2018/11/ganser.png?w=500}}

So ergeht es US-kritischen Forschern in der Schweiz: Der Historiker Dr.
Daniele Ganser geriet 2006 nach einer öffentlichen Inter­vention der
amerika­nischen Bot­schaf­terin unter Druck und musste seine Forschung
an der ETH Zürich schließlich aufgeben.

Ganser forschte zu ver­deckter Kriegs­führung und
\href{http://ofv.ch/sachbuch/detail/natogeheimarmeen-in-europa/3193/}{ins­ze­nier­tem
Terror} durch die NATO im Kalten Krieg sowie zu den An­schlägen vom 11.
September 2001 (s.
\href{http://archiv.ethlife.ethz.ch/articles/9.11.html}{Artikel im
ETH-Magazin}).

\href{https://swprs.org/anschlag-auf-die-forschungsfreiheit\#weiterlesen}{Weiterlesen
→}

\begin{center}\rule{0.5\linewidth}{\linethickness}\end{center}

\href{https://swprs.org/2017/03/01/anschlag-auf-die-forschungsfreiheit/}{**1.
March 2017}

\hypertarget{medienaufsicht-im-faktencheck}{%
\section{\texorpdfstring{\href{https://swprs.org/2017/03/01/medienaufsicht-faktencheck/}{Medienaufsicht
im
Faktencheck}}{Medienaufsicht im Faktencheck}}\label{medienaufsicht-im-faktencheck}}

\href{https://swprs.org/2017/03/01/medienaufsicht-faktencheck/}{\includegraphics{https://swprs.files.wordpress.com/2017/03/srf-ombudsstelle.png?w=600}}

Die Ombudsstelle des \emph{SRF} ist die erste Anlaufstelle für
Programm­be­schwerden des Publi­kums. Doch wie un­vor­ein­ge­nommen und
objektiv behandelt sie Beschwerden zu geo­po­li­tischen Themen?

Um dies zu über­prüfen, wurden während eines halben Jahres alle
Schluss­be­richte zum Syrien­kon­flikt einem Fakten­check unter­zogen.
Die Resul­tate sind bedenk­lich.

\href{https://swprs.org/srf-ombudsstelle-im-faktencheck/}{Zum
Faktencheck~→}

\begin{center}\rule{0.5\linewidth}{\linethickness}\end{center}

\href{https://swprs.org/2017/03/01/medienaufsicht-faktencheck/}{**1.
March 2017}

\hypertarget{swiss-policy-research}{%
\subsubsection{Swiss Policy Research}\label{swiss-policy-research}}

\begin{itemize}
\tightlist
\item
  \href{https://swprs.org/kontakt/}{Kontakt}
\item
  \href{https://swprs.org/uebersicht/}{Übersicht}
\item
  \href{https://swprs.org/donationen/}{Donationen}
\item
  \href{https://swprs.org/disclaimer/}{Disclaimer}
\end{itemize}

\hypertarget{english}{%
\subsubsection{English}\label{english}}

\begin{itemize}
\tightlist
\item
  \href{https://swprs.org/contact/}{About Us / Contact}
\item
  \href{https://swprs.org/media-navigator/}{The Media Navigator}
\item
  \href{https://swprs.org/the-american-empire-and-its-media/}{The CFR
  and the Media}
\item
  \href{https://swprs.org/donations/}{Donations}
\end{itemize}

\hypertarget{follow-by-email}{%
\subsubsection{Follow by email}\label{follow-by-email}}

Follow

\href{https://wordpress.com/?ref=footer_custom_com}{WordPress.com}.

\protect\hyperlink{}{Up ↑}

\includegraphics{https://pixel.wp.com/b.gif?v=noscript}
