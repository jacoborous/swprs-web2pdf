\protect\hyperlink{content}{Skip to content}

\href{https://swprs.org/}{}

\protect\hyperlink{search-container}{Search}

Search for:

\href{https://swprs.org/}{\includegraphics{https://swprs.files.wordpress.com/2020/05/swiss-policy-research-logo-300.png}}

\href{https://swprs.org/}{Swiss Policy Research}

Geopolitics and Media

Menu

\begin{itemize}
\tightlist
\item
  \href{https://swprs.org}{Start}
\item
  \href{https://swprs.org/srf-propaganda-analyse/}{Studien}

  \begin{itemize}
  \tightlist
  \item
    \href{https://swprs.org/srf-propaganda-analyse/}{SRF / ZDF}
  \item
    \href{https://swprs.org/die-nzz-studie/}{NZZ-Studie}
  \item
    \href{https://swprs.org/der-propaganda-multiplikator/}{Agenturen}
  \item
    \href{https://swprs.org/die-propaganda-matrix/}{Medienmatrix}
  \end{itemize}
\item
  \href{https://swprs.org/medien-navigator/}{Analysen}

  \begin{itemize}
  \tightlist
  \item
    \href{https://swprs.org/medien-navigator/}{Navigator}
  \item
    \href{https://swprs.org/der-propaganda-schluessel/}{Techniken}
  \item
    \href{https://swprs.org/propaganda-in-der-wikipedia/}{Wikipedia}
  \item
    \href{https://swprs.org/logik-imperialer-kriege/}{Kriege}
  \end{itemize}
\item
  \href{https://swprs.org/netzwerk-medien-schweiz/}{Netzwerke}

  \begin{itemize}
  \tightlist
  \item
    \href{https://swprs.org/netzwerk-medien-schweiz/}{Schweiz}
  \item
    \href{https://swprs.org/netzwerk-medien-deutschland/}{Deutschland}
  \item
    \href{https://swprs.org/medien-in-oesterreich/}{Österreich}
  \item
    \href{https://swprs.org/das-american-empire-und-seine-medien/}{USA}
  \end{itemize}
\item
  \href{https://swprs.org/bericht-eines-journalisten/}{Fokus I}

  \begin{itemize}
  \tightlist
  \item
    \href{https://swprs.org/bericht-eines-journalisten/}{Journalistenbericht}
  \item
    \href{https://swprs.org/russische-propaganda/}{Russische Propaganda}
  \item
    \href{https://swprs.org/die-israel-lobby-fakten-und-mythen/}{Die
    »Israel-Lobby«}
  \item
    \href{https://swprs.org/geopolitik-und-paedokriminalitaet/}{Pädokriminalität}
  \end{itemize}
\item
  \href{https://swprs.org/migration-und-medien/}{Fokus II}

  \begin{itemize}
  \tightlist
  \item
    \href{https://swprs.org/covid-19-hinweis-ii/}{Coronavirus}
  \item
    \href{https://swprs.org/die-integrity-initiative/}{Integrity
    Initiative}
  \item
    \href{https://swprs.org/migration-und-medien/}{Migration \& Medien}
  \item
    \href{https://swprs.org/der-fall-magnitsky/}{Magnitsky Act}
  \end{itemize}
\item
  \href{https://swprs.org/kontakt/}{Projekt}

  \begin{itemize}
  \tightlist
  \item
    \href{https://swprs.org/kontakt/}{Kontakt}
  \item
    \href{https://swprs.org/uebersicht/}{Seitenübersicht}
  \item
    \href{https://swprs.org/medienspiegel/}{Medienspiegel}
  \item
    \href{https://swprs.org/donationen/}{Donationen}
  \end{itemize}
\item
  \href{https://swprs.org/contact/}{English}
\end{itemize}

\protect\hyperlink{}{Open Search}

\hypertarget{tag-nato}{%
\section{Tag: NATO}\label{tag-nato}}

\hypertarget{die-partnerschaft-mit-der-nato}{%
\section{\texorpdfstring{\href{https://swprs.org/2017/03/01/schweizer-medien-nato/}{Die
Partnerschaft mit
der~NATO}}{Die Partnerschaft mit der~NATO}}\label{die-partnerschaft-mit-der-nato}}

\href{https://swprs.org/2017/03/01/schweizer-medien-nato/}{\includegraphics{https://swprs.files.wordpress.com/2016/07/nato-logo-3s.png?w=305}}

Die Schweiz ist nicht Mit­glied in der NATO, trat jedoch 1996 der
\emph{\href{http://www.pfp.admin.ch/}{»NATO Partner­ship for Peace«}}
und 1997 dem
\emph{\href{http://www.nato.int/docu/review/2007/issue2/german/art5.html}{Euro-Atlan­tischen
Par­tner­schafts­rat}} bei -- je­weils ohne Volks­ab­stimmung.

Seit­dem kommt das Schweizer Militär im Zuge von NATO-Inter­­ven­­tionen
zum \href{https://www.peace-support.ch/de/}{Einsatz}, so im Kosovo, in
Bosnien und in Afgha­ni­stan (ISAF). Auch der Schweizer
Nach­richten­dienst (NDB) wird inzwischen von einem
\href{https://www.admin.ch/gov/de/start/dokumentation/medienmitteilungen.msg-id-70400.html}{General}
geführt, der durch die NATO ausgebildet wurde.

Würden Schweizer Medien trotz NATO-Part­ner­schaft allzu kritisch über
Interventionen der US-Allianz berichten, so könnte dies als
\href{https://swprs.org/russische-propaganda/}{»feind­li­che
Pro­pa­gan­da«} ge­wer­tet werden -- was po­li­tisch und ökonomisch
wenig opportun wäre.

Auf diese Weise ergibt sich eine weitgehend
\href{https://swprs.org/medien-navigator/}{NATO-kon­forme} Darstellung
von geopolitischen Kon­flik­ten, so in Jugoslawien, Afgha­ni­stan, Irak,
Li­by­en, Syrien, Jemen oder der Ukraine.

An wirtschaftlichen Sanktionen muss sich die Schweiz auf Wunsch der USA
schon seit 1951
\href{https://de.wikipedia.org/wiki/Hotz-Linder-Agreement}{be­tei­li­gen}.
Jour­na­listen, die diese Ver­letzung der Neu­tra­lität damals
kri­ti­sierten,
\href{https://web.archive.org/web/20141206061445/http://buchundnetz.com/online-buch/schnueffelstaat-schweiz-ob/iii-modernisieren-oder-abschaffen/staatsschutz-je-nach-wetterlage/}{er­hielten}
15 Mo­nate Gefäng­nis wegen Landes­verrats.

\begin{center}\rule{0.5\linewidth}{\linethickness}\end{center}

\href{https://swprs.org/2017/03/01/schweizer-medien-nato/}{**1. March
2017}

\hypertarget{das-transatlantik-netzwerk}{%
\section{\texorpdfstring{\href{https://swprs.org/2017/03/01/das-netzwerk/}{Das
Transatlantik-Netzwerk}}{Das Transatlantik-Netzwerk}}\label{das-transatlantik-netzwerk}}

Wie sind Schweizer Medien in trans­at­lantische Netz­werke
ein­ge­bunden? Welche Personen, Organi­sa­tionen und Kon­fe­ren­zen sind
von Bedeutung? Unsere Info­grafik gibt Auskunft.

\href{https://swprs.org/netzwerk-medien-schweiz}{\includegraphics{https://swprs.files.wordpress.com/2019/10/medien-netzwerk-schweiz-hdz-s.png?w=736}}

\href{https://swprs.org/netzwerk-medien-schweiz}{Zur Infografik →}

\begin{center}\rule{0.5\linewidth}{\linethickness}\end{center}

\href{https://swprs.org/2017/03/01/das-netzwerk/}{**1. March 2017}

\hypertarget{der-korrespondent}{%
\section{\texorpdfstring{\href{https://swprs.org/2017/03/01/der-korrespondent/}{Der
Korrespondent}}{Der Korrespondent}}\label{der-korrespondent}}

\href{https://swprs.org/2017/03/01/der-korrespondent/}{\includegraphics{https://swprs.files.wordpress.com/2017/03/srf-gsteiger-nato.jpg?w=600}}

Wie wird man Kor­res­pon­dent beim \emph{Schwei­zer Radio und
Fern­sehen}? Fredy Gsteiger muss es wissen: Er ist
\href{http://www.persoenlich.com/medien/fredy-gsteiger-neu-in-der-radio-chefredaktion-232921}{stv.
Chef­redakteur}, Auslands­chef und
\href{http://www.srf.ch/radio-srf-1/radio-srf-1/fredy-gsteiger-unser-mann-in-der-uno}{diplo­ma­tischer
Korres­pon­dent} des \emph{Schwei­zer Radios SRF}. In dieser Funktion
be­richtet er etwa über die UNO, NATO und EU -- und damit z.B. auch über
\href{http://www.srf.ch/news/international/dieser-eu-rueckzieher-ist-peinlich}{Russ­land-Sanktionen}
und die Genfer
\href{http://www.srf.ch/news/international/assad-kommt-mit-giftgaseinsaetzen-vorlaeufig-davon}{Syrien-Ver­hand­lungen}.

Gsteiger begann seine journa­lis­tische Lauf­bahn Ende der 80er Jahre
als Nahost-Redakteur bei der
\href{https://swprs.org/netzwerk-medien-deutschland/}{deutsch-trans­atlan­tischen}
Wochen­zeitung \emph{Die Zeit}. Die Schwei­zer Neutra­lität war für ihn
schon vor dem Ersten Irak­krieg 1991 ein
\emph{\href{http://www.zeit.de/1990/44/ein-konzept-von-gestern}{»Konzept
von gestern«},} wirt­schaft­liche Neutralität ohnehin
\emph{\href{http://www.zeit.de/1990/44/ein-konzept-von-gestern}{»gänz­lich
über­holt«}.} Von 1997 bis 2001 war Gsteiger dann Chef­redakteur bei der
\emph{Welt­woche}. Unter seiner Leitung trat das Blatt »\emph{für den
Bei­tritt der Schweiz zur NATO«} ein, wie er in seinem
\href{https://web.archive.org/web/20040722094101/http://www.weltwoche.ch/artikel/?AssetID=400\&CategoryID=60}{Abschieds­artikel}
schrieb.

Damit kam Gsteiger 2002 zum Schweizer Radio. Be­schwer­den über eine
ein­sei­tige Be­richt­er­stattung wurden von der Ombuds­stelle mehr­fach
\href{https://www.srgd.ch/de/aktuelles/news/2016/09/28/sendung-info-3-auf-radio-srf-3-uber-waffenruhe-syrien-beanstandet/}{abge­lehnt}.
Und so
\href{http://www.swissinfo.ch/ger/kooperation_die-nato-umwirbt-die-schweiz/42225918}{be­tont}
Gsteiger auch heute noch die »\emph{vielen
Koope­ra­tions­möglich­keiten«} mit der NATO;
\href{http://www.srf.ch/news/international/dieser-eu-rueckzieher-ist-peinlich}{be­dauert},
dass die Russ­land-Sanktionen nicht ver­schärft werden; und
\href{http://www.srf.ch/news/international/assad-kommt-mit-giftgaseinsaetzen-vorlaeufig-davon}{weiß
genau}, wer in Syrien der Böse­wicht ist.

\emph{Update:} 2019 erhält das SRF eine neue
\href{https://www.srgd.ch/de/aktuelles/news/2017/12/05/luzia-tschirky-wird-neue-russland-korrespondentin/}{Russland-Korrespondentin}
-- die zuvor beim amerikanischen
\href{https://de.wikipedia.org/wiki/Radio_Free_Europe}{\emph{Radio Free
Europe}} arbeitete. (\emph{Foto oben:} Gsteiger 2014 auf einer
Jour­na­­listen-​Tour der US NATO-Mission.)

\begin{center}\rule{0.5\linewidth}{\linethickness}\end{center}

\href{https://swprs.org/2017/03/01/der-korrespondent/}{**1. March 2017}

\hypertarget{der-atlantic-council}{%
\section{\texorpdfstring{\href{https://swprs.org/2017/03/01/der-atlantic-council/}{Der
Atlantic Council}}{Der Atlantic Council}}\label{der-atlantic-council}}

\href{https://swprs.org/2017/03/01/der-atlantic-council/}{\includegraphics{https://swprs.files.wordpress.com/2018/11/atlantic-council.png?w=532}}

Der \emph{Atlantic Council} ist
\href{https://www.rubikon.news/artikel/facebook-als-waffe}{bekannt} für
sein En­ga­ge­ment gegen NATO-kritische »Des­in­for­ma­tion«, seine
Kooperation mit Facebook, die zur Lö­schung diverser Seiten führte,
sowie seine Ein­wir­kungen auf die eu­ro­pä­ische Außen­politik. Doch
wer ist der *Atlantic Council?\\
*

\href{https://swprs.org/atlantic-council/}{Weiterlesen →}

\begin{center}\rule{0.5\linewidth}{\linethickness}\end{center}

\href{https://swprs.org/2017/03/01/der-atlantic-council/}{**1. March
2017}

\hypertarget{russische-propaganda}{%
\section{\texorpdfstring{\href{https://swprs.org/2017/03/01/russische-propaganda/}{Russische
Propaganda}}{Russische Propaganda}}\label{russische-propaganda}}

\href{https://swprs.org/2017/03/01/russische-propaganda/}{\includegraphics{https://swprs.files.wordpress.com/2018/11/kreml.png?w=495}}

Wie funktioniert russische Propaganda, und was macht sie so
wirkungs­voll?

\href{https://swprs.org/russische-propaganda/}{Zum Beitrag →}

\begin{center}\rule{0.5\linewidth}{\linethickness}\end{center}

\href{https://swprs.org/2017/03/01/russische-propaganda/}{**1. March
2017}

\hypertarget{die-integrity-initiative}{%
\section{\texorpdfstring{\href{https://swprs.org/2017/03/01/die-integrity-initiative/}{Die
»Integrity
Initiative«}}{Die »Integrity Initiative«}}\label{die-integrity-initiative}}

\href{https://swprs.org/2017/03/01/die-integrity-initiative/}{\includegraphics{https://swprs.files.wordpress.com/2018/12/ii-logo-e1549798726940.png?w=350}}

Es ist die wohl größte Geheimdienstenthüllung seit Edward Snowden. Doch
von den etablierten Medien wurde sie nahezu vollständig ignoriert. Ein
Überblick.

\href{https://swprs.org/die-integrity-initiative/}{Zum Beitrag →}

\begin{center}\rule{0.5\linewidth}{\linethickness}\end{center}

\href{https://swprs.org/2017/03/01/die-integrity-initiative/}{**1. March
2017}

\hypertarget{die-republik-und-das-imperium}{%
\section{\texorpdfstring{\href{https://swprs.org/2017/03/01/die-republik-und-das-imperium/}{Die
Republik und
das~Imperium}}{Die Republik und das~Imperium}}\label{die-republik-und-das-imperium}}

\href{https://swprs.org/2017/03/01/die-republik-und-das-imperium/}{\includegraphics{https://swprs.files.wordpress.com/2018/11/republik.png?w=480}}

Das Online-Magazin \emph{Republik} startete 2018 mit dem
\href{https://www.persoenlich.com/medien/weltrekord-fur-journalistisches-crowdfunding-gebrochen}{erfolgreichsten}
Medien-Crowdfunding aller Zeiten. »Journalismus ist ein Kind der
Aufklä­rung. Seine Aufgabe ist die Kritik der Macht.«, proklamierte das
\href{https://www.republik.ch/manifest}{Manifest} verheißungsvoll. Doch
wie sieht es damit in der Realität aus?

\href{https://swprs.org/die-republik-und-das-imperium/}{Zum Beitrag →}

\begin{center}\rule{0.5\linewidth}{\linethickness}\end{center}

\href{https://swprs.org/2017/03/01/die-republik-und-das-imperium/}{**1.
March 2017}

\hypertarget{die-woz-und-die-weltpolitik}{%
\section{\texorpdfstring{\href{https://swprs.org/2017/03/01/die-woz-und-die-weltpolitik/}{Die
WOZ und
die~Weltpolitik}}{Die WOZ und die~Weltpolitik}}\label{die-woz-und-die-weltpolitik}}

\href{https://swprs.org/2017/03/01/die-woz-und-die-weltpolitik/}{\includegraphics{https://swprs.files.wordpress.com/2017/03/woz-logo-n.png?w=522}}

»Linksalternativ« und doch NATO-konform? Die WOZ zeigt wie's geht: In
Syrien etwa hätten ein paar
\href{https://www.woz.ch/1203/syrien/assad-geht-das-licht-aus}{Graffiti­sprayer}
eine
\href{https://www.woz.ch/1616/syriens-zukunft/assads-spiel-mit-dem-westen}{marxis­tisch}
ange­hauchte
\href{https://www.woz.ch/1511/kommentar-von-francois-moore/die-revolution-in-syrien-ist-am-ende}{»Revo­lution«}
junger
\href{https://www.woz.ch/1606/syrien/mithilfe-dieser-verdammten-russen-wird-dieser-bastard-noch-ueberleben}{Idealisten}
und \href{https://www.woz.ch/1235/syrien/kaempfen-und-beten}{frommer
Gottes­krieger} ausgelöst, während
\href{https://www.woz.ch/1324/syrien/ein-land-zersplittert-immer-mehr}{»das
Regime«} einen Krieg vom Zaun
\href{https://www.woz.ch/1321/syrien-und-der-westen/assad-kann-nur-gewinnen}{brach}
und mit »Fass­bomben« Kranken­häuser
\href{https://www.woz.ch/1416/syrien/fassbomben-gottes-wille-und-demokratie}{bombar­dierte},
sodass selbst eine NATO-Inter­vention
\href{https://www.woz.ch/1335/syrien/intervention-als-kleineres-uebel}{»das
kleinere Übel«} sei.

NATO-Kritiker Ganser hingegen
\href{https://www.woz.ch/1703/wahrheit-und-verschwoerung/das-ganser-phaenomen}{biete}
eine »Plattform für rechte Ver­schwö­rungs­theo­retiker«, und Wiki­leaks
-- an der Nieder­lage Clintons mitschuldig --
\href{https://www.woz.ch/1711/cia-dokumente/die-alternativen-fakten-von-wikileaks}{produziere}
»alter­native Fakten« für die »Neurechten«. Auch vor »alter­na­tiven
Medien« wird
\href{https://www.woz.ch/1743/qualitaet-der-medien/unterinformiert-und-ausgeliefert}{gewarnt}:
Diese »bedienen unverblümt Ver­schwörungs­theorien oder ver­breiten
rechte Propaganda«.

Wer die Global­isierung unvor­sichtig kriti­siert, sei womöglich ein
verkappter
\href{https://www.woz.ch/1708/wirtschaftlicher-protektionismus/die-voelkische-kritik-an-der-globalisierung}{»Rechts­nationa­list«},
und bei der Wachs­tums­politik des IWF dürfe man »nicht zu dogma­tisch
sein«, denn es
\href{https://www.woz.ch/1742/weltwirtschaft/die-hueterin-des-kapitalismus}{gelte},
»den Kapita­lismus vor der Rechten zu retten«. Selbst die
\href{https://www.woz.ch/1414/schweizerische-aussenpolitik/opportunistische-neutralitaet}{Schweizer
Neutra­lität} ist irgendwie »rechts«.

Medien­historisch erinnert die WOZ damit ein wenig an jene
\href{https://www.youtube.com/watch?v=3QAgCFjNXJE}{CIA-finanzierten
Publika­tionen}, die während des Kalten Krieges die potentiell kritische
Linke auf US-Kurs zu bringen versuchten. Und offenbar wird
geo­poli­tische Konfor­mität auch heute noch honoriert: Etwa mit
\href{https://swprs.files.wordpress.com/2017/10/amnesty-international-werbung.png}{ganz­seitigen
Farb­inseraten} von \emph{Amnesty Inter­national}, die in der WOZ den
Sturz von Washingtons Feinden
\href{https://consortiumnews.com/2012/06/18/amnestys-shilling-for-us-wars/}{bewerben}.

\begin{center}\rule{0.5\linewidth}{\linethickness}\end{center}

\href{https://swprs.org/2017/03/01/die-woz-und-die-weltpolitik/}{**1.
March 2017}

\hypertarget{die-logik-imperialer-kriege}{%
\section{\texorpdfstring{\href{https://swprs.org/2017/03/01/die-logik-imperialer-kriege/}{Die
Logik
imperialer~Kriege}}{Die Logik imperialer~Kriege}}\label{die-logik-imperialer-kriege}}

Wie lassen sich die amerikanischen Kriege der letzten Jahrzehnte
rational erklären?

Die folgende Analyse zeigt anhand des Modells der Professoren David
Sylvan und Stephen Majeski, dass diese Kriege auf einer eigenen, genuin
imperialen Handlungslogik basieren.

Eine besondere Rolle kommt dabei dem traditionellen Mediensystem zu.

\href{https://swprs.org/logik-imperialer-kriege/}{\includegraphics{https://swprs.files.wordpress.com/2018/05/logik-imperialer-kriege-spr-s.png?w=736}}

\href{https://swprs.org/logik-imperialer-kriege/}{Zur Analyse →}

\href{https://swprs.org/us-foreign-policy/}{Zur englischen Version →}

\begin{center}\rule{0.5\linewidth}{\linethickness}\end{center}

\href{https://swprs.org/2017/03/01/die-logik-imperialer-kriege/}{**1.
March 2017}

\hypertarget{swiss-policy-research}{%
\subsubsection{Swiss Policy Research}\label{swiss-policy-research}}

\begin{itemize}
\tightlist
\item
  \href{https://swprs.org/kontakt/}{Kontakt}
\item
  \href{https://swprs.org/uebersicht/}{Übersicht}
\item
  \href{https://swprs.org/donationen/}{Donationen}
\item
  \href{https://swprs.org/disclaimer/}{Disclaimer}
\end{itemize}

\hypertarget{english}{%
\subsubsection{English}\label{english}}

\begin{itemize}
\tightlist
\item
  \href{https://swprs.org/contact/}{About Us / Contact}
\item
  \href{https://swprs.org/media-navigator/}{The Media Navigator}
\item
  \href{https://swprs.org/the-american-empire-and-its-media/}{The CFR
  and the Media}
\item
  \href{https://swprs.org/donations/}{Donations}
\end{itemize}

\hypertarget{follow-by-email}{%
\subsubsection{Follow by email}\label{follow-by-email}}

Follow

\href{https://wordpress.com/?ref=footer_custom_com}{WordPress.com}.

\protect\hyperlink{}{Up ↑}

Post to

\protect\hyperlink{}{Cancel}

\includegraphics{https://pixel.wp.com/b.gif?v=noscript}
