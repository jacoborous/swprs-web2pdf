\protect\hyperlink{content}{Skip to content}

\href{https://swprs.org/}{}

\protect\hyperlink{search-container}{Search}

Search for:

\href{https://swprs.org/}{\includegraphics{https://swprs.files.wordpress.com/2020/05/swiss-policy-research-logo-300.png}}

\href{https://swprs.org/}{Swiss Policy Research}

Geopolitics and Media

Menu

\begin{itemize}
\tightlist
\item
  \href{https://swprs.org}{Start}
\item
  \href{https://swprs.org/srf-propaganda-analyse/}{Studien}

  \begin{itemize}
  \tightlist
  \item
    \href{https://swprs.org/srf-propaganda-analyse/}{SRF / ZDF}
  \item
    \href{https://swprs.org/die-nzz-studie/}{NZZ-Studie}
  \item
    \href{https://swprs.org/der-propaganda-multiplikator/}{Agenturen}
  \item
    \href{https://swprs.org/die-propaganda-matrix/}{Medienmatrix}
  \end{itemize}
\item
  \href{https://swprs.org/medien-navigator/}{Analysen}

  \begin{itemize}
  \tightlist
  \item
    \href{https://swprs.org/medien-navigator/}{Navigator}
  \item
    \href{https://swprs.org/der-propaganda-schluessel/}{Techniken}
  \item
    \href{https://swprs.org/propaganda-in-der-wikipedia/}{Wikipedia}
  \item
    \href{https://swprs.org/logik-imperialer-kriege/}{Kriege}
  \end{itemize}
\item
  \href{https://swprs.org/netzwerk-medien-schweiz/}{Netzwerke}

  \begin{itemize}
  \tightlist
  \item
    \href{https://swprs.org/netzwerk-medien-schweiz/}{Schweiz}
  \item
    \href{https://swprs.org/netzwerk-medien-deutschland/}{Deutschland}
  \item
    \href{https://swprs.org/medien-in-oesterreich/}{Österreich}
  \item
    \href{https://swprs.org/das-american-empire-und-seine-medien/}{USA}
  \end{itemize}
\item
  \href{https://swprs.org/bericht-eines-journalisten/}{Fokus I}

  \begin{itemize}
  \tightlist
  \item
    \href{https://swprs.org/bericht-eines-journalisten/}{Journalistenbericht}
  \item
    \href{https://swprs.org/russische-propaganda/}{Russische Propaganda}
  \item
    \href{https://swprs.org/die-israel-lobby-fakten-und-mythen/}{Die
    »Israel-Lobby«}
  \item
    \href{https://swprs.org/geopolitik-und-paedokriminalitaet/}{Pädokriminalität}
  \end{itemize}
\item
  \href{https://swprs.org/migration-und-medien/}{Fokus II}

  \begin{itemize}
  \tightlist
  \item
    \href{https://swprs.org/covid-19-hinweis-ii/}{Coronavirus}
  \item
    \href{https://swprs.org/die-integrity-initiative/}{Integrity
    Initiative}
  \item
    \href{https://swprs.org/migration-und-medien/}{Migration \& Medien}
  \item
    \href{https://swprs.org/der-fall-magnitsky/}{Magnitsky Act}
  \end{itemize}
\item
  \href{https://swprs.org/kontakt/}{Projekt}

  \begin{itemize}
  \tightlist
  \item
    \href{https://swprs.org/kontakt/}{Kontakt}
  \item
    \href{https://swprs.org/uebersicht/}{Seitenübersicht}
  \item
    \href{https://swprs.org/medienspiegel/}{Medienspiegel}
  \item
    \href{https://swprs.org/donationen/}{Donationen}
  \end{itemize}
\item
  \href{https://swprs.org/contact/}{English}
\end{itemize}

\protect\hyperlink{}{Open Search}

\hypertarget{tag-propaganda}{%
\section{Tag: Propaganda}\label{tag-propaganda}}

\hypertarget{srf-die-propaganda-analyse}{%
\section{\texorpdfstring{\href{https://swprs.org/2017/03/01/srf-propaganda-analyse/}{SRF:
Die
Propaganda-Analyse}}{SRF: Die Propaganda-Analyse}}\label{srf-die-propaganda-analyse}}

\href{https://swprs.org/2017/03/01/srf-propaganda-analyse/}{\includegraphics{https://swprs.files.wordpress.com/2016/10/srf-analyse-s.png?w=500}}

Das Schweizer Radio und Fern­se­hen (SRF) leistet mit seinen
Nach­rich­ten- und In­for­ma­tions­sen­dungen einen wich­tigen Bei­trag
zur öffent­lichen Meinungs­bildung in der Schweiz. Doch wie objektiv und
kritisch be­rich­tet das SRF über geo­po­li­tische The­men?

Um dies zu über­prü­fen, wurde erst­mals eine sys­te­ma­tische Ana­lyse
der SRF-​Be­richt­er­stat­tung zu einem geo­po­li­tisch relevanten
Ereig­nis durch­ge­führt.

Die Resul­tate sind alar­mie­rend: In allen unter­such­ten Bei­trä­gen
des SRF wurden Pro­pa­ganda- und Mani­pu­la­tions­tech­niken auf
re­dak­tio­nel­ler, sprach­licher und audio­vi­su­el­ler Ebene
fest­ge­stellt.

\href{https://swprs.org/srf-propaganda-analyse/}{Zur SRF
Propaganda-Analyse →}

\begin{center}\rule{0.5\linewidth}{\linethickness}\end{center}

\href{https://swprs.org/2017/03/01/srf-propaganda-analyse/}{**1. March
2017}

\hypertarget{das-gewuxfcnschte-narrativ}{%
\section{\texorpdfstring{\href{https://swprs.org/2017/03/01/das-gewuenschte-narrativ/}{Das
gewünschte
Narrativ}}{Das gewünschte Narrativ}}\label{das-gewuxfcnschte-narrativ}}

\href{https://swprs.org/2017/03/01/das-gewuenschte-narrativ/}{\includegraphics{https://swprs.files.wordpress.com/2016/02/medien-narrativ1.png?w=400}}

Bei geopolitischen Konflikten bestehen oftmals vordefinierte mediale
Narrative. Was geschieht, wenn ein Schweizer Jour­na­list davon abweicht
und über die »falschen« Themen be­richtet?

Heute kaum noch vorstellbar, doch mitten im
\href{https://de.wikipedia.org/wiki/Bosnienkrieg}{Bosnien­krieg}
(1992-95) veröffentlichte der damalige Aus­lands­chef der
\emph{Welt­woche} einen Artikel zu Kriegs­lügen in west­lichen Medien.

Daraufhin geschah Folgendes:

\href{https://swprs.org/das-gewuenschte-narrativ\#weiterlesen}{Weiterlesen
→}

\begin{center}\rule{0.5\linewidth}{\linethickness}\end{center}

\href{https://swprs.org/2017/03/01/das-gewuenschte-narrativ/}{**1. March
2017}

\hypertarget{warum-der-tagi-nichts-verpasst}{%
\section{\texorpdfstring{\href{https://swprs.org/2017/03/01/warum-der-tagesanzeiger-nichts-verpasst/}{Warum
der Tagi
nichts~verpasst}}{Warum der Tagi nichts~verpasst}}\label{warum-der-tagi-nichts-verpasst}}

\href{https://swprs.org/2017/03/01/warum-der-tagesanzeiger-nichts-verpasst/}{\includegraphics{https://swprs.files.wordpress.com/2016/06/lena-logo2.png?w=440}}

Ob Ukraine, Syrien oder Chi­na: Der \emph{Zürcher Tages-Anzeiger}
schreibt viele seiner Aus­lands­berichte nicht mehr selbst, sondern
bezieht sie im Rah­men einer
\href{https://www.tagesanzeiger.ch/schweiz/standard/In-eigener-Sache/story/24648194}{»umfassenden
Ko­ope­ra­tion«} von der \emph{Süd­deut­schen Zeitung.}

Deren Außen­politik­chef
\href{https://swprs.org/netzwerk-medien-deutschland/}{zählt} indes zu
den bekanntesten Trans­at­lan­tikern Deutsch­lands -- und ent­spre­chend
le­sen sich die Arti­kel im \emph{Tagi.} Aus dem ara­bi­schen Raum
\href{https://web.archive.org/web/20170606085220/http://www.icfj.org/sites/default/files/Kr\%C3\%BCger.pdf}{berichtet}
z.B. ein Absol­vent des ameri­ka­ni­schen
\emph{Arthur-F.-Burns-Fellowship}, aus Mos­kau ein
\href{https://spiegelkabinett-blog.blogspot.com/2016/09/julian-hans-von-der-suddeutschen.html}{Ab­gänger}
der \emph{Henri-Nannen-Schule}.

Über den Onlinedienst
\emph{\href{https://de.wikipedia.org/wiki/Newsnet}{Newsnet}} werden
Aus­lands­be­rich­te des \emph{Tagi} zudem an andere Schwei­zer
Zei­tungen wei­ter­ge­reicht. Auf diese Weise
\href{http://www.tagesanzeiger.ch/ausland/europa/Den-Ausloeser-zum-Krieg-habe-ich-gedrueckt/story/16330278}{er­scheinen}
Beiträge der \emph{Süd­deutschen Zeitung} via \emph{Tages­-Anzeiger} und
\emph{Newsnet} zu­sätz­lich im \emph{Berner Bund} und der \emph{Basler
Zeitung.}

Seit 2015 ist der \emph{Tages­-Anzeiger} über­dies Teil der
\emph{\href{https://de.wikipedia.org/wiki/Leading_European_Newspaper_Alliance}{Leading
European News­paper Alliance} (LENA).} Zweck des Ver­bunds ist die
``Ent­wick­lung und der Aus­tausch re­dak­tio­neller In­hal­te'' mit
anderen LENA- Mit­glie­dern wie \emph{Le Fi­g­aro}, \emph{Die Welt},
\emph{El País} oder \emph{La Re­pub­blica}.

Alle diese Zei­tungen sind in das
\href{https://swprs.files.wordpress.com/2016/07/bilderberg_teilnehmer_1954-2014.pdf}{Bilder­berg-Netz­werk}
ein­ge­bun­den -- kann es da noch über­raschen, dass auch
LENA- Repor­ta­gen zumeist auf
\href{http://www.tagesanzeiger.ch/ausland/europa/Wer-sagt-was-er-denkt-nobrriskiert-allesnobr/story/17225010}{trans­at­lan­tischer
Linie} sind?

\begin{center}\rule{0.5\linewidth}{\linethickness}\end{center}

\href{https://swprs.org/2017/03/01/warum-der-tagesanzeiger-nichts-verpasst/}{**1.
March 2017}

\hypertarget{die-nzz-studie}{%
\section{\texorpdfstring{\href{https://swprs.org/2017/03/01/die-nzz-studie/}{Die
NZZ-Studie}}{Die NZZ-Studie}}\label{die-nzz-studie}}

\href{https://swprs.org/2017/03/01/die-nzz-studie/}{\includegraphics{https://swprs.files.wordpress.com/2017/03/nzz-propaganda-gesamt-small.png?w=500}}

Die \emph{Neue Zürcher Zeitung} ist das Flagg­schiff unter den Schweizer
Tages­zei­tungen. Doch wie objektiv und kritisch berichtet die
\emph{NZZ} über geo­politische Konf‌likte?

Um dies zu be­ant­worten, wurde die Bericht­erstattung der \emph{NZZ}
zur Ukraine-Krise und zum Syrien­krieg während je eines Monats
unter­sucht. Die Ergebnisse sind eindeutig.

\href{https://swprs.org/die-nzz-studie/}{Zur NZZ-Studie →}

\begin{center}\rule{0.5\linewidth}{\linethickness}\end{center}

\href{https://swprs.org/2017/03/01/die-nzz-studie/}{**1. March 2017}

\hypertarget{der-propaganda-multiplikator}{%
\section{\texorpdfstring{\href{https://swprs.org/2017/03/01/propaganda-multiplikator/}{Der
Propaganda-Multiplikator}}{Der Propaganda-Multiplikator}}\label{der-propaganda-multiplikator}}

\href{https://swprs.org/2017/03/01/propaganda-multiplikator/}{\includegraphics{https://swprs.files.wordpress.com/2016/12/propaganda-multiplikator-300.png?w=346}}

Es ist einer der wichtigsten Aspekte unseres Medien­systems -- und
dennoch in der Öf‌fent­lich­keit nahezu unbekannt: Der größte Teil der
inter­na­tio­nalen Nach­rich­ten in all unseren Medien stammt von nur
drei glo­balen Nach­rich­ten­agen­turen aus~New York, London und Paris.

Die Schlüssel­rolle dieser Agen­turen be­wirkt, dass west­liche Medien
zu­meist über die glei­chen The­men be­richten und dabei sogar oftmals
dieselben For­mu­lie­rungen ver­wenden.

Zu­dem nutzen Re­gie­rungen, Mi­li­tärs und Ge­heim­dienste die
glo­balen Agen­turen als Mul­ti­pli­kator für die welt­weite
Ver­brei­tung ihrer Bot­schaf‌ten. Die trans­at­lan­tische Ver­netzung
der eta­blier­ten Medien ge­währ­leis­tet da­bei, dass die ge­wün­schte
Sicht­weise kaum hin­ter­fragt wird.

Eine Unter­suchung der Syrien-Bericht­er­stat­tung von je drei
füh­ren­den Tages­zei­tungen aus Deutsch­land, Öster­reich und der
Schweiz illus­triert diese Ef‌fekte deutlich.

\href{https://swprs.org/der-propaganda-multiplikator/}{Zur
Multiplikator-Studie →}

\begin{center}\rule{0.5\linewidth}{\linethickness}\end{center}

\href{https://swprs.org/2017/03/01/propaganda-multiplikator/}{**1. March
2017}

\hypertarget{das-gewuxfcnschte-narrativ-ii}{%
\section{\texorpdfstring{\href{https://swprs.org/2017/03/01/das-gewuenschte-narrativ-ii/}{Das
gewünschte
Narrativ~II}}{Das gewünschte Narrativ~II}}\label{das-gewuxfcnschte-narrativ-ii}}

\href{https://swprs.org/2017/03/01/das-gewuenschte-narrativ-ii/}{\includegraphics{https://swprs.files.wordpress.com/2017/03/zeitungen-schweiz.png?w=450}}

Im Dezember 2015 publi­zierte das News­portal \emph{Watson} (AZ Medien)
einen \href{https://www.watson.ch/!148360008}{Artikel} des lang­jährigen
Tages­schau-Kor­res­pon­denten Hel­mut Sche­­ben zum Syrien­krieg.
Scheben stellte den Krieg in einen geo­po­li­tischen Kontext und
kri­ti­sierte die westliche Be­richt­er­stattung als einseitig und
ma­ni­pu­la­tiv.

Der Artikel un­ter­schied sich deutlich von anderen Aus­lands­bei­trägen
auf \emph{Watson}, die meist vom deutsch-transatlantischen
\href{https://www.watson.ch/Corporate/articles/502582965-Spiegel-Online-und-watson-machen-gemeinsame-Sache}{Content
Partner} \emph{Spiegel Online} geliefert werden.

Keine zwei Tage später veröffentlichte \emph{Watson} jedoch einen
aufgebrachten \href{https://www.watson.ch/!491379853}{Rückruf}, in dem
sich das Portal vom Artikel distanzierte und Helmut Scheben wüst
beschimpfte: Man sei auf einen ``Putin-Troll'' herein­ge­fallen, der
wo­möglich in der ``russischen Propaganda-Maschinerie'' mit­wirke. Auch
Leser, die sich positiv zum ur­sprüng­lichen Artikel geäußert hatten,
wurden als »Trolle« verun­glimpft.

Wer oder was hat wohl hinter den Kulissen zu dieser selt­samen Reak­tion
geführt? Jeden­falls wurde den hiesigen Journa­listen damit einmal mehr
in Er­in­nerung gerufen: Wer sich in der Schweiz nicht an das
ge­wünschte Nar­ra­tiv hält, ris­kiert Ruf und Karriere.

\begin{center}\rule{0.5\linewidth}{\linethickness}\end{center}

\href{https://swprs.org/2017/03/01/das-gewuenschte-narrativ-ii/}{**1.
March 2017}

\hypertarget{propaganda-im-staatsauftrag}{%
\section{\texorpdfstring{\href{https://swprs.org/2017/03/01/propaganda-im-staatsauftrag/}{Propaganda
im
Staatsauftrag?}}{Propaganda im Staatsauftrag?}}\label{propaganda-im-staatsauftrag}}

\href{https://swprs.org/2017/03/01/propaganda-im-staatsauftrag/}{\includegraphics{https://swprs.files.wordpress.com/2016/02/srf-syrien11.png?w=600}}

Von den öf‌fentlichen Rund­funk­an­stalten er­war­tet das Pu­bli­kum
eine aus­ge­wogene Bericht­er­stattung. Doch of‌t ist ge­rade dort der
politische Druck be­sonders hoch, sich an das
\href{https://swprs.org/das-gewuenschte-narrativ/}{trans­at­lan­tische
Narra­tiv} zu halten.

So haben Mitarbeiter der \emph{ARD} gemäß internen Memos Weisung, bei
geo­po­li­tischen Kon­f‌lik­ten
\emph{\href{https://www.heise.de/tp/features/Ukraine-Konflikt-ARD-Programmbeirat-bestaetigt-Publikumskritik-3367400.html}{»west­liche
Posi­tionen zu ver­tei­di­gen«}}, ver­trau­liche
\href{https://www.heise.de/tp/features/Die-vertraulichen-Sprachregelungen-der-ARD-3758887.html}{Sprach­­re­­ge­lungen}
zu be­fol­gen und aus­­schließ­­lich
\href{https://www.oxmoxhh.de/magazin/story-interview/oxmox-exklusiv-interview-mit-volker-braeutigam-friedhelm-klinkhammer/}{konforme
Quellen} zu ver­wen­den.

Beim \emph{ZDF} machte der ehe­ma­lige Chef­re­dakteur publik, dass
Bei­träge zu US-Kriegen poli­tisch
\href{https://www.youtube.com/watch?v=i2423aDq_hE}{be­ein­f‌‌lusst}
werden. Nahost-Kor­res­pon­dent Ulrich Tilgner be­klagte
re­dak­tio­nelle Ein­grif‌fe aufgrund von
\href{http://www.berliner-zeitung.de/korrespondent-ulrich-tilgner-sucht-mehr-distanz-zum-zdf--ich-fuehle-mich-eingeschraenkt--15870684}{»Bünd­nis­rück­sich­ten«},
und der vormalige Leiter des *ZDF-*Studios Bonn be­stä­tig­te
\href{https://propagandaschau.wordpress.com/2016/01/30/wolfgang-herles-es-gibt-in-den-oeffentlich-rechtlichen-anweisungen-von-oben/}{»An­wei­sungen
von oben«} und eine
\href{http://www.rolandtichy.de/daili-es-sentials/meinungsfreiheit-anordnung-zur-anpassung/}{»frei­willige
Gleich­schal­tung«} der Jour­na­lis­ten.

Auch das \emph{SRF} verwendet diverse
\href{https://swprs.org/srf-propaganda-analyse/}{Mani­pu­lations­tech­niken}
zugunsten der Konflikt­partei USA \& NATO und
\href{http://www.srf.ch/sendungen/srfglobal/propagandagruesse-aus-moskau-2}{thematisiert}
Propaganda stets nur auf der Gegenseite. Selbst vor dem Einsatz
sub­tiler
\href{http://www.srf.ch/play/tv/10vor10/video/warum-assad-bleibt?id=a6d267c9-52b3-470b-868e-95bb919a0b96}{Grusel­musik}
in den Nach­rich­ten schreckt das \emph{SRF} nicht zurück, um Gegner der
US-Allianz zu dämo­ni­sieren.

Programmbe­schwer­den sind indes chan­cen­los, denn: Beiträge zu
inter­na­tio­nalen Kon­flik­ten müssten
\emph{\href{https://swprs.org/srf-ombudsstelle-im-faktencheck/}{``weder
neutral noch ausgewogen''}} sein, und \emph{``die­je­ni­gen, die dem SRF
vor­wer­fen, ein­sei­tig der US- und Nato-Pro­pa­gan­da zu er­lie­gen,
be­trei­ben ihrer­seits das Ge­schäf‌t der russischen Pro­pa­ganda''} --
so die erstaun­liche
\href{https://swprs.org/srf-ombudsstelle-im-faktencheck/}{Ar­gu­men­ta­tion}
der Om­buds­stelle.*\\
*

\begin{center}\rule{0.5\linewidth}{\linethickness}\end{center}

\href{https://swprs.org/2017/03/01/propaganda-im-staatsauftrag/}{**1.
March 2017}

\hypertarget{der-korrespondent}{%
\section{\texorpdfstring{\href{https://swprs.org/2017/03/01/der-korrespondent/}{Der
Korrespondent}}{Der Korrespondent}}\label{der-korrespondent}}

\href{https://swprs.org/2017/03/01/der-korrespondent/}{\includegraphics{https://swprs.files.wordpress.com/2017/03/srf-gsteiger-nato.jpg?w=600}}

Wie wird man Kor­res­pon­dent beim \emph{Schwei­zer Radio und
Fern­sehen}? Fredy Gsteiger muss es wissen: Er ist
\href{http://www.persoenlich.com/medien/fredy-gsteiger-neu-in-der-radio-chefredaktion-232921}{stv.
Chef­redakteur}, Auslands­chef und
\href{http://www.srf.ch/radio-srf-1/radio-srf-1/fredy-gsteiger-unser-mann-in-der-uno}{diplo­ma­tischer
Korres­pon­dent} des \emph{Schwei­zer Radios SRF}. In dieser Funktion
be­richtet er etwa über die UNO, NATO und EU -- und damit z.B. auch über
\href{http://www.srf.ch/news/international/dieser-eu-rueckzieher-ist-peinlich}{Russ­land-Sanktionen}
und die Genfer
\href{http://www.srf.ch/news/international/assad-kommt-mit-giftgaseinsaetzen-vorlaeufig-davon}{Syrien-Ver­hand­lungen}.

Gsteiger begann seine journa­lis­tische Lauf­bahn Ende der 80er Jahre
als Nahost-Redakteur bei der
\href{https://swprs.org/netzwerk-medien-deutschland/}{deutsch-trans­atlan­tischen}
Wochen­zeitung \emph{Die Zeit}. Die Schwei­zer Neutra­lität war für ihn
schon vor dem Ersten Irak­krieg 1991 ein
\emph{\href{http://www.zeit.de/1990/44/ein-konzept-von-gestern}{»Konzept
von gestern«},} wirt­schaft­liche Neutralität ohnehin
\emph{\href{http://www.zeit.de/1990/44/ein-konzept-von-gestern}{»gänz­lich
über­holt«}.} Von 1997 bis 2001 war Gsteiger dann Chef­redakteur bei der
\emph{Welt­woche}. Unter seiner Leitung trat das Blatt »\emph{für den
Bei­tritt der Schweiz zur NATO«} ein, wie er in seinem
\href{https://web.archive.org/web/20040722094101/http://www.weltwoche.ch/artikel/?AssetID=400\&CategoryID=60}{Abschieds­artikel}
schrieb.

Damit kam Gsteiger 2002 zum Schweizer Radio. Be­schwer­den über eine
ein­sei­tige Be­richt­er­stattung wurden von der Ombuds­stelle mehr­fach
\href{https://www.srgd.ch/de/aktuelles/news/2016/09/28/sendung-info-3-auf-radio-srf-3-uber-waffenruhe-syrien-beanstandet/}{abge­lehnt}.
Und so
\href{http://www.swissinfo.ch/ger/kooperation_die-nato-umwirbt-die-schweiz/42225918}{be­tont}
Gsteiger auch heute noch die »\emph{vielen
Koope­ra­tions­möglich­keiten«} mit der NATO;
\href{http://www.srf.ch/news/international/dieser-eu-rueckzieher-ist-peinlich}{be­dauert},
dass die Russ­land-Sanktionen nicht ver­schärft werden; und
\href{http://www.srf.ch/news/international/assad-kommt-mit-giftgaseinsaetzen-vorlaeufig-davon}{weiß
genau}, wer in Syrien der Böse­wicht ist.

\emph{Update:} 2019 erhält das SRF eine neue
\href{https://www.srgd.ch/de/aktuelles/news/2017/12/05/luzia-tschirky-wird-neue-russland-korrespondentin/}{Russland-Korrespondentin}
-- die zuvor beim amerikanischen
\href{https://de.wikipedia.org/wiki/Radio_Free_Europe}{\emph{Radio Free
Europe}} arbeitete. (\emph{Foto oben:} Gsteiger 2014 auf einer
Jour­na­­listen-​Tour der US NATO-Mission.)

\begin{center}\rule{0.5\linewidth}{\linethickness}\end{center}

\href{https://swprs.org/2017/03/01/der-korrespondent/}{**1. March 2017}

\hypertarget{der-kriegsreporter}{%
\section{\texorpdfstring{\href{https://swprs.org/2017/03/01/der-kriegsreporter/}{Der
Kriegsreporter}}{Der Kriegsreporter}}\label{der-kriegsreporter}}

\href{https://swprs.org/2017/03/01/der-kriegsreporter/}{\includegraphics{https://swprs.files.wordpress.com/2016/11/pelda-syrien.jpg?w=600}}

Wie wird man in den Schweizer Medien zum »Nahost-Experten«? Kurt Pelda
muss es wissen: Von der \emph{Welt­woche} bis zum \emph{Schwei­zer
Fern­se­hen} ist er der Mann, der die Ereig­nisse in Sy­ri­en und Irak
für das Publi­kum
\href{http://www.srf.ch/news/international/assad-ist-nur-noch-an-der-macht-weil-er-so-brutal-ist}{»ein­ord­nen«}
darf.

Pelda
\href{https://www.youtube.com/watch?v=dtV25eIECKY}{be­glei­tete}schon in
den 80er Jahren als junger Journa­list die Mudschahedin im von den USA
\href{https://www.voltairenet.org/article165889.html}{lancier­ten} Krieg
gegen die afgha­nische Regie­rung, die mit Moskau verbün­det war. Nach
Sta­tionen bei der \emph{Financial Times} und der \emph{NZZ} bereist er
heute als freier Journa­list erneut Kriegs­ge­biete -- wie damals meist
nur
\href{https://tageswoche.ch/politik/ein-basler-im-syrischen-kampfgebiet/}{auf
Seiten} der US-unter­stützten Milizen.

Ist diese Ein­seitig­keit ein Pro­blem? Nicht für Pelda, denn er sei
schließ­lich -- so erklärte er in einem
\href{https://www.tageswoche.ch/de/2014_36/international/667493/}{Interview}
-- ein »Mei­nungs­jour­na­list« und »kein objek­ti­ver Be­obach­ter«,
wes­wegen Neutra­li­tät für ihn »keine Option« ist; viel­mehr gehe es
ihm um »gute Ge­schich­ten«, für die die Medien zu zahlen be­reit sind.
Wer in diesen Ge­schich­ten die Guten sind -- und wer
\href{http://www.srf.ch/news/international/assad-ist-nur-noch-an-der-macht-weil-er-so-brutal-ist}{die
Bösen} -- dürf‌te dabei niemanden über­raschen.

Mit diesem Ansatz wurde Pelda 2014 zum
\href{http://www.srf.ch/news/panorama/kurt-pelda-ist-journalist-des-jahres}{»Jour­na­list
des Jahres«} gekürt. Andere Nahost-Ken­ner, denen Objek­ti­vi­tät und
Neutra­lität wich­ti­ger sind als eine »gute Ge­schichte«, kommen in
Schwei­zer Medien indes
\href{https://swprs.org/das-gewuenschte-narrativ-ii/}{kaum noch} zu
Wort. Statt »ein­ge­ordnet« wurde hier -- aus­sor­tiert.

\emph{Foto oben:} Pelda in Syrien.
\emph{(\href{https://tageswoche.ch/politik/ein-basler-im-syrischen-kampfgebiet/}{TW})}

\begin{center}\rule{0.5\linewidth}{\linethickness}\end{center}

\href{https://swprs.org/2017/03/01/der-kriegsreporter/}{**1. March 2017}

\hypertarget{die-srf-rundschau-hinterfragt}{%
\section{\texorpdfstring{\href{https://swprs.org/2017/03/01/srf-rundschau/}{Die
SRF-Rundschau
hinterfragt}}{Die SRF-Rundschau hinterfragt}}\label{die-srf-rundschau-hinterfragt}}

\href{https://swprs.org/2017/03/01/srf-rundschau/}{\includegraphics{https://swprs.files.wordpress.com/2018/07/rundschau.png?w=500}}

Die
\emph{\href{https://de.wikipedia.org/wiki/Rundschau_(SRF)}{Rundschau}}
ist das bekannteste Polit­ma­ga­zin des \emph{Schweizer Fernsehens.} Sie
\href{https://www.srf.ch/sendungen/rundschau/50-jahre-rundschau-die-jubilaeumssendung}{möchte}
»die Mächtigen hinterfragen« -- doch geht's um Geo­po­litik, so steht
sie meist selbst auf deren Seite.

Während die
\href{https://www.srf.ch/sendungen/rundschau/subventionierte-piloten-vaeter-am-limit-bombenhoelle-aleppo}{»Bombenhölle«}
Aleppo »fällt«, wird Mossul »befreit« -- von einem Familien­vater, der
gerne »US-Popmusik«
\href{https://www.srf.ch/sendungen/rundschau/buben-beschneidung-michel-bollag-pkb-west-mossul}{hört}.

Vom »Giftgasangriff« bei Ghouta
\href{https://www.srf.ch/sendungen/rundschau/kriminaltouristen-verhuetungsmittel-j-bitzer-giftgaseinsatz}{berichtet}
der »Augen­zeuge« einer »Hilfs­organisation« -- wer diese
\href{http://www.uossm.org/who_we_are}{finanziert}, verrät die
\emph{Rundschau} nicht.

Beim
\href{https://www.srf.ch/sendungen/rundschau/gehorsam-und-ehelos-klamauk-statt-kompromiss-vergessener-krieg}{»Vergessenen
Krieg«} im Jemen werden prompt die saudischen Luftangriffe »vergessen«
-- und deren westliche
\href{https://www.strategic-culture.org/news/2018/06/18/western-media-whitewash-yemen-genocide.html}{Unterstützung}
ebenso.

Putin indes hege
\href{https://www.srf.ch/sendungen/rundschau/gianni-infantino-fatma-samoura-iv-kosovaren-zittern-vor-russen}{»Expansionsgelüste«}
und füh­re einen
\href{https://www.srf.ch/sendungen/rundschau/putins-informationskrieg-milliarden-jongleur-bastos-camorra}{»Informationskrieg«},
seine Angriffe auf den Westen seien bereits »mehrfach be­legt« -- doch
statt Fakten
\href{https://www.srf.ch/sendungen/rundschau/putins-informationskrieg-milliarden-jongleur-bastos-camorra}{folgen}
finstere Sound­effekte.

Schon der Gründer der \emph{Rundschau} und spätere Leiter der
\emph{Tagesschau} nahm an der Konferenz der trans­atlantischen Elite
\href{https://wikileaks.org/plusd/cables/1978ZURICH00660_d.html}{teil}
-- ob man dort lernt, die Mächtigen zu hinterfragen?

\begin{center}\rule{0.5\linewidth}{\linethickness}\end{center}

\href{https://swprs.org/2017/03/01/srf-rundschau/}{**1. March 2017}

\hypertarget{der-schweizer-presserat}{%
\section{\texorpdfstring{\href{https://swprs.org/2017/03/01/der-schweizer-presserat/}{Der
Schweizer
Presserat}}{Der Schweizer Presserat}}\label{der-schweizer-presserat}}

\href{https://swprs.org/2017/03/01/der-schweizer-presserat/}{\includegraphics{https://swprs.files.wordpress.com/2016/07/presserat-logo.png?w=200}}

Der \href{https://presserat.ch/}{Schweizer Presse­rat} nimmt
Be­schwer­den zu Me­dien­be­rich­ten ent­ge­gen und prüft, ob die
Beiträge seinen
\href{https://presserat.ch/journalistenkodex/richtlinien/}{Richt­linien}
ent­spre­chen.

Aller­dings
\href{https://presserat.ch/der-presserat/presseratsmitglieder/}{besteht}
das Gre­mium selbst aus 15 Jour­na­listen und nur sechs
\emph{Pub­li­kums­ver­tre­tern} -- und auch diese werden von einem
\href{https://presserat.ch/der-presserat/stiftungsratsmitglieder/}{Stif‌­tungs­rat}
er­nannt, der gänz­lich von Medien­orga­ni­sa­tionen
\href{https://presserat.ch/der-presserat/geschaeftsreglement/}{kon­trol­liert}
wird.

Das Resultat ist naheliegend. Im Som­mer 2014 wurde etwa eine
Be­schwerde gegen die
\href{https://swprs.org/die-nzz-studie/}{no­to­risch ein­sei­tige}
Ukraine-Bericht­er­stattung der \emph{NZZ} ein­ge­legt. Ganze zwei Jahre
später kam der Presse­rat zu seinem
\href{https://presserat.ch/complaints/wahrheitspflicht-kommentarfreiheit-unterschlagen-wichtiger-informationen-entstellen-von-tatsachen/}{Verdikt}:
Die Rich­tig­keit der *NZZ- *Dar­stel­lung stehe \emph{»außer Frage«},
denn auf \emph{»amt­liche Ver­laut­ba­rungen und Agen­tur­mel­dungen«}
sei \emph{»Verlass«,} während russische Quel­len weder glaub­haf‌t noch
erforderlich wären; Kom­men­tare müss­ten nicht auf Fak­ten ba­sie­ren,
Ge­gen­mei­nungen ein­zu­holen sei \emph{»un­üb­lich«,} und an den
Aus­füh­rungen der \emph{NZZ} zu \emph{»Kreml- Trollen«} sei
\emph{»nicht zu zwei­feln«}. Be­schwerde ab­ge­lehnt.

Pikant: Einige der be­ur­teil­ten Ar­tikel stam­mten von einem
\href{http://www.nzz.ch/international/europa/beschwerde-beim-presserat-kritik-an-nzz-abgewiesen-ld.104814}{*NZZ-*Redak­teur},
der selbst im Stif‌­tungs­rat des Gremiums sitzt -- und inzwischen wurde
der damalige *NZZ-*Chef gar zu dessen
\href{http://www.nzz.ch/schweiz/medien-selbstregulierung-markus-spillmann-wird-praesident-des-presserats-ld.135619}{Prä­si­denten}
ernannt. Beim Presse­rat nennt man dies
\href{https://de.wikipedia.org/wiki/Schweizer_Presserat}{»Selbst­re­gu­lierung«\ldots{}}

\begin{center}\rule{0.5\linewidth}{\linethickness}\end{center}

\href{https://swprs.org/2017/03/01/der-schweizer-presserat/}{**1. March
2017}

\hypertarget{die-grenzen-der-pressefreiheit}{%
\section{\texorpdfstring{\href{https://swprs.org/2017/03/01/die-grenzen-der-pressefreiheit/}{Die
Grenzen
der~Pressefreiheit}}{Die Grenzen der~Pressefreiheit}}\label{die-grenzen-der-pressefreiheit}}

\href{https://swprs.org/2017/03/01/die-grenzen-der-pressefreiheit/}{\includegraphics{https://swprs.files.wordpress.com/2017/12/reporter_ohne_grenzen_logo_s.png?w=530}}

Der \href{http://pressclub.ch/?lang=en}{Schweizer Presseclub} in Genf
genießt einen ausgezeichneten Ruf: Seit seiner Gründung hat er über
zweitausend Anlässe mit illustren Rednern von Fidel Castro bis Henry
Kissinger und von Jean Ziegler bis Klaus Schwab organisiert.

Doch für Ende November 2017 war ein
\href{http://pressclub.ch/they-dont-care-about-us-white-helmets-true-agenda/?lang=en}{Vortrag}
angekündigt, der sich kritisch mit den in west­li­chen Medien populären
\href{https://www.hintergrund.de/globales/kriege/weisse-helme-ohne-weisse-westen/}{Syrischen
Weiß­helmen} befassen wollte. Daraufhin geschah Folgendes:

\href{https://swprs.org/die-grenzen-der-pressefreiheit/}{Weiterlesen →}

\begin{center}\rule{0.5\linewidth}{\linethickness}\end{center}

\href{https://swprs.org/2017/03/01/die-grenzen-der-pressefreiheit/}{**1.
March 2017}

\hypertarget{propaganda-in-der-wikipedia}{%
\section{\texorpdfstring{\href{https://swprs.org/2017/03/01/propaganda-in-der-wikipedia/}{Propaganda
in
der~Wikipedia}}{Propaganda in der~Wikipedia}}\label{propaganda-in-der-wikipedia}}

Die Online-Enzyklopädie Wikipedia ist ein integraler Bestandteil des
transatlantischen Medien- und Informationssystems. In der folgenden
Analyse werden zentrale Aspekte ihrer Organisationsstruktur,
Funktionsweise und Manipulation dargestellt.

\href{https://swprs.org/propaganda-in-der-wikipedia/}{\includegraphics{https://swprs.files.wordpress.com/2019/03/wikipedia-2019-s.png?w=736}}

\href{https://swprs.org/propaganda-in-der-wikipedia/}{Zur Analyse →}

\begin{center}\rule{0.5\linewidth}{\linethickness}\end{center}

\href{https://swprs.org/2017/03/01/propaganda-in-der-wikipedia/}{**1.
March 2017}

\hypertarget{medienaufsicht-im-faktencheck}{%
\section{\texorpdfstring{\href{https://swprs.org/2017/03/01/medienaufsicht-faktencheck/}{Medienaufsicht
im
Faktencheck}}{Medienaufsicht im Faktencheck}}\label{medienaufsicht-im-faktencheck}}

\href{https://swprs.org/2017/03/01/medienaufsicht-faktencheck/}{\includegraphics{https://swprs.files.wordpress.com/2017/03/srf-ombudsstelle.png?w=600}}

Die Ombudsstelle des \emph{SRF} ist die erste Anlaufstelle für
Programm­be­schwerden des Publi­kums. Doch wie un­vor­ein­ge­nommen und
objektiv behandelt sie Beschwerden zu geo­po­li­tischen Themen?

Um dies zu über­prüfen, wurden während eines halben Jahres alle
Schluss­be­richte zum Syrien­kon­flikt einem Fakten­check unter­zogen.
Die Resul­tate sind bedenk­lich.

\href{https://swprs.org/srf-ombudsstelle-im-faktencheck/}{Zum
Faktencheck~→}

\begin{center}\rule{0.5\linewidth}{\linethickness}\end{center}

\href{https://swprs.org/2017/03/01/medienaufsicht-faktencheck/}{**1.
March 2017}

\hypertarget{der-atlantic-council}{%
\section{\texorpdfstring{\href{https://swprs.org/2017/03/01/der-atlantic-council/}{Der
Atlantic Council}}{Der Atlantic Council}}\label{der-atlantic-council}}

\href{https://swprs.org/2017/03/01/der-atlantic-council/}{\includegraphics{https://swprs.files.wordpress.com/2018/11/atlantic-council.png?w=532}}

Der \emph{Atlantic Council} ist
\href{https://www.rubikon.news/artikel/facebook-als-waffe}{bekannt} für
sein En­ga­ge­ment gegen NATO-kritische »Des­in­for­ma­tion«, seine
Kooperation mit Facebook, die zur Lö­schung diverser Seiten führte,
sowie seine Ein­wir­kungen auf die eu­ro­pä­ische Außen­politik. Doch
wer ist der *Atlantic Council?\\
*

\href{https://swprs.org/atlantic-council/}{Weiterlesen →}

\begin{center}\rule{0.5\linewidth}{\linethickness}\end{center}

\href{https://swprs.org/2017/03/01/der-atlantic-council/}{**1. March
2017}

\hypertarget{russische-propaganda}{%
\section{\texorpdfstring{\href{https://swprs.org/2017/03/01/russische-propaganda/}{Russische
Propaganda}}{Russische Propaganda}}\label{russische-propaganda}}

\href{https://swprs.org/2017/03/01/russische-propaganda/}{\includegraphics{https://swprs.files.wordpress.com/2018/11/kreml.png?w=495}}

Wie funktioniert russische Propaganda, und was macht sie so
wirkungs­voll?

\href{https://swprs.org/russische-propaganda/}{Zum Beitrag →}

\begin{center}\rule{0.5\linewidth}{\linethickness}\end{center}

\href{https://swprs.org/2017/03/01/russische-propaganda/}{**1. March
2017}

\hypertarget{die-integrity-initiative}{%
\section{\texorpdfstring{\href{https://swprs.org/2017/03/01/die-integrity-initiative/}{Die
»Integrity
Initiative«}}{Die »Integrity Initiative«}}\label{die-integrity-initiative}}

\href{https://swprs.org/2017/03/01/die-integrity-initiative/}{\includegraphics{https://swprs.files.wordpress.com/2018/12/ii-logo-e1549798726940.png?w=350}}

Es ist die wohl größte Geheimdienstenthüllung seit Edward Snowden. Doch
von den etablierten Medien wurde sie nahezu vollständig ignoriert. Ein
Überblick.

\href{https://swprs.org/die-integrity-initiative/}{Zum Beitrag →}

\begin{center}\rule{0.5\linewidth}{\linethickness}\end{center}

\href{https://swprs.org/2017/03/01/die-integrity-initiative/}{**1. March
2017}

\hypertarget{die-woz-und-die-weltpolitik}{%
\section{\texorpdfstring{\href{https://swprs.org/2017/03/01/die-woz-und-die-weltpolitik/}{Die
WOZ und
die~Weltpolitik}}{Die WOZ und die~Weltpolitik}}\label{die-woz-und-die-weltpolitik}}

\href{https://swprs.org/2017/03/01/die-woz-und-die-weltpolitik/}{\includegraphics{https://swprs.files.wordpress.com/2017/03/woz-logo-n.png?w=522}}

»Linksalternativ« und doch NATO-konform? Die WOZ zeigt wie's geht: In
Syrien etwa hätten ein paar
\href{https://www.woz.ch/1203/syrien/assad-geht-das-licht-aus}{Graffiti­sprayer}
eine
\href{https://www.woz.ch/1616/syriens-zukunft/assads-spiel-mit-dem-westen}{marxis­tisch}
ange­hauchte
\href{https://www.woz.ch/1511/kommentar-von-francois-moore/die-revolution-in-syrien-ist-am-ende}{»Revo­lution«}
junger
\href{https://www.woz.ch/1606/syrien/mithilfe-dieser-verdammten-russen-wird-dieser-bastard-noch-ueberleben}{Idealisten}
und \href{https://www.woz.ch/1235/syrien/kaempfen-und-beten}{frommer
Gottes­krieger} ausgelöst, während
\href{https://www.woz.ch/1324/syrien/ein-land-zersplittert-immer-mehr}{»das
Regime«} einen Krieg vom Zaun
\href{https://www.woz.ch/1321/syrien-und-der-westen/assad-kann-nur-gewinnen}{brach}
und mit »Fass­bomben« Kranken­häuser
\href{https://www.woz.ch/1416/syrien/fassbomben-gottes-wille-und-demokratie}{bombar­dierte},
sodass selbst eine NATO-Inter­vention
\href{https://www.woz.ch/1335/syrien/intervention-als-kleineres-uebel}{»das
kleinere Übel«} sei.

NATO-Kritiker Ganser hingegen
\href{https://www.woz.ch/1703/wahrheit-und-verschwoerung/das-ganser-phaenomen}{biete}
eine »Plattform für rechte Ver­schwö­rungs­theo­retiker«, und Wiki­leaks
-- an der Nieder­lage Clintons mitschuldig --
\href{https://www.woz.ch/1711/cia-dokumente/die-alternativen-fakten-von-wikileaks}{produziere}
»alter­native Fakten« für die »Neurechten«. Auch vor »alter­na­tiven
Medien« wird
\href{https://www.woz.ch/1743/qualitaet-der-medien/unterinformiert-und-ausgeliefert}{gewarnt}:
Diese »bedienen unverblümt Ver­schwörungs­theorien oder ver­breiten
rechte Propaganda«.

Wer die Global­isierung unvor­sichtig kriti­siert, sei womöglich ein
verkappter
\href{https://www.woz.ch/1708/wirtschaftlicher-protektionismus/die-voelkische-kritik-an-der-globalisierung}{»Rechts­nationa­list«},
und bei der Wachs­tums­politik des IWF dürfe man »nicht zu dogma­tisch
sein«, denn es
\href{https://www.woz.ch/1742/weltwirtschaft/die-hueterin-des-kapitalismus}{gelte},
»den Kapita­lismus vor der Rechten zu retten«. Selbst die
\href{https://www.woz.ch/1414/schweizerische-aussenpolitik/opportunistische-neutralitaet}{Schweizer
Neutra­lität} ist irgendwie »rechts«.

Medien­historisch erinnert die WOZ damit ein wenig an jene
\href{https://www.youtube.com/watch?v=3QAgCFjNXJE}{CIA-finanzierten
Publika­tionen}, die während des Kalten Krieges die potentiell kritische
Linke auf US-Kurs zu bringen versuchten. Und offenbar wird
geo­poli­tische Konfor­mität auch heute noch honoriert: Etwa mit
\href{https://swprs.files.wordpress.com/2017/10/amnesty-international-werbung.png}{ganz­seitigen
Farb­inseraten} von \emph{Amnesty Inter­national}, die in der WOZ den
Sturz von Washingtons Feinden
\href{https://consortiumnews.com/2012/06/18/amnestys-shilling-for-us-wars/}{bewerben}.

\begin{center}\rule{0.5\linewidth}{\linethickness}\end{center}

\href{https://swprs.org/2017/03/01/die-woz-und-die-weltpolitik/}{**1.
March 2017}

\hypertarget{was-ist-medienqualituxe4t}{%
\section{\texorpdfstring{\href{https://swprs.org/2017/03/01/medienqualitaet/}{Was
ist
Medienqualität?}}{Was ist Medienqualität?}}\label{was-ist-medienqualituxe4t}}

\href{https://swprs.org/2017/03/01/medienqualitaet/}{\includegraphics{https://swprs.files.wordpress.com/2018/09/mqr_logo.png?w=300}}

2018 wurde die zweite Ausgabe des \emph{Schweizer
Medien­qualitäts­rankings}
\href{http://medienqualitaet-schweiz.ch/files/3115/3578/3114/MQR-18_Hauptbefunde.pdf}{vorgestellt}.
Zuoberst fanden sich erneut die \emph{NZZ} sowie einige
*SRF-*Nach­rich­ten­for­mate. Stehen diese Resultate im Widerspruch zu
unseren \href{https://swprs.org/die-nzz-studie/}{Untersuchungen}, wonach
gerade jene Medien eine besonders hohe Propaganda-Intensität
\href{https://swprs.org/srf-propaganda-analyse/}{aufweisen}?

Keineswegs, denn das Qualitätsranking basiert auf rein formalen
\href{http://www.medienqualitaet-schweiz.ch/index.php/qualitaetsrating/}{Kriterien}
wie Relevanz, Aktualität und Professionalität -- woraus sich im
Endeffekt eine weitgehend triviale Sortierung der Medien von
boulevardesk bis bildungs­bürger­lich ergibt. Wer über den Syrienkrieg
statt über Superstars berichtet und dazu noch den Experten vom
NATO-Thinktank befragt, der schwingt im Ranking schon oben aus.

Das Qualitätsranking ist gut gemeint, die Autoren sorgen sich um den
ökonomisch bedingten Niedergang der klassischen Medien und die
\href{https://www.nzz.ch/feuilleton/medien/was-die-medien-fuer-die-schweizer-demokratie-leisten-ld.1416854}{Auswirkungen}
auf das Staatswesen. Doch für den kritischen Leser genügt ein solch
formaler Ansatz längst nicht mehr -- denn gefragt ist wahrhaftiger
Journalismus, und nicht bloß
\href{https://swprs.org/der-propaganda-schluessel/}{Manipulation} auf
hohem Niveau.

\begin{center}\rule{0.5\linewidth}{\linethickness}\end{center}

\href{https://swprs.org/2017/03/01/medienqualitaet/}{**1. March 2017}

\hypertarget{die-propaganda-matrix}{%
\section{\texorpdfstring{\href{https://swprs.org/2017/03/01/propaganda-matrix/}{Die
Propaganda-Matrix}}{Die Propaganda-Matrix}}\label{die-propaganda-matrix}}

\href{https://swprs.org/2017/03/01/propaganda-matrix/}{\includegraphics{https://swprs.files.wordpress.com/2017/03/propaganda-matrix-fs.png?w=449}}

Ob Russland, Syrien oder Donald Trump: Um die geopolitische
Bericht­erstattung westlicher Medien zu verstehen, muss man die
Schlüssel­rolle des amerikanischen \emph{Council on Foreign Relations
(CFR)} kennen.

In der folgenden Studie wird erstmals dargestellt, wie der CFR einen in
sich weitgehend geschlossenen, trans­atlantischen
Informations­­kreislauf schuf, in dem nahezu alle relevanten Quellen und
Bezugs­punkte von Mitgliedern des Councils und seiner
Partner­­organisationen kontrolliert werden.

Auf diese Weise entstand eine historisch einzigartige
Informations­­matrix, die klassischer Regierungs­propaganda autoritärer
Staaten deutlich überlegen ist, indes durch den Erfolg unabhängiger
Medien zunehmend an Wirksamkeit verliert.

\href{https://swprs.org/die-propaganda-matrix}{Zur Studie →}

\begin{center}\rule{0.5\linewidth}{\linethickness}\end{center}

\href{https://swprs.org/2017/03/01/propaganda-matrix/}{**1. March 2017}

\hypertarget{swiss-policy-research}{%
\subsubsection{Swiss Policy Research}\label{swiss-policy-research}}

\begin{itemize}
\tightlist
\item
  \href{https://swprs.org/kontakt/}{Kontakt}
\item
  \href{https://swprs.org/uebersicht/}{Übersicht}
\item
  \href{https://swprs.org/donationen/}{Donationen}
\item
  \href{https://swprs.org/disclaimer/}{Disclaimer}
\end{itemize}

\hypertarget{english}{%
\subsubsection{English}\label{english}}

\begin{itemize}
\tightlist
\item
  \href{https://swprs.org/contact/}{About Us / Contact}
\item
  \href{https://swprs.org/media-navigator/}{The Media Navigator}
\item
  \href{https://swprs.org/the-american-empire-and-its-media/}{The CFR
  and the Media}
\item
  \href{https://swprs.org/donations/}{Donations}
\end{itemize}

\hypertarget{follow-by-email}{%
\subsubsection{Follow by email}\label{follow-by-email}}

Follow

\href{https://wordpress.com/?ref=footer_custom_com}{WordPress.com}.

\protect\hyperlink{}{Up ↑}

\includegraphics{https://pixel.wp.com/b.gif?v=noscript}
