\protect\hyperlink{content}{Skip to content}

\href{https://swprs.org/}{}

\protect\hyperlink{search-container}{Search}

Search for:

\href{https://swprs.org/}{\includegraphics{https://swprs.files.wordpress.com/2020/05/swiss-policy-research-logo-300.png}}

\href{https://swprs.org/}{Swiss Policy Research}

Geopolitics and Media

Menu

\begin{itemize}
\tightlist
\item
  \href{https://swprs.org}{Start}
\item
  \href{https://swprs.org/srf-propaganda-analyse/}{Studien}

  \begin{itemize}
  \tightlist
  \item
    \href{https://swprs.org/srf-propaganda-analyse/}{SRF / ZDF}
  \item
    \href{https://swprs.org/die-nzz-studie/}{NZZ-Studie}
  \item
    \href{https://swprs.org/der-propaganda-multiplikator/}{Agenturen}
  \item
    \href{https://swprs.org/die-propaganda-matrix/}{Medienmatrix}
  \end{itemize}
\item
  \href{https://swprs.org/medien-navigator/}{Analysen}

  \begin{itemize}
  \tightlist
  \item
    \href{https://swprs.org/medien-navigator/}{Navigator}
  \item
    \href{https://swprs.org/der-propaganda-schluessel/}{Techniken}
  \item
    \href{https://swprs.org/propaganda-in-der-wikipedia/}{Wikipedia}
  \item
    \href{https://swprs.org/logik-imperialer-kriege/}{Kriege}
  \end{itemize}
\item
  \href{https://swprs.org/netzwerk-medien-schweiz/}{Netzwerke}

  \begin{itemize}
  \tightlist
  \item
    \href{https://swprs.org/netzwerk-medien-schweiz/}{Schweiz}
  \item
    \href{https://swprs.org/netzwerk-medien-deutschland/}{Deutschland}
  \item
    \href{https://swprs.org/medien-in-oesterreich/}{Österreich}
  \item
    \href{https://swprs.org/das-american-empire-und-seine-medien/}{USA}
  \end{itemize}
\item
  \href{https://swprs.org/bericht-eines-journalisten/}{Fokus I}

  \begin{itemize}
  \tightlist
  \item
    \href{https://swprs.org/bericht-eines-journalisten/}{Journalistenbericht}
  \item
    \href{https://swprs.org/russische-propaganda/}{Russische Propaganda}
  \item
    \href{https://swprs.org/die-israel-lobby-fakten-und-mythen/}{Die
    »Israel-Lobby«}
  \item
    \href{https://swprs.org/geopolitik-und-paedokriminalitaet/}{Pädokriminalität}
  \end{itemize}
\item
  \href{https://swprs.org/migration-und-medien/}{Fokus II}

  \begin{itemize}
  \tightlist
  \item
    \href{https://swprs.org/covid-19-hinweis-ii/}{Coronavirus}
  \item
    \href{https://swprs.org/die-integrity-initiative/}{Integrity
    Initiative}
  \item
    \href{https://swprs.org/migration-und-medien/}{Migration \& Medien}
  \item
    \href{https://swprs.org/der-fall-magnitsky/}{Magnitsky Act}
  \end{itemize}
\item
  \href{https://swprs.org/kontakt/}{Projekt}

  \begin{itemize}
  \tightlist
  \item
    \href{https://swprs.org/kontakt/}{Kontakt}
  \item
    \href{https://swprs.org/uebersicht/}{Seitenübersicht}
  \item
    \href{https://swprs.org/medienspiegel/}{Medienspiegel}
  \item
    \href{https://swprs.org/donationen/}{Donationen}
  \end{itemize}
\item
  \href{https://swprs.org/contact/}{English}
\end{itemize}

\protect\hyperlink{}{Open Search}

\hypertarget{tag-zensur}{%
\section{Tag: Zensur}\label{tag-zensur}}

\hypertarget{zensur-und-selbstzensur}{%
\section{\texorpdfstring{\href{https://swprs.org/2017/03/01/zensur-in-schweizer-medien/}{Zensur
und
Selbstzensur}}{Zensur und Selbstzensur}}\label{zensur-und-selbstzensur}}

\href{https://swprs.org/2017/03/01/zensur-in-schweizer-medien/}{\includegraphics{https://swprs.files.wordpress.com/2016/07/zensur-selbstzensur-c.png?w=306}}

Zensur und Selbst­zensur bei geo­po­li­tischen Kon­f‌lik­ten sind in der
Schweiz keines­wegs un­be­kannt, wie ein Blick in die Ge­schichte zeigt.

Um das Land keinen un­nöti­gen Ri­si­ken aus­zu­setzen,
\href{http://www.amazon.de/Selbstzensur-schweizerische-Pressepolitik-Zweiten-Weltkrieg/dp/3719304566}{muss­ten}
sich Medien und
\href{https://www.chronos-verlag.ch/node/20528}{Buch­ver­lage} wäh­rend
des

\begin{enumerate}
\def\labelenumi{\arabic{enumi}.}
\tightlist
\item
  und 2. Welt­kriegs und während des
  \href{http://www.swissinfo.ch/ger/das-ende-eines-nationalen-maenner-netzwerks/4205194}{Kal­ten
  Kriegs} an einen po­li­tisch definierten Mei­nungs­korri­dor halten,
  der sich an den welt­wei­ten Kräfte­ver­hält­nissen orientierte.
\end{enumerate}

Durch die Ereignisse von 1990 und 2001 nahm der
\href{https://www.youtube.com/watch?v=a4eGtXFDFJA}{Druck} auf
Drittstaaten und ihre Medien wei­ter zu: »Entweder mit uns, oder gegen
uns.«

Aufgrund der Medien­kon­zen­tration werden in­zwi­schen zudem
\href{https://swprs.files.wordpress.com/2018/03/broschur_jahrbuch_foeg_deutsch_2015.pdf\#page=13}{über
90\%} des Schwei­zer Mark­tes von nur noch fünf Medien­häusern bedient:
Tamedia, Ringier, NZZ Medien und AZ Medien, sowie der SRG (siehe
\href{https://swprs.org/netzwerk-medien-schweiz/}{Info­grafik}).

Eine echte \href{https://swprs.org/medien-navigator/}{Medienvielfalt}
entstand mithin erst durch das Internet -- obschon auch hier bereits
diverse
\href{https://www.heise.de/tp/features/Facebook-Fake-News-und-die-Privatisierung-der-Zensur-3599878.html}{Zensurversuche}
zu beobachten sind.

\begin{center}\rule{0.5\linewidth}{\linethickness}\end{center}

\href{https://swprs.org/2017/03/01/zensur-in-schweizer-medien/}{**1.
March 2017}

\hypertarget{das-gewuxfcnschte-narrativ}{%
\section{\texorpdfstring{\href{https://swprs.org/2017/03/01/das-gewuenschte-narrativ/}{Das
gewünschte
Narrativ}}{Das gewünschte Narrativ}}\label{das-gewuxfcnschte-narrativ}}

\href{https://swprs.org/2017/03/01/das-gewuenschte-narrativ/}{\includegraphics{https://swprs.files.wordpress.com/2016/02/medien-narrativ1.png?w=400}}

Bei geopolitischen Konflikten bestehen oftmals vordefinierte mediale
Narrative. Was geschieht, wenn ein Schweizer Jour­na­list davon abweicht
und über die »falschen« Themen be­richtet?

Heute kaum noch vorstellbar, doch mitten im
\href{https://de.wikipedia.org/wiki/Bosnienkrieg}{Bosnien­krieg}
(1992-95) veröffentlichte der damalige Aus­lands­chef der
\emph{Welt­woche} einen Artikel zu Kriegs­lügen in west­lichen Medien.

Daraufhin geschah Folgendes:

\href{https://swprs.org/das-gewuenschte-narrativ\#weiterlesen}{Weiterlesen
→}

\begin{center}\rule{0.5\linewidth}{\linethickness}\end{center}

\href{https://swprs.org/2017/03/01/das-gewuenschte-narrativ/}{**1. March
2017}

\hypertarget{die-angst-vor-den-lesern}{%
\section{\texorpdfstring{\href{https://swprs.org/2017/03/01/leserkommentare/}{Die
Angst vor
den~Lesern}}{Die Angst vor den~Lesern}}\label{die-angst-vor-den-lesern}}

\href{https://swprs.org/2017/03/01/leserkommentare/}{\includegraphics{https://swprs.files.wordpress.com/2016/07/leserkommentare.png?w=600}}

Weil Propaganda von kritischen Lesern immer öfter und schneller entlarvt
wird, sind viele Medien dazu über­ge­gangen, die Kommentar­funktion auf
ihren Inter­net­­seiten stark zu zensieren oder ganz zu
\href{https://www.heise.de/tp/features/Konzentriertes-Gejammer-NZZ-schliesst-Kommentarspalte-3618957.html}{deaktivieren}.
Zuletzt griff selbst die vermeintlich liberale \emph{NZZ} zu dieser
\href{https://www.heise.de/tp/features/Konzentriertes-Gejammer-NZZ-schliesst-Kommentarspalte-3618957.html}{Maßnahme}.

Schließlich versuchten die ertappten Medien, die kri­ti­schen Leser als
Trolle
\href{https://www.nzz.ch/international/putins-internetpiraten-1.18324628}{dar­zu­stellen},
die womöglich von aus­län­dischen Re­gie­rungen fürs Kom­men­tieren
bezahlt würden. Be­lege da­für blie­ben aus, und inhaltlich wurde auf
die Leser­kritik ohnehin nicht ein­ge­gangen.

Doch nicht nur von den Medien, auch im Online-Lexikon \emph{Wikipedia}
werden die Leser an der freien Meinungs­bil­dung
\href{https://swprs.org/propaganda-in-der-wikipedia/}{gehindert}: Hier
sorgt eine kleine Gruppe anonymer »Adminis­tra­toren« dafür, dass bei
geo­po­li­tisch brisanten Themen ab­wei­chende Positionen gelöscht,
Autoren gesperrt und kritische Forscher diffamiert werden (siehe
\href{https://swprs.org/propaganda-in-der-wikipedia/}{Vertiefungsstudie}).

\begin{center}\rule{0.5\linewidth}{\linethickness}\end{center}

\href{https://swprs.org/2017/03/01/leserkommentare/}{**1. March 2017}

\hypertarget{die-grenzen-der-pressefreiheit}{%
\section{\texorpdfstring{\href{https://swprs.org/2017/03/01/die-grenzen-der-pressefreiheit/}{Die
Grenzen
der~Pressefreiheit}}{Die Grenzen der~Pressefreiheit}}\label{die-grenzen-der-pressefreiheit}}

\href{https://swprs.org/2017/03/01/die-grenzen-der-pressefreiheit/}{\includegraphics{https://swprs.files.wordpress.com/2017/12/reporter_ohne_grenzen_logo_s.png?w=530}}

Der \href{http://pressclub.ch/?lang=en}{Schweizer Presseclub} in Genf
genießt einen ausgezeichneten Ruf: Seit seiner Gründung hat er über
zweitausend Anlässe mit illustren Rednern von Fidel Castro bis Henry
Kissinger und von Jean Ziegler bis Klaus Schwab organisiert.

Doch für Ende November 2017 war ein
\href{http://pressclub.ch/they-dont-care-about-us-white-helmets-true-agenda/?lang=en}{Vortrag}
angekündigt, der sich kritisch mit den in west­li­chen Medien populären
\href{https://www.hintergrund.de/globales/kriege/weisse-helme-ohne-weisse-westen/}{Syrischen
Weiß­helmen} befassen wollte. Daraufhin geschah Folgendes:

\href{https://swprs.org/die-grenzen-der-pressefreiheit/}{Weiterlesen →}

\begin{center}\rule{0.5\linewidth}{\linethickness}\end{center}

\href{https://swprs.org/2017/03/01/die-grenzen-der-pressefreiheit/}{**1.
March 2017}

\hypertarget{swiss-policy-research}{%
\subsubsection{Swiss Policy Research}\label{swiss-policy-research}}

\begin{itemize}
\tightlist
\item
  \href{https://swprs.org/kontakt/}{Kontakt}
\item
  \href{https://swprs.org/uebersicht/}{Übersicht}
\item
  \href{https://swprs.org/donationen/}{Donationen}
\item
  \href{https://swprs.org/disclaimer/}{Disclaimer}
\end{itemize}

\hypertarget{english}{%
\subsubsection{English}\label{english}}

\begin{itemize}
\tightlist
\item
  \href{https://swprs.org/contact/}{About Us / Contact}
\item
  \href{https://swprs.org/media-navigator/}{The Media Navigator}
\item
  \href{https://swprs.org/the-american-empire-and-its-media/}{The CFR
  and the Media}
\item
  \href{https://swprs.org/donations/}{Donations}
\end{itemize}

\hypertarget{follow-by-email}{%
\subsubsection{Follow by email}\label{follow-by-email}}

Follow

\href{https://wordpress.com/?ref=footer_custom_com}{WordPress.com}.

\protect\hyperlink{}{Up ↑}

\includegraphics{https://pixel.wp.com/b.gif?v=noscript}
