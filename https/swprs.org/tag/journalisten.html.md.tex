\protect\hyperlink{content}{Skip to content}

\href{https://swprs.org/}{}

\protect\hyperlink{search-container}{Search}

Search for:

\href{https://swprs.org/}{\includegraphics{https://swprs.files.wordpress.com/2020/05/swiss-policy-research-logo-300.png}}

\href{https://swprs.org/}{Swiss Policy Research}

Geopolitics and Media

Menu

\begin{itemize}
\tightlist
\item
  \href{https://swprs.org}{Start}
\item
  \href{https://swprs.org/srf-propaganda-analyse/}{Studien}

  \begin{itemize}
  \tightlist
  \item
    \href{https://swprs.org/srf-propaganda-analyse/}{SRF / ZDF}
  \item
    \href{https://swprs.org/die-nzz-studie/}{NZZ-Studie}
  \item
    \href{https://swprs.org/der-propaganda-multiplikator/}{Agenturen}
  \item
    \href{https://swprs.org/die-propaganda-matrix/}{Medienmatrix}
  \end{itemize}
\item
  \href{https://swprs.org/medien-navigator/}{Analysen}

  \begin{itemize}
  \tightlist
  \item
    \href{https://swprs.org/medien-navigator/}{Navigator}
  \item
    \href{https://swprs.org/der-propaganda-schluessel/}{Techniken}
  \item
    \href{https://swprs.org/propaganda-in-der-wikipedia/}{Wikipedia}
  \item
    \href{https://swprs.org/logik-imperialer-kriege/}{Kriege}
  \end{itemize}
\item
  \href{https://swprs.org/netzwerk-medien-schweiz/}{Netzwerke}

  \begin{itemize}
  \tightlist
  \item
    \href{https://swprs.org/netzwerk-medien-schweiz/}{Schweiz}
  \item
    \href{https://swprs.org/netzwerk-medien-deutschland/}{Deutschland}
  \item
    \href{https://swprs.org/medien-in-oesterreich/}{Österreich}
  \item
    \href{https://swprs.org/das-american-empire-und-seine-medien/}{USA}
  \end{itemize}
\item
  \href{https://swprs.org/bericht-eines-journalisten/}{Fokus I}

  \begin{itemize}
  \tightlist
  \item
    \href{https://swprs.org/bericht-eines-journalisten/}{Journalistenbericht}
  \item
    \href{https://swprs.org/russische-propaganda/}{Russische Propaganda}
  \item
    \href{https://swprs.org/die-israel-lobby-fakten-und-mythen/}{Die
    »Israel-Lobby«}
  \item
    \href{https://swprs.org/geopolitik-und-paedokriminalitaet/}{Pädokriminalität}
  \end{itemize}
\item
  \href{https://swprs.org/migration-und-medien/}{Fokus II}

  \begin{itemize}
  \tightlist
  \item
    \href{https://swprs.org/covid-19-hinweis-ii/}{Coronavirus}
  \item
    \href{https://swprs.org/die-integrity-initiative/}{Integrity
    Initiative}
  \item
    \href{https://swprs.org/migration-und-medien/}{Migration \& Medien}
  \item
    \href{https://swprs.org/der-fall-magnitsky/}{Magnitsky Act}
  \end{itemize}
\item
  \href{https://swprs.org/kontakt/}{Projekt}

  \begin{itemize}
  \tightlist
  \item
    \href{https://swprs.org/kontakt/}{Kontakt}
  \item
    \href{https://swprs.org/uebersicht/}{Seitenübersicht}
  \item
    \href{https://swprs.org/medienspiegel/}{Medienspiegel}
  \item
    \href{https://swprs.org/donationen/}{Donationen}
  \end{itemize}
\item
  \href{https://swprs.org/contact/}{English}
\end{itemize}

\protect\hyperlink{}{Open Search}

\hypertarget{tag-journalisten}{%
\section{Tag: Journalisten}\label{tag-journalisten}}

\hypertarget{das-gewuxfcnschte-narrativ}{%
\section{\texorpdfstring{\href{https://swprs.org/2017/03/01/das-gewuenschte-narrativ/}{Das
gewünschte
Narrativ}}{Das gewünschte Narrativ}}\label{das-gewuxfcnschte-narrativ}}

\href{https://swprs.org/2017/03/01/das-gewuenschte-narrativ/}{\includegraphics{https://swprs.files.wordpress.com/2016/02/medien-narrativ1.png?w=400}}

Bei geopolitischen Konflikten bestehen oftmals vordefinierte mediale
Narrative. Was geschieht, wenn ein Schweizer Jour­na­list davon abweicht
und über die »falschen« Themen be­richtet?

Heute kaum noch vorstellbar, doch mitten im
\href{https://de.wikipedia.org/wiki/Bosnienkrieg}{Bosnien­krieg}
(1992-95) veröffentlichte der damalige Aus­lands­chef der
\emph{Welt­woche} einen Artikel zu Kriegs­lügen in west­lichen Medien.

Daraufhin geschah Folgendes:

\href{https://swprs.org/das-gewuenschte-narrativ\#weiterlesen}{Weiterlesen
→}

\begin{center}\rule{0.5\linewidth}{\linethickness}\end{center}

\href{https://swprs.org/2017/03/01/das-gewuenschte-narrativ/}{**1. March
2017}

\hypertarget{die-konferenz}{%
\section{\texorpdfstring{\href{https://swprs.org/2017/03/01/schweizer-medien-bilderberg-konferenz/}{Die
Konferenz}}{Die Konferenz}}\label{die-konferenz}}

\href{https://swprs.org/2017/03/01/schweizer-medien-bilderberg-konferenz/}{\includegraphics{https://swprs.files.wordpress.com/2016/02/bilderberg_2011.png?w=440}}

Die großen Schweizer Medien­­häuser sind in geo­poli­tische
\href{https://swprs.org/netzwerk-medien-schweiz/}{Netz­werke}
ein­ge­bun­den: So nehmen die wichtigsten Schweizer Verleger und
Chef­redakteure im Turnus an der jähr­lichen
\href{http://www.bilderbergmeetings.org/}{Bilderberg-Konferenz} teil, wo
sie im privaten Rahmen auf die trans­atlan­tische Elite aus
Wirt­schaf‌t, Politik und Militär treffen.

Teilnehmer seit 1991 (siehe
\href{https://swprs.org/netzwerk-medien-schweiz/}{Info­grafik}):

\includegraphics{https://swprs.files.wordpress.com/2017/03/teilnehmer-bilderberg-ch-1.png?w=736}

Auch der journa­lis­tische Nach­wuchs wird ge­för­dert: Sowohl der
\href{http://www.americanswiss.org/news/arthur-honegger-spotlight/}{*10vor10-*​Mode­ra­tor}
des SRF wie auch der
\href{http://www.americanswiss.org/news/niklaus-nuspliger-spotlight/}{*NZZ-*Korres­pon­dent}
für die EU \& NATO wurden von der
\href{http://www.americanswiss.org/}{\emph{Ameri­can Swiss
Foun­da­tion}} zu »Young Leaders« ernannt -- und neh­men in dieser Rolle
an
\href{http://www.americanswiss.org/ambassador-barras-hosts-dinner-for-young-leaders-1/}{exklu­siven
Dinners} mit hoch­rang­igen US-Ver­tre­tern teil.

\emph{Foto:} Bilder­berg-Meeting
\href{https://www.theguardian.com/world/gallery/2011/jun/15/bilderberg-in-pictures}{2011}
in St. Moritz.

\begin{center}\rule{0.5\linewidth}{\linethickness}\end{center}

\href{https://swprs.org/2017/03/01/schweizer-medien-bilderberg-konferenz/}{**1.
March 2017}

\hypertarget{warum-der-tagi-nichts-verpasst}{%
\section{\texorpdfstring{\href{https://swprs.org/2017/03/01/warum-der-tagesanzeiger-nichts-verpasst/}{Warum
der Tagi
nichts~verpasst}}{Warum der Tagi nichts~verpasst}}\label{warum-der-tagi-nichts-verpasst}}

\href{https://swprs.org/2017/03/01/warum-der-tagesanzeiger-nichts-verpasst/}{\includegraphics{https://swprs.files.wordpress.com/2016/06/lena-logo2.png?w=440}}

Ob Ukraine, Syrien oder Chi­na: Der \emph{Zürcher Tages-Anzeiger}
schreibt viele seiner Aus­lands­berichte nicht mehr selbst, sondern
bezieht sie im Rah­men einer
\href{https://www.tagesanzeiger.ch/schweiz/standard/In-eigener-Sache/story/24648194}{»umfassenden
Ko­ope­ra­tion«} von der \emph{Süd­deut­schen Zeitung.}

Deren Außen­politik­chef
\href{https://swprs.org/netzwerk-medien-deutschland/}{zählt} indes zu
den bekanntesten Trans­at­lan­tikern Deutsch­lands -- und ent­spre­chend
le­sen sich die Arti­kel im \emph{Tagi.} Aus dem ara­bi­schen Raum
\href{https://web.archive.org/web/20170606085220/http://www.icfj.org/sites/default/files/Kr\%C3\%BCger.pdf}{berichtet}
z.B. ein Absol­vent des ameri­ka­ni­schen
\emph{Arthur-F.-Burns-Fellowship}, aus Mos­kau ein
\href{https://spiegelkabinett-blog.blogspot.com/2016/09/julian-hans-von-der-suddeutschen.html}{Ab­gänger}
der \emph{Henri-Nannen-Schule}.

Über den Onlinedienst
\emph{\href{https://de.wikipedia.org/wiki/Newsnet}{Newsnet}} werden
Aus­lands­be­rich­te des \emph{Tagi} zudem an andere Schwei­zer
Zei­tungen wei­ter­ge­reicht. Auf diese Weise
\href{http://www.tagesanzeiger.ch/ausland/europa/Den-Ausloeser-zum-Krieg-habe-ich-gedrueckt/story/16330278}{er­scheinen}
Beiträge der \emph{Süd­deutschen Zeitung} via \emph{Tages­-Anzeiger} und
\emph{Newsnet} zu­sätz­lich im \emph{Berner Bund} und der \emph{Basler
Zeitung.}

Seit 2015 ist der \emph{Tages­-Anzeiger} über­dies Teil der
\emph{\href{https://de.wikipedia.org/wiki/Leading_European_Newspaper_Alliance}{Leading
European News­paper Alliance} (LENA).} Zweck des Ver­bunds ist die
``Ent­wick­lung und der Aus­tausch re­dak­tio­neller In­hal­te'' mit
anderen LENA- Mit­glie­dern wie \emph{Le Fi­g­aro}, \emph{Die Welt},
\emph{El País} oder \emph{La Re­pub­blica}.

Alle diese Zei­tungen sind in das
\href{https://swprs.files.wordpress.com/2016/07/bilderberg_teilnehmer_1954-2014.pdf}{Bilder­berg-Netz­werk}
ein­ge­bun­den -- kann es da noch über­raschen, dass auch
LENA- Repor­ta­gen zumeist auf
\href{http://www.tagesanzeiger.ch/ausland/europa/Wer-sagt-was-er-denkt-nobrriskiert-allesnobr/story/17225010}{trans­at­lan­tischer
Linie} sind?

\begin{center}\rule{0.5\linewidth}{\linethickness}\end{center}

\href{https://swprs.org/2017/03/01/warum-der-tagesanzeiger-nichts-verpasst/}{**1.
March 2017}

\hypertarget{eine-bruxfccke-uxfcber-den-atlantik}{%
\section{\texorpdfstring{\href{https://swprs.org/2017/03/01/eine-bruecke-ueber-den-atlantik/}{Eine
Brücke über
den~Atlantik}}{Eine Brücke über den~Atlantik}}\label{eine-bruxfccke-uxfcber-den-atlantik}}

\href{https://swprs.org/2017/03/01/eine-bruecke-ueber-den-atlantik/}{\includegraphics{https://swprs.files.wordpress.com/2016/07/atlantikbruecke-logo.png?w=600}}

Der deutsche Medien­gigant
\href{https://de.wikipedia.org/wiki/Axel_Springer_SE}{Axel Springer}
(\emph{BILD}, \emph{Die Welt,} etc.) gewinnt auch in der Schweiz
zu­neh­mend an Einfluss. Bereits 1999 wurde die \emph{Handels­zeitung}
\href{https://de.wikipedia.org/wiki/Handelszeitung}{über­nommen}, 2007
\href{https://de.wikipedia.org/wiki/Jean_Frey_AG}{folgten} u.a. die
\emph{Bilanz} und der \emph{Beobachter}. 2014 wurde eine umfang­reiche
\href{http://www.blick.ch/news/wirtschaft/medien-ringier-und-axel-springer-gruenden-gemeinschaftsunternehmen-in-der-schweiz-id3357037.html}{Koope­ration}
mit *Blick-*Verleger Ringier publik, und 2015 der
\href{http://www.persoenlich.com/marketing/die-werbeallianz-prasentiert-sich-zum-ersten-mal-der-branche}{Ein­stieg}
in die Werbe­allianz \emph{Admeira} mit Ringier, Swiss­com und der SRG.

Als angel­säch­sische
\href{https://de.wikipedia.org/wiki/Lizenzzeitung}{Li­zenz­grün­dung}
von 1946 ist Axel Springer -- wie die meisten deutschen Leit­medien --
bis heute tief in den US-Macht­­struk­turen ver­wur­zelt. So war
Konzern­­chef Mathias Döpfner Mit­glied im
\href{https://www.cfr.org/global-board-advisors}{Bei­rat} des \emph{U.S.
Council on Foreign Relations (CFR),} und der lang­jährige *BILD-*​Chef
Kai Diek­­mann ist Vor­stands­mitglied der ame­ri­ka­treuen
\emph{\href{https://de.wikipedia.org/wiki/Atlantik-Br\%C3\%BCcke}{Atlantik-Brücke},}
in der viele der bekanntesten Medien­leute Deutsch­lands ver­ei­nigt
sind (siehe
\href{https://swprs.org/netzwerk-medien-deutschland/}{Infografik}).

Springer-Journa­listen sind zudem
\href{https://bildblog.de/89290/axel-springer-gibt-sich-neue-alte-grundsaetze/}{ver­­trag­­lich
ver­pfli­ch­tet}, das \emph{»trans­at­lantische Bündnis«} bzw. die
\emph{»Soli­da­rität mit den USA«} zu unter­stützen. So auch der heutige
\href{https://de.wikipedia.org/wiki/Roger_K\%C3\%B6ppel}{Ver­leger} der
\emph{Welt­woche}, der zuvor Chef­re­dakteur von Springers \emph{Welt}
war und den Irak­krieg noch 2004 mit diesen Worten ver­tei­digte:

\emph{``Die UNO schützt die Welt­ordnung nicht, zu deren Hüterin sie
sich irr­tüm­licher­weise erklärt. Sie ist im Gegen­teil das Derivat
eines Friedens, den ameri­ka­nische Truppen sichern'',} weshalb
\emph{``Europa auf die USA als hege­mon­ialer Hüter der west­lichen
`Welt­gewalt­ordnung' nicht ver­zichten kann''.}

\begin{center}\rule{0.5\linewidth}{\linethickness}\end{center}

\href{https://swprs.org/2017/03/01/eine-bruecke-ueber-den-atlantik/}{**1.
March 2017}

\hypertarget{bericht-eines-journalisten}{%
\section{\texorpdfstring{\href{https://swprs.org/2017/03/01/bericht-eines-journalisten/}{Bericht
eines
Journalisten}}{Bericht eines Journalisten}}\label{bericht-eines-journalisten}}

\href{https://swprs.org/2017/03/01/bericht-eines-journalisten/}{\includegraphics{https://swprs.files.wordpress.com/2018/01/mainstreammedia.png?w=600}}

Wie entsteht der Mainstream in den Medien? Woher kommt die Propaganda?
Im folgenden Beitrag spricht erstmals ein Schweizer Top-Journalist über
seine langjährigen Erfahrungen.

\href{https://swprs.org/bericht-eines-journalisten/}{Zum Beitrag →}

\begin{center}\rule{0.5\linewidth}{\linethickness}\end{center}

\href{https://swprs.org/2017/03/01/bericht-eines-journalisten/}{**1.
March 2017}

\hypertarget{swiss-policy-research}{%
\subsubsection{Swiss Policy Research}\label{swiss-policy-research}}

\begin{itemize}
\tightlist
\item
  \href{https://swprs.org/kontakt/}{Kontakt}
\item
  \href{https://swprs.org/uebersicht/}{Übersicht}
\item
  \href{https://swprs.org/donationen/}{Donationen}
\item
  \href{https://swprs.org/disclaimer/}{Disclaimer}
\end{itemize}

\hypertarget{english}{%
\subsubsection{English}\label{english}}

\begin{itemize}
\tightlist
\item
  \href{https://swprs.org/contact/}{About Us / Contact}
\item
  \href{https://swprs.org/media-navigator/}{The Media Navigator}
\item
  \href{https://swprs.org/the-american-empire-and-its-media/}{The CFR
  and the Media}
\item
  \href{https://swprs.org/donations/}{Donations}
\end{itemize}

\hypertarget{follow-by-email}{%
\subsubsection{Follow by email}\label{follow-by-email}}

Follow

\href{https://wordpress.com/?ref=footer_custom_com}{WordPress.com}.

\protect\hyperlink{}{Up ↑}

Post to

\protect\hyperlink{}{Cancel}

\includegraphics{https://pixel.wp.com/b.gif?v=noscript}
