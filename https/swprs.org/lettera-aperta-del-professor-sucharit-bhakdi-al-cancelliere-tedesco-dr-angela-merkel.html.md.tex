\protect\hyperlink{content}{Skip to content}

\href{https://swprs.org/}{}

\protect\hyperlink{search-container}{Search}

Search for:

\href{https://swprs.org/}{\includegraphics{https://swprs.files.wordpress.com/2020/05/swiss-policy-research-logo-300.png}}

\href{https://swprs.org/}{Swiss Policy Research}

Geopolitics and Media

Menu

\begin{itemize}
\tightlist
\item
  \href{https://swprs.org}{Start}
\item
  \href{https://swprs.org/srf-propaganda-analyse/}{Studien}

  \begin{itemize}
  \tightlist
  \item
    \href{https://swprs.org/srf-propaganda-analyse/}{SRF / ZDF}
  \item
    \href{https://swprs.org/die-nzz-studie/}{NZZ-Studie}
  \item
    \href{https://swprs.org/der-propaganda-multiplikator/}{Agenturen}
  \item
    \href{https://swprs.org/die-propaganda-matrix/}{Medienmatrix}
  \end{itemize}
\item
  \href{https://swprs.org/medien-navigator/}{Analysen}

  \begin{itemize}
  \tightlist
  \item
    \href{https://swprs.org/medien-navigator/}{Navigator}
  \item
    \href{https://swprs.org/der-propaganda-schluessel/}{Techniken}
  \item
    \href{https://swprs.org/propaganda-in-der-wikipedia/}{Wikipedia}
  \item
    \href{https://swprs.org/logik-imperialer-kriege/}{Kriege}
  \end{itemize}
\item
  \href{https://swprs.org/netzwerk-medien-schweiz/}{Netzwerke}

  \begin{itemize}
  \tightlist
  \item
    \href{https://swprs.org/netzwerk-medien-schweiz/}{Schweiz}
  \item
    \href{https://swprs.org/netzwerk-medien-deutschland/}{Deutschland}
  \item
    \href{https://swprs.org/medien-in-oesterreich/}{Österreich}
  \item
    \href{https://swprs.org/das-american-empire-und-seine-medien/}{USA}
  \end{itemize}
\item
  \href{https://swprs.org/bericht-eines-journalisten/}{Fokus I}

  \begin{itemize}
  \tightlist
  \item
    \href{https://swprs.org/bericht-eines-journalisten/}{Journalistenbericht}
  \item
    \href{https://swprs.org/russische-propaganda/}{Russische Propaganda}
  \item
    \href{https://swprs.org/die-israel-lobby-fakten-und-mythen/}{Die
    »Israel-Lobby«}
  \item
    \href{https://swprs.org/geopolitik-und-paedokriminalitaet/}{Pädokriminalität}
  \end{itemize}
\item
  \href{https://swprs.org/migration-und-medien/}{Fokus II}

  \begin{itemize}
  \tightlist
  \item
    \href{https://swprs.org/covid-19-hinweis-ii/}{Coronavirus}
  \item
    \href{https://swprs.org/die-integrity-initiative/}{Integrity
    Initiative}
  \item
    \href{https://swprs.org/migration-und-medien/}{Migration \& Medien}
  \item
    \href{https://swprs.org/der-fall-magnitsky/}{Magnitsky Act}
  \end{itemize}
\item
  \href{https://swprs.org/kontakt/}{Projekt}

  \begin{itemize}
  \tightlist
  \item
    \href{https://swprs.org/kontakt/}{Kontakt}
  \item
    \href{https://swprs.org/uebersicht/}{Seitenübersicht}
  \item
    \href{https://swprs.org/medienspiegel/}{Medienspiegel}
  \item
    \href{https://swprs.org/donationen/}{Donationen}
  \end{itemize}
\item
  \href{https://swprs.org/contact/}{English}
\end{itemize}

\protect\hyperlink{}{Open Search}

\hypertarget{lettera-aperta-del-professor-sucharit-bhakdi-al-cancelliere-tedesco-dr-angela-merkel}{%
\section{Lettera aperta del Professor Sucharit Bhakdi al Cancelliere
tedesco Dr.
Angela~Merkel}\label{lettera-aperta-del-professor-sucharit-bhakdi-al-cancelliere-tedesco-dr-angela-merkel}}

~

\includegraphics{https://swprs.files.wordpress.com/2020/03/bakhdi-letter-header.png?w=736\&h=297}

\textbf{Lingue}:
\href{https://swprs.org/offener-brief-von-professor-sucharit-bhakdi-an-bundeskanzlerin-dr-angela-merkel/}{DE},
\href{https://swprs.org/open-letter-from-professor-sucharit-bhakdi-to-german-chancellor-dr-angela-merkel/}{EN},
\href{https://swprs.org/professor-sucharit-bhakdi-avalik-kiri-angela-merkelile/}{EE},
\href{http://piensachile.com/2020/03/carta-abierta-a-angela-merkel/}{ES},
\href{https://swprs.org/covid-19-lettre-ouverte-du-professeur-sucharit-bhakdi-a-la-chanceliere-allemande-dre-angela-merkel/}{FR},
\href{https://swprs.org/professor-bhakdi-open-letter-greek/}{GR},
\href{https://yanivhamo.com/open-letter-from-professor-sucharit-bhakdi-to-german-chancellor-dr-angela-merkel-hebrew/}{HE},
\href{https://swprs.org/lettera-aperta-del-professor-sucharit-bhakdi-al-cancelliere-tedesco-dr-angela-merkel/}{IT},
\href{https://swprs.org/open-brief-van-professor-sucharit-bhakdi-aan-de-duitse-bondskanselier-dr-angela-merkel/}{NL},
\href{https://swprs.org/carta-aberta-do-professor-sucharit-bhakdi-a-chanceler-alema-dra-angela-merkel/}{PT},
\href{https://swprs.org/\%d0\%be\%d1\%82\%d0\%ba\%d1\%80\%d1\%8b\%d1\%82\%d0\%be\%d0\%b5-\%d0\%bf\%d0\%b8\%d1\%81\%d1\%8c\%d0\%bc\%d0\%be-\%d0\%bf\%d1\%80\%d0\%be\%d1\%84\%d0\%b5\%d1\%81\%d1\%81\%d0\%be\%d1\%80\%d0\%b0-\%d1\%81\%d1\%83\%d1\%87\%d0\%b0\%d1\%80\%d0\%b8\%d1\%82\%d0\%b0/}{RU},
\href{https://alatyr.sk/open-letter-from-professor_sk.htm}{SK},
\href{https://swprs.org/prof-dr-sucharit-bhakdiden-basbakan-dr-angela-merkele-acik-mektup/}{TR}\\
\textbf{Share this letter on}:
\href{https://twitter.com/intent/tweet?url=https://swprs.org/lettera-aperta-del-professor-sucharit-bhakdi-al-cancelliere-tedesco-dr-angela-merkel/}{Twitter}
/
\href{https://www.facebook.com/share.php?u=https://swprs.org/lettera-aperta-del-professor-sucharit-bhakdi-al-cancelliere-tedesco-dr-angela-merkel/}{Facebook}

Lettera aperta del Dr. Sucharit Bhakdi, professore emerito di
Microbiologia medica presso l'Università Johannes Gutenberg di Magonza,
al Cancelliere tedesco Dr. Angela Merkel. Il professor Bhakdi sollecita
una rivalutazione urgente della risposta a Covid-19 e pone al
Cancelliere cinque domande chiave. La lettera è datata 26 marzo 2020.
(\href{https://swprs.org/offener-brief-von-professor-sucharit-bhakdi-an-bundeskanzlerin-dr-angela-merkel/}{PDF
in tedesco})

\hypertarget{lettera-aperta}{%
\subsubsection{Lettera aperta}\label{lettera-aperta}}

Cara cancelliera,

Come emerito dell'Università Johannes Gutenberg di Magonza e direttore
di lunga data del locale Istituto di microbiologia medica e igiene, mi
sento in dovere di giudicare criticamente le restrizioni di vasta
portata della vita pubblica che assumiamo per ridurre la diffusione del
virus COVID-19.

Non è mia esplicita intenzione minimizzare i pericoli della malattia
virale o diffondere un messaggio politico. Tuttavia, ritengo sia mio
dovere fornire un contributo scientifico alla corretta interpretazione
dei dati, mettendo in prospettiva i fatti che conosciamo finora e
ponendo anche domande che rischierebbero di perdersi nell'acceso
dibattito pubblico.

Il motivo principale della mia preoccupazione sono le conseguenze
socioeconomiche realmente imprevedibili delle drastiche misure di
contenimento, introdotte in gran parte dell'Europa già ampiamente
praticate in Germania.

Il mio desiderio è di dibattere in modo critico e con la necessaria
lungimiranza i vantaggi e gli svantaggi delle restrizioni della vita
pubblica e dei conseguenti effetti a lungo termine.

Ci sono cinque domande a cui finora è stata data una risposta
inadeguata, ma che sono essenziali per un'analisi equilibrata.

Con la presente le chiedo una rapida presa di posizione e chiedo
cortesemente al Governo Federale di sviluppare delle strategie che
proteggano efficacemente i gruppi ad alto rischio, senza restringere la
vita pubblica in modo generalizzato e senza piantare i semi per una
polarizzazione della società ancora più intensa di quella che sta già
avvenendo.

Distinti saluti,

\textbf{Prof. em. Dr. med. Sucharit Bhakdi}

\hypertarget{1-statistica}{%
\subparagraph{\texorpdfstring{\textbf{1.
Statistica}}{1. Statistica}}\label{1-statistica}}

Nell'infettivologia -- fondata dallo stesso Robert Koch -- viene fatta
una tradizionale distinzione tra infezione e malattia. Una malattia
richiede manifestazione clinica. {[}1{]} Pertanto, solo i pazienti con
sintomi come febbre o tosse dovrebbero essere inclusi nelle statistiche
come nuovi casi. In altre parole, una nuova infezione, rilevata tramite
un test COVID-19, non significa necessariamente che abbiamo a che fare
con un paziente malato, bisognoso di un letto d'ospedale.

Invece, attualmente, si presume che il cinque percento di tutte le
persone infette si ammali gravemente e richieda una ventilazione. Su
questa base, le proiezioni suggeriscono che il sistema sanitario
potrebbe essere sovraccaricato.

\textbf{Domanda}: le proiezioni distinguono tra pazienti infetti privi
di sintomi e pazienti realmente malati, cioè persone che sviluppano
sintomi?

\hypertarget{2-pericolosituxe0}{%
\subparagraph{\texorpdfstring{\textbf{2.
Pericolosità}}{2. Pericolosità}}\label{2-pericolosituxe0}}

Numerosi virus corona sono in circolazione da molto tempo, in gran parte
inosservati dai media.\\
Se si scoprisse che al virus COVID-19 non dovrebbe essere attribuito un
rischio significativamente più elevato rispetto ai virus corona già
circolanti, tutte le misure di contenimento sarebbero ovviamente
superflue.

La rivista ``International Journal of Antimicrobial Agents'',
riconosciuta a livello internazionale, presto pubblicherà un lavoro che
affronterà questa domanda. I risultati preliminari dello studio possono
già essere visti oggi e portano alla conclusione che il nuovo virus NON
differisce nella sua pericolosità dai tradizionali virus corona. Gli
autori lo spiegano nel titolo del loro lavoro ``SARS-CoV-2: Fear versus
Data''. {[}3{]}

\textbf{Domanda}: qual'è l'attuale tasso di utilizzo delle unità di
terapia intensiva con pazienti diagnosticati con il virus COVID-19
rispetto ad altre infezioni da coronavirus, e in che misura questi dati
sono presi in considerazione nell'ulteriore processo decisionale del
governo federale? Inoltre: lo studio di cui sopra è stato fin'ora
considerato nelle decisioni di pianificazione? Naturalmente, anche in
questo caso deve essere applicato quanto segue: ``diagnosticato''
significa che il virus svolge un ruolo significativo nelle condizioni
del paziente e che malattie precedenti non svolgono un ruolo importante.

\hypertarget{3-diffusione}{%
\subparagraph{\texorpdfstring{\textbf{3.
Diffusione}}{3. Diffusione}}\label{3-diffusione}}

Secondo un rapporto della Süddeutsche Zeitung, nemmeno il tanto citato
Robert Koch Institute sa esattamente l'estensione dei test di COVID-19.
Tuttavia, con l'aumento del volume dei test in Germania, è stato
recentemente osservato un rapido aumento del numero di casi. {[}4{]}

Si sospetta quindi che il virus si sia diffuso in modo inosservato nella
popolazione sana. Ciò avrebbe due conseguenze: in primo luogo,
significherebbe che il tasso di mortalità ufficiale -- il 26 marzo 2020
erano stati rilevati 206 decessi per circa 37.300 infezioni, il 0,55\%
{[}5{]} -- è troppo alto; e in secondo luogo, che non è più possibile
impedire la diffusione del virus nella popolazione sana.

\textbf{Domanda}: è già stata effettuata una rilevazione a campione
sulla popolazione sana per convalidare la reale diffusione del virus o
si prevede di farla in tempi rapidi?

\hypertarget{3-mortalituxe0}{%
\subparagraph{\texorpdfstring{\textbf{3.
Mortalità}}{3. Mortalità}}\label{3-mortalituxe0}}

La paura di un aumento del tasso di mortalità in Germania (attualmente
0,55\%) è attualmente dibattuta in modo particolarmente intenso nei
media. Molte persone temono che, senza misure tempestive, la mortalità
possa aumentare in modo vertigioso come in Italia (10\%) e in Spagna
(7\%).

Allo stesso tempo, in tutto il mondo viene commesso l'errore di riferire
i decessi correlati al virus non appena sia determinato che il virus era
presente al momento della morte, indipendentemente da altri fattori. Ciò
viola un principio basilare dell'infettivologia: la diagnosi può essere
fatta solo se si è sicuri che un agente abbia un ruolo significativo nel
causare la malattia o la morte. L'Associazione delle Società
Scientifiche Mediche scrive espressamente nelle sue linee guida: ``Oltre
alla causa della morte, nel certificato di morte è necessario
specificare una catena causale, con la corrispondente malattia di base
al terzo posto. Occasionalmente, devono essere fornite catene causali
quadrinomiali.'' {[}6{]}

Al momento non ci sono informazioni ufficiali sul fatto che, almeno a
posteriori, siano state intraprese analisi più critiche delle cartelle
cliniche per determinare quanti decessi siano realmente dovuti al virus.

\textbf{Domanda}: la Germania ha semplicemente seguito i sospetti
generalizzati sul COVID-19? E intende continuare questa categorizzazione
in modo acritico, come in altri paesi? Come si dovrebbero distinguere le
morti realmente legate al Covid-19 dalla presenza accidentale del virus
al momento della morte?

\hypertarget{4-comparabilituxe0}{%
\subparagraph{\texorpdfstring{\textbf{4.
Comparabilità}}{4. Comparabilità}}\label{4-comparabilituxe0}}

La terrificante situazione in Italia è utilizzata in continuazione come
scenario di riferimento. Tuttavia, il ruolo effettivo del virus in
questo paese non è completamente chiaro per molte ragioni, non solo
perché anche qui si applicano i sopracitati punti 3 e 4, ma anche perché
esistono fattori esterni eccezionali che rendono queste regioni
particolarmente vulnerabili.

Uno di questi è l'elevatissomo inquinamento atmosferico nel nord Italia.
Secondo le stime dell'OMS, questa condizione ha provocato nel 2006 oltre
8.000 decessi aggiuntivi nelle 13 città più grandi della sola Italia,
anche senza virus. {[}7{]} La situazione non è cambiata
significativamente da allora. {[}8{]} Infine, è stato anche dimostrato
che l'inquinamento atmosferico nelle persone molto giovani e in quelle
anziane aumenta fortemente il rischio di malattie polmonari virali.
{[}9{]}

Inoltre, in questo paese il 27,4\% della popolazione particolarmente
vulnerabile vive con i giovani, in Spagna addirittura il 33,5\%. In
confronto, in Germania è solo il 7\% {[}10{]}. Inoltre, secondo il Prof.
Dr. Reinhard Busse, capo del Dipartimento di Gestione della Sanità
presso la TU di Berlino, la Germania è significativamente più attrezzata
dell'Italia in termini di unità di terapia intensiva, con un fattore di
circa 2,5 {[}11{]}.

\textbf{Domanda}: quali sforzi vengono fatti per evidenziare alla
popolazione queste differenze elementari e per far capire alle persone
che scenari come quelli in Italia o in Spagna non sono realistici?

\hypertarget{riferimenti}{%
\subparagraph{\texorpdfstring{\textbf{Riferimenti:}}{Riferimenti:}}\label{riferimenti}}

{[}1{]} Fachwörterbuch Infektionsschutz und Infektionsepidemiologie.
\href{https://www.rki.de/DE/Content/Service/Publikationen/Fachwoerterbuch_Infektionsschutz.html}{Fachwörter
-- Definitionen -- Interpretationen}. Robert Koch-Institut, Berlin 2015.
(abgerufen am 26.3.2020)

{[}2{]} Killerby et al., Human Coronavirus Circulation in the United
States 2014--2017. J Clin Virol. 2018, 101, 52-56

{[}3{]} Roussel et al. SARS-CoV-2: Fear Versus Data. Int. J. Antimicrob.
Agents 2020, 105947

{[}4{]} Charisius, H.
\href{https://www.sueddeutsche.de/gesundheit/covid-19-coronavirus-testverfahren-1.4855487}{Covid-19:
Wie gut testet Deutschland?} Süddeutsche Zeitung. (abgerufen am
27.3.2020)

{[}5{]} Johns Hopkins University,
\href{https://coronavirus.jhu.edu/map.html}{Coronavirus Resource
Center}. 2020. (abgerufen am 26.3.2020)

{[}6{]} S1-Leitlinie 054-001,
\href{https://www.awmf.org/uploads/tx_szleitlinien/054-002l_S1_Regeln-zur-Durchfuehrung-der-aerztlichen-Leichenschau_2018-02_01.pdf}{Regeln
zur Durchführung der ärztlichen Leichenschau}. AWMF Online (abgerufen am
26.3.2020)

{[}7{]} Martuzzi et al. Health Impact of PM10 and Ozone in 13 Italian
Cities. World Health Organization Regional Office for Europe. WHOLIS
number E88700 2006

{[}8{]} European Environment Agency,
\href{https://www.eea.europa.eu/themes/air/country-fact-sheets/2019-country-fact-sheets}{Air
Pollution Country Fact Sheets 2019}, (abgerufen am 26.3.2020)

{[}9{]} Croft et al. The Association between Respiratory Infection and
Air Pollution in the Setting of Air Quality Policy and Economic Change.
Ann. Am. Thorac. Soc. 2019, 16, 321--330.

{[}10{]} United Nations, Department of Economic and Social Affairs,
Population Division. Living Arrange­ments of Older Persons: A Report on
an Expanded International Dataset (ST/ESA/SER.A/407). 2017

{[}11{]} Deutsches Ärzteblatt,
\href{https://www.aerzteblatt.de/nachrichten/111029/Ueberlastung-deutscher-Krankenhaeuser-durch-COVID-19-laut-Experten-unwahrscheinlich}{Überlastung
deutscher Krankenhäuser durch COVID-19 laut Experten unwahrscheinlich},
(abgerufen am 26.3.2020)

\hypertarget{torna-allarticolo-principale-i-fatti-su-covid-19}{%
\paragraph{\texorpdfstring{Torna all'articolo principale:
\href{https://swprs.org/un-medico-svizzero-su-covid-19/}{I fatti su
Covid-19}}{Torna all'articolo principale: I fatti su Covid-19}}\label{torna-allarticolo-principale-i-fatti-su-covid-19}}

\begin{center}\rule{0.5\linewidth}{\linethickness}\end{center}

\hypertarget{swiss-policy-research}{%
\subsubsection{Swiss Policy Research}\label{swiss-policy-research}}

\begin{itemize}
\tightlist
\item
  \href{https://swprs.org/kontakt/}{Kontakt}
\item
  \href{https://swprs.org/uebersicht/}{Übersicht}
\item
  \href{https://swprs.org/donationen/}{Donationen}
\item
  \href{https://swprs.org/disclaimer/}{Disclaimer}
\end{itemize}

\hypertarget{english}{%
\subsubsection{English}\label{english}}

\begin{itemize}
\tightlist
\item
  \href{https://swprs.org/contact/}{About Us / Contact}
\item
  \href{https://swprs.org/media-navigator/}{The Media Navigator}
\item
  \href{https://swprs.org/the-american-empire-and-its-media/}{The CFR
  and the Media}
\item
  \href{https://swprs.org/donations/}{Donations}
\end{itemize}

\hypertarget{follow-by-email}{%
\subsubsection{Follow by email}\label{follow-by-email}}

Follow

\href{https://wordpress.com/?ref=footer_custom_com}{WordPress.com}.

\protect\hyperlink{}{Up ↑}

Post to

\protect\hyperlink{}{Cancel}

\includegraphics{https://pixel.wp.com/b.gif?v=noscript}
