\protect\hyperlink{content}{Skip to content}

\href{https://swprs.org/}{}

\protect\hyperlink{search-container}{Search}

Search for:

\href{https://swprs.org/}{\includegraphics{https://swprs.files.wordpress.com/2020/05/swiss-policy-research-logo-300.png}}

\href{https://swprs.org/}{Swiss Policy Research}

Geopolitics and Media

Menu

\begin{itemize}
\tightlist
\item
  \href{https://swprs.org}{Start}
\item
  \href{https://swprs.org/srf-propaganda-analyse/}{Studien}

  \begin{itemize}
  \tightlist
  \item
    \href{https://swprs.org/srf-propaganda-analyse/}{SRF / ZDF}
  \item
    \href{https://swprs.org/die-nzz-studie/}{NZZ-Studie}
  \item
    \href{https://swprs.org/der-propaganda-multiplikator/}{Agenturen}
  \item
    \href{https://swprs.org/die-propaganda-matrix/}{Medienmatrix}
  \end{itemize}
\item
  \href{https://swprs.org/medien-navigator/}{Analysen}

  \begin{itemize}
  \tightlist
  \item
    \href{https://swprs.org/medien-navigator/}{Navigator}
  \item
    \href{https://swprs.org/der-propaganda-schluessel/}{Techniken}
  \item
    \href{https://swprs.org/propaganda-in-der-wikipedia/}{Wikipedia}
  \item
    \href{https://swprs.org/logik-imperialer-kriege/}{Kriege}
  \end{itemize}
\item
  \href{https://swprs.org/netzwerk-medien-schweiz/}{Netzwerke}

  \begin{itemize}
  \tightlist
  \item
    \href{https://swprs.org/netzwerk-medien-schweiz/}{Schweiz}
  \item
    \href{https://swprs.org/netzwerk-medien-deutschland/}{Deutschland}
  \item
    \href{https://swprs.org/medien-in-oesterreich/}{Österreich}
  \item
    \href{https://swprs.org/das-american-empire-und-seine-medien/}{USA}
  \end{itemize}
\item
  \href{https://swprs.org/bericht-eines-journalisten/}{Fokus I}

  \begin{itemize}
  \tightlist
  \item
    \href{https://swprs.org/bericht-eines-journalisten/}{Journalistenbericht}
  \item
    \href{https://swprs.org/russische-propaganda/}{Russische Propaganda}
  \item
    \href{https://swprs.org/die-israel-lobby-fakten-und-mythen/}{Die
    »Israel-Lobby«}
  \item
    \href{https://swprs.org/geopolitik-und-paedokriminalitaet/}{Pädokriminalität}
  \end{itemize}
\item
  \href{https://swprs.org/migration-und-medien/}{Fokus II}

  \begin{itemize}
  \tightlist
  \item
    \href{https://swprs.org/covid-19-hinweis-ii/}{Coronavirus}
  \item
    \href{https://swprs.org/die-integrity-initiative/}{Integrity
    Initiative}
  \item
    \href{https://swprs.org/migration-und-medien/}{Migration \& Medien}
  \item
    \href{https://swprs.org/der-fall-magnitsky/}{Magnitsky Act}
  \end{itemize}
\item
  \href{https://swprs.org/kontakt/}{Projekt}

  \begin{itemize}
  \tightlist
  \item
    \href{https://swprs.org/kontakt/}{Kontakt}
  \item
    \href{https://swprs.org/uebersicht/}{Seitenübersicht}
  \item
    \href{https://swprs.org/medienspiegel/}{Medienspiegel}
  \item
    \href{https://swprs.org/donationen/}{Donationen}
  \end{itemize}
\item
  \href{https://swprs.org/contact/}{English}
\end{itemize}

\protect\hyperlink{}{Open Search}

\hypertarget{die-propaganda-matrix}{%
\section{Die Propaganda-Matrix}\label{die-propaganda-matrix}}

Ob Russland, Syrien oder Donald Trump: Um die geopolitische
Bericht­erstattung westlicher Medien zu verstehen, muss man die
Schlüssel­rolle des amerikanischen \emph{Council on Foreign Relations
(CFR)} kennen.

In der folgenden Studie wird erstmals dargestellt, wie der CFR einen in
sich weit­ge­hend geschlossenen, trans­atlantischen
Informations­­kreislauf schuf, in dem nahezu alle relevanten Quellen und
Bezugs­punkte von Mitgliedern des Councils und seiner
Partner­­organisationen kontrolliert werden.

Auf diese Weise entstand eine historisch einzigartige
Informations­­matrix, die klassischer Regierungs­propaganda autoritärer
Staaten deutlich überlegen ist, indes durch den Erfolg unabhängiger
Medien zunehmend an Wirksamkeit verliert.

\href{https://swprs.files.wordpress.com/2018/07/die-propaganda-matrix-spr-hdv.pdf}{Studie
als PDF herunterladen}

\href{https://swprs.files.wordpress.com/2017/09/propaganda-matrix-ts.png}{\includegraphics{https://swprs.files.wordpress.com/2017/09/propaganda-matrix-ts.png?w=500}}

(Hinweis: Bei Interesse an der Studie bitte auf diese Seite verlinken.
Obige Zusammen­fassung und einzelne Auszüge können übernommen werden.
Keine Volltext-Kopie.)

\begin{center}\rule{0.5\linewidth}{\linethickness}\end{center}

\hypertarget{die-propaganda-matrix-1}{%
\subsection{Die Propaganda-Matrix:}\label{die-propaganda-matrix-1}}

\hypertarget{wie-der-cfr-den-geostrategischen}{%
\subsection{Wie der CFR den
geostrategischen}\label{wie-der-cfr-den-geostrategischen}}

Informationsfluss kontrolliert

\emph{Eine Studie von \href{https://swprs.org/}{Swiss Propaganda
Research}}

September 2017

\emph{»Wir sind jetzt ein Imperium, und wenn wir handeln,\\
so erschaffen wir unsere eigene Realität.«}\\
---\\
\href{https://en.wikiquote.org/wiki/Karl_Rove}{Karl Rove}, ehemaliger
Leiter des Büros für\\
Strategische Initiativen der US-Regierung

Inhaltsübersicht

\begin{enumerate}
\def\labelenumi{\arabic{enumi}.}
\tightlist
\item
  \protect\hyperlink{k1}{Der Council on Foreign Relations}
\item
  \protect\hyperlink{k2}{Die CFR-Matrix}
\item
  \protect\hyperlink{k3}{Journalisten in der Matrix}
\item
  \protect\hyperlink{k4}{Fazit}
\item
  \protect\hyperlink{k5}{Literatur}
\end{enumerate}

\hypertarget{1-der-council-on-foreign-relations}{%
\subsubsection{1. Der Council on Foreign
Relations}\label{1-der-council-on-foreign-relations}}

Der Ursprung des \emph{Council on Foreign Relations} liegt im
sogenannten
\href{http://magazin.spiegel.de/EpubDelivery/spiegel/pdf/41389590}{»Trauma
von 1920«}: Nach dem Ersten Weltkrieg hätten die USA erstmals die
globale Führungsrolle übernehmen können -- doch der Senat entschied sich
gegen den Beitritt zum Völkerbund und die kriegsmüde Bevölkerung wählte
mit Warren Harding einen Präsidenten, der eine
\href{https://en.wikipedia.org/wiki/Return_to_normalcy}{»Rückkehr zur
Normalität«} versprach und sich zuerst um die Angelegen­heiten und
Probleme Amerikas und der Amerikaner kümmern wollte.

Um einen solchen Rückschlag künftig zu vermeiden und
\href{http://www.nytimes.com/1971/11/21/archives/is-it-a-club-seminar-presidium-invisible-government-the-council-on-.html}{»Amerika
für seine weltweiten Pflichten zu erwecken«}, gründeten international
orientierte Bankiers, Unternehmer und Politiker im Folgejahr in der
Finanz- und Handelsmetropole New York den parteiübergreifenden CFR.
Durch die Mitarbeit führender Akademiker und Publizisten, darunter
\href{https://en.wikipedia.org/wiki/Archibald_Cary_Coolidge}{Archibald
Coolidge} (\emph{The United States as a World Power,} 1908) und
\href{https://en.wikipedia.org/wiki/Walter_Lippmann}{Walter Lippmann}
(\emph{Public Opinion,} 1922), sollten Ideen für eine aktive
Außen­politik entwickelt und in der Öffentlichkeit beliebt gemacht
werden.

Der Durchbruch gelang dem Council während des Zweiten Weltkriegs, als
CFR-Experten im Rahmen der
\href{https://swprs.files.wordpress.com/2017/09/cfr_imperial_brain_trust.pdf\#page=129}{\emph{War
and Peace Studies}} die amerikanische Kriegsstrategie sowie die
Grundsätze der Nachkriegsordnung formulierten -- inklusive der
\href{https://swprs.files.wordpress.com/2017/09/domhoff-cfr-2014.pdf}{Satzungen}
von UNO, Weltbank und Weltwährungsfonds. Dabei folgten sie der Vorgabe
von CFR-Gründungs­direktor Isaiah Bowman, wonach die USA künftig die
»globale Sicherheit garantieren« müssten, dabei jedoch »konventionelle
Formen des Imperialismus« zu vermeiden hätten, weswegen der Ausübung
amerikanischer Macht ein »internationaler Charakter« zu verleihen sei
(\href{https://swprs.files.wordpress.com/2017/09/cfr_imperial_brain_trust.pdf\#page=181}{Shoup
\& Minter}, 1977:169ff).

Auf diese Weise entstand -- nur 170 Jahre nach der
Unabhängigkeits­erklärung -- ein globales
\href{http://carnegieendowment.org/1998/06/01/benevolent-empire-pub-275}{\emph{American
Empire}}, dessen Schlüssel­positionen seitdem nahezu durchgehend von den
inzwischen knapp 5000 Vertretern des CFR besetzt wurden (siehe folgende
Abbildung sowie
\href{https://swprs.files.wordpress.com/2017/07/cfr-administration-members-1900-2014.pdf}{Auflistung
nach Administration}). Das Nachrichtenmagazin \emph{Der Spiegel}
\href{http://www.spiegel.de/spiegel/print/d-41389590.html}{bezeichnete}
den Council deshalb einst als die \emph{„einfluss­reichste private
Institution Amerikas und der westlichen Welt``} und als ein
\emph{„Politbüro für den Kapitalismus``}.

\href{https://swprs.files.wordpress.com/2018/02/cfr-imperial-council-hdv.png}{\includegraphics{https://swprs.files.wordpress.com/2018/02/cfr-imperial-council-hdv.png?w=736}}*1945
bis 2017: CFR-Mitglieder in den Schlüsselpositionen des American
Empire\\
\href{https://swprs.files.wordpress.com/2018/02/cfr-imperial-council-hdv.png}{Vergrößern}*🔎

Mit dem Zweiten Weltkrieg erweiterte sich das amerikanische
Einflussgebiet erstmals auf (West-)Europa und Ostasien (insbesondere
Japan). Um in diesen Regionen lokale Eliten aufzubauen und in die
eigenen Planungen miteinzubeziehen, musste der Council sein Netzwerk
ergänzen: Für Europa lancierte CFR-Mitglied
\href{https://en.wikipedia.org/wiki/Charles_Douglas_Jackson}{Charles D.
Jackson}, Eisenhowers Assistent für psychologische Kriegsführung, 1954
die sogenannte
\href{https://en.wikipedia.org/wiki/Bilderberg_Group}{Bilderberg-Gruppe},
während für Ostasien von CFR-Präsident David Rockefeller und
CFR-Direktor Zbigniew Brzezinski 1972 zusätzlich die
\href{https://en.wikipedia.org/wiki/Trilateral_Commission}{Trilaterale
Kommission} gegründet wurde.

Beide Organisationen haben zum Ziel, die zentralen geostrategischen
Heraus­forderungen zu erörtern und einen länder- und
partei­über­greifenden Konsens zu entwickeln. Der ehemalige französische
Premier­­minister (und Bilderberg-Teilnehmer) François Fillon dürfte
insofern nicht unrecht gehabt haben, als er 2013
\href{https://swprs.org/video-bilderberg-gruppe/}{konstatierte}: „Es
sind die Bilderberger, die uns regieren.``

\hypertarget{2-die-cfr-matrix}{%
\subsubsection{2. Die CFR-Matrix}\label{2-die-cfr-matrix}}

Die erfolgreiche Umsetzung einer geopolitischen Strategie -- in
Friedens- und insbesondere in Kriegszeiten -- wäre undenkbar ohne die
wirkungsvolle Beeinflussung der öffentlichen Meinung. Autoritäre Staaten
sind hierfür meist auf direkte Regierungs­propaganda angewiesen, die
indes oft rasch an Glaub­würdigkeit verliert.

Der Council ging dies klüger an: Mit seinen inzwischen knapp 5000
Mitgliedern baute er ein scheinbar vielfältiges und unabhängiges
Informations­system auf, in dem jedoch nahezu alle relevanten Quellen
und Bezugspunkte von Mitgliedern des CFR und seiner
Partner­organisationen kontrolliert werden. Auf diese Weise entstand
eine historisch einzigartige »Propaganda-Matrix«, deren Elemente und
Funktionsweise im Folgenden dargestellt werden.

\href{https://swprs.files.wordpress.com/2017/09/propaganda-matrix-ts.png}{\includegraphics{https://swprs.files.wordpress.com/2017/09/propaganda-matrix-ts.png?w=480\&h=775}}

\begin{quote}
»Die Bedeutung des CFR ist nicht leicht zu übertreiben.~ Es ist die
wichtigste nichtstaatliche außenpolitische Organisation der USA. Seine
zentrale Rolle besteht darin, die akzeptierten, legitimen und orthodoxen
Parameter der Diskussion über die US-Außenpolitik und damit
zusammenhängende Fragen zu definieren. () Der CFR entspricht also dem,
was die Sowjets die oberste Ebene der Nomenklatura nannten.«

Princeton-Professor und ehem. CFR-Mitglied Stephen F. Cohen, \emph{The
Nation},
\href{https://www.thenation.com/article/the-american-bipartisan-policy-establishment-declares-its-second-cold-war-vs-russia-after-years-of-denying-it/}{2018}
\end{quote}

\hypertarget{eingebettete-medien}{%
\paragraph{Eingebettete Medien}\label{eingebettete-medien}}

Ob Zeitungen, Magazine, Rundfunk oder Internet: Der \emph{Council on
Foreign Relations} war stets darauf bedacht, Eigentümer, Chef­redakteure
und Top-Journalisten der führenden Medien in seine Strukturen zu
integrieren.

In den \textbf{USA} wurden tatsächlich nahezu alle bekannten Medien von
CFR-Vertretern gegründet oder bereits vor Jahrzehnten aufgekauft (siehe
Abbildung unten). Dies war möglich, weil für den Betrieb eines
einflussreichen Mediums bislang erhebliche finanzielle Mittel sowie
Zugang zu politischen Entscheidungsträgern erforderlich waren -- und
über beides verfügt der Council und seine Mitglieder wie kaum eine
andere Gruppierung. Selbst moderne Internet­unternehmen wie Google und
Facebook sind auf höchster Ebene in das Netzwerk des Councils
eingebunden -- und bisweilen auch an dessen
\href{https://www.washingtonexaminer.com/clinton-email-reveals-google-sought-overthrow-of-syrias-assad}{internationalen
Operationen} beteiligt.

Die traditionellen Medien in \textbf{(West-)Deutschland} wurden nach dem
Krieg in einem
\href{https://de.wikipedia.org/wiki/Lizenzzeitung}{alliierten
Lizenzverfahren} gegründet und mit sorgfältig ausgewählten Verlegern und
Chef­redakteuren besetzt -- Strukturen, die sich über
verwandtschaftliche und andere Beziehungen bis heute erhalten haben.
Nebst der Bilderberg-Gruppe und der Trilateralen Kommission erfolgt die
Einbindung und Sozialisierung der führenden deutschen Medienleute dabei
insbesondere über die sogenannte
\href{https://de.wikipedia.org/wiki/Atlantik-Br\%C3\%BCcke}{Atlantik-Brücke},
die 1952 von CFR- und Weltbank-Präsident sowie Hochkommissar für
Deutschland, \href{https://de.wikipedia.org/wiki/John_Jay_McCloy}{John
J. McCloy}, zusammen mit CFR-Mitglied und Bankier
\href{https://de.wikipedia.org/wiki/Eric_M._Warburg}{Eric Warburg} --
dem Enkel von CFR-Direktor und \emph{Federal Reserve} Initiant Paul
Warburg -- gegründet wurde.

Auch die offiziell neutrale \textbf{Schweiz} ist seit dem Zweiten
Weltkrieg in die transatlantische Wirtschafts- und
Sicherheits­architektur integriert und hat in hohem Maße von ihr
profitiert. Deshalb liegt eine von transatlantischen Standards
abweichende, allzu kritische Medien­bericht­erstattung -- die rasch als
\href{https://swprs.org/russische-propaganda/}{»feindliche Propaganda«}
gewertet und zu unerwünschten politischen oder ökonomischen
Komplikationen führen könnte -- nicht im Landesinteresse.

In geopolitischen bzw. imperialen Angelegenheiten berichten mithin auch
die etablierten Schweizer Medien weitgehend
\href{https://swprs.org/medien-navigator/}{CFR- und NATO-konform}.
Begünstigt wird diese Konformität durch die zunehmende
Medien­konzentration, die dazu führte, dass inzwischen
\href{http://www.foeg.uzh.ch/jahrbuch.html}{über 90\%} des Schweizer
Marktes von nur fünf Medien­häusern kontrolliert werden. Die
strukturelle Einbindung dieser Verlage erfolgt dabei primär über die
Bilderberg-Gruppe sowie über zunehmend enge
\href{https://swprs.org/tagesanzeiger/}{Kooperationen} mit deutschen
Atlantikbrücke-Medien.

\includegraphics{https://swprs.files.wordpress.com/2016/02/srf-syrien11.png?w=500\&h=331}\emph{NATO-konform:
Die Tagesschau des Schweizer Fernsehens SRF}

In den folgenden Grafiken werden die soeben beschriebenen
Mediennetzwerke in den USA, Deutschland und der Schweiz anhand der
offiziellen Mitglieder- und Teilnehmerlisten (siehe Anhang) erstmals
grafisch dargestellt. Wie ersichtlich umfassen sie im Wesentlichen
sämtliche sogenannten »Mainstream-Medien«. Diese zugleich abwertende wie
anmaßende Bezeichnung kann insofern als eine Umschreibung für
CFR-konforme Publikationen begriffen werden.

\href{https://swprs.files.wordpress.com/2017/08/cfr-media-network-hdv-spr.png}{\includegraphics{https://swprs.files.wordpress.com/2017/08/cfr-media-network-hdv-spr.png?w=736\&h=526}\emph{Vergrößern}
🔎}

\href{https://swprs.files.wordpress.com/2019/10/medien-netzwerk-schweiz-hdz.png}{}

\includegraphics{https://swprs.files.wordpress.com/2019/10/medien-netzwerk-schweiz-hdz.png?w=329\&h=255}

Schweizer Medien: Das Transatlantik-Netzwerk

\href{https://swprs.files.wordpress.com/2017/08/netzwerk-medien-deutschland-spr-mt.png}{}

\includegraphics{https://swprs.files.wordpress.com/2017/08/netzwerk-medien-deutschland-spr-mt.png?w=399\&h=255}

Medien in Deutschland: Das Transatlantik-Netzwerk

Obige Medien -- sowie einige weitere, kleinere Publikationen -- bilden
den inneren Ring der Informationsmatrix. Sie suggerieren der Bevölkerung
eine scheinbare Informationsvielfalt, vermitteln ihr in Wirklichkeit
jedoch eine weitgehend homogene und CFR-konforme Sichtweise auf das
Welt­geschehen. Hierfür steht den Medien ein umfangreiches
\href{https://swprs.org/der-propaganda-schluessel/}{Instrumentarium} mit
über zwei Dutzend ver­schie­denen Methoden zur Verfügung, die von einer
tendenziösen Sprache über die selektive Themen­wahl und systematische
Ausblendung von Kontext bis hin zur gelegentlichen Falschbehauptung
reichen.

\begin{quote}
»Die Mitgliedschaft dieser Journalisten im Council, was auch immer sie
von sich selbst denken mögen, ist eine Bestätigung ihrer aktiven und
wichtigen Rolle in öffentlichen Angelegenheiten und ihres Aufstiegs in
die herrschende Klasse Amerikas. Sie analysieren und interpretieren die
Außenpolitik der USA nicht nur; sie helfen sie zu machen. () Sie sind
ein Teil des Establishments, ob sie es wollen oder nicht, und sie teilen
die meisten seiner Werte und Ansichten.«

\href{https://www.washingtonpost.com/archive/opinions/1993/10/30/ruling-class-journalists/761e7bf8-025d-474e-81cb-92dcf271571e/}{Richard
Harwood}, ehemaliger leitender Redakteur\\
und Ombudsmann der \emph{Washington Post}
\end{quote}

Um die langfristige Kohärenz dieser Medienmatrix sicherzustellen, ist
jedoch zusätzlich ein äußerer Ring erforderlich, der die Medien mit
geeigneten Informationen, Sichtweisen und Deutungs­mustern versorgt.
Dieser äußere Ring besteht aus trans­atlantischen Regierungen, Militärs,
Geheim­diensten, NGOs, Denkfabriken und Experten sowie Nachrichten- und
PR-Agenturen, die ihrerseits allesamt in das weitverzweigte Netzwerk des
CFR eingebunden sind, wie die folgenden Abschnitte zeigen.

\hypertarget{nichtregierungsorganisationen-ngos}{%
\paragraph{Nichtregierungsorganisationen
(NGOs)}\label{nichtregierungsorganisationen-ngos}}

Während Propaganda in autoritären Staaten zumeist direkt von der
Regierung ausgeht (und entsprechend einfach zu durchschauen ist),
spielen in der CFR-Matrix die sogenannten
\textbf{Nicht­­regierungs­­organisationen (NGOs)} eine besondere Rolle,
da sie der Bevölkerung eine Regierungsferne und mithin eine größere
Unabhängigkeit und Glaubwürdigkeit suggerieren.

Tatsächlich sind die
\href{http://foreignpolicy.com/2011/11/17/names-nossel-to-head-amnesty-international-usa/}{Direktoren}
von \emph{Amnesty International (AI)}, \emph{Human Rights Watch (HRW)}
und vieler weiterer vordergründig humanitärer Organisationen jedoch seit
Jahrzehnten in den Council
\href{https://swprs.files.wordpress.com/2018/02/cfr-imperial-council-hdv.png}{ein­ge­bunden},
während zahlreiche andere von CFR-Milliardären wie George Soros
finanziert und gelenkt werden. Letzterer betreibt dabei durchaus keine
eigen­ständige Außen­politik, sondern
\href{http://www.nybooks.com/articles/2015/10/08/ukraine-europe-what-should-be-done/}{unter­stützt}
ledig­lich die inter­na­tionalen Opera­tionen des Councils im Rahmen
seiner Möglichkeiten.

Während diese NGOs unterm Jahr bisweilen durchaus sinnvolle, indes
überwiegend folgenlose Arbeit leisten (z.B. Berichte zur
inter­nationalen Menschen­rechts­lage verfassen), kommt ihre
geopolitische Funktion immer dann zum Einsatz, wenn es gilt, einen
\emph{Regime Change} vorzubereiten oder eine~Militärintervention
humanitär zu legitimieren.

So
\href{https://en.wikipedia.org/wiki/Nayirah_(testimony)}{»verifizierte«}
\emph{Amnesty International} bereits 1991 öffentlich die von einer
amerikanischen PR-Firma erfundene
\href{https://de.wikipedia.org/wiki/Brutkastenl\%C3\%BCge}{»Brutkastenlüge«}
und trug damit wesentlich zur Lancierung des Golfkriegs bei. Auch auf
dem Balkan, in Afghanistan
(\emph{\href{https://consortiumnews.com/2012/06/18/amnestys-shilling-for-us-wars/}{„NATO:
keep the progress going!``}}) und Libyen forderten AI und HRW auf Basis
fragwürdiger bis falscher
\href{https://journal-neo.org/2014/08/01/hrw-human-rights-watch-or-hypocrites-representing-washington-part-1/}{Behauptungen}
»humanitäre« Militärinterventionen.

Im Syrienkrieg hatte \emph{Human Rights Watch} nach dem Giftgasangriff
vom Sommer 2013 alsbald ein
\href{https://www.hrw.org/news/2013/09/10/syria-government-likely-culprit-chemical-attack}{Gutachten}
zur Hand, welches die Täterschaft der syrischen Regierung belegen und
damit eine NATO-Intervention begründen sollte. In einer späteren Analyse
von MIT-Forschern stellte sich das Gutachten indes als
\href{http://blauerbote.com/2017/04/01/giftgasangriffe-in-ghouta-bei-damaskus-syrien/}{fabriziert}
heraus, doch für CFR-Medien dürfte dies auch künftig kein Grund zur
Skepsis sein.

\includegraphics{https://swprs.files.wordpress.com/2017/09/hrw-dw.png?w=500}\emph{HRW-Direktor
und CFR-Mitglied Kenneth Roth auf der Deutschen Welle}

Im ostafrikanischen Eritrea, das sich den amerikanischen
Hegemonial­ansprüchen seit seiner Unabhängigkeit von Äthiopien 1993
\href{http://www.globalresearch.ca/u-s-sets-stage-for-libya-like-regime-change-in-eritrea-africas-cuba-2/5531735}{widersetzt}
hat, wurden \emph{Amnesty International} und \emph{Human Rights Watch}
2011 sogar auf frischer Tat bei einer \emph{Regime-Change-Operation}
\href{https://www.tesfanews.net/hillary-amnesty-hrw-and-regime-change-in-africa/}{ertappt}:
In geheimer Mission infiltrierten teils als Nonnen getarnte Mitarbeiter
das Land, um ein verdecktes Netzwerk aufzubauen, das später auf Kommando
landesweite Proteste auslösen sollte. In einem abgefangenen
\href{https://www.tesfanews.net/hillary-amnesty-hrw-and-regime-change-in-africa/}{Schreiben}
der Amnesty-Direktorin für »Spezial­programme in Afrika« heißt es:
„Unser Ziel ist es, dass das Regime von Issayas Afewerky bis Ende des
Jahres ins Wanken gerät und gestürzt werden kann.``

Neben den permanenten NGOs wie \emph{Amnesty} und \emph{HRW} gründen und
\href{http://www.hintergrund.de/201611244160/globales/kriege/weisse-helme-ohne-weisse-westen.html}{finanzieren}
CFR-geführte Institutionen wie USAID und NED für einzelne Konflikte bei
Bedarf zusätzlich temporäre Organisationen, die lokale Aufgaben
übernehmen und sich nahtlos in die Matrix einfügen lassen. Im
Syrienkrieg
\href{http://www.globalresearch.ca/war-propaganda-and-the-aleppo-media-centre-funded-by-french-foreign-office-eu-and-us/5546925}{entstanden}
auf diese Weise etwa die \emph{Syrische Beobachtungs­stelle für
Menschen­rechte,} das \emph{Aleppo Media Center} oder die berüchtigten
\emph{Weißhelme}, die die westlichen Agenturen und Medien mit
dramatischen und nicht immer über alle Zweifel erhabenen Bildern und
Informationen
\href{http://21stcenturywire.com/2015/10/23/syrias-white-helmets-war-by-way-of-deception-part-1/}{versorgten}.

\includegraphics{https://swprs.files.wordpress.com/2016/10/srf-white-helmets.png?w=500\&h=281}\\
\emph{Ein Mitarbeiter der Weißhelme zeigt auf den UNO-Hilfskonvoi, der
am 20. September 2016 in der Nähe von Aleppo unter ungeklärten Umständen
ausbrannte, und beschuldigt Russland und Syrien.}

Selbstverständlich gibt es auch zahlreiche aufrichtige und unabhängige
NGOs, die sich ernsthaft für den Frieden und die Menschen­rechte
engagieren. Nur sind diese zumeist mit wesentlich weniger Mitteln
ausgestattet, und kommen in CFR-Medien kaum je zu Wort -- insbesondere
nicht in geostrategisch entscheidenden Momenten.

\textbf{Box 1: Der Friedensnobelpreis.} Eine besondere Rolle in der
Definition von »Gut« und »Böse« spielt der sogenannte
Friedens­nobel­preis. Dieser wird als einziger der Nobelpreise nicht von
der Akademie der Wissenschaften des neutralen Schweden verliehen,
sondern von einer
\href{https://de.wikipedia.org/wiki/Friedensnobelpreis}{Kommission
ehemaliger Politiker} des NATO-Gründungs­mitglieds Norwegen. Der
Friedens­nobel­preis wird deshalb im Allgemeinen nicht für die Wahrung
des Friedens an sich, sondern für die Wahrung des amerikanischen
Friedens -- der \emph{Pax Americana} -- vergeben. Persönlich­keiten, die
sich beispielsweise gegen völker­rechts­widrige NATO-Interventionen
engagiert haben, sucht man auf der
\href{https://de.wikipedia.org/wiki/Liste_der_Friedensnobelpreistr\%C3\%A4ger}{Liste
der Preisträger} daher vergeblich. Dafür findet man dort CFR-Vertreter
von Kissinger bis
\href{https://swprs.org/das-american-empire-und-seine-medien/\#wh}{Obama}
und ihre Gehilfen in Ländern von Burma über Tunesien und Jemen bis zur
EU.\\
\includegraphics{https://swprs.files.wordpress.com/2017/09/burma-ned-award.png?w=450\&h=419}\emph{Die
Friedensnobelpreisträgerin und Präsidentin Burmas, Aung San Suu Kyi,
erhält den
\href{http://www.prnewswire.com/news-releases/aung-san-suu-kyi-to-address-ned-2012-democracy-award-in-us-capitol-167734515.html}{NED
Democracy Award 2012} im US-Kapitol. Links NED-Präsident und
Council-Mitglied Carl Gershman, rechts die frühere US-Außenministerin
und Council-Direktorin Madeleine Albright. Burma ist Teil der
\href{https://journal-neo.org/2017/09/08/us-to-fight-us-saudi-sponsored-terrorism-in-asia/}{US-Einkreisungsstrategie}
gegenüber China.}

\hypertarget{denkfabriken-und-experten}{%
\paragraph{Denkfabriken und Experten}\label{denkfabriken-und-experten}}

Eine weitere wichtige Funktion in der CFR-Matrix nehmen die sogenannten
Denkfabriken \emph{(Think Tanks)} und Experten wahr. Diese versorgen die
Medien und die Öffentlichkeit mit scheinbar fundierten und objektiven
Einschätzungen und Analysen. Tatsächlich sind jedoch nahezu alle
Experten, die in CFR-konformen Medien zu Wort kommen, ihrerseits in das
transatlantische Netzwerk des Councils integriert -- wobei dies dem
Publikum zumeist nicht mitgeteilt wird.

In den USA betrifft dies etwa die \emph{Brookings Institution}, die
\emph{RAND Corporation}, den NATO-nahen \emph{Atlantic Council}, das
\emph{Aspen Institute} oder das \emph{Center for Strategic and
International Studies (CSIS),} die allesamt von CFR-Kadern geführt
werden. Auch der Gründer des »investigativen Journalisten-Kollektivs«
\href{https://en.wikipedia.org/wiki/Bellingcat}{Bellingcat} -- das
CFR-Medien in der Ukraine-Krise und im Syrienkrieg mit einschlägigen
Analysen belieferte -- tauchte alsbald als
\href{https://en.wikipedia.org/wiki/Eliot_Higgins}{\emph{Senior
Non-Resident Fellow}} beim \emph{Atlantic Council} auf.

\includegraphics{https://swprs.files.wordpress.com/2017/09/higgins-slomka-preis.jpg?w=500\&h=491}\\
\emph{Bellingcat-Gründer Eliot Higgins und ZDF-Moderatorin Marietta
Slomka erhalten 2015 den Hanns-Joachim-Friedrichs-Preis für
herausragenden Fernsehjournalismus.
(\href{https://presse.wdr.de/plounge/wdr/unternehmen/2015/10/20151028_hanns_joachim_friedrichs_preis.html}{WDR})}

Hinzu kommen Dutzende von Politik-, Wirtschafts- und
Geschichts­professoren sowie die Präsidenten der meisten amerikanischen
Elite­universitäten, die als CFR-Mitglieder für einen konformen
Forschungs- und Lehrbetrieb sorgen und den Medien als Experten zur
Verfügung stehen (siehe erste Abbildung).

In Deutschland zählen zu den in CFR-Medien gefragten Denkfabriken
insbesondere die \href{https://de.wikipedia.org/wiki/DGAP}{Deutsche
Gesellschaft für Auswärtige Politik (DGAP)} -- die 1955 vom CFR
mitgegründet und vom ehemaligen Atlantikbrücke-Chef Arendt Oetker
präsidiert wird -- sowie die von einem BND-Geheimdienstler auf Anraten
von CFR-Direktor Kissinger gegründete
\href{https://lobbypedia.de/wiki/Stiftung_Wissenschaft_und_Politik}{Stiftung
Wissenschaft und Politik (SWP)}\emph{.} Die SWP wird hauptsächlich von
der deutschen Bundes­regierung finanziert und von Volker Perthes
geleitet, der gleichzeitig
\href{https://spiegelkabinett-blog.blogspot.com/2014/05/stiftung-fur-wissenschaft-und-politik.html}{Mitglied}
in der Atlantikbrücke, der Trilateralen Kommission, der
Bilderberg-Gruppe und der DGAP ist und damit zu den führenden
Transatlantikern Deutschlands zählt.

\includegraphics{https://swprs.files.wordpress.com/2017/08/perthes-swp-ard.png?w=500}\emph{SWP-Direktor
Volker Perthes in den ARD-Tagesthemen (ARD)}

Die SWP ist indes nicht nur eine Denkfabrik, sondern auch ein
Planungsbüro: So organisierte sie 2012 in Berlin zusammen mit dem
\href{https://en.wikipedia.org/wiki/United_States_Institute_of_Peace}{»US
Institute of Peace«} -- das vom ehemaligen US-Sicherheitsberater und
CFR-Mitglied Stephen Hadley geleitet wird -- eine Serie von Workshops
mit syrischen Oppositionellen und Rebellen, um die Zeit nach dem
anvisierten Regierungssturz zu planen
(\href{http://www.zeit.de/2012/31/Syrien-Bundesregierung}{Projekt »Day
After«}).

In der Schweiz gibt es mit Ausnahme des
\href{http://www.css.ethz.ch/}{\emph{ETH Center for Security Studies}}
kaum noch nennenswerte geopolitische Institute. Das \emph{Schweizer
Fernsehen} und Zeitungen wie die \emph{NZZ} greifen für ihre Interviews
und Gastbeiträge deshalb ebenfalls gerne auf
\href{https://www.srf.ch/news/international/damaskus-ist-es-voellig-egal-wenn-viele-zivilisten-sterben}{SWP-Experten}
und andere deutsche Trans­atlantiker zurück -- wobei deren einschlägige
Verbindungen üblicherweise nicht offengelegt werden.

Unabhängige Experten -- an Fachwissen ihren transatlantischen Kollegen
nicht selten überlegen -- haben in CFR-Medien hingegen einen schweren
Stand: Die meisten von ihnen werden schlicht ignoriert, während
besonders kritische Köpfe sogar mit Diffamierungs­kampagnen rechnen
müssen, wie sie zuletzt der deutsche Islam­wissen­schaftler und
Syrienkenner \href{http://www.nachdenkseiten.de/?p=37845}{Michael
Lueders} oder der Schweizer Historiker und Experte für verdeckte
Kriegsführung
\href{https://swprs.org/anschlag-auf-die-forschungsfreiheit/}{Daniele
Ganser} erlebten.

\hypertarget{milituxe4r}
aller höheren US-Militärs vom CFR ausgebildet wurden -- inklusive nahezu
aller Generalstabschefs, NATO-Oberbefehlshaber und Gebietskommandeure
seit dem Zweiten Weltkrieg (siehe erste Abbildung und
\href{https://swprs.files.wordpress.com/2017/07/cfr-administration-members-1900-2014.pdf}{Auflistung
pro Administration}). Auf diese Weise hat sich der Council eine
ideologisch geschulte, imperiale Streitkraft aufgebaut, wie man dies
sonst fast nur von totalitären Regimen her kennt.

Ulrich Tilgner, der langjährige Nahost-Korrespondent des \emph{ZDF} und
\emph{Schweizer Fernsehens,}
\href{https://www.amazon.de/inszenierte-Krieg-T\%C3\%A4uschung-Wahrheit-Husseins/dp/3871344923}{beschrieb}
die Interaktion zwischen Medien und Militär im Rückblick auf den
Irak-Krieg von 2003 wie folgt:

``Mit Hilfe der Medien bestimmen die Militärs die öffentliche
Wahrnehmung und nutzen sie für ihre Planungen. Sie schaffen es,
Erwartungen zu wecken und Szenarien und Täuschungen zu verbreiten. In
dieser neuen Art von Krieg erfüllen die PR-Strategen der
US-Administration eine ähnliche Funktion wie sonst die Bomberpiloten.
Die Spezial-Abteilungen für Öffentlich­keits­­arbeit im Pentagon und in
den Geheim­diensten sind zu Kombattanten im Informationskrieg geworden.
()\\
Dabei nutzen die amerikanischen Militärs die mangelnde Transparenz der
Bericht­erstattung in den Medien gezielt für ihre Täuschungs­­manöver.
Die von ihnen gestreuten Informationen, die von Zeitungen und Rundfunk
aufgenommen und verbreitet werden, können Leser, Zuhörer oder Zuschauer
unmöglich bis zur Quelle zurückverfolgen. Somit gelingt es ihnen nicht,
die ursprüngliche Absicht der Militärs zu erkennen. \ldots{}
Journalisten werden so als Mittel genutzt, den Kriegsgegner in die Irre
zu führen. Information wird zum Bestandteil der Kriegsführung: zum
Informationskrieg.`` (Tilgner,
\emph{\href{https://www.amazon.de/inszenierte-Krieg-T\%C3\%A4uschung-Wahrheit-Husseins/dp/3871344923}{Der
inszenierte Krieg},} 2003/2007, S. 132ff)

Tilgners Einschätzung wurde von Tom Curley, dem ehemaligen Chef der
amerikanischen Nachrichten­agentur \emph{Associated Press}, bestätigt.
Curley machte in einem Vortrag von 2009 publik, dass allein das Pentagon
\href{http://www.tagesanzeiger.ch/ausland/amerika/27000-PRBerater-polieren-Image-der-USA/story/20404513}{27'000
PR-Spezialisten} beschäftigt, die mit einem jährlichen Budget von fast 5
Milliarden Dollar Propaganda und Desinformation produzieren. Zudem
hätten hohe US-Generäle gedroht, dass man die AP und ihn »ruinieren«
werde, falls die Reporter allzu kritisch über das US-Militär berichten
sollten. Dennoch -- oder deswegen -- übernehmen CFR-Medien die
Verlautbarungen des US- und NATO-Militärs zumeist gänzlich unkritisch.

Die Symbiose zwischen Militär und Medien reicht somit weit über die
berüchtigten »eingebetteten Journalisten« hinaus. Unabhängige
Investigativ-Journalisten haben hingegen einen schweren Stand: Sie
werden von NATO-Mitgliedern gemäß Wikileaks-Dokumenten als eines der
größten Sicherheits­risiken
\href{https://youtu.be/WCWdwlkx-rE?t=1h18m35s}{eingestuft} -- und
entsprechend
\href{http://www.zerohedge.com/news/2017-08-28/journalist-interrogated-fired-story-linking-cia-and-syria-weapons-flights}{behandelt}.

\includegraphics{https://swprs.files.wordpress.com/2017/08/cfr-military.jpg?w=550\&h=491}\\
*Mitglieder des US-Generalstabs auf einem CFR-Podium,
\href{https://commons.wikimedia.org/wiki/File:From_left,_the_35th_Commandant_of_the_Marine_Corps,_Gen._James_F._Amos;_Army_Chief_of_Staff_Gen._Raymond_T._Odierno;_Air_Force_Chief_of_Staff_Gen._Mark_A._Welsh_III;_Commandant_of_the_Coast_Guard,_Adm._Robert_J_130508-M-LU710-167.jpg}{2013}.\\
*

\hypertarget{geheimdienste}{%
\paragraph{Geheimdienste}\label{geheimdienste}}

Seit dem Zweiten Weltkrieg waren nahezu alle CIA-Direktoren Mitglieder
des Councils. Bereits die Vorgänger­organisation der CIA, das
\emph{\href{https://en.wikipedia.org/wiki/Office_of_Strategic_Services}{Office
of Strategic Services (OSS)},} wurde von den CFR-Mitgliedern Allen
Dulles und William J. Donovan gegründet und geleitet. Insofern dürfte
die CIA eher als ein verdeckt operierender Arm des Councils zu sehen
sein, und weniger als ein klassischer, ausschließlich dem US-Präsidenten
unterstellter Geheimdienst.

\includegraphics{https://swprs.files.wordpress.com/2017/09/dulles-kennedy.png?w=736}\\
\emph{CIA- und CFR-Direktor Allen Dulles (links), der nach der
gescheiterten Kuba-Invasion von Kennedy entlassen wurde und später die
Untersuchungs­kommission zum Mord an Kennedy mitleitete.}

Damit erscheint auch die bekannte \emph{\textbf{Operation Mockingbird}}
in einem etwas anderen Licht. Mitte der 70er Jahre wurde publik, dass
die CIA in nahezu allen US-Medien
\href{https://en.wikipedia.org/wiki/Operation_Mockingbird}{über
Konfidenten verfügte} und diese mit Information bzw. Desinformation
belieferte. Allerdings waren die Chefs dieser Medien ohnehin längst in
den Council eingebunden und saßen mit den Direktoren der CIA am selben
Tisch -- von einer subversiven Unter­wanderung ansonsten unabhängiger
Medien kann insofern nicht wirklich gesprochen
werden.~\href{https://en.wikipedia.org/wiki/Operation_Mockingbird}{Beendet}
haben soll dieses Programm schließlich CIA- und CFR-Direktor George H.W.
Bush -- jedenfalls stand dies damals so in den Zeitungen.

Der ehemalige CIA-Offizier und Whistleblower John Stockwell
\href{https://swprs.org/video-the-cia-and-the-media/}{sagte} zu seiner
Arbeit im Angola-Krieg: »Das grundsätzliche Ziel war, es wie eine
gegnerische Aggression aussehen zu lassen. In diesem Sinne schrieben wir
passende Geschichten und brachten sie in den Medien unter. Ein Drittel
meines Teams in dieser Mission waren PR-Experten, deren Aufgabe es war,
Nachrichten zu erfinden und sie in der Presse zu platzieren. Die
Redakteure in den meisten westlichen Zeitungen sind nicht allzu
skeptisch bei Meldungen, die den allgemeinen Ansichten und Vorurteilen
entsprechen. Einige unserer Geschichten liefen während Wochen. Aber es
war alles erfunden.«

Dass auch einige deutschsprachige Top-Journalisten eng mit
Geheimdiensten zusammenarbeiten, dies zeigt beispielhaft der Fall von
Otto Schulmeister. Schulmeister war langjähriger Chefredakteur der
\emph{Presse}, einer der traditionsreichsten Tages­zeitungen
Österreichs. Dabei unterhielt er enge Kontakte zur CIA und wurde vom
Geheim­dienst laufend mit »Material« versorgt. In der CIA-Zentrale
freute man sich über die gute Zusammenarbeit, wie in seinem kürzlich
deklassifizierten Dossier
\href{https://swprs.org/der-chefredakteur-und-die-cia}{nachzulesen} ist:
»Material ausgehändigt. Es erschien ein Leitartikel nach unseren
Anweisungen.«

Das Besondere an Geheimdiensten wie der CIA ist indes, dass sie nicht
nur in der Gewinnung und Verarbeitung von Informationen tätig sind,
sondern auch verdeckte Operationen durchführen. Beispiels­weise
organisierten britische und amerikanische Geheimdienste zusammen mit der
NATO während des Kalten Kriegs Dutzende Bombenanschläge in Westeuropa,
die sodann kommunistischen und arabischen Gruppierungen angelastet
wurden
(\href{https://web.archive.org/web/20190214075435/https://www.danieleganser.ch/assets/files/Inhalte/Publikationen/Fachzeitschriften/DanieleGanser_Terrorism_in_Western_Europe.pdf}{Operation
Gladio}). CFR-konforme Medien verbreiteten dabei stets das offizielle
Narrativ und stellten keine kritischen Fragen -- ein Mechanismus, der
sich bis heute beobachten lässt.

\includegraphics{https://swprs.files.wordpress.com/2017/09/bologna.png?w=736}\\
\emph{Anschlag auf den Bahnhof von Bologna, 1980: Eine Gladio-Operation}

Auf diese Weise kann das Netzwerk des Councils von der verdeckten
Operation bis hin zur medialen Bericht­erstattung eine ganze
Ereigniskette dirigieren und so eine künstliche Realität erschaffen, mit
der sich die Öffentlich­keit nahezu beliebig lenken lässt. Oder wie es
der ehemalige CIA-Direktor und Council-Vertreter William Casey einst
\href{https://www.quora.com/Did-CIA-Director-William-Casey-really-say-Well-know-our-disinformation-program-is-complete-when-everything-the-American-public-believes-is-false/answer/Barbara-Honegger}{formulierte}:
„Unser Desinformations­programm wird erst abgeschlossen sein, wenn
alles, was die Öffentlich­keit glaubt, falsch ist.``

\textbf{Box 2: Die SITE Intelligence Group.} Wenn ein neues
Al-Kaida-Video auftaucht oder die mysteriöse Terror­gruppe ISIS sich zu
einem Anschlag bekennt,
\href{https://www.hintergrund.de/globales/terrorismus/propaganda-und-wahrheit-die-botschaften-des-osama-bin-laden/}{erfahren}
CFR-konforme Medien dies zumeist aus derselben Quelle: Von der
israelisch-amerikanischen
\href{http://www.globalresearch.ca/who-is-behind-the-islamic-state-is-beheadings-probing-the-site-intelligence-group/5402082}{SITE
Intelligence Group}. Das Besondere an SITE: Die Organisation erhält
solche Informationen nicht nur meist als Erste, sie ist mitunter auch an
deren Produktion beteiligt. So stellte sich 2011 im Rahmen eines
Münchner Gerichtsverfahrens heraus, dass SITE zusammen mit
US-Geheimdiensten beim Aufbau der Al-Kaida-Plattform \emph{Global
Islamic Media Front} (GIMF) in Deutschland
\href{https://www.swr.de/blog/terrorismus/2011/09/09/eklat-im-gimf-verfahren-steuerte-das-fbi-die-zweite-generation/}{behilflich
war}.\\
\includegraphics{https://swprs.files.wordpress.com/2017/09/site-isis.png?w=736}*Ausschnitt
aus einem von SITE veröffentlichten ISIS-Video. Die Terrorgruppe
marschierte 2014
\href{https://journal-neo.org/2015/06/09/logistics-101-where-does-isis-get-its-guns/}{via
NATO-Mitglied Türkei und NATO-Partner Jordanien} in Syrien und im Irak
ein und provozierte eine Militärintervention der US-Allianz in diesen
Ländern.\\
*

\hypertarget{regierungen}{%
\paragraph{Regierungen}\label{regierungen}}

Bis zur überraschenden Wahl von Donald Trump (siehe:
\href{https://swprs.org/trump-medien-geopolitik/}{Trump, die Medien, und
die Geopolitik}) besetzte der Council während Jahrzehnten nahezu alle
Schlüssel­positionen in der US-Regierung und stellte pro Administration
-- ob demokratisch oder republikanisch --
\href{https://swprs.files.wordpress.com/2017/07/cfr-administration-members-1900-2014.pdf}{mehrere
hundert Spitzenbeamte und Berater}. Der ehemalige US-Senator Barry
Goldwater
\href{http://www.thirdworldtraveler.com/New_World_Order/Goldwater_NoApologies.html}{sagte}
dazu einst: „Wenn wir die Präsidenten wechseln, dann bedeutet dies, dass
die Wähler einen Wechsel in der nationalen Politik wünschen. () Bisher
gab es zwar stets einen großen Wechsel an Personal, aber keinen Wechsel
in der Politik, denn ein CFR-Mitglied löste das andere ab.``

Andere Regierungen im Einflussbereich der USA haben meist nur geringen
Einfluss und Spielraum, wenn es um geopolitische bzw. imperiale
Angelegenheiten geht. Werner Weidenfeld, der langjährige Koordinator der
deutschen Bundesregierung für die deutsch-amerikanische Zusammenarbeit,
\href{https://swprs.org/video-werner-weidenfeld/}{erklärte} dies in
einem Interview wie folgt: „Wenn wir in einer \emph{ernsten} Frage
anderer Auffassung sind {[}als die Amerikaner{]}, dann kommt
Geheim­dienst­material auf den Tisch, das Deutschland belastet, und
{[}es heißt:{]} entweder ihr macht mit, oder ihr seid dran.`` (siehe
Video)

\emph{Weidenfeld: »Entweder ihr macht mit, oder ihr seid dran.«
(\href{https://swprs.org/video-werner-weidenfeld/}{2013})}

Auch die offiziell neutrale Schweiz kann sich den geopolitischen Zwängen
nicht entziehen: Würde sich die Eidgenossen­schaft beispielsweise nicht
an den US-initiierten Sanktionen gegen Russland, Syrien oder den Iran
beteiligen, dann gäbe es eben Sanktionen gegen die Schweiz (wie sie im
Rahmen des
\href{https://de.wikipedia.org/wiki/Hotz-Linder-Agreement}{Hotz-Linder-Abkommens}
schon einmal angedroht wurden) -- mit verheerenden Folgen für die
Schweizer Wirtschaft und Gesellschaft. Entsprechend zurück­haltend
berichten staats­tragende Medien über derartige Themen.

\begin{quote}
»Diejenigen von uns, die in der Wahlkampagne von Kennedy mitwirkten,\\
wurden in der Regierung toleriert und durften mitreden, aber die
Außenpolitik\\
war dennoch in der Hand der Leute vom Council on Foreign Relations.«

John Kenneth Galbraith, Harvard-Ökonom und Kennedy-Unterstützer\\
Zitat aus:
\href{https://books.google.com/books?id=U_WoXcJq_1wC\&q=\%22council+on+foreign+relations+people\%22}{The
Best and the Brightest}, S. 60
\end{quote}

Innerhalb der Matrix nehmen Regierungen verschiedene Aufgaben wahr.
Einerseits zählen sie natürlich seit jeher zu den Hauptakteuren in der
unmittelbaren Verbreitung von Propaganda. Im Vergleich zu autoritären
Staaten profitieren Demokratien dabei vom Umstand, dass ihre durch
Propaganda »belasteten« Regierungen alle paar Jahre durch frische
Nachfolger mit neuem Vertrauens­vorschuss abgelöst werden -- wobei sich
an den geopolitischen Macht­ver­hältnissen und Mechanismen durch den
Regierungs­wechsel zumeist nichts ändert (siehe Abbildung).

\includegraphics{https://swprs.files.wordpress.com/2018/10/us-presidential-approval-tracker.png?w=700\&h=389}\\
\emph{Zustimmungswerte für US-Präsidenten seit 1946 (Gallup / USA
Today)}

Noch wesentlich grundlegender ist jedoch der staatliche Einfluss auf das
Bildungs­wesen, durch welches das Welt- und Geschichts­bild einer
Bevölkerung nachhaltig geformt wird. Insbesondere die
Geschichts­schreibung ist ein essentielles Instrument, um »Gut« und
»Böse« zu definieren und das Selbst­ver­ständnis von Ländern zu prägen.
Und obwohl jeder weiß, dass »der Sieger die Geschichte schreibt«, sind
sich nur wenige bewusst, dass dem \emph{tatsächlich} so ist.

CFR-Medien -- sowie das Online-Lexikon
\href{https://swprs.org/propaganda-in-der-wikipedia/}{Wikipedia} --
sorgen ihrerseits dafür, die imperiale Geschichts­schreibung in der
Öffent­lich­­keit präsent zu halten, während es kritischen Historikern
(»Revisionisten«) oft noch schlechter ergeht als ihren Kollegen im
Journalismus. Denn es gilt das
\href{http://www.sparknotes.com/lit/1984/quotes.html}{Diktum} von George
Orwell: \emph{„Wer die Vergangenheit kontrolliert, kontrolliert die
Zukunft. Wer die Gegenwart kontrolliert, kontrolliert die
Vergangenheit.``}

\textbf{Box 3: Die UNO.} Die Giftgasangriffe der syrischen Regierung
seien durch einen »sehr seriösen UNO-Bericht« belegt, schrieb der
Ombudsmann des \emph{Schweizer Fernsehens} in seiner
\href{https://swprs.org/srf-ombudsstelle-im-faktencheck/}{Antwort} an
einen Zuschauer, der dem SRF eine einseitige Bericht­erstattung vorwarf.
Doch~\href{https://consortiumnews.com/2016/09/08/un-team-heard-claims-of-staged-chemical-attacks/}{amerikanische
Investigativ-Journalisten} kamen zu einem gegenteiligen Ergebnis: Der
\href{http://daccess-ods.un.org/access.nsf/GetFile?OpenAgent\&DS=S/2016/738\&Lang=E\&Type=PDF}{UNO-Bericht}
zu Gift­gas­einsätzen in Syrien weise gravierende Mängel auf,~ habe
Manipu­lationen ignoriert und basiere letztlich auf Behauptungen
regierungs­­feind­licher Milizen.\\
--\\
Weshalb publiziert die UNO einen solch fragwürdigen Bericht? Womöglich
deshalb,
\href{https://www.rubikon.news/artikel/die-vereinten-nationen-in-den-handen-der-kriegstreiber-1-3}{weil}
die federführende UNO-Abteilung für Politische Angelegenheiten von einem
US-Diplomaten geleitet wird, der zuvor in der Besatzungs­behörde für den
Irak (CPA)
\href{https://en.wikipedia.org/wiki/Jeffrey_D._Feltman}{diente}, während
die von der UNO mit der Untersuchung beauftragte \emph{Organisation für
das Verbot chemischer Waffen (OPCW)} von einem türkischen Diplomaten
geführt wird, der zuvor NATO-Funktionär (und Bilderberg-Teilnehmer)
\href{https://en.wikipedia.org/wiki/Ahmet_\%C3\%9Cz\%C3\%BCmc\%C3\%BC}{war}.
Auch UNO-Berichte sind mithin stets kritisch zu prüfen -- zumal
CFR-Medien diese Arbeit aus naheliegenden Gründen kaum übernehmen
werden. (Mehr zur Rolle der UNO: siehe
\href{https://consortiumnews.com/2013/10/16/how-us-pressure-bends-un-agencies/}{hier}
und
\href{https://consortiumnews.com/2017/09/08/u-n-enablers-of-aggressive-war/}{hier}).\\
\includegraphics{https://swprs.files.wordpress.com/2017/09/uno-syrien-giftgas.png?w=450}\emph{Die
amerikanische UNO-Botschafterin präsentiert Bilder von Opfern eines
angeblichen Giftgas-Angriffs der syrischen Regierung; April 2017 (AP).}

\hypertarget{hollywood}{%
\paragraph{Hollywood}\label{hollywood}}

Nebst den traditionellen Medien ist auch die \textbf{Filmindustrie in
Hollywood} ein fester Bestandteil der CFR-Matrix, zumal die Chefs aller
bekannten Filmstudios -- von \emph{Disney} über \emph{Universal} bis
\emph{20th Century Fox} -- in den Council eingebunden sind. Deshalb
erstaunt es nicht, dass Hollywood von \emph{American Sniper} bis
\emph{Zero Dark Thirty} einen Propaganda­streifen nach dem anderen in
die Kinos bringt und damit -- zusätzlich zum Schulunterricht -- das
Welt- und Geschichtsbild breiter Bevölkerungs­schichten auf mehr oder
weniger subtile Art und Weise beeinflusst.

Die Filmstudios agieren dabei nicht unabhängig von den übrigen Akteuren
der CFR-Matrix: Gemäß
\href{https://www.globalresearch.ca/documents-expose-how-hollywood-promotes-war-on-behalf-of-the-pentagon-cia-and-nsa/5597891}{kürzlich
veröffentlichten Dokumenten} haben Pentagon und CIA die Drehbücher von
mindestens 800 Kinofilmen und über 1000 TV-Produktionen bis in einzelne
Dialoge und Figuren hinein bearbeitet, um dem Publikum die gewünschten
Botschaften und Stereotype zu vermitteln. Besonders lohnenswert ist
dieser Aufwand, wenn die jeweilige Filmproduktion Ende des Jahres einen
Oscar erhält -- so wie zuletzt der
\href{http://edition.cnn.com/2017/02/26/us/white-helmets-oscar/}{»Dokumentarfilm«}
über die ominösen Weißhelme in Syrien.

\begin{quote}
\emph{»Der Kinofilm ist eines der mächtigsten Propagandawerkzeuge,}\\
\emph{die den Vereinigten Staaten zur Verfügung stehen.«}

Zitat aus: Der Film als Waffe der psychologischen Kriegsführung,*

\begin{itemize}
\tightlist
\item
  \href{https://web.archive.org/web/20170913122557/http://www.spyculture.com/docs/US/OSS-motionpicturesasweapons.pdf}{Strategiepapier
  des US-Geheimdienstes OSS}
\end{itemize}
\end{quote}

Doch nicht nur Filmstudios, auch einige der bekanntesten Hollywood-Stars
sind Mitglied im CFR und engagieren sich für dessen internationale
Projekte. Wenn \textbf{Angelina Jolie} nach Libyen fliegt, um mit den
NATO-Revolutionären
\href{https://www.reuters.com/article/us-libya-jolie-idUSTRE79A3S820111011}{Solidarität}
zu zeigen und sie
\href{https://www.reuters.com/article/us-libya-jolie-idUSTRE79A3S820111012}{für
ihren Einsatz zu loben}, oder wenn \textbf{George Clooney} sich (der
hungernden Kinder wegen) für die Aufspaltung des (ölreichen und
China-freundlichen) Sudans unter US-Aufsicht
\href{http://edition.cnn.com/2011/WORLD/africa/01/06/george.clooney.sudan.vote/index.html}{einsetzt},
dann berichten CFR-Medien
\href{http://www.thedailybeast.com/george-clooney-leverages-celebrity-to-bring-change-to-sudan}{ausführlich}
darüber -- und erwähnen dabei nur eines nicht: dass diese Schauspieler
ebenfalls Mitglieder des Councils sind.

\includegraphics{https://swprs.files.wordpress.com/2017/08/clooney-jolie-cfr.png?w=736}
*Clooney 2012 im (Süd-)Sudan, Jolie 2011 in Libyen. (Abaca/Reuters)\\
*

\hypertarget{nachrichtenagenturen}{%
\paragraph{Nachrichtenagenturen}\label{nachrichtenagenturen}}

Eine besondere Rolle in der Informationsmatrix nehmen die
\textbf{Nachrichten­agenturen} ein. Der ehemalige Geschäftsführer der
österreichischen Nachrichten­agentur APA beschrieb ihre Funktion mit
diesen Worten: »Nachrichten­agenturen stehen selten im Blickpunkt des
öffentlichen Interesses. Dennoch sind sie eine der einflussreichsten und
gleichzeitig eine der am wenigsten bekannten Mediengattungen. Sie sind
Schlüssel­insti­tutionen mit substanzieller Bedeutung für jedes
Mediensystem. Sie sind das unsichtbare Nerven­zentrum, das alle Teile
dieses Systems verbindet.« (Segbers 2007, S.10)

Tatsächlich stammen bei internationalen Ereignissen nahezu alle Texte
und Bilder, die CFR-Medien verwenden, von nur \emph{drei} globalen
Nachrichten­agenturen: der amerikanischen \emph{Associated Press (AP),}
der britisch-kanadischen \emph{Thomson-Reuters}, und der französischen
\emph{Agence France-Presse (AFP).} Selbst internationale Korrespondenten
müssen sich für ihre Arbeit zumeist auf diese Agenturen verlassen, wie
der langjährige holländische Kriegs­bericht­erstatter Joris Luyendijk in
seinem Buch
\href{https://www.klett-cotta.de/buch/Tropen-Sachbuch/Von_Bildern_und_Luegen_in_Zeiten_des_Krieges/48944}{»Von
Bildern und Lügen in Zeiten des Krieges«} eindrucksvoll beschrieb.

\href{https://swprs.files.wordpress.com/2016/06/schlagzeilen-agenturen-r.png}{\includegraphics{https://swprs.files.wordpress.com/2016/06/schlagzeilen-agenturen-r.png?w=600\&h=986}}\\
\emph{»Putin droht«, »Iran provoziert«, »NATO besorgt«,
»Assad-Hochburg«: Inhaltliche und sprachliche Ähnlichkeit der
geopolitischen Bericht­erstattung aufgrund von Meldungen der globalen
Agenturen.}

Die Dominanz der drei globalen Agenturen hat einerseits zur Folge, dass
sich in CFR-Medien von Wien bis Washington meist in etwa dieselben
Informationen finden -- und dieselben Informationen fehlen. Andererseits
erleichtert die zentrale Informations­distribution jenen Akteuren die
Arbeit, die in entscheidenden Momenten Propaganda und Desinformation in
das weltweite Mediensystem einspeisen möchten.

Reuters-Kriegskorrespondent Fred Bridgland beschrieb dies in einer
\href{https://swprs.org/video-the-cia-and-the-media/}{bemerkenswerten
Reportage} des britischen \emph{Channel 4} wie folgt: »Wir basierten
unsere Berichte auf offiziellen Mitteilungen. Erst Jahre später erfuhr
ich, dass in der US-Botschaft ein Desinformations-Experte der CIA saß
und diese Mitteilungen erfand, die überhaupt keinen Bezug zur Realität
hatten. () Aber ehrlich gesagt, egal was die Agenturen publizieren, es
wird von den Redaktionen sowieso aufgenommen.«

Während Reuters und AP direkt in den Council eingebunden sind, gehört
die AFP dem französischen Staat, der seinerseits über die
Bilderberg-Gruppe und die NATO in die transatlantischen Strukturen
integriert ist. Im Endeffekt fungieren die globalen Agenturen dadurch
als eine Art
\href{https://swprs.org/der-propaganda-multiplikator/}{»Propaganda-Multiplikator«},
mit dem CFR-Operateure und ihre Partner die gewünschten Botschaften
weltweit verbreiten können (siehe
\href{https://swprs.org/der-propaganda-multiplikator/}{Vertiefungsstudie}).
Dabei profitieren sie vom Umstand, dass die Agenturen im Normalfall
absolut seriös arbeiten und deshalb einen ausgezeichneten Ruf genießen.

\href{https://swprs.files.wordpress.com/2019/02/propaganda-multiplikator.png}{\includegraphics{https://swprs.files.wordpress.com/2019/02/propaganda-multiplikator.png?w=450}}\\
\emph{Der Propaganda-Multiplikator}

Nur dank den drei globalen Agenturen gelangten etwa die zweifelhaften
Berichte der \emph{Syrischen Beobachtungsstelle für Menschenrechte} oder
die fragwürdigen Ukraine-Analysen von \emph{Bellingcat} an hunderte
internationale Medien und dadurch an ein weltweites Milliardenpublikum.

Auch die Bilder des ausgebrannten UN-Hilfskonvois bei Aleppo im
September 2016 sowie des »Giftgasangriffs« auf Khan Sheikhoun im April
2017 -- beide Ereignisse sind bis heute nicht aufgeklärt -- gingen um
die Welt und sorgten für massive diplomatische und sogar militärische
Reaktionen. Die Fotos stammten in beiden Fällen von den zwei selben, in
US-unterstützte Milizen eingebetteten,
\href{https://de.sputniknews.com/blogs/20170507315669095-reuters-afp-getty-vermarkten-fake-syrien/}{Agenturfotografen}.

Die Arbeiten unabhängiger Reporter schaffen es bei geopolitisch
brisanten Ereignissen hingegen kaum in die Nachrichten. Der Norweger Jan
Oberg war im Dezember 2016 als einer von wenigen Fotografen im
rückeroberten Aleppo vor Ort, doch seine
\href{https://janoberg.exposure.co/}{Bilder} konnte er in keinem Medium
unterbringen -- sie hätten
\href{https://janoberg.exposure.co/humans-in-liberated-aleppo}{»nicht
ins westliche Narrativ gepasst«}. Und der langjährigen
Nahost-Korrespondentin und Syrien-Kennerin Karin Leukefeld wurde
mitgeteilt, man könne ihre Reportagen nicht mehr verwenden, da sie sich
nicht an die
\href{https://www.watson.ch/International/Kommentar/148360008-Spielball-der-M\%C3\%A4chte--Weshalb-der-Syrien-Konflikt-in-erster-Linie-ein-Stellvertreterkrieg-ist}{»einschlägigen
Agentur­meldungen«} halte.

\includegraphics{https://swprs.files.wordpress.com/2017/09/oberg-syrien.png?w=500}\\
\emph{Passten nicht ins »westliche Narrativ«: Bilder von Jan Oberg aus
Aleppo.}

Chefs von Nachrichtenagenturen haben aufgrund ihres Überblicks über die
Medien­land­schaft mitunter aber noch ganz andere Aufgaben: So war der
Direktor der \emph{Schweizerischen Depeschen­agentur SDA} während des
Kalten Krieges persönlich dafür
\href{https://www.infosperber.ch/Medien/Demokratie-Pressevielfalt-Medien}{zuständig},
der »Links­lastigkeit« verdächtigte Schweizer Journalisten zwecks
Fichierung und Observation der Bundes­polizei zu melden.

\hypertarget{pr-agenturen}{%
\paragraph{PR-Agenturen}\label{pr-agenturen}}

Was Regierungen, Militärs und Geheimdienste nicht selbst durchführen
können oder wollen, das übernehmen externe PR-Agenturen. So wurde etwa
die bekannte
\href{https://en.wikipedia.org/wiki/Nayirah_(testimony)}{»Brutkastenlüge«}
(siehe oben) von der US-Agentur \emph{Hill \& Knowlton} inszeniert,
indem die Tochter des kuwaitischen Botschafters zur Kranken­schwester
gemacht und auf ihre Falschaussage vor dem US-Kongress vorbereitet
wurde.

Die Schlüsselfigur war damals
\href{https://en.wikipedia.org/wiki/Nayirah_(testimony)}{John E.
Porter}, der den Kongress­ausschuss leitete und gleichzeitig mit der
PR-Agentur kooperierte. Angesichts solcher Kollusion
\href{https://en.wikipedia.org/wiki/Nayirah_(testimony)\#Revelation}{forderte}
selbst die CFR-affine \emph{New York Times} Konsequenzen -- und die gab
es tatsächlich: Porter wurde kurz darauf in den Council
\href{https://swprs.files.wordpress.com/2017/07/council-on-foreign-relations-membership-rosters-1922-2013.pdf}{gewählt}.

\includegraphics{https://swprs.files.wordpress.com/2017/08/nayirah-orf.png?w=480\&h=360}\\
\emph{Die Brutkastenlüge: »Krankenschwester« Nayirah vor dem
US-Kongress, 1991}

Der Golfkrieg war eben erst beendet und die Brutkasten­lüge aufgedeckt,
da war die US-Agentur \emph{Rudder Finn} bereits in den Balkan­kriegen
\href{https://www.hintergrund.de/globales/kriege/operation-balkan-werbung-fuer-krieg-und-tod/}{aktiv}
und bereitete den publizistischen Boden für die folgende
NATO-Intervention (siehe Becker/Beham,
\emph{\href{https://www.hintergrund.de/globales/kriege/operation-balkan-werbung-fuer-krieg-und-tod/}{Operation
Balkan: Werbung für Krieg und Tod},} 2008). Der damalige Direktor von
\emph{Rudder Finn} erklärte in einem späteren
\href{http://www.sourcewatch.org/index.php/James_Harff}{Interview},
warum seine Firma beispiels­weise die Falsch­meldung von serbischen
»Todeslagern« in Bosnien verbreitete:

„Unsere Arbeit ist nicht, Informationen zu überprüfen. Dafür sind wir
nicht ausgerüstet. Unsere Arbeit ist es, die Zirkulation von für uns
günstigen Informationen zu beschleunigen und sorgfältig ausgewählte
Ziele zu erreichen. Wir haben die Existenz von Todeslagern in Bosnien
nicht bestätigt, wir haben nur bekannt gemacht, dass {[}das
US-Magazin{]} \emph{Newsday} diese behauptet hat. () Wir sind
Professionals. Wir hatten einen Auftrag und wir erledigten ihn. Wir sind
nicht dafür bezahlt, moralisch zu sein.``

Als der Auslandschef einer Schweizer Wochen­zeitung diese und andere
Kriegslügen Mitte der 1990er Jahre einem deutsch­sprachigen Publikum
darlegen wollte, intervenierten umgehend bekannte Medien­häuser aus
Deutschland und der Schweiz bei seinem Verleger und sorgten dafür, dass
er zu Bosnien vorläufig nichts mehr schreiben durfte und gar seine
Absetzung \href{https://swprs.org/das-gewuenschte-narrativ/}{diskutiert
wurde}.

\includegraphics{https://swprs.files.wordpress.com/2017/08/bana-rowling.png?w=500}\\
*Danke für die Bücher: »Bana Alabed« und J.K. Rowling, Syrienkrieg,
2016. (AP)\\
*

Auch im Syrienkrieg waren PR-Profis gefragt. Ein Höhepunkt war hierbei
zweifellos das
\href{http://blauerbote.com/2017/07/05/bana-alabed-aus-aleppo/}{»siebenjährige
Twitter-Mädchen Bana Alabed«}, welches der Bevölkerung in den
NATO-Staaten in bestem Englisch versicherte, dass die Rückeroberung
Aleppos durch die syrische Armee und Russland keine Befreiung, sondern
ein neuer »Holocaust« sei. CFR-Medien berichteten
\href{https://www.rubikon.news/artikel/das-twittermadchen-aus-syrien}{während
Wochen} über das Kind.

Schließlich wurde bekannt, dass »Bana« bei der britischen PR-Agentur
\emph{The Blair Partnership}
\href{http://blauerbote.com/2017/04/23/bana-alabed-ist-klientin-der-londoner-pr-agentur-the-blair-partnership/}{unter
Vertrag ist}, bei der auch Harry-Potter-Autorin J.K. Rowling mitwirkt,
die ihr zuvor medienwirksam einige ihrer Bücher zukommen ließ. Keine
Geschenke gab es hingegen für einen deutschen Blogger, der dem Magazin
\emph{Stern} wegen unkritischer Verbreitung der Bana-Geschichte
\emph{»Fake News«} vorwarf: Er wurde sogleich
\href{http://blauerbote.com/2017/07/15/urteil-zu-gerichtsverfahren-gegen-marc-drewello-und-stern/}{verklagt}.

Ein weiteres Spezialgebiet von PR-Agenturen ist das sogenannte
\href{https://de.wikipedia.org/wiki/Astroturfing}{\emph{Astroturfing}},
bei dem eine künstliche öffentliche Bewegung lanciert wird, um ein
politisches Ziel zu erreichen. Besonders beliebt sind hierzu
Online-Petitionen vermeintlich humanitärer Organisationen wie
\href{http://www.nachdenkseiten.de/?p=35284}{Avaaz} oder
\href{http://www.nachdenkseiten.de/?p=35689}{Campact}, die dann statt
Unterstützung für den Regenwald plötzlich eine »Flug­verbots­zone« über
Libyen
\href{https://consortiumnews.com/2016/04/14/duping-progressives-into-wars/}{fordern}.

\includegraphics{https://swprs.files.wordpress.com/2017/08/avaaz-libya.png?w=600}\\
\emph{Avaaz-Petition zur Errichtung einer Flugverbotszone über Libyen.}

\hypertarget{3-journalisten-in-der-matrix}{%
\subsubsection{3. Journalisten in der
Matrix}\label{3-journalisten-in-der-matrix}}

Ein entscheidender Aspekt der CFR-Matrix besteht darin, dass auch
gewöhnliche Journalisten in sie eingeschlossen sind. Viele Journalisten
dürften mithin selbst an die ihnen vorgesetzten Narrative glauben,
während andere wie PR-Profis arbeiten und ihre Beiträge einfach mit dem
gewünschten \emph{Spin} abliefern. Wieder andere mögen in der
Konformität gar eine Karrierechance für sich erblicken.

Doch vorselektierte Quellen, Gruppendruck und die Abhängigkeit von
Vorgesetzten und Auftrag­gebern sorgen dafür, dass es selbst für
aufrichtige und intelligente Journalisten schwierig bis unmöglich ist,
die Informations­matrix von innen heraus zu durchbrechen und abweichende
Standpunkte einzubringen, sofern imperiale Angelegenheiten tangiert
sind.

So haben Mitarbeiter der \emph{ARD} gemäß internen Memos die Vorgabe,
bei geopolitischen Konflikten
\href{https://www.heise.de/tp/features/Ukraine-Konflikt-ARD-Programmbeirat-bestaetigt-Publikumskritik-3367400.html}{\emph{»west­liche
Posi­tionen zu ver­tei­di­gen«}}, vertrauliche
\href{https://www.heise.de/tp/features/Die-vertraulichen-Sprachregelungen-der-ARD-3758887.html}{Sprachregelungen}
zu beachten und ausschließlich
\href{https://www.oxmoxhh.de/magazin/story-interview/oxmox-exklusiv-interview-mit-volker-braeutigam-friedhelm-klinkhammer/}{konforme
Quellen} zu verwenden. Der ehemalige Chefredakteur des \emph{ZDF} machte
zudem publik, dass Bei­träge zu US-Kriegen
\href{https://www.youtube.com/watch?v=i2423aDq_hE}{politisch
beeinflusst} werden. Nahost-Korrespondent Ulrich Tilgner be­klagte
re­dak­tio­nelle Ein­­griffe aufgrund von
\href{http://www.berliner-zeitung.de/korrespondent-ulrich-tilgner-sucht-mehr-distanz-zum-zdf--ich-fuehle-mich-eingeschraenkt--15870684}{»Bünd­nis­rück­sich­ten«},
und der vormalige Leiter des *ZDF-*Studios Bonn be­stä­tig­te
\href{https://propagandaschau.wordpress.com/2016/01/30/wolfgang-herles-es-gibt-in-den-oeffentlich-rechtlichen-anweisungen-von-oben/}{»An­wei­sungen
von oben«} und eine
\href{http://www.rolandtichy.de/daili-es-sentials/meinungsfreiheit-anordnung-zur-anpassung/}{»frei­willige
Gleich­­schal­­tung«} der Jour­na­lis­ten.

\includegraphics{https://swprs.files.wordpress.com/2017/09/tilgner-zdf.png?w=500\&h=491}\\
\emph{Re­dak­tio­nelle Ein­­griffe aufgrund von
»Bünd­nis­rück­sich­ten«: Nahost-Korrespondent Tilgner}

Abweichler werden entsprechend sanktioniert: In der Schweiz wurde etwa
der langjährige SRF-Korrespondent Helmut Scheben als »Putin-Troll« und
»Teil der russischen Propaganda-Maschinerie«
\href{https://swprs.org/das-gewuenschte-narrativ-ii/}{beschimpft}, als
er es wagte, die Syrien­bericht­erstattung westlicher Medien kritisch zu
hinterfragen. Auch ein NZZ-Autor, der sich anmerken ließ, dass er noch
offene Fragen zu den Ereignissen vom 11. September 2001 hat, wurde von
seinem Chef umgehend
\href{https://swprs.org/anschlag-auf-die-forschungsfreiheit/}{öffentlich
zurechtgewiesen}.

Amerikanischen Journalisten ergeht es nicht besser. \textbf{Gary Webb},
der in den 90er Jahren aufdeckte, dass die CIA Kokain aus Kolumbien
importierte und mit den Erlösen Milizen in Nicaragua finanzierte, wurde
von den US-Medien
\href{https://consortiumnews.com/2014/11/02/gary-webb-and-media-manipulation/}{so
lange diffamiert}, bis sein Ruf ruiniert war und er einige Jahre später
Selbstmord beging. \textbf{Phil Donahue}, der 2003 als beinahe einziger
US-Topjournalist den geplanten Irak-Krieg kritisierte, wurde von MSNBC
trotz hervorragender Quoten
\href{https://www.democracynow.org/2013/3/21/phil_donahue_on_his_2003_firing}{kurzerhand
entlassen}.

\textbf{Amber Lyon}, die im Auftrag von CNN eine Dokumentation über den
US-Verbündeten Bahrein drehte und darin die Menschenrechtslage
kritisierte, wurde die Ausstrahlung von ihrem eigenen Sender
\href{https://www.theguardian.com/world/2012/sep/04/cnn-international-documentary-bahrain-arab-spring-repression}{verweigert},
worauf sie den Sender von sich aus verließ. Und \textbf{Sean Hannity},
der auf \emph{Fox News}
den\href{http://www.globalresearch.ca/the-case-of-seth-richs-assassination-will-it-expose-the-truth-behind-the-russia-elections-hacking-narrative/5593866}{ungeklärten
Mord} an DNC-Mitarbeiter Seth Rich thematisieren wollte, sah sich mit
dem Absprung mehrerer Sponsoren und der möglichen Absetzung seiner
Sendung
\href{https://www.theguardian.com/media/2017/may/26/sean-hannity-advertiser-boycott-fox-news-seth-rich}{konfrontiert}
-- sowie mit
\href{http://foreignpolicy.com/2017/05/22/the-seth-rich-scandal-shows-that-fox-news-is-morally-bankrupt/}{erbosten
Kommentaren} von führenden Council-Mitarbeitern.

\includegraphics{https://swprs.files.wordpress.com/2017/09/seth-rich-wikileaks.png?w=500\&h=450}\\
\emph{Für CFR-Medien ein Tabu: Erhielt Wikileaks die Emails der
Demokratischen Partei 2016 nicht} \emph{von »russischen Hackern«,
sondern} \emph{vom kurz darauf ermordeten DNC-Mitarbeiter Seth Rich?}

Nun könnte man annehmen, dass in solch offensichtlichen
Missbrauchs­fällen das amerikanische
\emph{\href{https://en.wikipedia.org/wiki/Committee_to_Protect_Journalists}{Committee
to Protect Journalists (CPJ)},} das sich für die Rechte der Journalisten
einsetzt, intervenieren würde. Dem ist jedoch nicht so -- denn die
Direktoren sowie fast der gesamte Vorstand des CPJ sind selbst
Mitglieder des \emph{Council on Foreign Relations.}

Immerhin könnten solche Journalisten für ihre Arbeit eine Auszeichnung
erhalten, beispiels­weise den renommierten
\href{https://en.wikipedia.org/wiki/Pulitzer_Prize}{Pulitzer-Preis}.
Auch hier wartet man jedoch vergeblich, denn der Präsident des
Pulitzer-Komitees (aktuell ein \emph{Washington-Post}-Redakteur) sowie
diverse Vorstands­mitglieder entstammen ebenso dem Council. Generell ist
die Vergabe von Preisen und Auszeichnungen ein wirksames Mittel, um
festzulegen, was „guter`` Journalismus und wer „renommierter``
Journalist ist.

Der deutsche Investigativ-Journalist und Dokumentar­filmer Dirk Pohlmann
\href{https://youtu.be/mJrA1lnMcv8?t=5m24s}{beschrieb} die Situation mit
den folgenden Worten, nachdem eines seiner geopolitisch brisanten
\href{http://www.imdb.com/name/nm0688383/}{Filmprojekte} vom ZDF auf
höchster Ebene gestoppt wurde:

„Das war eben ein Thema, bei dem man an die Grenzen dessen kommt,
worüber man berichten darf. Diese Grenzen gibt es, auch in unserem
sogenannten »Freien Westen«. Das merkt man, wenn man sie betritt: Dann
gehen auf einmal die Scheinwerfer an, die Hunde fangen an zu bellen und
man hört, wie die Leute näher kommen. Und dann weiß man, ok, jetzt bin
ich in dem Territorium, von dem vorher behauptet wurde, dass es das gar
nicht gibt: nämlich das verminte Territorium der Grenzen der
Informationsfreiheit.``

Bedeutet dies, dass kritischer Journalismus in CFR-konformen Medien
nicht erwünscht ist? Im Gegenteil: Seriöser Journalismus bildet
überhaupt erst die Grundlage für die Glaubwürdigkeit der traditionellen
Medien, auf deren Basis dann gezielt und wirkungsvoll geopolitische (und
andere) Propaganda lanciert werden kann. Denn der arglose Leser und
Zuschauer hat kaum eine Chance, zwischen zwei ehrlichen Beiträgen die
geschickte Manipulation zu erkennen oder auch nur zu erahnen.

Von allen Propaganda-Prinzipien ist dieses -- langfristig gesehen --
vielleicht sogar das wichtigste: Nur Medien, die vertrauenswürdig
erscheinen, können dieses Vertrauen auch missbrauchen.

\hypertarget{4-fazit}{%
\subsubsection{4. Fazit**}\label{4-fazit}}

**

Jahrzehntelang hatte das Netzwerk des \emph{Council on Foreign
Relations} eine nahezu uneingeschränkte Kontrolle über den
geostrategischen Informationsfluss in den NATO-Ländern. Die meisten
Menschen hatten keine Möglichkeit zu realisieren, dass sie sich trotz
scheinbarer Medienvielfalt tatsächlich in einer dicht gewobenen
Informationsmatrix befanden.

Weshalb betrieb und betreibt der Council einen derartigen Aufwand zur
Täuschung der eigenen Bevölkerung? Der inzwischen verstorbene Nationale
Sicherheitsberater und Council-Direktor Zbigniew Brzezinski brachte es
in seinem Buch
\href{https://archive.org/details/TheGrandChessboardAmericanPrimacyAndItsGeostrategicImperatives1997ZbigniewBrzezinski}{\emph{The
Grand Chessboard: American Primacy and Its Geostrategic Imperatives}}auf
den Punkt: „Demokratie ist der imperialen Mobilisierung abträglich, denn
das Streben nach imperialer Macht läuft den demokratischen Instinkten
zuwider.`` (Brzezinski 1998, S. 20)

Tatsächlich mussten die USA seit dem
\href{https://de.wikipedia.org/wiki/Spanisch-Amerikanischer_Krieg}{Krieg
gegen Spanien von 1898} für nahezu alle ihre Interventionen einen
\href{https://www.amazon.de/Zerst\%C3\%B6rung-Hoffnung-Killing-Hope-Interventionen/dp/3889751415}{Vorwand
kreieren}, um die stetige Expansion der eigenen Einflusssphäre -- nach
der letztlich alle Imperien streben -- moralisch zu legitimieren und die
eigene Bevölkerung dafür zu gewinnen -- zumal kaum je ein Land so
leichtfertig war, die Vereinigten Staaten ohne Not von sich aus
anzugreifen.

Hierzu erschufen die USA -- wie Karl Rove es
\href{https://en.wikiquote.org/wiki/Karl_Rove}{ausdrückte} -- eine
eigene, »imperiale« Realität, die von verdeckten Operationen über die
mediale Bericht­erstattung bis hin zur Geschichts­schreibung reicht und
von den Mitgliedern des Councils und seiner Partner­organisationen
inszeniert und propagiert wird.

Damit wird zugleich verständlich, warum CFR-Medien bisweilen derart
\href{http://www.zeit.de/politik/ausland/2014-11/rt-deutsch-russland-propaganda-luegen}{nervös}
auf den Erfolg russischer Medien wie \emph{Russia Today (RT)} reagieren:
Diese erweitern eben nicht nur die gepriesene Mei­nungs­vielfalt,
sondern destabilisieren die umfassende Informations­matrix des Councils
-- zumindest dort, wo dies russischen Interessen dienlich ist.

Durch das Internet entwickelte sich zudem die Möglichkeit, Informationen
dezentral und kostengünstig zu verbreiten und so die \emph{Gatekeeper}
des Councils zu umgehen. Inzwischen existieren auch im
deutsch­sprachigen Raum eine Vielzahl
\href{https://swprs.org/medien-navigator/}{leserfinanzierter Medien und
Plattformen}, die das konventionelle Narrativ kritisch hinterfragen und
neue Sichtweisen ermöglichen (siehe den
\href{https://swprs.org/medien-navigator/}{Medien-Navigator}).

Aus Sicht des Councils stellen solche Publikationen indes eine
zunehmende Bedrohung der eigenen Informations- und Deutungs­hoheit dar.
CFR-konforme Medien und Internet­unternehmen reagierten hierauf mit der
\href{https://www.heise.de/tp/features/Konzentriertes-Gejammer-NZZ-schliesst-Kommentarspalte-3618957.html}{Schließung}
von Leserforen,
\href{https://www.heise.de/tp/features/Facebook-Fake-News-und-die-Privatisierung-der-Zensur-3599878.html}{Zensur}
auf sozialen Netzwerken, der
\href{https://www.wsws.org/de/articles/2017/10/03/goog-o03.html}{»Bereinigung«}
von Suchresultaten sowie zunehmender
\href{https://www.humanrights.ch/de/menschenrechte-schweiz/inneres/person/sicherheit/schweizer-nachrichtendienstgesetz}{Überwachung}.
Auf Illusion folgt somit Repression -- es bleibt die Frage, ob dadurch
das Vertrauen der Bevölkerung zurückgewonnen werden kann.

***

\href{https://swprs.files.wordpress.com/2018/07/die-propaganda-matrix-spr-hdv.pdf}{Studie
als PDF herunterladen}

\hypertarget{5-literatur}{%
\subsubsection{5. Literatur}\label{5-literatur}}

\hypertarget{mitgliederverzeichnisse}{%
\subparagraph{\texorpdfstring{\textbf{Mitgliederverzeichnisse}}{Mitgliederverzeichnisse}}\label{mitgliederverzeichnisse}}

CFR-Mitgliederverzeichnisse von
\href{https://swprs.files.wordpress.com/2017/07/council-on-foreign-relations-membership-rosters-1922-2013.pdf}{1922
bis 2013} und von
\href{https://swprs.files.wordpress.com/2017/07/cfr-members-2016.pdf}{2016}

CFR-Mitglieder in der US-Regierung von
\href{https://swprs.files.wordpress.com/2017/07/cfr-administration-members-1900-2014.pdf}{1900
bis 2014}

Bilderberg-Konferenzen: Teilnehmerlisten
\href{https://swprs.files.wordpress.com/2016/07/bilderberg_teilnehmer_1954-2014.pdf}{1954
bis 2014} und
\href{http://www.bilderbergmeetings.org/latest-meetings.html}{2015-2017}

Trilaterale Kommission: Mitgliederlisten von
\href{https://swprs.files.wordpress.com/2017/07/trilateral-commission-members-1973.pdf}{1973};
\href{https://swprs.files.wordpress.com/2017/07/trilateral-commission-members-1978.pdf}{1978};
\href{https://swprs.files.wordpress.com/2017/07/trilateral-commission-members-1985.pdf}{1985};
\href{https://swprs.files.wordpress.com/2017/07/trilateral-commission-members-1995.pdf}{1995};
\href{https://swprs.files.wordpress.com/2017/07/trilateral-commission-members-2010.pdf}{2010};
und
\href{https://swprs.files.wordpress.com/2017/07/trilateral-commission-members-2017.pdf}{2017}

Atlantik-Brücke: Jahresberichte von
\href{https://www.atlantik-bruecke.org/unsere-arbeit/publikationen/jahresberichte/}{2006
bis 2016}***

\begin{center}\rule{0.5\linewidth}{\linethickness}\end{center}

\hypertarget{council-on-foreign-relations}{%
\subparagraph{\texorpdfstring{\textbf{Council on Foreign
Relations}}{Council on Foreign Relations}}\label{council-on-foreign-relations}}

Website des \href{https://www.cfr.org/}{CFR} sowie seines Magazins
\href{https://www.foreignaffairs.com/}{\emph{Foreign Affairs}}

Domhoff, William G. (2014): The Council on Foreign Relations and the
Grand Area: Case Studies on the Origins of the IMF and the Vietnam War;
\emph{Class, Race and Corporate Power: Vol. 2 : Iss. 1.}
(\href{https://swprs.files.wordpress.com/2017/09/domhoff-cfr-2014.pdf}{PDF})

Grose, Peter (1996/2006): Continuing the Inquiry -- The Council on
Foreign Relations from 1921 to 1996. \emph{CFR Press,} New York.
(\href{https://www.cfr.org/book/continuing-inquiry}{PDF})

Parmar, Inderjeet (1999): Mobilizing America for an Internationalist
Foreign Policy: The Role of the Council on Foreign Relations.
\emph{Studies in American Political Development}, 13(2), pp. 337--373.
(\href{https://swprs.files.wordpress.com/2017/09/cfr-parmar-1999.pdf}{PDF})

Schulzinger, R.D. (1984): The Wise Men of Foreign Affairs. History of
the Council on Foreign Relations. \emph{Columbia University Press,} New
York.

Shoup, Laurence H. (2015): Wall Street's Think Tank: The Council on
Foreign Relations and the Empire of Neoliberal Geopolitics, 1976-2014*.
Monthly Review Press,* New York.
(\href{https://monthlyreview.org/product/wall-streets-think-tank/}{Web})

Shoup, Laurence \& Minter, William (1977): Imperial Brain Trust -- The
Council on Foreign Relations and United States Foreign Policy.
\emph{Monthly Review Press,} New York.
(\href{https://swprs.files.wordpress.com/2017/09/cfr_imperial_brain_trust.pdf}{PDF})

\hypertarget{propaganda-theorie-und-praxis}{%
\subparagraph{**Propaganda: Theorie und
Praxis}\label{propaganda-theorie-und-praxis}}

**

Baines, Paul R (2013): Propaganda. Volume I-IV. \emph{SAGE Library of
Military and Strategic Studies}, London.

Altschull, Herbert J. (1984/1995): Agents of power. The media and public
policy. \emph{Longman,} New York.

Becker, Jörg (2015): Medien im Krieg -- Krieg in den Medien.
\emph{Springer Verlag für Sozialwissenschaften,} Wiesbaden.

Becker, Jörg \& Beham, Miram (2008): Operation Balkan: Werbung für Krieg
und Tod. \emph{Nomos}, Baden-Baden.

Bittermann, Klaus (1994): Serbien muss sterbien. Wahrheit und Lüge im
jugoslawischen Bürgerkrieg. \emph{Edition TIAMAT}, Berlin.

Bussemer, Thymian (2008): Propaganda. Konzepte und Theorien. \emph{VS
Verlag für Sozialwissenschaften}, Wiesbaden.

Chomsky, Noam (1997): What Makes Mainstream Media Mainstream. \emph{Z
Magazine}. (\href{https://chomsky.info/199710__/}{PDF})

Chomsky, Noam \& Herman, Edward (1988): A Propaganda Model.
(\href{https://chomsky.info/consent01/}{Web})

Gritsch, Kurt (2010): Inszenierung eines gerechten Krieges?
Intellektuelle, Medien und der „Kosovo-Krieg`` 1999. \emph{Georg Olms
Verlag,} Hildesheim.

Hird, Christopher (1985): Standard Techniques. \emph{Diverse Reports,
Channel 4 TV.} 30. Oktober 1985.
(\href{https://swprs.org/video-the-cia-and-the-media/}{Web})

Krüger, Uwe (2013): Meinungsmacht. Der Einfluss von Eliten auf
Leitmedien und Alpha-Journalisten -- eine kritische Netzwerkanalyse.
\emph{Herbert von Halem Verlag,} Köln.

Luyendijk, Joris (2015): Von Bildern und Lügen in Zeiten des Krieges:
Aus dem Leben eines Kriegsberichterstatters -- Aktualisierte Neuausgabe.
\emph{Tropen,} Stuttgart.

Morelli, Anne (2004): Die Prinzipien der Kriegspropaganda. \emph{zu
Klampen}, Springe.

Mükke, Lutz (2014): Korrespondenten im Kalten Krieg. Zwischen Propaganda
und Selbstbehauptung. \emph{Herbert von Halem Verlag,} Köln.

Ponsonby, Arthur (1928): Falsehood in War-Time. \emph{George Allen \&
Unwin}, London.

Starkulla, Heinz jr. (2015): Propaganda: Begriffe, Typen, Phänomene.
\emph{Nomos}, Baden-Baden.

Tilgner, Ulrich (2003): Der inszenierte Krieg -- Täuschung und Wahrheit
beim Sturz Saddam Husseins. \emph{Rowohlt}, Reinbek.

\hypertarget{nachrichtenagenturen-1}{%
\subparagraph{\texorpdfstring{\textbf{Nachrichtenagenturen}}{Nachrichtenagenturen}}\label{nachrichtenagenturen-1}}

Blum, Roger et al. (Hrsg.) (1995): Die AktualiTäter.
Nachrichtenagenturen in der Schweiz. \emph{Verlag Paul Haupt,} Bern.

Höhne, Hansjoachim (1977): Report über Nachrichtenagenturen. Band 1: Die
Situation auf den Nachrichtenmärkten der Welt. Band 2: Die Geschichte
der Nachricht und ihrer Verbreiter. \emph{Nomos Verlagsgesellschaft,}
Baden-Baden.

Johnston, Jane \& Forde, Susan (2011): The Silent Partner: News Agencies
and 21st Century News. \emph{International Journal of Communication 5
(2011),} p. 195--214.
(\href{http://ijoc.org/index.php/ijoc/article/view/928/519}{PDF})

MacGregor, Phil (2013): International News Agencies. Global eyes that
never blink. In: Fowler-Watt/Allan (ed.): Journalism: New Challenges.
\emph{Centre for Journalism \& Communication Research,} Bournemouth
University.
(\href{https://microsites.bournemouth.ac.uk/cjcr/files/2013/10/JNC-2013-Chapter-3-MacGregor.pdf}{PDF})

Schulten-Jaspers, Yasmin (2013): Zukunft der Nachrichtenagenturen.
Situation, Entwicklung, Prognosen. \emph{Nomos,} Baden-Baden.

Segbers, Michael (2007): Die Ware Nachricht. Wie Nachrichtenagenturen
ticken. \emph{UVK,} Konstanz.

Steffens, Manfred {[}Ziegler, Stefan{]} (1969): Das Geschäft mit der
Nachricht. Agenturen, Redaktionen, Journalisten. \emph{Hoffmann und
Campe}, Hamburg.

Wilke, Jürgen (Hrsg.) (2000): Von der Agentur zur Redaktion.
\emph{Böhlau}, Köln.

\hypertarget{geopolitik}{%
\subparagraph{\texorpdfstring{\textbf{Geopolitik}}{Geopolitik}}\label{geopolitik}}

Barnett, Thomas P.M. (2005): The Pentagon's New Map: War and Peace in
the Twenty-First Century. \emph{Putnam Publishing Group,} New York.
(\href{http://www.esquire.com/news-politics/a1546/thomas-barnett-iraq-war-primer/}{Web})

Blum, William (2014): Killing Hope: US Military and CIA Interventions
Since World War II -- Updated Edition. \emph{ZED BOOKS,} London.*\\
*

Brzezinski, Zbigniew (1998): The Grand Chessboard: American Primacy And
Its Geostrategic Imperatives. \emph{Basic Books,} New York.
(\href{https://archive.org/details/TheGrandChessboardAmericanPrimacyAndItsGeostrategicImperatives1997ZbigniewBrzezinski}{PDF})*\\
*

Brzezinski, Zbigniew (2005): The Choice: Global Domination or Global
Leadership. \emph{Basic Books,} New York.*\\
*

Haass, Richard (2017): A World in Disarray: American Foreign Policy and
the Crisis of the Old Order. \emph{Penguin Press,} London.*\\
*

Kagan, Robert (1998): The Benevolent Empire. \emph{Foreign Policy
Magazine.}
(\href{http://carnegieendowment.org/1998/06/01/benevolent-empire-pub-275}{PDF})

Kissinger, Henry (2015): World Order. \emph{Penguin Books,} London.*\\
*

Sylvan, David \& Majeski, Stephen (2009): U.S. Foreign Policy in
Perspective: Clients, Enemies and Empire. \emph{Routledge}, London.
(\href{http://www.us-foreign-policy-perspective.org/}{Web})

\begin{center}\rule{0.5\linewidth}{\linethickness}\end{center}

Studie teilen auf:
\href{https://twitter.com/intent/tweet?url=https://swprs.org/die-propaganda-matrix/}{Twitter}
/
\href{https://www.facebook.com/share.php?u=https://swprs.org/die-propaganda-matrix/}{Facebook}

\hypertarget{swiss-policy-research}{%
\subsubsection{Swiss Policy Research}\label{swiss-policy-research}}

\begin{itemize}
\tightlist
\item
  \href{https://swprs.org/kontakt/}{Kontakt}
\item
  \href{https://swprs.org/uebersicht/}{Übersicht}
\item
  \href{https://swprs.org/donationen/}{Donationen}
\item
  \href{https://swprs.org/disclaimer/}{Disclaimer}
\end{itemize}

\hypertarget{english}{%
\subsubsection{English}\label{english}}

\begin{itemize}
\tightlist
\item
  \href{https://swprs.org/contact/}{About Us / Contact}
\item
  \href{https://swprs.org/media-navigator/}{The Media Navigator}
\item
  \href{https://swprs.org/the-american-empire-and-its-media/}{The CFR
  and the Media}
\item
  \href{https://swprs.org/donations/}{Donations}
\end{itemize}

\hypertarget{follow-by-email}{%
\subsubsection{Follow by email}\label{follow-by-email}}

Follow

\href{https://wordpress.com/?ref=footer_custom_com}{WordPress.com}.

\protect\hyperlink{}{Up ↑}

Post to

\protect\hyperlink{}{Cancel}

\includegraphics{https://pixel.wp.com/b.gif?v=noscript}
