\protect\hyperlink{content}{Skip to content}

\href{https://swprs.org/}{}

\protect\hyperlink{search-container}{Search}

Search for:

\href{https://swprs.org/}{\includegraphics{https://swprs.files.wordpress.com/2020/05/swiss-policy-research-logo-300.png}}

\href{https://swprs.org/}{Swiss Policy Research}

Geopolitics and Media

Menu

\begin{itemize}
\tightlist
\item
  \href{https://swprs.org}{Start}
\item
  \href{https://swprs.org/srf-propaganda-analyse/}{Studien}

  \begin{itemize}
  \tightlist
  \item
    \href{https://swprs.org/srf-propaganda-analyse/}{SRF / ZDF}
  \item
    \href{https://swprs.org/die-nzz-studie/}{NZZ-Studie}
  \item
    \href{https://swprs.org/der-propaganda-multiplikator/}{Agenturen}
  \item
    \href{https://swprs.org/die-propaganda-matrix/}{Medienmatrix}
  \end{itemize}
\item
  \href{https://swprs.org/medien-navigator/}{Analysen}

  \begin{itemize}
  \tightlist
  \item
    \href{https://swprs.org/medien-navigator/}{Navigator}
  \item
    \href{https://swprs.org/der-propaganda-schluessel/}{Techniken}
  \item
    \href{https://swprs.org/propaganda-in-der-wikipedia/}{Wikipedia}
  \item
    \href{https://swprs.org/logik-imperialer-kriege/}{Kriege}
  \end{itemize}
\item
  \href{https://swprs.org/netzwerk-medien-schweiz/}{Netzwerke}

  \begin{itemize}
  \tightlist
  \item
    \href{https://swprs.org/netzwerk-medien-schweiz/}{Schweiz}
  \item
    \href{https://swprs.org/netzwerk-medien-deutschland/}{Deutschland}
  \item
    \href{https://swprs.org/medien-in-oesterreich/}{Österreich}
  \item
    \href{https://swprs.org/das-american-empire-und-seine-medien/}{USA}
  \end{itemize}
\item
  \href{https://swprs.org/bericht-eines-journalisten/}{Fokus I}

  \begin{itemize}
  \tightlist
  \item
    \href{https://swprs.org/bericht-eines-journalisten/}{Journalistenbericht}
  \item
    \href{https://swprs.org/russische-propaganda/}{Russische Propaganda}
  \item
    \href{https://swprs.org/die-israel-lobby-fakten-und-mythen/}{Die
    »Israel-Lobby«}
  \item
    \href{https://swprs.org/geopolitik-und-paedokriminalitaet/}{Pädokriminalität}
  \end{itemize}
\item
  \href{https://swprs.org/migration-und-medien/}{Fokus II}

  \begin{itemize}
  \tightlist
  \item
    \href{https://swprs.org/covid-19-hinweis-ii/}{Coronavirus}
  \item
    \href{https://swprs.org/die-integrity-initiative/}{Integrity
    Initiative}
  \item
    \href{https://swprs.org/migration-und-medien/}{Migration \& Medien}
  \item
    \href{https://swprs.org/der-fall-magnitsky/}{Magnitsky Act}
  \end{itemize}
\item
  \href{https://swprs.org/kontakt/}{Projekt}

  \begin{itemize}
  \tightlist
  \item
    \href{https://swprs.org/kontakt/}{Kontakt}
  \item
    \href{https://swprs.org/uebersicht/}{Seitenübersicht}
  \item
    \href{https://swprs.org/medienspiegel/}{Medienspiegel}
  \item
    \href{https://swprs.org/donationen/}{Donationen}
  \end{itemize}
\item
  \href{https://swprs.org/contact/}{English}
\end{itemize}

\protect\hyperlink{}{Open Search}

\hypertarget{wikipedia-mauxdfnahmen-bei-missbrauch}{%
\section{Wikipedia: Maßnahmen
bei~Missbrauch}\label{wikipedia-mauxdfnahmen-bei-missbrauch}}

\textbf{Wirkungsvolle Maßnahmen gegen Missbrauch der Wikipedia.}

Die Hierarchie der Wikipedia wird derzeit wesentlich durch ideologische
Gruppierungen
\href{https://swprs.org/wikipedia-manipulation-autoren/}{dominiert}.
Eine Korrektur manipulierter Artikel ist deshalb oftmals nicht direkt
möglich, da Anpassungen blockiert und ungewollte Autoren gesperrt
werden.

Dennoch kann dem Missbrauch der Wikipedia intern und extern wirkungsvoll
begegnet werden.

\hypertarget{auuxdferhalb-der-wikipedia}{%
\paragraph{Außerhalb der Wikipedia}\label{auuxdferhalb-der-wikipedia}}

Bei rufschädigenden Wikipedia-Artikeln bestehen für Betroffene folgende
Möglichkeiten:

\begin{enumerate}
\def\labelenumi{\arabic{enumi}.}
\tightlist
\item
  Klage gegen Wikimedia am eigenen Wohnort
  (\href{https://swprs.org/weiteres-urteil-im-fall-wikipedia/}{mehr
  dazu})
\item
  Klage gegen den verantwortlichen Wikipedia-Autor
  (\href{https://swprs.org/der-wikipedia-prozess/}{mehr dazu})
\item
  Klage gegen Personen/Medien, die den Artikel verbreiten
  (\href{https://www.mll-news.com/oger-zh-liken-ehrverletzender-posts-ist-strafrechtliches-weiterverbreiten-und-ueble-nachrede/}{mehr
  dazu})
\end{enumerate}

Rufschädigung entsteht dabei nicht nur durch falsche
Tatsachen­be­hauptungen, sondern auch durch Vermittlung ehrverletzender
Teil­wahr­heiten, d.h. Verschweigen oder Entfernen positiver Tatsachen
-- eine gerade in der Wikipedia naheliegende Form der Rufschädigung
(\href{https://lexetius.com/1999,1148}{mehr dazu}).

Im August 2018 wurde Wikipedia durch das Landgericht Berlin zudem das
sogenannte »Laienprivileg«
\href{https://www.heise.de/newsticker/meldung/Urteil-gegen-Wikipedia-Keine-rufschaedigende-Kritik-ohne-Recherche-4209610.html}{aberkannt}.
Dies bedeutet, dass Wikipedia-Autoren rufschädigende Behauptungen auch
aus Medien­beiträgen oder anderen Quellen nicht mehr ungeprüft
übernehmen dürfen.

Die Bestimmung des für eine Textstelle verantwortlichen Autors erfolgt
mit dem Tool \href{https://f-squared.org/whovisual/}{WikiWho}. Die
Bestimmung des für eine Löschung verantwortlichen Autors erfolgt durch
die
\href{https://de.wikipedia.org/wiki/Hilfe:Versionen}{Versionsgeschichte}.

Auch die ungerechtfertigte Sperrung eines Benutzers dürfte juristisch
anfechtbar sein. Die oftmals arbiträren Wikipedia-internen Vorgänge und
Regeln sind dabei aufgrund der monopol­­artigen Stellung und
gesell­schaft­lichen Bedeutung des Online-Lexikons rechtlich kaum von
Belang.

Bei systematisch rufschädigenden Bearbeitungen ist es denkbar, dass
einem Autor die Mitarbeit an der Wikipedia gerichtlich gänzlich
untersagt werden kann.

Schätzungen zufolge existieren derzeit allein im deutschsprachigen Raum
bis zu eintausend gegen Wikipedia und deren Autoren klageberechtigte
Personen und Organisationen. Auf öffentliche üble Nachrede
\href{https://dejure.org/gesetze/StGB/186.html}{stehen} bis zu zwei, auf
öffentliche Verleumdung bis zu fünf Jahre Haftstrafe.

Alternativ zum Strafrecht können Betroffene auch anwaltliche Abmahnungen
erwirken.

\includegraphics{https://swprs.files.wordpress.com/2019/03/transparenz2.png?w=736}

\hypertarget{innerhalb-der-wikipedia}{%
\paragraph{Innerhalb der Wikipedia}\label{innerhalb-der-wikipedia}}

Innerhalb der Wikipedia besteht die Möglichkeit, manipulative
Admini­stra­toren jederzeit
\href{https://de.wikipedia.org/wiki/Wikipedia:Adminwiederwahl}{abzuwählen}:
25 Antragsteller können eine Neuwahl ver­la­ngen, bei der sich ein
Drittel der Teil­neh­menden gegen den fraglichen Admi­ni­strator
aussprechen muss. Hierzu
\href{https://de.wikipedia.org/wiki/Wikipedia:Adminkandidaturen/Archiv/2018}{reichen}
meist 100 bis 150 Stimmen.

\href{https://de.wikipedia.org/wiki/Wikipedia:Stimmberechtigung}{Stimmberechtigt}
sind alle Wikipedia-Autoren, die seit mindestens zwei Monaten aktiv
mitarbeiten sowie mindestens 200 Bearbeitungen vorgenommen haben, davon
mindestens 50 in den zurück­liegenden 12 Monaten. Derzeit gibt es circa
3000 stimmberechtigte Wikipedia-Autoren.

Administratoren mit
\href{https://de.wikipedia.org/wiki/Hilfe:Benutzer}{Sonderfunktionen}
(Aufseher, Bürokraten, Checkuser, etc.) können nicht unmittelbar
abgewählt werden; sie müssen sich jedoch alle zwei Jahre einer
Wiederwahl stellen. Manipulative Sichter können durch seriöse
Administratoren jederzeit gesperrt werden.

Generell ist zu erwägen, Biographien lebender Personen und andere
sensible Artikel nur noch von besonders verantwortungsvollen und
identifizierbaren Autoren schreiben zu lassen.

\hypertarget{medienbelege}{%
\paragraph{Medienbelege}\label{medienbelege}}

Die deutschsprachige Wikipedia
\href{https://de.wikipedia.org/wiki/Wikipedia:Belege}{akzeptiert}
derzeit primär traditionelle Medien als Belege. Durch eine
\href{https://de.wikipedia.org/wiki/Wikipedia:Meinungsbilder}{Abstimmung}
können jedoch zwecks Verbreiterung der Quellenbasis auch bedeutende
Internet­medien zugelassen werden, wie dies in der englischsprachigen
Wikipedia bereits der Fall ist.

Der Fall »Relotius« -- von dem auch die Wikipedia durch Artikelquellen
betroffen war -- hat gezeigt, dass es für die ohnehin anachronistische
Bevorzugung traditioneller Medien keine legitime Begründung gibt.

\hypertarget{siehe-auch}{%
\paragraph{Siehe auch}\label{siehe-auch}}

\begin{itemize}
\tightlist
\item
  \href{https://de.wikipedia.org/wiki/Wikipedia:Rechtsprechung_zur_Wikipedia}{Rechtsprechung
  zur Wikipedia}
\item
  \href{https://swprs.org/propaganda-in-der-wikipedia/}{Propaganda in
  der Wikipedia}
\item
  \href{https://swprs.org/wikipedia-manipulation-autoren/}{Ideologisch
  agierende Autoren}
\end{itemize}

\begin{center}\rule{0.5\linewidth}{\linethickness}\end{center}

Beitrag teilen auf:
\href{https://twitter.com/intent/tweet?url=https://swprs.org/wikipedia-missbrauch-massnahmen/}{Twitter}
/
\href{https://www.facebook.com/share.php?u=https://swprs.org/wikipedia-missbrauch-massnahmen/}{Facebook}

Publiziert: März 2019

\hypertarget{swiss-policy-research}{%
\subsubsection{Swiss Policy Research}\label{swiss-policy-research}}

\begin{itemize}
\tightlist
\item
  \href{https://swprs.org/kontakt/}{Kontakt}
\item
  \href{https://swprs.org/uebersicht/}{Übersicht}
\item
  \href{https://swprs.org/donationen/}{Donationen}
\item
  \href{https://swprs.org/disclaimer/}{Disclaimer}
\end{itemize}

\hypertarget{english}{%
\subsubsection{English}\label{english}}

\begin{itemize}
\tightlist
\item
  \href{https://swprs.org/contact/}{About Us / Contact}
\item
  \href{https://swprs.org/media-navigator/}{The Media Navigator}
\item
  \href{https://swprs.org/the-american-empire-and-its-media/}{The CFR
  and the Media}
\item
  \href{https://swprs.org/donations/}{Donations}
\end{itemize}

\hypertarget{follow-by-email}{%
\subsubsection{Follow by email}\label{follow-by-email}}

Follow

\href{https://wordpress.com/?ref=footer_custom_com}{WordPress.com}.

\protect\hyperlink{}{Up ↑}

Post to

\protect\hyperlink{}{Cancel}

\includegraphics{https://pixel.wp.com/b.gif?v=noscript}
