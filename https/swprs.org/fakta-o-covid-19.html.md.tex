\protect\hyperlink{content}{Skip to content}

\href{https://swprs.org/}{}

\protect\hyperlink{search-container}{Search}

Search for:

\href{https://swprs.org/}{\includegraphics{https://swprs.files.wordpress.com/2020/05/swiss-policy-research-logo-300.png}}

\href{https://swprs.org/}{Swiss Policy Research}

Geopolitics and Media

Menu

\begin{itemize}
\tightlist
\item
  \href{https://swprs.org}{Start}
\item
  \href{https://swprs.org/srf-propaganda-analyse/}{Studien}

  \begin{itemize}
  \tightlist
  \item
    \href{https://swprs.org/srf-propaganda-analyse/}{SRF / ZDF}
  \item
    \href{https://swprs.org/die-nzz-studie/}{NZZ-Studie}
  \item
    \href{https://swprs.org/der-propaganda-multiplikator/}{Agenturen}
  \item
    \href{https://swprs.org/die-propaganda-matrix/}{Medienmatrix}
  \end{itemize}
\item
  \href{https://swprs.org/medien-navigator/}{Analysen}

  \begin{itemize}
  \tightlist
  \item
    \href{https://swprs.org/medien-navigator/}{Navigator}
  \item
    \href{https://swprs.org/der-propaganda-schluessel/}{Techniken}
  \item
    \href{https://swprs.org/propaganda-in-der-wikipedia/}{Wikipedia}
  \item
    \href{https://swprs.org/logik-imperialer-kriege/}{Kriege}
  \end{itemize}
\item
  \href{https://swprs.org/netzwerk-medien-schweiz/}{Netzwerke}

  \begin{itemize}
  \tightlist
  \item
    \href{https://swprs.org/netzwerk-medien-schweiz/}{Schweiz}
  \item
    \href{https://swprs.org/netzwerk-medien-deutschland/}{Deutschland}
  \item
    \href{https://swprs.org/medien-in-oesterreich/}{Österreich}
  \item
    \href{https://swprs.org/das-american-empire-und-seine-medien/}{USA}
  \end{itemize}
\item
  \href{https://swprs.org/bericht-eines-journalisten/}{Fokus I}

  \begin{itemize}
  \tightlist
  \item
    \href{https://swprs.org/bericht-eines-journalisten/}{Journalistenbericht}
  \item
    \href{https://swprs.org/russische-propaganda/}{Russische Propaganda}
  \item
    \href{https://swprs.org/die-israel-lobby-fakten-und-mythen/}{Die
    »Israel-Lobby«}
  \item
    \href{https://swprs.org/geopolitik-und-paedokriminalitaet/}{Pädokriminalität}
  \end{itemize}
\item
  \href{https://swprs.org/migration-und-medien/}{Fokus II}

  \begin{itemize}
  \tightlist
  \item
    \href{https://swprs.org/covid-19-hinweis-ii/}{Coronavirus}
  \item
    \href{https://swprs.org/die-integrity-initiative/}{Integrity
    Initiative}
  \item
    \href{https://swprs.org/migration-und-medien/}{Migration \& Medien}
  \item
    \href{https://swprs.org/der-fall-magnitsky/}{Magnitsky Act}
  \end{itemize}
\item
  \href{https://swprs.org/kontakt/}{Projekt}

  \begin{itemize}
  \tightlist
  \item
    \href{https://swprs.org/kontakt/}{Kontakt}
  \item
    \href{https://swprs.org/uebersicht/}{Seitenübersicht}
  \item
    \href{https://swprs.org/medienspiegel/}{Medienspiegel}
  \item
    \href{https://swprs.org/donationen/}{Donationen}
  \end{itemize}
\item
  \href{https://swprs.org/contact/}{English}
\end{itemize}

\protect\hyperlink{}{Open Search}

\hypertarget{fakta-o-covid-19-ux10desky}{%
\section{Fakta o Covid-19~(Česky)}\label{fakta-o-covid-19-ux10desky}}

\textbf{Aktualizace}: 18 května, 2020; \textbf{Publikováno}: 14 březen,
2020; \\
\textbf{Jazyky}: \href{https://swprs.org/covid-19-hinweis-ii/}{DE},
\href{https://swprs.org/a-swiss-doctor-on-covid-19/}{EN},
\href{https://swprs.org/hechos-sobre-covid-19/}{ES},
\href{https://swprs.org/faktoja-covid-19sta/}{FI},
\href{https://swprs.org/coronavirus-un-medecin-suisse-parle/}{FR},
\href{https://swprs.org/facts-about-covid19-greek/}{GR},
\href{https://swprs.org/covid-19-cinjenice/}{HBS},
\href{https://yanivhamo.com/facts-about-covid-19-hebrew/}{HE},
\href{https://swprs.org/egy-svajci-orvos-a-covid-19-rol/}{HU},
\href{https://swprs.org/un-medico-svizzero-su-covid-19/}{IT},
\href{https://swprs.org/covid19-facts-japanese/}{JP},
\href{https://swprs.org/covid19-korean/}{KO},
\href{https://www.globalinfo.nl/Achtergrond/een-kritische-kijk-op-het-coronabeleid-transparantie-in-tijden-van-crisis}{NL},
\href{https://midtifleisen.wordpress.com/2020/03/14/en-sveitsisk-lege-om-covid-19/}{NO},
\href{https://swprs.org/szwajcarski-lekarz-o-covid-19/}{PL},
\href{https://swprs.org/fatos-sobre-covid-19/}{PT},
\href{https://swprs.org/informatii-despre-covid-19/}{RO},
\href{https://swprs.org/\%d0\%bd\%d0\%b0-\%d0\%ba\%d0\%be\%d0\%b2\%d0\%b8\%d0\%b4-19/}{RU},
\href{https://swprs.org/fakta-om-covid-19/}{SE},
\href{http://www.ninamvseeno.org/pregled-clanka.aspx?naslov=pomembne-informacije-o-novem-koronavirusu-covid-19\&id=148}{SI},
\href{https://alatyr.sk/covid-19_swiss_propaganda_research.htm}{SK},
\href{https://swprs.org/isvicreli-bir-doktordan-kovid-19-uezerine/}{TR}\\
\textbf{Sdílejte to}:
\href{https://twitter.com/intent/tweet?url=https://swprs.org/fakta-o-covid-19/}{Twitter}
/
\href{https://www.facebook.com/share.php?u=https://swprs.org/fakta-o-covid-19/}{Facebook}\\

Plně ozdrojovaná fakta o Covid-19 poskytnutá odborníky v oboru,
umožňující našim čtenářům učinit realistický odhad rizika. (aktualizace
níže).

\textbf{``Jediným prostředkem v boji s nákazou je upřímnost.'' Albert
Camus, The Plague (1947)}

\hypertarget{pux159ehled},
  což je srovnatelné se
  \href{https://www.ebm-netzwerk.de/en/publications/covid-19}{závažnou
  chřipkou}, a asi dvacetkrát nižší, než původně
  \href{https://www.businessinsider.com/coronavirus-death-rate-by-age-countries-2020-3}{předpokládala}
  Světová zdravotnická organizace (WHO).
\item
  I v globálních „hotspotech`` (tj. nejvíce zasažené oblasti světa) je
  riziko úmrtí pro obecnou populaci školního a produktivního věku
  srovnatelné s
  \href{https://www.medrxiv.org/content/10.1101/2020.04.05.20054361v1}{každodenní
  jízdou autem} do práce. Riziko bylo zpočátku přeceňováno, protože
  mnoho lidí s pouze mírnými nebo žádnými příznaky nebylo započítano do
  statistik.
\item
  Až 80\% všech osob s pozitivním testem
  \href{https://www.bmj.com/content/369/bmj.m1375}{zůstává bez
  příznaků}. Dokonce i ve věkové skupině 70-79 let,
  \href{https://www.niid.go.jp/niid/en/2019-ncov-e/9407-covid-dp-fe-01.html}{asi
  60\%} osob zůstává bez příznaků. Více než 95\% všech osob vykazuje
  nanejvýš
  \href{https://swprs.org/studies-on-covid-19-lethality/\#hospitalizations}{mírné
  příznaky}.
\item
  Až třetina všech osob již má určitou
  \href{https://www.medrxiv.org/content/10.1101/2020.04.17.20061440v1}{předchozí
  imunitu} vůči Covid-19 díky kontaktu s koronaviry (tj. běžnými
  nachlazeními) v minulosti.
\item
  Střední věk (medián) zemřelých je ve většině zemí (včetně
  \href{https://www.epicentro.iss.it/coronavirus/sars-cov-2-decessi-italia}{Itálie})
  přes 80 let a osoby bez závažných předchozích nemocí tvoří pouze
  \href{https://www.bloomberg.com/news/articles/2020-03-18/99-of-those-who-died-from-virus-had-other-illness-italy-says}{asi
  1\%} zemřelých. Věkový a rizikový profil úmrtí tedy v podstatě
  odpovídá
  \href{https://www.vienna.at/analyse-zeigt-covid-19-opferkurve-entspricht-normaler-mortalitaet/6581246}{normální
  úmrtnosti}.
\item
  Ve většině západních zemí se 50 až 70\% všech úmrtí odehrálo
  \href{https://ltccovid.org/2020/04/12/mortality-associated-with-covid-19-outbreaks-in-care-homes-early-international-evidence/}{v
  domovech s pečovatelskou službou}, které jsou zasaženy všeobecným
  omezením pohybu. Navíc v mnoha případech
  \href{https://www.hsj.co.uk/commissioning/thousands-of-extra-deaths-outside-hospital-not-attributed-to-covid-19/7027459.article}{není
  jasné}, zda tito lidé skutečně zemřeli na Covid-19 nebo kvůli
  \href{https://www.nytimes.com/2020/04/16/world/canada/montreal-nursing-homes-coronavirus.html}{extrémnímu
  stresu}, strachu či osamělosti.
\item
  Až 50\% nadměrné úmrtnosti mohlo být způsobeno
  \href{https://www.thetimes.co.uk/edition/news/coronavirus-record-weekly-death-toll-as-fearful-patients-avoid-hospitals-bm73s2tw3}{nikoliv
  virem Covid-19}, ale důsledkem
  \href{https://www.telegraph.co.uk/global-health/science-and-disease/two-new-waves-deaths-break-nhs-new-analysis-warns/}{omezení
  pohybu, paniky a strachu}. Například počet hospitalizací pro srdeční
  infarkty a mozkové příhody se
  \href{https://www.nytimes.com/2020/04/06/well/live/coronavirus-doctors-hospitals-emergency-care-heart-attack-stroke.html}{snížil}
  až o 60\%, protože mnoho pacientů se v současnosti neodváží jít do
  nemocnice.
\item
  Dokonce ani v takzvaných „obětí koronaviru``
  \href{https://spectator.us/understand-report-figures-covid-deaths/}{není
  často objasněno}, jestli byl příčinou úmrtí tento virus nebo jiná, již
  existující nemoc, a jestli jsou do této statistiky zahrnuty
  \href{https://www.youtube.com/watch?v=V0lIWZpiRU0}{„předpokládané
  případy``}, tj. mrtví, kteří vůbec nebyli testováni. Oficiální údaje
  všechny tyto rozdíly obvykle
  \href{https://www.hsj.co.uk/coronavirus/systematic-reviews-to-discover-true-cause-of-outbreak-deaths/7027491.article}{nezohledňují}.
  (česky:
  \href{https://www.idnes.cz/technet/veda/co-se-o-koronaviru-rika-spatne-mylne-informace-zkreslene-covid-19-koronavirus-statistik-karel-helman.A200423_113928_veda_mla}{Statistik:
  Ve zprávách o covid-19 dělají novináři opakovaně vážné chyby})
\item
  Ukázalo se, že mnoho mediálních zpráv o mladých a zdravých lidech,
  kteří zemřeli na Covid-19, bylo nepravdivých: mnoho z těchto mladých
  lidí buď
  \href{https://www.dailymail.co.uk/news/article-8193487/Coroner-refuses-rule-COVID-19-cause-death-six-week-old-Connecticut-baby.html}{nezemřelo}
  na Covid-19, nebo byli již
  \href{https://sports.yahoo.com/spanish-football-coach-francisco-garcia-163153573.html}{vážně
  nemocní} (např. z nediagnostikované leukémie jako v~případě
  španělského fotbalového trenéra), nebo jim ve skutečnosti bylo
  \href{https://www.tagesanzeiger.ch/bund-muss-in-seiner-todesfallstatistik-fehler-korrigieren-584308129723}{ne
  9, ale 109 let}.
\item
  Normální celková denní úmrtnost je v USA
  \href{https://www.cdc.gov/mmwr/volumes/68/wr/mm6826a5.htm}{asi 8 000
  lidí}, v Německu asi 2 600 a v Itálii asi 1 800 lidí. Počet úmrtí
  za~chřipkovou sezónu je v USA
  \href{https://www.statnews.com/2018/09/26/cdc-us-flu-deaths-winter/}{až
  80 000}, v Německu i Itálii
  \href{https://www.sciencedirect.com/science/article/pii/S1201971219303285}{až
  25 000}. V několika zemích je celkový počet úmrtí navzdory Covid-19
  \href{https://www.euromomo.eu/graphs-and-maps/}{menší než} při silných
  chřipkových sezónách v minulosti.
\item
  Regionální nárůst úmrtnosti může být ovlivněn dalšími rizikovými
  faktory, jako je
  \href{https://www.theguardian.com/environment/2020/apr/20/air-pollution-may-be-key-contributor-to-covid-19-deaths-study?utm_medium}{vysoká
  úroveň znečištění ovzduší} a mikrobiální kontaminace, jakož i
  \href{https://swprs.org/covid-19-a-report-from-italy/}{kolaps péče o
  starší a nemocné} v důsledku infekcí, hromadné paniky a omezení
  pohybu. Zvláštní
  \href{https://www.ecdc.europa.eu/sites/default/files/documents/COVID-19-safe-handling-of-bodies-or-persons-dying-from-COVID19.pdf}{předpisy}
  pro jednání se zesnulými někdy vedly k dalším komplikacím v pohřebních
  nebo kremačních službách.
\item
  V zemích, jako je Itálie a Španělsko a do jisté míry i ve Velké
  Británii a USA,
  \href{https://off-guardian.org/2020/04/02/coronavirus-fact-check-1-flu-doesnt-overwhelm-our-hospitals/}{není}
  přetížení nemocnic v důsledku silných chřipkových vln neobvyklé. Navíc
  bylo až 15\% lékařů a zdravotnického personálu
  \href{https://www.reuters.com/article/us-health-coronavirus-spain-morgue-idUSKBN21B1PP}{zařazeno
  do karantény}, i když se u nich nevyskytly žádné příznaky.
\item
  Často uváděné exponenciální křivky „koronových případů`` jsou
  \href{https://fivethirtyeight.com/features/coronavirus-case-counts-are-meaningless/}{zavádějící},
  protože i počet testů exponenciálně vzrostl. Ve většině zemí zůstal
  poměr pozitivních testů k celkovým testům (tj. pozitivní míra)
  \href{https://swprs.org/rate-of-positive-covid19-tests/}{konstantní na
  5\% až 25\%} nebo se zvýšil jen nepatrně. V mnoha zemích začlo šíření
  zpomalovat již
  \href{https://www.dailymail.co.uk/news/article-8235979/UKs-coronavirus-crisis-peaked-lockdown-Expert-argues-draconian-measures-unnecessary.html}{před
  omezeními pohybu}.
\item
  Země bez zákazu vycházení a zákazů kontaktu, jako je
  \href{https://www.japantimes.co.jp/news/2020/03/20/national/coronavirus-explosion-expected-japan/}{Japonsko},
  \href{https://www.businessinsider.com/south-korea-coronavirus-testing-death-rate-2020-3?op=1}{Jižní
  Korea} nebo
  \href{https://www.youtube.com/watch?v=bfN2JWifLCY}{Švédsko},
  \href{https://www.washingtontimes.com/news/2020/apr/15/sweden-coronavirus-rates-easing-despite-loose-rule/}{nezažily}
  negativnější průběh událostí než jiné země. Světová zdravotnická
  organizace (WHO) dokonce
  \href{https://nypost.com/2020/04/29/who-lauds-sweden-as-model-for-resisting-coronavirus-lockdown/}{ocenila}
  Švédsko, které nyní těží z vyšší imunity ve srovnání s~zeměmi s
  omezeními.
\item
  Strach z nedostatku ventilátorů byl
  \href{https://apnews.com/8ccd325c2be9bf454c2128dcb7bd616d}{neopodstatněný}.
  Podle odborníků na plícní onemocnění je invazivní ventilace (intubace)
  pacientů s~Covid-19, často používaná
  \href{https://www.dailymail.co.uk/news/article-8262351/Nurse-New-York-claims-city-killing-COVID-19-patients-putting-ventilators.html}{kvůli
  strachu} z rozšíření viru, ve skutečnosti mnohdy
  \href{https://www.medscape.com/viewarticle/928156}{kontraproduktivní}
  a poškozuje plíce.
\item
  Na rozdíl od původních předpokladů různé studie ukázaly, že
  \href{https://www.who.int/news-room/commentaries/detail/modes-of-transmission-of-virus-causing-covid-19-implications-for-ipc-precaution-recommendations}{neexistuje
  důkaz} o tom, že by se virus šířil aerosoly (tj. částicemi proudícími
  se vzduchem) nebo prostřednictvím
  \href{https://www.telegraph.co.uk/news/2020/04/02/no-proof-coronavirus-can-spread-shopping-says-leading-german/}{povrchových
  infekcí} (např. na klikách dveří, smartphonech nebo v kadeřnictví).
\item
  Neexistují ani
  \href{https://www.researchgate.net/publication/340570735_Masks_Don't_Work_A_review_of_science_relevant_to_COVID-19_social_policy}{vědecké
  důkazy} o účinnosti roušek u zdravých nebo asymptomatických jedinců.
  Naopak odborníci varují, že takové masky narušují normální dýchání a
  mohou se stát
  \href{https://de.sputniknews.com/interviews/20200425326953541-corona-gefahr-virologe/}{„nositeli
  nákazy``}. Přední lékaři to označují za „mediální humbuk`` a
  \href{https://www.aerztezeitung.de/Politik/Montgomery-haelt-Maskenpflicht-fuer-falsch-408844.html}{„absurditu``}.
\item
  Mnoho klinik v Evropě a USA zůstalo i během vrcholu nákazy Covid-19
  \href{https://www.hsj.co.uk/acute-care/nhs-hospitals-have-four-times-more-empty-beds-than-normal/7027392.article}{silně
  nevyužitých} nebo téměř prázdných a v některých případech muselo
  \href{https://www.usatoday.com/story/news/health/2020/04/02/coronavirus-pandemic-jobs-us-health-care-workers-furloughed-laid-off/5102320002/}{poslat
  zaměstnance domů}. Bylo
  \href{https://www.sfchronicle.com/bayarea/article/Stanford-hospital-system-to-cut-pay-20-furlough-15227591.php}{zrušeno}
  mnoho operací a terapií, včetně transplantací orgánů a screeningu
  rakoviny.
\item
  Několik médií bylo přistiženo jak se pokoušejí
  \href{https://nypost.com/2020/04/01/cbs-admits-to-using-footage-from-italy-in-report-about-nyc/}{dramatizovat}
  situaci v nemocnicích, někdy dokonce s manipulativními obrázky a
  videi.
  \href{https://onlinelibrary.wiley.com/doi/full/10.1111/eci.13222}{Neprofesionální
  zpravodajství} mnoha médií šířilo všeobecný strach a paniku mezi
  lidmi.
\item
  Mezinárodně používané virové testovací sady jsou
  \href{https://www.ncbi.nlm.nih.gov/pubmed/32219885}{náchylné k chybám}
  a mohou vykazovat falešně pozitivním nebo falešně negativním výsledky.
  Navíc oficiální virový test nebyl kvůli nedostatku času
  \href{https://www.youtube.com/watch?v=p_AyuhbnPOI}{klinicky schválen}
  a někdy může vyjít pozitivně při přítomnosti jiných koronavirů.
\item
  Četní
  \href{https://off-guardian.org/2020/03/24/12-experts-questioning-the-coronavirus-panic/}{mezinárodně
  uznávaní odborníci} v oblasti virologie, imunologie a epidemiologie
  považují přijatá opatření za
  \href{https://off-guardian.org/2020/03/28/10-more-experts-criticising-the-coronavirus-panic/}{kontraproduktivní}
  a doporučují rychlou
  \href{https://off-guardian.org/2020/04/17/8-more-experts-questioning-the-coronavirus-panic/}{přirozenou
  imunizaci} obecné populace a ochranu rizikových skupin. Rizika pro
  děti jsou
  \href{https://www.thelancet.com/journals/lanchi/article/PIIS2352-4642(20)30095-X/fulltext}{prakticky
  nulová} a uzavření škol nebylo z~lékařského hlediska nikdy
  opodstatněno.
\item
  Několik odborných lékařů označilo vakcíny proti koronavirům za
  \href{https://www.youtube.com/watch?v=vrL9QKGQrWk}{zbytečné} nebo
  dokonce
  \href{https://www.nature.com/articles/d41586-020-00751-9}{nebezpečné}.
  Například vakcína proti
  \href{https://www.forbes.com/2010/02/05/world-health-organization-swine-flu-pandemic-opinions-contributors-michael-fumento.html\#658c006c48e8}{tzv.
  prasečí chřipce} z roku 2009 ve skutečnosti někdy vedla k
  \href{https://www.ibtimes.co.uk/brain-damaged-uk-victims-swine-flu-vaccine-get-60-million-compensation-1438572}{závažným
  neurologickým poruchám} a soudním sporům o náhradu škody v~řádu
  milionů.
\item
  Počet lidí trpících nezaměstnaností,
  \href{https://www.indystar.com/story/news/health/2020/04/03/coronavirus-indiana-how-get-help-mental-health-addiction/5104357002/}{psychickými
  problémy} a domácím násilím v důsledku těchto opatření
  \href{https://www.reuters.com/article/us-health-coronavirus-usa-layoffs/us-weekly-jobless-claims-seen-at-record-high-again-idUSKBN21K0FX}{prudce
  stoupá} po celém světě. Několik odborníků se domnívá, že omezení mohou
  vyžadovat
  \href{https://www.nytimes.com/2020/03/20/opinion/coronavirus-pandemic-social-distancing.html}{více
  životů} než samotný virus. Podle OSN mohou
  \href{https://www.theguardian.com/global-development/2020/apr/21/coronavirus-pandemic-will-cause-famine-of-biblical-proportions}{miliony
  lidí} na celém světě upadnout do absolutní chudoby a hladomoru.
\item
  Whistleblower Edward Snowden varoval, že důsledkem „koronové krize``
  bude \href{https://www.youtube.com/watch?v=-pcQFTzck_c}{rozsáhlé a
  trvalé rozšíření globálního sledování}. Proslulý virolog Pablo
  Goldschmidt
  \href{https://www.rubikon.news/artikel/der-corona-totalitarismus}{hovořil}
  o „globálním mediálním teroru`` a „totalitních opatřeních``. Přední
  britský virologický profesor John Oxford
  \href{https://novuscomms.com/2020/03/31/a-view-from-the-hvivo-open-orphan-orph-laboratory-professor-john-oxford/}{hovořil}
  o „mediální epidemii``.
\item
  Více než 500 vědců
  \href{https://www.esat.kuleuven.be/cosic/sites/contact-tracing-joint-statement/}{varovalo}
  před „bezprecedentním špehování společnosti`` prostřednictvím
  problematických aplikací umožňujících „chytrou karanténu``. V
  některých zemích je taková „chytrá karanténa`` již prováděna
  \href{https://www.jewishpress.com/news/the-courts/state-to-high-court-even-more-shin-bet-involvement-in-fighting-the-coronavirus/2020/04/14/}{kontrarozvědkou}.
  V několika částech světa je již populace
  \href{https://off-guardian.org/2020/04/25/50-headlines-darker-more-of-the-new-normal/}{sledována
  drony} a čelí závážné policejní šikaně.
\end{enumerate}

\hypertarget{viz-takuxe9-otevux159enuxfd-dopis-profesora-bhakdiho-nux11bmeckuxe9-kancluxe9ux159ce-merkelovuxe9}{%
\subparagraph{\texorpdfstring{\textbf{Viz také}:
\href{https://swprs.org/open-letter-from-professor-sucharit-bhakdi-to-german-chancellor-dr-angela-merkel/}{Otevřený
dopis} profesora Bhakdiho německé kancléřce
Merkelové.}{Viz také: Otevřený dopis profesora Bhakdiho německé kancléřce Merkelové.}}\label{viz-takuxe9-otevux159enuxfd-dopis-profesora-bhakdiho-nux11bmeckuxe9-kancluxe9ux159ce-merkelovuxe9}}

\begin{center}\rule{0.5\linewidth}{\linethickness}\end{center}

\hypertarget{novux11bjux161uxed-aktualizace-v-angliux10dtinux11b-nebo-nux11bmux10dinux11b}{%
\paragraph{\texorpdfstring{Novější aktualizace v
\href{https://swprs.org/a-swiss-doctor-on-covid-19/}{angličtině} nebo
\href{https://swprs.org/covid-19-hinweis-ii/}{němčině}}{Novější aktualizace v angličtině nebo němčině}}\label{novux11bjux161uxed-aktualizace-v-angliux10dtinux11b-nebo-nux11bmux10dinux11b}}

\hypertarget{25-dubna-2020}{%
\paragraph{25. dubna 2020}\label{25-dubna-2020}}

\hypertarget{luxe9kaux159skuxe9-aktualizace}{%
\subparagraph{\texorpdfstring{\textbf{Lékařské
aktualizace:}}{Lékařské aktualizace:}}\label{luxe9kaux159skuxe9-aktualizace}}

\begin{itemize}
\tightlist
\item
  Video: Kalifornský lékař Dr. Dan Erickson popsal na tiskové konferenci
  \href{https://www.turnto23.com/news/coronavirus/video-interview-with-dr-dan-erickson-and-dr-artin-massihi-taken-down-from-youtube}{svá
  pozorování ohledně Covid -19}. Nemocnice a jednotky intenzivní péče v
  Kalifornii a dalších státech dosud zůstaly do značné míry prázdné. Dr.
  Erickson uvádí, že lékaři z několika amerických států byli „pod
  tlakem``, aby udávali Covid-19 jako příčinu smrti na úmrtních listech,
  i když oni sami nesouhlasili. Dr. Erickson tvrdí, že zákazy vzcházení
  a strach oslabují imunitní systém a zdraví lidí. Již došlo k výraznému
  nárůstu „sekundárních účinků``, jako je alkoholismus, deprese,
  sebevražda a zneužívání dětí a manželů. Dr. Erickson doporučuje, aby
  do karantény byli zařazeni pouze nemocní, nikoli zdraví, natož celá
  společnost. Podle Dr. Ericksona mají roušky smysl pouze v akutních
  situacích, jako je v nemocnici, ale ne v každodenním životě.
\item
  Profesor Detlef Krüger, přímý předchůdce známého německého virologa
  Christiana Drostena na klinice Charité v Berlíně,
  \href{https://de.sputniknews.com/interviews/20200425326953541-corona-gefahr-virologe/}{v
  nedávném rozhovoru vysvětluje}, že Covid-19 je „v mnoha ohledech
  srovnatelná s chřipkou`` a „není nebezpečnější než určité varianty
  chřipkového viru``. Profesor Krüger považuje „ochranu úst a nosu
  objevenou politiky`` za „akcionismus`` a potenciální „rozsévač
  nákazy``. Zároveň varuje před „masivními kolaterálními škodami``
  způsobeným přijatými opatřeními.
\item
  Švýcarští patologové dospěli k závěru, že mnoho z testovaných
  pozitivních zemřelých
  \href{https://www.welt.de/wissenschaft/article207417811/Corona-Tote-In-den-wenigsten-Faellen-eine-Lungenentzuendung.html}{netrpělo
  zápalem plic}, ale mělo poruchu krevních cév a tím i kyslíkové výměny
  v plicích. To by mohlo vysvětlit, proč ventilace kritických pacientů
  Covid-19 často není účinná a proč pacienti s existujícími
  kardiovaskulárními problémy patří do rizikové skupiny. Ve skutečnosti
  všichni pitvaní pacienti trpěli vysokým krevním tlakem, velká část
  byla vážně obézní a dvě třetiny měly již dříve poškozené koronární
  tepny.
\item
  \href{https://www.epicentro.iss.it/coronavirus/bollettino/Bollettino-sorveglianza-integrata-COVID-19_16-aprile-2020.pdf\#page=13}{Nejnovější
  údaje z Itálie} ukazují (s. 12/13), že 60 z téměř 17 000 lékařů a
  sester, kteří byli pozitivně testováni, zemřelo. Výsledkem je míra
  úmrtnosti Covid-19 nižší než 0,1\% u osob mladších 50 let, 0,27\% u
  osob ve věku 50 až 60 let, 1,4\% u osob ve věku 60 až 70 let a 12,6\%
  u osob ve věku 70 až 80 let. I tato čísla jsou pravděpodobná příliš
  vysoká, protože se jedná o úmrtí s koronavirem a ne nutně na
  koronavirus, a až 80\% z těchto lidí zůstává asymptomatických a
  někteří možná nebyli testováni. Celkově jsou však hodnoty v souladu s
  hodnotami z dřívějšího testování v Jižní Koreji a nasvědčují tomu, že
  úmrtnost u obecné populace je srovnatelná s chřipkou.
\item
  Vedoucí italské Civilní Obrany
  \href{https://www.theguardian.com/world/2020/apr/16/italian-police-broaden-care-home-coronavirus-milan}{v
  polovině dubna prohlásil}, že v pečovatelských domovech v Lombardii
  zemřelo více než 1,800 lidí a že v mnoha případech nebyla příčina
  smrti dosud jasná. Již dříve vyšlo najevo, že se zhroutila péče o
  seniory v pečovatelských domovech a v důsledku toho i celý zdravotní
  systém v částech Lombardie, částečně
  \href{https://swprs.org/covid-19-a-report-from-italy/}{kvůli strachu z
  viru a omezením pohybu}.
\item
  Také
  \href{https://covid-19.sciensano.be/sites/default/files/Covid19/Meest\%20recente\%20update.pdf}{nejnovější
  údaje z Belgie} ukazují, že k více než 50\% všech dalších úmrtí
  dochází v domovech s pečovatelskou službou, kterým rozhodně neprospíva
  všeobecný zákaz pohybu osob. Covid-19 byl „potvrzen`` v 6\% těchto
  úmrtí, v 94\% úmrtí na něj „bylo podezření``. Přibližně 70\% osob
  testovaných pozitivně (zaměstnanců a obyvatel) nevykazovalo žádné
  příznaky.
\item
  Bývalý švédský a evropský hlavní epidemiolog, profesor Johan Giesecke,
  dal rakouskému časopisu Addendum
  \href{https://www.addendum.org/coronavirus/interview-johan-giesecke/}{upřímný
  rozhovor}. Profesor Giesecke říká, že epidemie je v 75 až 90\% případů
  „neviditelná``, protože u mnoha lidí se nevyskytují žádné nebo téměř
  žádné symptomy. Omezení pohybu by proto bylo „zbytečné`` a poškodilo
  by společnost. Základem švédské strategie bylo, že „lidé nejsou
  hloupí``. Giesecke očekává úmrtnost mezi 0,1 a 0,2\%, podobnou
  úmrtnosti na chřipku. Profesor Giesecke tvrdí, že v Itálii i New Yorku
  byly na virus velmi špatně připraveni a nechránili své rizikové
  skupiny.
\item
  Britský deník The Guardian cituje
  \href{https://www.theguardian.com/environment/2020/apr/20/air-pollution-may-be-key-contributor-to-covid-19-deaths-study?utm_medium}{nové
  studie}, podle nichž by znečištění ovzduší mohlo být „klíčovým
  faktorem`` úmrtí s Covid-19. Například 80\% úmrtí ve čtyřech zemích
  bylo v nejvíce znečištěných regionech (jako je Lombardie nebo Madrid).
\item
  Německé noviny DIE ZEIT se zaměřují
  \href{https://www.zeit.de/2020/18/kliniken-coronavirus-intensivbetten-patienten-behandlung-notaufnahme}{na
  vysokou míru neobsazenosti} v německých nemocnicích, které v některých
  odděleních dosahují až 70\%. Dokonce i vyšetření rakoviny a
  transplantace orgánů, které nebyly nezbytně nutné pro přežití, byly
  zrušeny, aby se vytvořil prostor pro pacienty s Covid-19, ale tyto
  doposud do značné míry chyběly.
\item
  V Německu bylo nařízeno nosit roušku ve veřejné dopravě a v
  maloobchodních prodejnách. Prezident Světové lékařské asociace Frank
  Montgomery to
  \href{https://www.aerztezeitung.de/Politik/Montgomery-haelt-Maskenpflicht-fuer-falsch-408844.html}{kritizoval}
  jako „špatné`` a zamýšlené použití šátků a roušek jako „směšné``.
  Studie skutečně ukazují, že používání masek v každodenním životě
  nepřináší měřitelné výhody zdravým a asymptomatickým lidem, a proto
  švýcarský infektolog Dr. Vernazza hovořil o
  \href{https://infekt.ch/2020/04/atemschutzmasken-fuer-alle-medienhype-oder-unverzichtbar/}{„mediálním
  humbuku``}. Jiní kritici hovoří o symbolu
  \href{https://multipolar-magazin.de/artikel/maskenpflicht-gesellschaftliches-klima}{„vynucené,
  okázalé poslušnosti``}.
\item
  Studie WHO (Světová Zdravotnická Organizace) z roku 2019 zjistila
  \href{https://www.heise.de/tp/features/COVID-19-WHO-Studie-findet-kaum-Belege-fuer-die-Wirksamkeit-von-Eindaemmungsmassnahmen-4706446.html}{„malý
  až žádný vědecký důkaz``} o účinnosti opatření, jako je „sociální
  distancování``, omezení cestování a zákazy vycházení.
  (\href{https://www.who.int/influenza/publications/public_health_measures/publication/en/}{Původní
  studie})
\item
  Německá laboratoř na začátku dubna
  \href{http://www.labor-augsburg-mvz.de/de/aktuelles/coronavirus}{uvedla},
  že podle doporučení WHO jsou testy viru Covid-19 nyní považovány za
  pozitivní, i když je konkrétní cílová sekvence viru Covid-19 negativní
  a pouze obecnější cílová sekvence koronavire je pozitivní. To však
  může vést k dalším koronovým virům (studeným virům), které také
  vyvolávají falešně pozitivní výsledek testu. Laboratoř také
  vysvětlila, že protilátky Covid-19 jsou často zjistitelné pouze dva až
  tři týdny po nástupu příznaků. Toto je třeba vzít v úvahu, aby
  skutečný počet lidí, kteří jsou již vůči Covid-19 imunní, nebyl
  statisticky podhodnocen.
\item
  Ve
  \href{https://www.20min.ch/schweiz/news/story/-rzte-und-Politiker-fordern-Corona-Impfzwang-20853917}{Švýcarsku}
  i
  \href{https://www.faz.net/agenturmeldungen/dpa/soeder-waere-fuer-deutschlandweite-impfpflicht-gegen-corona-16738369.html}{Německu}
  požadovali někteří politici „povinné očkování proti koroně``. Přestože
  očkování proti tzv. „prasečí chřipce`` v sezónně 2009/2010 vedlo v
  některých případech k
  \href{https://www.ibtimes.co.uk/brain-damaged-uk-victims-swine-flu-vaccine-get-60-million-compensation-1438572}{závažnému
  neurologickému poškození}, zejména u dětí, a k nárokům na náhradu
  škody v řádu milionů.
\item
  Profesor Christopher Kuhbandner:
  \href{https://www.heise.de/tp/features/Von-der-fehlenden-wissenschaftlichen-Begruendung-der-Corona-Massnahmen-4709563.html?seite=all}{O
  nedostatku vědeckého zdůvodnění koronových opatření}: „Hlášené údaje o
  nových infekcích velmi dramaticky přeceňují skutečné šíření koronového
  viru. Pozorovaný rychlý nárůst nových infekcí je téměř výhradně
  způsoben skutečností, že počet testů v průběhu času rychle rostl (viz
  obrázek níže). Alespoň podle uváděných údajů tedy ve skutečnosti nikdy
  nedošlo k exponenciálnímu rozšíření koronaviru. Uváděné údaje o nových
  infekcích skrývají skutečnost, že počet nových infekcí od začátku nebo
  poloviny března klesá. ``
\end{itemize}

\includegraphics{https://swprs.files.wordpress.com/2020/04/zunahme-infektionen-tests-tag.png?w=550\&h=404}

\hypertarget{ux161vuxe9dsko-muxe9dia-versus-realita}{%
\subparagraph{\texorpdfstring{\textbf{Švédsko: Média versus
realita}}{Švédsko: Média versus realita}}\label{ux161vuxe9dsko-muxe9dia-versus-realita}}

Někteří čtenáři byli překvapeni poklesem počtu úmrtí ve Švédsku, protože
většina médií vykazuje prudce rostoucí křivku. Jaký je důvod? Většina
médií uvádí kumulativní údaje podle data hlášení, zatímco švédské úřady
zveřejňují mnohem významnější denní údaje podle data úmrtí.

 Švédské orgány vždy zdůrazňují, že ne všechny nově nahlášené případy za
posledních 24 hodin zemřely, ale mnoho médií to ignoruje (viz graf
níže). Ačkoli nejnovější švédská čísla se mohou, jako ve všech zemích,
stále ještě zvyšit, nic to nemění na tom, že trend je obecně klesající.

 Navíc tato čísla představují úmrtí s koronavirem a ne nutně na
koronavirus. Přibližně 50\% úmrtí se odehrálo v pečovatelských domovech,
zatímco účinek na běžnou populaci zůstal minimální, přestože Švédsko má
jednu z
\href{https://link.springer.com/article/10.1007/s00134-012-2627-8}{nejnižších
kapacit intenzivní péče} v Evropě. Průměrný věk zemřelých ve Švédsku je
přes 80 let.

 Švédská vláda však díky „koroně`` získala nové
\href{https://www.tagesschau.de/faktenfinder/ausland/corona-kursaenderung-schweden-103.html}{nouzové
pravomoci} a mohla by tak pokračovat programy sledování kontaktů i v
budoucnosti.

\includegraphics{https://swprs.files.wordpress.com/2020/04/sweden-corona-media-vs-reality.png?w=736\&h=338}

\hypertarget{situace-ve-velkuxe9-brituxe1nii} dalších úmrtí vyskytuje v domovech s pečovatelskou službou, které
jsou zasaženy všeobecným omezením pohybu.

Navíc \href{https://archive.is/2eKCW}{až 50\%} nedávných úmrtí je
hlášeno jako úmrtí bez Covid-19 a
\href{https://www.ft.com/content/67e6a4ee-3d05-43bc-ba03-e239799fa6ab}{až
25\%} nedávných úmrtí se odehrává doma. Není tedy vůbec jasné, zda je
obecné omezení pohybu prospěšné nebo zda skutečně škodí celé
společnosti.

Frasor Nelson, redaktor týdeníku British Spectator,
\href{https://www.telegraph.co.uk/politics/2020/04/09/boris-worried-lockdown-has-gone-far-can-end/}{prohlásil},
že vládní agentury očekávají, že omezení pohybu povedou v dlouhodobém
horizontu k 150 000 dalším úmrtím, což je výrazně více, než se očekává
od Covid19. Naposledy byl
\href{https://sports.yahoo.com/coronavirus-bethany-palmer-teenager-death-suicide-152707750.html}{zveřejněn}
případ 17leté studentky a zpěvačky, která si kvůli zákazům vycházení
vzala vlastní život.

Je překvapující, že Anglie má na rozdíl od většiny ostatních zemí
(včetně Švédska) výrazně zvýšenou úmrtnost i
\href{https://www.euromomo.eu/}{mezi 15 až 64letými}. Může to být
způsobeno častými kardiovaskulárními problémy nebo to může být způsobeno
důsledky omezení pohybu.

Projekt \href{http://inproportion2.talkigy.com/}{InProportion}
publikoval řadu nových grafů, které uvádějí současnou úmrtnost ve
Spojeném království v souvislosti s předchozím výskytem chřipky a jinými
příčinami úmrtí. Další webové stránky, které kriticky hodnotí britskou
situaci a opatření, jsou \href{https://lockdownsceptics.org/}{Lockdown
Skeptics} a \href{https://www.ukcolumn.org/}{UK Column}.

\includegraphics{https://swprs.files.wordpress.com/2020/04/inproportion2_chart5.png?w=736\&h=363}

\hypertarget{ux161vuxfdcarsko-nadmux11brnuxe1-uxfamrtnost-hluboko-pod-silnuxfdmi-chux159ipkovuxfdmi-vlnami}{%
\subparagraph{\texorpdfstring{\textbf{Švýcarsko: Nadměrná úmrtnost
hluboko pod silnými chřipkovými
vlnami}}{Švýcarsko: Nadměrná úmrtnost hluboko pod silnými chřipkovými vlnami}}\label{ux161vuxfdcarsko-nadmux11brnuxe1-uxfamrtnost-hluboko-pod-silnuxfdmi-chux159ipkovuxfdmi-vlnami}}

\begin{itemize}
\tightlist
\item
  První sérologická studie na Ženevské univerzitě
  \href{https://www.hug-ge.ch/medias/communique-presse/seroprevalence-covid-19-premiere-estimation}{dospěla
  k závěru}, že nejméně šestkrát více lidí v kantonu Ženeva mělo kontakt
  s Covid-19, než se dříve myslelo. To znamená, že smrtelnost Covid-19
  ve Švýcarsku také klesá výrazně pod jedno procento, zatímco oficiální
  zdroje stále hovoří až o 5\%.
\item
  Dokonce i v nejvíce postiženém kantonu Ticino se
  \href{https://www.bluewin.ch/de/news/schweiz/sp-chef-levrat-will-die-reichen-schropfen-383977.html}{téměř
  polovina} nedávných úmrtí odehrála v domovech s pečovatelskou službou,
  které jsou zasaženy všeobecným omezením pohybu.
\item
  Ve Švýcarsku již bylo na krátkodobou práci zaregistrováno
  \href{https://www.bluewin.ch/de/news/schweiz/sp-chef-levrat-will-die-reichen-schropfen-383977.html}{1,85
  milionu lidí} , tj. více než třetina všech zaměstnanců. Ekonomické
  náklady se odhadují na 32 miliard za období od března do června.
\item
  Infosperber:
  \href{https://www.infosperber.ch/Artikel/Medien/Corona-NZZ-deckt-das-Nachplappern-anderer-Medien-auf}{Corona:
  Papouškování médií.} „Nejvlivnější média skrývají skutečnost, že
  statistiky o Covid-19 se spoléhají na neprůhledná data.``
\item
  Ktipp:
  \href{https://www.ktipp.ch/artikel/artikeldetail/bund-fast-alle-zahlen-ohne-gewaehr/}{Švýcarské
  orgány: Téměř všechna čísla „bez záruky``.} „Letos za prvních 14 týdnů
  zemřelo v kategorii do 65 let nejméně lidí za posledních pět let. V
  kategorii nad 65 let byl počet také relativně nízký. ``\\
\end{itemize}

Následující graf ukazuje, že celková úmrtnost ve Švýcarsku byla v prvním
čtvrtletí roku 2020 v normálním rozmezí a že do poloviny dubna se stále
pohybovala kolem 2,000 lidí, a tedy pod chřipkovou vlnou v roce 2015.
50\% úmrtí se vyskytlo
\href{https://www.nzz.ch/zuerich/coronavirus-zuerich-aendert-nun-das-testregime-in-heimenauch-viele-aeltere-covid-19-infizierte-entwickeln-keine-symptome-zuerich-aendert-nun-das-testregime-in-heimen-ld.1552089}{v
domovech s pečovatelskou službou}, které jsou zasaženy všeobecným
omezením pohybu.

 Celkově se kolem 75\% nových úmrtí odehrálo
\href{https://www.tagesspiegel.de/wissen/woran-sterben-corona-patienten-wirklich-ein-schweizer-forscher-macht-hoffnung-im-kampf-gegen-covid-19/25750666.html}{doma},
zatímco nemocnice a jednotky intenzivní péče zůstávají
\href{https://swprs.files.wordpress.com/2020/04/intensivbettenbelegung-schweiz-2020-04-14.png}{silně
nevyužité} a četné operace byly zrušeny. I ve Švýcarsku tedy vyvstává
velmi závažná otázka, zda „izolace`` může stát více životů než jich
zachránit.

\includegraphics{https://swprs.files.wordpress.com/2020/04/schweiz-todesfaelle-2010-2020.png?w=736\&h=357}

\hypertarget{politickuxe9-aktualizace}{%
\subparagraph{\texorpdfstring{\textbf{Politické
aktualizace:}}{Politické aktualizace:}}\label{politickuxe9-aktualizace}}

\begin{itemize}
\tightlist
\item
  Video: V australském státě Queensland sledoval policejní vrtulník se
  zařízením pro noční vidění
  \href{https://twitter.com/Independent/status/1252911273597120513}{tři
  mladé muže}, kteří v noci pili pivo na střeše domu, čímž porušili
  „koronská nařízení``. Muži byli prostřednictvím megafonu informováni,
  že budova je „obklíčena policií`` a že musí pokračovat k východu.
  Každý z mužů dostal
  \href{https://www.dailystar.co.uk/news/world-news/police-helicopter-uses-night-vision-21899640}{pokutu}
  asi 1,000 dolarů.
\item
  V Izraeli byla od poloviny března
  \href{https://www.jewishpress.com/news/the-courts/state-to-high-court-even-more-shin-bet-involvement-in-fighting-the-coronavirus/2020/04/14/}{pověřena}
  domácí a protiteroristická zpravodajská agentura Shin Bet, aby ve
  spolupráci s policií sledovala mobilní telefony obyvatelstva s cílem
  sledovat kontakty a zatýkat v souvislosti s Covid-19. Tato opatření
  byla původně vydána bez souhlasu parlamentu a měla by zůstat v
  platnosti nejméně do konce dubna.
\item
  OffGuardian:
  \href{https://off-guardian.org/2020/04/23/the-seven-step-path-from-pandemic-to-totalitarianism/}{Sedm
  kroků od pandemie k totalitě.}
\item
  UK Column:
  \href{https://www.ukcolumn.org/article/who-controls-british-government-response-covid19-part-one}{Kdo
  kontroluje reakci britské vlády na Covid-19?}
\item
  Mezitím byl propuštěn švýcarský lékař, který byl zatčen zvláštní
  jednotkou švýcarské policie a poslán na psychiatrickou kliniku pro
  svou kritiku „koronových opatření``. Zpráva časopisu Weltwoche
  \href{https://uncut-news.ch/wp-content/uploads/2020/04/Wer-l\%C3\%B6ste-den-Fehlalarm-aus.pdf}{odhalila},
  že doktor byl zatčen z falešných důvodů: nehrozilo nebezpečí pro
  příbuzné nebo úřady a nedošlo k držení nabité zbraně. Zdá se tedy, že
  šlo pravděpodobně o politicky motivovanou operaci. 
\item
  Mnichovská místní rozhlasová stanice, která v březnu dotazovala lékaře
  kritické vůči koronovým politikám,
  \href{https://norberthaering.de/medienversagen/radiomuenchen-blm-meinungsvielfalt/}{byla
  informována} odpovědným orgánem pro dohled nad médii po stížnostech,
  že „takové problematické vysílání musí být v budoucnu zastaveno``.
\item
  Webové stránky \href{https://kollateral.news/}{kollateral.news}
  německého odborného právníka shromažďují zprávy o „utrpení v důsledku
  omezení pohybu`` a o skutečné situaci v německých nemocnicích.
\item
  Němečtí praktičtí lékaři zveřejnili
  \href{https://aerzteinnenvorort.de/der-appell}{výzvu politikům a
  vědcům}, v níž požadují „odpovědnější řešení koronové krize``.
\item
  Jak v
  \href{https://www.wochenblick.at/corona-kritik-aerztekammer-droht-unbotmaessigem-arzt-mit-ausschluss/}{Rakousku},
  tak v
  \href{https://magyarhang.org/belfold/2020/04/16/etikai-vizsgalat-indul-az-orvos-ellen-aki-szerint-nincs-jarvany-es-az-idosek-csak-a-felelemtol-halnak-meg/}{Maďarsku}
  hrozí lékařům, kteří kritizovali koronová opatření, vyloučení z jejich
  profese.
\item
  V Nigérii bylo podle oficiálních údajů policie zabito
  \href{https://www.bbc.com/news/world-africa-52317196}{při vynucování}
  koronových zákazů vycházení více lidí než samotným koronavirem.
\end{itemize}

\hypertarget{21-dubna-2020}{%
\paragraph{21. dubna 2020}\label{21-dubna-2020}}

\hypertarget{luxe9kaux159skuxe9-aktualizace-1}{%
\subparagraph{\texorpdfstring{\textbf{Lékařské
aktualizace:}}{Lékařské aktualizace:}}\label{luxe9kaux159skuxe9-aktualizace-1}}

\begin{itemize}
\tightlist
\item
  Stanfordský profesor medicíny John Ioannidis vysvětluje v novém
  \href{https://www.youtube.com/watch?v=cwPqmLoZA4s}{hodinovém
  rozhovoru} výsledky několika nových studií o Covid-19. Podle profesora
  Ioannidise je úmrtnost Covid-19 „srovnatelná se sezónní chřipkou``. U
  lidí mladších 65 let je riziko úmrtnosti srovnatelné s každodenní
  jízdou autem, a to i v „centrech nákazy``, zatímco u zdravých lidí do
  65 let je riziko úmrtnosti „zcela zanedbatelné``. Pouze v New Yorku
  bylo riziko úmrtnosti osob mladších 65 let srovnatelné s rizikem
  řidičů kamionů na velké vzdálenosti.
\item
  Profesor Carl Heneghan, který řídí Centre for Evidence-Based Medicine
  na Oxfordské univerzitě, varuje
  \href{https://news.yahoo.com/lockdown-damage-outweighs-coronavirus-warning-121940675.html}{v
  novém článku}, že škody způsobené omezeními pohybu mohou být větší než
  škody způsobené virem. Vrchol epidemie byl ve většině zemí již dosažen
  před těmito omezeními, tvrdí profesor Heneghan.
\item
  Nová
  \href{http://publichealth.lacounty.gov/phcommon/public/media/mediapubhpdetail.cfm?prid=2328}{sérologická
  studie} v Los Angeles zjistila, že 28 až 55krát více lidí, než se
  původně předpokládalo již mělo Covid-19 (aniž by vykazovaly významné
  příznaky), z čehož vyplývá, že i nebezpečnost tohoto onemocnění je
  mnohem nižší.
\item
  Ve městě Chelsea poblíž Bostonu měla
  \href{https://archive.is/20200418222442/https:/www.bostonglobe.com/2020/04/17/business/nearly-third-200-blood-samples-taken-chelsea-show-exposure-coronavirus/}{asi
  třetina z 200 dárců krve} protilátky proti patogenu Covid-19. Polovina
  z nich uvedla, že v posledním měsíci měla nachlazení. V útulku pro
  bezdomovce poblíž Bostonu jen něco málo přes třetinu lidí testovalo
  pozitivně, ale
  \href{https://www.wsbtv.com/news/trending/coronavirus-cdc-reviewing-stunning-universal-testing-results-boston-homeless-shelter/ZADQ45HCAZEVJAZA3OTCUR7M6M/}{nikdo
  nevykazoval žádné příznaky}.
\item
  Skotsko uvádí, že polovina (speciálně vyhrazených) lůžek intenzivní
  péče
  \href{https://www.heraldscotland.com/news/18377095.coronavirus-scotland-half-icu-beds-empty/}{zůstala
  prázdná}. Podle úředníků počty nově přijatých pacientů „stagnují``.
\item
  Pohotovost v městské nemocnici v Bergamu byla na začátku tohoto týdne
  poprvé za 45 dní \href{https://orf.at/stories/3162642/}{úplně
  prázdná}. Mezitím je znovu léčeno více lidí s jinými chorobami než
  pacientů s Covid-19.
\item
  Zpráva v lékařském časopise Lancet dospěla k závěru, že školní
  uzávěrky s cílem zastavit šíření koronaviru mají
  \href{https://www.thelancet.com/journals/lanchi/article/PIIS2352-4642(20)30095-X/fulltext}{jen
  minimální nebo žádný účinek}.
\item
  Devítileté francouzské dítě nakažené koronavirem mělo kontakt se 172
  lidmi, ale
  \href{https://www.n-tv.de/panorama/172-Kontaktpersonen-von-Corona-verschont-article21727469.html}{nikdo
  z nich nebyl infikován}. To potvrzuje dřívější výsledky, že
  koronavirus (na rozdíl od chřipky) není nikdy nebo téměř nikdy
  přenášený dětmi.
\item
  Německý emeritní profesor mikrobiologie Sucharit Bhakdi poskytnul
  \href{https://kenfm.de/kenfm-am-set-gespraech-mit-prof-dr-sucharit-bhakdi-zu-covid-19/}{nový
  hodinový rozhovor} o Covid-19. Profesor Bhakdi tvrdí, že většina médií
  během epidemie Covid-19 jednala „zcela nezodpovědně``.
\item
  Německá Iniciativa pro etickou péči
  \href{http://pflegeethik-initiative.de/2020/04/15/corona-krise-falsche-prioritaeten-gesetzt-und-ethische-prinzipien-verletzt/}{kritizuje}
  paušální zákazy návštěv a bolestivé léčbě intenzivní péčí o kojící
  pacienty: „I před „koronou`` úmíralo v Německu denně asi 900 starých
  lidí závislých na péči v domovech, aniž by byli odvezeni do nemocnice.
  Ve skutečnosti by byla pro tyto pacienty vhodnější paliativní léčba,
  pokud vůbec nějakou léčbu potřebovali. Podle všeho, co zatím víme o
  koronaviru, neexistuje jediný věrohodný důvod, proč i nadále
  nadřazovat ochranu před touto infekcí nad základní práva občanů.
  Zvedněte nelidské zákazy návštěv! ``
\item
  Nejstarší žena ve švýcarském kantonu St. Gallen zemřela minulý týden
  ve věku 109. Přežila „španělskou chřipku`` z roku 1918, nebyla
  infikována koronou a „na svůj věk si vedla velmi dobře``. „Izolace
  kvůli korony`` ji však
  \href{https://swprs.files.wordpress.com/2020/04/tagblatt-109.jpg}{„velmi
  ovlivnila``}: „Zmizela bez každodenních návštěv svých rodinných
  příslušníků.``
\item
  Švýcarský kardiolog Dr. Nils Kucher uvádí, že ve Švýcarsku v současné
  době asi 75\% všech dalších úmrtí
  \href{https://www.tagesspiegel.de/wissen/woran-sterben-corona-patienten-wirklich-ein-schweizer-forscher-macht-hoffnung-im-kampf-gegen-covid-19/25750666.html}{nenastává
  v nemocnici, ale doma}. To rozhodně vysvětluje
  \href{https://swprs.files.wordpress.com/2020/04/intensivbettenbelegung-schweiz-2020-04-14.png}{převážně
  prázdné} švýcarské nemocnice a jednotky intenzivní péče. Je také již
  známo, že přibližně 50\% všech dalších úmrtí se vyskytuje
  \href{https://www.nzz.ch/zuerich/coronavirus-zuerich-aendert-nun-das-testregime-in-heimenauch-viele-aeltere-covid-19-infizierte-entwickeln-keine-symptome-zuerich-aendert-nun-das-testregime-in-heimen-ld.1552089}{v
  domovech s pečovatelskou službou}. Dr. Kucher má podezření, že někteří
  z těchto lidí umírají na náhlou plicní embolii. To je myslitelné.
  Vyvstává však otázka, jakou roli hraje „izolace`` v těchto nedávných
  úmrtích.
\item
  Italský zdravotní úřad (ISS)
  \href{https://www.iss.it/en/rapporti-covid-19/-/asset_publisher/btw1J82wtYzH/content/id/5334891}{varuje},
  že pacienti s Covid-19 ze středomořského regionu, kteří mají často
  genetickou metabolickou zvláštnost zvanou favismus, by neměli být
  léčeni antimalariky, jako je chlorochin, protože to může vést k úmrtí.
  To je
  \href{https://www.sciencedaily.com/releases/2020/02/200206110703.htm}{další
  náznak}, že špatné nebo příliš agresivní léky mohou nemoc ještě
  zhoršit.
\item
  Rubicon:
  \href{https://www.rubikon.news/artikel/120-expertenstimmen-zu-corona}{120
  odborných stanovisek ke koronaviru}. Celosvětově uznávaní vědci,
  lékaři, právníci a další odborníci kritizují opatření kvůli
  koronaviru. (německy)
\end{itemize}

\hypertarget{klasifikace-pandemie}{%
\subparagraph{\texorpdfstring{\textbf{Klasifikace
pandemie:}}{Klasifikace pandemie:}}\label{klasifikace-pandemie}}

V roce 2007 americké zdravotnické orgány definovaly
\href{https://www.cidrap.umn.edu/news-perspective/2007/02/hhs-ties-pandemic-mitigation-advice-severity}{pětistupňovou
klasifikaci} pandemické chřipky a protiopatření. Pět kategorií je
založeno na pozorované letalitě pandemie (CFR), od kategorie 1
(\textless{}0,1\%) do kategorie 5 (\textgreater{} 2\%). Podle tohoto
klíče by současná koronová pandemie byla pravděpodobně zařazena do
kategorie 2 (0,1\% až 0,5\%). V této kategorii se jako hlavní opatření v
té době uvažovalo pouze o „dobrovolné izolaci nemocných``.

 V roce 2009 však WHO
\href{https://www.forbes.com/2010/02/05/world-health-organization-swine-flu-pandemic-opinions-contributors-michael-fumento.html\#5ae32fb848e8}{odstranila}
závažnost ze své definice pandemie. Od té doby lze v zásadě každou
celosvětovou vlnu chřipky prohlásit za pandemii, což se stalo s velmi
mírnou „prasečí chřipkou`` z roku 2009/2010, na kterou byly prodány
vakcíny v hodnotě přibližně 18 miliard dolarů.

 Dokument TrustWHO („Trust who?``), který se zabývá pochybnou rolí WHO v
souvislosti s „prasečí chřipkou``, byl nedávno
\href{https://www.youtube.com/watch?v=VjQGyqVN5RM}{vymazán ze serveru
VIMEO}.

\hypertarget{ux161vuxfdcarskuxfd-hlavnuxed-luxe9kaux159-pietro-vernazza-elementuxe1rnuxed-opatux159enuxed-jsou-dostateux10dnuxe1}{%
\subparagraph{\texorpdfstring{\textbf{Švýcarský hlavní lékař Pietro
Vernazza: Elementární opatření jsou
dostatečná}}{Švýcarský hlavní lékař Pietro Vernazza: Elementární opatření jsou dostatečná}}\label{ux161vuxfdcarskuxfd-hlavnuxed-luxe9kaux159-pietro-vernazza-elementuxe1rnuxed-opatux159enuxed-jsou-dostateux10dnuxe1}}

Ve svém
\href{https://infekt.ch/2020/04/sind-wir-tatsaechlich-im-blindflug/}{posledním
příspěvku} používá švýcarský hlavní lékař pro infektiologii, Pietro
Vernazza, výsledky německého Institutu Roberta Kocha a ETH v Curychu,
aby ukázal, že epidemie Covid-19 byla pod kontrolou ještě před zavedením
„izolace``:

 „Tyto výsledky jsou revoluční: Obě studie ukazují, že elementární
opatření, jako je rušení velkých shromáždění a zavedení hygienických
opatření, jsou vysoce účinná. Populace je schopna tato doporučení dobře
implementovat a opatření mohou epidemii téměř zastavit. V každém případě
jsou tato opatření dostatečná k ochraně našeho zdravotního systému tak,
aby nemocnice nebyly přetíženy``.

\includegraphics{https://swprs.files.wordpress.com/2020/04/ch-reproduktionszahl-eth-infekt.png?w=650\&h=379}

\hypertarget{ux161vuxfdcarsko-kumulativnuxed-celkovuxe1-uxfamrtnost-v-normuxe1lnuxedm-rozmezuxed}{%
\subparagraph{\texorpdfstring{\textbf{Švýcarsko: Kumulativní celková
úmrtnost v normálním
rozmezí}}{Švýcarsko: Kumulativní celková úmrtnost v normálním rozmezí}}\label{ux161vuxfdcarsko-kumulativnuxed-celkovuxe1-uxfamrtnost-v-normuxe1lnuxedm-rozmezuxed}}

Ve Švýcarsku byla kumulativní celková úmrtnost v prvním čtvrtletí (do 5.
dubna)
n\href{https://swprs.files.wordpress.com/2020/04/ch-sterblichkeit-kumuliert-q1-2020.pdf}{a
průměrné očekávané hodnotě} a více než 1 500 jednotek pod horní
očekávanou hodnotou. Navíc do poloviny dubna byla celková úmrtnost stále
více než 2 000 úmrtí pod srovnávací hodnotou z období vážné chřipky v
roce 2015 (viz obrázek níže).

\hypertarget{ux161vuxe9dsko-epidemie-konux10duxed-i-bez-omezenuxed-pohybu}{%
\subparagraph{\texorpdfstring{\textbf{Švédsko: Epidemie končí i bez
omezení
pohybu}}{Švédsko: Epidemie končí i bez omezení pohybu}}\label{ux161vuxe9dsko-epidemie-konux10duxed-i-bez-omezenuxed-pohybu}}

Nejnovější údaje o pacientech a úmrtích ukazují, že epidemie ve Švédsku
končí. Ve Švédsku, stejně jako ve většině ostatních zemí, došlo k
nadměrné úmrtnosti hlavně v pečovatelských domovech, které nebyly
dostatečně chráněny,
\href{https://www.washingtontimes.com/news/2020/apr/15/sweden-coronavirus-rates-easing-despite-loose-rule/}{vysvětlil}
hlavní epidemiolog.

 Ve srovnání s jinými zeměmi může švédská populace nyní těžit z vyšší
imunity vůči viru Covid-19, který by je mohl lépe chránit před možnou
„druhou vlnou`` příští zimu.

 Lze předpokládat, že do konce roku 2020 nebude Covid-19 ve švédské
celkové úmrtnosti viditelný. Švédský příklad ukazuje, že „omezení`` byla
nejen sociálně a ekonomicky devastující, ale i zbytečná nebo dokonce
kontraproduktivní z lékařského hlediska.

\textbf{Video}: \href{https://www.youtube.com/watch?v=bfN2JWifLCY}{Proč
jsou zablokování nesprávnou politikou -- švédský odborník, profesor
Johan Giesecke}

\href{https://swprs.files.wordpress.com/2020/04/sweden-deaths-day-2.png}{\includegraphics{https://swprs.files.wordpress.com/2020/04/sweden-deaths-day-2.png?w=736\&h=293}}Počet
úmrtí s pozitivním testem ve Švédsku
(\href{https://en.wikipedia.org/wiki/2020_coronavirus_pandemic_in_Sweden\#Charts7be6f9f87457ed9aa}{FOHM
/ Wikipedia}; hodnoty mohou být ještě upraveny)

\hypertarget{anekdoty-vs-dux16fkazy}{%
\subparagraph{\texorpdfstring{\textbf{Anekdoty vs.
důkazy:}}{Anekdoty vs. důkazy:}}\label{anekdoty-vs-dux16fkazy}}

S ohledem na nedostatek vědeckých důkazů se některá média stále více
spoléhají na hrozivé anekdoty, aby v populaci udrželi strach. Typickým
příkladem jsou „zdravé děti``, které údajně zemřely na Covid-19, ale
později se často ukázalo, že na Covid-19
\href{https://www.dailymail.co.uk/news/article-8193487/Coroner-refuses-rule-COVID-19-cause-death-six-week-old-Connecticut-baby.html}{nezemřely},
nebo že již byly
\href{https://sports.yahoo.com/spanish-football-coach-francisco-garcia-163153573.html}{vážně
nemocné}.

Rakouská média nedávno informovala o
\href{https://www.rainews.it/tgr/tagesschau/articoli/2020/04/tag-Coronavirus-Lungeschaden-Forschung-Uniklinik-Innsbruck-6708e11e-28dc-4843-a760-e7f926ace61c.html}{potápěčích},
kteří šest týdnů po onemocnění Covid-19 s plicním postižením stále
vykazovali sníženou výkonnost a nepříznivé výsledky testů. Jedna část
hovoří o „nevratných škodách``, další vysvětluje, že tento závěr je
„nepodložený a spekulativní``. Není ale uvedeno, že by potápěči měli
obvykle mít
\href{https://www.deeperblue.com/pulmonary-considerations-in-diving/}{6
až 12 měsíční pauzu} po závažné pneumonii.

Často se zmiňují také neurologické účinky, jako je dočasná ztráta čichu
nebo chuti. Také zde obvykle není vysvětleno, že se jedná o
\href{https://www.ncbi.nlm.nih.gov/pubmed/25294743}{známý účinek}
nachlazení a chřipky, a Covid-19 není v tomto ohledu
\href{https://www.ncbi.nlm.nih.gov/pubmed/23948436}{nijak závažný}.

V jiných zprávách jsou zdůrazněny možné účinky na různé orgány, jako
jsou ledviny, játra nebo mozek, aniž by bylo uvedeno, že mnoho
postižených pacientů bylo již velmi starých a mělo
\href{https://www.epicentro.iss.it/coronavirus/sars-cov-2-decessi-italia}{již
předtím závažné chronické nemoci}.

\hypertarget{politickuxe9-aktualizace-1}{%
\subparagraph{\texorpdfstring{\textbf{Politické
aktualizace:}}{Politické aktualizace:}}\label{politickuxe9-aktualizace-1}}

\begin{itemize}
\tightlist
\item
  WOZ:
  \href{https://www.woz.ch/2016/grundrechte/wenn-die-angst-regiert}{Když
  vládne strach.} „U dronů, aplikací a zákazů demonstrací: V důsledku
  koronavirové krize dochází k porušování základních svobod. Může se
  jednoduše stát, že to tak zůstane tak i po zablokování -- ale extrémní
  situace také nabízí důvod k naději. `` (německy)
\item
  Multipolar:
  \href{https://multipolar-magazin.de/artikel/die-massnahmen-wirken}{Co
  je jejich cílem?} „Vláda se chválí, šíří slogany vytrvalosti a zároveň
  zpomaluje shromažďování základních údajů, které by umožnilo spolehlivé
  měření šíření a nebezpečí viru. Naproti tomu úřady jednají rychle a
  rozhodně při rozšiřování pochybných nástrojů, jako jsou nové „koronové
  aplikace`` pro hromadné měření pulzů a sledování kontaktů``. (německy)
\item
  Profesor Christian Piska, odborník na veřejné právo a právní techniku
  ​​ve Vídni: „Rakousko se změnilo. Velmi. Avšak zdá se, že to většina
  lidí prostě akceptuje. Krok za krokem, ať už ekonomika prosperuje nebo
  ne, náhle žijeme s podmínkami policejního státu a přísnými omezeními
  našich základních a lidských práv, čímž se vyrovnáváme diktátorským
  režimům. Toto je Pandořina skříňka, která už se jednou otevřela, a
  nikdy se nemůže
  \href{https://kurier.at/meinung/das-juristische-totschlagargument-vom-menschenleben/400814570}{znovu
  zavřít}. `` (německy)
\item
  Více než 300 vědců z 26 zemí varuje před
  \href{https://www.golem.de/news/corona-app-300-wissenschaftler-warnen-vor-zentraler-datenspeicherung-2004-147973.html}{„bezprecedentním
  sledováním společnosti``} koronovými aplikacemi, které porušují
  ochranu dat. Několik vědců a univerzit již opustilo evropský projekt
  sledování kontaktů PEPP-PT z důvodu nedostatečné transparentnosti.
  Nedávno vyšlo najevo, že do projektu se zapojila švýcarská společnost
  AGT, která dříve vytvořila systémy hromadného sledování pro arabské
  státy.
\item
  V Izraeli
  \href{https://edition.cnn.com/2020/04/20/middleeast/israel-protest-social-distancing-intl/index.html}{demonstrovalo
  proti opatřením} Netanjahuovi vlády asi 5 000 lidí (všichni v
  rozestupu 2 m): „Mluví o exponenciálním nárůstu případů koronaviru,
  ale jediná věc, která exponenciálně roste, je počet lidí, kteří se
  postaví, aby ochránili naše zemi a její demokracii ``.
\item
  Irský novinář Jason O'Toole žijící v Madridu
  \href{https://www.rt.com/op-ed/486350-spain-tough-rules-covid-19-lockdown/}{popisuje}
  situaci ve Španělsku: „Vzhledem k tomu, že v ulicích španělských měst
  je armáda všudypřítomná, není možné situaci popsat jinak než jako
  válečné právo ve skutečnosti, ale bez vyhlášení. Velký bratr George
  Orwella je živ a zdráv, protože španělská policie sleduje všechny
  pomocí CCTV nebo dronů létajících nad jejich hlavami. Během prvních
  čtyř týdnů bylo pokutováno neuvěřitelných 650 000 lidí a 5 568 osob
  bylo zatčeno. Byl jsem šokován, když jsem sledoval videoklip
  policisty, který použil těžkou sílu, aby zatkl duševně nemocného
  mladíka, který podle všeho právě šel domů s chlebem.``
\item
  OffGuardian:
  \href{https://off-guardian.org/2020/04/18/the-disturbing-developments-in-uk-policing/}{Znepokojivý
  vývoj v britské policejní práci.}
\item
  Americký investigativní novinář Whitney Webb píše ve svém novém článku
  o tom,
  \href{https://www.thelastamericanvagabond.com/top-news/techno-tyranny-how-us-national-security-state-using-coronavirus-fulfill-orwellian-vision/}{„jak
  americký národní bezpečnostní stát používá koronaviry k naplnění
  orwellovské vize``} : „Minulý rok vyzvala vládní komise Američany, aby
  přijaly systém hromadného sledování řízeného umělou inteligencí, který
  jde daleko za hranici dosaženou v jakékoli jiné zemi, a to, aby byla
  zajištěna americká hegemonie v umělé inteligenci. „Překážky``, které
  bráni jeho implementaci, jsou nyní rychle odstraňovány pod rouškou
  boje proti kronavírové krizi.``
\item
  V
  \href{https://www.thelastamericanvagabond.com/top-news/all-roads-lead-dark-winter/}{předchozím
  článku} se Whitney Webb zabýval ústřední rolí, kterou hraje Johns
  Hopkins University a její „Centrum pro zdravotní bezpečnost`` v řízení
  reakce na tuto pandemii, jakož i roli tohoto centra v předchozích
  simulacích pandemie a biologických zbraní a jeho úzkými vazbami na
  americké tajné služby.
\item
  Myšlenka použití pandemie k rozšíření nástrojů globálního dozoru a
  kontroly není nová. Již v roce 2010 popsala americká Rockefeller
  Foundation
  \href{https://swprs.files.wordpress.com/2020/04/rockefeller-foundation-scenarios-2010.pdf}{v
  předběžné zprávě} o budoucím technologickém a společenském vývoji
  „scénář izolace``, který předpověděl současný vývoj s působivou
  přesností (od strany 18).
\item
  \href{https://childrenshealthdefense.org/news/the-truth-about-fauci-featuring-dr-judy-mikovits/}{„Pravda
  o Faucim``}: V novém rozhovoru hovoří americký virolog Dr. Judy
  Mikovits o svých zkušenostech s doktorem Anthonym Faucim, který v
  současné době hraje hlavní roli při utváření opatření americké vlády
  proti Covid-19.
\item
  Organizace poskytující pomoc varují, že „mnohem více lidí`` zemře na
  \href{https://www.welt.de/wirtschaft/article207092745/Corona-Pandemie-Rezession-beschert-der-Welt-die-noch-groessere-Katastrophe.html}{ekonomické
  důsledky} opatření než na samotný Covid-19. Prognózy nyní
  předpovídají, že 35 až 65 milionů lidí upadne do absolutní chudoby, a
  mnohým z nich hrozí hladovění.
\item
  V Německu se očekává, že v roce 2020 bude
  \href{https://www.boeckler.de/pdf/p_wsi_pb_38_2020.pdf}{2,35 milionu
  lidí} závislých na dočasných pracích, což je více než dvojnásobek ve
  srovnání se situací po finanční krizi v roce 2008/2009.
\end{itemize}

\includegraphics{https://swprs.files.wordpress.com/2020/04/kurzarbeit-de-corona.png?w=650\&h=461}

\hypertarget{18-dubna-2020}{%
\paragraph{18. dubna 2020}\label{18-dubna-2020}}

\hypertarget{luxe9kaux159skuxe9-aktualizace-2}{%
\subparagraph{\texorpdfstring{\textbf{Lékařské
aktualizace:}}{Lékařské aktualizace:}}\label{luxe9kaux159skuxe9-aktualizace-2}}

\begin{itemize}
\tightlist
\item
  Nová
  \href{https://www.medrxiv.org/content/10.1101/2020.04.14.20062463v1}{sérologická
  studie} Stanfordské univerzity našla protilátky u 50 až 85krát více
  lidí, než se dříve myslelo v okrese Santa Clara v Kalifornii, z čehož
  vyplývá, že smrtelnost Covid-19 je 0,12\% až 0,2\% nebo dokonce nižší
  (tj. srovnatelná s těžkou chřipkou). Profesor John Ioannidis
  vysvětluje studii \href{https://www.youtube.com/watch?v=jGUgrEfSgaU}{v
  novém videu}.
\item
  Centre for Evidence-Based Medicine (CEBM) na Oxfordské univerzitě ve
  své nové analýze
  \href{https://www.cebm.net/covid-19/global-covid-19-case-fatality-rates/}{tvrdí},
  že úmrtnost v souvislosti s Covid-19 (IFR) je mezi 0,1\% a 0,36\% (tj.
  srovnatelná s těžkou chřipkou). U lidí starších 70 let bez vážných
  předchozích nemocí je očekáváná úmrtnost nižší než 1\%. U lidí
  starších 80 let se úmrtnost pohybuje mezi 3\% a 15\% v závislosti na
  tom, zda k úmrtí doposud došlo převážně s onemocněním nebo kvůli němu.
  Na rozdíl od chřipky je dětská úmrtnost téměř nulová. Pokud jde o
  vysokou úmrtnost v severní Itálii, výzkumná skupina zdůrazňuje, že
  Itálie má v Evropě
  \href{https://www.ansa.it/english/news/science_tecnology/2019/11/19/italy-top-in-eu-in-antibiotic-resistance_369e0123-0107-445e-8c17-f11932c9d27c.html}{nejvyšší
  odolnost vůči antibiotikům}. Údaje italských orgánů skutečně ukazují,
  že přibližně 80\% zemřelých bylo léčeno antibiotiky, což ukazuje na
  závažné bakteriální infekce.
\item
  Finský profesor epidemiologie Mikko Paunio z Helsinské University
  vyhodnotil několik mezinárodních studií v
  \href{https://lockdownsceptics.org/wp-content/uploads/2020/04/How-the-World-got-Fooled-by-COVID-ed-2c.pdf}{předběžné
  zprávě} a dospěl k závěru, že letalita Covid-19 (IFR) je 0,1\% nebo
  méně (tj. srovnatelná se sezonní chřipkou). Podle Paunia byl vytvořen
  dojem vyšší letality, protože virus se šířil velmi rychle, zejména v
  domácnostech s více generacemi v Itálii a Španělsku, ale také ve
  městech, jako je New York. „Izolace`` přišla příliš pozdě a nebyla
  účinná.
\item
  Velká Británie: Londýnská dočasná nemocnice Nightingale zůstala
  \href{https://www.hsj.co.uk/service-design/exclusive-nightingale-largely-empty-as-icus-handle-surge/7027398.article}{z
  velké části prázdná} a během velikonočního víkendu bylo v zařízení
  léčeno pouze 19 pacientů. Londýnské nemocnice zdvojnásobily kapacitu
  JIP (jednotek intenzivní péče) a zatím se s nárůstem vyrovnávají.
\item
  V Kanadě
  \href{https://www.nytimes.com/2020/04/16/world/canada/montreal-nursing-homes-coronavirus.html}{zemřelo
  31 lidí} v pečovatelském domě poté, co „téměř veškerý ošetřovatelský
  personál opustil zařízení ve spěchu ze strachu z rozšíření koronového
  viru. Zdravotní úřady našli lidi v domě v Dorvalu nedaleko Montrealu
  jen o několik dní později -- ti, co přežili, byli dehydratováni,
  podvyživení a apatičtí.`` Podobné tragédie již byly
  \href{https://swprs.org/covid-19-a-report-from-italy/}{zaznamenány} v
  severní Itálii, kde východoevropské sestry opustily zemi ve spěchu,
  když vypukla panika a byla oznámena opatření omezující pohyb osob. 
\item
  Skotský lékař, který také pečuje o pečovatelské domovy,
  \href{https://drmalcolmkendrick.org/2020/04/17/care-homes-and-covid19/}{píše}:
  „Jaká byla vládní strategie pro pečovatelské domovy? Dosud podniknuté
  kroky situaci ještě zhoršily.``
\item
  Ve Švýcarsku se i přes Covid-19 celková úmrtnost v prvním čtvrtletí
  roku 2020 (do 5. dubna) pohybovala
  \href{https://swprs.files.wordpress.com/2020/04/ch-sterblichkeit-kumuliert-q1-2020.pdf}{v
  normálním rozmezí}. Jedním z důvodů by mohla být mírná chřipková
  sezóna v důsledku mírné zimy, kterou nyní Covid-19 částečně
  „vyrovnal``.
\item
  Podle zprávy ze 14. dubna jsou švýcarské nemocnice a dokonce i
  jednotky intenzivní péče i nadále
  \href{https://swprs.files.wordpress.com/2020/04/intensivbettenbelegung-schweiz-2020-04-14.png}{nedostatečně
  využívány}. To opět vyvolává otázku, kde a jak přesně se ve Švýcarsku
  skutečně vyskytují úmrtí s pozitivním testem (průměrný věk 84).
\item
  Prezident Německé asociace nemocnic
  \href{https://www.bz-berlin.de/deutschland/kliniken-verband-schlaegt-alarm-wegen-corona-regeln}{vyhlásil
  poplach}: více než 50 procent všech plánovaných operací v Německu bylo
  zrušeno a „počet nevyřízených operací`` se blíží k tisícům. Kromě toho
  je v nemocnicích léčeno o 30 až 40\% méně pacientů se srdečními
  infarkty a mozkovými příhodami, protože ti se již neodváží chodit do
  nemocnic ze strachu z korony. V celé zemi bylo nevyužito 150 000
  nemocničních lůžek a 10 000 lůžek pro intenzivní péči. V Berlíně jsou
  pacienti s koronou pouze na 68 lůžkách intenzivní péče, pohotovostní
  klinika s 1 000 lůžky se momentálně nepoužívá.
\item
  Nové údaje německých úřadů ukazují, že i v Německu již míra reprodukce
  Covid-19 klesla pod kritickou hodnotu 1
  \href{https://www.rki.de/DE/Content/Infekt/EpidBull/Archiv/2020/Ausgaben/17_20_SARS-CoV2_vorab.pdf?__blob=publicationFile\#page=5}{před
  omezeními pohybu}. Základní hygienická opatření proto postačovala k
  zabránění exponenciálního rozšíření. To již ukázala ETH Curych v
  případě Švýcarska.
\item
  \href{https://www.ouest-france.fr/sante/virus/coronavirus/coronavirus-au-moins-940-marins-positifs-sur-le-charles-de-gaulle-et-son-escorte-6810816}{1
  081 vojáků testováno pozitivně} na francouzské letadlové lodi. Téměř
  50\% z nich zůstalo dosud bez příznaků a asi 50\% mělo mírné příznaky.
  Čtyřiadvacet z nich bylo hospitalizováno, a jeden z nich je v
  intenzivní péči (nejsou známy předchozí nemoci).
\item
  Přední německý virolog Christian Drosten si myslí, že je možné, že
  někteří lidé již vybudovali proti novému koronaviru účinnou
  \href{https://www.watson.de/!324026684}{„imunitu z minulosti``} díky
  předchozímu kontaktu s koronovými viry spojených s běžným nachlazením.
\item
  Klaus Püschsel, soudní lékař z Hamburku, který již prozkoumal mnoho
  zemřelých s pozitivním testem,
  \href{https://www.abendblatt.de/hamburg/article228908865/hamburg-corona-virus-uke-infektion-covid-19-pueschel-coronavirus-krise-patienten-impfstoff-immunitaet-krankenhaeuser-kontaktverbot-kliniken-infektionsrate-krankheit-pandemie-test-lungenkrankheit-sars-cov-epidemie-sars-cov-2.html}{vysvětluje
  v novém článku}: „Čísla neospravedlňují strach z korony``. Jeho
  zjištění: „Korona je relativně neškodné virové onemocnění. Musíme se
  vypořádat s tím, že Korona je normální infekce a musíme se s ní naučit
  žít bez karantény``. Úmrtí, které zkoumal, by všichni měli takové
  vážné již existující nemoci, že „i když to zní tvrdě, všichni by v
  průběhu tohoto roku zemřeli``. Püschel dodává: „Čas virologů skončil.
  Nyní bychom se měli zeptat i jiných, co je třeba dělat v případě
  koronové krize, například lékařů intenzivní péče.``
\item
  \href{https://emedicine.medscape.com/article/227820-overview}{Přehled
  na Medscape} ukazuje, že běžné nachlazení infekce způsobené koronaviry
  obvykle klesají na konci dubna -- s omezeními pohybu nebo bez nich.
\item
  Švýcarský časopis Infosperber píše:
  \href{https://www.infosperber.ch/Artikel/Gesundheit/Weniger-Corona-Falle-Einfach-weniger-testen}{„Méně
  případů korony? Jen testujte méně!``} Denní počet„ nových případů
  ``uvádí málo o stavu epidemie. Vyvolávání strachu pomocí křivky
  kumulativních úmrtí s pozitivním testem bylo bezohledné, tvrdí.
\item
  OffGuardian:
  \href{https://off-guardian.org/2020/04/17/8-more-experts-questioning-the-coronavirus-panic/}{Dalších
  osm odborníků zpochybňujících koronavirová opatření.}
\item
  Video: \href{https://www.youtube.com/watch?v=bfN2JWifLCY}{Proč jsou
  omezení pohybu nesprávnou politikou -- švédský expert Prof. Johan
  Giesecke.} Švédský profesor epidemiologie Johan Giesecke hovoří o
  „tsunami mírného onemocnění`` a považuje omezení pohybu za
  kontraproduktivní. Nejdůležitější věc, říká, je poskytnout účinnou
  ochranu rizikovým skupinám, zejména domovům s pečovatelskou službou.
\end{itemize}

\includegraphics{https://swprs.files.wordpress.com/2020/04/rki-reproduktion-lockdown.png?w=600\&h=421}

\hypertarget{ventilace-s-covid19}{%
\subparagraph{\texorpdfstring{\textbf{Ventilace s
Covid19:}}{Ventilace s Covid19:}}\label{ventilace-s-covid19}}

Další odborníci v Evropě a USA vyjádřili svůj názor na léčbu kritických
pacientů s Covid-19 a důrazně nedoporučují invazivní ventilaci
(intubaci). Pacienti s Covid-19 netrpí akutním respiračním selháním
(ARDS), ale nedostatkem kyslíku, pravděpodobně způsobeným problémem s
difúzí kyslíku vyvolaným virem nebo imunitní odpovědí na něj.

\begin{itemize}
\tightlist
\item
  AP: \href{https://apnews.com/8ccd325c2be9bf454c2128dcb7bd616d}{Někteří
  lékaři upouští od ventilátorů při léčbě pacientů s viry} 
\item
  Video: \href{https://www.youtube.com/watch?v=QPlEUAVjxV8}{Covid-19:
  Kritická diskuse o doporučení pro včasnou intubaci}
\item
  Video: \href{https://www.youtube.com/watch?v=NmRlvX3VrAQ}{Newyorský
  lékař intenzivní péče mluví o Covid-19 jako možné difúzní hypoxémii}
\item
  Žurnál:
  \href{https://link.springer.com/article/10.1007/s00134-020-06033-2}{COVID-19
  pneumonie: různá respirační léčba pro různé fenotypy?}
\item
  (Německy)
  \href{https://www.welt.de/vermischtes/article207221877/Corona-Pandemie-Sterberate-bei-Beatmungspatienten-gibt-Raetsel-auf.html}{Die
  WELT: Míra úmrtnosti mezi pacienty s dýcháním}
\end{itemize}

\hypertarget{politickuxe9-aktualizace-2}{%
\subparagraph{\texorpdfstring{\textbf{Politické
aktualizace:}}{Politické aktualizace:}}\label{politickuxe9-aktualizace-2}}

\begin{itemize}
\tightlist
\item
  Video:
  \href{https://archive.org/details/what-in-the-world-is-actually-going-on-document-reveals-plans-step-by-step}{Policejní
  násilí a monitorování během koronových omezení pohybu po celém světě.}
\item
  V několika státech USA došlo k
  \href{https://news.yahoo.com/protests-draw-thousands-over-state-024328374.html}{demonstracím
  proti omezením pohybu}.
\item
  Německý ekonom Norbert Haering \href{https://norberthaering.de/}{v
  několika článcích vysvětluje}, jak se „koronová krize`` používá k
  zavedení již dlouhou dobu plánováných celosvětových špehovacích
  systémů -- v oblasti cestování, plateb, trasování kontaktů a
  biometrie.
\item
  Giorgio Agamben, italský filosof,
  \href{https://www.nzz.ch/feuilleton/coronavirus-giorgio-agamben-zum-zusammenbruch-der-demokratie-ld.1551896}{o
  protikoronových opatřeních}: „Země, a celá kultura, se právě teď ničí,
  a zdá se, že to nikomu moc nevadí. Co se děje před našimi zraky v
  zemích, které tvrdí, že jsou civilizované? ``
\item
  Italští právníci podali
  \href{https://www.tvprato.it/2020/04/la-camera-civile-degli-avvocati-pratesi-chiede-lannullamento-del-dpcm-del-10-aprile-e-illegittimo/}{stížnost}
  proti koronovým opatřením vlády.
\item
  Německý profesor ekonomie Stefan Homburg v DIE WELT:
  \href{https://www.msn.com/de-de/nachrichten/coronavirus/warum-deutschlands-lockdown-falsch-ist-\%E2\%80\%93-und-schweden-vieles-besser-macht/ar-BB12E6km}{„Proč
  je německé omezení pohybu špatné -- a proč si Švédsko vede mnohem
  lépe``}: „Souhrnně řečeno, země jako Švédsko, Jižní Korea nebo
  Tchaj-wan jednaly moudře tím, že nepoužívaly omezení pohybu.
  Virologové v těchto zemích konzistentně vedli populaci a politiky
  skrze krizi, místo aby je mátli neustále se měnícími odhady.
  Koronavirus byl úspěšně zvládnut, aniž by došlo k poškození základních
  práv a pracovních míst. Německo by mělo tuto politiku brát jako model.
  ``
\item
  Švýcarský občan naléhavě
  \href{https://faktenb-covid-19-massnahmen.jimdofree.com/}{požádal}
  federální správní soud a federální radu o okamžité zrušení omezení.
\item
  Video: \href{https://www.youtube.com/watch?v=eU6IdglI-wc}{„Švýcarští
  lékaři byli zmateni, Federální rada je rozdělena.``} Rozhovor s
  doktorem Stephanem Rietikerem, zakladatelem společnosti
  InsideCorona.ch
\item
  Video: \href{https://www.youtube.com/watch?v=SO2JMkKtq40}{„Švýcarská
  vláda patří do vězení. Polemika. ``}
\end{itemize}

\hypertarget{16-dubna-2020}{%
\paragraph{16. dubna 2020}\label{16-dubna-2020}}

\begin{itemize}
\tightlist
\item
  The London Times \href{https://archive.is/2eKCW}{hlásí}, že 50\%
  současné britské nadměrné úmrtnosti nemusí být způsobeno koronaviry,
  ale důsledky omezení pohybu, obecného panického a částečného
  sociálního rozpadu. To činí
  \href{https://www.ons.gov.uk/peoplepopulationandcommunity/birthsdeathsandmarriages/deaths/bulletins/deathsregisteredweeklyinenglandandwalesprovisional/weekending3april2020}{3
  000 úmrtí týdně}. Ve skutečnosti by toto číslo mohlo být ještě vyšší,
  protože britská definice smrti koronavirem zahrnuje všechna úmrtí s
  koronavirem (ne jen ta kvůli němu), a dokonce i případy, kdy je na
  virus „podezření``. Kromě toho přibližně 50\% „koronových úmrtí``
  \href{https://ltccovid.org/2020/04/12/mortality-associated-with-covid-19-outbreaks-in-care-homes-early-international-evidence/}{nastává}
  v pečovatelských domovech, které trpí všeobecnými omezeními.
\item
  V Dánsku nyní
  \href{https://jyllands-posten.dk/debat/breve/ECE12074246/vi-skulle-aldrig-have-trykket-paa-stopknappen/}{litují}
  zavedení omezení: „Nikdy jsme neměli stisknout tlačítko stop. Situaci
  měl pod kontrolou dánský systém zdravotní péče. Totální izolace šla
  příliš daleko``, tvrdí profesor Jens Otto Lunde Jørgensen z Aarhusské
  universitní nemocnice. Dánsko v současné době obnovuje provoz škol.
\item
  David Katz, profesor z univerzity v Yale, který včas varoval před
  negativními důsledky omezení, poskytl
  \href{https://www.youtube.com/watch?v=VK0Wtjh3HVA}{podrobný hodinový
  rozhovor} o současné situaci.
\item
  Německý virolog Hendrik Streeck
  \href{https://today.rtl.lu/news/science-and-environment/a/1498185.html}{vysvětluje},
  že dosud nebyly v supermarketech, restauracích nebo kadeřnických
  salónech zjištěny žádné „nátěrové infekce``.
\item
  Nové údaje o protilátkách z italské komunity Robbio v Lombardii
  ukazují, že
  \href{https://www.tgcom24.mediaset.it/cronaca/a-robbio-pv-il-22-ha-o-ha-avuto-il-coronavirus-ok-del-sindaco-ai-test-per-tutti_17285128-202002a.shtml}{asi
  desetkrát více lidí} než se myslelo mělo koronavirus, ale protože se u
  nich nevyskytly žádné nebo jen mírné příznaky, tak se o tom nevědělo.
  Skutečná míra imunizace je 22\%.
\item
  Nové údaje ze švýcarského kantonu Curych
  \href{https://www.nzz.ch/zuerich/coronavirus-zuerich-aendert-nun-das-testregime-in-heimenauch-viele-aeltere-covid-19-infizierte-entwickeln-keine-symptome-zuerich-aendert-nun-das-testregime-in-heimen-ld.1552089}{ukazují},
  že přibližně 50\% všech úmrtí souvisejících s Covid-19 se vyskytlo v
  domovech pro seniory nebo v pečovatelských domovech. Přesto i tam asi
  40\% všech pozitivně testovaných lidí nevykazovalo žádné příznaky.
  Střední věk úmrtí pozitivních na test ve Švýcarsku je v současné době
  asi 84 let.
\item
  Pietro Vernazza, švýcarský hlavní lékař pro infekci, komentuje
  strategii
  \href{https://infekt.ch/2020/04/exitstrategie-lockdown/}{„žít s
  virem``} a mimo jiné doporučuje individuálně optimalizovanou ochranu
  ohrožených osob. Imunita obyvatelstva je také ochranou ohrožených
  lidí.
\item
  Nový britský web \href{https://lockdownsceptics.org/}{Lockdown
  Skeptics} kriticky informuje o přijatých opatřeních a mediálním
  zpravodajstvím souvisejícími s Covid-19.
\item
  Rakouská občanská společnost
  \href{https://www.initiative-corona.info/}{„Iniciativa pro koronové
  informace založené na důkazech``} poskytuje přehled studií a analýz
  nového koronaviru
\item
  Dokument: \href{https://www.youtube.com/watch?v=dYlia_fQOLk}{„WHO -- V
  sevření lobbistů``} (ARTE, 2017, německy)
\end{itemize}

\hypertarget{7-dubna-2020}{%
\paragraph{7. dubna 2020}\label{7-dubna-2020}}

\hypertarget{posudek-ux161uxe9fa-hamburskuxe9ho-soudnuxedho-luxe9kaux159stvuxed}{%
\subparagraph{\texorpdfstring{\textbf{Posudek šéfa Hamburského soudního
lékařství}}{Posudek šéfa Hamburského soudního lékařství}}\label{posudek-ux161uxe9fa-hamburskuxe9ho-soudnuxedho-luxe9kaux159stvuxed}}

Profesor Klaus Püschel, šéf Hamburského soudního lékařství,
\href{https://www.pressreader.com/germany/hamburger-morgenpost/20200403/281487868456736}{prohlásil
ke Covid19}: „Tento virus ovlivňuje zcela přemrštěným způsobem náš
život. To není v žádném poměru k nebezpečí, které z něho vyplývá. Ani
astronomické hospodářské škody, které teď vznikají, nejsou nebezpečí
přiměřené. Jsem přesvědčen, že se koronová úmrtnost na roční úmrtnosti
neprojeví ani jako výchylka.`` V Hamburku dosud na virus nezemřel
„jediný člověk bez předchozích onemocnění``. „Všichni, které jsme dosud
vyšetřili, měli rakovinu, chronické onemocnění dýchacích cest, byli
silní kuřáci a otylí, trpěli na diabetes nebo měli kardiovaskulární
onemocnění \ldots{} Covid19 je smrtelný jen výjimečně, ve většině
případů je převážně neškodnou virovou infekcí.``

\href{https://www.abendblatt.de/hamburg/article228828787/rechtsmedizin-pueschel-hamburg-corona-virus-infektion-covid-19-coronavirus-krise-patienten-krankenhaeuser-kliniken-infektionsrate-krankheit-pandemie-test-lungenkrankheit-sars-cov-epidemie-sars-cov-2.html\%20/t\%20_blank}{K
tomu dodává dr.}
\href{https://www.abendblatt.de/hamburg/article228828787/rechtsmedizin-pueschel-hamburg-corona-virus-infektion-covid-19-coronavirus-krise-patienten-krankenhaeuser-kliniken-infektionsrate-krankheit-pandemie-test-lungenkrankheit-sars-cov-epidemie-sars-cov-2.html\%20/t\%20_blank}{Püschel}:``V
nemálo případech jsme také zjistili, že aktuální koronová infekce neměla
s úmrtím nic společného, protože jeho příčiny byly jiné, například
krvácení do mozku nebo srdeční infarkt.`` Korona sama o sobě je
„nepříliš nebezpečné virové onemocnění``, říká soudní lékař. Doporučuje
statistiky založené na konkrétních výsledcích vyšetření. „Všechny dohady
o jednotlivých případech úmrtí, které nebyly odborně přezkoumány, pouze
šíří strach.``

Svobodné hansovní město Hamburk v poslední době začalo, v rozporu se
specifikacemi Institutu Roberta Kocha, rozlišovat mezi úmrtími „na``
koronový virus a „s`` koronovým virem, což vedlo ke
\href{https://www.t-online.de/nachrichten/deutschland/id_87636856/coronavirus-hamburg-will-nur-echte-covid-19-tote-zaehlen.html}{snížení}
případů úmrtí na Covid19.

\hypertarget{dalux161uxed-luxe9kaux159skuxe9-informace}{%
\subparagraph{\texorpdfstring{\textbf{Další lékařské
informace}}{Další lékařské informace}}\label{dalux161uxed-luxe9kaux159skuxe9-informace}}

\begin{itemize}
\tightlist
\item
  \href{https://multipolar-magazin.de/artikel/coronavirus-regierung-ignoriert-daten\%20/t\%20_blank}{Nejnovější
  čísla ve}
  \href{https://multipolar-magazin.de/artikel/coronavirus-regierung-ignoriert-daten\%20/t\%20_blank}{zvláštví}
  \href{https://multipolar-magazin.de/artikel/coronavirus-regierung-ignoriert-daten\%20/t\%20_blank}{zprávě}
  německého Institutu Roberta Kocha ukazují, že se tzv. poměr pozitivů
  (t.j. pozitivní výsledky v poměru k počtu testů) zvyšuje zřetelně
  pomaleji než v médiích publikované exponenciální křivky a že koncem
  března byl kolem 10\%, což je pro koronární viry typická hodnota. O
  nějakém „nebezpečně rychlém šíření`` proto podle časopisu Multipolar
  „nemůže být řeč``.
\item
  Německý virolog Hendrik Streeck v současnosti pracuje na pilotní
  studii o šíření původce Covid19 a jeho cestách. V
  \href{https://www.zeit.de/wissen/gesundheit/2020-04/hendrik-streeck-covid-19-heinsberg-symptome-infektionsschutz-massnahmen-studie/komplettansicht}{jednom
  rozhovoru prohlásil}: „Podíval jsem se ještě jednou podrobněji na 31
  ze 40 případů úmrtí v okresu Heinsberg -- a nebyl jsem moc překvapen,
  že ti lidé zemřeli. Jednomu z nich bylo přes 100 let, to by smrt mohla
  způsobit i normální rýma.`` Přenosy přes kliky a podobné tzv. stěrové
  infekce oproti původním předpokladům dosud prokázat nemůže.
\item
  První švýcarská nemocnice musí kvůli velmi nízkému vytížení
  \href{https://www.engadinerpost.ch/2020/4/04/Engadiner-Spitaeler-haben-freie-Kapazitaeten\%20/t\%20_blank}{nahlásit}\href{https://www.engadinerpost.ch/2020/4/04/Engadiner-Spitaeler-haben-freie-Kapazitaeten\%20/t\%20_blank}{kurzarbeit}.
  „Personál ve všech odděleních nemá co dělat a v prvním kroku jsme
  vyčerpali náhrady přesčasů. Teď nahlásíme kurzarbeit. Finanční
  následky jsou velké.`` K připomenutí: na nerealistických předpokladech
  založená studie curyšské ETH
  \href{https://www.toponline.ch/news/coronavirus/detail/news/studie-bestaetigt-engpass-bei-spitalbetten-steht-kurz-bevor-00131333/}{předpovídala}
  k 2. dubnu první přetížení švýcarských nemocnic. K tomu dodnes nedošlo
  nikde.
\item
  Ve Švýcarsku proběhla počátkem 2017 výrazná chřipková vlna. V průběhu
  prvních 6 týdnů došlo ke
  \href{https://www.srf.ch/news/schweiz/todesursachen-statistik-woran-die-meisten-schweizerinnen-und-schweizer-sterben}{1500
  úmrtím navíc} u obyvatelstva nad 65 let. Normálně zemře ve Švýcarsku
  \href{https://www.nzz.ch/lungenentzuendung-1.4550285}{1300 osob} ročně
  na následky zápalu plic, z toho 95\% starších než 65 let. Pro
  srovnání: v současnosti je ve Švýcarsku zaznamenáno celkem
  \href{https://www.corona-data.ch/}{762 úmrtí} „s`` (nikoliv „na``)
  Covid19.
\item
  Vedoucí jedné německé laboratoře pro životní prostředí se domnívá, že
  obyvatelé severoitalské Lombardie jsou v důsledku chronicky vysoké
  zátěže legionelami
  \href{https://m.apotheke-adhoc.de/nachrichten/detail/coronavirus/erhoehen-legionellen-die-todesrate-einer-corona-infektion/}{obzvlášť
  náchylní k virovým onemocněním} jako Covid19. „Jsou-li plíce tak jako
  v současné situaci oslabené virózou, mají bakterie snadnou práci a
  mohou průběh negativně ovlivnit a způsobit komplikace.`` V Lombardii
  docházelo již v minulosti k místním vlnám zápalů plic zapříčiněných
  odpařovacími chladicími zařízeními zamořenými legionelami.
\item
  Na základě údajů z Číny byly celosvětově definovány lékařské
  protokoly, předpokládající pro pozitivně testované intenzivní pacienty
  \textbf{invazivní} \textbf{umělé dýchání} \textbf{intubací}. Na jedné
  straně vycházejí z toho, že neinvazivní dýchání s kyslíkovou maskou
  nestačí a na druhé existuje obava, že by se mohl „nebezpečný virus``
  šířit aerosoly. Avšak již v březnu němečtí lékaři
  \href{https://www.doccheck.com/de/detail/articles/26271-covid-19-beatmung-und-dann}{poukázali
  na to}, že intubace může vést k dalším poškozením plic a má celkově
  špatné vyhlídky úspěšnosti. Mezitím se ozvali také lékaři z USA,
  popisující, že intubace pacientům
  \href{https://www.youtube.com/watch?v=k9GYTc53r2o}{„víc škodí než
  prospívá``}. Pacienti podle nich často netrpí selháním plic, ale
  druhem akutní horské nemoci, kterou umělé dýchání pod zvýšeným tlakem
  ještě zhoršuje. Již v únoru
  \href{https://www.upi.com/Top_News/World-News/2020/02/14/Oxygen-therapy-working-for-coronavirus-patient-Seoul-says/6651581696794/}{informovali
  jihokorejští lékaři}, že pacienti s Covid19 v kritickém stavu dobře
  reagují na kyslíkovou terapii bez dýchacích přístrojů, Výše jmenovaný
  americký lékař varuje, že je naléhavě třeba nasazení dýchacích
  přístrojů přehodnotit, aby nevznikaly dodatečné škody.
\item
  Oficiální model USA pro šíření Covid19 zatím
  \href{https://twitter.com/NikolovScience/status/1246823479820693505}{nadhodnotil}
  hospitalizace osmkrát, intenzivní pacienty šestkrát a potřebu
  dýchacích přístrojů čtyřicetkrát.
\item
  Známý americký statistik Nate Silver upozornil, jak
  \href{https://fivethirtyeight.com/features/coronavirus-case-counts-are-meaningless/}{„nesmyslná``}
  jsou čísla koronových případů, pokud nevíme víc o počtu a provádění
  testů.
\item
  \href{https://www.youtube.com/watch?v=EpSdCh1KT1A}{Příspěvek v relaci
  ARD Monitor} z roku 2009 o přehnaném podání „prasečí chřipky``
  vykazuje úžasné podobnosti se současnou situací. Závěr příspěvku ARD
  tenkrát byl, že „tou skutečnou pandemií je strach z ní``.
\end{itemize}

\hypertarget{5-dubna-2020}{%
\paragraph{5. dubna 2020}\label{5-dubna-2020}}

\begin{itemize}
\tightlist
\item
  Ve velmi poučném
  \href{https://www.youtube.com/watch?v=lGC5sGdz4kg}{40-minutovém
  rozhovoru} prohlásil mezinárodně renomovaný profesor epidemiologie
  Knut Witkowski z New Yorku, že přijatá opatření ke Covid19 jsou vesměs
  kontraproduktivní. Namísto „social distancing, uzavírání škol ,
  „lock-down``, roušek, masového testování a očkování musí život
  pokračovat pokud možno nerušeně a pokud možno rychle vytvořit imunitu
  obyvatelstva. Covid19 není podle všech dosavadních zjištění
  nebezpečnější než předchozí chřipkové epidemie.
\item
  Britský Medical Journal (BMJ)
  \href{https://www.bmj.com/content/369/bmj.m1375}{informuje}, že podle
  nejnovějších dat z Číny 78\% pozitivně testovaných osob nevykazuje
  žádné symptomy. Jeden oxfordský epidemiolog k tomu dodává „Tyhle
  výsledky jsou velmi důležité \ldots{} Jestli jsou reprezentativní,
  musíme se ptát, proč k čertu děláme lock-down?``.
\item
  Dr. Andreas Sönnichsen, vedoucí oddělení pro všeobecné a rodinné
  lékařství Vídeňské univerzity a předseda Sítě pro lékařství založené
  na faktech,
  \href{https://www.diepresse.com/5794224/was-machen-wir-da-auf-den-intensivstationen-eigentlich}{považuje
  dosud přijatá opatření za ``zmatená''}. Kvůli „ochraně nemnohých,
  kteří by mohli být ohroženi``, je ochromen celý stát.
\item
  Švédská vláda
  \href{https://www.telegraph.co.uk/news/2020/04/03/coronavirus-swedish-experiment-could-prove-britain-wrong/\%20/t\%20_blank}{jako
  první na
  světě}\href{https://www.telegraph.co.uk/news/2020/04/03/coronavirus-swedish-experiment-could-prove-britain-wrong/\%20/t\%20_blank}{ozmámila},
  že napříště bude rozlišovat mezi úmrtími „z`` koronaviru a „s``
  koronavirem. To by mohlo vést k redukci ohlašovaného počtu úmrtí. Je
  zajímavé, že mezitím stoupá mezinárodní tlak na Švédsko, aby od své
  liberální strategie upustilo.
\item
  Hamburský zdravotní úřad nechává nově případy pozitivně testovaných
  úmrtí
  \href{https://www.t-online.de/nachrichten/deutschland/id_87636856/coronavirus-hamburg-will-nur-echte-covid-19-tote-zaehlen.html\%20/t\%20_blank}{vyšetřovat
  soudními lékaři}, aby počítal pouze „pravá`` úmrtí na koronu. Tím se
  počet úmrtí oproti údajům Institutu Roberta Kocha snížil až o 50\%.
\item
  Německý Ärzteblatt referoval již 2018 o
  \href{https://www.aerzteblatt.de/nachrichten/97750/Vielzahl-an-Lungenentzuendungen-beunruhigen-Behoerden-in-Norditalien}{„množství
  plicních onemocnění``} v severní Itálii, které znepokojovalo úřady.
  Mezi jinými bylo tenkrát jako příčina uváděno znečištění vody.
\item
  Německá Pharmazeutische Zeitung
  \href{https://www.pharmazeutische-zeitung.de/atemstillstand-koennte-auch-zentrale-ursache-haben-116664/}{poukazuje},
  že v současné situaci často „pacienti těžce onemocní, dokonce zemřou,
  aniž by se předtím vyvinuly problémy dýchání.`` Neurologové z toho
  soudí, že koronové viry mohou škodit i nervovým buňkám. Jiným
  vysvětlením by mohlo být, že tito pacienti často vyžadující péči
  zemřou v důsledku vysokého stresu.
\item
  Podle
  \href{https://www.bag.admin.ch/dam/bag/de/dokumente/mt/k-und-i/aktuelle-ausbrueche-pandemien/2019-nCoV/covid-19-lagebericht.pdf.download.pdf/COVID-19_Epidemiologische_Lage_Schweiz.pdf}{nejnovějších
  čísel ze Švýcarska} jsou nejčastějšími symptomy pozitivně testovaných
  pacientů v nemocnicích horečka, kašel a potíže dýchání. U 43\% nebo
  900 osob se vyskytuje zápal plic. Také v těchto případech není apriori
  jasné, zda ho způsobuje koronový virus nebo jiný původce. Věkový
  medián pozitivně testovaných zesnulých je kolem 83 let, rozpětí
  dosahuje až 101 let.
\item
  Britský projekt
  \href{http://inproportion2.talkigy.com/\%20/t\%20_blank}{„}\href{http://inproportion2.talkigy.com/\%20/t\%20_blank}{In}
  \href{http://inproportion2.talkigy.com/\%20/t\%20_blank}{Proportion}\href{http://inproportion2.talkigy.com/\%20/t\%20_blank}{``}
  sleduje úmrtnost „s`` Covid19 ve srovnání s úmrtností na chřipku a s
  celkovou úmrtností, která je i ve Velké Británii.nadále v normálu a
  pod ním a v současnosti klesá.
\item
  Ve státu Indiana, USA, se z důvodu lock-down
  \href{https://twitter.com/JesseKellyDC/status/1246449878219145216}{zvýšil
  počet volání} na tísňovou linku pro psychické problémy a sebevražedné
  myšlenky o víc než 2000\% z 1000 na 25 000 denně.
\item
  Lékařská odborný portál Rxisk
  \href{https://rxisk.org/medications-compromising-covid-infections/}{poukazuje},
  že různé léky mohou zvýšit riziko nákazy koronovým virem zčásti až o
  200\%.
\end{itemize}

\hypertarget{dalux161uxed-zpruxe1vy}{%
\subparagraph{\texorpdfstring{\textbf{Další
zprávy}}{Další zprávy}}\label{dalux161uxed-zpruxe1vy}}

\begin{itemize}
\tightlist
\item
  Britský žurnalista Peter Hitchens popisuje v článku s názvem
  \href{https://www.firstthings.com/web-exclusives/2020/04/we-love-big-brother\%20/t\%20_blank}{„}\href{https://www.firstthings.com/web-exclusives/2020/04/we-love-big-brother\%20/t\%20_blank}{We}\href{https://www.firstthings.com/web-exclusives/2020/04/we-love-big-brother\%20/t\%20_blank}{love}\href{https://www.firstthings.com/web-exclusives/2020/04/we-love-big-brother\%20/t\%20_blank}{Big}\href{https://www.firstthings.com/web-exclusives/2020/04/we-love-big-brother\%20/t\%20_blank}{Brother``},
  jak se i dosud kritičtí lidé i bez zdravotních důkazů nechají
  naočkovat strachem. V jednom rozhovoru prohlásil s ohledem na ohrožená
  základní práva, že kritika je dnes
  \href{https://www.spiked-online.com/podcast-episode/in-this-lockdown-dissent-is-a-moral-duty/}{„morální
  povinností``}.
\item
  Německý historik René Schlott píše o
  \href{https://www.spiegel.de/politik/deutschland/corona-krise-und-buergerrechte-rendezvous-mit-dem-polizeistaat-a-68611322-f4d4-453f-aba5-5ec5a49ae329\%20/t\%20_blank}{„}\href{https://www.spiegel.de/politik/deutschland/corona-krise-und-buergerrechte-rendezvous-mit-dem-polizeistaat-a-68611322-f4d4-453f-aba5-5ec5a49ae329\%20/t\%20_blank}{Rendesvouz}\href{https://www.spiegel.de/politik/deutschland/corona-krise-und-buergerrechte-rendezvous-mit-dem-polizeistaat-a-68611322-f4d4-453f-aba5-5ec5a49ae329\%20/t\%20_blank}{s
  policejním státem``}: „Koupit si knihu, sedět na lavičce v parku,
  setkat se s přáteli -- to je dnes zakázáno, kontrolováno a
  denuncováno. Jak to tak vypadá, pojistky demokracie shořely. Kde a jak
  to má skončit?``
\item
  V Německu vypracovává více právních kanceláří žaloby proti přijatým
  opatřením a nařízením. Jedna odbornice na lékařské právo
  \href{http://beatebahner.de/lib.medien/aktualisierte\%20Pressemitteilung.pdf}{píše
  v tiskovém sdělení}: „Opatření spolkové a zemských vlád jsou v do očí
  bijícím rozporu s ústavou a porušují v dosud nevídaném rozsahu řadu
  základních práv občanek a občanů Německa. To se vztahuje na všechna
  nařízení 16 spolkových zemí. Zejména nejsou tato opatření oprávněná
  Zákonem o ochraně před infekcemi, který byl teprve před několika dny
  překotně novelizován \ldots{} Dostupná čísla a statistiky totiž
  ukazují, že koronová infekce probíhá (či dokonce již snad proběhla) u
  95\% obyvatel neškodně a nepředstavuje tedy pro veřejnost žádné
  mimořádně závažné ohrožení.``
\item
  \href{https://swprs.org/offener-brief-von-professor-sucharit-bhakdi-an-bundeskanzlerin-dr-angela-merkel/}{Otevřený
  dopis} profesora Sucharita Bhagdi spolkové kancléřce Angele Merkelové
  existuje mezitím v němčině, angličtině, francouzštině, španělštině,
  ruštině, turečtině, holandštině a estonštině, další jazyky následují.
\item
  V jednom \href{https://www.youtube.com/watch?v=-pcQFTzck_c}{Interview
  (EN/DE)} varuje whistleblower Edward Snowden, že Covid19 je nebezpečný
  ale přechodný, zatímco odstranění základních práv je smrtelné a
  trvalé.
\end{itemize}

\hypertarget{3-dubna-2020}{%
\paragraph{3. dubna 2020}\label{3-dubna-2020}}

\textbf{USA}: Další
\href{https://www.youtube.com/watch?v=5pIMD1enwd4}{videa občanských
žurnalistů} ukazují, že v některých nemocnicích popisovaných americkými
médii jako „válečné pole`` je ve skutečnosti velmi poklidně (poznámka:
někteří autoři z toho vyvozují
\href{https://www.politifact.com/factchecks/2020/apr/03/facebook-posts/hospital-beds-being-kept-empty-prepare-covid-influ/}{chybné
závěry}).

\textbf{Rakousko}: Také v Rakousku jsou „koronová úmrtí`` zjevně
definována „velmi liberálně``,
\href{https://www.heute.at/s/osterreich-bei-corona-todesstatistik-sehr-liberal-48665863}{jak
informují média}: „Počítá se ke koronovým úmrtím také, když je dotyčný
virem infikovaný, ale zemřel na něco jiného?`` Ano, říkají Rudi
Anschober a Bernhard Benka, člen Corona Task Force na ministerstvu
zdravotnictví. „Platí jasné pravidlo: úmrtí s koronovým virem anebo
úmrtí na koronový virus,`` dodává Benka. Všechny tyto případy se
započítají do statistiky. Příčina úmrtí se nerozlišuje. Obrazně řečeno,
i devadesátník umírající se zlomeninou stehenního klíčku, který se
několik hodin před smrtí nakazil, platí za koronové úmrtí. To jen jako
příklad.

\textbf{Německo}: Německý Institut Roberta Kocha nově odrazuje od
autopsií pozitivně testovaných, neboť riziko kapénkové infekce aerosoly
má být \href{https://www.youtube.com/watch?v=gSn_YaOYYcY}{příliš
vysoké}. Pak ovšem již v mnoha případech nelze příčinu úmrtí zjistit.

Jeden odborný lékař-patolog to
\href{https://www.youtube.com/watch?v=gSn_YaOYYcY}{komentuje} následovně
(dopis je otištěn pod videem): „Darebák, kdo si přitom pomyslí něco
ošklivého. Dosud bylo pro patology samozřejmé, že, při odpovídajících
opatřeních, obdukují i infekční případy jako HIV/AIDS, hepatitis,
tuberkulóza, prionová onemocnění atd. Je pozoruhodné, že u infekce,
která sklátí tisíce pacientů po celém globu a téměř zablokuje
hospodářství celých zemí, jsou k dispozici jen velmi skrovné obdukční
nálezy (šest pacientů z Číny). Ze zdravotně politického i vědeckého
hlediska by přece měl být obzvlášť velký veřejný zájem na obdukčních
nálezech. Avšak opak je skutečností. Panuje strach dozvědět se skutečnou
příčinu úmrtí pozitivně testovaných? Mohlo by se stát, že by se čísla
koronových úmrtí rozpustila jako sníh na jarním slunci.``

\textbf{Itálie}: Ruský odborný personál si povšiml
\href{https://de.sputniknews.com/panorama/20200402326767475-fachpersonal-todesfaelle-lombardei-zeitung/}{„pozoruhodných
úmrtí``} v pečovatelských domech v Lombardii: „Tak bylo podle novinových
zpráv ve městě Gromo zaznamenáno několik případů, kdy údajně infikovaní
koronovým virem prostě usnuli a už se neprobudili. U zesnulých nebyly do
té doby zaznamenány žádné vážnější příznaky \ldots{} Jak ředitel domu
pro seniory později v rozhovoru pro RIA Novosti upřesnil, není jasné,
zda zesnulí byli skutečně infikováni koronovým virem, neboť v domě nikdo
testován nebyl \ldots{} V domech, kde jsou činné ruské lékařské a
pečovatelské týmy, se chodby, pokoje a jídelny desinfikují.``

Podobné případy již byly
\href{https://web.archive.org/web/20200330082928/https:/www.sueddeutsche.de/panorama/coronavirus-news-deutschland-wolfsburg-laschet-1.4828033}{zaznamenány}
i v Německu. Pacienti v pečovatelských zařízeních zemřou v současné
výjimečné situaci náhle a jsou pak vedeni jako „koronová úmrtí``. Opět
tak vyvstává vážná otázka: kdo zemřel na virus a kdo na zčásti extrémní
opatření?

\textbf{Pečovatelský personál}: Süddeutsche Zeitung
\href{https://www.sueddeutsche.de/politik/coronavirus-pflegekraefte-ausland-1.4866124}{informuje}:
„Po celé Evropě ohrožuje pandemie zabezpečení starých lidí doma, neboť
pečovatelské síly k nim již nesmějí -- anebo zemi úprkem opustily směrem
k domovu.``

\textbf{Další}: Profesor lékařství Stanfordské univerzity Dr. Jay
Bhattacharya poskytl
\href{https://www.youtube.com/watch?v=-UO3Wd5urg0}{půlhodinový
rozhovor}, ve kterém zpochybnil „konvenční vědomosti`` o Covid19.
Dosavadní opatření jsou podle něj přijímána na základě velmi nejistých a
zčásti pochybných datových podkladů.

\hypertarget{2-dubna-2020-i}{%
\paragraph{2. dubna 2020 (I)}\label{2-dubna-2020-i}}

\hypertarget{nux11bmecko}{%
\subparagraph{\texorpdfstring{\textbf{Německo}}{Německo}}\label{nux11bmecko}}

Podle
\href{https://influenza.rki.de/Wochenberichte/2019_2020/2020-13.pdf}{nejnovější
zprávy o chřipce} německého Institutu Roberta Kocha počet akutních
onemocnění dýchacích cest „se ve všech spolkových zemích velmi snížil``.
Hodnoty „ve všech věkových skupinách silně poklesly``.

Do 20. března (12. týdne) celkový počet stacionárních případů s akutními
dýchacími potížemi zřetelně poklesl. Ve věkové skupině 80 let a víc
poklesl počet případů v porovnání s předchozím týdnem téměř na polovinu.

V 73 sledovaných nemocnicích byl v 7\% případů onemocnění dýchacích cest
diagnozován rovněž COVID-19. Ve věkové skupině 35-59 let byl COVID-19
diagnostikován v 16\% a ve skupině 60-79 let v 13\% pacientů.

Tato čísla odpovídají údajům z jiných zemí i základnímu typickému
rozšíření koronových virů (5-15\%).

\href{https://swprs.files.wordpress.com/2020/04/rki-ili-kw13.png}{}

\includegraphics{https://swprs.files.wordpress.com/2020/04/rki-ili-kw13.png?w=279\&h=171}

Chřipková onemocnění (RKI, 13.kal. týden)

\href{https://swprs.files.wordpress.com/2020/04/rki-sari-kw12.png}{}

\includegraphics{https://swprs.files.wordpress.com/2020/04/rki-sari-kw12.png?w=449\&h=171}

Akutní onemocnění dýchacích cest v nemocnicích

Chřipková onemocnění celkem a akutní onemocnění dýchacích cest v
nemocnicích (Institut Roberta Kocha, 13. a 12. kalendářní týden)

\hypertarget{ux10dluxe1nek-v-die-zeit-se-zabuxfdvuxe1-otuxe1zkou-o-pacienty-s-intenzivnuxed-puxe9ux10duxed-v-nux11bmecku}{%
\subparagraph{\texorpdfstring{\href{https://www.zeit.de/wissen/2020-04/krankenhaeuser-kapazitaeten-coronavirus-patienten-deutschland/seite-2\%20/t\%20_blank}{Článek
v Die}
\href{https://www.zeit.de/wissen/2020-04/krankenhaeuser-kapazitaeten-coronavirus-patienten-deutschland/seite-2\%20/t\%20_blank}{Zeit}
se zabývá otázkou o pacienty s intenzivní péčí v
Německu:}{Článek v Die Zeit se zabývá otázkou o pacienty s intenzivní péčí v Německu:}}\label{ux10dluxe1nek-v-die-zeit-se-zabuxfdvuxe1-otuxe1zkou-o-pacienty-s-intenzivnuxed-puxe9ux10duxed-v-nux11bmecku}}

„V současnosti sledují politici, odborníci a mnozí občané exponenciálně
rostoucí počty nově infikovaných s obavami. To však není rozhodující
příznak pro posouzení, jak těžce koronová krize Německo postihuje a
ještě postihne. Je totiž zkreslený zejména každý týden se zvyšujícím
počtem testů..

Pro měření zátěže zdravotního systému je namísto toho důležitý počet
těžce onemocnělých vyžadujících umělé dýchaní. Pokud pro ně existuje
dostatek lůžek, může být mnoho z nich zachráněno. Teprve když se volná
lůžka vyčerpají, hrozí situace jako v Itálii.

Registr lůžek JIP (DIVI-Register) nyní ukazuje, že situace v německých
intenzivních stanicích je uvolněná. ‚Zatím jsme v pohodovém pásmu`, říká
Grabenhenrich. Počty těžkých onemocnění nestoupají zdaleka tak strmě
jako počty nakažených a i kdyby, ještě mnoho lůžek JIP s velmi dobrým
vybavením bychom mohli poskytnout.``

\hypertarget{ux161vuxfdcarsko}{%
\subparagraph{\texorpdfstring{\textbf{Švýcarsko}}{Švýcarsko}}\label{ux161vuxfdcarsko}}

Švýcarský spolkový úřad pro zdraví (BAG)
\href{https://www.bag.admin.ch/bag/de/home/krankheiten/ausbrueche-epidemien-pandemien/aktuelle-ausbrueche-epidemien/novel-cov/situation-schweiz-und-international.html}{hlásí},
že dosud bylo procedeno kolem 139 330 testů na Covid19, z toho vykázalo
15\% pozitivní výsledek (PDF). I toto číslo odpovídá hodnotě typické pro
koronové viry známé z jiných zemí a zdá se, že se ani ve Švýcarsku
zjevně nezvyšuje.

Pouze v médiích často uváděné počty testů se zvyšují exponenciálně, ne
však počty „infikovaných``, onemocnělých či dokonce zemřelých.

31. března byla publikována
\href{https://www.bfs.admin.ch/bfs/de/home/statistiken/gesundheit/gesundheitszustand/sterblichkeit-todesursachen.html}{nová
týdenní úmrtní statistika}, která ve Švýcarsku pro 12. kalendářní týden
(do 22. března) poprvé předpovídá zvýšenou celkovou mortalitu ve věkové
skupině 65+ (viz graf níže). Konkrétně se má celková mortalita zvýšit o
asi 200 případů \emph{týdně navíc}.

Toto zvýšení má být „projevem současné pandemie``. Vyvstává však přitom
následující problém: až do 22. března existovalo ve Švýcarsku
\emph{celkem}
\href{https://de.wikipedia.org/wiki/COVID-19-Pandemie_in_der_Schweiz\#Todesf\%C3\%A4lle}{106
pozitivně testovaných úmrtí}. Zvýšení o 200 úmrtí \emph{týdně} by tedy
znamenalo, že většina úmrtí navíc by nebyla způsobena virem, ale
„protiopatřeními``.

Jiné vysvětlení by bylo, že již bylo započítáno asi 200 případů úmrtí v
\emph{následném týdnu}
(\href{https://de.wikipedia.org/wiki/COVID-19-Pandemie_in_der_Schweiz\#Todesf\%C3\%A4lle}{13.
kalendářní týden}). To by znamenalo, že všechna pozitivně testovaná
úmrtí jsou považována za \emph{úmrtí navíc}. Vzhledem k věkovému a
zdravotnímu profilu i k
\href{https://swprs.org/rki-relativiert-corona-todesfaelle/}{mezinárodním
zkušenostem} by to ovšem byl zcela pochybný předpoklad.

Skutečně zpráva také poznamenává, že „tyto první odhady jsou dosud velmi
nejisté, takže nelze publikovat žádná přesná čísla.``

Pokud by se ukázalo, že většina pozitivně testovaných úmrtí (věkový
medián 83 let) nejsou \emph{dodatečná}, celková úmrtnost by se nezvýšila
anebo by se zvýšila zejména kvůli drastickým opatřením, jak se
\href{https://swprs.org/offener-brief-von-professor-sucharit-bhakdi-an-bundeskanzlerin-dr-angela-merkel/}{obávají}
někteří odborníci.

\hypertarget{2-dubna-2020-ii}{%
\paragraph{2. dubna 2020 (II)}\label{2-dubna-2020-ii}}

\begin{itemize}
\tightlist
\item
  Již 2018 informoval britský Guardien
  „\href{https://www.theguardian.com/society/2018/dec/09/steep-rise-lung-related-illness-hospitals-nhs\%20/t\%20_blank}{Pollution}
  \href{https://www.theguardian.com/society/2018/dec/09/steep-rise-lung-related-illness-hospitals-nhs\%20/t\%20_blank}{and}
  \href{https://www.theguardian.com/society/2018/dec/09/steep-rise-lung-related-illness-hospitals-nhs\%20/t\%20_blank}{flu}
  \href{https://www.theguardian.com/society/2018/dec/09/steep-rise-lung-related-illness-hospitals-nhs\%20/t\%20_blank}{bring}
  \href{https://www.theguardian.com/society/2018/dec/09/steep-rise-lung-related-illness-hospitals-nhs\%20/t\%20_blank}{steep}
  \href{https://www.theguardian.com/society/2018/dec/09/steep-rise-lung-related-illness-hospitals-nhs\%20/t\%20_blank}{rise}
  \href{https://www.theguardian.com/society/2018/dec/09/steep-rise-lung-related-illness-hospitals-nhs\%20/t\%20_blank}{in}
  \href{https://www.theguardian.com/society/2018/dec/09/steep-rise-lung-related-illness-hospitals-nhs\%20/t\%20_blank}{lung}\href{https://www.theguardian.com/society/2018/dec/09/steep-rise-lung-related-illness-hospitals-nhs\%20/t\%20_blank}{--}\href{https://www.theguardian.com/society/2018/dec/09/steep-rise-lung-related-illness-hospitals-nhs\%20/t\%20_blank}{related}
  \href{https://www.theguardian.com/society/2018/dec/09/steep-rise-lung-related-illness-hospitals-nhs\%20/t\%20_blank}{illnesses}.
  Shortage of specialists adds to worries that surge in respiratory
  diseases is putting pressure on A\&Es.
\item
  Mezitím si stěžují i
  \href{https://pflege-prisma.de/2020/03/31/sterbezahlen-in-pflegeheimen/}{zástupci
  pečovatelských služeb} na restriktivní opatření a nepřiměřené
  zpravodajství médií o Covid19. „I před koronou docházelo v zimních
  měsících v relativně krátkém čase k úmrtí mnoha klientů, aniž by přede
  dveřmi stály televizní týmy a ukazovaly osoby v ochranných oděvech
  vystavující se hrdinně nebezpečí infekce.``
\item
  Čísla ze severoitalského města Tresivo (u Benátek) ukazují, že i přes
  108 pozitivně testovaných zůstává celková úmrtnost v městských
  nemocnicích do konce března
  \href{https://swprs.files.wordpress.com/2020/04/reppublica-treviso.jpg}{přibližně
  stejná} jako v loňském roce. To je další známka, že přechodně zvýšená
  úmrtnost v některých místech by mohly souviset spíše s jinými faktory
  jako panika a kolaps než z koronovým virem.
\item
  Profesor Martin Haditsch, odborný lékař pro mikrobiologii, virologii a
  infekční epidemiologii,
  \href{https://www.youtube.com/watch?v=PtzHH8DhgZM}{ostře kritizuje
  opatření kolem Covid19}. Jsou podle něj „zcela neodůvodněná`` a hrubě
  porušují „přiměřenost a etické zásady``.
\item
  Profesor John Oxford z univerzity Queen Mary v Londýně, světový
  špičkový virolog a specialista na chřipku, dochází k
  \href{https://novuscomms.com/2020/03/31/a-view-from-the-hvivo-open-orphan-orph-laboratory-professor-john-oxford/}{následujícímu
  odhadu vzhledem ke Covid19}: „Osobně bych řekl, že nejlepší rada je
  trávit méně času před televizním zpravodajstvím, které je senzační a
  nepříliš dobré. Osobně považuji tenhle Covid19 za těžkou zimní
  chřipkovou epidemii. V minulém roce jsme v tomto případě měli v
  rizikových skupinách, tedy přes 65\% lidí se srdečními chorobami atd.,
  8000 úmrtí. Nemyslím, že současný Covid19 tenhle počet překoná. Trpíme
  epidemií médií.``
\end{itemize}

\hypertarget{1-dubna-2020}{%
\paragraph{1. dubna 2020}\label{1-dubna-2020}}

\hypertarget{situace-v-ituxe1lii}{%
\subparagraph{\texorpdfstring{\textbf{Situace v
Itálii}}{Situace v Itálii}}\label{situace-v-ituxe1lii}}

Italští lékaři
\href{https://www.scmp.com/news/china/society/article/3076334/coronavirus-strange-pneumonia-seen-lombardy-november-leading}{uvedli},
že již na konci minulého roku pozorovali závažné případy zápalu plic v
severní Itálii. Genetické analýzy však nyní ukazují, že virus Covid-19
se v Itálii objevil až v lednu tohoto roku. „Těžká pneumonie
diagnostikovaná v Itálii v listopadu a prosinci proto musí být způsobena
jiným patogenem,``
\href{https://www.nzz.ch/wissenschaft/coronavirus-der-stammbaum-verraet-woher-es-kommt-ld.1548271}{poznamenal}
virolog. To opět vyvolává otázku, jakou roli hraje v italské situaci
virus Covid-19 a jakou jiné faktory.

30. března jsme zmínili seznam italských lékařů, kteří zemřeli „během
koronové krize``, z nichž mnozí byli starší než 90 let a na léčení se
aktivně nepodíleli. Dnes byly ze seznamu
\href{https://portale.fnomceo.it/elenco-dei-medici-caduti-nel-corso-dellepidemia-di-covid-19/}{odstraněny}
všechny data narození (viz poslední verze v
\href{https://web.archive.org/web/20200328152430/https://portale.fnomceo.it/elenco-dei-medici-caduti-nel-corso-dellepidemia-di-covid-19/}{archivu}).
Zvláštní postup.

Obdrželi jsme také následující zprávu od pozorovatele v Itálii. Ten
uvádí další podrobnosti o dramatické situaci v této zemi, která je
evidentně způsobena něčím jiným než virem:

„V posledních týdnech většina východoevropských sester v Itálii, které
pracovaly 24 hodin denně, 7 dní v týdnu, a starali se o lidi, kteří
potřebují péči, uprchla ze země. A to zejména kvůli neskutečnému
panikaření, zákazům vycházení a hrozbě uzavření hranic v rámci
„nouzových opatření``. Výsledkem bylo, že staří lidé, kteří potřebují
péči, a lidé se zdravotním postižením, někteří bez jakýcholiv
příbuzných, byli necháni napospas osudu.

Mnoho z těchto opuštěných lidí skončilo po několika dnech v nemocnicích,
které byly už dlouhé roky trvale přetížené, v řadě případů kvůli
dehydrataci. Nemocnice bohužel postrádaly personál, který se musel
starat o děti v domácnosti, protože školy a školky byly zavřeny. To pak
vedlo k úplnému kolapsu péče o zdravotně postižené a seniory, zejména v
oblastech, kde byla nařízena ještě tvrdší „opatření``, do už tak
chaotických podmínek.

Ošetřovatelská pohotovost, která byla způsobena panikou, dočasně vedla k
mnoha úmrtím mezi těmi, kteří potřebují péči, a stále více i mezi
mladšími pacienty v nemocnicích. Tyto smrtelné následky pak způsobily
ještě větší paniku mezi odpovědnými osobami a médii, kteří hlásili
zprávy jako „dalších 475 obětí``, nebo „mrtví jsou z nemocnic
odstraňováni armádou``, doprovázené obrázky kolon vojenských náklaďáků s
rakvemi.

To, co donutilo pohřební služby odmítat své služby byl strach ze
„zabijáckého viru``. Na jedné straně tady byla naakumulovaná poptávka po
pohřebních službách a na straně druhé vláda schválila zákon, podle
kterého musely být všichni mrtví s koronavirem spáleni. V katolické
Itálii bylo přitom v minulosti prováděno jen málo kremací, a tak
existovalo jen několik malých krematorií, které velmi rychle dosáhly
svých limitů. Proto museli být zesnulí rozváženi do vzdálených kostelů.

Co se stalo je v~podstatě stejné ve všech zemích. Různá kvalita
zdravotních systémů však vede k~různým dopadům. Proto je v Německu,
Rakousku nebo Švýcarsku méně problémů než v Itálii, Španělsku nebo USA.
Jak však lze vidět na oficiálních údajích, nedošlo k významnému zvýšení
úmrtnosti. Jen malá odchylka, kterou způsobila tato tragédie.``

\hypertarget{situace-v-nemocnicuxedch-v-usa-nux11bmecku-a-ux161vuxfdcarsku}{%
\subparagraph{\texorpdfstring{\textbf{Situace v~nemocnicích v USA,
Německu a
Švýcarsku:}}{Situace v~nemocnicích v USA, Německu a Švýcarsku:}}\label{situace-v-nemocnicuxedch-v-usa-nux11bmecku-a-ux161vuxfdcarsku}}

\begin{itemize}
\tightlist
\item
  Americká televizní stanice CBS byla
  \href{https://www.theblaze.com/news/cbs-news-footage-italy-hospital-nyc}{přistižena},
  jak ukazuje záznamy z italské jednotky intenzivní péče a vydává je
  jako dokumentaci současné situace v New Yorku.\\
\item
  V~rozporu ze zprávami v médiích ani rejstřík německých jednotek
  intenzivní péče
  \href{https://www.divi.de/register/intensivregister}{nevykazuje
  zvýšenou obsazenost}. Zaměstnanec mnichovské kliniky vysvětlil, že
  „týdny čekali na vlnu``, ale že „nedošlo ke zvýšení počtu pacientů``.
  Řekl, že prohlášení politiků neodpovídají jejich vlastním zkušenostem,
  které nepodporují „mýtus o zabijáckém viru``.
\item
  Také na švýcarských klinikách nebyla dosud pozorována zvýšená
  obsazenost. Návštěvník kantonální nemocnice v Lucernu uvádí, že „je
  méně aktivní než normálně``. Celá podlaží byla vyhrazena pro Covid-19,
  ale personál „stále čeká na pacienty``. Také nemocnice v Bernu,
  Basileji, Zugu a Curychu „zely prázdnotou``. I v Ticinu (pozn. překl.:
  italsky mluvící kanton sousedící s~Lombardií)
  \href{https://www.nzz.ch/schweiz/tessin-verlegt-erste-corona-patienten-in-deutschschweizer-spitaeler-ld.1549417}{nejsou
  jednotky intenzivní péče plně využity}, přesto však jsou pacienti
  převáděni do německy-mluvících oddělení. Z čistě lékařského hlediska
  to nedává smysl.\\
\end{itemize}

\hypertarget{dalux161uxed-luxe9kaux159skuxe9-postux159ehy}{%
\subparagraph{\texorpdfstring{\textbf{Další lékařské
postřehy:}}{Další lékařské postřehy:}}\label{dalux161uxed-luxe9kaux159skuxe9-postux159ehy}}

\begin{itemize}
\tightlist
\item
  ~Dr. Ansgar Lohse, virolog a ředitel univerzitní nemocnice
  Hamburg-Eppendorf,
  \href{https://www.mopo.de/hamburg/uke-infektiologe-fordert-es-muessen-sich-mehr-menschen-mit-corona-infizieren-36483636}{požaduje}
  okamžitý konec izolací a zákazů vycházení. \emph{Více lidí by mělo být
  nakaženo koronou.} Střediska denní péče a školy musí být co nejdříve
  znovu otevřeny, aby děti a jejich rodiče mohli být imunní proti
  infekci koronavirem. Pokračování přísných opatření by podle lékaře
  vedlo k hospodářské krizi, která také stojí lidské životy.
\item
  Ve Španělsku je
  \href{https://www.heise.de/tp/features/Das-ist-keine-Krise-sondern-eine-Katastrophe-4694104.html}{15\%
  lidí pozitivních na testy} lékaři a zdravotní sestry. Ačkoli tito
  většinou zůstávají většinou bez příznaků, musí jít do karantény, což
  znamená, že španělský zdravotní systém se stále více zhroutí.
\item
  Dr. John Lee, emeritní profesor patologie,
  \href{https://www.spectator.co.uk/article/how-to-understand-and-report-figures-for-covid-19-deaths-}{píše}
  o velmi zavádějící definici „koronových úmrtí`` a zpravodajství o nich
  v britském deníku The Spectator.
\item
  Nejnovější
  \href{https://swprs.files.wordpress.com/2020/04/die-lage-in-norwegen.pdf}{zprávy
  z Norska}, které vyhodnotil doktor specializující se na
  environmentální toxikologii, opět ukazují, že míra pozitivních
  výsledků testů se nezvyšuje -- jak by se dalo očekávat v případě
  epidemie -- ale kolísá v normálním rozmezí pro koronaviry, mezi 2 a 10
  \%. Průměrný věk zesnulého s pozitivním testem je 84 let, příčiny
  smrti nejsou zveřejnovány, a nedochází k nadměrné úmrtnosti.
\item
  Švédsko, které se dosud obešlo bez radikálních opatření a
  nezaznamenalo zvýšenou úmrtnost (podobně tomu bylo v asijských zemích,
  jako je Japonsko nebo Jižní Korea), je
  \href{https://www.theguardian.com/world/2020/mar/30/catastrophe-sweden-coronavirus-stoicism-lockdown-europe}{pod
  silným tlakem} mezinárodních médií, aby změnilo svou strategii.
\item
  Data ze státu New York ukazují, že míra hospitalizace u osob s
  pozitivním testem by mohla být
  \href{https://www.nytimes.com/2020/03/27/nyregion/new-rochelle-coronavirus.html}{více
  než dvacetkrát nižší}, než se původně předpokládalo.
\item
  Článek na specializovaném portálu
  \href{https://www.doccheck.com/de/detail/articles/26271-covid-19-beatmung-und-dann}{DocCheck}
  se zabývá problémem ventilace pacientů pozitivních na test. U pacientů
  pozitivních na test se oficiálně nedoporučuje jednoduchá ventilace
  přes masku, mimo jiné proto, aby se zabránilo šíření koronaviru
  aerosoly. Proto jsou pacienti s pozitivní intenzivní péčí často
  intubováni přímo. Intubace má však nízké šance na úspěch a často vede
  k dalšímu poškození plic (tzv. Ventilatorem vyvolané poškození plic).
  Stejně jako u léků vyvstává otázka, zda by šetrnější léčba pacientů
  nebyla z lékařského hlediska citlivější.
\end{itemize}

\hypertarget{zpruxe1vy-o-politickuxe9m-vuxfdvoji}{%
\subparagraph{\texorpdfstring{\textbf{Zprávy o politickém
vývoji:}}{Zprávy o politickém vývoji:}}\label{zpruxe1vy-o-politickuxe9m-vuxfdvoji}}

\begin{itemize}
\tightlist
\item
  Německý ministr zahraničí
  \href{https://de.nachrichten.yahoo.com/strobl-b\%C3\%BCrger-verst\%C3\%B6\%C3\%9Fe-gegen-corona-regeln-polizei-melden-095746341.html}{vyzval}
  obyvatelstvo, aby „bylo ostražité a hlásilo porušení pravidel pro
  zadržování koronové epidemie policii``.
  \href{https://www.br.de/nachrichten/bayern/buerger-melden-eifrig-verstoesse-gegen-corona-regeln,RuGXp1h}{„Obzvlášě
  nutné``} je hlásit případy jako „zakázaná shromáždění, děti hrající si
  na hřištích, nebo oslavy``, a také turismus.
\item
  Němečtí odborníci na ústavní právo
  \href{https://www.focus.de/politik/deutschland/corona-regelungen-der-regierung-medizin-darf-nicht-gefaehrlicher-sein-als-die-krankheit_id_11827625.html}{upozorňují}
  na „závažný zásah do základních práv``. Odborník na ústavní právo Hans
  Michael Heinig varuje, že „demokratický ústavní stát se v žádném
  okamžiku nemůže změnit na fašisticko-hysterický hygienický stát``.
  Profesor Christoph Möllers z berlínské Humboldtovy univerzity
  vysvětluje, že zákon o ochraně před infekcemi „nemůže sloužit jako
  ospravedlnění pro tak dalekosáhlá omezení svobod občanů``. Podle
  bývalého předsedy Spolkového ústavního soudu Německa Hanse Jürgena
  Papiera „nouzová opatření neospravedlňují omezení občanských svobod a
  zavádění autoritářského státu a úředního špehování``.
\item
  V několika zemích byly iniciovány petice, jejichž cílem je ukončit
  zákaz vycházení a další zásahy do základních práv. Současně se stále
  více odstraňují videa disidentů na internetu, dokonce i lékařů. V
  Berlíně
  \href{https://www.heise.de/tp/features/Wenn-Demonstranten-zu-Gefaehrdern-erklaert-werden-4692869.html}{policie
  ukončila} povolenou akci na obranu základních práv, při které byla
  distribuována německá ústava.
\end{itemize}

\hypertarget{31-bux159ezna-2020-i}{%
\paragraph{31. března 2020 (I)}\label{31-bux159ezna-2020-i}}

\begin{itemize}
\tightlist
\item
  Dr. Richard Capek a další vědci
  \href{https://coronadaten.wordpress.com/}{již prokázali}, že počet
  testovaných pozitivních jedinců ve vztahu k počtu provedených testů
  zůstává ve všech dosud studovaných zemích konstantní, což hovoří proti
  exponenciálnímu šíření („epidemie``) viru a pouze indikuje
  exponenciální nárůst počtu testů.
\item
  Podíl testovaných jedinců v~jednotlivých zemích se pohybuje mezi 5 a
  15\%, což odpovídá obvyklému šíření koronavirů. Je zajímavé, že tyto
  konstantní číselné hodnoty nejsou aktivně sdělovány
  (\href{https://multipolar-magazin.de/artikel/coronavirus-irrefuhrung-fallzahlen}{nebo
  jsou dokonce odstraňovány}) státními orgány a médii. Místo toho jsou
  zobrazovány exponenciální, ale irelevantní a zavádějící křivky bez
  kontextu.
\item
  Takové chování samozřejmě neodpovídá profesionálním lékařským
  standardům, jak ukazuje přehled tradiční
  \href{https://influenza.rki.de/Saisonberichte/2017.pdf}{zprávy o
  chřipce} německého Robert Koch-Institut (s. 130, viz tabulka níže).
  Zde se kromě počtu detekcí (vpravo) zobrazuje také počet vzorků
  (vlevo, šedé pruhy) a pozitivní rychlost (vlevo, modrá křivka).
\item
  To jasně ukazuje, že během chřipkové sezóny vzroste pozitivní
  frekvence z 0 na 10\% až do 80\% vzorků a po několika týdnech klesne
  zpět na normální hodnotu. Ve srovnání, Covid-19 testy ukazují
  konstantní pozitivní rychlost v normálním rozmezí (viz níže).
\item
  Konstantní Covid19-pozitivní míra pomocí amerických dat (Dr. Richard
  Capek). To platí obdobně pro všechny ostatní země, pro které jsou v
  současné době k dispozici údaje o počtu vzorků.
\end{itemize}

\includegraphics{https://swprs.files.wordpress.com/2020/03/rki-influenza-report-2017.png?w=650\&h=530}

Konstantní Covid19-pozitivní míra pomocí amerických dat (Dr. Richard
Capek). To platí obdobně pro všechny ostatní země, pro které jsou v
současné době k dispozici údaje o počtu vzorků.

\includegraphics{https://swprs.files.wordpress.com/2020/03/infizierte-pro-test2603.jpg?w=600\&h=325}

\hypertarget{31-bux159ezna-2020-ii}{%
\paragraph{31. března 2020 (II)}\label{31-bux159ezna-2020-ii}}

\begin{itemize}
\tightlist
\item
  ~\href{https://off-guardian.org/2020/03/30/covid19-yet-to-impact-europes-overall-mortality/}{Grafická
  analýza evropských monitorovacích údajů} působivě ukazuje, že bez
  ohledu na přijatá opatření zůstala celková úmrtnost v celé Evropě v
  normálním rozmezí nebo níže do 25. března a často výrazně pod úrovní
  předchozích let. Pouze v Itálii (65+) se celková úmrtnost poněkud
  zvýšila (pravděpodobně z několika důvodů), ale stále byla pod
  předchozími chřipkovými obdobími.
\item
  Předseda německého Robert Koch-Institute znovu potvrdil, že již
  existující nemoci a skutečná příčina smrti
  \href{https://swprs.org/rki-relativiert-corona-todesfaelle/}{nehrají
  roli} při definování tzv. „koronových úmrtí``. Z lékařského hlediska
  je taková definice zjevně zavádějící. Má zřejmý a obecně známý účinek,
  který staví politiku a společnost do strachu.
\item
  V Itálii se nyní situace
  \href{https://www.tagesspiegel.de/politik/die-verlangsamung-ist-da-in-italien-zeichnet-sich-die-wende-in-der-coronakrise-ab/25698124.html}{začíná
  zklidňovat}. Jak je známo, dočasně zvýšené úmrtnosti (65+) byly spíše
  lokálními účinky, často doprovázené masovou panikou a zhroucením
  zdravotní péče. Politik ze severní Itálie se například ptá: „Jak je
  možné, že Covid pacienti z Brescie jsou transportováni do Německa,
  zatímco v nedaleké Veroně jsou dvě třetiny lůžek pro intenzivní péči
  prázdné?``
\item
  V článku zveřejněném v Evropském časopise Clinical Investigation,
  profesor medicíny ze Stanfordu John C. Ioannidis
  \href{https://onlinelibrary.wiley.com/doi/abs/10.1111/eci.13222}{kritizuje}
  „škodlivost opatření přijatých na základě nadhodnocených dat a bez
  důkazů``. Dokonce i lékařské publikace zpočátku publikovaly pochybná
  tvrzení.
\item
  Čínská studie zveřejněná v~Chinese Journal of Epidemiology na začátku
  března, která naznačila nespolehlivost testů na virus Covid-19
  (přibližně 50\% falešně pozitivních výsledků u asymptomatických
  pacientů), byla od té doby stažena. Vedoucí autor studie, děkan
  lékařské fakulty, nechtěl uvést důvod pro její stažení a hovořil o
  \href{https://www.npr.org/sections/health-shots/2020/03/26/822084429/in-defense-of-coronavirus-testing-strategy-administration-cited-retracted-study}{„choulostivé
  věci``}, což by mohlo naznačovat politický tlak, jak poznamenal
  novinář NPR. Avšak i bez této studie je nespolehlivost tzv. PCR
  virových testů již dlouho známa: Například v roce 2006 byla
  „nalezena`` masová infekce v kanadském pečovatelském domě koronaviry
  SARS, které se později
  \href{https://www.ncbi.nlm.nih.gov/pmc/articles/PMC2095096/}{ukázaly
  být} koronaviry související s běžným nachlazením (i ty mohou být pro
  rizikové skupiny fatální).
\item
  Autoři německé sítě Risk Management Networks (RiskNET)
  \href{https://www.risknet.de/themen/risknews/covid-19-und-der-blindflug/}{hovoří
  v analýze Covid-19} o „letu naslepo``, jakož i „nedostatečnou
  kompetenci i etiku v zacházení s daty``. Místo provádění více a více
  testů a měření je nutný reprezentativní vzorek. „Smysl a poměr``
  opatření musí být kriticky zkoumán.
\item
  Španělský rozhovor s mezinárodně uznávaným argentinsko-francouzským
  virologem Pablem Goldschmidtem byl
  \href{https://www.rubikon.news/artikel/der-corona-totalitarismus}{přeložen
  do němčiny}. Goldschmidt považuje zavedená opatření za medicínsky
  kontraproduktivní a poznamenává, že člověk musí nyní „číst Hannah
  Arendtovou``, aby pochopil „původ totality``.
\item
  Maďarský premiér Viktor Orban, stejně jako ostatní premiéři a
  prezidenti před ním, \href{https://www.krone.at/2127086}{z velké části
  odstavil} maďarský parlament pomocí „nouzového zákona`` a nyní může v
  zásadě vládnout jako diktátor.
\end{itemize}

\hypertarget{30-bux159ezna-2020-i}{%
\paragraph{30. března 2020 (I)}\label{30-bux159ezna-2020-i}}

\begin{itemize}
\tightlist
\item
  Některé kliniky v~Německu již nemohou pacienty přijímat -- ne proto,
  že je příliš mnoho pacientů nebo příliš málo lůžek, ale proto, že
  \href{https://web.archive.org/web/20200330082928/https:/www.sueddeutsche.de/panorama/coronavirus-news-deutschland-wolfsburg-laschet-1.4828033}{ošetřující
  personál testoval pozitivně}, i když ve většině případů jen stěží
  vykazují jakékoli příznaky. Tento případ znovu ilustruje, jak a proč
  jsou systémy zdravotní péče ochromeny.
\item
  V německém domově pro seniory s pečovatelskou službou a domovem pro
  seniory
  \href{https://web.archive.org/web/20200330082928/https:/www.sueddeutsche.de/panorama/coronavirus-news-deutschland-wolfsburg-laschet-1.4828033}{zemřelo}
  15 lidí pozitivních na test. „Překvapivě však mnoho lidí zemřelo, aniž
  by projevilo příznaky korony.`` Německý lékařský specialista nás
  informuje: „Podle mého lékařského úsudku je prokazatelné, že někteří z
  těchto lidí mohli zemřít v důsledku přijatých opatření. Lidé s demencí
  se dostávají do vysokého stresu, když se v jejich každodenním životě
  dělají velké změny: izolace, žádný fyzický kontakt, případně personál
  s kapucí.``
\item
  Podle
  \href{https://twitter.com/sneatio/status/1244157986832101376}{švýcarského
  farmakologa} Schweizer Inselspital v Bernu donutil zaměstnance, aby
  odešli, zastavili terapie a odložili operace kvůli strachu z Covid-19.
\item
  Profesor Gérard Krause, vedoucí oddělení epidemiologie v německém
  Helmholtzově centru pro výzkum infekce, varuje v německé veřejné
  televizi ZDF, že protikoronová opatření
  \href{https://www.zdf.de/nachrichten/politik/coronavirus-epidemiologe-folgen-helmholtz-100.html}{„mohou
  vést k více úmrtím než samotný virus``}.
\item
  Různá média informovala, že více než 50 lékařů v Itálii již zemřelo
  „během koronové krize``, jako vojáci v bitvě. Pohled na
  \href{https://portale.fnomceo.it/elenco-dei-medici-caduti-nel-corso-dellepidemia-di-covid-19/}{příslušný
  seznam} však ukazuje, že většina zemřelých jsou lékaři v důchodu, a
  zahrnuje mimo jiné i 90leté psychiatry a pediatry, z nichž mnozí
  zřejmě zemřeli na přirozené příčiny.
\item
  Rozsáhlý
  \href{https://www.buzzfeed.com/albertonardelli/coronavirus-testing-iceland}{průzkum
  na Islandu} zjistil, že 50\% všech testovaných pozitivních osob
  nevykázalo „vůbec žádné příznaky``, zatímco ostatních 50\% většinou
  vykázalo „velmi mírné symptomy podobné nachlazení``. Podle islandských
  údajů je úmrtnost Covid-19 v~řádu promile, tj. na úrovni chřipky nebo
  níže. Jedním ze dvou
  \href{https://www.government.is/news/article/?newsid=c65cf658-6eb6-11ea-9462-005056bc4d74}{zemřelých}
  s~pozitivním testem byl „turista s neobvyklými příznaky``.
  (\href{https://www.covid.is/data}{Více informací o Islandu})
\item
  Britský novinář Daily Mail Peter Hitchens
  \href{https://hitchensblog.mailonsunday.co.uk/2020/03/theres-powerful-evidence-this-great-panic-is-foolish-yet-our-freedom-is-still-broken-and-our-economy.html}{píše}:
  „Existují silné důkazy o tom, že tato velká panika je hloupá. Přesto
  je naše svoboda stále narušována a naše hospodářství je zmrzačeno.``
  Hitchens zdůrazňuje, že v některých částech Velké Británie policejní
  roboti \href{https://www.youtube.com/watch?v=fHNxDzLsPeg}{monitorují a
  hlásí} „nadbytečné`` procházky v přírodě. V některých případech
  policejní drony
  \href{https://www.youtube.com/watch?v=D4GEZjUTkqc}{volají na lidi
  prostřednictvím reproduktoru}, aby šli domů a „zachránili tak
  životy``. (Poznámka: Tohle nenapadlo ani George Orwella.)
\item
  Italská tajná služba
  \href{https://www.focus.de/panorama/welt/sorge-vor-sozialen-unruhen-supermaerkte-gepluendert-apotheken-ueberfallen-italiens-geheimdienst-warnt-vor-aufstaenden_id_11826664.html}{varuje}
  před sociálními nepokoji a povstáními. Supermarkety jsou již rabovány
  a lékárny přepadávany.
\item
  Profesor Sucharit Bhakdi mezitím
  \href{https://www.youtube.com/watch?v=LsExPrHCHbw\&feature=emb_title}{zveřejnil
  video} (německy, anglické titulky), ve kterém vysvětluje svůj
  \href{https://swprs.org/open-letter-from-professor-sucharit-bhakdi-to-german-chancellor-dr-angela-merkel/}{otevřený
  dopis} německé kancléři Dr. Angele Merkelové.
\end{itemize}

\hypertarget{30-bux159ezna-2020-ii}{%
\paragraph{30. března 2020 (II)}\label{30-bux159ezna-2020-ii}}

\begin{itemize}
\tightlist
\item
  V několika zemích existuje v souvislosti s Covid-19 stále více důkazů,
  že „léčba může být horší než nemoc``.
\item
  Na jedné straně existuje riziko tzv.
  \href{https://en.wikipedia.org/wiki/Hospital-acquired_infection}{nozokomiálních
  infekcí}, tj. infekcí, které pacient, který může být jen mírně
  nemocný, získá v nemocnici. Odhaduje se, že v Evropě je ročně
  přibližně 2,5 milionu nozokomiálních infekcí a 50 000 úmrtí. I v
  německých jednotkách intenzivní péče asi 15\% pacientů získává
  nozokomiální infekci, včetně pneumonie při umělém dýchání. V
  nemocnicích je také problém bakterií stále odolnějších vůči
  antibiotikům.
\item
  Dalším aspektem jsou zcela jistě dobře míněné, ale někdy velmi
  agresivní metody léčby, které se stále častěji používají u pacientů
  Covid-19. Patří sem zejména podávání steroidů, antibiotik a
  antivirových léčiv (nebo jejich kombinací). Již při léčbě pacientů s
  SARS-1 se ukázalo, že výsledek této léčby byl často
  \href{https://www.sciencedaily.com/releases/2020/02/200206110703.htm}{horší
  a fatálnější} než bez takové léčby.\\
\end{itemize}

\hypertarget{29-bux159ezna-2020}{%
\paragraph{29. března 2020}\label{29-bux159ezna-2020}}

\begin{itemize}
\tightlist
\item
  ~Doktor Sucharit Bhakdi, emeritní profesor lékařské mikrobiologie v
  Mohuči, napsal německé kancléřce Dr. Angele Merkelové
  \href{https://swprs.org/open-letter-from-professor-sucharit-bhakdi-to-german-chancellor-dr-angela-merkel/}{otevřený
  dopis}, v němž vyzval k naléhavému přehodnocení reakce na Covid-19 a
  položil kancléři pět zásadních otázek.
\item
  Poslední
  \href{https://multipolar-magazin.de/artikel/coronavirus-irrefuhrung-fallzahlen}{údaje
  německého Robert Koch-Institut} ukazují, že nárůst osob testovaných
  pozitivně je úměrný nárůstu počtu testů, tj. zůstává zhruba stejný v
  procentuálním vyjádření. To může naznačovat, že nárůst počtu případů
  je způsoben hlavně zvýšením počtu testů, a nikoli kvůli probíhající
  epidemii.
\item
  Milánská mikrobioložka Maria Rita Gismondo
  \href{https://www.secoloditalia.it/2020/03/coronavirus-la-gismondo-ammonisce-duramente-basta-snocciolare-numeri-sui-positivi-sono-dati-falsati/}{vyzývá
  italskou vládu}, aby přestala sdělovat denní počet „korona
  pozitivních``, protože tato čísla jsou „falešná`` a obyvatelstvo
  zbytečně panikaří. Počet pozitivních testů velmi závisí na typu a
  počtu testů a nehovoří nic o zdravotním stavu.
\item
  Dr. John Ioannidis, profesor medicíny a epidemiologie ze Stanfordu,
  provedl obsáhlý
  \href{https://www.youtube.com/watch?v=d6MZy-2fcBw}{hodinový rozhovor}
  o nedostatku údajů pro opatření Covid-19.
\item
  Argentinský virolog Pablo Goldschmidt, který žije ve Francii, považuje
  politickou reakci na Covid-19 za „zcela přehnanou`` a varuje před
  \href{https://www.infobae.com/coronavirus/2020/03/28/para-un-prestigioso-cientifico-argentino-el-coronavirus-no-merece-que-el-planeta-este-en-un-estado-de-parate-total/}{„totalitními
  opatřeními``}. V některých částech Francie je pohyb lidí již sledován
  drony.
\item
  Italský autor Fulvio Grimaldi, narozený v roce 1934, vysvětluje, že
  státní opatření, která jsou v současné době v Itálii prováděna, jsou
  \href{https://www.youtube.com/watch?v=O3BuNp01vpc}{„horší než za
  fašismu``}. Parlament i celá společnost byli degradování do role
  statistů.\\
\end{itemize}

\hypertarget{28-bux159ezna-2020}{%
\paragraph{28. března 2020}\label{28-bux159ezna-2020}}

\begin{itemize}
\tightlist
\item
  Nová
  \href{https://news.yahoo.com/oxford-study-suggests-millions-people-221100162.html}{studie
  Oxfordské university} dospěla k závěru, že Covid-19 může existovat ve
  Velké Británii již od ledna 2020 a že polovina populace může být již
  imunizována, přičemž většina lidí má žádné nebo jen mírné příznaky. To
  by znamenalo, že pro Covid-19 by musela být hospitalizován pouze jeden
  z tisíce lidí.
  (\href{https://www.medrxiv.org/content/10.1101/2020.03.24.20042291v1}{Studie})
\item
  Britská média
  \href{https://www.bbc.com/news/uk-england-beds-bucks-herts-52041709}{informovala}
  o 21leté ženě „která zemřela na Covid-19 bez předchozích nemocí``. Od
  té doby se však
  \href{https://archive.is/20200329015127/https:/www.theguardian.com/world/2020/mar/27/chloe-middleton-death-21-year-old-not-recorded-nhs-covid-19-related}{zjistilo},
  že žena neměla pozitivní test na Covid-19 a zemřela na srdeční
  selhání. Mylná informace o Covid-19 vznikla „protože měla mírný
  kašel``.
\item
  Německý mediální odborník profesor Otfried Jarren kritizoval, že mnoho
  médií
  \href{https://www.deutschlandfunk.de/covid-19-scharfe-kritik-an-ard-und-zdf-wegen.2849.de.html?drn:news_id=1114517}{praktikuje
  nekritickou žurnalistiku}, která zdůrazňuje hrozby a výkonnou moc.
  Podle profesora Jarrena mezi odborníky není téměř žádná diferenciace a
  skutečná debata.\\
\end{itemize}

\hypertarget{27-bux159ezna-2020-i}{%
\paragraph{27. března 2020 (I)}\label{27-bux159ezna-2020-i}}

\begin{itemize}
\tightlist
\item
  \textbf{Itálie}: Podle
  \href{http://www.salute.gov.it/portale/caldo/SISMG_sintesi_ULTIMO.pdf}{nejnovějších
  údajů} zveřejněných italským ministerstvem zdravotnictví je nyní
  celková úmrtnost ve všech věkových skupinách nad 65 let po úmrtí pod
  mírnou zimou výrazně vyšší. Až do 14. března byla celková úmrtnost
  stále pod chřipkovou sezónou 2016/2017, ale mezitím ji možná již
  překročila. Většina z této nadměrné úmrtnosti v současnosti pochází ze
  severní Itálie. Přesná role Covid-19 však ve srovnání s dalšími
  faktory, jako je panika, kolaps zdravotní péče a samotné uzavírání
  oblastí, není dosud jasná.\\
\end{itemize}

\includegraphics{https://swprs.files.wordpress.com/2020/03/italia-mortalita-marzo-14.png?w=600\&h=343}

\begin{itemize}
\tightlist
\item
  \textbf{Francie}: Podle
  \href{https://www.santepubliquefrance.fr/maladies-et-traumatismes/maladies-et-infections-respiratoires/infection-a-coronavirus/documents/bulletin-national/covid-19-point-epidemiologique-du-24-mars-2020}{posledních
  údajů z Francie} zůstává celková úmrtnost na národní úrovni po mírné
  chřipkové sezóně v normálním rozmezí. V některých regionech, zejména
  na severovýchodě Francie, však celková úmrtnost ve věkové skupině nad
  65 let prudce stoupla v souvislosti s virem Covid-19 (viz obrázek
  níže).
\end{itemize}

\includegraphics{https://swprs.files.wordpress.com/2020/03/france-mortality.png?w=650\&h=400}

Francie rovněž poskytuje
\href{https://www.santepubliquefrance.fr/maladies-et-traumatismes/maladies-et-infections-respiratoires/infection-a-coronavirus/documents/bulletin-national/covid-19-point-epidemiologique-du-24-mars-2020}{podrobné
informace} o rozložení věku a již existujících nemocech u pacientů v
intenzivní péčí a zesnulých pacientů pozitivních na test (viz obrázek
níže):

\begin{itemize}
\tightlist
\item
  Průměrný věk zesnulého je 81,2 let.
\item
  78\% zemřelých bylo starších 75 let; 93\% bylo starších 65 let.
\item
  2,4\% zemřelých bylo mladších 65 let a nemělo (identifikovanou)
  předchozí onemocnění.
\item
  Průměrný věk pacientů s intenzivní péčí je 65 let.
\item
  26\% pacientů s intenzivní péčí je starších 75 let; 67\% má předchozí
  onemocnění.
\item
  17\% pacientů s intenzivní péčí je mladších 65 let a nemá žádné
  předchozí onemocnění.
\end{itemize}

Francouzské orgány dodávají, že „podíl epidemie (Covid-19) na celkové
úmrtnosti je třeba určit.``

\href{https://swprs.files.wordpress.com/2020/03/france-age-distribution-march-24.png}{\includegraphics{https://swprs.files.wordpress.com/2020/03/france-age-distribution-march-24.png?w=736\&h=349}}Věkové
rozdělení hospitalizovaných pacientů (vlevo nahoře), pacientů s
intenzivní péčí (vpravo nahoře), pacientů doma (vlevo dole) a zesnulých
(vpravo dole). Zdroj: SPF / 24. března 2020

\begin{itemize}
\tightlist
\item
  \textbf{USA}: Výzkumník Stephen McIntyre
  \href{https://twitter.com/ClimateAudit/status/1243019315462516736}{vyhodnotil}
  oficiální údaje o úmrtích na zápal plic v USA. Obvykle je mezi 3 000 a
  5 500 úmrtí týdně, a tedy výrazně více, než jsou současné údaje pro
  Covid-19. Celkový počet úmrtí v USA je mezi 50 000 a 60 000 za týden.
  (Poznámka: V níže uvedeném grafu nebyly nejnovější údaje za březen
  2020 ještě zcela aktualizovány, takže křivka klesá).
\end{itemize}

\includegraphics{https://swprs.files.wordpress.com/2020/03/us-pneumonia-deaths.png?w=400\&h=360}

\hypertarget{-brituxe1nie}{%
\subparagraph{\texorpdfstring{\textbf{~Británie:}}{~Británie:}}\label{-brituxe1nie}}

\begin{itemize}
\tightlist
\item
  Neil Ferguson z Imperial College London nyní
  \href{https://www.newscientist.com/article/2238578-uk-has-enough-intensive-care-units-for-coronavirus-expert-predicts/}{předpokládá},
  že Spojené království má dostatečnou kapacitu v jednotkách intenzivní
  péče k léčbě pacientů Covid-19.
\item
  John Lee, emeritní profesor patologie,
  \href{https://www.spectator.co.uk/article/The-evidence-on-Covid-19-is-not-as-clear-as-we-think}{tvrdí},
  že konkrétní způsob, jakým jsou registrovány případy Covid-19, vede k
  nadhodnocení odhadu rizika, které představuje Covid-19 ve srovnání s
  běžnými případy chřipky a chladu.
\end{itemize}

\hypertarget{dalux161uxed-tuxe9mata}{%
\subparagraph{\texorpdfstring{\textbf{Další
témata:}}{Další témata:}}\label{dalux161uxed-tuxe9mata}}

\begin{itemize}
\tightlist
\item
  \href{https://medium.com/@nigam/higher-co-infection-rates-in-covid19-b24965088333}{Předběžná
  studie} vědců ze Stanfordské univerzity ukázala, že 20 až 25\%
  pacientů pozitivních na Covid-19 testovalo pozitivně i na jiné viry
  chřipky nebo nachlazení.
\item
  Počet žádostí o pojištění pro případ nezaměstnanosti v USA prudce
  stoupl na rekord
  \href{https://www.businessinsider.com/us-weekly-jobless-claims-record-coronavirus-unemployment-insurance-labor-recession-2020-3}{přes
  tři miliony}. V této souvislosti se také očekává prudký
  \href{https://twitter.com/KoenSwinkels/status/1243066532390977544}{nárůst
  sebevražd}.
\item
  První pacient s pozitivním testem v Německu se nyní zotavil. Podle
  jeho vlastního prohlášení zažil 33letý muž nemoc
  \href{https://www.br.de/nachrichten/bayern/coronavirus-patient-nummer-1-wie-ich-die-quarantaene-erlebte,Rrm4Ul8}{„ne
  tak hroznou jako chřipka``}.
\item
  Španělská média
  \href{https://elpais.com/sociedad/2020-03-25/los-test-rapidos-de-coronavirus-comprados-en-china-no-funcionan.html}{uvádějí},
  že rychlotesty na Covid-19 mají citlivost pouze 30\%, i když by měla
  být alespoň 80\%.
\item
  \href{https://ehjournal.biomedcentral.com/articles/10.1186/1476-069X-2-15}{Studie
  z Číny} v roce 2003 dospěla k závěru, že pravděpodobnost úmrtí na SARS
  je o 84\% vyšší u lidí vystavených mírnému znečištění ovzduší než u
  pacientů z oblastí s čistým vzduchem. Riziko je dokonce o 200\% vyšší
  u lidí z oblastí se silně znečištěným vzduchem.
\item
  Německá síť Evidenzbasierte Medizin (EbM)
  \href{https://www.ebm-netzwerk.de/en/publications/covid-19}{kritizuje
  zpravodajství médií} o viru Covid-19: „Mediální zpravodajství v žádném
  případě nezohledňuje kritéria komunikace rizik založených na důkazech,
  kterou jsme požadovali. Prezentace prvotních údajů bez odkazu na jiné
  příčiny smrti vede k nadhodnocení rizika ``.
\end{itemize}

\hypertarget{27-bux159ezna-2020-ii}{%
\paragraph{27. března 2020 (II)}\label{27-bux159ezna-2020-ii}}

\begin{itemize}
\tightlist
\item
  Německý vědec Dr. Richard Capek v
  \href{https://coronadaten.wordpress.com/}{kvantitativní analýze
  tvrdí}, že „korona epidemie`` je ve skutečnosti „epidemií testů``.
  Capek ukazuje, že zatímco počet testů exponenciálně vzrostl, podíl
  infekcí zůstal stabilní a úmrtnost klesla, což hovoří proti
  exponenciálnímu šíření samotného viru (viz níže).
\item
  Německý profesor virologie Dr. Carsten Scheller z University of
  Würzburg \href{https://www.youtube.com/watch?v=w-uub0urNfw}{v podcastu
  vysvětluje}, že Covid-19 je určitě srovnatelný s chřipkou a dosud vedl
  k menšímu počtu úmrtí. Profesor Scheller má podezření, že
  exponenciální křivky často prezentované v médiích mají více společného
  s rostoucím počtem testů než s neobvyklým rozšířením samotného viru.
  Země jako Německo by se měli inspirovat radši Japonskem a Jižní Koreou
  než Itálií. Navzdory milionům čínských turistů a pouze minimálním
  sociálním omezením tyto země ještě nezažily krizi Covid-19. Jedním z
  důvodů by mohlo být nošení ústních roušek: Ty však nechrání před
  infekcí, jen omezují šíření viru infikovanými lidmi.
\item
  Poslední
  \href{https://www.ecodibergamo.it/stories/bergamo-citta/a-bergamo-decessi-4-volte-oltre-la-medialeco-lancia-unindagine-nei-comuni_1346651_11/}{údaje
  z Bergama (města)} ukazují, že celková úmrtnost v březnu 2020 vzrostla
  z obvykle 150 lidí za měsíc na přibližně 450 lidí. Stále není jasné,
  jaký podíl na tom měl Covid-19 a jaký podíl byl způsoben jinými
  faktory, jako je masová panika, systémový kolaps a omezení pohybu
  osob. Městskou nemocnici zřejmě zaplavili lidé z celého regionu a tak
  zkolabovala.
\item
  Dva Stanfordští profesoři medicíny, Dr. Eran Bendavid a Dr. Jay
  Bhattacharya,
  \href{https://web.archive.org/web/20200325103650/https:/www.wsj.com/articles/is-the-coronavirus-as-deadly-as-they-say-11585088464}{v
  článku} vysvětlují, že letalita Covid-19 je přeceňována o několik řádů
  a je pravděpodobně dokonce v Itálii pouze o 0,01\% až 0,06\%, a tedy
  pod úrovní chřipky. Důvodem tohoto nadhodnocení je velmi podhodnocený
  počet infikovaných lidí (bez příznaků). Jako příklad lze uvést plně
  testovanou italskou komunitu Vo, která ukázala, že
  \href{https://www.repubblica.it/salute/medicina-e-ricerca/2020/03/16/news/coronavirus_studio_il_50-75_dei_casi_a_vo_sono_asintomatici_e_molto_contagiosi-251474302/}{50
  až 75\% pozitivně testovaných osob bylo bez příznaků}.
\item
  Dr. Gerald Gaß, prezident Spolku německých nemocnic, v
  \href{https://www.handelsblatt.com/politik/deutschland/coronakrise-deutsche-krankenhausgesellschaft-wir-sind-besser-vorbereitet-als-italien/25651268.html}{rozhovoru
  s Handelsblatt} vysvětlil, že „extrémní situace v Itálii je způsobena
  především velmi nízkými kapacitami intenzivní péče``.
\item
  Dr. Wolfgang Wodarg, jeden z
  \href{https://www.youtube.com/watch?v=p_AyuhbnPOI}{prvních a vokálních
  kritiků} paniky okolo „Covid19``, byl
  \href{https://www.transparency.de/aktuelles/detail/article/in-eigener-sache-vorstand-beschliesst-ruhen-der-mitgliedschaft-von-wolfgang-wodarg-1/}{prozatímně
  vyloučen} představenstvem Transparency Internantional Germany, kde
  vedl pracovní skupinu pro zdraví. Wodarg už byl kvůli své kritice
  médii vážně napaden.
\item
  Whistleblower NSA Edward Snowden
  \href{https://www.cnet.com/news/snowden-warns-government-surveillance-amid-covid-19-could-be-long-lasting/}{varuje},
  že vlády využívají současnou situaci k rozšíření státního špehování a
  omezení základních práv. Autritářská opatření, která jsou v
  současnosti zaváděna, nemusí být po krizi odvolána.
\end{itemize}

\href{https://swprs.org/a-swiss-doctor-on-covid-19/anzahl-infizierte-und-tests-2603/}{}

\includegraphics{https://swprs.files.wordpress.com/2020/03/anzahl-infizierte-und-tests-2603.jpg?w=356\&h=202}

Number of tests and test-positives (proportional)

\href{https://swprs.org/covid-19-hinweis-ii/infizierte-pro-test2603/}{}

\includegraphics{https://swprs.files.wordpress.com/2020/03/infizierte-pro-test2603.jpg?w=372\&h=202}

Test-positives per number of tests (constant)

Rostoucí počet testů nalézá poměrný počet infekcí, poměr zůstává
konstantní, mluví proti probíhající virové epidemii (Dr. Richard Capek,
data z USA)

\hypertarget{26-bux159ezna-2020-i}{%
\paragraph{26. března 2020 (I)}\label{26-bux159ezna-2020-i}}

\begin{itemize}
\tightlist
\item
  \textbf{USA}: Nejnovější \href{https://healthweather.us/}{údaje z USA}
  z 25. března ukazují klesající počet nemocí podobných chřipce v celé
  zemi, jejichž četnost je nyní výrazně pod víceletým průměrem. Důvodem
  mohou být vládní opatření, která byla platná méně než týden.\\
\end{itemize}

~

\href{https://swprs.org/covid-19-hinweis-ii/us-influenza-trend/}{}

\includegraphics{https://swprs.files.wordpress.com/2020/03/us-influenza-trend.png?w=404\&h=242}

US Influenza Trend (March 25, 2020)

\href{https://swprs.org/covid-19-hinweis-ii/us-illness-levels/}{}

\includegraphics{https://swprs.files.wordpress.com/2020/03/us-illness-levels.png?w=324\&h=242}

US Influenza Trend (March 25, 2020)

USA: Ústup chřipkových nemocí (25. března 2020, KINSA)

\begin{itemize}
\tightlist
\item
  \textbf{Německo}: Poslední
  \href{https://influenza.rki.de/Wochenberichte/2019_2020/2020-12.pdf}{zpráva
  o chřipce} Německého Robert Koch-Institut (RKI) z 24. března
  dokumentuje „celostátní pokles aktivity akutních onemocnění dýchacích
  cest``: Počet chřipkových chorob a počet nemocničních pobytů, které
  způsobují, je pod úrovní předchozí let a v současné době stále klesá.
  RKI pokračuje: „Nárůst počtu návštěv u lékaře nelze v současné době
  vysvětlit ani chřipkovými viry cirkulujícími v populaci, ani
  SARS-CoV-2.``
\end{itemize}

~

\href{https://swprs.org/covid-19-hinweis-ii/rki-atemwegserkrankungen-20-2-2020/}{}

\includegraphics{https://swprs.files.wordpress.com/2020/03/rki-atemwegserkrankungen-20-2-2020.png?w=327\&h=202}

Deutschland: Atemwegserkrankungen 2019/2020 ggü. Vorjahren

\href{https://swprs.org/covid-19-hinweis-ii/rki-kliniken-belegung/}{}

\includegraphics{https://swprs.files.wordpress.com/2020/03/rki-kliniken-belegung.png?w=401\&h=202}

Deutschland: Krankenhausaufenthalte durch Atemwegserkrankungen nach
Altersgruppen

Německo: Snižování chřipkových chorob (20. března 2020, RKI)

\begin{itemize}
\tightlist
\item
  \textbf{Itálie}: Proslulý italský virolog Giulio Tarro
  \href{https://www.cybermednews.eu/index.php/it/health/70871-interview-to-the-virologist-giulio-tarro-the-death-rate-of-covid-19-is-less-than-1-as-confirmed-by-the-national-institute-of-allergy-and-infectious-diseases}{tvrdí},
  že úmrtnost Covid-19 je i v Itálii pod 1\%, a je proto srovnatelná s
  chřipkou. Vyšší hodnoty vznikají pouze proto, že se nerozlišuje mezi
  úmrtími s Covid-19 a úmrtím na Covid-19 a protože počet infikovaných
  osob (bez příznaků) je značně podhodnocován.
\item
  \textbf{Británie}: Autoři studie British Imperial College, kteří
  předpovídali až 500 000 úmrtí, opět snižují své předpovědi. Poté, co
  již \href{https://www.bbc.com/news/health-51979654}{připustili}, že
  velká část úmrtí s~pozitivním testem jsou součástí normální úmrtnosti,
  nyní uvádějí, že vrchol nemoci může být
  \href{https://www.thetimes.co.uk/article/nhs-now-likely-to-cope-with-coronavirus-says-key-scientist-rn5m6nggk}{dosažen
  již za dva až tři týdny}.
\item
  \textbf{Británie}: The British Guardian
  \href{https://www.theguardian.com/society/2019/feb/20/britons-urged-to-get-flu-vaccine-as-critical-cases-rise-above-2000}{informoval
  v únoru 2019}, že dokonce i ve všeobecně slabé chřipkové sezóně
  2018/2019 bylo ve Velké Británii více než 2180 lidí přijato do
  jednotky intenzivní péče v souvislosti s chřipkou.
\item
  \textbf{Švýcarsko}: Ve Švýcarsku je nadměrná úmrtnost na Covid-19
  zřejmě stále nulová. Poslední „fatální obětí`` představenou v médiích
  je
  \href{https://www.nau.ch/ort/basel/drei-weitere-covid-19-todesfalle-in-basel-stadt-65684099}{100letá
  žena}. Švýcarská vláda nicméně nadále zpřísňuje omezující opatření.
\end{itemize}

\hypertarget{26-bux159ezna-2020-ii}{%
\paragraph{26. března 2020 (II)}\label{26-bux159ezna-2020-ii}}

\begin{itemize}
\tightlist
\item
  \textbf{Švédsko}: Švédsko doposud ve svém přístupu ke Covid-19
  uplatňovalo nejliberálnější strategii, která je
  \href{https://www.zeit.de/politik/ausland/2020-03/coronavirus-schweden-stockholm-oeffentliches-leben/komplettansicht}{založena
  na dvou zásadách}: rizikové skupiny jsou chráněny a lidé s příznaky
  chřipky zůstávají doma. ``Pokud se budete řídit těmito dvěma pravidly,
  není třeba dalších opatření, jejichž účinek je stejně marginální,''
  uvedl hlavní epidemiolog Anders Tegnell. Společenský a ekonomický
  život bude pokračovat normálně. Velký nával v nemocnicích dosud
  nenastal, řekl Tegnell.
\item
  Německá expertka na trestní a ústavní právo Dr. Jessica Hamed
  \href{https://www.fr.de/politik/coronakrise-deutschland-sind-kontaktsperren-ausgangsbeschraenkungen-rechtswidrig-13611821.html}{tvrdí},
  že opatření, jako jsou obecné zákazy vycházení a zákaz kontaktů, jsou
  masivním a nepřiměřeným zásahem do základních práv svobody, a proto
  jsou pravděpodobně „nezákonná``.
\item
  Poslední \href{https://www.euromomo.eu/index.html}{evropská
  monitorovací zpráva} o celkové úmrtnosti nadále vykazuje normální nebo
  podprůměrné hodnoty ve všech zemích a ve všech věkových skupinách, ale
  nyní s
  \href{https://www.euromomo.eu/outputs/zscore_country65.html}{jednou
  výjimkou}: ve věkové skupině nad 65 let v Itálii se v současnosti
  předpokládá zvýšená celková úmrtnost (tzv. delay-adjusted z-score) --
  to je však stále pod hodnotami chřipkových vln v letech 2017 a 2018.
\end{itemize}

\hypertarget{25-bux159ezna-2020}{%
\paragraph{25. března 2020}\label{25-bux159ezna-2020}}

\begin{itemize}
\tightlist
\item
  Německý imunolog a toxikolog, profesor Stefan Hockertz, v
  \href{https://www.youtube.com/watch?v=7wfb-B0BWmo}{rozhlasovém
  rozhovoru} vysvětluje, že Covid-19 není nebezpečnější než chřipka, ale
  že je jednoduše sledován mnohem podrobněji. Mnohem nebezpečnější než
  virus jsou strach a panika vyvolaná médii a „autoritářská reakce``
  mnoha vlád. Profesor Hockertz také poznamenává, že u většiny tzv.
  „koronových úmrtí`` ve skutečnosti lidé zemřeli na jiné příčiny a
  zároveň pozitivně testovali na koronaviry. Hockertz věří, že až
  desetkrát více lidí, než bylo uvedeno, již Covid-19 mělo, akorát nebyl
  zjištěn, neboť příznaky byly nulové nebo minimální.
\item
  Argentinský virolog a biochemik Pablo Goldschmidt vysvětluje, že
  Covid-19
  \href{https://www.clarin.com/buena-vida/coronavirus-panico-injustificado-dice-virologo-argentino-francia_0_yVcmJ4RM.html}{není
  nebezpečnější než závažná nachlazení nebo chřipka}. Je dokonce možné,
  že virus Covid-19 cirkuloval již v předchozích letech, ale nebyl
  objeven, protože ho nikdo nehledal. Dr. Goldschmidt hovoří o
  „globálním teroru`` vytvořeném médii a politikou. Každý rok, jak říká,
  zemřou na zápal plic tři miliony novorozenců po celém světě a 50 000
  dospělých v USA.
\item
  Profesor Martin Exner, vedoucí Ústavu hygieny na Univerzitě v Bonnu,
  \href{https://www.youtube.com/watch?v=9mI9trSm3PY}{v rozhovoru
  vysvětluje}, proč je zdravotnický personál v současné době pod tlakem,
  přestože v Německu dosud nedošlo k téměř žádnému nárůstu počtu
  pacientů: Na jedné straně , lékaři a zdravotní sestry, kteří byli
  pozitivně testováni, musí být v karanténě a často je obtížné je
  nahradit. Na druhé straně sestry ze sousedních zemí, které poskytují
  důležitou část péče, v současné době nemohou vstoupit do země kvůli
  uzavřeným hranicím.
\item
  Profesor Julian Nida-Ruemelin, bývalý německý státní tajemník pro
  kulturu a profesor etiky,
  \href{https://www.zdf.de/nachrichten/zdf-morgenmagazin/julian-nida-ruemelin-zur-corona-krise-100.html}{zdůrazňuje},
  že Covid-19 nepředstavuje žádné riziko pro zdravou obecnou populaci, a
  že extrémní opatření, jako jsou zákaz vycházení, tedy nejsou
  opodstatněná.
\item
  Na základě údajů z výletní lodi Diamond Princess profesor Stanford
  John Ioannidis
  \href{https://www.statnews.com/2020/03/17/a-fiasco-in-the-making-as-the-coronavirus-pandemic-takes-hold-we-are-making-decisions-without-reliable-data/}{ukázal},
  že letalita Covid-19 s korekcí na věk je mezi 0,025\% a 0,625\%, tj. v
  rozmezí závažných nachlazení nebo chřipky.
  \href{https://www.niid.go.jp/niid/en/2019-ncov-e/9407-covid-dp-fe-01.html}{Japonská
  studie} navíc ukázala, že ze všech cestujících pozitivních na test a
  navzdory vysokému průměrnému věku zůstalo 48\% zcela bez příznaků;
  dokonce mezi 80-89 lety 48\% zůstalo bez příznaků, zatímco mezi 70 až
  79 lety to bylo ohromujících 60\%, u kterých se nevyskytly žádné
  příznaky. To opět vyvolává otázku, zda již existující nemoci nejsou
  možná důležitějším faktorem než samotný virus. Italský příklad ukázal,
  že
  \href{https://www.bloomberg.com/news/articles/2020-03-18/99-of-those-who-died-from-virus-had-other-illness-italy-says}{99\%
  úmrtí s pozitivním testem} mělo jednu nebo více již existujících
  nemocí, a dokonce i mezi nimi pouze
  \href{https://web.archive.org/web/20200324214448/https:/www.telegraph.co.uk/global-health/science-and-disease/have-many-coronavirus-patients-died-italy/}{12\%
  úmrtních listů} uvádí jako příčinný faktor Covid19.
\end{itemize}

\hypertarget{24-bux159ezna-2020}{%
\paragraph{24. března 2020}\label{24-bux159ezna-2020}}

\begin{itemize}
\tightlist
\item
  Spojené království odstranilo Covid-19 z oficiálního seznamu
  infekčních nemocí s vysokým výskytem (HCID) a uvedlo, že úmrtnost je
  \href{https://www.gov.uk/guidance/high-consequence-infectious-diseases-hcid\#status-of-covid-19}{„celkově
  nízká``}.
\item
  Ředitel Německého národního zdravotního ústavu (RKI)
  \href{https://swprs.org/rki-relativiert-corona-todesfaelle/}{připustil},
  že všechny úmrtí s pozitivním testem se bez ohledu na skutečnou
  příčinu smrti považují za „úmrtí na koronaviry``. Průměrný věk
  zesnulého je 82 let, nejvíce se závažnými předpoklady. Stejně jako ve
  většině ostatních zemí bude nadměrná úmrtnost kvůli Covid-19 v Německu
  pravděpodobně nulová.
\item
  Lůžka ve švýcarských jednotkách intenzivní péče vyhrazená pro pacienty
  s Covid-19 jsou stále
  \href{https://www.aargauerzeitung.ch/aargau/kanton-aargau/erst-3-von-100-aargauer-betten-der-intensivstationen-sind-belegt-so-ruesten-sich-die-spitaeler-auf-die-epidemie-137332716}{„většinou
  prázdná``}.
\item
  Německý profesor Karin Moelling, bývalý předseda lékařské virologie na
  univerzitě v Curychu, uvedl v
  \href{https://www.radioeins.de/programm/sendungen/die_profis/archivierte_sendungen/beitraege/corona-virus-kein-killervirus.html}{rozhovoru},
  že Covid-19 „není žádný virus zabiják`` a že „panika musí skončit``.
\end{itemize}

\hypertarget{23-bux159ezna-2020-i}{%
\paragraph{23. března 2020 (I)}\label{23-bux159ezna-2020-i}}

\begin{itemize}
\tightlist
\item
  Nová francouzská studie v Journal of Antimicrobial Agents, nazvaná
  \href{https://www.sciencedirect.com/science/article/abs/pii/S0924857920300972}{SARS-CoV-2:
  fear versus data}, dospěla k závěru, že „problém SARS-CoV-2 je
  pravděpodobně přeceňován``, protože „úmrtnost na SARS-CoV-2 je
  významně se neliší od běžných koronavirů identifikovaných ve studijní
  nemocnici ve Francii ``.
\item
  \href{https://www.ijidonline.com/article/S1201-9712(19)30328-5/fulltext}{Italská
  studie z srpna 2019} zjistila, že úmrtí na chřipku v Itálii bylo v
  posledních letech 7 000 až 25 000. Tato hodnota je vyšší než ve
  většině ostatních evropských zemí v důsledku velké starší populace v
  Itálii a mnohem vyšší než cokoli, co bylo dosud připisováno Covid-19.
\item
  Světová zdravotnická organizace (WHO) ve
  \href{https://www.who.int/news-room/q-a-detail/q-a-similarities-and-differences-covid-19-and-influenza}{své
  nové publikaci} uvádí, že se Covid-19 ve skutečnosti šíří pomaleji, ne
  rychleji, než chřipka, a to přibližně o 50\%. Navíc se zdá, že
  předsymptomatický přenos je u přípravku Covid-19 mnohem nižší než u
  chřipky.
\item
  Přední italský lékař uvádí, že
  \href{https://www.scmp.com/news/china/society/article/3076334/coronavirus-strange-pneumonia-seen-lombardy-november-leading}{„podivné
  případy zápalu plic``} byly v oblasti Lombardie zaznamenány už v
  listopadu 2019, což opět vyvolává otázku, zda byly způsobeny novým
  virem (který se v Itálii oficiálně objevil až v únoru 2020) nebo jiným
  faktory, jako je
  \href{https://www.thelocal.it/20170131/our-lungs-are-breaking-smog-levels-way-above-safe-limits-in-northern-italy}{nebezpečně
  vysoká úroveň smogu} v severní Itálii.
\item
  Dánský vědec Peter Gøtzsche, zakladatel renomované Cochrane Medical
  Collaboration, píše, že Corona je
  \href{https://www.deadlymedicines.dk/corona-an-epidemic-of-mass-panic/}{„epidemií
  masové paniky``} a „logika byla jednou z prvních obětí``.
\end{itemize}

\hypertarget{23-bux159ezna-2020-ii}{%
\paragraph{23. března 2020 (II)}\label{23-bux159ezna-2020-ii}}

\begin{itemize}
\tightlist
\item
  Bývalý izraelský ministr zdravotnictví, profesor Yoram Lass,
  \href{https://en.globes.co.il/en/article-lockdown-lunacy-1001322696}{říká},
  že nový koronavirus je „méně nebezpečný než chřipka`` a uzavírání
  oblastí „zabijí více lidí než virus``. Dodává, že „čísla neodpovídají
  panice`` a „psychologie převládá nad vědou``. Rovněž poznamenává, že
  „Itálie je známá svou obrovskou morbiditou při respiračních
  problémech, což je více než třikrát více než jakákoli jiná evropská
  země.``
\item
  Pietro Vernazza, švýcarský specialista na infekční choroby, tvrdí, že
  mimořádná opatření většinou
  \href{https://www.tagblatt.ch/leben/ostschweizer-infektiologe-pietro-vernazza-die-zahlen-zu-den-jungen-corona-virus-erkrankten-sind-irrefuehrend-ld.1206440}{nejsou
  založena na vědeckých poznatcích} a měla by být proto odvolána. Podle
  Vernazzy nemá hromadné testování smysl, protože 90\% populace neuvidí
  žádné příznaky a uzamčení a uzavření školy jsou dokonce
  „kontraproduktivní``. Doporučuje chránit pouze rizikové skupiny a
  zároveň nenarušovat chod hospodářství a společnosti.
\item
  Prezident Světové federace lékařů (World Doctors Federation) Frank
  Ulrich Montgomery
  \href{https://www.general-anzeiger-bonn.de/news/politik/deutschland/interview-mit-weltaerztepraesident-montgomery-ueber-corona-pandemie-ist-chaos_aid-49609561}{tvrdí},
  že uzavírání oblastí jako v Itálii jsou „nepřiměřená`` a
  „kontraproduktivní`` a měla by být odvolána.
\item
  Švýcarsko: Přes paniku médií byla nadměrná úmrtnost stále na nule nebo
  blízko nuly: posledními pozitivně testovanými
  \href{https://www.bluewin.ch/de/newsregional/zuerich/1068-bestatigte-corona-falle-und-funf-todesfalle-im-kanton-zurich-371873.html}{„oběťmi``}
  byla 96letá osoba v paliativní péči a 97letá osoba s existujícími
  onemocněními.
\item
  Poslední statistická zpráva Italského národního zdravotního institutu
  je nyní
  \href{https://www.epicentro.iss.it/coronavirus/bollettino/Report-COVID-2019_20_marzo_eng.pdf}{k
  dispozici v angličtině}.
\end{itemize}

\hypertarget{22-bux159ezna-2020-i}{%
\paragraph{22. března 2020 (I)}\label{22-bux159ezna-2020-i}}

\begin{itemize}
\tightlist
\item
  Pokud jde o situaci v Itálii: Většina hlavních médií falešně uvádí, že
  Itálie má až 800 úmrtí denně na coronavirus. Prezident italské Služby
  civilní ochrany (Protezione Civile) ve skutečnosti zdůrazňuje, že se
  jedná o úmrtí „s koronavirem a nikoli na koronavirus`` (minuta 03:30
  \href{https://youtu.be/0M4kbPDHGR0?t=210}{tiskové konference}). Jinými
  slovy, tyto osoby zemřely, zatímco byly také pozitivní.
\item
  Jak
  \href{https://www.statnews.com/2020/03/17/a-fiasco-in-the-making-as-the-coronavirus-pandemic-takes-hold-we-are-making-decisions-without-reliable-data/}{ukázali}
  profesoři Ioannidis a Bhakdi, v zemích jako Jižní Korea a Japonsko,
  které nezavedly žádná omezení pohybu osob, došlo v souvislosti s
  Covidem-19 k téměř nulové nadměrné úmrtnosti, zatímco výletní loď
  Diamond Princess zažila extrapolovanou úmrtnost v řádu promile, tj. na
  úrovni nebo pod úrovní sezónní chřipky.
\item
  Současné údaje o úmrtích s pozitivním testem v Itálii jsou stále nižší
  než 50\% běžné denní úmrtnosti v Itálii, což je kolem 1800 úmrtí
  denně. Je tedy možné, snad i pravděpodobné, že velká část normální
  denní úmrtnosti se nyní jednoduše počítá jako úmrtí „Covid19``
  (protože mají pozitivní testy). To je to, co zdůrazňoval předseda
  italské Služby civilní ochrany.
\item
  Dosud je však zřejmé, že v některých regionech v severní Itálii, tj. v
  regionech, které čelí nejdrastičtějším
  \href{https://en.wikipedia.org/wiki/2020_Italy_coronavirus_lockdown}{omezením},
  se výrazně zvyšuje denní úmrtnost. Je také známo, že v oblasti
  Lombardie se 90\% úmrtí s~pozitivním testem neodehrává na jednotkách
  intenzivní péče, ale spíše
  \href{https://www.tgcom24.mediaset.it/cronaca/coronavirus-in-lombardia-9-morti-su-10-mai-giunti-in-terapia-intensiva_16362350-202002a.shtml}{doma}.
  A více než 99\% má vážné již existující nemoci.
\item
  Profesor Sucharit Bhakdi
  \href{https://www.youtube.com/watch?v=JBB9bA-gXL4}{označil} obecné
  zákazy vycházení za „zbytečnou``, „sebezničující`` „kolektivní
  sebevraždu``. Vyvstává tedy mimořádně znepokojivá otázka, do jaké míry
  může být zvýšená úmrtnost těchto starších, izolovaných, vysoce
  stresovaných lidí s mnohonásobnými již existujícími onemocněními ve
  skutečnosti způsobena týdny trvajícími a stále platnými opatřeními
  omezující pohyb osob.
\item
  Pokud ano, může to být jeden z případů, kdy je léčba horší než nemoc.
  (Viz aktualizace níže: pouze 12\% úmrtních listů uvádí koronavirus
  jako příčinu.)
\end{itemize}

\includegraphics{https://swprs.files.wordpress.com/2020/03/borrelli2.jpg?w=550\&h=309}

\hypertarget{22-bux159ezna-2020-ii}{%
\paragraph{22. března 2020 (II)}\label{22-bux159ezna-2020-ii}}

\begin{itemize}
\tightlist
\item
  Ve Švýcarsku je v současné době 56 úmrtí s~pozitivním testem, z nichž
  všichni byli
  \href{https://www.nzz.ch/schweiz/coronavirus-in-der-schweiz-die-neusten-entwicklungen-ld.1542664\#subtitle-wie-viele-infizierte-und-todesf-lle-gibt-es-second}{„vysoce
  rizikovými pacienty``} kvůli jejich pokročilému věku a / nebo již
  existujícím zdravotním stavům. Jejich skutečná příčina smrti, tj.
  kvůli viru nebo jednoduše s virem, nebyla sdělena.
\item
  Švýcarská vláda tvrdila, že situace v jižním Švýcarsku (sousedícím s
  Itálií) je „dramatická``, ale místní lékaři to
  \href{https://www.nzz.ch/schweiz/punkto-intensivbetten-sind-wir-im-tessin-besser-ausgeruestet-als-der-rest-der-schweiz-ld.1547728}{popřeli}
  a uvedli, že je vše normální.
\item
  Podle
  \href{https://www.blick.ch/news/schweiz/nicht-nur-beatmungsgeraete-werden-knapp-im-kampf-gegen-corona-es-droht-ein-engpass-beim-sauerstoff-id15808185.html}{zpráv
  z tisku} hrozí nedostatek kyslíkových láhví. Důvodem však není v
  současné době vyšší využití, ale spíše hromadění kvůli strachu z
  budoucího nedostatku.
\item
  V mnoha zemích již roste
  \href{https://www.washingtonpost.com/health/covid-19-hits-doctors-nurses-emts-threatening-health-system/2020/03/17/f21147e8-67aa-11ea-b313-df458622c2cc_story.html}{nedostatek}
  lékařů a zdravotních sester. Důvodem je především to, že zdravotničtí
  pracovníci, kteří testují pozitivně, musí být v karanténě, i když v
  mnoha případech zůstanou zcela nebo téměř bez příznaků.
\end{itemize}

\hypertarget{22-bux159ezna-2020-iii}{%
\paragraph{22. března 2020 (III)}\label{22-bux159ezna-2020-iii}}

\begin{itemize}
\tightlist
\item
  Model z Imperial College London předpověděl, že mezi 250 000 a 500 000
  úmrtími ve Velké Británii „kvůli`` Covid-19. Nyní však autoři studie
  \href{https://www.bbc.com/news/health-51979654}{připustili}, že mnoho
  z těchto úmrtí nebude mimořádných, ale spíše v rámci běžné roční
  úmrtnosti, která je ve Velké Británii asi 600 000 lidí ročně. Jinými
  slovy, nadměrná úmrtnost by zůstala nízká.
\item
  Dr. David Katz, zakládající ředitel Výzkumného centra pro prevenci na
  univerzitě v Yale, se v
  \href{https://www.nytimes.com/2020/03/20/opinion/coronavirus-pandemic-social-distancing.html}{New
  York Times} ptá: „Je náš boj proti Coronavirus horší než nemoc?
  Nemohou existovat cílenější způsoby, jak porazit pandemii.``
\item
  Podle italského profesora Waltera Ricciardiho
  \href{https://web.archive.org/web/20200324214448/https:/www.telegraph.co.uk/global-health/science-and-disease/have-many-coronavirus-patients-died-italy/}{„pouze
  12\% úmrtních listů prokázalo přímou kauzalitu na koronaviry``},
  zatímco ve zprávách v médiích „všichni lidé, kteří zemřou v
  nemocnicích s koronaviry, jsou považováni za zemřelé na koronaviry``.
  To znamená, že údaje o italských úmrtích hlášené médii musí být
  vydělené nejméně osmi, aby se získaly skutečné úmrtí způsobené virem.
  Pokud vezmeme v úvahu, že ~celková úmrtností činí 1800 úmrtí denně a
  až 20 000 lidí umírá ročně na chřipku, dostaneme nadměrnou denní
  úmrtnost nanejvýš v řádu desítek.\\
\end{itemize}

\hypertarget{21-bux159ezna-2020-i}{%
\paragraph{21. března 2020 (I)}\label{21-bux159ezna-2020-i}}

\begin{itemize}
\tightlist
\item
  Španělsko hlásí pouze tři úmrtí s pozitivním testem
  \href{https://www.20minutos.es/noticia/4193883/0/media-edad-coronavirus-espana/}{v~kategorii
  pod 65 let} (z celkem asi 1000). Jejich dosavadní zdravotní stav a
  skutečná příčina smrti nejsou dosud známy.
\item
  20. března Itálie
  \href{https://www.msn.com/en-au/news/coronavirus/italy-coronavirus-deaths-surge-by-627-in-a-day-lifting-total-death-toll-to-4032/ar-BB11tDnS}{oznámila}
  627 celostátních úmrtí s pozitivním testem za jeden den. Pro srovnání,
  normální celková úmrtnost v Itálii je asi 1800 úmrtí denně. Od 21.
  února oznámila Itálie asi 4000 úmrtí s pozitivním testem. Normální
  celková úmrtnost v tomto časovém rámci je až 50 000 úmrtí. Dosud není
  známo, do jaké míry se normální celková úmrtnost zvýšila, nebo do jaké
  míry se jednoduše zvýšil podíl smrtí s pozitivním testem. Itálie a
  Evropa navíc prošly v letech 2019/2020 velmi mírným chřipkovým
  obdobím, které ušetřilo mnoho jinak zranitelných lidí.
\item
  Podle
  \href{https://www.tgcom24.mediaset.it/cronaca/coronavirus-in-lombardia-9-morti-su-10-mai-giunti-in-terapia-intensiva_16362350-202002a.shtml}{zpráv
  v italských mediích}, 90\% zemřelých s pozitivním testem v oblasti
  Lombardie zemřelo mimo jednotky intenzivní péče, většinou doma nebo v
  zařízeních všeobecné péče. Stále není jasné jaká je příčina jejich
  smrti a jestli k ní nějak přispěly karanténní opatření. Na JIP zemřelo
  pouze 260 z 2168 pozitivních osob.
\item
  Bloomberg vydává zprávu s~titulkem:
  \href{https://www.bloomberg.com/news/articles/2020-03-18/99-of-those-who-died-from-virus-had-other-illness-italy-says}{„99\%
  hlášených obětí viru mělo jinou nemoc, říká Itálie.``}
\end{itemize}

\includegraphics{https://swprs.files.wordpress.com/2020/03/covid-iss-stat-bloomberg.png?w=550\&h=301}

\hypertarget{21-bux159ezna-2020-ii}{%
\paragraph{21. března 2020 (II)}\label{21-bux159ezna-2020-ii}}

\begin{itemize}
\tightlist
\item
  The Japan Times se ptá:
  \href{https://www.japantimes.co.jp/news/2020/03/20/national/coronavirus-explosion-expected-japan/}{Japonsko
  očekávalo explozi koronaviru. Kde je?} Přestože je Japonsko jednou z
  prvních zemí, které získaly pozitivní výsledky testů a nezavedlo žádná
  omezení pohzbu osob, je jednou z nejméně postižených zemí. Citace: „I
  když Japonsko možná podhodnocuje počty nakažených, nemocnice nejsou
  přetížené a v případech zápalu plic nedošlo k žádné eskalaci.``
\item
  Italští vědci tvrdí, že extrémní smog v severní Itálii, nejhorší v
  Evropě,
  \href{https://www.heise.de/tp/features/Feinstaubpartikel-als-Viren-Vehikel-4687454.html}{může
  hrát příčinnou roli} při současném výskytu ohniska pneumonie, stejně
  jako tomu bylo ve Wu-chanu.
\item
  V \href{https://www.youtube.com/watch?v=JBB9bA-gXL4}{novém rozhovoru}
  profesor Sucharit Bhakdi, světově uznávaný odborník v lékařské
  mikrobiologii, říká, že obviňovat samotný nový koronavirus z úmrtí je
  „nesprávné`` a „nebezpečně zavádějící``, protože ve hře jsou další,
  důležitější faktory, zejména již existující zdravotní stav a špatná
  kvalita ovzduší v čínských a severoitalských městech. Profesor Bhakdi
  popisuje aktuálně diskutovaná nebo zavedená opatření jako
  „groteskní``, „zbytečná``, „sebezničující`` a jako „kolektivní
  sebevraždu``, která zkrátí život seniorů a společnost by je neměla
  akceptovat.
\end{itemize}

\hypertarget{20-bux159ezna-2020}{%
\paragraph{20. března 2020}\label{20-bux159ezna-2020}}

\begin{itemize}
\tightlist
\item
  Podle \href{https://www.euromomo.eu/index.html}{nejnovější evropské
  monitorovací zprávy} zůstává celková úmrtnost ve všech zemích (včetně
  Itálie) a ve všech věkových skupinách dosud v rámci nebo dokonce pod
  běžným rozsahem.
\item
  \href{https://de.wikipedia.org/wiki/COVID-19-Pandemie_in_Deutschland\#Todesf\%C3\%A4lle_in_den_Medien}{Nejnovější
  německé statistiky} o lidech umírajících s pozitivním testem ukazují,
  že jejich střední věk~ je asi 83 let a většina z nich má již
  existující onemocnění, které mohli být příčinou smrti.
\item
  \href{https://www.ncbi.nlm.nih.gov/pmc/articles/PMC2095096/}{Kanadská
  studie z roku 2006}, na kterou se odkazuje profesor John Ioannidis ze
  Stanfordu, zjistila, že koronaviry suvisející s běžným nachlazením
  mohou také způsobovat úmrtnost až 6\% v rizikových skupinách, jako
  jsou obyvatelé pečovatelských zařízení, a že soupravy na testování
  virů je zpočátku falešně určovaly jako infekci koronaviry SARS.
\end{itemize}

\hypertarget{19-bux159ezna-2020-i}{%
\paragraph{19. března 2020 (I)}\label{19-bux159ezna-2020-i}}

Italský národní zdravotní ústav ISS zveřejnil
\href{https://www.epicentro.iss.it/coronavirus/bollettino/Report-COVID-2019_17_marzo-v2.pdf}{novou
zprávu} o testech pozitivních úmrtích:

\begin{itemize}
\tightlist
\item
  Střední věk je 80,5 let (79,5 pro muže, 83,7 pro ženy).
\item
  10\% zemřelých bylo starších 90-ti let; 90\% zemřelých bylo starších
  70-ti let.
\item
  Nanejvýš 0,8\% zemřelých nemělo žádné již existující chronické
  onemocnění.
\item
  Přibližně 75\% zemřelých mělo dvě nebo více již existujících
  onemocnění, 50\% mělo tři a více existujících onemocnění, zejména
  srdeční onemocnění, cukrovku a rakovinu.
\item
  Pět zemřelých bylo ve věku 31 až 39 let, všichni s vážnými již
  existujícími onemocněními (např. rakovina nebo srdeční choroby).
\item
  Národní zdravotní ústav dosud nestanovil, na co vyšetřovaní pacienti
  nakonec zemřeli, a obecně se na ně odkazuje jako na úmrtí lidí s
  pozitivním testem na Covid19.
\end{itemize}

\hypertarget{19-bux159ezna-2020-ii}{%
\paragraph{19. března 2020 (II)}\label{19-bux159ezna-2020-ii}}

\begin{itemize}
\tightlist
\item
  \href{https://milano.corriere.it/notizie/cronaca/18_gennaio_10/milano-terapie-intensive-collasso-l-influenza-gia-48-malati-gravi-molte-operazioni-rinviate-c9dc43a6-f5d1-11e7-9b06-fe054c3be5b2.shtml}{Zpráva}
  v italských novinách Corriere della Sera zdůrazňuje, že jednotky
  intenzivní péče v Itálii se již v roce 2017/2018 zhroutily pod silnou
  chřipkovou vlnou. Museli odložit operace, zavolat sestry zpět z
  dovolené a došly jim dárci krve.
\item
  Německý virolog Hendrik Streeck
  \href{https://www.faz.net/aktuell/gesellschaft/gesundheit/coronavirus/virologe-hendrik-streeck-ueber-corona-neue-symptome-entdeckt-16681450.html?printPagedArticle=true\#pageIndex_2}{tvrdí},
  že Covid-19 pravděpodobně nezvýší celkovou úmrtnost v Německu, která
  je obvykle kolem 2500 lidí denně. Streeck zmiňuje případ 78letého muže
  s předchozími onemocněními, který zemřel na srdeční selhání, následně
  byl testován na Covid-19 pozitivně, a byl tedy zahrnut do statistiky
  úmrtí Covid19.
\item
  Podle Stanfordského profesora Johna Ioannidise
  \href{https://www.statnews.com/2020/03/17/a-fiasco-in-the-making-as-the-coronavirus-pandemic-takes-hold-we-are-making-decisions-without-reliable-data/}{nemusí
  být} nový koronavirus nebezpečnější než některé běžné koronaviry, a to
  i u starších lidí. Ioannidis tvrdí, že neexistují spolehlivé lékařské
  statistiky ospraveldňující opatření, o nichž se v současnosti
  rozhoduje.\\
\end{itemize}

\hypertarget{18-bux159ezna-2020}{%
\paragraph{18. března 2020}\label{18-bux159ezna-2020}}

\begin{itemize}
\tightlist
\item
  ~\href{https://www.medrxiv.org/content/10.1101/2020.02.12.20022434v2}{Nová
  epidemiologická studie} (předběžný tisk) dospěla k závěru, že úmrtnost
  v souvislosti s Covid-19 i v čínském městě Wu-chan byla pouze 0,04\%
  až 0,12\%, a tedy spíše nižší než u sezónní chřipky, která má úmrtnost
  asi 0,1\%. Jako důvod k nadhodnocené úmrtnosti Covid-19 vědci
  předpokládají, že zpočátku bylo ve Wu-chanu zaznamenáno pouze malé
  množství případů, protože u mnoha lidí byla nemoc pravděpodobně
  asymptomatická nebo mírná.
\item
  Čínští vědci tvrdí, že
  \href{https://www.eurasiareview.com/01022020-polluted-air-could-be-an-important-cause-of-wuhan-pneumonia-oped/}{extrémní
  zimní smog} ve městě Wu-chan mohl být jednou z příčin vypuknutí
  pneumonie. Už v létě roku 2019 se ve Wu-chanu konaly
  \href{https://www.cnn.com/2019/07/10/asia/china-wuhan-pollution-problems-intl-hnk/index.html}{veřejné
  protesty} kvůli špatné kvalitě ovzduší.
\item
  Nové satelitní snímky ukazují, jak severní Itálie má
  \href{https://twitter.com/esa/status/1238480433047916545}{nejvyšší
  úrovně znečištění ovzduší} v Evropě a jak toto znečištění ovzduší bylo
  karanténou výrazně sníženo.
\item
  Výrobce testovací soupravy Covid-19 uvádí, že by se měla
  \href{https://www.creative-diagnostics.com/sars-cov-2-coronavirus-multiplex-rt-qpcr-kit-277854-457.htm}{používat
  pouze pro výzkumné účely}, a nikoli pro diagnostické aplikace, protože
  dosud nebyla klinicky ověřena.
\end{itemize}

\includegraphics{https://swprs.files.wordpress.com/2020/03/covid-testkit.png?w=550\&h=149}

\hypertarget{17-bux159ezna-2020-i}{%
\paragraph{17. března 2020 (I)}\label{17-bux159ezna-2020-i}}

\begin{itemize}
\tightlist
\item
  Profil úmrtnosti zůstává z virologického hlediska záhadný, protože na
  rozdíl od virů chřipky jsou děti ušetřeny a muži jsou postiženi
  dvakrát častěji než ženy. Na druhé straně tento profil odpovídá
  \href{http://www.gbe-bund.de/gbe10/abrechnung.prc_abr_test_logon?p_uid=gast\&p_aid=0\&p_knoten=FID\&p_sprache=D\&p_suchstring=820}{přirozené
  úmrtnosti}, která se blíží nule u dětí a téměř dvakrát vyšší u
  75letých mužů než u žen stejného věku.
\item
  Mladší zesnulí s~pozitivním testem měli téměř vždy vážné již
  existující nemoci. Například španělský fotbalový trenér ve věku 21 let
  zemřel s~pozitivním testem, což bylo zdůrazněno v novinových titulcích
  po celém světě. Lékaři však
  \href{https://sports.yahoo.com/spanish-football-coach-francisco-garcia-163153573.html}{diagnostikovali}
  dříve nerozpoznanou leukémii, jejíž typické komplikace zahrnují těžký
  zápal plic.
\item
  Rozhodujícím faktorem při hodnocení nebezpečí nemoci proto není počet
  testovaných pozitivních osob a zemřelých, který je často uváděn v
  médiích, ale aktuální počet osob u kterých se nečekaně vyvíjí zápal
  plic nebo na něj umírají (tzv. nadměrná úmrtnost). Ve většině zemí je
  tato hodnota zatím velmi nízká.
\item
  Ve Švýcarsku jsou některé pohotovostní jednotky přetíženy jednoduše
  kvůli velkému počtu lidí,
  \href{https://insideparadeplatz.ch/2020/03/16/notfall-stationen-bereits-seit-tagen-am-anschlag/}{kteří
  chtějí být testováni}. To ukazuje, že současnou situaci vyhrocují
  psychologické a logistické faktory.\\
\end{itemize}

\hypertarget{17-bux159ezna-2020-ii}{%
\paragraph{17. března 2020 (II)}\label{17-bux159ezna-2020-ii}}

\begin{itemize}
\tightlist
\item
  Italský profesor imunologie Sergio Romagnani z Florentské univerzity
  dospěl k závěru ve studii na 3 000 lidech, že 50 až 75\% lidí
  s~pozitivním testem všech věkových skupin zůstává
  \href{https://www.repubblica.it/salute/medicina-e-ricerca/2020/03/16/news/coronavirus_studio_il_50-75_dei_casi_a_vo_sono_asintomatici_e_molto_contagiosi-251474302/}{zcela
  bez příznaků} -- výrazně více, než se dříve předpokládalo.
\item
  Míra obsazenosti severoitalských JIP (jednotek intenzivní péče) v
  zimních měsících je obvykle
  \href{https://jamanetwork.com/journals/jama/fullarticle/2763188}{85 až
  90\%}. Někteří nebo mnozí z těchto stávajících pacientů by nyní mohli
  být také pozitivní na testy. Počet dalších neočekávaných případů
  zápalu plic však dosud není znám.
\item
  Lékař z~nemocnice ve španělské Malaze
  \href{https://twitter.com/NeurologaenSAS/status/1239498772570308609}{píše
  na Twitteru}, že lidé v současné době častěji umírají na panický a
  systémový kolaps než na virus. Nemocnici zahlcují lidé s nachlazením,
  chřipkou a možná Covid-19 a lékaři tak ztratili kontrolu.
\end{itemize}

\hypertarget{14-bux159ezna-2020-ii}{%
\paragraph{14. března 2020 (II)}\label{14-bux159ezna-2020-ii}}

Podle
\href{https://www.epicentro.iss.it/coronavirus/sars-cov-2-decessi-italia}{posledních
dat} ~Italského národního zdravotnického institutu ISS je průměrný věk
zesnulých, kteří byli pozitivně testováni, kolem 81 let. Věk 10\%
zesnulých je přes 90 let. 90\% zesnulých je starších 70 let.

80\% zesnulých trpělo jednou nebo více chronickými chorobami. 50\%
zesnulých trpělo třemi a více chronickými chorobami. Tyto chronické
choroby zahrnují zejména kardiovaskulární problémy, diabetes, problémy
dýchací soustavy a novotvary.

Méně než 1\% zesnulých byly zdravé osoby, tzn. osoby bez předchozích
~chronických onemocnění. Pouze 30\% zesnulých tvoří ženy.

Italský zdravotnický institut
navíc~\href{https://youtu.be/0M4kbPDHGR0?t=210}{rozlišuje}~mezi těmi,
kteří zemřeli na koronavirus a těmi, kteří zemřeli s koronavirem. V
mnoha případech není dosud jasné, zda osoby zemřely kvůli viru nebo
kvůli jejich chronickým onemocněním, nebo kvůli kombinaci obou.

Dva Italové mladší 40 let (oba 39 let), byli pacienti s rakovinou a
diabetem s dodatečnými komplikacemi. Také v těchto případech není
příčina úmrtí dosud jasná (tj. zda se jednalo o úmrtí kvůli viru nebo
kvůli již existujícím chorobám).

Částečné přetížení nemocnic je zapříčiněno obecným přísunem pacientů a
zvýšeným počtem pacientů vyžadujících speciální nebo intenzivní péči.
Cílem je zvláště stabilizace dýchacích funkcí a v závažných případech
poskytnutí antivirové terapie.

(Aktuálně: Italský národní zdravotnický institut publikoval
\href{https://www.epicentro.iss.it/coronavirus/bollettino/Report-COVID-2019_17_marzo-v2.pdf}{statistickou
rešerši}~o pozitivně testovaných pacientech a zesnulých, který výše
uvedená data potvrzuje.)

\textbf{Rovněž je třeba vzít v~úvahu následující aspekty:}

Severní Itálie má jednu z nejstarších populací a
\href{https://twitter.com/esa/status/1238480433047916545}{nejhorší
kvalitu ovzduší}~v Evropě, což již v minulosti vedlo ke zvýšenému počtu
respiračních onemocnění a úmrtí a je jedním z dodatečných rizikových
faktorů v současné epidemii.

Kupříkladu Jižní Korea zaznamenala mnohem hladší průběh než Itálie a již
překročila vrchol epidemické křivky. Dosud bylo z Jižní Koreje hlášeno
pouze cca 70 úmrtí s pozitivním testem na koronavir. Stejně jako v
Itálii, postiženými byli především vysoce rizikoví pacienti.

Několik desítek dosud hlášených pozitivně testovaných úmrtí ve Švýcarsku
bylo rovněž ze skupiny vysoce rizikových pacientů s chronickými
onemocněními, průměrným věkem přes 80 let a nejvyšším věkem 97 let, kde
přesná příčina smrti (kvůli koronaviru nebo kvůli existujícím nemocem)
není dosud známa.

Dále studie ukázaly, že mezinárodně užívané testovací sady mohou v
některých případech dávat falešně pozitivní výsledky. V těchto případech
se testované osoby mohly nakazit nikoli novým koronavirem, ale
pravděpodobně jedním z mnoha existujících, lidmi přenášených koronavirů,
které jsou součástí právě probíhající každoroční sezony chřipek a
nachlazení. (1)

Proto nejdůležitějším indikátorem pro ohodnocení nebezpečnosti tohoto
onemocnění nejsou neustále hlášená čísla o pozitivně testovaných osobách
a úmrtích, ale počty osob, u nichž se skutečně a nečekaně vyvinul zápal
plic (tzv. excesivní mortalita -- navýšená úmrtnost).

Podle všech aktuálních dat, pro zdravou celkovou populaci ve školním a
produktivním věku lze předpokládat lehký až mírný průběh choroby.
Senioři a občané s existujícími chronickými chorobami by měli být
chráněni. Potřebné zdravotnické kapacity by měly být optimálně
připraveny.

\textbf{Zdravotnická literatura:}

(1) Patrick et
al.,~\href{https://www.ncbi.nlm.nih.gov/pmc/articles/PMC2095096/}{An
Outbreak of Human Coronavirus OC43 Infection and Serological
Cross-reactivity with SARS Coronavirus}, CJIDMM, 2006.

(2) Grasselli et
al.,~\href{https://jamanetwork.com/journals/jama/fullarticle/2763188}{Critical
Care Utilization for the COVID-19 Outbreak in Lombardy}, JAMA, March
2020.

(3)
WHO,~\href{https://www.who.int/docs/default-source/coronaviruse/who-china-joint-mission-on-covid-19-final-report.pdf}{Report
of the WHO-China Joint Mission on Coronavirus Disease 2019}, February
2020.

\textbf{Referenční hodnoty}

Důležité referenční hodnoty zahrnují počty ročních úmrtí na chřipku,
kterých je v Itálii do 8000 a v USA do 60,000; obvyklá celková úmrtnost,
která je v Itálii do 2000 denně; a průměrný počet případů zápalu plic,
který je v Itálii 120,000 ročně.

Současný souhrn úmrtnosti v Evropě a v Itálii je stále normální nebo
dokonce podprůměrný. Jakákoli navýšená úmrtnost kvůli Covid-19 by se
měla v \href{https://www.euromomo.eu/index.html}{Evropských
monitorovacích statistikách} projevit.

\includegraphics{https://swprs.files.wordpress.com/2020/03/italy-smog.png?w=550\&h=309}

\begin{center}\rule{0.5\linewidth}{\linethickness}\end{center}

Share this on:
\href{https://twitter.com/intent/tweet?url=https://swprs.org/fakta-o-covid-19/}{Twitter}
/
\href{https://www.facebook.com/share.php?u=https://swprs.org/fakta-o-covid-19/}{Facebook}\\

\hypertarget{swiss-policy-research}{%
\subsubsection{Swiss Policy Research}\label{swiss-policy-research}}

\begin{itemize}
\tightlist
\item
  \href{https://swprs.org/kontakt/}{Kontakt}
\item
  \href{https://swprs.org/uebersicht/}{Übersicht}
\item
  \href{https://swprs.org/donationen/}{Donationen}
\item
  \href{https://swprs.org/disclaimer/}{Disclaimer}
\end{itemize}

\hypertarget{english}{%
\subsubsection{English}\label{english}}

\begin{itemize}
\tightlist
\item
  \href{https://swprs.org/contact/}{About Us / Contact}
\item
  \href{https://swprs.org/media-navigator/}{The Media Navigator}
\item
  \href{https://swprs.org/the-american-empire-and-its-media/}{The CFR
  and the Media}
\item
  \href{https://swprs.org/donations/}{Donations}
\end{itemize}

\hypertarget{follow-by-email}{%
\subsubsection{Follow by email}\label{follow-by-email}}

Follow

\href{https://wordpress.com/?ref=footer_custom_com}{WordPress.com}.

\protect\hyperlink{}{Up ↑}

Post to

\protect\hyperlink{}{Cancel}

\includegraphics{https://pixel.wp.com/b.gif?v=noscript}
