\protect\hyperlink{content}{Skip to content}

\href{https://swprs.org/}{}

\protect\hyperlink{search-container}{Search}

Search for:

\href{https://swprs.org/}{\includegraphics{https://swprs.files.wordpress.com/2020/05/swiss-policy-research-logo-300.png}}

\href{https://swprs.org/}{Swiss Policy Research}

Geopolitics and Media

Menu

\begin{itemize}
\tightlist
\item
  \href{https://swprs.org}{Start}
\item
  \href{https://swprs.org/srf-propaganda-analyse/}{Studien}

  \begin{itemize}
  \tightlist
  \item
    \href{https://swprs.org/srf-propaganda-analyse/}{SRF / ZDF}
  \item
    \href{https://swprs.org/die-nzz-studie/}{NZZ-Studie}
  \item
    \href{https://swprs.org/der-propaganda-multiplikator/}{Agenturen}
  \item
    \href{https://swprs.org/die-propaganda-matrix/}{Medienmatrix}
  \end{itemize}
\item
  \href{https://swprs.org/medien-navigator/}{Analysen}

  \begin{itemize}
  \tightlist
  \item
    \href{https://swprs.org/medien-navigator/}{Navigator}
  \item
    \href{https://swprs.org/der-propaganda-schluessel/}{Techniken}
  \item
    \href{https://swprs.org/propaganda-in-der-wikipedia/}{Wikipedia}
  \item
    \href{https://swprs.org/logik-imperialer-kriege/}{Kriege}
  \end{itemize}
\item
  \href{https://swprs.org/netzwerk-medien-schweiz/}{Netzwerke}

  \begin{itemize}
  \tightlist
  \item
    \href{https://swprs.org/netzwerk-medien-schweiz/}{Schweiz}
  \item
    \href{https://swprs.org/netzwerk-medien-deutschland/}{Deutschland}
  \item
    \href{https://swprs.org/medien-in-oesterreich/}{Österreich}
  \item
    \href{https://swprs.org/das-american-empire-und-seine-medien/}{USA}
  \end{itemize}
\item
  \href{https://swprs.org/bericht-eines-journalisten/}{Fokus I}

  \begin{itemize}
  \tightlist
  \item
    \href{https://swprs.org/bericht-eines-journalisten/}{Journalistenbericht}
  \item
    \href{https://swprs.org/russische-propaganda/}{Russische Propaganda}
  \item
    \href{https://swprs.org/die-israel-lobby-fakten-und-mythen/}{Die
    »Israel-Lobby«}
  \item
    \href{https://swprs.org/geopolitik-und-paedokriminalitaet/}{Pädokriminalität}
  \end{itemize}
\item
  \href{https://swprs.org/migration-und-medien/}{Fokus II}

  \begin{itemize}
  \tightlist
  \item
    \href{https://swprs.org/covid-19-hinweis-ii/}{Coronavirus}
  \item
    \href{https://swprs.org/die-integrity-initiative/}{Integrity
    Initiative}
  \item
    \href{https://swprs.org/migration-und-medien/}{Migration \& Medien}
  \item
    \href{https://swprs.org/der-fall-magnitsky/}{Magnitsky Act}
  \end{itemize}
\item
  \href{https://swprs.org/kontakt/}{Projekt}

  \begin{itemize}
  \tightlist
  \item
    \href{https://swprs.org/kontakt/}{Kontakt}
  \item
    \href{https://swprs.org/uebersicht/}{Seitenübersicht}
  \item
    \href{https://swprs.org/medienspiegel/}{Medienspiegel}
  \item
    \href{https://swprs.org/donationen/}{Donationen}
  \end{itemize}
\item
  \href{https://swprs.org/contact/}{English}
\end{itemize}

\protect\hyperlink{}{Open Search}

\hypertarget{propaganda-in-der-wikipedia}{%
\section{Propaganda in
der~Wikipedia}\label{propaganda-in-der-wikipedia}}

\textbf{Publiziert}: Oktober 2018; \textbf{Aktualisiert}: Mai 2020;\\
\textbf{Sprachen}:
\href{https://swprs.files.wordpress.com/2020/03/wikipedia-disinformation-operation-arabic.pdf}{AR},
\href{https://swprs.org/propaganda-in-der-wikipedia/}{DE},
\href{https://swprs.org/wikipedia-disinformation-operation/}{EN},
\href{https://swprs.files.wordpress.com/2020/03/wikipedia-disinformation-operation-spanish.pdf}{ES},
\href{https://lumieresurgaia.com/wikipedia-une-operation-de-desinformation/}{FR},
\href{https://swprs.files.wordpress.com/2020/03/wikipedia-disinformation-operation-dutch.pdf}{NL},
\href{https://swprs.files.wordpress.com/2020/03/wikipedia-disinformation-operation-polish.pdf}{PL},
\href{https://evz.ro/wikipedia-o-operatie-de-dezinformare-cine-se-ascunde-in-spatele-acestei-enciclopedii-online.html/2}{RO},
\href{https://kiwibyrd.org/2020/03/20/20h33/}{RU}\\
\textbf{Teilen auf}:
\href{https://twitter.com/intent/tweet?url=https://swprs.org/propaganda-in-der-wikipedia/}{Twitter}
/
\href{https://www.facebook.com/share.php?u=https://swprs.org/propaganda-in-der-wikipedia/}{Facebook}

Die Online-Enzyklopädie Wikipedia ist ein integraler Bestandteil des
transatlantischen Medien- und Informationssystems. In der folgenden
Analyse werden zentrale Aspekte ihrer Organisations­struktur,
Funktionsweise und Manipulation dargestellt. Sodann werden die Rollen
der \emph{Wikimedia Foundation} und der traditionellen Medien diskutiert
sowie verschiedene Analysetools vorgestellt.

\href{https://swprs.files.wordpress.com/2019/09/wikipedia-propaganda-analyse-2019.pdf}{Analyse
als PDF herunterladen}

\href{https://swprs.files.wordpress.com/2019/03/wikipedia_2019.png}{\includegraphics{https://swprs.files.wordpress.com/2019/03/wikipedia_2019.png?w=736\&h=1269}Vergrößern
🔎}

\hypertarget{bedeutung-der-wikipedia}{%
\paragraph{Bedeutung der Wikipedia}\label{bedeutung-der-wikipedia}}

Die deutschsprachige Wikipedia verfügt derzeit über rund 2.2 Millionen
Artikel, die pro Monat von rund 100 Millionen Geräten knapp 1 Milliarde
mal aufgerufen
\href{https://stats.wikimedia.org/v2/\#/de.wikipedia.org}{werden}. Im
deutschsprachigen Raum
\href{https://de.wikipedia.org/wiki/Wikipedia:Statistik}{zählt} die
Wikipedia damit zu den sieben meistbesuchten Websites und ist bei vielen
Internetrecherchen eines der ersten Suchergebnisse und mithin eine der
ersten Anlaufstellen.

Im Bereich der Online-Lexika kommt der Wikipedia damit eine zentrale,
monopolartige Stellung zu.

\hypertarget{organisation-der-wikipedia}{%
\paragraph{Organisation der
Wikipedia}\label{organisation-der-wikipedia}}

Die Wikipedia gilt gemeinhin als ein freies und offenes Lexikon, an dem
jeder mitwirken kann. Der Großteil der deutschsprachigen Inhalte wird
indes von nur rund 800 Autoren mit über 100 Edits pro Monat
\href{https://stats.wikimedia.org/DE/TablesWikipediansEditsGt100.htm}{verfasst};
durchschnittlich erreichen nur wenige Tausend Autoren mehr als fünf
Edits pro Monat.

Zudem existiert innerhalb der Wikipedia eine strenge
\href{https://commons.wikimedia.org/wiki/File:Vereinfachtes_Benutzergruppenschema_dewiki.svg}{Hierarchie},
an deren Spitze circa 30 Funktionäre und 150 Administratoren stehen.
Diese entscheiden Konflikte, können Artikel löschen sowie Benutzer
sperren. Administratoren werden von den ca. 3000
\href{https://de.wikipedia.org/wiki/Wikipedia:Stimmberechtigung}{stimmberechtigten}
Wikipedianern ohne Befristung gewählt, wobei die Stimmbeteiligung meist
bei weniger als 10\% liegt (ca. 300 Stimmen); die
Wikipedia-Administration ernennt und bestätigt sich mithin größtenteils
selbst. Trotz ihrer Machtfülle agieren ca. 90\% der Administratoren
pseudonym, in der Öffentlichkeit ist meist nur wenig über sie bekannt.

Unterhalb der Administratoren befinden sich die sogenannten
\href{https://de.wikipedia.org/wiki/Wikipedia:Gesichtete_Versionen}{Sichter},
die Bearbeitungen von Neu-Autoren überprüfen, freigeben oder
revertieren. Dies ist ein wichtiger Kontrollmechanismus, der in der
Praxis aber auch zu Konflikten führen kann. Neben den angemeldeten
Autoren (mit Benutzernamen) gibt es zudem noch unangemeldete Autoren,
durch die circa 20\% aller Artikelbearbeitungen erfolgen.

\hypertarget{manipulation-der-wikipedia}{%
\paragraph{Manipulation der
Wikipedia}\label{manipulation-der-wikipedia}}

Das Problem der Manipulation der Wikipedia besteht seit deren Gründung.
Der Fokus liegt dabei meist auf dem sogenannten »Vandalismus«
(Verunstaltung von Artikeln) sowie auf kommerziell motivierter
Manipulation durch Konzerne, PR-Agenturen und bezahlte Autoren.

Bekannt ist jedoch auch die politische und geopolitische Manipulation
durch Aktivisten, Regierungen und Geheimdienste. So wurde bereits 2007
durch den sogenannten WikiScanner
\href{https://www.reuters.com/article/us-security-wikipedia/cia-fbi-computers-used-for-wikipedia-edits-idUSN1642896020070816}{nachgewiesen},
dass Mitarbeiter der US-Geheimdienste CIA und FBI Wikipedia-Einträge
beispielsweise zum Irak-Krieg und dem US-Militärgefängnis in Guantanamo
editierten. 2016 wurde
\href{https://www.heise.de/newsticker/meldung/Beamte-der-Schweizer-Bundesverwaltung-manipulieren-Wikipedia-Texte-3098396.html}{publik},
dass Schweizer Bundesbeamte kritische Abschnitte aus dem Artikel zum
eigenen Nachrichtendienst entfernten.

Inzwischen ist indes offenkundig, dass die Manipulation der Wikipedia
insbesondere bei geopolitischen und gesellschaftspolitischen Themen
nicht mehr nur vereinzelt und »von außen« geschieht, sondern
systematisch und »von innen«: Einflussreiche Interessensgruppen haben
ihre Akteure in der
\href{https://commons.wikimedia.org/wiki/File:Vereinfachtes_Benutzergruppenschema_dewiki.svg}{Hierarchie}
der Wikipedia als Sichter und Administratoren platziert und können
dadurch relevante Artikel gezielt bearbeiten, unerwünschte Bearbeitungen
entfernen und unerwünschte Autoren sperren.

Zahlreiche Wikipedianer
\href{https://www.golem.de/news/mobbing-auf-wikipedia-content-vandalismus-drohungen-und-beschimpfung-1606-121626.html}{beklagen}
denn auch ein aggressives und frustrierendes Klima innerhalb der
Wikipedia; die Anzahl der aktiven Autoren ist seit Jahren
\href{http://www.spiegel.de/netzwelt/web/wikipedia-wird-15-nicht-nur-ein-grund-zum-feiern-a-1072059.html}{rückläufig},
wodurch politisch oder ideo­lo­gisch motivierte Akteure ihren Einfluss
schrittweise ausbauen konnten.

\hypertarget{manipulation-durch-netzwerke}{%
\paragraph{Manipulation durch
Netzwerke}\label{manipulation-durch-netzwerke}}

Die systematische Manipulation der Wikipedia geschieht durch
netzwerkartige Gruppierungen. Diese bestehen aus einigen höchst aktiven
Benutzern (oder Benutzergruppen), die zumeist anonym bzw. pseudonym
auftreten. Aufgrund von Recherchen unabhängiger
Investigativ­journalisten konnten die Mitglieder dieser Netzwerke
inzwischen teilweise eruiert werden. Siehe hierzu insbesondere:

\begin{itemize}
\tightlist
\item
  Die beiden Filmdokus
  \href{https://www.youtube.com/watch?v=wHfiCX_YdgA}{»Die dunkle Seite
  der Wikipedia«} (2015) und
  \href{https://www.youtube.com/watch?v=tef7bgwInjY}{»Zensur«} (2017)
\item
  Die mehrteilige Investigativ-Serie
  \href{https://www.youtube.com/channel/UCQWqzh6Wcc_2mkBJ5sy3SqA/}{»Geschichten
  aus Wikihausen«} (seit 2018)
\item
  Die aktualisierte
  \href{https://swprs.org/wikipedia-manipulation-autoren/}{Übersichtsgrafik}
  zu den manipulativ agierenden Autoren (2019)
\end{itemize}

Mithilfe der weiter unten vorgestellten Analyse-Werkzeuge lassen sich
die umfangreichen Aktivitäten dieser Netzwerke rekonstruieren. Während
bei geopolitischen Themen die transatlantische Sichtweise dominiert (zu
Themen mit Israel-Bezug siehe zudem
\href{http://www.unz.com/article/how-israel-and-its-partisans-work-to-censor-the-internet/}{hier}),
besteht bei gesell­schafts­politischen Themen eine bemerkenswerte
Übereinstimmung mit Positionen regierungsnaher Berliner
\href{http://www.belltower.news/artikel/task-force-gegen-hassinhalte-im-internet-es-gibt-noch-viel-zu-tun-10780}{Stiftungen}.

Politiker, Publizisten und Forscher mit abweichenden Standpunkten werden
in der Wikipedia bisweilen geradezu
\href{https://kenfm.de/tagesdosis-19-6-2018-die-schauprozesse-der-wikipedia-junta/}{diffamiert};
aufgrund der dargestellten Machtstruktur und der Anonymität können
Betroffene im Allgemeinen weder auf die Diffamierungen reagieren noch
deren Urheber belangen.

2019 \href{https://swprs.org/der-wikipedia-prozess/}{entschied} das
Landgericht Hamburg im Präzedenzfall »Feliks« hingegen erstmals, dass
die Identität eines denunziativ agierenden Wikipedia-Autors von
öffentlichem Interesse ist. Der Wikipedia-Autor »Feliks« wurde in der
Folge von mehreren Betroffenen rechtlich belangt.

Die systematische Manipulation der Wikipedia ist ein weltweites
Phänomen. Ähnliche Operationen wurden auch in der englischsprachigen
Wikipedia
\href{https://de.sputniknews.com/panorama/20180531320955722-philip-cross-wikipedia/}{aufgedeckt},
inklusive Hinweisen auf eine nach­rich­ten­dienstliche Koordination, die
auch im deutschsprachigen Raum nicht auszuschließen ist.

So wurde bereits 2007 einer der einflussreichsten Administratoren der
englischen Wikipedia als ehemaliger MI5-Informant
\href{https://web.archive.org/web/20200528235242/http://english.ohmynews.com/articleview/article_view.asp?menu=c10400\&no=374006\&rel_no=1}{enttarnt},
der unter falscher Identität in Kanada lebte und Wikipedia im Sinne der
britischen Regierung bearbeitete. Wikipedia-Gründer Jimmy Wales
\href{https://gosint.wordpress.com/2018/06/02/wikipedia-the-spooks-the-remake-update-philip-cross-identified/}{verteidigte}
ihn dennoch.

\hypertarget{die-rolle-der-wikimedia-foundation}{%
\paragraph{Die Rolle der Wikimedia
Foundation}\label{die-rolle-der-wikimedia-foundation}}

Die amerikanische
\emph{\href{https://de.wikipedia.org/wiki/Wikimedia_Foundation}{Wikimedia
Foundation}} ist die Trägerstiftung der Wikipedia (sowie weiterer
Wiki-Projekte). Sie verfügt inzwischen über jährliche Spenden-Einnahmen
von rund 90 Millionen Dollar und ein Vermögen von rund 120 Millionen
Dollar. Zu den größten Spendern
\href{https://wikimediafoundation.org/support/benefactors/}{zählen}
dabei diverse US-Konzerne sowie einflussreiche Stiftungen, wodurch es
wiederholt zu Interessenskonflikten
\href{https://www.dailydot.com/business/wikipedia-conflict-editing-donation-benefactors/}{kam}.
Wiki­media-Gründer Jimmy Wales war ein
\href{http://reports.weforum.org/the-forum-of-young-global-leaders-2014/ygl-stories/crowdsourced/}{\emph{Young
Global Leader}} des \emph{World Economic Forum (WEF)} Davos und ist ein
privater und geschäftlicher
\href{http://wikipediocracy.com/2015/06/01/how-jimmy-wales-rode-tony-blairs-coattails/}{Partner}
des ehemaligen britischen Premierministers Tony Blair.

Auch die Unterorganisation \emph{Wikimedia Deutschland} ist einschlägig
vernetzt: So war der ehemalige
\href{https://de.wikipedia.org/wiki/Christian_Rickerts}{Geschäftsführer}
der Wikimedia Deutschland zuvor Vizepräsident für
Unter­nehmens­kommunikation bei der Bertelsmann-Stiftung und wechselte
danach als Staats­sekretär in die Berliner Landes­regierung. Der im
Besitz der Bertelsmann-Stiftung befindliche Bertelsmann-Medien­konzern
\href{https://www.bertelsmann.com/corporate-responsibility/facts-and-figures/cooperations/}{ist}
Unternehmenspartner der \emph{Atlantik-Brücke} und
\href{https://meedia.de/2018/02/15/facebook-polizei-bertelsmann-tochter-arvato-fahndet-fuer-us-plattform-auch-weiterhin-nach-verbotenen-inhalten/}{betreibt}
über den Dienstleister \emph{Arvato} die sogenannte »Inhaltsmoderation«
für das deutschsprachige Facebook.

Seitens der \emph{Wikimedia Foundation} ist insofern kaum mit Kritik an
der (geo-)politischen Ausrichtung und Manipulation der Wikipedia zu
rechnen. Liegen jedoch gerichtliche Verfügungen beispielsweise aufgrund
von Verleumdungen vor, muss Wikimedia die fraglichen Passagen entfernen.
Betroffene können hierfür unter Berufung auf das Internationale
Privatrecht eine Klage an ihrem Wohnsitz
\href{https://swprs.org/wikipedia-missbrauch-massnahmen/}{einreichen}.

\hypertarget{die-rolle-der-traditionellen-medien}{%
\paragraph{Die Rolle der traditionellen
Medien}\label{die-rolle-der-traditionellen-medien}}

Traditionelle Medien sind einerseits häufig verwendete Quellen für
Wikipedia-Artikel, andererseits greifen Journalisten für Recherchen
bisweilen selbst auf Wikipedia zurück. Hierdurch können geschlossene
Informationskreisläufe
\href{https://de.wikipedia.org/wiki/Kritik_an_Wikipedia\#Zweifelhafte_Quellenlage}{entstehen},
bei denen sich traditionelle Medien selbst bestätigen und alternative
Sichtweisen ausgeblendet bleiben, zumal etwa leserfinanzierte
Online-Medien von den maßgebenden Wikipedia-Administratoren meist nicht
als
\href{https://de.wikipedia.org/wiki/Wikipedia:Relevanzkriterien}{»relevante
Quellen«} zugelassen werden.

Traditionelle Medien berichteten verschiedentlich über die Manipulation
der Wikipedia durch einzelne
\href{http://www.spiegel.de/wirtschaft/wikipedia-das-geschoente-bild-vom-daimler-konzern-a-817802.html}{Konzerne},
\href{https://www.fr.de/politik/steckt-hinter-afd-freund-lukati-11059673.html}{Parteien}
oder
\href{https://www.vice.com/de/article/znky4j/wikipedia-manipulation-812}{Agenturen},
nicht jedoch über die systematische (geo-)politische Manipulation. Dies
könnte daran liegen, dass traditionelle Medien im deutschsprachigen Raum
ihrerseits in trans­atlantische Elitennetzwerke
\href{https://swprs.org/netzwerk-medien-deutschland/}{eingebunden} sind
und deshalb im Allgemeinen dieselben (geo-)politischen Positionen
vertreten, die auch in der Wikipedia dominieren.

Amerikanische Social-Media-Plattformen wie Youtube und Facebook haben
2018 zudem
\href{https://www.heise.de/newsticker/meldung/Maechtigstes-Werkzeug-zur-Radikalisierung-Youtube-will-mit-Wikipedia-Links-Verschwoerungstheorien-3994112.html}{angekündigt},
bei »umstrittenen Themen« künftig auf entsprechende Wikipedia-Artikel zu
verlinken. Damit gewinnt die Wikipedia im transatlantischen Mediensystem
zusätzlich an Bedeutung.

\hypertarget{analyse-werkzeuge}{%
\paragraph{Analyse-Werkzeuge}\label{analyse-werkzeuge}}

Verschiedene Online-Werkzeuge ermöglichen eine professionelle Analyse
von Wikipedia-Beiträgen und ihren Autoren. Zu den wichtigsten Werkzeugen
zählen hierbei:

\begin{itemize}
\tightlist
\item
  \href{https://f-squared.org/whovisual/}{WikiWho}: Diese am
  \emph{Karlsruher Institut für Technologie} und dem
  \emph{Leibniz-Institut für Sozialwissenschaften} entwickelte
  Browser-Erweiterung ermöglicht es, Artikel-Autoren sowie kürzlich
  hinzugefügte und besonders umstrittene Textstellen farblich
  hervorzuheben.
\item
  \href{https://xtools.wmflabs.org/?uselang=de}{X-Tools}: Die X-Tools
  sind eine von langjährigen Wikipedianern erstellte Sammlung an
  Analyse- und Statistikwerkzeugen, mit denen sich Benutzerbeiträge,
  Seitenhistorien und viele weitere Aspekte untersuchen und grafisch
  darstellen lassen.
\item
  \href{http://wikibu.ch/}{Wikibu}: Wikibu ist ein an der
  \emph{Pädagogischen Hochschule Bern} entwickeltes Werkzeug, das die
  Qualität von Wikipedia-Artikeln anhand formaler Kriterien wie der
  Anzahl der Autoren, Verweise und Quellen bewertet und zusätzlich auf
  mögliche Qualitätsrisiken hinweist.
\end{itemize}

Für den professionellen Umgang mit Wikipedia empfiehlt sich zudem das
Studium der öffentlich einsehbaren
\href{https://de.wikipedia.org/wiki/Wikipedia:Diskussionsseiten}{Diskussionsseiten}
und Versionsverläufe der fraglichen Artikel.

\includegraphics{https://swprs.files.wordpress.com/2018/10/wiki-who.png?w=736}\\
\emph{\href{https://f-squared.org/whovisual/}{WikiWho}: Wer hat an einem
Wikipedia-Artikel mitgewirkt?}

\hypertarget{perspektiven}{%
\paragraph{Perspektiven}\label{perspektiven}}

Wikipedia hat sich, wie dargestellt, zu einem integralen Bestandteil des
transatlantischen Medien- und Informationssystems entwickelt.
Insbesondere bei geopolitischen und einigen gesellschaftspolitischen
Themen kann die Wikipedia im Allgemeinen keine objektive Darstellung
bieten.

Die Erfolgsaussichten für eine interne Reform der Wikipedia erscheinen
bislang gering, da manipulative Gruppierungen ihre Stellung in der
Wikipedia-Hierarchie durch mehrjährige, koordinierte, verdeckte und
mindestens teilweise extern finanzierte Aktivitäten weitgehend
abgesichert haben.

In Bezug auf Wikipedia stehen derzeit somit primär die Aufklärung der
Manipulationsstrukturen, die Schulung der persönlichen Medienkompetenz,
die Nutzung alternativer Onlineressourcen, der Fokus auf Primärquellen,
sowie Maßnahmen im Falle von Verleumdungen im Vordergrund.

*»Was mit der Wikipedia geschieht ist ein Skandal. Ich habe früher
mitgeschrieben und kann die Diktatur der Admins bestätigen. Es geht
nicht um Argumente, sondern um Macht.«\\
*--- Ein ehemaliger
\href{https://www.tichyseinblick.de/feuilleton/medien/enttarnung-eines-wiederholungstaeters-wikipedia-das-kontaminierte-lexikon/\#comment-502004}{Wikipedianer}
*---\\
*

\hypertarget{weiterfuxfchrende-literatur}{%
\paragraph{Weiterführende Literatur}\label{weiterfuxfchrende-literatur}}

Weiterführende Literatur zur Manipulation der Wikipedia (sortiert nach
Publikationsdatum).

2010--2015:

\begin{itemize}
\tightlist
\item
  \href{http://www.bpb.de/gesellschaft/digitales/wikipedia/145809/die-macht-der-wenigen?p=all}{Wikipedia:
  Die Macht der Wenigen} (Bundeszentrale für politische Bildung, 2012)
\item
  \href{https://sciencefiles.org/2012/07/29/feindliche-ideologische-ubernahme-deutsche-wikipedia-droht-im-desaster-zu-enden/}{Wikipedia:
  Feindliche, ideologische Übernahme} (ScienceFiles, 2012)
\item
  \href{http://deutsche-wirtschafts-nachrichten.de/2013/11/04/wikipedia-wir-machen-meinung/}{Wikipedia:
  Wir machen Meinung} (Deutsche Wirtschafts-Nachrichten, 2013)
\item
  \href{https://www.otto-brenner-stiftung.de/wissenschaftsportal/informationsseiten-zu-studien/studien-2014/verdeckte-pr-in-wikipedia/}{Verdeckte
  PR in Wikipedia} (Studie der Otto Brenner Stiftung, 2014)
\item
  \href{https://www.infosperber.ch/Medien/Wikipedia-PR-Manipulation-Unternehmen}{Wie
  Unternehmen Wikipedia manipulieren} (Infosperber, 2014)
\item
  \href{https://rotefahne.eu/2014/01/wikipedia-desinformation-im-auftrag-der-nato-doktrin/}{Wikipedia:
  Desinformation im Auftrag der NATO-Doktrin} (Rote Fahne, 2014)
\item
  \href{https://www.heise.de/tp/features/Verschwoerungstheorie-3363979.html?seite=all}{Wikipedia,
  9/11 und das Problem mit dem Dissens} (Telepolis, 2014)
\item
  \href{http://de.pluspedia.org/wiki/Merck-Wikipedia-Skandal}{Der
  Merck-Wikipedia-Skandal} (Unternehmens-PR, Junge Welt, 2015)
\item
  \href{https://www.heise.de/tp/features/Jimmy-Wales-eine-Ikone-mit-Schoenheitsfehlern-3377061.html?seite=all}{Jimmy
  Wales: Eine Ikone mit Schönheitsfehlern} (Telepolis, 2015)
\item
  \href{https://blog.wikimedia.de/2015/03/12/monsters-of-law-nr-5-wiki-immunity-bleibt-die-wikipedia-in-deutschland-rechtlich-geschuetzt/}{Bleibt
  Wikipedia in Deutschland rechtlich geschützt?} (Wikimedia, 2015)
\item
  \href{https://www.vice.com/de/article/znky4j/wikipedia-manipulation-812}{``Ich
  habe für eine Agentur Wikipedia-Artikel manipuliert''} (Vice, 2015)
\item
  \href{http://www.free21.org/wp-content/uploads/2015/11/03-McClean-Gesinnungsw\%C3\%A4chter-Wikipedia3.pdf}{Die
  Gesinnungswächter der Wikipedia} (Free21, 2015)
\end{itemize}

2016--2018:

\begin{itemize}
\tightlist
\item
  \href{https://www.nachdenkseiten.de/?p=37340}{Wie die Wikipedia sich
  selbst zerstört} (Nachdenkseiten, 2017)
\item
  \href{https://www.aargauerzeitung.ch/schweiz/wikipedia-umschreiben-loeschen-manipulieren-die-schoenfaerberei-der-bundesbeamten-130051388}{Wikipedia:
  Die Schönfärberei der Bundesbeamten} (Aargauer Zeitung, 2017)
\item
  \href{https://medium.com/@Klarsager/wikipedia-mafi\%C3\%B6se-strukturen-manipulationen-und-millionenverm\%C3\%B6gen-59f7e2086915}{Wikipedia:
  Mafiöse Strukturen, Manipulationen und Millionen} (Medium, 2017)
\item
  EN:
  \href{https://www.purdue.edu/newsroom/releases/2017/Q4/results-of-wikipedia-study-may-surprise.html}{Wikipedia:
  Top 1 percent create 80 percent of content} (Purdue University, 2017)
\item
  \href{https://www.youtube.com/watch?v=lRrZrJZYXJc}{Wikipedia und
  transatlantische Thinktanks} (Interview, Gruppe 42, 2018)
\item
  \href{https://www.heise.de/tp/features/Wikipedia-auf-dem-Weg-zum-Orwellschen-Wahrheitsministerium-4059211.html?seite=all}{Wikipedia:
  Auf dem Weg zum Orwellschen Wahrheitsministerium} (Telepolis, 2018)
\item
  \href{https://www.heise.de/newsticker/meldung/Urteil-gegen-Wikipedia-Keine-rufschaedigende-Kritik-ohne-Recherche-4209610.html}{Urteil
  gegen Wikipedia: Keine rufschädigende Kritik ohne Recherche} (Heise,
  2018)
\item
  \href{https://www.tichyseinblick.de/feuilleton/medien/enttarnung-eines-wiederholungstaeters-wikipedia-das-kontaminierte-lexikon/}{Wikipedia:
  Das kontaminierte Lexikon} (Tichys Einblick, 2018)
\item
  \href{https://www.golem.de/news/wikipedia-autoren-verifiziert-und-manipuliert-1812-137610.html}{Unternehmens-PR:
  Verifiziert und manipuliert} (Golem, 2018)
\item
  EN:
  \href{https://web.archive.org/web/20181201154510/https://medium.com/@helen.buyniski/wikipedia-rotten-to-the-core-dcc435781c45}{Wikipedia:
  Rotten to the Core} (Helen Buyniski, Medium, 2018)
\item
  EN:
  \href{https://www.mintpressnews.com/phillip-cross-the-mystery-wikipedia-editor-targeting-anti-war-sites/250824/}{Phillip
  Cross: The Mystery Wikipedia Editor Targeting Anti-War Sites} (ML,
  2018)
\end{itemize}

2019--2020:

\begin{itemize}
\tightlist
\item
  \href{https://web.archive.org/web/20190705140019/https://www.journalist-magazin.de/hintergrund/zur-loeschung-vorgeschlagen}{Wikipedia:
  Zur Löschung vorgeschlagen} (Medienmagazin Journalist, 2019)
\item
  \href{https://swprs.org/der-wikipedia-prozess/}{Landgericht Hamburg:
  Urteil im Wikipedia-Prozess} (SPR, 2019)
\item
  \href{https://www.sueddeutsche.de/digital/wikipedia-werbung-manipulation-schleichwerbung-wissen-1.4496890}{Gekaufte
  Wahrheiten auf Wikipedia} (Süddeutsche, 2019)
\item
  \href{https://www.tagesanzeiger.ch/news/standard/faelschen-fuer-den-meisterfaelscher/story/10949518}{Fälschen
  für den Meisterfälschen} (Tages-Anzeiger, 2019)
\item
  \href{https://kanzleikompa.de/2020/02/18/olg-hamburg-deanonymisierung-von-autoren-politischer-beitraege-zulaessig/}{Deanonymisierung
  von Autoren politischer Beiträge zulässig} (Kompa, 2020)
\item
  \textbf{EN}:
  \href{https://swprs.org/wikipedia-disinformation-operation/}{Wikipedia:
  A Disinformation Operation?} (SPR, 2020)
\item
  \textbf{Buch}:
  \href{https://www.amazon.de/Schwarzbuch-Wikipedia-Diffamierung-Falschinformationen-Online-Enzyklop\%C3\%A4die/dp/3943007278}{Schwarzbuch
  Wikipedia} (Zeitgeist Verlag, 364 Seiten, 2020)
\item
  \href{https://www.faz.net/aktuell/feuilleton/medien/deutscher-rat-fuer-public-relations-ruegt-wikipedia-16733775.html}{PR-Rat
  rügt Wikipedia : Da fehlt es an Transparenz} (FAZ, 2020)
\item
  \textbf{EN}:
  \href{https://www.craigmurray.org.uk/archives/2020/04/information-wars/}{Wikipedia
  -- Information Wars} (Craig Murray, 2020)
\end{itemize}

Weiteres:

\begin{itemize}
\tightlist
\item
  \href{https://web.archive.org/web/20160907093409/http://de.pluspedia.org/wiki/Wikipedia_Blacklist}{Die
  Wikipedia-Blacklist} (PlusPedia, 2016, Archiv)
\item
  \href{http://de.pluspedia.org/wiki/Wikipedia}{Kritischer Artikel zur
  Wikipedia auf Pluspedia} (Pluspedia)
\item
  \href{https://de.wikipedia.org/wiki/Kritik_an_Wikipedia}{Kritik an
  Wikipedia} (Artikel auf Wikipedia)
\end{itemize}

Siehe auch:

\begin{itemize}
\tightlist
\item
  \href{https://swprs.org/wikipedia-missbrauch-massnahmen/}{Maßnahmen
  bei Missbrauch der Wikipedia} (SPR, 2019)
\end{itemize}

\begin{center}\rule{0.5\linewidth}{\linethickness}\end{center}

Beitrag teilen auf:
\href{https://twitter.com/intent/tweet?url=https://swprs.org/propaganda-in-der-wikipedia/}{Twitter}
/
\href{https://www.facebook.com/share.php?u=https://swprs.org/propaganda-in-der-wikipedia/}{Facebook}\\
Publiziert: Oktober 2018; Aktualisiert: Mai 2020

\hypertarget{swiss-policy-research}{%
\subsubsection{Swiss Policy Research}\label{swiss-policy-research}}

\begin{itemize}
\tightlist
\item
  \href{https://swprs.org/kontakt/}{Kontakt}
\item
  \href{https://swprs.org/uebersicht/}{Übersicht}
\item
  \href{https://swprs.org/donationen/}{Donationen}
\item
  \href{https://swprs.org/disclaimer/}{Disclaimer}
\end{itemize}

\hypertarget{english}{%
\subsubsection{English}\label{english}}

\begin{itemize}
\tightlist
\item
  \href{https://swprs.org/contact/}{About Us / Contact}
\item
  \href{https://swprs.org/media-navigator/}{The Media Navigator}
\item
  \href{https://swprs.org/the-american-empire-and-its-media/}{The CFR
  and the Media}
\item
  \href{https://swprs.org/donations/}{Donations}
\end{itemize}

\hypertarget{follow-by-email}{%
\subsubsection{Follow by email}\label{follow-by-email}}

Follow

\href{https://wordpress.com/?ref=footer_custom_com}{WordPress.com}.

\protect\hyperlink{}{Up ↑}

\includegraphics{https://pixel.wp.com/b.gif?v=noscript}
