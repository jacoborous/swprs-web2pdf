\protect\hyperlink{content}{Skip to content}

\href{https://swprs.org/}{}

\protect\hyperlink{search-container}{Search}

Search for:

\href{https://swprs.org/}{\includegraphics{https://swprs.files.wordpress.com/2020/05/swiss-policy-research-logo-300.png}}

\href{https://swprs.org/}{Swiss Policy Research}

Geopolitics and Media

Menu

\begin{itemize}
\tightlist
\item
  \href{https://swprs.org}{Start}
\item
  \href{https://swprs.org/srf-propaganda-analyse/}{Studien}

  \begin{itemize}
  \tightlist
  \item
    \href{https://swprs.org/srf-propaganda-analyse/}{SRF / ZDF}
  \item
    \href{https://swprs.org/die-nzz-studie/}{NZZ-Studie}
  \item
    \href{https://swprs.org/der-propaganda-multiplikator/}{Agenturen}
  \item
    \href{https://swprs.org/die-propaganda-matrix/}{Medienmatrix}
  \end{itemize}
\item
  \href{https://swprs.org/medien-navigator/}{Analysen}

  \begin{itemize}
  \tightlist
  \item
    \href{https://swprs.org/medien-navigator/}{Navigator}
  \item
    \href{https://swprs.org/der-propaganda-schluessel/}{Techniken}
  \item
    \href{https://swprs.org/propaganda-in-der-wikipedia/}{Wikipedia}
  \item
    \href{https://swprs.org/logik-imperialer-kriege/}{Kriege}
  \end{itemize}
\item
  \href{https://swprs.org/netzwerk-medien-schweiz/}{Netzwerke}

  \begin{itemize}
  \tightlist
  \item
    \href{https://swprs.org/netzwerk-medien-schweiz/}{Schweiz}
  \item
    \href{https://swprs.org/netzwerk-medien-deutschland/}{Deutschland}
  \item
    \href{https://swprs.org/medien-in-oesterreich/}{Österreich}
  \item
    \href{https://swprs.org/das-american-empire-und-seine-medien/}{USA}
  \end{itemize}
\item
  \href{https://swprs.org/bericht-eines-journalisten/}{Fokus I}

  \begin{itemize}
  \tightlist
  \item
    \href{https://swprs.org/bericht-eines-journalisten/}{Journalistenbericht}
  \item
    \href{https://swprs.org/russische-propaganda/}{Russische Propaganda}
  \item
    \href{https://swprs.org/die-israel-lobby-fakten-und-mythen/}{Die
    »Israel-Lobby«}
  \item
    \href{https://swprs.org/geopolitik-und-paedokriminalitaet/}{Pädokriminalität}
  \end{itemize}
\item
  \href{https://swprs.org/migration-und-medien/}{Fokus II}

  \begin{itemize}
  \tightlist
  \item
    \href{https://swprs.org/covid-19-hinweis-ii/}{Coronavirus}
  \item
    \href{https://swprs.org/die-integrity-initiative/}{Integrity
    Initiative}
  \item
    \href{https://swprs.org/migration-und-medien/}{Migration \& Medien}
  \item
    \href{https://swprs.org/der-fall-magnitsky/}{Magnitsky Act}
  \end{itemize}
\item
  \href{https://swprs.org/kontakt/}{Projekt}

  \begin{itemize}
  \tightlist
  \item
    \href{https://swprs.org/kontakt/}{Kontakt}
  \item
    \href{https://swprs.org/uebersicht/}{Seitenübersicht}
  \item
    \href{https://swprs.org/medienspiegel/}{Medienspiegel}
  \item
    \href{https://swprs.org/donationen/}{Donationen}
  \end{itemize}
\item
  \href{https://swprs.org/contact/}{English}
\end{itemize}

\protect\hyperlink{}{Open Search}

\hypertarget{anschlag-auf-die-forschungsfreiheit}{%
\section{Anschlag auf die
Forschungsfreiheit}\label{anschlag-auf-die-forschungsfreiheit}}

\includegraphics{https://swprs.files.wordpress.com/2018/11/ganser.png?w=400\&h=268}

So ergeht es US-kritischen Forschern in der Schweiz: Der Historiker
\textbf{Dr. Daniele Ganser} geriet 2006 nach einer öffentlichen
Inter­vention der amerikanischen Botschafterin unter Druck und musste
seine Forschung an der ETH Zürich schließlich aufgeben. Ganser forschte
zu verdeckter Kriegs­führung und
\href{http://ofv.ch/sachbuch/detail/natogeheimarmeen-in-europa/3193/}{inszeniertem
Terror} durch die NATO im Kalten Krieg sowie zu den Anschlägen vom 11.
September 2001 (siehe
\href{http://archiv.ethlife.ethz.ch/articles/9.11.html}{Artikel} im
ETH-Magazin).

Der Zürcher Tages-Anzeiger berichtete zum Eklat um Ganser:

»Mit seiner Kritik an der offiziellen 9/11-Version und seinen
öffentlichen Stellungnahmen seit 2005 hat Ganser teils heftige
Reaktionen ausgelöst. Gewisse Fachkollegen hätten ihm abgeraten, seine
Forschungen zu den Terror­anschlägen von 2001 weiter­zu­führen. Einige
hätten zwar unter vorge­haltener Hand zugestimmt, dass vieles ungeklärt
sei. Andere hätten aber gemahnt, solche Fragen seien zu politisch und
könnten auch in der Schweiz eine Karriere als Wissenschaftler ruinieren,
erzählt Ganser. Er sei auch beschimpft worden, selbst die US-Botschaft
in Bern habe protestiert.«
(\href{http://www.tagesanzeiger.ch/ausland/amerika/WTC7-und-andere-Raetsel-um-911/story/26888372?dossier_id=544}{WTC7
und andere Rätsel um 9/11}, Tages-Anzeiger vom 7. September 2011)*\\
*

Nach der ETH Zürich musste Ganser aufgrund seiner Forschung auch die
Universitäten Zürich und Basel
\href{https://www.aargauerzeitung.ch/schweiz/daniele-ganser-star-der-verschwoerungsszene-verstossenes-kind-der-wissenschaft-132406642}{verlassen}.
Der emeritierte ETH-Professor Albert Stahel, der die Anschläge zusammen
mit Ganser untersuchte,
\href{http://www.srf.ch/play/tv/einstein/video/die-anatomie-von-verschwoerungstheorien?id=94905414-61a4-40fe-ba8c-1c8048ed7a76\&startTime=778}{bestätigte}
in einem Interview: »Mir persönlich wurde sogar gesagt, es gibt Fragen,
die darf man im Zusammenhang mit 9/11 nicht stellen. Es war fast eine
Art Diffamierungskampagne. Man hat angefangen, uns beide auf breiter
Front zu diffamieren.«

Seit Ganser öffentlich als NATO-Kritiker auftritt, nehmen auch die
negativen Medienberichte über ihn zu: So in der Schweiz am Sonntag
(\href{http://www.schweizamsonntag.ch/ressort/basel/die_ganser-verschwoerung/}{»Die
Ganser-Verschwörung«)}, der NZZ
(\href{https://www.nzz.ch/feuilleton/medien/die-stimmen-des-digitalen-untergrunds-1.18627359}{»Die
Stimmen des digitalen Untergrunds«}), der WOZ
(\href{https://www.woz.ch/-768a}{»Das Ganser-Phänomen«}), der Berner
Zeitung
(\href{https://www.bernerzeitung.ch/schweiz/standard/das-geschaeft-mit-dem-zweifel/story/11814825}{»Das
Geschäft mit dem Zweifel«}), der TagesWoche
(\href{https://tageswoche.ch/gesellschaft/der-manipulator/}{»Der
Manipulator«}), der
\href{https://www.tagesanzeiger.ch/sonntagszeitung/wenn-alles-mit-allem-zu-tun-hat/story/23220523}{SonntagsZeitung}
und weiteren Medien.

Den vorläufigen Höhepunkt erreichte die Kampagne gegen den unbequemen
Historiker im Februar 2017, als das Schweizer Fernsehen SRF in der
Polit-Sendung »Arena« Ganser durch Einblendung einer privaten und
gekürzten E-Mail zu diskreditieren
\href{https://www.youtube.com/watch?v=vBYPSuiY8eE}{versuchte}. Dies
führte indes zu einer
\href{http://www.tagesanzeiger.ch/kultur/fernsehen/ganserarena-noch-nie-gab-es-so-viele-beschwerden/story/14245665}{rekordhohen}
Anzahl Zuschauerbeschwerden und einer
\href{https://www.srgd.ch/de/aktuelles/news/2017/04/11/arena-uber-unehrliche-medien-beanstandet/}{Rüge}
durch die Ombudsstelle.

Gleichwohl wurde daraufhin die Universität St. Gallen -- an der Ganser
zur »Geschichte und Zukunft von Energiesystemen« unterrichtete -- von
mehreren Medien unter Druck
\href{https://www.aargauerzeitung.ch/schweiz/ein-lehrauftrag-fuer-verschwoerungstheoretiker-ganser-professoren-kritisieren-die-hsg-131311387}{gesetzt}
und zur Beendigung seines Lehrauftrages
\href{https://www.aargauerzeitung.ch/schweiz/daniele-ganser-star-der-verschwoerungsszene-verstossenes-kind-der-wissenschaft-132406642}{bewegt}.
Damit war Ganser aus allen Schweizer Universitäten entfernt.

Der Redaktionsleiter der SRF Tagesschau bestätigte derweil in einem
bemerkenswerten
\href{https://www.srgd.ch/de/aktuelles/news/2017/06/11/srf-online-beitrag-flugzeug-absturz-auf-pentagon-die-erinnerung-911-beanstandet/}{Schreiben},
dass sich das SRF bezüglich 9/11 trotz »vieler Ungereimtheiten« an die
»offizielle Sicht der Dinge« halten müsse. Auch ein NZZ-Journalist, der
auf offene Fragen zu 9/11 hinweisen wollte, wurde von seinem CEO und
anderen Schweizer Journalisten sogleich öffentlich
\href{http://www.stefan-schaer.ch/2017/03/01/jan-flueckigers-kehrtwende-oder-wie-man-journalisten-mundtot-macht/}{zurechtgewiesen}.

Forschungsfreiheit ist ein hohes Gut. Doch wie frei sind Schweizer
Forscher, wenn es um geopolitisch brisante Themen geht?

\textbf{Hinweis:} Dr. Ganser ist nicht Mitglied der SPR-Forschungsgruppe
und kennt keines ihrer Mitglieder.

\hypertarget{siehe-auch}{%
\paragraph{Siehe auch}\label{siehe-auch}}

\begin{itemize}
\tightlist
\item
  \href{https://swprs.org/netzwerk-medien-schweiz/}{Medien-Netzwerk
  Schweiz}
\item
  \href{https://swprs.org/netzwerk-medien-deutschland/}{Medien-Netzwerk
  Deutschland}
\item
  \href{https://swprs.org/propaganda-in-der-wikipedia/}{Propaganda in
  der Wikipedia}
\end{itemize}

\begin{center}\rule{0.5\linewidth}{\linethickness}\end{center}

Beitrag teilen auf:
\href{https://twitter.com/intent/tweet?url=https://swprs.org/anschlag-auf-die-forschungsfreiheit/}{Twitter}
/
\href{https://www.facebook.com/share.php?u=https://swprs.org/anschlag-auf-die-forschungsfreiheit/}{Facebook}\\
Publiziert: März 2016; Aktualisiert: März 2018

\hypertarget{swiss-policy-research}{%
\subsubsection{Swiss Policy Research}\label{swiss-policy-research}}

\begin{itemize}
\tightlist
\item
  \href{https://swprs.org/kontakt/}{Kontakt}
\item
  \href{https://swprs.org/uebersicht/}{Übersicht}
\item
  \href{https://swprs.org/donationen/}{Donationen}
\item
  \href{https://swprs.org/disclaimer/}{Disclaimer}
\end{itemize}

\hypertarget{english}{%
\subsubsection{English}\label{english}}

\begin{itemize}
\tightlist
\item
  \href{https://swprs.org/contact/}{About Us / Contact}
\item
  \href{https://swprs.org/media-navigator/}{The Media Navigator}
\item
  \href{https://swprs.org/the-american-empire-and-its-media/}{The CFR
  and the Media}
\item
  \href{https://swprs.org/donations/}{Donations}
\end{itemize}

\hypertarget{follow-by-email}{%
\subsubsection{Follow by email}\label{follow-by-email}}

Follow

\href{https://wordpress.com/?ref=footer_custom_com}{WordPress.com}.

\protect\hyperlink{}{Up ↑}

Post to

\protect\hyperlink{}{Cancel}

\includegraphics{https://pixel.wp.com/b.gif?v=noscript}
