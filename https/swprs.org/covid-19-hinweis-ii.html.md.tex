\protect\hyperlink{content}{Skip to content}

\href{https://swprs.org/}{}

\protect\hyperlink{search-container}{Search}

Search for:

\href{https://swprs.org/}{\includegraphics{https://swprs.files.wordpress.com/2020/05/swiss-policy-research-logo-300.png}}

\href{https://swprs.org/}{Swiss Policy Research}

Geopolitics and Media

Menu

\begin{itemize}
\tightlist
\item
  \href{https://swprs.org}{Start}
\item
  \href{https://swprs.org/srf-propaganda-analyse/}{Studien}

  \begin{itemize}
  \tightlist
  \item
    \href{https://swprs.org/srf-propaganda-analyse/}{SRF / ZDF}
  \item
    \href{https://swprs.org/die-nzz-studie/}{NZZ-Studie}
  \item
    \href{https://swprs.org/der-propaganda-multiplikator/}{Agenturen}
  \item
    \href{https://swprs.org/die-propaganda-matrix/}{Medienmatrix}
  \end{itemize}
\item
  \href{https://swprs.org/medien-navigator/}{Analysen}

  \begin{itemize}
  \tightlist
  \item
    \href{https://swprs.org/medien-navigator/}{Navigator}
  \item
    \href{https://swprs.org/der-propaganda-schluessel/}{Techniken}
  \item
    \href{https://swprs.org/propaganda-in-der-wikipedia/}{Wikipedia}
  \item
    \href{https://swprs.org/logik-imperialer-kriege/}{Kriege}
  \end{itemize}
\item
  \href{https://swprs.org/netzwerk-medien-schweiz/}{Netzwerke}

  \begin{itemize}
  \tightlist
  \item
    \href{https://swprs.org/netzwerk-medien-schweiz/}{Schweiz}
  \item
    \href{https://swprs.org/netzwerk-medien-deutschland/}{Deutschland}
  \item
    \href{https://swprs.org/medien-in-oesterreich/}{Österreich}
  \item
    \href{https://swprs.org/das-american-empire-und-seine-medien/}{USA}
  \end{itemize}
\item
  \href{https://swprs.org/bericht-eines-journalisten/}{Fokus I}

  \begin{itemize}
  \tightlist
  \item
    \href{https://swprs.org/bericht-eines-journalisten/}{Journalistenbericht}
  \item
    \href{https://swprs.org/russische-propaganda/}{Russische Propaganda}
  \item
    \href{https://swprs.org/die-israel-lobby-fakten-und-mythen/}{Die
    »Israel-Lobby«}
  \item
    \href{https://swprs.org/geopolitik-und-paedokriminalitaet/}{Pädokriminalität}
  \end{itemize}
\item
  \href{https://swprs.org/migration-und-medien/}{Fokus II}

  \begin{itemize}
  \tightlist
  \item
    \href{https://swprs.org/covid-19-hinweis-ii/}{Coronavirus}
  \item
    \href{https://swprs.org/die-integrity-initiative/}{Integrity
    Initiative}
  \item
    \href{https://swprs.org/migration-und-medien/}{Migration \& Medien}
  \item
    \href{https://swprs.org/der-fall-magnitsky/}{Magnitsky Act}
  \end{itemize}
\item
  \href{https://swprs.org/kontakt/}{Projekt}

  \begin{itemize}
  \tightlist
  \item
    \href{https://swprs.org/kontakt/}{Kontakt}
  \item
    \href{https://swprs.org/uebersicht/}{Seitenübersicht}
  \item
    \href{https://swprs.org/medienspiegel/}{Medienspiegel}
  \item
    \href{https://swprs.org/donationen/}{Donationen}
  \end{itemize}
\item
  \href{https://swprs.org/contact/}{English}
\end{itemize}

\protect\hyperlink{}{Open Search}

\hypertarget{fakten-zu-covid-19}{%
\section{Fakten zu Covid-19}\label{fakten-zu-covid-19}}

\textbf{Aktualisiert}:~\protect\hyperlink{latest}{August 2020};
\textbf{Teilen auf}:
\href{https://twitter.com/intent/tweet?url=https://swprs.org/covid-19-hinweis-ii/}{Twitter}
/
\href{https://www.facebook.com/share.php?u=https://swprs.org/covid-19-hinweis-ii/}{Facebook}\\
\textbf{Lang.}: \href{https://swprs.org/fakta-o-covid-19/}{CZ},
\href{https://swprs.org/covid-19-hinweis-ii/}{DE},
\href{https://swprs.org/a-swiss-doctor-on-covid-19/}{EN},
\href{https://swprs.org/faktoj-pri-kovim-19/}{EO},
\href{https://swprs.org/hechos-sobre-covid-19/}{ES},
\href{https://swprs.org/faktoja-covid-19sta/}{FI},
\href{https://swprs.org/coronavirus-un-medecin-suisse-parle/}{FR},
\href{https://swprs.org/facts-about-covid19-greek/}{GR},
\href{https://swprs.org/covid-19-cinjenice/}{HBS},
\href{https://yanivhamo.com/facts-about-covid-19-hebrew/}{HE},
\href{https://swprs.org/egy-svajci-orvos-a-covid-19-rol/}{HU},
\href{https://swprs.org/un-medico-svizzero-su-covid-19/}{IT},
\href{https://swprs.org/covid19-facts-japanese/}{JP},
\href{https://swprs.org/covid19-korean/}{KO},
\href{https://swprs.org/feiten-over-covid19/}{NL},
\href{https://midtifleisen.wordpress.com/2020/04/15/fakta-om-covid-19/}{NO},
\href{https://swprs.org/szwajcarski-lekarz-o-covid-19/}{PL},
\href{https://swprs.org/fatos-sobre-covid-19/}{PT},
\href{https://swprs.org/informatii-despre-covid-19/}{RO},
\href{https://swprs.org/\%d0\%bd\%d0\%b0-\%d0\%ba\%d0\%be\%d0\%b2\%d0\%b8\%d0\%b4-19/}{RU},
\href{https://swprs.org/fakta-om-covid-19/}{SE},
\href{http://www.ninamvseeno.org/pregled-clanka.aspx?naslov=pomembne-informacije-o-novem-koronavirusu-covid-19\&id=148}{SI},
\href{https://alatyr.sk/covid-19_swiss_propaganda_research.htm}{SK},
\href{https://swprs.org/isvicreli-bir-doktordan-kovid-19-uezerine/}{TR}

Von Fachleuten präsentierte, vollständig referenzierte Fakten zu
Covid-19, die unseren Lesern eine realistische Risikobeurteilung
ermöglichen sollen. (Updates siehe unten)*\\
*

\textbf{``Die einzige Art, gegen die Pest zu kämpfen, ist die
Ehrlichkeit.'' (Albert Camus, 1947)}

\hypertarget{uxfcbersicht}{%
\paragraph{Übersicht}\label{uxfcbersicht}}

\begin{enumerate}
\def\labelenumi{\arabic{enumi}.}
\tightlist
\item
  Laut den neuesten immunologischen Studien liegt die \textbf{Letalität}
  von Covid-19 (IFR) bei
  \href{https://swprs.org/studies-on-covid-19-lethality/}{insgesamt
  circa 0.1\% bis 0.3\%} und damit im Bereich einer starken Influenza
  (Grippe).
\item
  Bei Personen mit hohem Risiko oder hoher Exposition (inklusive
  Pflegemitarbeiter) ist eine
  \href{https://swprs.org/zur-behandlung-von-covid-19/}{frühzeitige oder
  sogar prophylaktische} \textbf{Behandlung} entscheidend.
\item
  In Ländern wie den USA, UK oder auch Schweden (ohne Lockdown) liegt
  die \textbf{Gesamt­mortalität} seit Jahresbeginn
  \href{https://swprs.org/studies-on-covid-19-lethality/\#overall-mortality}{im
  Bereich einer} starken Grippesaison; in Ländern wie Deutschland und
  der Schweiz liegt die Gesamtmortalität bisher im Bereich einer milden
  Grippesaison.
\item
  Das \textbf{Sterberisiko} für die Allgemeinbevölkerung im Schul- und
  Arbeitsalter liegt in den meisten Regionen im Bereich einer
  \href{https://www.medrxiv.org/content/10.1101/2020.04.05.20054361v1}{täglichen
  Autofahrt} zur Arbeit. Das Risiko wurde zunächst überschätzt, da
  Personen mit milden oder keinen Symptomen nicht erfasst wurden.
\item
  Bis zu 80\% aller testpositiven Personen bleiben
  \href{https://www.bmj.com/content/369/bmj.m1375}{symptomlos}. Selbst
  unter den 70- bis 79-Jährigen bleiben
  \href{https://www.niid.go.jp/niid/en/2019-ncov-e/9407-covid-dp-fe-01.html}{rund
  60\%} \textbf{symptomlos}. Circa 95\% aller Personen zeigen
  \href{https://swprs.org/studies-on-covid-19-lethality/\#hospitalizations}{höchstens
  moderate} Symptome.
\item
  Bis zu 60\% aller Personen
  \href{https://www.cell.com/cell/fulltext/S0092-8674(20)30610-3}{verfügen
  bereits} über eine gewisse zelluläre \textbf{Hinter­grund­immunität}
  gegen das neue Virus durch den Kontakt mit bisherigen Coronaviren
  (d.h. Erkältungsviren). Die Annahme, es gebe keine Immunität gegen das
  neue Coronavirus, war nicht zutreffend.
\item
  Das \textbf{Medianalter} der Verstorbenen liegt in den meisten Ländern
  (inkl. Italien) bei
  \href{https://swprs.org/studies-on-covid-19-lethality/\#age}{über 80
  Jahren} (z.B. in Schweden bei 86 Jahren) und nur
  \href{https://www.bloomberg.com/news/articles/2020-05-26/italy-says-96-of-virus-fatalities-suffered-from-other-illnesses}{circa
  4\%} der Verstorbenen hatten keine ernsthaften Vorerkrankungen. Das
  Sterbeprofil entspricht damit im Wesentlichen der
  \href{https://www.vienna.at/analyse-zeigt-covid-19-opferkurve-entspricht-normaler-mortalitaet/6581246}{normalen
  Sterblichkeit}.
\item
  In vielen Ländern ereigneten sich
  \href{https://swprs.org/studies-on-covid-19-lethality/\#care-homes}{bis
  zu zwei Drittel} aller Todesfälle in \textbf{Pflegeheimen}, die von
  einem allgemeinen Lockdown nicht profitieren. Zudem ist oftmals
  \href{https://www.hsj.co.uk/commissioning/thousands-of-extra-deaths-outside-hospital-not-attributed-to-covid-19/7027459.article}{nicht
  klar}, ob diese Menschen wirklich an Covid-19 starben oder an
  wochenlangem
  \href{https://www.theguardian.com/world/2020/jun/05/covid-19-causing-10000-dementia-deaths-beyond-infections-research-says}{Stress
  und Isolation}.
\item
  Bis zu 30\% aller \textbf{zusätzlichen Todesfälle}
  \href{https://www.ons.gov.uk/peoplepopulationandcommunity/birthsdeathsandmarriages/deaths/articles/analysisofdeathregistrationsnotinvolvingcoronaviruscovid19englandandwales28december2019to1may2020/technicalannex}{wurden
  nicht} durch Covid-19 verursacht, sondern durch die Folgen von
  \href{https://www.telegraph.co.uk/global-health/science-and-disease/two-new-waves-deaths-break-nhs-new-analysis-warns/}{Lockdown,
  Panik und Angst}. So ging etwa die Behandlung von Herzinfarkten und
  Hirnschlägen um bis zu 60\%
  \href{https://www.nytimes.com/2020/04/06/well/live/coronavirus-doctors-hospitals-emergency-care-heart-attack-stroke.html}{zurück},
  da sich Patienten nicht mehr in die Kliniken wagten.
\item
  Selbst bei den sogenannten \textbf{``Covid19-Todesfällen''} ist
  oftmals
  \href{https://spectator.us/understand-report-figures-covid-deaths/}{nicht
  klar}, ob sie \emph{an} oder \emph{mit} Coronaviren (d.h. an den
  \href{https://www.morgenpost.de/vermischtes/article228994571/Rechtsmediziner-Alle-Corona-Toten-hatten-Vorerkrankungen.html}{Vorerkrankungen})
  starben oder als ``Verdachtsfälle'' gar nicht getestet
  \href{https://www.youtube.com/watch?v=V0lIWZpiRU0}{wurden}. Die
  offiziellen Zahlen reflektieren diese Unterscheidungen jedoch
  \href{https://swprs.org/rki-relativiert-corona-todesfaelle/}{oftmals
  nicht}.
\item
  Viele Medienberichte, wonach auch \textbf{junge und gesunde Personen}
  an Covid-19 starben, stellten sich als falsch heraus: Viele dieser
  jungen Menschen starben entweder
  \href{https://www.dailymail.co.uk/news/article-8193487/Coroner-refuses-rule-COVID-19-cause-death-six-week-old-Connecticut-baby.html}{nicht}
  an Covid-19, waren doch bereits schwer
  \href{https://sports.yahoo.com/spanish-football-coach-francisco-garcia-163153573.html}{vorerkrankt}
  (z.B. an Leukämie), oder sie waren
  \href{https://www.n-tv.de/panorama/Neunjaehrige-Corona-Tote-war-109-Jahre-alt-article21753784.html}{109
  statt 9 Jahre} alt. Die angebliche Zunahme der Kawasaki-Krankheit bei
  Kindern war ebenso eine
  \href{https://www.societi.org.uk/kawasaki-disease-covid-19/responding-to-press-coverage-28-april-2020/}{Übertreibung}.
\item
  Die meisten \textbf{Covid-19-Symptome} können auch durch eine starke
  Influenza ausgelöst werden (inklusive Lungen­ent­zündungen,
  \href{https://www.sciencedaily.com/releases/2009/10/091014111549.htm}{Thrombosen}
  und der temporäre Verlust des
  \href{https://pubmed.ncbi.nlm.nih.gov/23948436/}{Geruchssinns}), aber
  bei starkem Covid-19 sind diese Symptome tatsächlich häufiger und
  ausgeprägter.
\item
  Regional stark \textbf{erhöhte Sterblichkeiten} können entstehen, wenn
  es zu einem infektions- oder panikbedingten
  \href{https://swprs.org/covid19-bericht-aus-italien/}{Kollaps der
  Alten- und Krankenpflege} kommt oder zusätzliche Risikofaktoren wie
  \href{https://www.heise.de/tp/features/Feinstaubpartikel-als-Viren-Vehikel-4687454.html}{starke
  Luftverschmutzung} bestehen. Fragwürdige
  \href{https://www.ecdc.europa.eu/sites/default/files/documents/COVID-19-safe-handling-of-bodies-or-persons-dying-from-COVID19.pdf}{Vorschriften}
  zum Umgang mit Verstorbenen führten teilweise zu
  \href{https://www.globalresearch.ca/truth-behind-refrigerated-morgue-truck-stories/5711475}{zusätzlichen
  Engpässen} bei Bestattungen und Kremierungen.
\item
  In Ländern wie Italien und Spanien sowie teilweise Großbritannien und
  den USA haben
  \href{https://off-guardian.org/2020/04/02/coronavirus-fact-check-1-flu-doesnt-overwhelm-our-hospitals/}{auch
  Grippewellen} bereits bisher zu einer \textbf{Überlastung der
  Kliniken} geführt. Derzeit müssen zudem
  \href{https://www.nytimes.com/2020/03/24/world/europe/coronavirus-europe-covid-19.html}{bis
  zu 15\%} der Ärzte und Pfleger, auch ohne Symptome, in Quarantäne.
\item
  Die oft gezeigten Exponentialkurven mit \textbf{``Coronafällen''} sind
  \href{https://multipolar-magazin.de/artikel/coronavirus-irrefuhrung-fallzahlen}{irreführend},
  da auch die Anzahl der Tests exponentiell zunahm. In den meisten
  Ländern blieb das Verhältnis von positiven Tests zu Tests insgesamt
  (sog. Positivenrate)
  \href{https://swprs.org/rate-of-positive-covid19-tests/}{konstant bei
  5\% bis 20\%} oder nahm nur leicht zu. Der Höhepunkt der Ausbreitung
  war in den meisten Ländern bereits
  \href{https://infekt.ch/2020/04/sind-wir-tatsaechlich-im-blindflug/}{vor
  dem Lockdown} erreicht.
\item
  Länder \emph{ohne} \textbf{Ausgangssperren}, wie z.B.
  \href{https://www.bloomberg.com/news/articles/2020-05-22/did-japan-just-beat-the-virus-without-lockdowns-or-mass-testing}{Japan},
  \href{https://www.businessinsider.com/south-korea-coronavirus-testing-death-rate-2020-3?op=1}{Südkorea},
  \href{https://www.forbes.com/sites/jamesrodgerseurope/2020/04/04/in-belarus-lukashenko-has-his--own-ways-for-the-country-to-face-coronavirus/}{Weißrussland}
  und
  \href{https://www.addendum.org/coronavirus/interview-johan-giesecke/}{Schweden},
  haben
  \href{https://www.washingtontimes.com/news/2020/apr/15/sweden-coronavirus-rates-easing-despite-loose-rule/}{keinen
  negativeren Verlauf} als viele andere Länder erlebt. Schweden wurde
  von der WHO sogar als
  \href{https://www.nau.ch/news/schweiz/coronavirus-who-nennt-schweden-ein-vorbild-65701044}{vorbildliches
  Modell} gelobt und profitiert nun von einer
  \href{https://news.ki.se/immunity-to-covid-19-is-probably-higher-than-tests-have-shown}{hohen
  Immunität}. 75\% der schwedischen Todesfälle
  \href{https://www.thelocal.se/20200525/swedish-death-toll-passes-4000-as-coronavirus-cases-in-care-homes-start-to-fall}{erfolgten
  in} Pflege­ein­rich­tungen, die zu spät geschützt wurden.
\item
  Die Angst vor einer Knappheit an \textbf{Beatmungsgeräten} war
  \href{https://off-guardian.org/2020/05/06/covid19-are-ventilators-killing-people/}{unberechtigt}.
  Laut
  \href{https://www.vpneumo.de/fileadmin/pdf/f2004071.007_Voshaar.pdf}{Lungenfachärzten}
  ist die invasive Beatmung (Intubation) von Covid19-Patienten, die
  teilweise
  \href{https://nypost.com/2020/05/29/northwell-health-probing-use-of-ventilators-for-covid-patients/}{aus
  Angst} vor dem Virus geschah, zudem oftmals
  \href{https://www.doccheck.com/de/detail/articles/26271-covid-19-beatmung-und-dann}{kontraproduktiv}
  und schädigt die Lungen zusätzlich.
\item
  Verschiedene Studien
  \href{https://www.nordkurier.de/ratgeber/es-gibt-keine-gefahr-jemandem-beim-einkaufen-zu-infizieren-0238940804.html}{zeigten},
  dass die haupsächliche \textbf{Übertragung des Virus} weder durch
  weitreichende
  \href{https://www.who.int/news-room/commentaries/detail/transmission-of-sars-cov-2-implications-for-infection-prevention-precautions}{Aerosole}
  (in der Luft \emph{schwebende} Partikel) noch
  \href{https://www.yahoo.com/lifestyle/cdc-coronavirus-mainly-spreads-through-persontoperson-contact-and-does-not-spread-easily-on-contaminated-surfaces-153317029.html}{über
  Oberflächen} geschieht, sondern durch direkten Körperkontakt und
  Tröpfchen. In Innenräumen ist eine aerosolartige Übertragung jedoch
  unter
  \href{https://jamanetwork.com/journals/jama/fullarticle/2768396}{gewissen
  Bedingungen} möglich.
\item
  Für die Wirksamkeit von \textbf{Masken} bei \emph{gesunden} und
  \emph{symptomlosen} Personen gibt es weiterhin
  \href{https://swprs.org/face-masks-evidence/}{kaum Evidenz}. Experten
  warnen zudem, dass solche Masken die Atmung beeinträchtigen und bei
  Mehr­fach­verwendung zu
  \href{https://de.sputniknews.com/interviews/20200425326953541-corona-gefahr-virologe/}{``Keimschleudern''}
  werden können.
\item
  Viele \textbf{Kliniken} in Europa und den USA
  \href{https://www.usatoday.com/story/news/health/2020/04/02/coronavirus-pandemic-jobs-us-health-care-workers-furloughed-laid-off/5102320002/}{blieben}
  während der Lockdowns stark
  \href{https://www.spiegel.de/wirtschaft/unternehmen/trotz-corona-pandemie-warum-kliniken-jetzt-kurzarbeit-anmelden-a-3dc61bc9-fb12-4298-8022-bb4c2be39d7d}{unterbelegt}
  oder mussten sogar
  \href{https://www.20min.ch/schweiz/news/story/Spitaeler-28949526}{Kurzarbeit}
  anmelden. Millionen von Operationen und Therapien wurden
  \href{https://www.birmingham.ac.uk/news/latest/2020/05/covid-disruption-28-million-surgeries-cancelled.aspx}{abgesagt},
  darunter auch zahlreiche Krebs­unter­suchungen und
  Organ­trans­plan­ta­tionen.
\item
  Mehrere \textbf{Medien} wurden dabei
  \href{https://nypost.com/2020/04/01/cbs-admits-to-using-footage-from-italy-in-report-about-nyc/}{ertappt},
  wie sie die Situation in Kliniken zu dramatisieren versuchten,
  teilweise sogar mit
  \href{https://www.wsj.com/articles/cbs-says-fake-news-wasnt-theirs-11588789238}{manipulativen}
  Bildern. Generell bewirkte die
  \href{https://www.infosperber.ch/Artikel/Medien/Corona-Medien-verbreiten-weiter-unbeirrt-statistischen-Unsinn}{unseriöse
  Berichterstattung} vieler Medien eine Maximierung der Angst in der
  Bevölkerung.
\item
  Die international verwendeten \textbf{Virentestkits} sind
  \href{https://www.ncbi.nlm.nih.gov/pubmed/32219885}{fehleranfällig}
  und können falsche positive und falsche negative Resultate
  \href{https://multipolar-magazin.de/artikel/warum-die-pandemie-nicht-endet}{ergeben}.
  Der offizielle Virentest wurde aus Zeitdruck zudem
  \href{https://www.youtube.com/watch?v=p_AyuhbnPOI}{nicht klinisch
  validiert} und kann auch auf andere Coronaviren (Erkältungsviren)
  positiv reagieren.
\item
  Zahlreiche renommierte
  \href{https://www.rubikon.news/artikel/120-expertenstimmen-zu-corona}{Experten}aus
  den Bereichen Virologie, Immunologie und Epidemiologie
  \href{https://off-guardian.org/2020/03/24/12-experts-questioning-the-coronavirus-panic/}{halten}
  die getroffenen \textbf{Maßnahmen} für
  \href{https://off-guardian.org/2020/03/28/10-more-experts-criticising-the-coronavirus-panic/}{kontraproduktiv}
  und empfehlen eine rasche
  \href{https://off-guardian.org/2020/04/17/8-more-experts-questioning-the-coronavirus-panic/}{natürliche
  Immunisierung} der Allgemeinbevölkerung und den Schutz von
  Risikogruppen.
\item
  Für die Schließung von \textbf{Schulen} gab es zu
  \href{https://infekt.ch/2020/04/schulen-schliessen-hilfreich-oder-nicht/}{keinem
  Zeitpunkt} einen medizinischen Grund, da das Erkrankung- und
  Übertragungs­risiko bei Kindern
  \href{https://thehill.com/opinion/education/500349-science-says-open-the-schools}{äußerst
  gering ist}. Auch für Kleinklassen, Masken oder Abstandsregeln in
  Schulen gibt es
  \href{https://www.welt.de/politik/deutschland/article208075525/Corona-Kitas-und-Grundschulen-vollstaendig-oeffnen-uneingeschraenkt.html}{keinen
  medizinischen Grund}.
\item
  Mehrere Experten
  \href{https://www.statnews.com/2020/07/31/covid-19-vaccine-amazingly-close-why-am-i-so-worried/}{bezeichneten}
  forcierte \textbf{Impfstoffe} gegen Coronaviren als
  \href{https://www.news.com.au/lifestyle/health/health-problems/no-vaccine-for-coronavirus-a-possibility/news-story/34e678ae205b50ea983cc64ab2943608}{unnötig}
  oder sogar
  \href{https://www.nature.com/articles/d41586-020-00751-9}{gefährlich}.
  Tatsächlich führte etwa der Impfstoff gegen die
  \href{https://www.forbes.com/2010/02/05/world-health-organization-swine-flu-pandemic-opinions-contributors-michael-fumento.html}{sog.
  Schweinegrippe} von 2009 zu teilweise schweren
  \href{https://www.ibtimes.co.uk/brain-damaged-uk-victims-swine-flu-vaccine-get-60-million-compensation-1438572}{neurologischen
  Schäden} und Klagen in Millionenhöhe. Auch bei Tests von
  Corona-Impfstoffen
  \href{https://www.forbes.com/sites/williamhaseltine/2020/05/16/did-the-oxford-covid-vaccine-work-in-monkeys-not-really/}{kam
  es bereits} zu gravierenden
  \href{https://childrenshealthdefense.org/news/vaccine-trial-catastrophe-moderna-vaccine-has-20-serious-injury-rate-in-high-dose-group/}{Komplikationen}.
\item
  Eine globale Pandemie kann sich durchaus über
  \href{https://www.britannica.com/event/1968-flu-pandemic}{mehrere
  Jahre} erstrecken, doch viele Studien zu einer \textbf{``Zweiten
  Welle''} basieren auf
  \href{https://www.heise.de/tp/features/Fellay-Studie-Zweite-Corona-Welle-4726303.html}{sehr
  unrealistischen Annahmen}, wie z.B. einem konstanten Erkrankungs- und
  Sterberisiko über alle Altersgruppen.
\item
  In den USA beschrieben Krankenschwestern eine oft
  \href{https://www.youtube.com/watch?v=UIDsKdeFOmQ}{tödliche
  \textbf{Fehlbehandlung}} von Covid-Patienten aufgrund fragwürdiger
  \href{https://www.usatoday.com/story/news/factcheck/2020/04/24/fact-check-medicare-hospitals-paid-more-covid-19-patients-coronavirus/3000638001/}{finanzieller
  Anreize} und ungeeigneter Methoden.
\item
  Die Anzahl an Menschen, die aufgrund der Maßnahmen an
  \textbf{Arbeitslosigkeit},
  \href{https://www.indystar.com/story/news/health/2020/04/03/coronavirus-indiana-how-get-help-mental-health-addiction/5104357002/}{Depression}
  und häuslicher Gewalt leiden, hat weltweit
  \href{https://www.businessinsider.com/us-weekly-jobless-claims-unemployment-filings-coronavirus-labor-market-layoffs-2020-5}{Höchstwerte
  erreicht}. Mehrere Experten gehen davon aus, dass die Maßnahmen
  deutlich mehr Leben
  \href{https://www.nytimes.com/2020/03/20/opinion/coronavirus-pandemic-social-distancing.html}{fordern
  werden} als das Virus selbst. Laut UNO sind weltweit
  \href{https://www.theguardian.com/world/2020/apr/29/half-of-worlds-workers-at-immediate-risk-of-losing-livelihood-due-to-coronavirus}{1.6
  Milliarden Menschen} vom akuten Verlust ihrer Lebens­grund­lagen
  bedroht.
\item
  NSA-Whistleblower Edward Snowden warnte, dass ``Corona'' für den
  \href{https://www.youtube.com/watch?v=-pcQFTzck_c}{permanenten Ausbau}
  von technologischen \textbf{Überwachungs­instrumenten} genutzt wird.
  Weltweit kam es zum Monitoring der Zivilbevölkerung
  \href{https://off-guardian.org/2020/04/25/50-headlines-darker-more-of-the-new-normal/}{durch
  Drohnen} und zu teilweise massiver Polizeigewalt.
\item
  Eine WHO-Studie von 2019 zu Maßnahmen gegen Grippepandemien ergab,
  dass \textbf{``Kontakt­verfolgung''} aus medizinischer Sicht
  \href{https://apps.who.int/iris/bitstream/handle/10665/329438/9789241516839-eng.pdf\#page=9}{``unter
  keinen Umständen zu empfehlen''} ist. Dennoch wurden Tracing-Apps in
  mehreren
  \href{https://www.heise.de/tp/features/CuidAR-Argentinien-ueberwacht-mit-einer-App-4720143.html}{Ländern}
  bereits teilweise
  \href{https://www.technologyreview.com/2020/05/07/1001360/india-aarogya-setu-covid-app-mandatory/}{obligatorisch}.
  In einigen Ländern wird diese Kontakt­ver­folgung direkt vom
  Geheimdienst
  \href{https://www.jewishpress.com/news/the-courts/state-to-high-court-even-more-shin-bet-involvement-in-fighting-the-coronavirus/2020/04/14/}{durchgeführt}.
\end{enumerate}

\hypertarget{siehe-auch}{%
\subparagraph{\texorpdfstring{\textbf{Siehe
auch}}{Siehe auch}}\label{siehe-auch}}

\begin{enumerate}
\def\labelenumi{\arabic{enumi}.}
\tightlist
\item
  \href{https://swprs.org/zur-behandlung-von-covid-19/}{Studien zur
  Behandlung von Covid-19}
\item
  \href{https://swprs.org/studies-on-covid-19-lethality/}{Studien zur
  Letalität von Covid-19 (IFR)}
\item
  \href{https://swprs.org/ursprung-des-covid-19-virus-die-mojiang-minenarbeiter-hypothese/}{Zum
  Ursprung des neuen Coronavirus}
\item
  \href{https://swprs.org/face-masks-evidence/}{Zur Wirksamkeit von
  Masken (Übersicht)}
\item
  \href{https://euromomo.eu}{Das europäische Mortalitätsmonitoring}
\end{enumerate}

\begin{center}\rule{0.5\linewidth}{\linethickness}\end{center}

\hypertarget{uxfcbersichtsgrafiken}{%
\paragraph{Übersichtsgrafiken}\label{uxfcbersichtsgrafiken}}

\includegraphics{https://swprs.files.wordpress.com/2020/06/covid-19-comparison-e1592927192181.png?w=736\&h=600}

\includegraphics{https://swprs.files.wordpress.com/2020/07/global-cases-deaths.png?w=736\&h=533}

\includegraphics{https://swprs.files.wordpress.com/2020/07/sweden-projection-reality-june-28.png?w=736\&h=496}

\includegraphics{https://swprs.files.wordpress.com/2020/07/nys-vs-sweden-deaths.png?w=736\&h=365}

\includegraphics{https://swprs.files.wordpress.com/2020/07/sweden-vs-england-deaths.png?w=736\&h=354}

\includegraphics{https://swprs.files.wordpress.com/2020/06/sweden-all-cause-nov-may-1990.jpg?w=736\&h=499}

\includegraphics{https://swprs.files.wordpress.com/2020/07/usa-daily-deaths.png?w=736\&h=209}

\includegraphics{https://swprs.files.wordpress.com/2020/06/us-cumulative-deaths.png?w=736\&h=438}

\includegraphics{https://swprs.files.wordpress.com/2020/07/us-age-adjusted-death-rate.png?w=736\&h=503}

\includegraphics{https://swprs.files.wordpress.com/2020/07/us-care-home-deaths-june-19.png?w=736}

\includegraphics{https://swprs.files.wordpress.com/2020/05/care-home-deaths-may-21.png?w=736\&h=443}

\includegraphics{https://swprs.files.wordpress.com/2020/06/us-2020-recession.jpg?w=736\&h=482}

\includegraphics{https://swprs.files.wordpress.com/2020/07/england-wales-2020-1999-comparison.png?w=736\&h=393}

\includegraphics{https://swprs.files.wordpress.com/2020/06/uk-flu-comparison.png?w=736\&h=437}

\includegraphics{https://swprs.files.wordpress.com/2020/07/schweiz-todesfaelle-2010-2020_woche_29.png?w=736\&h=322}

\includegraphics{https://swprs.files.wordpress.com/2020/06/sterbefallzahlen-de-25-05.png?w=736\&h=414}

\hypertarget{august-2020}{%
\paragraph{August 2020}\label{august-2020}}

\hypertarget{a-allgemeiner-teil}{%
\subparagraph{A. Allgemeiner Teil}\label{a-allgemeiner-teil}}

\hypertarget{zur-immunituxe4t-gegen-das-neue-coronavirus}{%
\subparagraph{**Zur Immunität gegen das neue
Coronavirus}\label{zur-immunituxe4t-gegen-das-neue-coronavirus}}

**

Zu Beginn der Corona-Pandemie bestand die Befürchtung, gegen das neue
Coronavirus gebe es keine Immunität in der Bevölkerung. Dies war einer
der wesentlichen Gründe für die große Angst vor dem Virus und für die
ursprüngliche Strategie des ``flatten the curve''.

Ab März und April erschienen jedoch die ersten Studien, die aufzeigten,
dass ein erheblicher Teil der Bevölkerung doch bereits über eine
\href{https://www.cell.com/cell/fulltext/S0092-8674(20)30610-3}{gewisse
Immunität} gegen das neue Virus verfügt, die durch den Kontakt mit
früheren Coronaviren (Erkältungsviren) erworben wurde.

Im Juli erschienen nun weitere wichtige Studien zu diesem Thema:

\begin{itemize}
\tightlist
\item
  Eine \href{https://www.researchsquare.com/article/rs-35331/v1}{neue
  Studie aus Deutschland} kam zum Ergebnis, dass bis zu 81\% der
  Personen, die noch keinen Kontakt mit dem neuen Coronavirus hatten,
  bereits über kreuzreaktive \textbf{T-Zellen} (durch frühere
  Coronaviren) und damit über eine gewisse Hinter­grund­immunität
  verfügen. Dies bestätigt frühere Studien zur T-Zellen-Immunität.
\item
  Eine
  \href{https://www.biorxiv.org/content/10.1101/2020.05.14.095414v2}{britische
  Studie} fand überdies, dass \emph{bis zu 60\% der Kinder und
  Jugendlichen} und circa 6\% der Erwachsenen bereits über kreuzreaktive
  \textbf{Antikörper} gegen das neue Coronavirus verfügen, die durch den
  Kontakt mit bisherigen Coronaviren entstanden sind. Dies dürfte ein
  weiterer wichtiger Aspekt zur Erklärung der sehr geringen
  Krankheitsrate bei Kindern und Jugendlichen sein.
\item
  Eine Studie im Fachblatt \emph{Nature} kam im Falle von Singapur
  \href{https://www.nature.com/articles/s41586-020-2550-z}{zum
  Ergebnis}, dass Personen, die 2002/2003 an SARS-1 erkrankt waren, auch
  17 Jahre später noch über T-Zellen verfügten, die auch gegen das neue
  SARS-2-Coronavirus reaktiv sind. Zudem fanden die Forscher bei rund
  der Hälfte der Personen, die weder an SARS-1 noch an SARS-2 erkrankt
  waren, bereits kreuzreaktive T-Zellen, die durch den Kontakt mit
  anderen, \emph{teilweise unbekannten} Coronaviren entstanden sind. Die
  Forscher vermuten, dass die unterschiedliche Verbreitung solcher
  Coronaviren und T-Zellen miterklären können, warum manche Länder
  stärker oder schwächer vom neuen Coronavirus betroffen sind,
  unabhängig davon, welche politisch-medizinischen Maßnahmen sie
  ergreifen.
\item
  Bereits zuvor haben Analysten
  \href{https://twitter.com/boriquagato/status/1280212958392446977}{darauf
  aufmerksam} gemacht, dass \textbf{pazifische Länder} und insbesondere
  die Nachbarländer Chinas bisher sehr tiefe Covid-Todesraten aufweisen,
  und zwar unabhängig von ihrer Bevölkerungsstruktur (jung oder alt) und
  den ergriffenen Maßnahmen (mit oder ohne Lockdown, Massentests, Masken
  etc.). Eine mögliche Erklärung dafür könnte die Verbreitung früherer
  Coronaviren sein.
\item
  Der Harvard-Immunologe Michael Mina erklärte, dass das von einigen
  Medien dramatisierte ``Abfallen der Antikörper-Konzentration'' nach
  einer Covid-Erkrankung
  \href{https://www.nytimes.com/2020/07/22/health/covid-antibodies-herd-immunity.html}{``völlig
  normal''} und ``wie im Lehrbuch'' sei. Der Körper stelle die
  längerfristige Immunität durch T-Zellen und Erinnerungs­zellen im
  Knochenmark sicher, die bei Bedarf rasch neue Antikörper erzeugen
  können.
\end{itemize}

\textbf{Siehe auch}:
\href{https://swprs.org/studies-on-covid-19-lethality/}{Immunologische
Studien zum neuen Coronavirus}

\hypertarget{weitere-medizinische-updates}{%
\subparagraph{\texorpdfstring{\textbf{Weitere medizinische
Updates}}{Weitere medizinische Updates}}\label{weitere-medizinische-updates}}

\textbf{Wuhan}: Eine Harvard-Modellierungsstudie im Fachblatt
\emph{Nature} kam zum Ergebnis, dass selbst im Corona-Epizentrum Wuhan
bis zu 87\% der Infektionen
\href{https://news.harvard.edu/gazette/story/2020/07/study-finds-early-wuhan-covid-cases-largely-undetected/}{unbemerkt
blieben}, d.h. symptomlos oder mild verliefen
(\href{https://www.n-tv.de/wissen/Bis-zu-87-Prozent-der-Infektionen-unerkannt-article21917095.html}{deutscher
Artikel dazu}). Damit dürfte die Covid19-Letalität (IFR) auch in Wuhan
auf circa 0.1\% oder darunter fallen. Die Nature-Studie bestätigt eine
\href{https://www.medrxiv.org/content/10.1101/2020.02.12.20022434v3}{japanische
Studie} im Fachblatt \emph{BMC Medicine}, die für Wuhan bereits im März
eine IFR von 0.12\% errechnete.

Die chinesischen Behörden konnten diese vergleichsweise geringe
Letalität im Januar und Februar allerdings noch nicht kennen und bauten
deshalb kurzfristig zusätzliche Kliniken, die dann allerdings
\href{https://www.theguardian.com/world/2020/feb/12/what-chinas-empty-new-coronavirus-hospitals-say-about-its-secretive-system}{größtenteils
ungenutzt} blieben. Erst die systematischen Testresultate aus Südkorea
und vom Kreuzfahrtschiff Diamond Princess zeigten, dass die Letalität
des neuen Coronavirus in der Allgemeinbevölkerung viel geringer ist als
zunächst befürchtet.

\textbf{Italien}: Die italienische Gesundheitsbehörde ISS publizierte
\href{https://www.reuters.com/article/us-health-coronavirus-italy-study-idUSKCN24H2VZ}{eine
neue Analyse} zur Todesursache bei knapp 5000 Corona-Patienten. Demnach
sei bei 89\% der Todesfälle Covid die \emph{direkte Todesursache}
gewesen. Bei 11\% seien andere Erkrankungen wie Herzprobleme, Krebs oder
Demenz die primäre Todesursache gewesen. Bei 28\% sei Covid die
\emph{alleinige Todesursache} gewesen. Bekannt ist zudem, dass bei
\href{https://www.bloomberg.com/news/articles/2020-05-26/italy-says-96-of-virus-fatalities-suffered-from-other-illnesses}{circa
4\% der Todesfälle} keine Vorerkrankungen bestanden.

\textbf{Zur Covid-Letalität}: Die US-Gesundheitsbehörde CDC publizierte
im Mai eine vorsichtige ``beste Schätzung'' der Covid-Letalität (IFR)
von 0.26\% (unter Annahme von 35\% asymptomatischen Infektionen). Im
Juli wurde nun ein
\href{https://www.cdc.gov/coronavirus/2019-ncov/hcp/planning-scenarios.html}{neuer
IFR-Wert} von 0.65\% publiziert. Dieser neue Wert basiert jedoch nicht
auf eigenen Berechnungen oder neuen Studien, sondern auf
\href{https://www.medrxiv.org/content/10.1101/2020.05.03.20089854v4}{einer
Meta-Studie}, in der die bestehende Literatur einfach nach allen
bisherigen IFR-Werten durchsucht wurde.

Dadurch besteht die Meta-Studie hauptsächlich aus früheren
Modellierungsstudien sowie aus ``rohen IFR-Werten'', die im Vergleich zu
den tatsächlichen, populationsbasierten IFR-Werten aus
Antikörper-Studien~\href{https://swprs.org/covid-19-letalitat-wie-man-es-nicht-macht/}{viel
zu hoch} sind. Die wirklichen IFR-Werte liegen mit wenigen Ausnahmen bei
0.1\% bis 0.4\%, unter Berücksichtigung von mukosaler und zellulärer
Immunität \href{https://swprs.org/studies-on-covid-19-lethality/}{bei
ca. 0.1\%} oder darunter.

Das neue Coronavirus verbreitete sich jedoch deutlich schneller als
angenommen, was in einigen Regionen zu einer vorübergehend sehr hohen
Sterberate führte, insbesondere wenn Pflegeheime oder Krankenhäuser
betroffen waren.

\textbf{Nicht-infektiöse Virenfragmente}: Die US-Gesundheitsbehörde CDC
\href{https://www.cdc.gov/coronavirus/2019-ncov/hcp/duration-isolation.html}{macht
darauf aufmerksam}, dass sich bei den meisten Covid-Erkrankten zehn bis
fünfzehn Tage nach den ersten Symptomen keine infektiösen Virenpartikel
mehr auffinden lassen. Allerdings lassen sich \emph{bis zu drei Monate}
nach den ersten Symptomen noch nicht-infektiöse Virenfragmente (RNA)
auffinden. Dies dürfte mit Blick auf PCR-Tests ein erhebliches Problem
darstellen, da viele Menschen, die längst nicht mehr ansteckend sind,
immer noch positiv getestet werden und dadurch weitreichende Tracing-
und Quarantäne-Fehlalarme ausgelöst werden.

\textbf{Todesfälle mit oder durch oder ohne Coronaviren}: In England und
einigen anderen Ländern
\href{https://www.cebm.net/covid-19/why-no-one-can-ever-recover-from-covid-19-in-england-a-statistical-anomaly/}{wurde
bekannt}, dass alle verstorbenen Personen, die seit Anfang Jahr einmal
positiv auf das neue Coronavirus testeten, als Covid-Todesfälle gezählt
wurden -- unabhängig vom Zeitpunkt des Tests, einer möglichen Genesung,
und der wirklichen Todesursache. Im US-Bundesstaat Colorado zeigte sich,
dass \href{https://covid19.colorado.gov/data/case-data}{circa 10\% der
Todesfälle} \emph{mit} aber \emph{nicht an} Coronaviren erfolgten. In
den USA wurden weitere Fälle von ``Corona-Toten'' bekannt, bei denen es
sich in Wirklichkeit um testpositive
\href{https://cbs12.com/news/local/i-team-deaths-incorrectly-attributed-to-covid-19-in-palm-beach-county}{Mordopfer}
und
\href{https://www.fox35orlando.com/news/fox-35-investigates-questions-raised-after-fatal-motorcycle-crash-listed-as-covid-19-death}{Motorradunfälle}
handelte.

\hypertarget{kinder-und-schulen}{%
\subparagraph{\texorpdfstring{\textbf{Kinder und
Schulen}}{Kinder und Schulen}}\label{kinder-und-schulen}}

Bereits seit März ist bekannt, dass das Erkrankungs- und
Übertragungsrisiko bei Kindern im Falle von Covid19 minimal ist.
Hauptgrund dafür dürfte eine vorbestehende Immunität sein durch den
häufigen Kontakt mit bisherigen Coronaviren (d.h. Erkältungsviren). Für
die Schließung von Grundschulen, Kindergärten und Kitas sowie für
besondere Schutzmaßnahmen in Schulen gab und gibt es deshalb keinen
medizinischen Grund.

Inzwischen wurden einige weitere Aspekte zu diesem Thema bekannt:

\begin{itemize}
\tightlist
\item
  Der britische Epidemiologe Professor Mark Woolhouse erklärte, es gebe
  weltweit\href{https://www.thetimes.co.uk/article/no-known-case-of-teacher-catching-coronavirus-from-pupils-says-scientist-3zk5g2x6z}{keinen
  einzigen bestätigten Fall} einer Ansteckung eines Lehrers durch einen
  Schüler.**\\
  **
\item
  Tracing-Pionier Island fand
  ``\href{https://www.sciencemuseumgroup.org.uk/blog/hunting-down-covid-19/}{keinen
  einzigen Fall}, in dem ein Kind unter 10 Jahren seine Eltern
  angesteckt hat.''
\item
  Ein gemeinsamer Bericht von Schweden (ohne Schließung der
  Grundschulen) und Finnland (mit Schließung der Grundschulen) kam zum
  Ergebnis, dass sich die Infektionsraten bei Kindern in den beiden
  Länder
  \href{https://www.reuters.com/article/us-health-coronavirus-sweden-schools-idUSKCN24G2IS}{nicht
  unterschieden}.
\item
  In den USA starben laut der Gesundheitsbehörde CDC seit Jahresbeginn
  \href{https://childrenshealthdefense.org/news/if-covid-fatalities-were-90-2-lower-how-would-you-feel-about-schools-reopening/}{dreimal
  mehr} Kinder bis 14 Jahre an Influenza als an Covid-19 (101 versus
  31).
\item
  Eine
  \href{https://www.theglobeandmail.com/canada/article-new-syndrome-in-children-thought-to-be-linked-to-covid-19-yields/}{kanadische
  Untersuchung} kam zum Ergebnis, dass die meisten der Kinder mit~
  ``Kawasaki-ähnlichen'' Entzündungssymptomen gar keine Corona-Infektion
  hatten. Die medial stark dramatisierte Erkrankung bei Kindern sei
  ``sehr, sehr selten'', so die Forscher.
\item
  Eine deutsche Studie
  \href{https://www.faz.net/aktuell/politik/inland/corona-studie-an-schulen-kinder-eher-bremskloetze-der-infektion-16858827.html}{kam
  zum Ergebnis}, dass Kinder epidemiologisch ``wie Bremsklötze'' wirken
  und die Ausbreitung des neuen Coronavirus verlangsamen.
\end{itemize}

\hypertarget{kritische-expertenstimmen}{%
\subparagraph{\texorpdfstring{\textbf{Kritische
Expertenstimmen}}{Kritische Expertenstimmen}}\label{kritische-expertenstimmen}}

\begin{itemize}
\tightlist
\item
  Der deutsche Virologe \textbf{Hendrik Streeck} plädiert für einen
  \href{https://www.merkur.de/welt/coronavirus-zweite-welle-impfstoff-streeck-virologe-warnung-drosten-massentest-infektion-zr-13834907.html}{pragmatischen
  Umgang} mit dem neuen Coronavirus und gezielte Maßnahmen für Menschen
  mit hohem Risiko oder schweren Krankheitsverläufen. Das langfristige
  Unterdrücken des Virus und das Hoffen auf einen möglichen Impfstoff
  sind laut Streeck keine sinnvollen Strategien.
\item
  Professor \textbf{Carl Heneghan}, der Direktor des Oxford Centre for
  Evidence-Based Medicine,
  \href{https://www.youtube.com/watch?v=Z3plSbCbkSA}{erklärt in einem
  Interview}, dass es für die Wirksamkeit von Masken in der
  Allgemeinbevölkerung weiterhin keine Evidenz gebe. Eine permanente
  Unterdrückungs-Strategie wie in Neuseeland sei nicht sinnvoll und
  richte längerfristig hohe Schäden an. Die Letalität (IFR) von Covid-19
  liege bei ca. 0.1 bis 0.3\% und sei damit vergleichbar mit früheren
  Grippe-Epidemien und -Pandemien.
\item
  Der schwedische Chef-Epidemiologe \textbf{Anders Tegnell}
  \href{https://www.youtube.com/watch?v=xh9wso6bEAc}{erklärt in einem
  Interview}, dass die Ausrottung des Virus keine Option sei. In
  Schweden hätten sich die Infektionen auch ohne Lockdown stark
  verlangsamt, die neuen Todesfälle lägen inzwischen nahe bei null. Die
  Evidenz für den Nutzen von Masken sei immer noch ``sehr schwach'' und
  sie könnten sogar kontraproduktiv sein. Eine Einführung zum jetzigen
  Zeitpunkt mache keinen Sinn. Die Letalität von Covid-19 liege zwischen
  0.1\% bis 0.5\% und unterscheide sicht nicht radikal von einer
  Influenza. Schweden müsse aufgrund der Bevölkerungsstruktur und
  Reisetätigkeit epidemiologisch eher mit den Niederlanden verglichen
  werden als mit Norwegen und Finnland.
\item
  Der Epidemiologe und Systembiologe Professor \textbf{Francois
  Balloux}, Direktor des britischen UCL Genetics Institute,
  \href{https://twitter.com/BallouxFrancois/status/1284513419454971905}{erklärt
  in einem Beitrag}, dass Covid-19 mit einer pandemischen Influenza
  vergleichbar sei. Der Hauptunterschied bestehe in der
  Alters-Risikoverteilung: Während Covid-19 hauptsächlich für ältere
  Menschen gefährlich sei, sei eine pandemische Influenza zusätzlich
  auch für jüngere Menschen und Kinder lebensgefährlich. Professor
  Balloux weist darauf hin, dass die ``russische Grippepandemie'' von
  1889 womöglich vom Coronavirus OC-43 ausgelöst wurde, das heute als
  eines der vier typischen Erkältungsviren gelte.
\item
  Der Schweizer Chefarzt für Infektiologie, \textbf{Dr. Pietro
  Vernazza},
  \href{https://corona-transition.org/der-infektiologe-prof-pietro-vernazza-sieht-covid-19-im-bereich-einer}{plädiert
  für eine} ``kontrollierte Durchseuchung'' der Gesellschaft als
  Alternative zur ``Ausrottungsstrategie''. In den meisten Fällen
  verlaufe Covid-19 mild, die tatsächliche Sterblichkeit liege bei etwa
  0.1\% und damit im Bereich einer starken Influenza. Die Schweden
  hätten mit ihrer Strategie ``nichts falsch gemacht''.
\item
  Der ehemalige Direktor des Instituts für Immunologie der Universität
  Bern, \textbf{Professor Beda Stadler},
  \href{https://www.achgut.com/artikel/coronas_zeugen}{plädiert
  ebenfalls für} eine kontrollierte Durchseuchung der Gesellschaft. Die
  Gefährlichkeit des Virus sei aufgrund der falschen Annahme einer
  fehlenden Immunität überschätzt worden. Maskenpflicht und Massentests
  sieht Professor Stadler kritisch. In einem
  \href{https://www.youtube.com/watch?v=GBRcK-od50Q}{weiteren Interview}
  erklärt der emeritierte Professor Stadler, dass sich viele jüngere
  Immunologen aufgrund der extremen Polarisierung der Debatte durch
  Politik und Medien nicht mehr getrauen, sich öffentlich zum Thema zu
  äußern.
\end{itemize}

Andererseits hat \textbf{Professor Karin Mölling}, die ehemalige
Leiterin der Virologie an der Universität Zürich und eine der frühesten
kritischen Stimmen zu den Corona-Maßnahmen, ihre Meinung inzwischen
\href{https://www.bazonline.ch/eine-zweite-welle-laesst-sich-vermeiden-zumindest-bis-im-winter-959708338566}{teilweise
geändert}: Aufgrund der teilweise ernsthaften Lungenschäden dürfe das
Virus nicht unterschätzt werden und seien Maßnahmen zur Eindämmung
wichtig.

\hypertarget{zum-krankheitsbild-von-covid-19}{%
\subparagraph{\texorpdfstring{\textbf{Zum Krankheitsbild von
Covid-19}}{Zum Krankheitsbild von Covid-19}}\label{zum-krankheitsbild-von-covid-19}}

Die geringere Letalität von Covid-19 sollte nicht darüber
hinwegtäuschen, dass das neue Coronavirus aufgrund seiner effizienten
Nutzung des menschlichen ACE2-Zellrezeptors in einigen Fällen zu
\href{https://www.nature.com/articles/s41591-020-0968-3}{schweren
Krankheitsverläufen} mit Komplikationen in der Lunge, dem Gefäß- und
Nervensystem und weiteren Organen führen kann, die teilweise
\href{https://jamanetwork.com/journals/jama/fullarticle/2768351}{monatelang
nachwirken} können.

Es ist zwar richtig, dass die meisten dieser Symptome auch bei schwerer
Influenza auftreten können (dies gilt auch für
\href{https://www.sciencedaily.com/releases/2009/10/091014111549.htm}{Thrombosen}
und den temporären Verlust des
\href{https://pubmed.ncbi.nlm.nih.gov/23948436/}{Geruchssinns}), aber
bei der neuartigen Covid-19-Erkrankung treten sie tatsächlich häufiger
und ausgeprägter auf.

Hinzu kommt die Tatsache, dass auch scheinbar ``milde'' Verläufe (ohne
Hospitalisierung) in manchen Fällen zu langwierigen Komplikationen mit
Atemproblemen, Müdigkeit oder anderen Symptomen führen können. Die
US-Behörde CDC kam
\href{https://www.cdc.gov/mmwr/volumes/69/wr/mm6930e1.htm}{zum
Ergebnis}, dass nach einem Monat noch etwa ein Drittel der scheinbar
``milden'' Fälle solche Symptome aufweisen. Selbst bei den 18- bis
34-Jährigen ohne Vorerkrankungen waren es noch etwa 20\% mit
Nachwirkungen.

Immerhin vermeldeten Forscher am Klinikum Stuttgart zuletzt
\href{https://www.tagesschau.de/investigativ/kontraste/coronavirus-langzeitfolgen-101.html}{gute
Heilungschancen}: ``Wir können erkennen, dass die Lunge gut heilen kann,
auch bei Patienten, die drei Wochen Intensivstation hinter sich haben.''
Nach drei Monaten hatten 20\% der Intensivpatienten wieder eine gesunde
Lunge, bei den restlichen Patienten war eine deutliche Regeneration
sichtbar.

Dennoch sollte es oberstes Ziel sein, eine Progression der Erkrankung zu
vermeiden.

\hypertarget{zur-behandlung-von-covid-19}{%
\subparagraph{\texorpdfstring{\textbf{Zur Behandlung von
Covid-19}}{Zur Behandlung von Covid-19}}\label{zur-behandlung-von-covid-19}}

\textbf{Hinweis}: Patienten wenden sich an einen Arzt.

Viele Länder setzten auf die Strategie, während oder nach einer
Infektions­welle einen Lockdown zu verhängen und dadurch bereits
infizierte Risikopersonen ohne Behandlung zuhause einzuschließen, bis
sie schwere Atemprobleme entwickelten und direkt eine risikoreiche
intensivmedizinische Behandlung benötigten. Noch heute werden
testpositive Risikopersonen oftmals ohne Behandlung einfach unter
Quarantäne gestellt.

Dies ist kein optimaler Ansatz. Zahlreiche Studien und Ärzteberichte
haben inzwischen dargelegt, dass bei Personen mit hohem Risiko oder
hoher Exposition eine \emph{frühzeitige Behandlung} unmittelbar bei
Auftreten der ersten typischen Symptome entscheidend ist, um eine
Progression der Erkrankung und eine Hospitalisierung zu vermeiden.

Studien und Ärzteberichte aus verschiedenen Ländern in Ost und West
empfehlen hierfür insbesondere
\href{https://swprs.org/zur-behandlung-von-covid-19/}{ein
Kombinationsprotokoll} aus Zink (das die RNA-Replikation von Coronaviren
hemmt), dem Malariamittel HCQ (das die zelluläre Aufnahme von Zink
fördert und weitere antivirale Eigenschaften besitzt), sowie bei Bedarf
einem Antibiotikum (zur Verhinderung von bakteriellen Superinfektionen)
und einem Blutverdünner (zur Verhinderung von Thrombosen und Embolien).

Der Yale-Professor und Arzt Harvey A. Risch argumentiert in einem
\href{https://www.newsweek.com/key-defeating-covid-19-already-exists-we-need-start-using-it-opinion-1519535}{neuen
Kommentar}, dass sich die Frühbehandlung mit HCQ und Zink sowie einem
Antibiotikum als hochwirksam erwiesen habe. Allein in den USA hätten
durch den systematischen Einsatz von HCQ laut Professor Risch 70,000 bis
100,000 Todesfälle
\href{https://www.foxnews.com/media/hydroxychloroquine-could-save-lives-ingraham-yale-professor}{verhindert
werden} können. Risch fordert deshalb eine sofortige und rezeptfreie
Freigabe dieser Medikation, wie dies in anderen Ländern bereits seit
langem der Fall ist.

Um den Einsatz des kostengünstigen HCQ -- das
\href{https://swprs.files.wordpress.com/2020/07/hcq-white-paper-dr-simone-gold.pdf}{seit
Jahrzehnten erfolgreich und risikoarm} in der Prophylaxe und Behandlung
von Malaria und einigen anderen Krankheiten verwendet wird -- ist in
westlichen Industrieländern indessen ein bizarrer Kampf entbrannt, der
vor allem politisch und kommerziell motiviert zu sein scheint und der
sehr viele Opfer in Kauf nimmt.

HCQ-Gegner sind dabei selbst vor gefälschten Studien und tödlichen
Dosierungen nicht zurückgeschreckt,
\href{https://www.medicineuncensored.com/a-study-out-of-thin-air}{wie
Dr. James Todaro erklärt}, der einen dieser Betrugsfälle aufdeckte, auf
den führende Fachjournale, die WHO und Gesundheitsexperten weltweit
``hereingefallen'' sind.

Oftmals gibt es in diesem Zusammenhang Verbindungen zur
\href{https://omnij.org/Gilead:_Twenty-one_billion_reasons_to_discredit_hydroxychloroquine_(ORIGINAL_ARTICLE)}{Pharmafirma
Gilead}, die ein über einhundert mal teureres Medikament vertreiben
möchte (Remdesivir), das jedoch erst bei Intensivpatienten eingesetzt
wird und zudem
\href{https://www.sciencedirect.com/science/article/pii/S1201971220305282}{schwere
Nebenwirkungen} aufweist.

Zudem steht eine potentiell wirkungsvolle Frühbehandlung der
milliardenschweren globalen Impfstrategie entgegen, die von zahlreichen
Regierungen, Pharmaunternehmen und Impfinvestor Bill Gates verfolgt
wird. Direktoren von Impfstoffunternehmen haben bisher allein mit
Aktien- und Optionsgewinnen und noch ohne Impfstoff
\href{https://www.nytimes.com/2020/07/25/business/coronavirus-vaccine-profits-vaxart.html}{bereits
eine Milliarde Dollar} verdient.

Die Hoffnung auf einen sicheren und wirkungsvollen Impfstoff erscheint
hingegen weiterhin fragwürdig: So reagierten beim RNA-Impfstoff der
US-Firma Moderna in der zweiten Testrunde -- entgegen der
Mediendarstellung -- 80\% der Probanden (Durchschnittsalter 33 Jahre und
gesund) in den mittel- und hoch-dosierten Gruppen mit
\href{https://childrenshealthdefense.org/news/vaccines/letter-to-wv-legislators-the-moderna-covid-19-vaccine-is-likely-to-make-more-people-sick-than-covid-19/}{moderaten
bis schweren Nebenwirkungen}.

\textbf{Weiterlesen}:
\href{https://swprs.org/zur-behandlung-von-covid-19/}{Zur Behandlung von
Covid-19}

\hypertarget{zur-wirksamkeit-von-masken}{%
\subparagraph{\texorpdfstring{\textbf{Zur Wirksamkeit von
Masken}}{Zur Wirksamkeit von Masken}}\label{zur-wirksamkeit-von-masken}}

Verschiedene Länder diskutieren eine teilweise Maskenpflicht in der
Allgemeinbevölkerung oder haben diese bereits eingeführt. In den Updates
vom Juni und Juli wurde indes dargestellt, dass die Evidenz für die
Wirksamkeit von Stoffmasken in der Bevölkerung, entgegen der Darstellung
in vielen Medien,
\href{https://swprs.org/face-masks-evidence/}{weiterhin sehr schwach}
ist.

Bei früheren Grippe-Pandemien hatten Stoffmasken
\href{https://wwwnc.cdc.gov/eid/article/26/5/19-0994_article}{keinen
Einfluss} auf das Infektions­geschehen. Das oft genannte Maskenland
Japan hatte seine letzte Grippe-Epidemie mit
\href{https://www.upi.com/Top_News/World-News/2019/02/01/Millions-in-Japan-affected-as-flu-outbreak-grips-country/9191549043797/}{über
fünf Millionen Erkrankten} erst gerade vor einem Jahr, im Januar und
Februar 2019. Auch den Ausbruch der Covid-Pandemie in Wuhan konnten die
dort verbreiteten Masken nicht aufhalten.

Aufgrund der deutlich geringeren Hospitalisierungs- und Sterberate von
Covid-19 (im Vergleich zu den ursprünglichen Annahmen) ist eine
Maskenpflicht nicht unbedingt erforderlich, um ``die Kurve
abzuflachen''. Eine Maskenpflicht macht -- wenn überhaupt -- nur im
Rahmen einer Impfstrategie Sinn, bei der man das Virus bis zu einem
Impfstoff unterdrücken möchte.

Die BBC-Medizinkorrespondentin Deborah Cohen erklärte Mitte Juli, dass
die teilweise Anpassung der WHO-Empfehlung bezüglich Masken nicht
aufgrund neuer Evidenz erfolgte, sondern aufgrund von
\href{https://twitter.com/ClarkeMicah/status/1282987860090593280}{``politischem
Lobbying''}: ``We had been told by various sources WHO committee
reviewing the evidence had not backed masks but they recommended them
due to political lobbying. This point was put to WHO who did not deny.''

Beim ``politischen Lobbying'' dürfte es sich insbesondere um die Gruppe
\href{https://masks4all.co/about-us/}{``Masks for All''} handeln, die
von einem ``Young Leader'' des WEF Davos gegründet wurde und die sich
bei Behörden und Regierungen für eine weltweite Maskenpflicht einsetzt.

Im Zusammenhang mit Masken stellt sich auch die Frage, ob das neue
Coronavirus durch \textbf{Aerosole} weiträumig übertragen werden kann.
Eine echte Aerosol-Übertragung auch im Freien erscheint laut Fachleuten
\href{https://jamanetwork.com/journals/jama/fullarticle/2768396}{weiterhin
unwahrscheinlich} -- die Verbreitung des Virus würde sonst eine andere
Dynamik aufweisen und wäre entgegen der Realität oftmals nicht
rückverfolgbar.

Doch eine aerosolartige Übertragung in Innenräumen -- insbesondere bei
einer geschlossenen Luftumwälzung durch Ventilatoren oder bei intensiven
Aktivitäten wie Singen und Tanzen -- erscheint aufgrund verschiedener
Vorfälle zunehmend wahrscheinlich bzw. gesichert.

Bei einer Übertragung durch Aerosole dürften Stoffmasken aufgrund ihrer
Porengröße und ungenauen Passform allerdings noch weniger Schutz bieten
als bei Tröpfchen. Dies zeigte sich beispielsweise beim Corona-Ausbruch
beim deutschen Fleischverarbeiter Tönnies, der am klimatisierten
Arbeitsplatz
\href{https://swprs.org/tonnies-corona-ausbruch-trotz-maskenpflicht/}{trotz
Maskenpflicht} über bis zu acht Meter Distanz erfolgte.

Zur Frage der \textbf{``asymptomatischen Übertragung''} lässt sich
derzeit sagen, dass eine \emph{echte} \emph{asymptomatische} Übertragung
laut WHO weiterhin
\href{https://www.webmd.com/lung/news/20200609/who-clairifies-comments-on-asymptomatic-covid-spread}{selten
zu sein scheint} (was unter anderem die sehr geringe Übertragbarkeit bei
Kindern erklären dürfte), während jedoch eine \emph{prä-symptomatische}
Übertragung in den Tagen vor den ersten Symptomen (mit bereits hoher
Viruslast) sehr wahrscheinlich ist und die rasche Verbreitung des Virus
erklären dürfte.

Die prä-symptomatische Übertragung ist auch
\href{https://virologydownunder.com/influenza-virus-transmission-with-or-without-symptoms-youre-dropping-flu-virus/}{von
der Influenza} bekannt, allerdings ist dort die Inkubationszeit
wesentlich kürzer, sodass dies etwas weniger relevant sein dürfte.

Es folgen einige \textbf{aktuelle Fachkommentare und Artikel}, die sich
kritisch mit der Wirksamkeit von Stoffmasken in der Allgemeinbevölkerung
befassen.

\begin{itemize}
\tightlist
\item
  \textbf{Profs. Tom Jefferson and Carl Heneghan} (Oxford):
  \href{https://www.cebm.net/covid-19/masking-lack-of-evidence-with-politics/}{Masking
  lack of evidence with politics}
\item
  \textbf{Dr. Lisa Brosseau and Dr. Margaret Sietsema}, Center for
  Infectious Disease Research and Policy, University of Minnesota:
  \href{https://www.cidrap.umn.edu/news-perspective/2020/04/commentary-masks-all-covid-19-not-based-sound-data}{Masks-for-all
  for COVID-19 not based on sound data}
\item
  \textbf{Professor Michael T. Osterholm}, Center for Infectious Disease
  Research and Policy, University of Minnesota:
  \href{https://www.cidrap.umn.edu/news-perspective/2020/07/commentary-my-views-cloth-face-coverings-public-preventing-covid-19}{My
  views on cloth face coverings for the public for preventing COVID-19}
\item
  \textbf{Naoya Kon}:
  \href{http://www.asahi.com/ajw/articles/13523664}{Cloth face masks
  offer zero shield against virus, a study shows}
\item
  \textbf{Eliza
  McGraw}:\href{https://www.seattletimes.com/nation-world/everyone-wore-masks-during-the-1918-flu-pandemic-they-were-useless/}{Everyone
  wore masks during the 1918 flu pandemic. They were useless.}
\end{itemize}

Der schwedische Chefepidemiologe Anders Tegnell
\href{https://www.bloomberg.com/news/articles/2020-07-28/sweden-unveils-promising-covid-19-data-as-new-cases-plunge}{erklärte
zuletzt}, die Einführung von Masken zum jetzigen Zeitpunkt wäre
angesicht der in Schweden rasch sinkenden Fallzahlen selbst im
öffentlichen Verkehr ``sinnlos''. Die niederländische Regierung
\href{https://www.reuters.com/article/us-health-coronavirus-netherlands-idUSKCN24U2UJ}{erklärte},
sie werde das allgemeine Tragen von Masken nicht empfehlen, da die
wissenschaftliche Evidenz dafür schwach sei.

Masken sind im Übrigen durchaus nicht harmlos, wie die folgenden Aspekte
zeigen:

\begin{itemize}
\tightlist
\item
  Die WHO warnt vor diversen
  \href{https://www.who.int/publications/i/item/advice-on-the-use-of-masks-in-the-community-during-home-care-and-in-healthcare-settings-in-the-context-of-the-novel-coronavirus-(2019-ncov)-outbreak}{``Nebenwirkungen''}
  wie Atembeschwerden und Hautausschlägen.
\item
  Das Universitätsklinikum Leipzig kam bei Tests
  \href{https://science.orf.at/stories/3201213/}{zum Ergebnis}, dass
  Schutzmasken die Belastbarkeit und Leistungsfähigkeit gesunder
  Personen deutlich verringert.
\item
  Eine deutsche psychologische Studie mit rund 1000 Teilnehmern fand
  teilweise
  \href{https://corona-transition.org/der-maskenzwang-ist-verantwortlich-fur-schwere-psychische-schaden-und-die}{``schwere
  psychosoziale Folgen''} durch die eingeführte Maskenpflicht in
  Deutschland.
\item
  Das Hamburger Umweltinstitut warnte vor dem
  \href{https://corona-transition.org/maskentragen-noch-ungesunder-als-gedacht}{Einatmen
  von Chlorverbindungen} in Polyester-Masken sowie vor Problemen im
  Zusammenhang mit der Entsorgung.
\item
  Das europäische Schnellwarnsystem RAPEX rief bereits
  \href{https://corona-transition.org/maskentragen-noch-ungesunder-als-gedacht}{70
  Maskenmodelle zurück}, da sie nicht den EU-Qualitätsstandards
  entsprachen und zu ``schwerwiegenden Risiken'' führen können.
\item
  In China
  \href{https://www.ibtimes.com/2-chinese-boys-wearing-masks-during-gym-class-dropped-dead-reports-say-2971556}{starben
  zwei Buben}, die beim Sportunterricht eine Maske tragen mussten.
\item
  In den USA wurde ein Autofahrer mit einer N95-Maske (FFP2) ohnmächtig
  und
  \href{https://nypost.com/2020/04/24/driver-crashes-car-after-passing-out-from-wearing-n95-mask/}{verunfallte
  tödlich}.
\end{itemize}

\textbf{Fazit}: Es ist weiterhin möglich, dass Stoffmasken in der
Allgemeinbevölkerung das Infektions­geschehen verlangsamen können, aber
die Evidenz dafür ist bisher gering und der mögliche Nutzen
hauptsächlich im Rahmen einer langfristigen und unsicheren Impfstrategie
relevant.

\textbf{Mehr dazu}: \href{https://swprs.org/face-masks-evidence/}{Are
face masks effective? The evidence.}

\hypertarget{ist-covid-19-eine-reine-test-epidemie-gewiss-nicht}{%
\subparagraph{**Ist Covid-19 eine reine ``Test-Epidemie''? Gewiss
nicht.}\label{ist-covid-19-eine-reine-test-epidemie-gewiss-nicht}}

**

Einige besonders skeptische Beobachter scheinen Covid-19 weiterhin
vorwiegend als eine ``Test-Epidemie'' zu sehen. Diese Position ist
jedoch schon seit Monaten nicht mehr haltbar.

Die bekannteste ``Test-Epidemie'' ist die sogenannte Schweinegrippe von
2009/2010. Es handelte sich um ein eher mildes Grippevirus, das
lediglich aufgrund von Massentests und medialer Panik für weltweite
Aufregung sorgte. Eine Kommission des Europarates bezeichnete die
Schweinegrippe später als eine
\href{https://www.forbes.com/2010/02/05/world-health-organization-swine-flu-pandemic-opinions-contributors-michael-fumento.html\#658c006c48e8}{``Fake-Pandemie''}
und einen Pharma-Betrug.

Brisant war damals, dass die WHO wenige Monate zuvor ihre
\href{https://www.forbes.com/2010/02/05/world-health-organization-swine-flu-pandemic-opinions-contributors-michael-fumento.html\#658c006c48e8}{Pandemie-Richtlinien
änderte} und das Kriterium der erhöhten Letalität entfernte.
Pharmafirmen schlossen zudem milliarden­schwere
\href{https://www.dailymail.co.uk/news/article-1242147/The-false-pandemic-Drug-firms-cashed-scare-swine-flu-claims-Euro-health-chief.html}{Geheimverträge}
mit Regierungen für einen Impfstoff, der später zu teilweise
gravierenden
\href{https://www.ibtimes.co.uk/brain-damaged-uk-victims-swine-flu-vaccine-get-60-million-compensation-1438572}{neurologischen
Schäden} führte und am Ende größtenteils entsorgt werden musste.

Schließlich stellten Forscher fest, dass das Schweinegrippe-Virus
vermutlich selbst
\href{https://link.springer.com/article/10.1186/1743-422X-6-207}{aus der
Impfstoff-Forschung stammte} und durch ein Leck (oder Schlimmeres)
freigesetzt wurde.

Das neue Coronavirus ist aufgrund seiner besonderen Eigenschaften --
insbesondere der sehr effizienten Nutzung des ACE2-Zellrezeptors --
hingegen ein gefährliches und leicht übertragbares SARS-Virus, das in
der Lunge, den Blutgefäßen und anderen Organen schwere Schäden anrichten
kann. Das große Glück ist, dass viele Menschen bereits über eine gewisse
Immunität gegen das neue Virus verfügen oder es bereits auf der
Schleimhaut abwehren können.

Covid-19 ist deshalb eine echte und ernsthafte Pandemie und am ehesten
vergleichbar mit den
(\href{https://swprs.files.wordpress.com/2020/06/covid-19-comparison-e1592927192181.png}{noch
stärkeren}) Grippe-Pandemien von 1957/58 (Asiatische Grippe) und 1968
bis 1970 (Hong Kong Grippe). Der Vergleich mit der Schweinegrippe von
2009 ist nur deshalb möglich, weil die Todesfälle durch die
Schweinegrippe damals und bis heute
\href{https://www.cbsnews.com/news/swine-flu-cases-overestimated/}{stark
übertrieben wurden}.

(Andererseits sei daran erinnert, dass während der Grippepandemie von
1968/1970 -- bzw. im Sommer zwischen den beiden Hauptwellen -- das
\href{https://nypost.com/2020/05/16/why-life-went-on-as-normal-during-the-killer-pandemic-of-1969/}{bekannte
Woodstock-Festival} stattfand und das gesellschaftliche Leben im
Allgemeinen normal weiterlief.)

Allerdings kann man argumentieren, dass auf die echte Covid19-Pandemie
durch die Massentests in der Allgemeinbevölkerung \emph{zusätzlich} noch
eine ``Test-Pandemie'' aufgesetzt wurde, die eine zusätzliche Panik und
sehr hohe Kosten verursacht.

Stanford-Professor Scott Atlas
\href{https://www.facebook.com/cnn/posts/10160799274796509}{argumentierte
bereits im Mai}, dass die Massentests in der Allgemeinbevölkerung wenig
bringen und die Tests stattdessen auf bedrohte Einrichtungen wie
Pflegeheime und Krankenhäuser (auch für Besucher) beschränkt werden
sollten.

Die tägliche Testerei ist auch deshalb nicht zielführend, weil das Virus
\href{https://swprs.org/studies-on-covid-19-lethality/}{laut
Antikörperstudien} ohnehin bereits viel weiter verbreitet ist als durch
PCR-Tests sichtbar wird und die Tests anfällig für falsch-positive (und
falsch-negative) Resultate sowie nicht-infektiöse Virenfragmente sind.

Eine
\href{https://infekt.ch/2020/04/corona-testen-testen-und-kein-ende/}{einfachere
Empfehlung} ist es, bei Krankheitssymptomen zuhause zu bleiben und bei
bestehenden Risiken mit einer ärztlich verordneten frühzeitigen
Behandlung zu beginnen.

Länder wie Japan, Schweden und Weißrussland haben gezeigt, dass sich die
Pandemie auch ohne Lockdown und ohne Massentests -- und im Falle
Schwedens und Weißrusslands auch ohne Masken -- kontrollieren lässt,
sofern die sensiblen Einrichtungen geschützt werden.

\textbf{Fazit}: Bei Covid-19 handelt es sich um eine echte und
ernsthafte Pandemie vergleichbar mit den (noch stärkeren)
Grippepandemien von 1957 und 1968. Zur echten Covid19-Pandemie kommt
jedoch noch eine ``Test-Pandemie'' hinzu, die zu zusätzlicher Panik und
hohen Kosten führt.

\includegraphics{https://swprs.files.wordpress.com/2020/07/global-cases-deaths.png?w=600\&h=434}

\hypertarget{zum-ursprung-des-neuen-coronavirus}{%
\subparagraph{\texorpdfstring{\textbf{Zum Ursprung des neuen
Coronavirus}}{Zum Ursprung des neuen Coronavirus}}\label{zum-ursprung-des-neuen-coronavirus}}

Der Ursprung des neuen SARS-Coronavirus ist weiterhin unklar. Allerdings
konnten Rechercheure mit Zugang zu chinesischen Dokumenten
\href{https://swprs.org/ursprung-des-covid-19-virus-die-mojiang-minenarbeiter-hypothese/}{im
Mai nachweisen}, dass das am nächsten verwandte Coronavirus aus einer
Bergbau-Mine in Südwest-China stammte, in der 2012 sechs Minenarbeiter
an einer Covid-ähnlichen Lungenentzündung erkrankten und drei von ihnen
starben.

Die Erkrankung der Minenarbeiter war mit dem heutigen (schweren)
Covid-19 klinisch sozusagen identisch, weshalb einige Analysten
inzwischen statt von Covid-19 von
\href{https://twitter.com/search?q=\%23covid12\&f=live}{\textbf{Covid-12}}sprechen.

Das Virologische Institut in Wuhan erhielt 2012 und 2013 Virenproben
sowohl aus der Mine als auch aus dem Gewebe der verstorbenen
Minenarbeiter. Es ist denkbar, dass diese im Herbst 2019 aus dem Labor
entwichen.

Neben dem chinesischen Institut
\href{https://armswatch.com/project-g-2101-pentagon-biolab-discovered-mers-and-sars-like-coronaviruses-in-bats/}{arbeiteten
jedoch auch} die amerikanische Gesundheits­behörde CDC und das
amerikanische Militär nachweislich mit SARS-artigen Viren aus
Fledermäusen. Die US-NGO ``Eco Health Alliance'' kooperierte zu diesem
Thema sowohl mit dem Pentagon als auch mit dem Virologischen Institut in
Wuhan.

Eine direkte Übertragung durch ein Tier ist ebenfalls weiterhin denkbar,
obschon die bisherigen Kandidaten wie der bekannte Tiermarkt in Wuhan
oder die Pangolin-Theorie von Fachleuten inzwischen
\href{https://www.news-medical.net/news/20200708/Research-sheds-doubt-on-the-Pangolin-link-to-SARS-CoV-2.aspx}{ausgeschlossen}
wurden.

\textbf{Weiterlesen}:
\href{https://swprs.org/ursprung-des-covid-19-virus-die-mojiang-minenarbeiter-hypothese/}{Ursprung
des Covid-19-Virus: Die Mojiang-Minenarbeiter-Hypothese}

\hypertarget{b-luxe4nder-und-regionen}{%
\subparagraph{\texorpdfstring{\textbf{B. Länder und
Regionen}}{B. Länder und Regionen}}\label{b-luxe4nder-und-regionen}}

\hypertarget{usa}{%
\subparagraph{\texorpdfstring{\textbf{USA}}{USA}}\label{usa}}

Die USA gehören zu den bisher am stärksten vom neuen Coronavirus
betroffenen Länder. Dies könnte politische und medizinische Gründe
haben.

Medizinisch bestehen in den USA viele relevante Vorerkrankungen wie
Fettleibigkeit, Herzprobleme und Diabetes. Klimaanlagen könnten eine
aerosolartige Übertragung in Innenräumen begünstigen. Politisch kam es
zu gravierenden Fehlern im Umgang mit Pflegeheimen, zu Fehlanreizen bei
der Behandlung von Patienten, und zu einem problematischen Hin-und-Her
mit Lockdowns.

\begin{itemize}
\tightlist
\item
  Die USA zählen bereits über 150,000 Corona-Todesfälle und befinden
  sich damit im Bereich einer pandemischen Influenza, vergleichbar mit
  den
  \href{https://www.visualcapitalist.com/history-of-pandemics-deadliest/}{Pandemien
  von 1957 und 1968}.
\item
  45\% der Corona-Todesfälle erfolgten
  \href{https://freopp.org/the-covid-19-nursing-home-crisis-by-the-numbers-3a47433c3f70}{in
  Pflegeheimen}. Über 50\% aller Todesfälle erfolgten
  in\href{https://twitter.com/pdubdev/status/1280195926238261248/photo/1}{jenen
  sechs Bundesstaaten}, die Covid-Patienten aktiv in Pflegeheimen
  unterbrachten.
\item
  Für Personen im Schul- und Arbeitsalter (bis 65) ist die
  Corona-Sterblichkeit laut CDC vergleichbar mit der Sterblichkeit
  \href{https://childrenshealthdefense.org/news/if-covid-fatalities-were-90-2-lower-how-would-you-feel-about-schools-reopening/}{durch
  andere Lungenentzündungen} (z.B. durch Influenza). Für Kinder und
  Jugendliche ist Covid dreimal weniger gefährlich als Influenza.
\item
  Eine
  \href{https://www.the-scientist.com/news-opinion/largest-seroprevalence-study-in-us-shows-vast-covid-19-undercount-67762}{landesweite
  Antikörperstudie} ergab, dass das neue Coronavirus je nach Region 6
  bis 24 mal weiter verbreitet ist als aufgrund von PCR-Tests vermutet.
  Die Antikörperwerte liegen allerdings in den meisten Regionen noch im
  einstelligen Prozentbereich, was auf eine Verbreitung des Coronavirus
  von weniger als 50\% hindeutet.
\item
  Während die Anzahl der täglichen positiven Tests aufgrund der hohen
  Anzahl an Tests Mitte Juli einen Höchststand erreichte, lag die Anzahl
  der \href{https://covidusa.net/}{täglichen Todesfälle} nur noch halb
  so hoch wie im April, zuletzt allerdings wieder mit leicht steigender
  Tendenz (siehe Grafik unten).
\item
  In Florida wurde von zeitweise angeblich sehr hohen Positivraten
  berichtet. Eine Analyse
  \href{https://www.fox35orlando.com/news/fox-35-investigates-florida-department-of-health-says-some-labs-have-not-reported-negative-covid-19-results}{zeigte
  jedoch}, dass verschiedene Labore nur noch die Anzahl positiver Tests
  und damit eine scheinbare Positivenrate von 100\% meldeten. Die
  tatsächliche Positivenrate lag in Florida zumeist im einstelligen
  Prozentbereich. Bei den Todesfällen pro Einwohner liegt Florida im
  Vergleich mit den anderen Bundesstaaten weiterhin im unteren
  Mittelfeld.
\item
  Das Median-Alter der Covid-Todesfälle liegt in den USA
  \href{https://twitter.com/kylamb8/status/1287952340822175744}{bei 78.5
  Jahren}. Das ist höher als das Median-Alter der übrigen Todesfälle,
  aber tiefer als das Median-Alter der Covid-Todesfälle in Europa
  (typischerweise 80 bis 86 Jahre).
\item
  Yale-Professor und Epidemiologe Harvey A. Risch forderte zuletzt eine
  \href{https://www.newsweek.com/key-defeating-covid-19-already-exists-we-need-start-using-it-opinion-1519535}{umgehende
  rezeptfreie Abgabe} von HCQ zur Frühbehandlung von Covid-Erkrankungen.
\item
  Eine Ärztegruppe namens ``America's Frontline Doctors'' hielt eine
  Pressekonferenz mit demselben Ziel. Das
  \href{https://omnij.org/AmericasFrontlineDoctors}{Video der
  Pressekonferenz} wurde innerhalb eines Tages von 20 Millionen Menschen
  gesehen, bevor es von Facebook \& Co. als ``Desinformation'' gelöscht
  wurde.
\end{itemize}

~

\includegraphics{https://swprs.files.wordpress.com/2020/07/us-age-adjusted-death-rate.png?w=736\&h=503}

\includegraphics{https://swprs.files.wordpress.com/2020/07/usa-daily-deaths.png?w=736\&h=209}

\includegraphics{https://swprs.files.wordpress.com/2020/07/new-york-vs-florida-deaths.png?w=736\&h=391}

\includegraphics{https://swprs.files.wordpress.com/2020/07/us-care-home-deaths-june-19.png?w=736}

\hypertarget{grouxdfbritannien}{%
\subparagraph{\texorpdfstring{\textbf{Großbritannien}}{Großbritannien}}\label{grouxdfbritannien}}

\begin{itemize}
\tightlist
\item
  In England und Wales kam es bisher zu ca. 50,000
  ``Corona-Todesfällen''. Die Gesamtsterblichkeit liegt damit um circa
  10,000 Todesfälle \href{http://inproportion2.talkigy.com/}{unter der
  starken Grippewelle} von 1999/2000.
\item
  Bei den unter 45-Jährigen kam es bisher zu
  \href{http://inproportion2.talkigy.com/}{keiner Übersterblichkeit} im
  Vergleich mit den letzten fünf Jahren.
\item
  Die kumulierten Corona-Todesfälle seit März entsprechen
  \href{http://inproportion2.talkigy.com/}{ziemlich genau} den
  kumulierten Influenza- und Pneumonie-Todesfällen seit Winterbeginn im
  Dezember 2019.
\item
  Seit Mitte Juni befinden sich England und Wales in einer
  \href{https://www.ons.gov.uk/peoplepopulationandcommunity/birthsdeathsandmarriages/deaths/bulletins/deathsregisteredweeklyinenglandandwalesprovisional/weekending17july2020}{relativen
  Untersterblichkeit} und die täglichen Corona-Todesfälle liegen seither
  unter den täglichen Influenza- und Pneumonie-Todesfällen.
\item
  Mitte April seien bereits
  \href{https://www.thesun.co.uk/news/12182503/nearly-half-nhs-workers-infected-coronavirus-peak-epidemic/}{45\%
  der NHS-Pflegemitarbeiter} mit Corona infiziert gewesen. Ein
  erheblicher Teil der Patienten könnte sich im Krankenhaus mit Corona
  angesteckt haben. Zudem wurden auch in England Corona-Patienten in
  Pflegeheime verlegt, was zu zusätzlichen Todesfällen führte.
\end{itemize}

\includegraphics{https://swprs.files.wordpress.com/2020/07/england-wales-2020-1999-comparison.png?w=736\&h=393}

\hypertarget{frankreich}{%
\subparagraph{\texorpdfstring{\textbf{Frankreich}}{Frankreich}}\label{frankreich}}

Frankreich ist von der Corona-Pandemie relativ stark betroffen und
registrierte bis Ende Mai
\href{https://www.santepubliquefrance.fr/maladies-et-traumatismes/maladies-et-infections-respiratoires/infection-a-coronavirus/documents/rapport-synthese/surveillance-de-la-mortalite-au-cours-de-l-epidemie-de-covid-19-du-2-mars-au-31-mai-2020-en-france}{laut
der Gesundheitsbehörde SPF} circa 30,000 Corona-Todesfälle. Rund 50\%
dieser Todesfälle ereigneten sich in Pflegeheimen, das
Durchschnittsalter der Todesfälle liegt bei 81.3 Jahren. Das Medianalter
der
\href{https://www.santepubliquefrance.fr/content/download/260891/2645733}{Intensivpatienten}
lag bei circa 67 Jahren.

Besonders stark betroffen waren die Region um Paris, Ostfrankreich und
Nordfrankreich, während weite Teile Westfrankreichs und
Südwest-Frankreichs bisher kaum betroffen waren.

Obschon also erst ein Teil Frankreichs von Covid betroffen war, liegt
die kumulierte Übersterblichkeit seit Jahresbeginn (im Vergleich zum
Basiswert) rund 50\% höher als während der saisonalen Grippewellen der
vergangen fünf Jahre. Im Großraum Paris liegt die Übersterblichkeit
sogar rund 500\% bzw. 10,000 Personen höher als in den vergangenen
Jahren (siehe Grafiken).

Die Covid-Todesfälle machten landesweit rund 16\% aller Todesfälle aus,
im Großraum Paris waren es jedoch knapp 40\% aller Todesfälle von Anfang
März bis Ende Mai. Die wöchentliche Spitzenmortalität durch Covid-19 ist
vergleichbar mit dem Hitzesommer von 2003 (siehe Grafik unten).

Der bekannte Medizinprofessor und HCQ-Pionier Didier Raoult aus
Marseille kritisierte
\href{http://covexit.com/shock-testimony-of-professor-raoult-at-french-national-assembly/}{bei
einer parlamentarischen Anhörung} Ende Juni die fehlende Frühbehandlung
und das Verbot von HCQ. Bis 2019 sei HCQ in Frankreich rezeptfrei
erhältlich gewesen. Im Rahmen der Pandemie sei die Verwendung jedoch auf
Kliniken eingeschränkt und schließlich
\href{https://www.france24.com/en/20200527-france-revokes-decree-authorising-use-of-hydroxychloroquine-to-treat-covid-19}{ganz
verboten} worden. Der Anlass für das Verbot war die
\href{https://omnij.org/Gilead:_Twenty-one_billion_reasons_to_discredit_hydroxychloroquine_(ORIGINAL_ARTICLE)}{gefälschte
Lancet-Studie} von Ende Mai (die später zurückgezogen wurde).

In seiner Klinik hatte Prof. Raoult die Case Fatality Rate durch eine
Frühbehandlung mit HCQ laut einer
\href{https://www.sciencedirect.com/science/article/pii/S1477893920302817}{publizierten
retrospektiven Analyse} auf sehr tiefe 0.9\% senken können.

\includegraphics{https://swprs.files.wordpress.com/2020/07/france-mortality-cumulative.png?w=736\&h=225}

\includegraphics{https://swprs.files.wordpress.com/2020/07/france-mortality-idf-grand-est.png?w=736\&h=216}

\includegraphics{https://swprs.files.wordpress.com/2020/07/france-mortality-2003.png?w=736\&h=401}

\includegraphics{https://swprs.files.wordpress.com/2020/07/france-mortality-map.png?w=736\&h=486}

\textbf{Grafiken und Bericht}:
\href{https://www.santepubliquefrance.fr/maladies-et-traumatismes/maladies-et-infections-respiratoires/infection-a-coronavirus/documents/rapport-synthese/surveillance-de-la-mortalite-au-cours-de-l-epidemie-de-covid-19-du-2-mars-au-31-mai-2020-en-france}{Santé
Public France}

\hypertarget{deutschland}{%
\subparagraph{\texorpdfstring{\textbf{Deutschland}}{Deutschland}}\label{deutschland}}

Deutschland zählt bisher nur circa 9000 Corona-Todesfälle und erlebte
bisher keine wesentliche Übersterblichkeit (bevölkerungsangepasst sogar
eine
\href{https://swprs.files.wordpress.com/2020/06/breyer-deutschland-untersterblichkeit.pdf}{leichte
Untersterblichkeit}).

Ende Juni hatten unter Blutspendern jedoch
\href{https://www.rki.de/DE/Content/InfAZ/N/Neuartiges_Coronavirus/Projekte_RKI/SeBluCo_Zwischenbericht.html}{lediglich
1.3\%} IgG-Antikörper gegen das neue Coronavirus. Dieser Wert ist sehr
gering. Selbst wenn man Nicht-Blutspender (inklusive Kinder und
erkrankte Personen), T-Zellen und mukosale (IgA) Immunität
berücksichtigt, dürfte sich eine Exposition der Bevölkerung von kaum
mehr als 10\% bis 15\% ergeben.

Dies bedeutet, dass sich das neue Coronavirus in Deutschland noch nicht
stark verbreitet hat. Die Maßnahmen oder -- wahrscheinlicher -- die
Antizipation der Maßnahmen durch die Bevölkerung scheinen in diesem
Sinne also erfolgreich gewesen zu sein (siehe Grafik unten).

Andererseits bedeutet dies, dass Deutschland epidemiologisch im
Wesentlichen noch dort steht, wo es im April war, und dass das Risiko
für eine neuen und stärkeren Anstieg der Infektionen und Erkrankungen
sehr real ist. Der Vergleich mit Frankreich zeigt, was das bedeuten
kann.

Die deutsche Regierung scheint derzeit auf eine Suppressions- und
Impfstoff-Strategie zu setzen. Diese Strategie ist gesellschaftlich und
ökonomisch aufwändig und ihr Erfolg bleibt ungewiss. Alternativ oder
ergänzend könnte ein
\href{https://swprs.org/zur-behandlung-von-covid-19/}{Frühbehandlungskonzept}
geprüft werden.

Die politische Corona-Situation in Deutschland bleibt angespannt.
Wiederholt kam es zu Sanktionen gegen Corona-kritische Ärzte,
Professoren, Juristen und Beamte sowie zu teilweise gravierenden
Übergriffen auf Corona-kritische Journalisten und Aktivisten.

Seit Juli befasst sich ein
\href{https://corona-ausschuss.de/}{Außer­parlamen­tarischer
Untersuchungs­­ausschuss} bestehend aus Anwälten und medizinischen
Fachleuten mit der deutschen Corona-Regierungspolitik. Dabei sollte
indes nicht vergessen werden, dass die Corona-Pandemie in Deutschland
angesichts von nur 1.3\% IgG-Antikörpern gegen das Virus vermutlich noch
keineswegs vorbei ist.

\includegraphics{https://swprs.files.wordpress.com/2020/07/cidm-de-effekte.jpg?w=736\&h=411}

\hypertarget{schweiz}{%
\subparagraph{\texorpdfstring{\textbf{Schweiz}}{Schweiz}}\label{schweiz}}

\begin{itemize}
\tightlist
\item
  Die Schweizer \textbf{Jahresübersterblichkeit} tendiert derzeit gegen
  null (siehe Grafik) und liegt damit unter den meisten Grippewellen der
  letzten zehn Jahre. Grund dafür ist der milde Winter und das sehr hohe
  Medianalter der rund 1700 Corona-Todesfälle (84 Jahre). Ca. 50\% der
  Todesfälle ereigneten sich in Pflegeheimen. Die Wirkung des Lockdowns
  bleibt fraglich.
\item
  In den ehemaligen \textbf{Hotspots Tessin und Genf} lagen die
  IgG-Antikörper-Werte im Mai
  \href{https://infekt.ch/2020/07/sind-es-doch-10x-mehr-covid-19-faelle/}{bei
  circa 10\%} und damit rund zehnmal höher als durch die täglichen
  PCR-Tests vermutet. Unter Berücksichtigung von
  \href{https://swprs.org/coronavirus-antibody-tests-show-only-one-fifth-of-infections/}{mukosaler
  und zellulärer Immunität} könnte die Exposition in der Südschweiz und
  Westschweiz bereits bei circa 50\% liegen. In der Deutschschweiz
  dürfte die Exposition hingegen geringer sein. Das Risiko einer
  ``zweiten Welle'' ist deshalb real.
\item
  Das \textbf{Kantonsspital Aargau}
  \href{https://www.aargauerzeitung.ch/aargau/kanton-aargau/viele-schwere-coronafaelle-was-das-kantonsspital-aarau-ueber-covid-19-herausgefunden-hat-138523777}{publizierte
  die Daten} zu den 99 bisher behandelten Covid-Patienten. Rund einer
  Drittel der hospitalisierten Patienten zeigte schwere Verläufe, 18
  Patienten verstarben (CFR von 18\%). Insgesamt 30 Patienten waren
  unter 60 Jahre alt, davon 7 ohne Vorerkrankung, davon waren 3 auf der
  Intensivstation (keine Todesfälle).
\item
  Die Schweizer Regierung setzt grundsätzlich auf eine
  \textbf{Suppressions- und Impfstoff-Strategie}, die sie durch weitere
  Maßnahmen wie Massentests, Kontaktverfolgung und eine teilweise
  Maskenpflicht ergänzt. Als Alternative brachte Infektiologe Dr. Pietro
  Vernazza eine
  \href{https://corona-transition.org/der-infektiologe-prof-pietro-vernazza-sieht-covid-19-im-bereich-einer}{kontrollierte
  Durchseuchung} mit Schutz der Risikogruppen nach schwedischem Vorbild
  ins Gespräch.
\item
  Die Schweiz verfügt weiterhin über keine
  \href{https://swprs.org/zur-behandlung-von-covid-19/}{Frühbehandlungsstrategie}
  und riskiert dadurch eine unnötig hohe Hospitalisierungs- und
  Sterberate.
\item
  Schweden und Weißrussland, die beide ohne Lockdown und ohne
  Maskenpflicht durch die Corona-Pandemie kamen,
  stehen\href{https://www.nau.ch/news/schweiz/42-lander-schweiz-hat-liste-der-corona-risikolander-erweitert-65747991}{seit
  Mitte Juli} nicht mehr auf der \textbf{BAG-Liste der
  ``Risikoländer''}. Schweden hatte zuvor die Schweiz auf die eigene
  Risikoliste gesetzt. Tatsächlich entstand der Anstieg der schwedischen
  ``Fälle'' durch einen Anstieg der Tests.
\item
  Gegen die Corona-Tracing-App ``\textbf{SwissCovid''} wurde
  \href{https://corona-transition.org/referendum-gegen-die-swisscovid-app-gestartet}{ein
  Referendum gestartet}. Die Initianten machen datenschutzrechtliche und
  sicherheitstechnische Bedenken geltend. Zuvor veröffentlichte der
  westschweizer Professor Serge Vaudenay
  \href{https://swprs.org/corona-app-ein-eklatanter-betrug/}{eine
  kritische Analyse} zur App: Diese sei nicht so transparent wie
  behauptet. Die Kontrolle liege bei Google und Apple.
\item
  Auch gegen das \textbf{``Covid-19-Gesetz''}, das das Corona-Notrecht
  bis Ende 2022 verlängert, ist
  \href{https://notrecht-referendum.ch/}{ein Referendum in
  Vorbereitung}. Zudem wurde \href{https://fruehling2020.com/}{eine
  Petition gestartet}, die eine außer­parla­mentarische
  Untersuchungs­kommission zu den Corona-Maßnahmen fordert.
\item
  Für Aufsehen sorgte zudem
  \href{https://www.20min.ch/story/bag-warnt-vor-fake-news-flyer-der-maskengegner-559612676323}{eine
  Flyer-Aktion} gegen die Maskenpflicht im öffentlichen Verkehr. Der
  BAG-Direktor bezeichnete die Argumente der Kritiker etwas vorschnell
  als ``Fake News''.
\item
  \textbf{InsideCorona}:
  \href{https://www.insidecorona.ch/2020/07/29/beispiellose-desinformation/}{Covid-19-Taskforce:
  Beispiellose Desinformation}
\item
  \textbf{Infosperber}:
  \href{https://www.infosperber.ch/Artikel/Gesundheit/Task-Force-tauscht-Offentlichkeit-uber-den-Nutzen-der-Masken}{Die
  Covid-19-Task Force übertrieb den Nutzen der Masken massiv}
\item
  Für weitere aktuelle und kritische \textbf{Corona-Analysen} siehe
  \href{https://corona-transition.org/}{corona-transition.org}
\end{itemize}

\includegraphics{https://swprs.files.wordpress.com/2020/07/schweiz-todesfaelle-2010-2020_woche_29.png?w=736\&h=322}

\hypertarget{schweden}{%
\subparagraph{\texorpdfstring{\textbf{Schweden}}{Schweden}}\label{schweden}}

\begin{itemize}
\tightlist
\item
  In Schweden liegen die täglichen Corona-Todesfälle inzwischen nahe bei
  null. Die \textbf{Gesamtsterblichkeit} liegt im Bereich früherer
  starker Grippewellen. Selbst die monatliche Spitzensterblichkeit (im
  April 2020) blieb
  \href{https://emanuelkarlsten.se/more-swedes-died-in-one-month-1993-and-2000-compared-to-april-2020-why/}{unter
  den starken Grippewellen} der 1990er Jahre.
\item
  Das Beispiel Schwedens (sowie Weißrusslands) zeigt, dass ein
  \textbf{Lockdown} bei guter Vorbereitung der Bevölkerung nicht
  erforderlich war. Aus Sicht vieler Lockdown-Befürworter -- Regierungen
  und Medien -- ist dies allerdings
  \href{https://emanuelkarlsten.se/multiple-errors-in-the-new-york-times-article-about-swedens-corona-strategy/}{sehr
  schwierig einzugestehen}.
\item
  Schweden hat als eines der wenigen westlichen Länder auf Basis der
  medizinischen Evidenz die \textbf{Grundschulen} nicht geschlossen.
  Auch diese Entscheidung
  \href{https://www.reuters.com/article/us-health-coronavirus-sweden-schools-idUSKCN24G2IS}{war
  richtig}.
\item
  Schweden machte \textbf{zwei wirkliche Fehler}, die von den meisten
  Medien ironischerweise nicht thematisiert werden: 1) Die
  \emph{Pflegeheime} in der Region Stockholm wurden zu spät geschützt
  und verursachten über 50\% der schwedischen Todesfälle. 2) Schweden
  hatte keine \emph{Früh­behand­lungs­strategie}, mit der die
  Hospitalisierungs- und Sterberate hätte gesenkt werden können.
\item
  Schwedische Städte zeigten im Juli eine
  \textbf{IgG-Antikörper-Prävalenz}\href{https://www.sll.se/verksamhet/halsa-och-vard/nyheter-halsa-och-vard/2020/07/20-juli-lagesrapport-om-arbetet-med-det-nya-coronaviruset/}{zwischen
  10\% und 20\%}, was zusammen mit mukosaler und zellulärer Immunität
  auf eine Exposition der Bevölkerung zwischen 50\% und 100\% hindeutet.
  Schweden dürfte damit von allen westlichen Ländern vermutlich die
  beste Ausgangslage für den kommenden Winter haben.
\end{itemize}

Die folgenden Grafiken vergleichen die Todesfälle in Schweden mit
England und New York.

\includegraphics{https://swprs.files.wordpress.com/2020/07/sweden-vs-england-deaths.png?w=736\&h=354}

\includegraphics{https://swprs.files.wordpress.com/2020/07/nys-vs-sweden-deaths.png?w=736\&h=365}

Grafiken:
\href{https://twitter.com/pwyowell/status/1283504629393895425}{Paul
Yowell}

\hypertarget{indien}{%
\subparagraph{\texorpdfstring{\textbf{Indien}}{Indien}}\label{indien}}

Indien, das auf Frühbehandlung und
\href{https://health.economictimes.indiatimes.com/news/diagnostics/hcq-beneficial-as-preventive-drug-sms-doctors-told-icmr/76464620}{sogar
Prophylaxe} mit dem Malariamittel HCQ setzt, zählt bisher bei 1.3
Milliarden Einwohnern offiziell lediglich
\href{https://en.wikipedia.org/wiki/COVID-19_pandemic_in_India}{circa
35,000} Corona-Todesfälle.

Eine indische Antikörperstudie kam
\href{https://www.telegraphindia.com/india/covid-infected-and-cured-undetected-in-delhi/cid/1787002}{zum
Ergebnis}, dass rund 23\% der 20 Millionen Bewohner der indischen
Hauptstadt Delhi bereits über Antikörper gegen das neue Coronavirus
verfügen. Dies sind rund 35 mal mehr Personen als durch PCR-Tests
bestätigt.

Damit könnte sich Delhi (und einige andere Städte), unter
Berücksichtigung von mukosaler und zellulärer Immunität, bereits im
Bereich oder in der Nähe der Herdenimmunität befinden.

\hypertarget{lateinamerika}{%
\subparagraph{\texorpdfstring{\textbf{Lateinamerika}}{Lateinamerika}}\label{lateinamerika}}

Brasilien liegt derzeit mit 90,000 Todesfällen auf die Bevölkerung
bezogen zwischen den Niederlanden und Frankreich. Inzwischen hat
Brasilien ein Frühbehandlungskonzept mit Zink und HCQ eingeführt.

Eine noch höhere Todesrate (bezogen auf die Bevölkerung) weisen derzeit
Chile und Peru auf. Peru liegt mit knapp 20,000 Todesfällen im Bereich
von Italien und Spanien.

\hypertarget{c-weitere-meldungen}, der höchste Wert seit 1947. Der zweithöchste Rückgang war
  1958 mit 10\% -- im Zuge der Asiatischen Grippepandemie.
\item
  In den USA könnten aufgrund der Corona-Lockdowns
  \href{https://www.cnbc.com/2020/07/10/looming-evictions-may-soon-make-28-million-homeless-expert-says.html}{bis
  zu 28 Millionen Menschen} ihr Heim verlieren und obdachlos werden, was
  eine neue Hypothekenkrise auslösen könnte.
\item
  Die deutsche Wirtschaft schrumpfte im zweiten Quartal
  \href{https://www.marketwatch.com/story/german-gdp-slumps-by-most-since-1970-in-the-second-quarter-11596096434}{um
  10.1\%} im Vergleich um Vorjahresquartal -- der größte Rückgang seit
  1970.
\item
  Laut UNO könnten die Corona-Lockdowns und die globale
  Wirtschaftsdepression bis Ende Jahr welweit
  \href{https://www.telegraph.co.uk/global-health/climate-and-people/coronavirus-set-trigger-brutal-tragedies-fragile-countries-un/}{bis
  zu 225 Millionen Menschen} in eine Hungersnot stürzen.
\item
  Die EU-Kommission fordert oder plant
  \href{https://corona-transition.org/eu-kommission-will-alle-nationalen-corona-warn-apps-vernetzen}{die
  ``Vernetzung''} der nationalen Corona-Apps.
\item
  Die NGO Privacy International warnt derweil~ vor einem
  \href{https://privacyinternational.org/long-read/4074/looming-disaster-immunity-passports-and-digital-identity}{``drohenden
  Desaster''} durch Immunitätspässe und digitale Identitätskarten.
\item
  In Turkmenistan verbot die autoritäre Regierung laut ``Reporter ohne
  Grenzen'' offenbar
  \href{https://www.npr.org/sections/coronavirus-live-updates/2020/03/31/824611607/turkmenistan-has-banned-use-of-the-word-coronavirus}{die
  Benutzung des Wortes} ``Coronavirus''. Corona-Todesfälle gebe es dort,
  folglich, offiziell auch keine. Wer eine Maske trägt, werde von der
  Polizei festgenommen.
\end{itemize}

\hypertarget{juli-2020}{%
\paragraph{Juli 2020}\label{juli-2020}}

\hypertarget{zur-entwicklung-der-pandemie}{%
\subparagraph{**Zur Entwicklung der
Pandemie}\label{zur-entwicklung-der-pandemie}}

**

In den meisten westlichen Ländern war der Höhepunkt der
Coronavirus-Infektionen bereits im März oder April und oftmals noch vor
dem Lockdown erreicht. Der Höhepunkt der Todesfälle lag in den meisten
westlichen Ländern im April. Seither gehen die Hospitalisierungen und
Todesfälle in den meisten westlichen Ländern wieder zurück (siehe
Grafiken unten).

Diese Entwicklung gilt auch für Länder ohne Lockdown, wie z.B. Schweden,
Weißrussland und Japan. Die kumulierte Jahressterblichkeit liegt in den
meisten westlichen Ländern weiterhin
\href{https://swprs.org/studies-on-covid-19-lethality/\#overall-mortality}{im
Bereich} einer milden (z.B. CH, AT, DE) bis starken (z.B. USA, UK)
Grippewelle.

Nach dem Ende des Lockdowns wurde die Anzahl der Corona-Tests in der
risikoschwachen Allgemeinbevölkerung in vielen Ländern stark erhöht,
beispielsweise im Zusammenhang mit der Rückkehr der Menschen an die
Arbeitsplätze und in die Schulen.

Dies führte in einigen Ländern oder Regionen zu einem gewissen Anstieg
der positiven Testresultate, was von vielen Medien und Behörden als ein
gefährlicher Anstieg der ``Fallzahlen''
\href{https://www.infosperber.ch/Artikel/Gesundheit/Corona-Viele-grosse-Medien-ubertreiben-noch-immer-massiv}{dargestellt
wurde} und teilweise zu
\href{https://www.zeitpunkt.ch/index.php/die-zweite-welle-keine-wissenschaft-kaum-kranke-aber-mehr-einschraenkungen}{neuen
Restriktionen} führte, selbst wenn die Positivenrate sehr tief blieb.

Die ``Fallzahlen'' sind indes eine
\href{https://swprs.org/corona-medien-propaganda/}{irreführende Größe}
und nicht mit Erkrankten oder Infizierten gleichzusetzen. Bei einem
positiven Test kann es sich etwa um nicht-infektiöse Virenfragmente,
Mehrfachtests, eine asymptomatische Infektion, oder um ein
falsch-positives Resultat handeln.

Das Zählen von angeblichen ``Fallzahlen'' ist auch deshalb nicht
zielführend, da Anti­körper­studien und immunologische Untersuchungen
ohnehin \href{https://swprs.org/studies-on-covid-19-lethality/}{längst
gezeigt haben}, dass das Coronavirus bis zu fünfzigmal weiter verbreitet
ist als aufgrund der täglichen PCR-Tests angenommen wird.

Entscheidend sind vielmehr die Erkrankungen, die Hospitalisierungen, und
die Todesfälle. Bei den Hospitalisierungen ist zu beachten, dass viele
Kliniken inzwischen wieder im Normalbetrieb sind und alle Patienten,
auch die asymptomatischen, zusätzlich auf Coronaviren testen.
Entscheidend ist deshalb auch hier die Anzahl der tatsächlichen
Covid-Patienten.

Im Falle Schwedens musste die WHO z.B. die Einstufung als ``Risikoland''
\href{https://www.n-tv.de/panorama/Langfristig-koennte-Schweden-richtig-liegen-article21876864.html}{zurücknehmen},
nachdem klar wurde, dass die scheinbare Zunahme der ``Fälle'' auf einer
Zunahme der Tests beruhte. Tatsächlich sind die Hospitalisierungen und
Todesfälle in Schweden
\href{https://swprs.files.wordpress.com/2020/07/sweden-icu-deaths-june-28.png}{seit
April rückläufig}.

Mehrere Länder befinden sich seit Mai sogar in einer
\href{https://www.euromomo.eu/graphs-and-maps/\#z-scores-by-country}{relativen
Untersterblichkeit}. Der Grund dafür ist, dass das Medianalter der
Corona-Todesfälle oftmals über der durchschnittlichen Lebenserwartung
lag. Bis zu 80\% der Todesfälle ereigneten sich
\href{https://swprs.org/studies-on-covid-19-lethality/\#care-homes}{in
Pflegeeinrichtungen}.

In Ländern und Regionen, in denen die Ausbreitung des Coronavirus bisher
stark reduziert wurde, ist es dennoch absolut möglich, dass es zu einem
erneuten Anstieg an Erkrankungen kommt. In diesen Fällen ist eine
frühzeitige und wirkungsvolle Behandlung wichtig (siehe unten).

Die globale Covid-19-Mortalität liegt derzeit -- trotz der heute
deutlich älteren Bevölkerung -- eine ganze Größen­ordnung unter den
Grippepandemien von 1957 (asiatische Grippe) und 1968 (Hongkong-Grippe)
und
\href{https://swprs.files.wordpress.com/2020/06/covid-19-comparison-e1592927192181.png}{im
Bereich} der eher milden ``Schweinegrippe-Pandemie'' von 2009.

\hypertarget{die-folgenden-grafiken-illustrieren-die-diskrepanz-zwischen-fuxe4llen-und-todesfuxe4llen}{%
\subparagraph{Die folgenden Grafiken illustrieren die Diskrepanz
zwischen ``Fällen'' und
Todesfällen:}\label{die-folgenden-grafiken-illustrieren-die-diskrepanz-zwischen-fuxe4llen-und-todesfuxe4llen}}

\href{https://swprs.files.wordpress.com/2020/07/global-cases-deaths.png}{}

\includegraphics{https://swprs.files.wordpress.com/2020/07/global-cases-deaths.png?w=435\&h=315}

Worlwide ``cases'' versus deaths

\href{https://swprs.files.wordpress.com/2020/07/us-cases-deaths-july.jpg}{}

\includegraphics{https://swprs.files.wordpress.com/2020/07/us-cases-deaths-july.jpg?w=293\&h=156}

USA: ``cases'' vs. deaths

\href{https://swprs.files.wordpress.com/2020/07/florida-cases-deaths.jpg}{}

\includegraphics{https://swprs.files.wordpress.com/2020/07/florida-cases-deaths.jpg?w=293\&h=155}

Florida: ``cases'' vs. deaths

\href{https://swprs.files.wordpress.com/2020/07/sweden-cases-icu.png}{}

\includegraphics{https://swprs.files.wordpress.com/2020/07/sweden-cases-icu.png?w=732\&h=401}

Sweden: Reported ``cases'', adjusted for testing, and ICU usage

\hypertarget{die-folgenden-grafiken-vergleichen-covid-19-mit-fruxfcheren-grippewellen-mehr}{%
\subparagraph{\texorpdfstring{Die folgenden Grafiken vergleichen
Covid-19 mit früheren Grippewellen
(\href{https://swprs.org/studies-on-covid-19-lethality/\#overall-mortality}{mehr}):}{Die folgenden Grafiken vergleichen Covid-19 mit früheren Grippewellen (mehr):}}\label{die-folgenden-grafiken-vergleichen-covid-19-mit-fruxfcheren-grippewellen-mehr}}

\href{https://swprs.files.wordpress.com/2020/06/sweden-all-cause-nov-may-1990.jpg}{}

\includegraphics{https://swprs.files.wordpress.com/2020/06/sweden-all-cause-nov-may-1990.jpg?w=340\&h=231}

Sweden: All-cause mortality (Nov. to May) since 1990

\href{https://swprs.files.wordpress.com/2020/06/uk-flu-comparison.png}{}

\includegraphics{https://swprs.files.wordpress.com/2020/06/uk-flu-comparison.png?w=388\&h=231}

UK: Mortality 2020 (shifted) vs. 1999 and 2000

\href{https://swprs.files.wordpress.com/2020/07/schweiz-todesfaelle-2010-2020_woche_25.png}{}

\includegraphics{https://swprs.files.wordpress.com/2020/07/schweiz-todesfaelle-2010-2020_woche_25.png?w=405\&h=182}

Switzerland: Cumulative mortality vs. expectation value (2010-2020)

\href{https://swprs.files.wordpress.com/2020/06/sterbefallzahlen-de-25-05.png}{}

\includegraphics{https://swprs.files.wordpress.com/2020/06/sterbefallzahlen-de-25-05.png?w=323\&h=182}

Germany: Mortality (2017 to 2020)

\hypertarget{die-folgende-grafik-vergleicht-schweden-kein-lockdown-mit-new-york-state}{%
\subparagraph{Die folgende Grafik vergleicht Schweden (kein Lockdown)
mit New York
State:}\label{die-folgende-grafik-vergleicht-schweden-kein-lockdown-mit-new-york-state}}

\includegraphics{https://swprs.files.wordpress.com/2020/07/nys-vs-sweden-deaths.png?w=736\&h=365}

\hypertarget{die-folgende-grafik-vergleicht-covid-19-mit-fruxfcheren-pandemien}{%
\subparagraph{Die folgende Grafik vergleicht Covid-19 mit früheren
Pandemien:}\label{die-folgende-grafik-vergleicht-covid-19-mit-fruxfcheren-pandemien}}

\includegraphics{https://swprs.files.wordpress.com/2020/06/covid-19-comparison-e1592927192181.png?w=700\&h=571}

\hypertarget{zur-letalituxe4t-von-covid-19}{%
\subparagraph{\texorpdfstring{\textbf{Zur Letalität von
Covid-19}}{Zur Letalität von Covid-19}}\label{zur-letalituxe4t-von-covid-19}}

Die meisten Antikörperstudien ergaben eine bevölkerungsbasierte
Infection Fatality Rate (IFR) zwischen
\href{https://swprs.org/studies-on-covid-19-lethality/}{0.1\% und
0.3\%}. Die US-Gesundheitsbehörde CDC publizierte im Mai eine immer noch
vorsichtige
\href{https://reason.com/2020/05/24/the-cdcs-new-best-estimate-implies-a-covid-19-infection-fatality-rate-below-0-3/}{``beste
Schätzung''} von 0.26\% (basierend auf 35\% asymptomatischen Fällen).

Ende Mai erschien indes eine immunologische Studie der Universität
Zürich, die erstmals nachwies, dass die üblichen Antikörper-Tests, die
Antikörper im Blut messen (IgG und IgM), \textbf{höchstens ca. ein
Fünftel} aller Coronavirus-Infektionen
\href{https://swprs.org/coronavirus-antibody-tests-show-only-one-fifth-of-infections/}{erkennen
können}.

Der Grund dafür ist, dass das neue Coronavirus bei den meisten Menschen
bereits durch Antikörper auf der Schleimhaut (IgA) oder durch eine
zelluläre Immunität (T-Zellen) neutralisiert wird und sich dabei keine
oder nur milde Symptome ausbilden.

Dies bedeutet, dass das neue Coronavirus vermutlich viel weiter
verbreitet ist als bisher angenommen und die Letalität pro Infektion
rund fünfmal niedriger liegt als bisher vermutet. Die wirkliche
Letalität könnte somit \textbf{deutlich unter 0.1\%} und damit im
Bereich der Influenza liegen.

Die Schweizer Studie dürfte zugleich erklären, warum Kinder im
Normalfall gar nicht oder nur mild am neuen Coronavirus erkranken
(aufgrund des häufigen Kontakts mit bisherigen Corona-Erkältungsviren),
und warum selbst Hotspots wie New York City eine Antikörper-Verbreitung
(IgG/IgM) von höchstens 20\% fanden -- denn dies
\href{https://theconversation.com/coronavirus-could-it-be-burning-out-after-20-of-a-population-is-infected-141584}{entspricht
bereits} der Herdenimmunität.

Die Schweizer Studie wurde inzwischen von weiteren Studien bestätigt:

\begin{enumerate}
\def\labelenumi{\arabic{enumi}.}
\tightlist
\item
  Eine
  \href{https://news.ki.se/immunity-to-covid-19-is-probably-higher-than-tests-have-shown}{schwedische
  Studie} ergab, dass Personen mit milder oder asymptomatischer
  Erkrankung das Virus oftmals mit T-Zellen neutralisieren, ohne
  Antikörper ausbilden zu müssen. Die T-Zellen-Immunität war etwa
  doppelt so häufig wie die Antikörper-Immunität.
\item
  Eine umfangreiche
  \href{https://www.thelancet.com/journals/lancet/article/PIIS0140-6736(20)31483-5/fulltext}{spanische
  Studie} ergab, dass weniger als 20\% der~ symptomatischen Personen und
  ca. 2\% der getesteten asymptomatischen Personen IgG-Antikörper
  hatten.
\item
  Eine
  \href{https://www.researchsquare.com/article/rs-35331/v1}{deutsche
  Studie} (Preprint) ergab, dass 81\% der Personen, die noch
  \emph{keinen} Kontakt mit dem neuen Coronavirus hatten, bereits über
  kreuzreaktive T-Zellen und damit über eine gewisse
  Hintergrundimmunität verfügen. Der Grund dafür dürfte der Kontakt mit
  bisherigen Coronaviren (Erkältungsviren) sein.
\item
  Eine
  \href{https://www.nature.com/articles/s41591-020-0965-6}{chinesische
  Studie} im Fachmagazin Nature ergab, dass bei 40\% der
  asymptomatischen Personen und bei 12.9\% der symptomatischen Personen
  nach der Erholungsphase keine IgG-Antikörper mehr nachweisbar sind.
\item
  Eine
  \href{https://www.medrxiv.org/content/10.1101/2020.06.13.20130252v1}{weitere
  chinesische Studie} mit knapp 25,000 Klinikmitarbeitern in Wuhan
  ergab, dass höchstens ein Fünftel der vermutlich infizierten
  Mitarbeiter IgG-Antikörper aufwiesen.
\item
  Eine
  \href{https://www.medrxiv.org/content/10.1101/2020.06.21.20132449v1}{kleine
  französische Studie} (Preprint) ergab, dass sechs Familienmitglieder
  von Covid-Patienten eine T-Zellen-Immunität ohne Antikörper
  entwickelten.
\end{enumerate}

\textbf{Video-Interview}:
\href{https://www.youtube.com/watch?v=CwQpg62Kflg}{Swedish Doctor:
T-cell immunity and the truth about Covid-19 in Sweden}

In diesem Zusammenhang kam eine US-Studie im Fachjournal Science
Translational Medicine anhand verschiedener Indikatoren
\href{https://news.psu.edu/story/623797/2020/06/22/research/initial-covid-19-infection-rate-may-be-80-times-greater-originally}{zum
Ergebnis}, dass die Letalität von Covid-19 weit tiefer liegt als
ursprünglich angenommen, seine Ausbreitung in einigen Hotspots aber bis
zu 80-mal schneller erfolgte als vermutet, was den raschen Anstieg an
Erkrankungen erklären würde.

Eine Untersuchung im österreichischen \textbf{Skiort Ischgl}, einem der
ersten europäischen ``Corona-Hotspots'', fand Antikörper
\href{https://www.derstandard.at/story/2000118306133/42-4-prozent-der-bewohner-ischgls-haben-antikoerper-gegen-sars}{bei
42\% der Bevölkerung}. 85\% der Infektionen blieben ``unbemerkt'' (d.h.
sehr mild), ca. 50\% der Infektionen verliefen ganz ohne (spürbare)
Symptome.

Der hohe Antikörperwert von 42\% in Ischgl ergab sich, weil in Ischgl
auch auf IgA-Antikörper im Blut getestet wurde (statt nur auf IgM/IgG),
und dies relativ zeitnah bereits im April. Wäre zusätzlich auf mukosale
IgA und auf T-Zellen getestet worden, hätte sich zweifellos eine
nochmals deutlich höhere Immunität im Bereich der Herdenimmunität
ergeben.

In Ischgl kam es zu zwei Corona-Todesfällen (beides vorerkrankte Männer
über 80 Jahren), was einer rohen IFR von 0.26\% entspricht. Angepasst an
die Gesamtbevölkerung und die tatsächliche Immunität dürfte die
Covid-Letalität auch in Ischgl bei unter 0.1\% liegen.

Aufgrund der eher geringen Letalität fällt Covid-19 höchstens in die
Stufe 2 des von den US-Gesundheitsbehörden entwickelten fünfstufigen
\href{https://www.cidrap.umn.edu/news-perspective/2007/02/hhs-ties-pandemic-mitigation-advice-severity}{Pandemie-Plans}.
Für diese Stufe ist lediglich die \textbf{``freiwillige Isolierung
kranker Personen''} als Hauptmaßnahme vorgesehen. Weitergehende
Maßnahmen wie Mundschutzpflicht, Schulschließungen, Abstandsregeln,
Kontaktverfolgung, Impfungen und Lockdowns ganzer Gesellschaften sind
hingegen nicht angezeigt.

Die neuen immunologischen Resultate bedeuten zudem, dass
\href{https://www.nature.com/articles/d41586-020-01451-0}{``Immunitätsausweise''}
und Massenimpfungen nicht funktionieren können und mithin keine
sinnvollen Strategien sind.

Einige Medien sprechen weiterhin von angeblich viel höheren
Covid-Letalitätswerten. Diese Medien beziehen sich jedoch auf veraltete
Simulationsmodelle, verwechseln Mortalität und Letalität, oder CFR und
IFR, oder ``rohe IFR'' und bevölkerungsbasierte IFR.
\href{https://swprs.org/covid-19-letalitat-wie-man-es-nicht-macht/}{Mehr
zu diesen Fehlern hier.}

Im Juli wurde von einer angeblichen Antikörper-Verbreitung von
\href{https://www.nytimes.com/2020/07/09/nyregion/nyc-coronavirus-antibodies.html}{``bis
zu 70\%''} in einigen Stadtteilen New Yorks berichtet. Dabei handelt es
sich jedoch nicht um einen bevölkerungs­basierten Wert, sondern um
Antikörper bei Menschen, die eine Notfallstation aufgesucht hatten.

Die folgende Grafik zeigt die tatsächliche Entwicklung der
Corona-Todesfälle in Schweden (kein Lockdown, keine Maskenpflicht) im
Vergleich mit den Prognosen des Imperial College London (orange: keine
Maßnahmen; grau: moderate Maßnahmen). Die schwedische
Jahres­gesamt­sterblichkeit liegt im Bereich einer
\href{https://swprs.files.wordpress.com/2020/06/sweden-all-cause-nov-may-1990.jpg}{mittleren
Grippewelle} und 3.6\% unter den Vorjahren.

\includegraphics{https://swprs.files.wordpress.com/2020/07/sweden-projection-reality-june-28.png?w=736\&h=496}

\hypertarget{zu-den-gesundheitsrisiken-durch-covid-19}{%
\subparagraph{\texorpdfstring{\textbf{Zu den Gesundheitsrisiken durch
Covid-19}}{Zu den Gesundheitsrisiken durch Covid-19}}\label{zu-den-gesundheitsrisiken-durch-covid-19}}

Warum ist das neue Coronavirus für viele Menschen ungefährlich, für
einige Menschen aber sehr gefährlich? Der Grund liegt bei Besonderheiten
des Virus und des Immunsystems.

Viele Menschen, darunter fast alle Kinder, können das neue Coronavirus
mit einer bestehenden Immunität (durch den Kontakt mit früheren
Corona-Erkältungsviren) oder durch Antikörper bereits auf der
Schleimhaut (IgA) neutralisieren, ohne dass es viel Schaden anrichten
kann.

Gelingt das jedoch nicht, kann das Virus in den Organismus eindringen.
Dort kann das Virus aufgrund seiner effizienten Nutzung des
ACE2-Zellrezeptors zu Komplikationen in der Lunge (Pneumonie), den
Gefäßen (Thrombosen, Embolien), und weiteren Organen führen.

Reagiert das Immunsystem in diesem Fall zu schwach (bei älteren
Menschen) oder zu stark (bei einigen jüngeren Menschen), kann es zu
einem
\href{https://www.hollywoodreporter.com/news/nick-cordero-dead-bullets-broadway-waitress-actor-was-41-1301841}{kritischen
Krankheitsverlauf} kommen.

Es ist auch zutreffend, dass die Symptome oder Komplikationen einer
ernsthaften Covid-19-Erkrankungen in einigen Fällen während Wochen oder
sogar Monaten
\href{https://www.theatlantic.com/health/archive/2020/06/covid-19-coronavirus-longterm-symptoms-months/612679/}{anhalten
können}.

Deshalb ist das neue Coronavirus keinesfalls zu unterschätzen und eine
\emph{frühzeitige} und wirkungsvolle Behandlung bei Risikopatienten
absolut entscheidend.

Längerfristig könnte sich das neue Coronavirus zu einem typischen
Erkältungsvirus entwickeln, ähnlich dem
\href{https://en.wikipedia.org/wiki/Human_coronavirus_NL63}{Coronavirus
NL63}, das ebenfalls den ACE2-Zellrezeptor nutzt und heutzutage
Atemwegs- und Lungenentzündungen vor allem bei Kleinkindern und
Pflegepatienten auslöst.

\hypertarget{zur-behandlung-von-covid-19-1}{%
\subparagraph{\texorpdfstring{\textbf{Zur Behandlung von
Covid-19}}{Zur Behandlung von Covid-19}}\label{zur-behandlung-von-covid-19-1}}

\textbf{Hinweis}: Patienten wenden sich an einen Arzt.

Mehrere Studien haben inzwischen belegt, was einige behandelnde Ärzte
bereits seit März festgestellt haben: Eine frühzeitige Behandlung von
Covid-Patienten mit Zink und dem Malariamittel Hydroxychloroquin (HCQ)
ist \href{https://swprs.org/zur-behandlung-von-covid-19/}{tatsächlich
wirkungsvoll}.

US-Ärzte berichten von einer Reduktion der
Hospitalisierungsrate\href{https://www.preprints.org/manuscript/202007.0025/v1}{um
bis zu 84\%} und von einer Stabilisierung des Gesundheits­zustandes
oftmals \href{https://www.youtube.com/watch?v=eVs_EWVCVPc}{innerhalb von
wenigen Stunden}.

Zink besitzt antivirale Eigenschaften, HCQ unterstützt die Zinkaufnahme
und besitzt zusätzliche antivirale Eigenschaften. Diese Medikamente
werden von Ärzten bei Bedarf ergänzt durch ein Antibiotikum (zur
Verhinderung einer bakteriellen Superinfektion) und durch ein
Blutverdünnungsmittel (zur Verhinderung infektionsbedinger Thrombosen
und Embolien).

Die angeblich oder tatsächlich negativen Resultate mit HCQ im Rahmen
einiger Studien beruhten nach heutigem Wissensstand auf einem
verspäteten \href{https://c19study.com/}{Einsatz}, stark überhöhten
\href{http://www.francesoir.fr/politique-monde/oxford-recovery-et-solidarity-overdosage-two-clinical-trials-acts-considered}{Dosen}
(bis 2400mg/T), manipulierten
\href{https://www.theguardian.com/world/2020/jun/03/covid-19-surgisphere-who-world-health-organization-hydroxychloroquine}{Datensätzen},
oder
\href{https://www.iss.it/en/rapporti-covid-19/-/asset_publisher/btw1J82wtYzH/content/id/5334891}{Kontraindikationen}
(z.B. Favismus).

Die WHO, viele Medien und einige Behörden könnten durch ihr ablehnendes
Verhalten, das vielleicht politisch motiviert oder durch pharmazeutische
Interessen beeinflusst war, in den vergangenen Monaten erheblichen und
unnötigen
\href{https://www.youtube.com/watch?v=UIDsKdeFOmQ}{gesundheitlichen
Schaden} angerichtet haben.

So geht der französische Medizinprofessor Jaouad Zemmouri davon aus,
dass Europa mit einer konsequenten HCQ-Behandlungsstrategie bis zu 78\%
der Covid-Todesfälle
\href{https://www.moroccoworldnews.com/2020/06/306587/moroccan-scientist-moroccos-chloroquine-success-reveals-european-failures/}{hätte
vermeiden können}.\\

HCQ-Kontraindikationen wie zum Beispiel Favismus oder Herzprobleme
müssen beachtet werden, aber die neue
\href{https://www.henryford.com/news/2020/07/hydro-treatment-study}{Ford-Studie}
erreichte auch mit 56\% afroamerikanischen Patienten (die häufiger
Favismus aufweisen) eine Reduktion der Sterblichkeit in Kliniken um rund
50\%.

Der entscheidende Punkt bei der Behandlung von Risikopatienten ist
jedoch die \textbf{frühzeitige Intervention} bereits bei Entwicklung der
ersten typischen Symptome, um eine Progression der Erkrankung zu
verhindern und eine intensivmedizinische Hospitalisierung zu vermeiden.

Die meisten Länder reagierten auch in diesem Punkt unglücklich: Nach der
Infektionswelle im März kam ein Lockdown, sodass die bereits infizierten
und verängstigten Menschen ohne Behandlung bei sich zuhause
eingeschlossen waren und oftmals solange warteten, bis sie eine schwere
Atemnot entwickelten und direkt auf die Intensivstation gebracht werden
mussten, wo sie dann oftmals noch sediert und intubiert wurden und mit
hoher Wahrscheinlichkeit starben.

Es ist denkbar, dass ein Zink-HCQ-Protokoll, das einfach, sicher und
kostengünstig ist, komplexere Medikamente, Impfungen und Maßnahmen
weitgehend obsolet machen könnte.

Zuletzt zeigte eine
\href{https://www.sciencedirect.com/science/article/pii/S1201971220305282}{Fallstudie
aus Frankreich}, dass bei vier der ersten fünf Patienten, die mit dem
teuren Medikament Remdesivir des Pharmakonzerns Gilead behandelt wurden,
die Behandlung wegen Leberproblemen und Nierenversagen abgebrochen
werden musste.

\textbf{Mehr dazu}:
\href{https://swprs.org/zur-behandlung-von-covid-19/}{Zur Behandlung von
Covid-19}

\hypertarget{zur-wirksamkeit-von-masken-1}{%
\subparagraph{\texorpdfstring{\textbf{Zur Wirksamkeit von
Masken}}{Zur Wirksamkeit von Masken}}\label{zur-wirksamkeit-von-masken-1}}

Verschiedene Länder haben eine Maskenpflicht im öffentlichen Verkehr, im
Detailhandel oder allgemein in der Öffentlichkeit eingeführt oder
diskutieren diese aktuell.

Einige mögen argumentieren, dass sich die Diskussion aufgrund der
deutlich geringeren Letalität und
\href{https://swprs.org/studies-on-covid-19-lethality/\#hospitalizations}{Hospitalisierungsrate}
von Covid-19 und der Behandlungs­möglichkeiten bereits erübrigt hat, da
das ursprüngliche Ziel bezüglich ``flatten the curve'' nicht mehr im
Fokus steht.

Dennoch kann man die Frage nach der generellen Wirksamkeit von Masken
stellen. Im Falle von Influenza-Epidemien und -Pandemien ist die Antwort
\href{https://wwwnc.cdc.gov/eid/article/26/5/19-0994_article}{aus
wissenschaftlicher Sicht} bereits klar: Masken im Alltag habe keine oder
eine sehr geringe Wirkung auf das Infektionsgeschehen. Bei unsachgemäßer
Verwendung können sie das Infektionsrisiko sogar erhöhen.

Das beste und aktuellste Beispiel dafür ist ironischerweise das oft
genannte Maskenland Japan: Japan erlebte seine
\href{https://www.upi.com/Top_News/World-News/2019/02/01/Millions-in-Japan-affected-as-flu-outbreak-grips-country/9191549043797/}{letzte
starke Grippewelle} mit rund fünf Millionen Erkankten trotz Masken
gerade erst vor einem Jahr, im Januar und Februar 2019.

Bei Influenza-Viren kommt allerdings im Unterschied zu SARS-Coronaviren
der wichtige Faktor hinzu, dass sie sehr wesentlich durch Kinder
übertragen werden. Japan musste 2019 beispielsweise rund zehntausend
Schulen wegen akuter Krankheitsausbrüche schließen.

Beim SARS-1-Virus von 2002 und 2003 gibt es eine
\href{https://onlinelibrary.wiley.com/doi/10.1111/j.1750-2659.2011.00307.x}{gewisse
Evidenz}, dass \emph{medizinische} Masken vor einer Infektion teilweise
schützen können. Doch SARS-1 verbreitete sich fast nur in
Krankenhäusern, d.h. in einem professionellen Umfeld, und kaum in der
Allgemeinheit.

Eine bekannte Studie von 2015
\href{https://bmjopen.bmj.com/content/5/4/e006577}{zeigte hingegen},
dass die heutzutage verbreiteten Stoffmasken aufgrund ihrer Porengröße
für 97\% der viralen Partikel durchlässig sind und das Infektionsrisiko
durch die Speicherung von Feuchtigkeit zusätzlich erhöhen können.

Einige Studien argumentierten zuletzt, Masken im Alltag seien im Falle
des neuen Coronavirus dennoch wirksam und könnten zumindest die
Ansteckung anderer Personen verhindern.

Diese Studien sind jedoch methodisch schwach und belegen teilweise eher
das Gegenteil. Typischerweise ignorieren diese Studien den Effekt
anderer zeitgleicher Maßnahmen, die natürliche Entwicklung der
Infektionszahlen, die Veränderung der Test-Aktivität, oder sie
vergleichen Länder mit sehr unterschiedlichen Voraussetzungen.

Ein Überblick:

\begin{enumerate}
\def\labelenumi{\arabic{enumi}.}
\tightlist
\item
  Eine deutsche Studie
  \href{http://ftp.iza.org/dp13319.pdf\#page=28}{behauptete}, die
  Einführung einer Maskenpflicht in deutschen Städten habe zu einem
  Rückgang der Infektionen geführt. Doch die Daten belegen das nicht: In
  einigen Städten gab es keine Veränderung, in anderen eine Abnahme, in
  weiteren eine Zunahme der Infektionen (s. Grafik unten). Die als
  Vorbild präsentierte Stadt Jena führte gleichzeitig die strengsten
  \href{https://www.mdr.de/thueringen/ost-thueringen/jena/corona-jena-seit-einer-woche-keine-neuinfektion-100.html}{Quarantäneregeln}
  Deutschlands ein, was die Studie jedoch nicht erwähnte.
\item
  Eine Studie im Fachjournal PNAS
  \href{https://www.pnas.org/content/117/26/14857}{behauptete}, Masken
  hätten in drei Hotspots (darunter New York City) zu einem Rückgang der
  Infektionen geführt. Dabei wurden weder der natürliche Rückgang der
  Infektionen noch andere Maßnahmen berücksichtigt. Die Studie war so
  fehlerhaft, dass über 40 Wissenschaftler einen
  \href{https://reason.com/2020/06/22/prominent-researchers-say-a-widely-cited-study-on-wearing-masks-is-badly-flawed/}{Rückzug
  der Studie} empfahlen.
\item
  Eine amerikanische Studie
  \href{https://www.healthaffairs.org/doi/full/10.1377/hlthaff.2020.00818}{behauptete},
  die Maskenpflicht habe in 15 Bundesstaaten zu einem Rückgang der
  Infektionen geführt. Die Studie berücksichtigte nicht, dass das
  Infektionsgeschehen zu diesem Zeitpunkt in den meisten Bundesstaaten
  ohnehin bereits rückläufig war. Ein Vergleich mit anderen
  Bundesstaaten wurde nicht durchgeführt.
\item
  Eine amerikanisch-kanadische Studie
  \href{https://www.medrxiv.org/content/10.1101/2020.05.22.20109231v3.full.pdf}{behauptete},
  Länder mit einer Maskenpflicht hätten weniger Todesfälle als Länder
  ohne Maskenpflicht. Doch dabei wurden afrikanische,
  lateinamerikanische, asiatische und osteuropäsiche Länder mit sehr
  unterschiedlichen Infektionsgeschehen und Bevölkerungsstrukturen
  verglichen.
\item
  Eine Metastudie im Fachjournal \emph{Lancet}
  \href{https://www.thelancet.com/journals/lancet/article/PIIS0140-6736(20)31142-9/fulltext}{behauptete},
  Masken ``könnten'' zu einer Reduktion des Infektionsrisikos führen,
  doch die untersuchten Studien bezogen sich hauptsächlich auf
  Krankenhäuser (Sars-1) und die Stärke der Evidenz wurde mit ``gering''
  angegeben.
\end{enumerate}

Der medizinische Nutzen einer Maskenpflicht bleibt daher
\href{https://www.cidrap.umn.edu/news-perspective/2020/04/commentary-masks-all-covid-19-not-based-sound-data}{weiterhin
fraglich}. Eine länder­ver­gleichende Untersuchung der University of
East Anglia kam etwa zum Ergebnis, dass eine Maskenpflicht
\href{https://www.uea.ac.uk/about/-/new-study-reveals-blueprint-for-getting-out-of-covid-19-lockdown}{keinen
messbaren Effekt} auf die Covid-Infektionen oder Todesfälle hatte.

Bekannt ist auch, dass die sehr weit verbreiteten Masken den
ursprünglichen Ausbruch in der chinesischen Stadt Wuhan nicht verhindern
konnten.

Schweden zeigte, dass es auch ohne Lockdown, ohne Maskenpflicht und mit
einer der
\href{https://link.springer.com/article/10.1007/s00134-012-2627-8}{tiefsten
Intensivbettkapazitäten} Europas nicht zu einer Überlastung der
Krankenhäuser kommen muss. Tatsächlich liegt die schwedische
Jahresgesamtmortalität im Bereich
\href{https://swprs.files.wordpress.com/2020/06/sweden-all-cause-nov-may-1990.jpg}{früherer
Grippewellen}.

Problematisch wird es, wenn Behörden der Bevölkerung suggerieren, durch
eine Maskenpflicht sinke das Infektionsrisiko etwa im öffentlichen
Verkehr. Dafür gibt es keine Evidenz. Ob mit oder ohne Masken besteht in
dicht besetzten Innenräumen ein erhöhtes Infektionsrisiko.

Interessanterweise steht an der Spitze der Forderung nach einer
weltweiten Maskenpflicht eine Lobbygruppe namens
\href{https://masks4all.co/about-us/}{``masks4all''} (Masken für alle),
die von einem ``Young Leader'' des World Economic Forum (WEF) Davos
gegründet wurde.

\includegraphics{https://swprs.files.wordpress.com/2020/06/germany-face-masks-april-2020.png?w=736\&h=547}

\hypertarget{zur-rolle-von-kontaktverfolgung}{%
\subparagraph{\texorpdfstring{\textbf{Zur Rolle von
Kontaktverfolgung}}{Zur Rolle von Kontaktverfolgung}}\label{zur-rolle-von-kontaktverfolgung}}

Zahlreiche Länder haben für teils sehr viel Geld
Smartphone-Applikationen und Spezialeinheiten zur
gesamtgesellschaftlichen Kontaktverfolgung eingerichtet. Es gibt indes
keine Evidenz, dass diese einen epidemiologisch relevanten Beitrag zum
Pandemie-Management leisten können.

Beim Tracing-Pionier Island ist die App bereits
\href{https://www.technologyreview.com/2020/05/11/1001541/iceland-rakning-c19-covid-contact-tracing/}{weitgehend
gescheitert}, in Norwegen wurde sie aus Daten­schutz­gründen
\href{https://www.msn.com/en-us/news/technology/norway-to-halt-covid-19-track-and-trace-app-on-data-protection-concerns/ar-BB15uMEM}{gestoppt},
in Indien, Argentinien, Singapur und weiteren Ländern wurde sie
nachträglich doch noch
\href{https://www.tagesschau.de/ausland/indien-app-101.html}{obligatorisch},
in Israel wird sie direkt vom Geheimdienst
\href{https://www.jewishpress.com/news/the-courts/state-to-high-court-even-more-shin-bet-involvement-in-fighting-the-coronavirus/2020/04/14/}{betrieben}.

Eine WHO-Studie zu Grippepandemien kam 2019 zudem zum Ergebnis, dass
Kontaktverfolgung aus medizinischer Sicht nicht sinnvoll und
\href{https://apps.who.int/iris/bitstream/handle/10665/329438/9789241516839-eng.pdf\#page=9}{``unter
keinen Umständen zu empfehlen''} ist. Das typische Anwendungsgebiet
dafür sind eher sexuell übertragbare Krankheiten oder Vergiftungen.

Zudem bestehen weiterhin ernsthafte Bedenken bezüglich Datenschutz und
Bürgerrechten.

NSA-Whistleblower Edward Snowden warnte bereits im März, dass
Regierungen die Corona-Krise zum Anlass oder Vorwand für den Ausbau der
\href{https://www.youtube.com/watch?v=-pcQFTzck_c}{gesellschaftlichen
Überwachung und Kontrolle} nehmen und damit eine ``Architektur der
Unterdrückung'' errichten könnten.

Eine Whistleblowerin, die an einem Ausbildungsprogramm für
Kontaktverfolger in den USA
\href{https://www.youtube.com/watch?v=qFUyZWw7qoc}{teilgenommen hatte},
beschrieb dieses als ``totalitär'' und eine ``Gefahr für die
Gesellschaft''.

Der Schweizer Informatik-Professor Serge Vaudenay
\href{https://swprs.org/corona-app-ein-eklatanter-betrug/}{machte
öffentlich}, dass die Protokolle zur Kontaktverfolgung keineswegs
``dezentral'' und ``transparent'' sind, denn die eigentliche
Funktionalität wird durch eine Schnittstelle von Google und Apple (GAEN)
implementiert, die für die Öffentlichkeit nicht einsehbar und nicht
kontrollierbar ist (d.h. kein ``Open Source'').

Diese nicht-einsehbare Schnittstelle wurde von Google und Apple
inzwischen in
\href{https://www.bloomberg.com/news/articles/2020-04-10/apple-google-bring-covid-19-contact-tracing-to-3-billion-people}{drei
Milliarden Mobiltelefone} integriert. Laut Professor Vaudenay kann diese
Schnittstelle zudem
\href{https://swprs.org/corona-app-ein-eklatanter-betrug/}{alle
Kontakte}, nicht nur die medizinisch ``relevanten'', aufzeichnen und
speichern. Ein deutscher Informatik-Experte bezeichnete Tracing-Apps
seinerseits als ein
\href{https://www.heise.de/news/Informatiker-Die-Corona-App-ist-wie-ein-trojanisches-Pferd-4764560.html}{``Trojanisches
Pferd''}.

Für weitere Informationen zur ``Kontaktverfolgung'' siehe das Update vom
Juni.

\textbf{Siehe auch}:
\href{https://www.wired.com/story/inside-the-nsas-secret-tool-for-mapping-your-social-network/}{Inside
the NSA's Secret Tool for Mapping Your Social Network} (Wired)

\includegraphics{https://swprs.files.wordpress.com/2020/06/covid-google-apple.jpg?w=600\&h=338}

\hypertarget{weitere-meldungen}{%
\subparagraph{\texorpdfstring{\textbf{Weitere
Meldungen}}{Weitere Meldungen}}\label{weitere-meldungen}}

\begin{itemize}
\tightlist
\item
  Inside Corona:
  \href{https://www.insidecorona.ch/2020/06/24/eind\%C3\%A4mmungsstrategien-entwickeln-sich-zum-flop/}{Eindämmungsstrategien
  entwickeln sich zum Flop}
\item
  Professor Beda Stadler:
  \href{https://www.achgut.com/artikel/corona_aufarbeitung_warum_alle_falsch_lagen}{Coronavirus:
  Warum alle falsch lagen}*\\
  *
\item
  Weitere Meldungen und Analysen auf
  \href{https://corona-transition.org/}{Corona-Transition.org}
\end{itemize}

\hypertarget{zum-ursprung-des-neuen-coronavirus-1}{%
\subparagraph{\texorpdfstring{\textbf{Zum Ursprung des neuen
Coronavirus}}{Zum Ursprung des neuen Coronavirus}}\label{zum-ursprung-des-neuen-coronavirus-1}}

Im Juni-Update wurde dargestellt, dass renommierte Virologen einen
Labor-Ursprung des neuen Coronavirus für
\href{https://www.independentsciencenews.org/health/the-case-is-building-that-covid-19-had-a-lab-origin/}{``mindestens
so plausibel''} halten wie einen natürlichen Ursprung. Der Grund dafür
sind einige genetische Besonderheiten des Virus im Bereich der
Rezeptorbindung, die zu einer besonders hohen Übertragbarkeit und
Infektiosität beim Menschen führen.

Inzwischen gibt es weitere Evidenzfür diese Hypothese. Siehe dazu:

\begin{itemize}
\tightlist
\item
  \href{https://swprs.org/ursprung-des-covid-19-virus-die-mojiang-minenarbeiter-hypothese/}{Ursprung
  des Covid-19-Virus: Die Mojiang-Minenarbeiter-Hypothese} (SPR)
\item
  \href{https://www.thetimes.co.uk/article/seven-year-covid-trail-revealed-l5vxt7jqp}{Seven
  year coronavirus trail from bat cave via Wuhan lab} (London Times)
\item
  \href{https://armswatch.com/project-g-2101-pentagon-biolab-discovered-mers-and-sars-like-coronaviruses-in-bats/}{Pentagon
  biolab discovered MERS and SARS-like coronaviruses in bats} (DG)
\end{itemize}

Die Entwicklungen seit Anfang 2020 zeigen, dass das neue Coronavirus
nicht als ``Biowaffe'' im engeren Sinne zu sehen ist, da es zu wenig
tödlich und zu wenig gezielt einsetzbar ist. Allerdings kann es durchaus
die Bevölkerung in Angst versetzen und politisch genutzt werden.

Dennoch bleibt, neben einem möglichen Labor-Ursprung, auch ein
natürlicher Ursprung des neuen Coronavirus eine realistische
Möglichkeit, obschon die Hypothese vom
\href{https://thebulletin.org/2020/06/did-the-sars-cov-2-virus-arise-from-a-bat-coronavirus-research-program-in-a-chinese-laboratory-very-possibly}{``Wuhan-Tiermarkt''}
und zuletzt die
\href{https://www.news-medical.net/news/20200708/Research-sheds-doubt-on-the-Pangolin-link-to-SARS-CoV-2.aspx}{Pangolin-Hypothese}
von Experten inzwischen verworfen wurden.

\includegraphics{https://swprs.files.wordpress.com/2020/07/wiv-p4.jpg?w=600\&h=337}

(Zuletzt aktualsiert: 17. Juli)

\hypertarget{juni-2020}{%
\paragraph{Juni 2020}\label{juni-2020}}

\hypertarget{a-allgemeiner-teil-1}{%
\subparagraph{\texorpdfstring{\textbf{A. Allgemeiner
Teil}}{A. Allgemeiner Teil}}\label{a-allgemeiner-teil-1}}

\hypertarget{studien-zur-letalituxe4t-von-covid-19}{%
\subparagraph{\texorpdfstring{\textbf{Studien zur Letalität von
Covid-19}}{Studien zur Letalität von Covid-19}}\label{studien-zur-letalituxe4t-von-covid-19}}

Stanford-Professor John Ioannidis publizierte im Mai eine
\href{https://swprs.org/studies-on-covid-19-lethality/}{Übersicht der
bisherigen Covid19-Antikörper-Studien}. Demnach liegt die Letalität von
Covid19 (IFR) in den meisten Ländern und Regionen bei unter 0.16\%. Für
drei Hotspots fand Ioannidis eine Obergrenze von 0.40\%.

Auch die US-Gesundheitsbehörde CDC reduzierte in ihrem
\href{https://reason.com/2020/05/24/the-cdcs-new-best-estimate-implies-a-covid-19-infection-fatality-rate-below-0-3/}{neuesten
Bericht} die Covid19-Letalität (IFR) auf 0.26\% (best estimate). Selbst
dieser Wert ist noch als Obergrenze zu sehen, da die CDC konservativ von
35\% asymptomatischen Fällen ausgeht, während die meisten Studien
auf\href{https://www.bmj.com/content/369/bmj.m1375}{50 bis 80\%}
asymptomatische Fälle hindeuten.

Ende Mai publizierten Schweizer Immunologen um Professor Onur Boyman
allerdings die bisher wohl
\href{https://swprs.files.wordpress.com/2020/06/tagesanzeiger-antibody-study-june-2020.pdf}{wichtigste
Studie} zur Covid19-Letalität. Diese Preprint-Studie kam zum Ergebnis,
dass die üblichen Antikörper-Tests, die Antikörper im Blut messen (IgG
und IgM), \textbf{höchstens ca. ein Fünftel} aller Covid19-Infektionen
erkennen können
(\href{https://swprs.org/coronavirus-antibody-tests-show-only-one-fifth-of-infections/}{Zeitungsbericht
auf Englisch};
\href{https://www.biorxiv.org/content/10.1101/2020.05.21.108308v1}{Originalstudie}).

Der Grund dafür ist, dass das neue Coronavirus bei den meisten Menschen
bereits durch Antikörper auf der Schleimhaut (IgA) oder durch eine
zelluläre Immunität (T-Zellen) neutralisiert wird und sich dabei keine
oder nur milde Symptome ausbilden.

Dies bedeutet, dass das neue Coronavirus vermutlich noch viel weiter
verbreitet ist als bisher angenommen und die Letalität pro Infektion bis
zu fünfmal niedriger liegt als bisher vermutet. Die wirkliche Letalität
könnte somit \textbf{deutlich unter 0.1\%} und damit im Bereich der
Influenza liegen.

Tatsächlich zeigten inzwischen mehrere Studien, dass bis zu 60\% aller
Menschen bereits über eine gewisse
\href{https://www.cell.com/cell/fulltext/S0092-8674(20)30610-3}{zelluläre
Immunität} gegen Covid-19 verfügen, die durch den Kontakt mit bisherigen
Coronaviren (Erkältungsviren) erworben wurde. Insbesondere Kinder kommen
oft in Kontakt mit solchen Coronaviren, was ihre Unempflindlichkeit
gegenüber Covid19 miterklären könnte.

Die neue Schweizer Studie dürfte zudem erklären, warum
Antikörper-Studien selbst in Hotspots wie New York oder Madrid eine
Infektionsrate von höchstens ca. 20\%
\href{https://www.governor.ny.gov/news/amid-ongoing-covid-19-pandemic-governor-cuomo-announces-results-completed-antibody-testing}{fanden},
denn dies würde einer tatsächlichen Rate von nahezu 100\% entsprechen.
In vielen Regionen dürfte die tatsächliche Ausbreitung bereits bei
deutlich \href{https://swprs.org/studies-on-covid-19-lethality/}{über
50\%} und damit im Bereich der Herdenimmunität liegen.

Sollte sich die Schweizer Untersuchung bestätigen, so würde die
Einschätzung der Oxford-Epidemiologin Prof. \textbf{Sunetra Gupta}
zutreffen**,** die bereits früh von einer sehr weiten Verbreitung von
Covid-19 und einer geringen Letalität
\href{https://unherd.com/2020/05/oxford-doubles-down-sunetra-gupta-interview/}{zwischen
0.01\% unter 0.1\%}ausging.

Trotz der vergleichsweise geringen \textbf{Letalität} von Covid-19
(Todesfälle pro Infektionen) kann die \textbf{Mortalität} (Todesfälle
pro Bevölkerung) regional und kurzfristig dennoch stark erhöht sein,
wenn sich das Virus rasch ausbreitet und dabei Risikogruppen wie
insbesondere Patienten in Pflegeheimen erreicht, wie das in mehreren
Hotspots tatsächlich geschah (siehe unten).

Aufgrund der eher geringen Letalität fällt Covid-19 höchstens in die
Stufe 2 des von den US-Gesundheitsbehörden entwickelten fünfstufigen
\href{https://www.cidrap.umn.edu/news-perspective/2007/02/hhs-ties-pandemic-mitigation-advice-severity}{Pandemie-Plans}.
Für diese Stufe ist lediglich die \textbf{``freiwillige Isolierung
kranker Personen''} als Hauptmaßnahme vorgesehen. Weitergehende
Maßnahmen wie Mundschutzpflicht, Schulschließungen, Abstandsregeln,
Kontaktverfolgung, Impfungen und Lockdowns ganzer Gesellschaften sind
hingegen nicht angezeigt.

Bezüglich \textbf{Kontaktverfolgung} kam eine WHO-Studie zu
Influenza-Pandemien von 2019 zudem zum Ergebnis, dass diese aus
medizinischer Sicht
\href{https://apps.who.int/iris/bitstream/handle/10665/329438/9789241516839-eng.pdf\#page=9}{``unter
keinen Umständen zu empfehlen''} ist, da bei leicht übertragbaren und
insgesamt eher milden Atemwegs­erkrankungen nicht zielführend.

Manchmal wird argumentiert, man habe die eher geringe Letalität zu
Beginn nicht gekannt. Das ist nicht ganz richtig, denn die Daten aus
Südkorea, der Kreuzfahrtschiffe und selbst aus Italien
\href{https://www.statnews.com/2020/03/17/a-fiasco-in-the-making-as-the-coronavirus-pandemic-takes-hold-we-are-making-decisions-without-reliable-data/}{zeigten
bereits im März}, dass das Risiko für die Allgemeinbevölkerung ziemlich
gering ist.

Viele Gesundheitsbehörden wussten dies auch, wie etwa
\href{https://www.thelocal.dk/20200529/leaked-emails-show-how-denmarks-pm-steam-rollered-her-own-health-agency}{geleakte
Emails aus Dänemark} von Mitte März zeigen: ``Die dänische
Gesundheitsbehörde ist weiterhin der Ansicht, dass Covid-19 nicht als
allgemein gefährliche Krankheit bezeichnet werden kann, da es weder
einen normalerweise schwerwiegenden Verlauf noch eine hohe
Sterblichkeitsrate aufweist.''

Einige Medien berechnen jedoch weiterhin eine angeblich viel höhere
Covid19-Letalität von teilweise über 1\%, indem sie einfach Todesfälle
durch ``Infektionen''
\href{https://english.elpais.com/society/2020-05-14/antibody-study-shows-just-5-of-spaniards-have-contracted-the-coronavirus.html}{dividieren},
ohne die Alters- und Risikoverteilung zu berücksichtigen, die gerade bei
Covid19 absolut entscheidend ist.

Die aktuellen Daten des europäischen Mortalitätsmonitorings
\href{https://www.euromomo.eu/graphs-and-maps/\#z-scores-by-country}{Euromomo}
zeigen, dass sich in mehreren Ländern inzwischen eine
\textbf{Untersterblichkeit} abzeichnet, so in Frankreich, Italien,
Spanien und auch der Schweiz. Der Grund dafür ist, dass der
Altersdurchschnitt der Covid19-Todesfälle sehr hoch lag, und in dieser
Altersgruppe nun bereits weniger Menschen als üblich sterben.

\textbf{Siehe auch}:
\href{https://swprs.org/studies-on-covid-19-lethality/}{Studies on
Covid-19 lethality}

\includegraphics{https://swprs.files.wordpress.com/2020/05/death-rate-age-groups-ma.png?w=600\&h=339}

\hypertarget{zur-rolle-der-pflegeheime}{%
\subparagraph{\texorpdfstring{\textbf{Zur Rolle der
Pflegeheime}}{Zur Rolle der Pflegeheime}}\label{zur-rolle-der-pflegeheime}}

Pflegeheime spielten eine
\href{https://ltccovid.org/2020/04/12/mortality-associated-with-covid-19-outbreaks-in-care-homes-early-international-evidence/}{absolute
Schlüsselrolle} in der Covid-19-Pandemie. In den meisten Ländern
ereigneten sich ein bis zwei Drittel aller Covid19-Todesfälle in
Pflegeheimen, in Kanada und einigen US-Bundesstaaten
sogar\href{https://freopp.org/the-covid-19-nursing-home-crisis-by-the-numbers-3a47433c3f70}{bis
zu 80\%}. Auch in Schweden, das keinen Lockdown verhängte, erfolgten
\href{https://www.thelocal.se/20200525/swedish-death-toll-passes-4000-as-coronavirus-cases-in-care-homes-start-to-fall}{75\%
der Todesfälle} in Pflegeheimen und Pflegewohnungen.

Umso bedenklicher ist es, dass manche Behörden ihre Pflegeheime
\emph{verpflichteten}, erkrankte Personen aus den Kliniken bei sich
aufzunehmen, was in der Folge fast immer zu zahlreichen Neuinfektionen
und Todesfällen führte. Dies geschah etwa in
\href{https://www.trtworld.com/magazine/the-massacre-of-italy-s-elderly-nursing-home-residents-35575}{Norditalien},
England und den stark betroffenen US-Bundesstaaten
\href{https://nypost.com/2020/05/12/calls-for-independent-probe-of-gov-cuomos-nursing-home-policies/}{New
York}, New Jersey und
\href{https://nypost.com/2020/05/13/pennsylvania-health-official-moved-mother-from-nursing-home/}{Pennsylvania}.

Aus Norditalien ist zudem bekannt, dass die verbreitete Angst vor dem
Virus und der angekündigte Lockdown zu einer Flucht der vorwiegend
osteuropäischen Pflegekräfte führte, wodurch der Zusammenbruch der
Alterspflege \href{https://swprs.org/covid-19-a-report-from-italy/}{noch
beschleunigt wurde}.

In den USA entfallen insgesamt
\href{https://www.forbes.com/sites/theapothecary/2020/05/26/nursing-homes-assisted-living-facilities-0-6-of-the-u-s-population-43-of-u-s-covid-19-deaths/\#70834ed374cd}{mindestens
42\%} aller Covid19-Todesfälle auf jene 0.6\% der Bevölkerung, die in
Pflegeheimen leben. Dabei erfordern Pflegeheime einen gezielten Schutz
und profitieren gerade nicht von einem allgemeinen Lockdown der gesamten
Gesellschaft.

Es ist bekannt, dass auch gewöhnliche Coronaviren (Erkältungsviren) für
Menschen in Pflegeheimen gefährlich sein können. Stanford-Professor John
Ioannidis machte bereits Mitte März darauf aufmerksam, dass Coronaviren
dort eine
\href{https://www.statnews.com/2020/03/17/a-fiasco-in-the-making-as-the-coronavirus-pandemic-takes-hold-we-are-making-decisions-without-reliable-data/}{Fallsterblichkeit
von bis zu 8\%} erreichen.

Zudem ist oft nicht klar, ob diese Menschen wirklich an Covid-19 starben
oder am wochenlangen Stress und der totalen Isolation. So gab es in
englischen Pflegeheimen ca. 30,000 zusätzliche Todesfälle, aber bei nur
10,000 ist Covid19 \href{https://www.bmj.com/content/369/bmj.m1931}{auf
dem Totenschein vermerkt}.

Allein im April starben in England und Wales rund
\href{https://www.theguardian.com/world/2020/jun/05/covid-19-causing-10000-dementia-deaths-beyond-infections-research-says}{10,000
zusätzliche Demenzpatienten} ohne Corona-Infektion aufgrund der
wochenlangen Isolation. In mehreren Ländern wurden inzwischen
Untersuchungen zur Situation in Pflegeheimen
\href{https://www.theguardian.com/world/2020/apr/16/italian-police-broaden-care-home-coronavirus-milan}{eingeleitet}
oder
\href{https://nypost.com/2020/05/12/calls-for-independent-probe-of-gov-cuomos-nursing-home-policies/}{gefordert}.

\includegraphics{https://swprs.files.wordpress.com/2020/05/care-home-deaths-may-21.png?w=736\&h=443}

\hypertarget{zur-rolle-der-krankenhuxe4user}{%
\subparagraph{\texorpdfstring{\textbf{Zur Rolle der
Krankenhäuser}}{Zur Rolle der Krankenhäuser}}\label{zur-rolle-der-krankenhuxe4user}}

Der zweite zentrale Faktor bezüglich Infektionen und Todesfällen, neben
den Pflegeheimen, sind die Krankenhäuser selbst. Bereits in einer
\href{https://jamanetwork.com/journals/jama/fullarticle/2761044}{Fallstudie
in Wuhan} zeigte sich, dass sich ca. 41\% der hospitalisierten Patienten
im Krankenhaus selbst mit Covid19 angesteckt hatten.

Auch in Norditalien, Spanien, England und weiteren stark betroffenen
Regionen spielte die Ansteckung in Krankenhäusern eine
\href{https://medium.com/@tepper_jonathan/ground-zero-when-the-cure-is-worse-than-the-disease-3c513d91393d}{entscheidende
Rolle}, d.h. die Kliniken wurden selbst zum Hauptübertragungsort von
Covid19 auf bereits geschwächte Menschen (sog. nosokomiale Infektion) --
ein Problem, das bereits beim SARS-Ausbruch von 2003
\href{https://onlinelibrary.wiley.com/doi/10.1046/j.1440-1843.2003.00523.x}{beobachtet}
wurde.

Nach heutigem Kenntnisstand hatten jene Länder, die Infektions­ausbrüche
in Pflegeheimen und Krankenhäusern vermeiden konnten, vergleichsweise
wenige Todesfälle zu beklagen. Der allgemeine Lockdown spielte hingegen
keine bzw. eine kontraproduktive Rolle (siehe unten).

Hinzu kommt eine mitunter tödliche Fehlbehandlung von Covid19-Patienten
durch aggressive Medikamente oder invasive Beatmung (Intubation), vor
deren Risiken Fachleute
\href{https://off-guardian.org/2020/05/06/covid19-are-ventilators-killing-people/}{seit
Monaten warnten}. So gibt es in den USA
\href{https://eu.usatoday.com/story/news/factcheck/2020/04/24/fact-check-medicare-hospitals-paid-more-covid-19-patients-coronavirus/3000638001/}{finanzielle
Anreize}, Covid-Patienten an Beatmungs­maschinen anzuschließen. In New
York wurde diesbezüglich inzwischen
\href{https://nypost.com/2020/05/29/northwell-health-probing-use-of-ventilators-for-covid-patients/}{eine
Untersuchung} eingeleitet.

\textbf{Siehe auch}:
\href{https://www.youtube.com/watch?v=UIDsKdeFOmQ}{Eine
Undercover-Krankenschwester berichtet aus New York City} (Video)

\hypertarget{zum-krankheitsbild-von-covid-19-1}{%
\subparagraph{\texorpdfstring{\textbf{Zum Krankheitsbild von
Covid-19}}{Zum Krankheitsbild von Covid-19}}\label{zum-krankheitsbild-von-covid-19-1}}

Der bekannte Hamburger Rechtsmediziner Professor Klaus Püschel stellte
im Mai seine
\href{https://www.acpjournals.org/doi/10.7326/M20-2003}{weltweit
erstmalige Studie} (englisch) zu den ersten 12 von 190 detaillierten
Corona-Obduktionen an einer
\href{https://www.youtube.com/watch?v=GXhxorBBPYI}{Pressekonferenz}
(deutsch) vor.

Professor Püschel betonte erneut, dass Covid-19 ``nicht ansatzweise so
bedrohlich ist, wie zunächst vermutet wurde''. Die Gefahr sei ``durch
Medienbilder viel zu sehr beeinflusst'' worden. Die Medien hätten auf
schwere Einzelfälle fokusiert und mit ``völlig falschen Botschaften''
Panik geschürt. Covid-19 sei kein ``Killervirus'' und der Ruf nach neuer
Medizin oder Impfstoffen ``von Angst geprägt''.

Die konkrete Todesursache der untersuchten Todesfälle sei jeweils eine
Lungenentzündung gewesen, allerdings lagen in circa 50\% der Fälle auch
Venenthrombosen in den Beinen vor, die in der Folge zu tödlichen
Lungenembolien führen können. Teilweise seien zudem die Nieren und der
Herzmuskel betroffen gewesen. Professor Püschel empfiehlt deshalb bei
ernsthaften Covid-Erkrankungen die präventive Abgabe von
blutverdünnenden Medikamenten.

Bezüglich der Thrombosen und Lungenembolien betonte Professor Püschel --
wie zuvor bereits andere Experten -- dass ein ``Lockdown'' mit
Quarantäne zuhause ``genau die falsche Maßnahme'' sei, da der
Bewegungsmangel Thrombosen zusätzlich fördert. Auch US-Fachärzte haben
vor diesem Risiko
\href{https://twitter.com/AlexBerenson/status/1258625618431954945}{gewarnt},
nachdem selbst bei \emph{Covid-negativen} Personen
\href{https://twitter.com/AlexBerenson/status/1259634922161147910}{unerwartete}
Thrombosen
\href{https://twitter.com/AlexBerenson/status/1259548620724080640}{auftraten}.

Viele Medien interpretierten die Obduktionsbefunde wiederum falsch und
sprachen von Covid-19 als einer besonders gefährlichen Krankheit, die
angeblich im Unterschied zur Influenza zu Thrombosen und Lungenembolien
führe. Das ist nicht richtig: Bereits seit 50 Jahren ist
\href{https://www.thieme-connect.com/products/ejournals/abstract/10.1055/s-0028-1108874}{bekannt},
dass auch eine schwere Influenza
\href{https://www.sciencedaily.com/releases/2009/10/091014111549.htm}{das
Risiko} für Thrombosen und Embolien
\href{https://www.ejinme.com/article/S0953-6205(15)00284-8/pdf}{stark
erhöht} und den
\href{https://academic.oup.com/ije/article-abstract/7/3/231/755276}{Herzmuskel}
und andere Organe betreffen kann. Sogar die Empfehlung bezüglich
präventivem Blutverdünner bei schwerer Influenza ist schon seit 50
Jahren
\href{https://www.thieme-connect.com/products/ejournals/abstract/10.1055/s-0028-1108874}{bekannt}.

\hypertarget{kinder-und-schulen-1}{%
\subparagraph{\texorpdfstring{\textbf{Kinder und
Schulen}}{Kinder und Schulen}}\label{kinder-und-schulen-1}}

Zahlreiche Studien haben inzwischen
\href{https://thehill.com/opinion/education/500349-science-says-open-the-schools}{belegt},
dass Kinder an Covid19 kaum erkranken und das Virus nicht oder kaum
übertragen, was bereits vom SARS-Ausbruch von 2003
\href{https://www.thelancet.com/journals/lanchi/article/PIIS2352-4642(20)30095-X/fulltext}{bekannt}
war. Für die Schließung von Schulen gab es deshalb
\href{https://infekt.ch/2020/04/schulen-schliessen-hilfreich-oder-nicht/}{zu
keinem Zeitpunkt} einen medizinischen Grund.

Dementsprechend registrierten all jene
\href{https://www.reuters.com/article/us-health-coronavirus-denmark-reopening/opening-schools-in-denmark-did-not-worsen-outbreak-data-shows-idUSKBN2341N7}{Länder},
die ihre Schulen bereits im Mai wieder öffneten,
\href{https://www.cgdev.org/blog/back-school-tracking-covid-cases-schools-reopen}{keine
Zunahme} an Infektionsfällen. Länder wie Schweden, die ihre Grundschulen
ohnehin nie schlossen, hatten damit ebenfalls keine Probleme.

Eine Preprint-Studie des deutschen Virologen Christian Drosten
argumentierte, die Ansteckungsgefahr durch Kinder sei doch vergleichbar
mit Erwachsenen und Schulen sollten deshalb geschlossen bleiben. Mehrere
Forscher wiesen jedoch
\href{https://www.nau.ch/news/europa/coronavirus-forscher-drangen-drosten-zum-ruckzug-der-kinder-studie-65714012}{methodische
Fehler} in der Studie nach. Drosten nahm die Empfehlung bzgl.
Schulschließungen daraufhin zurück.

In einigen Schulen, beispielsweise in Frankreich und Israel, soll es
dennoch zu angeblichen
\href{https://www.npr.org/sections/coronavirus-live-updates/2020/06/03/868507524/israel-orders-schools-to-close-when-covid-19-cases-are-discovered}{``Corona-Ausbrüchen''}
gekommen sein. Es dürfte sich dort jedoch um Übertragungen von Lehrern
auf Schüler handeln, die zu ihrem Leidwesen regelmäßig getestet werden,
obschon sie kaum Symptome zeigen und selbst kaum oder gar nicht
ansteckend sind.

Zum Thema Kawasaki-Krankheit kritisierte die britische \emph{Kawasaki
Disease Foundation} erneut die
\href{https://www.societi.org.uk/kawasaki-disease-and-covid-19/}{unseriöse
und reißerische Medienberichterstattung}. Tatsächlich gebe es bisher
keine signifikante Zunahme an Kawasaki-Fällen und keinen belegten
Zusammenhang mit Covid-19. Allgemeine entzündliche Reaktionen in
einzelnen Kindern seien auch von anderen Vireninfektionen bekannt, die
Anzahl der bisher gemeldeten Fälle sei indes äußerst gering.

Auch deutsche medizinische Verbände
\href{https://www.welt.de/politik/deutschland/article208075525/Corona-Kitas-und-Grundschulen-vollstaendig-oeffnen-uneingeschraenkt.html}{veröffentlichten
eine Entwarnung}: Covid-19 verlaufe bei nahezu allen Kindern unmerklich
oder sehr mild. Schulen und Kitas seien deshalb umgehend und \emph{ohne
Einschränkungen} zu öffnen, d.h. es brauche keine Kleingruppen,
Abstandsregeln oder Masken.

\includegraphics{https://swprs.files.wordpress.com/2020/06/school-kids-france.jpg?w=550\&h=366}

\hypertarget{zur-wirksamkeit-von-masken-2}{%
\subparagraph{\texorpdfstring{\textbf{Zur Wirksamkeit von
Masken}}{Zur Wirksamkeit von Masken}}\label{zur-wirksamkeit-von-masken-2}}

Unabhängig von der ohnehin vergleichsweise geringen Letalität von
Covid19 in der Allgemein­bevölkerung (siehe oben) gibt es für die
Wirksamkeit von Masken bei gesunden und symptomlosen Menschen im Alltag
weiterhin keine wissenschaftlichen Belege.

Eine ländervergleichende Untersuchung der Universität von East Anglia
kam zum Ergebnis, dass eine Maskenpflicht
\href{https://www.uea.ac.uk/about/-/new-study-reveals-blueprint-for-getting-out-of-covid-19-lockdown}{keinen
Nutzen brachte} und das Infektionsrisiko sogar erhöhen könnte.

Zwei US-Professoren und Experten für Atem- und Infektionsschutz der
Universität von Illinois
erklären\href{https://www.cidrap.umn.edu/news-perspective/2020/04/commentary-masks-all-covid-19-not-based-sound-data}{in
einem Aufsatz}, dass Masken im Alltag keine Wirkung haben, weder als
Selbstschutz \emph{noch zum Schutz Dritter} (sogenannte
Quellenkontrolle). Auch hätten die weitverbreiteten Masken den Ausbruch
in der chinesischen Stadt Wuhan nicht verhindert.

Eine Studie vom April 2020 in der Fachzeitschrift \emph{Annals of
Internal Medicine}
\href{https://www.acpjournals.org/doi/10.7326/M20-1342}{kam zum
Ergebnis}, dass weder Stoffmasken noch chirurgische Masken die
Verbreitung des Covid19-Virus durch Husten verhindern können.

Ein Artikel im \emph{New England Journal of Medicine} vom Mai 2020 kommt
ebenfalls zum Ergebnis, dass Masken im Alltag
\href{https://www.nejm.org/doi/full/10.1056/NEJMp2006372}{keinen oder
kaum Schutz} bieten. Der Ruf nach einer Maskenpflicht sei ein
``irrationaler Angstreflex''.

Eine von der US-Gesundheitsbehörde CDC veröffentliche Metastudie vom Mai
2020 zu Influenza-Pandemien kam
\href{https://wwwnc.cdc.gov/eid/article/26/5/19-0994_article}{ebenfalls
zum Ergebnis}, dass Masken weder das Infektions- noch das
Übertragungsrisiko (Quellenkontrolle) reduzieren können.

Eine Studie im British Medical Journal von 2015 kam
\href{https://bmjopen.bmj.com/content/5/4/e006577}{zum Ergebnis}, dass
Stoffmasken 97\% aller Partikel durchließen und das Infektionsrisiko
durch Speicherung der Feuchtigkeit erhöhen können.

Die WHO erklärte im Juni überdies, dass die oft diskutierte
``asymptomatische Übertragung'' in
Wirklichkeit\href{https://www.nationalreview.com/news/who-says-transmission-by-asymptomatic-covid-patients-very-rare/}{``sehr
selten''} ist, wie Daten aus zahlreichen Ländern zeigten. Die wenigen
bestätigten Fälle erfolgten zumeist durch
\href{https://wwwnc.cdc.gov/eid/article/26/8/20-1235_article}{direkten
Körperkontakt} wie Handschlag oder Küsse.

In Österreich wird die Maskenpflicht im Handel und in der Gastronomie ab
Mitte Juni
\href{https://www.derstandard.at/story/2000117792769/die-maske-wird-wieder-zurueckgedraengt}{wieder
aufgehoben}. In Schweden wurde
\href{https://www.thelocal.se/20200514/explained-why-is-sweden-not-recommending-face-masks-to-the-public}{nie
eine Maskenpflicht eingeführt}, da diese ``keinen zusätzlichen Schutz
für die Bevölkerung bietet'', wie die Gesundheitsbehörde erklärte.

Zahlreiche Politiker, Medienleute und Polizisten wurden bereits dabei
\href{https://www.youtube.com/watch?v=y2qAKS6Hl4I}{erwischt}, wie sie
ihre Masken in einer Menschenmenge extra für die Fernsehkameras
\href{https://twitter.com/DailyCaller/status/1265382872631980032}{anzogen}
oder sofort wieder
\href{https://twitter.com/realPowerTie/status/1261445611594723330}{ablegten},
als sie glaubten, es werde nicht mehr gefilmt.

In manchen Fällen kam es zu
\href{https://twitter.com/DrRJKavanagh/status/1260739779177656325}{brutalen
Polizeiübergriffen}, da eine Person ihre Maske ``nicht richtig
getragen'' habe. In anderen Fällen durften Menschen mit einer
Behinderung, die ärztlich attestiert keine Maske tragen können und
müssen, Einkaufhäuser
\href{https://bnn.de/lokales/karlsruhe/maskenpflicht-behinderte-sind-befreit-und-werden-dafuer-teils-beschimpft}{nicht
betreten}.

Entgegen all dieser Evidenz propagiert eine Gruppe namens
\href{https://masks4all.co/about-us/}{``masks4all''}, die von einem
Young Leader des World Economic Forum (WEF) Davos gegründet wurde, eine
weltweite Maskenpflicht. Verschiedene Regierungen
\href{https://www.businessinsider.com/who-updates-face-masks-recommendations-amid-the-coronavirus-pandemic-2020-6}{und
die WHO} scheinen darauf anzusprechen.

Kritiker vermuten in diesem Zusammenhang, dass die Masken eher eine
psychologische oder politische
\href{https://multipolar-magazin.de/artikel/maskenpflicht-gesellschaftliches-klima}{Funktion
erfüllen} (``Maulkorb'' bzw. ``sichtbares Zeichen des Gehorsams'') und
bei häufigem Tragen womöglich zu zusätzlichen gesundheitlichen Problemen
führen können.

Eine Studie aus Deutschland zeigte empirisch, dass die Einführung von
Masken \href{http://ftp.iza.org/dp13319.pdf\#page=28}{keine Auswirkung}
auf die Infektionszahlen hatte (siehe Grafik). Nur in der Stadt Jena kam
es zu einer deutlichen Reduktion, doch Jena führte zeitgleich sehr
strenge
\href{https://www.mdr.de/thueringen/ost-thueringen/jena/corona-jena-seit-einer-woche-keine-neuinfektion-100.html}{Quarantänevorschriften}
ein.

\includegraphics{https://swprs.files.wordpress.com/2020/06/germany-face-masks-april-2020.png?w=736\&h=547}

\hypertarget{zum-ursprung-des-virus}{%
\subparagraph{\texorpdfstring{\textbf{Zum Ursprung des
Virus}}{Zum Ursprung des Virus}}\label{zum-ursprung-des-virus}}

Mitte März argumentierten einige Forscher in einem Brief an die
Fachzeitschrift Nature Medicine, dass das Covid19-Virus
\href{https://www.nature.com/articles/s41591-020-0820-9}{natürlichen
Ursprungs} sein müsse und nicht ``aus einem Labor'' stammen könne. Als
Grund nannten sie die Struktur des Virus und den Umstand, dass die
Bindung an den menschlichen ACE2-Zellrezeptor nicht dem theoretischen
Maximum entspreche.

Inzwischen haben jedoch zahlreiche
\href{https://www.independentsciencenews.org/health/the-case-is-building-that-covid-19-had-a-lab-origin/}{renommierte
Virologen} dieser Argumentation
\href{https://gmwatch.org/en/news/latest-news/19412-lab-escape-theory-of-sars-cov-2-origin-gaining-scientific-support}{widersprochen}.
Ein künstlicher Ursprung im Rahmen von virologischer
\href{https://medium.com/@yurideigin/lab-made-cov2-genealogy-through-the-lens-of-gain-of-function-research-f96dd7413748}{Funktionsforschung}
sei ``mindestens so plausibel'' wie ein natürlicher Ursprung.
Tatsächlich werde an solchen Coronaviren seit bald 20 Jahren (dem
SARS-Ausbruch von 2003) in mehreren Laboren intensiv
\href{https://gmwatch.org/en/news/latest-news/19410-chinese-and-us-scientists-genetically-engineered-bat-coronaviruses-in-dangerous-gain-of-function-research-stretching-back-years}{geforscht}.

Als Argumente \emph{für} einen künstlichen Ursprung werden insbesondere
angeführt, dass die Bindung an den menschlichen ACE2-Zellrezeptor
\href{https://arxiv.org/abs/2005.06199}{deutlich stärker} sei als bei
allen üblichen Ursprungstieren, und dass bisher kein direktes
Ursprungstier identifiziert werden konnte. Zudem enthalte das Virus
einige auffällige funktionale
\href{https://gmwatch.org/en/news/latest-news/19403-wuhan-and-us-scientists-used-undetectable-methods-of-genetic-engineering-on-bat-coronaviruses}{Gensequenzen},
die künstlich eingefügt worden sein könnten (s. Grafik).

Die anfängliche Theorie vom Tiermarkt in Wuhan wurde inzwischen
\href{https://thebulletin.org/2020/06/did-the-sars-cov-2-virus-arise-from-a-bat-coronavirus-research-program-in-a-chinese-laboratory-very-possibly}{verworfen},
da keines der dortigen Tiere positiv getestet wurde, und ein Drittel der
allerersten Patienten keine Verbindung zum Tiermarkt hatten. Der
Tiermarkt wird nun als sekundärer Übertragungsort gesehen.

Es ist bekannt, dass das virologische Labor in Wuhan in Zusammenarbeit
mit den USA und Frankreich an Coronaviren
\href{https://www.newsweek.com/dr-fauci-backed-controversial-wuhan-lab-millions-us-dollars-risky-coronavirus-research-1500741}{forschte}
und dabei auch ``potentiell pandemische Pathogene'' (PPP) erzeugte, die
besonders leicht übertragbar und/oder besonders gefährlich sind.
Außerdem kam es in China und auch in den USA bereits zu
\href{https://www.independentsciencenews.org/health/the-long-history-of-accidental-laboratory-releases-of-potential-pandemic-pathogens/}{mehreren
Laborunfällen} mit Freisetzung von Viren.

Der unvoreingenommene Beobachter muss daher weiterhin mehrere
realistische Möglichkeiten in Betracht ziehen: Ein natürlicher Ursprung
des Virus (wie er bei SARS 2003 angenommen wird), ein
\href{https://project-evidence.github.io/}{Laborunfall} im Rahmen von
Funktionsforschung (vermutlich in Wuhan), oder sogar eine gezielte
Freisetzung durch einen geopolitisch interessierten Akteur in Ost oder
West.

Gleichwohl ist beim Covid19-Virus nicht von einer ``Biowaffe'' im
klassichen Sinne zu sprechen: Das Virus ist zwar
\href{https://leelabvirus.host/covid19/origins-part2}{sehr leicht
übertragbar}, aber für die Allgemeinbevölkerung nicht besonders
gefährlich. Tierversuche haben gezeigt, dass wesentlich tödlichere
Coronaviren erzeugt werden können.

\includegraphics{https://swprs.files.wordpress.com/2020/06/virus-sequence-comparison.png?w=736\&h=227}

\hypertarget{impfungen-gegen-covid-19}{%
\subparagraph{\texorpdfstring{\textbf{Impfungen gegen
Covid-19}}{Impfungen gegen Covid-19}}\label{impfungen-gegen-covid-19}}

Verschiedene Politiker in Europa und den USA haben erklärt, dass die
``Corona-Krise'' erst durch einen noch zu entwicklenden Impfstoff
beendet werden könne.

Allerdings haben viele Experten bereits darauf
\href{https://www.nature.com/articles/d41586-020-00751-9}{hingewiesen},
dass ein forcierter Impfstoff gegen das neue Coronavirus
\href{https://www.youtube.com/watch?v=vrL9QKGQrWk}{aufgrund} der
insgesamt geringen Letalität (siehe oben) und der bereits
\href{https://www.bitchute.com/video/DXxeej9CW1pz/}{abklingenden}
Verbreitung
\href{https://www.news.com.au/lifestyle/health/health-problems/no-vaccine-for-coronavirus-a-possibility/news-story/34e678ae205b50ea983cc64ab2943608}{nicht
unbedingt erforderlich} ist. Der Schutz von Risikogruppen, insbesondere
in Pflegeheimen, könne
\href{https://www.reuters.com/article/us-health-coronavirus-haseltine-newsmake-idUSKBN22W34T}{wesentlich
gezielter} erfolgen.

Einige Experten wie der Schweizer Infektiologe Dr. Pietro Vernazza haben
zudem darauf hingewiesen, dass gerade die Hochrisikogruppe
erfahrungsgemäß
\href{https://infekt.ch/2020/05/corona-impfung-als-ultimative-rettung/}{am
wenigsten} von einer Impfung profitiert, da ihr Immunsystem nicht mehr
ausreichend auf den Impfstoff reagiert.

Verschiedene Experten haben zudem auf die
\href{https://www.reuters.com/article/us-health-coronavirus-vaccines-insight-idUSKBN20Y1GZ}{erheblichen
Gesundheitsrisiken} eines forcierten Impfstoffes
\href{https://www.nature.com/articles/d41586-020-00751-9}{hingewiesen}.
Tatsächlich führte beispielsweise die Impfung gegen die sogenannte
``Schweinegrippe'' von 2009/2010 zu teilweise
\href{https://www.ibtimes.co.uk/brain-damaged-uk-victims-swine-flu-vaccine-get-60-million-compensation-1438572}{schweren
neurologischen Schäden} insbesondere bei Kindern und zu
Schaden­­ersatz­forderungen in Millionenhöhe.

Dennoch wurden bereits mehrere Milliarden Dollar privater und
öffentlicher Gelder für die Entwicklung eines Impfstoffes
\href{https://www.nytimes.com/2020/05/04/world/europe/eu-coronavirus-vaccine.html}{eingesammelt}.
Auch ein ``Immunitätsausweis'' für Arbeit und Reisen wird
\href{https://www.theguardian.com/politics/2020/may/03/coronavirus-health-passports-for-uk-possible-in-months}{weiterhin
diskutiert}. Allerdings kam es bei den beiden führenden
\href{https://www.thelancet.com/journals/lancet/article/PIIS0140-6736(20)31252-6/fulltext}{Impfstoff-Projekten},
entgegen den meisten Mediendarstellungen, zu teilweise gravierenden
Komplikationen.

Im Falle des Impfstoffes der \textbf{Universität Oxford} erkrankten im
Tierversuch
\href{https://www.dailymail.co.uk/sciencetech/article-8331709/Oxford-coronavirus-vaccine-does-not-stop-infection-experts-warn.html}{alle
sechs Rhesusaffen} trotz Impfung an Covid19 und waren gleich infektiös
wie die ungeimpften Affen. Dennoch wurde der Impfstoff in die Testphase
mit Menschen weitergezogen. Der Projektleiter erklärte jedoch, dass das
Coronavirus in der Bevölkerung bereits \emph{so selten geworden}
\emph{sei}, dass der klinische Versuch mit 50\%iger Wahrscheinlichkeit
\href{https://www.telegraph.co.uk/news/2020/05/23/oxford-university-covid-19-vaccine-trial-has-50-per-cent-chance/}{kein
Ergebnis liefern} werde.

Im Falle des neuartigen RNA-Impfstoffes der Firma \textbf{Moderna}, der
ungewöhnlicherweise direkt im Menschenversuch
\href{https://www.statnews.com/2020/03/11/researchers-rush-to-start-moderna-coronavirus-vaccine-trial-without-usual-animal-testing/}{getestet}
wurde, hatten 20\% der Teilnehmer in der hochdosierten Gruppe eine
\href{https://childrenshealthdefense.org/news/vaccine-trial-catastrophe-moderna-vaccine-has-20-serious-injury-rate-in-high-dose-group/}{``schwere
Nebenwirkung''}, obschon Moderna nur sehr gesunde Personen zum Versuch
zuließ.

Einer der Moderna-Teilnehmer wurde von CNN danach als ``Held''
präsentiert und interviewt. Dabei wurde jedoch vereinbart, nicht zu
erwähnen, dass der Teilnehmer nach der Impfung ohnmächtig und
\href{https://childrenshealthdefense.org/news/modernas-guinea-pig-sickest-in-his-life-after-being-injected-with-experimental-vaccine/}{``so
krank wie noch nie in seinem Leben''} wurde. Mehrere Experten
kritisierten zudem, dass Moderna ihre klinischen Daten
\href{https://www.statnews.com/2020/05/19/vaccine-experts-say-moderna-didnt-produce-data-critical-to-assessing-covid-19-vaccine/}{nicht
ausreichend offengelegt} habe.

Der Leiter des US-Programms zur raschen Entwicklung eines
Corona-Impfstoffes war zuvor selbst
\href{https://www.businessinsider.com/moncef-slaoui-leading-trump-vaccine-push-10m-holding-moderna-conflict-2020-5}{Direktor
bei Moderna}. Präsident Trump kündigte zudem an, den Impfstoff womöglich
\href{https://www.msn.com/en-us/news/politics/trump-says-hes-mobilizing-military-to-distribute-potential-coronavirus-vaccine/ar-BB1463uZ}{mit
dem US-Militär} landesweit zu verteilen. Einige Länder wie etwa Dänemark
schufen bereits
\href{https://www.thelocal.dk/20200313/denmark-passes-far-reaching-emergency-coronavirus-law}{gesetzliche
Grundlagen} für eine Pflichtimpfung der gesamten Bevölkerung. Auch in
Deutschland haben sich
\href{https://www.swp.de/panorama/corona-impfpflicht-wird-die-impfung-bald-zur-vorschrift-in-deutschland_-das-ist-der-momentane-stand-45785279.html}{verschiedene
Politiker} für eine Impfpflicht ausgesprochen.

Befürworter einer Impfpflicht wie der Weltarztpräsident Frank Montgomery
\href{https://www.n-tv.de/panorama/Weltaerztepraesident-fordert-Impfpflicht-article21793158.html}{argumentieren},
die Bevölkerung müsse sich impfen lassen, um jene zu schützen, die sich
aus gesundheitlichen Gründen nicht impfen lassen können. Angesichts der
eher \href{https://swprs.org/studies-on-covid-19-lethality/}{geringen
Letalität} von Covid19 und der bereits sehr weiten Verbreitung erscheint
diese Argumentation jedoch überaus fragwürdig. Hinzu kommen die oben
dargestellen, durchaus ernsthaften Impfstoffrisiken.

Der Chef des größten europäischen \textbf{Ticketportals Eventim}
\href{https://www.n-tv.de/mediathek/audio/Ohne-Impfstoff-keine-Konzerte-article21821926.html}{erklärte}
gleichwohl, dass es womöglich ``erst wieder Veranstaltungen geben könne,
wenn es einen Impfstoff oder ein entsprechend wirksames Medikament gibt
-- da werden wir noch einige Zeit warten müssen.''

Der britische Premierminister Boris Johnson, der den Impfstoff-Gipfel
Anfang Juni zusammen mit US-Milliardär Bill Gates leitete, nannte die
Impfallianz GAVI eine Art
\href{https://www.youtube.com/watch?v=1S0LAbObZV0}{``Gesundheits-NATO''}.
Ein ``Impfausweis'' dürfte allerdings daran scheitern, dass selbst
Antikörper-Tests höchstens 20\% der Infektionen
\href{https://swprs.org/coronavirus-antibody-tests-show-only-one-fifth-of-infections/}{nachweisen
können}, wie die Studie von Professor Boyman erstmals zeigte.

\includegraphics{https://swprs.files.wordpress.com/2020/06/gvs-super-family-1200.jpg?w=736\&h=414}

\hypertarget{medikamente-gegen-covid-19}{%
\subparagraph{\texorpdfstring{\textbf{Medikamente gegen
Covid-19}}{Medikamente gegen Covid-19}}\label{medikamente-gegen-covid-19}}

Die Situation im Bereich hilfreicher Medikamente bei schweren
Covid19-Fällen ist weiterhin sehr unklar. Ein gewisser Konsens besteht
lediglich darin, dass Blutverdünner
\href{https://www.webmd.com/lung/news/20200506/blood-thinners-could-boost-covid19-survival\#1}{hilfreich
sind}, um lebensbedrohende Thrombosen und Embolien vorzubeugen (wie bei
schwerer Influenza).

Heftige Diskussionen gibt es seit Monaten um das Malaria-Medikament
Hydroxychloroquin (HCQ). Die Fachzeitschrift Lancet publizierte Ende Mai
eine Studie, wonach HCQ vermehrt zu Herzproblemen führe. Die WHO stellte
daraufhin alle ihre HCQ-Studien ein. Kurz darauf wurde jedoch bekannt,
dass die Lancet-Studie auf
\href{https://www.theguardian.com/world/2020/jun/03/covid-19-surgisphere-who-world-health-organization-hydroxychloroquine}{einem
manipulierten Datensatz} beruhte.

Die Lancet-Studie sowie eine weitere Studie im \emph{New England Journal
of Medicine (NEJM)} mussten
\href{https://www.bloomberg.com/news/articles/2020-06-04/researchers-retract-lancet-study-linking-malaria-drug-to-risks}{zurückgezogen}
werden, was einen der größten Medizinskandale der letzten Jahre
darstellt. Der Grund für die manipulierte Studie ist nicht klar,
allerdings scheint der Hauptautor zugleich an einer Studie zum
konkurrierenden Medikament Remdesivir
\href{https://www.medicineuncensored.com/a-study-out-of-thin-air}{beteiligt
zu sein}.

Die Verwendung von Remdesivir des Pharmakonzerns Gilead kam ihrerseits
unter Druck nachdem
\href{https://edition.cnn.com/2020/05/22/health/remdesivir-covid-19-trial-results-nejm-study/index.html}{eine
erste Studie zeigte}, dass das Medikament das Sterberisiko nicht senken
konnte. Viele Medien ignorierten das jedoch und berichteten dennoch
positiv über das Medikament.

Ein ehemaliger französischer
Gesundheitsminister~\href{https://www.youtube.com/watch?v=ZYgiCALEdpE}{verriet
in einem Interview}, dass die Editoren von Lancet und NEJM in einer
vertraulichen Diskussionsrunde erklärten, der Einfluss von Pharmafirmen
sei so groß, ja ``kriminell'', geworden, dass man nicht mehr von
Wissenschaft sprechen könne.

Verschiedene Kliniken
\href{https://www.webmd.com/lung/news/20200409/chloroquine-zinc-tested-to-block-covid-infection}{nutzen
oder studieren} HCQ bei Covid19-Patienten, teilweise in Kombination mit
Zink, Vitaminen oder anderen Medikamenten. Allerdings ist bekannt, dass
HCQ bei der Stoffwechselbesonderheit Favismus, die insbesondere bei
Menschen aus Afrika und dem Mittelmeerraum vorkommt, zu
\href{https://off-guardian.org/2020/05/13/covid19-a-case-for-medical-detectives/}{tödlichen
Komplikationen} führen kann.

Es ist leider davon auszugehen, dass eine falsche oder zu aggressive
Medikation mit HCQ,
\href{https://www.sciencedaily.com/releases/2020/02/200206110703.htm}{Steroiden},
Antibiotika und antiviralen Mitteln sowie
\href{https://nypost.com/2020/05/29/northwell-health-probing-use-of-ventilators-for-covid-patients/}{invasiver
Beatmung} während der Corona-Pandemie zu zahlreichen zusätzlichen und
vermeidbaren Todesfällen geführt hat.

\hypertarget{expertenstimmen-auswahl}{%
\subparagraph{**Expertenstimmen
(Auswahl)}\label{expertenstimmen-auswahl}}

**

\begin{itemize}
\tightlist
\item
  Der deutsche Virologe \textbf{Prof. Hendrik Streeck} kritisierte den
  Lockdown
  \href{https://www.welt.de/wissenschaft/article209299157/Corona-Krise-Virologe-Streeck-kritisiert-deutschen-Lockdown.html}{und
  erklärte}, dass mittlerweise ``alle Experten zur Einschätzung der
  Anfangszeit zurückkehren'', dass Covid-19 ``nicht bagatellisiert
  werden sollte, aber auch nicht dramatisiert werden'' dürfe. Der Grund
  der sinkenden Risiko­bewertung sei die ``enorme Anzahl von
  Infektionen, die folgenlos blieben''. Streeck rechnet in Deutschland
  weiterhin mit \emph{keiner Über­sterb­lichkeit} bis Ende Jahr, da das
  Durch­schnitts­alter der Todesfälle mit 81 Jahren ``eher oberhalb der
  Lebens­erwartung'' liege. ``Corona-Apps'' und massenweise Corona-Tests
  hält Streeck nicht für sinnvoll. Auch die allgemeine Verwendung von
  Masken kritisiert er, diese böten kaum Schutz und würden rasch zum
  ``Nährboden für Bakterien und Pilze''. Schulen und Kitas sollten
  möglichst rasch wieder geöffnet werden, da Erkrankungs- und
  Ansteckungsrisiko bei Kindern sehr gering seien.
\item
  Ein \textbf{Oberarzt für Intensivmedizin am Universitätsspital
  Zürich}, der selbst kritische Covid-19-Patienten betreute, kritisierte
  in einem
  \href{https://www.medinside.ch/de/post/corona-intensivmediziner-spricht-klartext}{vielbeachteten
  Video} die ``Angstmacherei'' im Zusammenhang mit der Krankheit. Für
  die überwiegende Mehrzahl aller Menschen bestehe kein signifikantes
  Sterberisiko, die Zahlen seien insgesamt vergleichbar mit früheren
  Grippewellen. Risikogruppen könnten gezielt geschützt werden, während
  der generelle Lockdown lediglich die Immunisierung der
  Allgemeinbevölkerung verhindere. Zudem sterben Menschen aufgrund der
  verordneten medizinischen Unterversorgung in anderen Bereichen. Der
  medizinische und gesellschaftliche Schaden sei längst höher als der
  Nutzen. Auch der teilweise obligatorische Mundschutz für Schulkinder
  habe ``keinen medizinischen Sinn und Nutzen'' und belaste die Kinder
  stark. Die ``tägliche Zählerei'' der Fälle sei unsinnig und verbreite
  lediglich Angst. Die kontraproduktiven Maßnahmen müssten rasch
  gestoppt werden. Schweizer Medien versuchten nach der weiten
  Verbreitung des Videos das Universitätsspital Zürich
  \href{https://www.blick.ch/news/schweiz/unispital-distanziert-sich-von-umstrittenen-aussagen-corona-kritiker-feiern-videobotschaft-von-zuercher-oberarzt-id15903859.html}{unter
  Druck zu setzen}. Das Originalvideo wurde vom Arzt inzwischen
  entfernt.
\item
  Der Schweizer Chefarzt für Infektiologie, \textbf{Dr. Pietro
  Vernazza},
  \href{https://infekt.ch/2020/05/expedition-mit-ueberraschendem-ausgang-ein-epidemiologisches-lehrstueck/}{erklärt
  am Beispiel aktueller Fallstudien}, dass Fiebermessungen und
  Kontaktverfolgungen aufgrund der oft symptomlosen Verläufe und
  leichten Übertragbarkeit von Covid19 nicht sinnvoll sind.
\item
  Der bekannte Schweizer Immunologe \textbf{Dr. Beda Stadler} erklärt
  \href{https://www.swissinfo.ch/ita/seconda-ondata-con-lockdown--in-svizzera-no-di-sicuro/45797198}{in
  einem Beitrag}, dass Covid19 eine ``sehr selektive Krankheit'' sei und
  nur für sehr wenige Menschen ein echtes Risiko darstelle. Die Medien
  hätten sich hingegen auf die wenigen atypischen Einzelfälle
  konzentriert, die es aber bei jeder Krankheit gebe. Viele
  Wissenschaftler hätten sich zu stark auf ihre Modelle und zu wenig auf
  die Realität konzentriert. Die geplante Kontaktverfolgung sei
  medizinisch ``sinnlos'' und
  \href{https://www.blick.ch/news/schweiz/behoerden-und-experten-raetseln-wo-ist-das-virus-hin-id15915001.html}{verbreite}
  ``höchstens Panik''.
\item
  Stanford-Professor \textbf{John Ioannidis} erklärt
  in\href{https://twitter.com/cnn/status/1256579248342564865}{einem
  Interview mit CNN}, dass Covid19 eine ``verbreitete und milde
  Erkrankung'' sei, die für die Allgemeinbevölkerung gleich gefährlich
  oder sogar weniger gefährlich als die Influenza (Grippe) sei. Zu
  schützen seien insbesondere Patienten in Pflegeheimen und
  Krankenhäusern.
\item
  Stanford-Professor \textbf{Dr. Scott Atlas} erklärt in
  \href{https://www.facebook.com/cnn/posts/10160799274796509}{einem
  CNN-Interview}, dass man ``durch die falsche Idee, Covid19 stoppen zu
  müssen, eine katastrophale Situation im Gesundheitsbereich
  geschaffen'' habe. Es seien irrationale Ängste erzeugt worden, denn
  die Erkrankung sei ``insge­samt mild''. Deshalb gebe es auch ``absolut
  keinen Grund'' für umfangreiche Testungen in der Allgemeinbevölkerung,
  diese seien nur gezielt in Krankenhäusern und Pflegeheimen
  erforderlich.
\item
  Der britische Chefmediziner \textbf{Dr. Chris Whitty} erklärte, dass
  Covid19 ``für den größten Teil der Bevölkerung''
  \href{https://off-guardian.org/2020/05/15/watch-uk-chief-medic-confirms-again-covid19-harmless-to-vast-majority/}{keine
  Gefahr darstelle}. Die meisten Menschen würden nicht oder nur mild
  daran erkranken und selbst bei jenen, die stark erkranken, seien die
  Heilungschancen gut.
\item
  Stanford-Professor und Chemie-Nobelpreisträger \textbf{Michael Levitt}
  erklärt
  \href{https://www.telegraph.co.uk/news/2020/05/23/lockdown-saved-no-lives-may-have-cost-nobel-prize-winner-believes/}{in
  einem neuen Beitrag}, dass die Lockdowns keine Leben gerettet aber
  viele gekostet haben. Es habe sich weltweit ein ``Panik-Virus'' unter
  den Politikern verbreitet.
\item
  Professor \textbf{Karel Sikora} von der University of Buckingham
  argumentiert \href{https://www.youtube.com/watch?v=uk2YZfnsOPg}{in
  einem Interview}, dass die Angst letztlich ``viel mehr Menschen töten
  wird als das Virus'', darunter unbehandelte Herz- und Krebspatienten.
  Schulen sollten rasch geöffnet werden und Masken eine individuelle
  Entscheidung bleiben, da ihr Nutzen nicht belegt sei. Man solle zurück
  in ein ``altes Normal'' und kein ``neues Normal''. (\textbf{Hinweis}:
  Das Video mit Professor Sikora wurde von Youtube zeitweise gelöscht
  und erst nach Protesten wieder aufgeschaltet).\\
\item
  Der ehemalige israelische Gesundheitsminister \textbf{Professor Yoram
  Lass} hält die Lockdown-Maßnahmen für
  \href{https://www.spiked-online.com/2020/05/22/nothing-can-justify-this-destruction-of-peoples-lives/}{``völlig
  unverhältnismäßig''} und eine akute Bedrohung für weltweit hunderte
  Millionen Menschen. Covid19 sei vergleichbar mit einer Grippe-Epidemie
  und hätte niemals eine solche politische Zerstörung von Existenzen
  gerechtfertigt. Man habe die Menschen eingeschüchtert und
  ``gehirngewaschen''.
\item
  Die Oxford-Professorin für Epidemiologie, \textbf{Sunetra Gupta},
  argumentiert in einem
  \href{https://unherd.com/2020/05/oxford-doubles-down-sunetra-gupta-interview/}{neuen
  Interview}, dass die Letalität von Covid19 unter 0.1\% liegen dürfte
  und bereits sehr viele Menschen in Kontakt mit dem Virus kamen, ohne
  Antikörper auszubilden.
\item
  Ein \textbf{Mitarbeiter des deutschen Innenministeriums}, zuständig
  für Katastrophenschutz, verfasste zusammen mit externen Fachleuten ein
  rund
  \href{https://www.ichbinanderermeinung.de/Dokument93.pdf}{100-seitiges
  Analysepapier} zum Corona-Krisenmanagement, das Anfang Mai der Presse
  zugespielt wurde und für
  \href{https://www.zeit.de/politik/deutschland/2020-05/corona-pandemie-bekaempfung-massnahmen-innenministerium-verschwoerungstheorien-regierungsrat/komplettansicht}{heftige
  Reaktionen} sorgte. Im Analysepapier wird Covid-19 als ein ``globaler
  Fehlalarm'' beschrieben, da für die Allgemeinbevölkerung ``vermutlich
  zu keinem Zeitpunkt eine über das Normalmaß hinausgehende Gefahr''
  bestanden habe. Der Kollateralschaden durch den Lockdown sei
  inzwischen deutlich höher ist als der erkennbare Nutzen und übertreffe
  das Gefahrenpotential des Coronavirus bei weitem. So seien allein im
  März und April in Deutschland über eine Million Operationen nicht
  durchgeführt worden. Das Krisen­management und die Gefahrenanalyse
  hätten weitgehend versagt, die vom RKI gelieferten Daten seien als
  Grundlage für die Entscheidungsfindung zudem ``nicht zu gebrauchen''.
  Der Beamte wurde in der Folge beurlaubt und erhielt ein Dienstverbot,
  da er das Papier ``ohne Autorisierung'' erstellt habe.
\item
  Eine Gruppe um \textbf{Professor Sucharit Bhakdi} hat den Verein
  \href{https://www.mwgfd.de/}{Mediziner und Wissenschaflter für
  Gesundheit, Freiheit und Demokratie} (MWGFD) gegründet, dem sich
  bereits über 16,000 Unterstützer angeschlossen haben. Anfang Juni
  veröffentlichte die Gruppe einen
  \href{https://www.mwgfd.de/category/topaktuell/}{Aufruf an die
  Bundesregierung und alle Landesregierungen}, die verhängten
  Corona-Maßnahmen sofort und vollständig aufzuheben
  (\href{https://www.youtube.com/watch?v=rNev1_UBxvQ}{Video zur
  Initiative}). Das Ende Juni erscheinende Buch von Professor Bhakdi,
  \href{https://www.amazon.de/Corona-Fehlalarm-Daten-Fakten-Hintergr\%C3\%BCnde/dp/3990601911}{Corona
  Fehlalarm?}, ist allein aufgrund der Vorbestellungen bereits
  Amazon-Bestseller.
\item
  \textbf{Übersicht}:
  \href{https://www.rubikon.news/artikel/weltweiter-widerstand}{250
  weltweite Expertenstimmen gegen Corona-Dogmen}
\end{itemize}

\hypertarget{erfolgsbeispiele}{%
\subparagraph{\texorpdfstring{\textbf{Erfolgsbeispiele}}{Erfolgsbeispiele}}\label{erfolgsbeispiele}}

\textbf{Schweden}: Schweden verhängte keinen Lockdown, keine
Maskenpflicht und keine Grundschul­schließungen, sondern setzte
hauptsächlich auf die Eigenverantwortung und Kooperation der
Bevölkerung. Dieses Vorgehen
\href{https://www.nationalreview.com/2020/05/coronavirus-crisis-sweden-refused-lockdown-other-countries-following/}{bewährte
sich} und Schweden sah in der Allgemein­bevölkerung lediglich eine
geringe Sterblichkeit im Rahmen einer saisonalen Grippewelle.

Dennoch fiel die Gesamt­sterblichkeit in Schweden
\href{https://www.statista.com/statistics/1104709/coronavirus-deaths-worldwide-per-million-inhabitants/}{tatsächlich
höher} aus als in den skandinavischen Nachbarländern oder in
Deutschland, was von vielen Medien als ein angebliches ``Scheitern der
schwedischen No-Lockdown-Strategie'' dargestellt wurde.

Dabei wird jedoch zumeist verschwiegen, dass sich circa
\href{https://www.thelocal.se/20200525/swedish-death-toll-passes-4000-as-coronavirus-cases-in-care-homes-start-to-fall}{drei
Viertel der schwedischen Todesfälle} in Pflegeheimen und Pflegewohnungen
ereigneten, die einen gezielten Schutz erfordern und von einem
allgemeinen Lockdown der Gesellschaft gerade nicht profitieren. Das
Medianalter der schwedischen Todesfälle liegt mit 86 Jahren denn auch
weltweit vermutlich am höchsten.

Die schwedische Regierung hat sich zudem als eine der wenigen für den
ungenügenden Schutz der Pflegepatienten entschuldigt und eine
Untersuchung angekündigt, was von vielen Medien indes erneut als
angebliches ``Scheitern der No-Lockdown-Strategie''
\href{https://www.spiked-online.com/2020/06/08/no-sweden-has-not-changed-its-mind-about-lockdown/}{dargestellt
wurde}.

Tatsächlich blieb die Gesamtsterblichkeit in Schweden aber immer noch
\href{https://www.reuters.com/article/us-health-coronavirus-sweden-toll/coronavirus-pushes-swedish-deaths-to-highest-since-1993-in-april-idUSKBN22U1S4}{unter
den starken saisonalen Grippewellen} der letzten dreißig Jahre. Zudem
dürfte Schweden nun von einer sehr hohen
\href{https://medicalxpress.com/news/2020-05-stockholm-virus-antibodies-sweden.html}{natürlichen
Immunität} profitieren, insbesondere in Anbetracht der neuen Studien zur
effektiven Reichweite der Antikörper-Tests (siehe oben).

\includegraphics{https://swprs.files.wordpress.com/2020/06/sweden-mortality-comparison-1990-e1591998033581.jpg?w=650\&h=399}

\textbf{Florida}: Florida führte trotz zahlreicher Senioren nur geringe
Einschränkungen ein, sogar die beliebten Strände wurden frühzeitig
wieder geöffnet, was von vielen US-Medien heftig kritisiert wurde.
Dennoch schnitt Florida im Vergleich mit anderen Bundesstaaten
\href{https://www.statista.com/statistics/1109011/coronavirus-covid19-death-rates-us-by-state/}{sehr
gut ab} und zählte zuletzt bei 21 Millionen Einwohnern ca.
\href{https://www.clickorlando.com/news/local/2020/03/16/interactive-map-shows-florida-coronavirus-cases-in-real-time/}{2300
Todesfälle}, was in etwa der Mortalität von Deutschland entspricht.

Der Gouverneur
\href{https://www.nationalreview.com/2020/05/coronavirus-crisis-ron-desantis-florida-covid-19-strategy/}{erklärte
in einem Interview}, dass man aufgrund der Zahlen aus Südkorea und
Italien und \emph{entgegen der Mediendarstellung} frühzeitig erkannt
habe, dass Covid19 nur für eine sehr kleine Risikogruppe gefährlich ist
und deshalb die Pflegeheime bestmöglich geschützt habe. Pflegeheime
seien im Sinne der Prävention sogar \emph{noch wichtiger} gewesen als
die Kliniken selbst, und diese Strategie habe sich bewährt. Bereits Ende
Mai gab der Gouverneur bekannt, dass Sommerlager und Jugendaktivitäten
\href{https://www.usnews.com/news/best-states/florida/articles/2020-05-22/miami-beach-officials-agree-to-reopen-hotels-beaches-june-1}{uneingeschränkt
durchgeführt} werden können.

\textbf{Japan}: Japan registrierte einige der ersten Covid19-Fälle
außerhalb Chinas, führte jedoch keinen Lockdown ein. Schon Ende März
fragte die Japan Times:
\href{https://www.japantimes.co.jp/news/2020/03/20/national/coronavirus-explosion-expected-japan/}{``Wo
bleibt die Coronavirus-Explosion?''}.~ Bloomberg berichtet nun, dass
diese
\href{https://www.bloomberg.com/news/articles/2020-05-22/did-japan-just-beat-the-virus-without-lockdowns-or-mass-testing}{bis
heute ausblieb}: Es gab keine Bewegungseinschränkungen, Restaurants und
Friseure blieben offen, es gab keine ``Tracking-Apps'' und keine
Maßentests der Allgemeinbevölkerung. Dennoch -- oder deshalb -- weist
Japan nun im Vergleich mit den G7-Industrieländern bei weitem am
wenigsten Todesfälle auf (\textless{}1000).

Manche argumentieren, die Atemschutzmasken seien für Japans Erfolg
entscheidend gewesen. Doch Atemschutzmasken sind in Japan freiwillig und
hielten auch den Ausbruch in der chinesischen Stadt Wuhan nicht auf,
während Schweden, Florida und andere erfolgreiche Regionen keine Masken
in der Allgemeinbevölkerung einsetzten.

\textbf{Weißrussland}: Weißrussland hat von allen Industrienationen wohl
am wenigsten Einschränkungen eingeführt und sogar Großveranstaltungen
wie die 75-Jahre-Feierlichkeiten zum Ende des Zweiten Weltkriegs
\href{https://www.youtube.com/watch?v=NZliKGoUN4E}{durchgeführt}.
Dennoch zählt Weißrussland auch nach über drei Monaten offiziell weniger
als
\href{https://en.wikipedia.org/wiki/COVID-19_pandemic_in_Belarus}{300
Covid-Todesfälle}. Langzeitpräsident Lukaschenko, der Corona wiederholt
als ``Psychose'' bezeichnet hatte, erklärte Mitte Mai, die Hauptstadt
Minsk habe den Höhepunkt bereits
\href{https://eng.belta.by/president/view/belarus-president-coronavirus-situation-on-the-mend-130571-2020/}{überschritten}.
Die Entscheidung, Covid19-Fälle wie eine normale Lungen­ent­zündung zu
behandeln, sei richtig gewesen. Ob die weißrussischen Zahlen wirklich
korrekt sind, wird aber letztlich erst die Statistik zur
Gesamtsterblichkeit zeigen können.

\hypertarget{weitere-meldungen-1}{%
\subparagraph{\texorpdfstring{\textbf{Weitere
Meldungen}}{Weitere Meldungen}}\label{weitere-meldungen-1}}

\begin{itemize}
\tightlist
\item
  \textbf{Übertragungswege}: Ein neuer Bericht der US-Gesundheitsbehörde
  CDC kommt
  \href{https://www.yahoo.com/lifestyle/cdc-coronavirus-mainly-spreads-through-persontoperson-contact-and-does-not-spread-easily-on-contaminated-surfaces-153317029.html}{zum
  Ergebnis}, dass das Virus hauptsächlich durch direkten Personenkontakt
  übertragen wird und sich ``nicht leicht auf Oberflächen verbreiten
  kann.'' Bereits der deutsche Virologe Hendrik Streeck
  \href{https://www.telegraph.co.uk/news/2020/04/02/no-proof-coronavirus-can-spread-shopping-says-leading-german/}{konnte
  nachweisen}, dass sich das neue Coronavirus nicht oder kaum über
  Gegenstände und in der Luft schwebende Aerosole verbreitet.
\item
  \textbf{Abstandsregeln}: Island hat die Abstandregeln Ende Mai
  \href{https://icelandmonitor.mbl.is/news/news/2020/05/25/two_meter_rule_optional_in_iceland/}{für
  optional erklärt} und Bars und Clubs wieder geöffnet. Die Schweiz
  wandelte die Abstandsvorschriften in eine freiwillige
  \href{https://www.20min.ch/story/bund-schafft-die-2-meter-abstands-busse-heimlich-ab-763738047957}{Empfehlung}
  um. Eine
  \href{https://www.cochranelibrary.com/cdsr/doi/10.1002/14651858.CD006207.pub4/full}{Cochrane-Untersuchung
  von 2011} ergab bereits, dass es für die Wirksamkeit von ``social
  distancing'' Maßnahmen bislang kaum Evidenz gebe.
\item
  \textbf{Operationen}: Laut einem Artikel im British Journal of Surgery
  wurden aufgrund der Corona-Maßnahmen während 12 Wochen
  weltweit~\href{https://bjssjournals.onlinelibrary.wiley.com/doi/abs/10.1002/bjs.11746}{rund
  28 Millionen Operationen} abgesagt oder verschoben, darunter auch
  viele Krebsoperationen.
\item
  \textbf{Lebensjahre}: Eine Auswertung von vier US-Professoren kommt
  zum Ergebnis, dass der Lockdown in den USA etwa
  \href{https://thehill.com/opinion/healthcare/499394-the-covid-19-shutdown-will-cost-americans-millions-of-years-of-life}{doppelt
  soviele Lebensjahre} kosten wird wie Covid-19 und damit auch aus rein
  medizinischer Sicht eine stark kontraproduktive Maßnahme war.
\item
  \textbf{Virustest}: Ein deutscher Mathematiker erklärt, warum die
  geringen verbleibenden Infektionszahlen in vielen Ländern selbst bei
  ziemlich genauen Virentests größtenteils aus
  \href{https://multipolar-magazin.de/artikel/warum-die-pandemie-nicht-endet}{falschen
  positiven Resultaten} bestehen und die Pandemie damit scheinbar nie
  ganz endet.
\item
  \textbf{``Zweite Welle''}: Studien zu einer ``zweiten Welle'' gehen
  teilweise
  von~\href{https://www.heise.de/tp/features/Fellay-Studie-Zweite-Corona-Welle-4726303.html}{sehr
  unrealistischen Annahmen} aus wie etwa einem konstanten Infektions-
  und Sterberisiko über alle Altersgruppen. Dennoch zeigt das Beispiel
  der
  \href{https://www.britannica.com/event/Hong-Kong-flu-of-1968}{Hongkong-Grippe
  von 1968}, dass sich die globale Ausbreitung von Pandemien durchaus
  über einige Saisons hinziehen kann.
\item
  \textbf{Italien}: In Mailand hatten bereits Mitte Februar, d.h. noch
  vor Ausbruch der dortigen Epidemie,
  \href{https://www.medrxiv.org/content/10.1101/2020.05.11.20098442v2}{knapp
  5\% der Bevölkerung} Antikörper gegen Covid19. Dies deutet erneut
  daraufhin, dass das Virus bereits früher als bisher angenommen in
  Europa zirkulierte, ohne aufzufallen.
\item
  \textbf{Arbeitslosigkeit}: Die Internationale Arbeitsagentur ILO
  rechnet damit, dass aufgrund der politischen Corona-Maßnahmen die
  Hälfte der weltweiten Arbeitnehmer oder 1.6 Milliarden Menschen
  \href{https://www.theguardian.com/world/2020/apr/29/half-of-worlds-workers-at-immediate-risk-of-losing-livelihood-due-to-coronavirus}{akut
  vom Verlust ihrer Lebensgrundlagen} bedroht sind.
\item
  \textbf{Faktencheck}:
  \href{https://www.infosperber.ch/Artikel/Gesundheit/13-irrefuhrende-und-falsche-Behauptungen-zur-Corona-Epidemie}{13
  irreführende und falsche Behauptungen zur Corona-Epidemie}
\item
  \textbf{Rückblick}: Warum das Leben während der Grippe-Pandemien
  \href{https://nypost.com/2020/05/16/why-life-went-on-as-normal-during-the-killer-pandemic-of-1969/}{von
  1968/69} (Hongkong-Grippe) und
  \href{http://www.ronpaulinstitute.org/archives/featured-articles/2020/may/06/the-great-pandemic-of-1957-and-why-nobody-remembers-it/}{von
  1957} (Asiatische Grippe) weitestgehend normal weiterging.
\end{itemize}

\hypertarget{hat-der-lockdown-leben-gerettet}{%
\subparagraph{\texorpdfstring{\textbf{Hat der Lockdown Leben
gerettet?}}{Hat der Lockdown Leben gerettet?}}\label{hat-der-lockdown-leben-gerettet}}

Viele Medien berichteten von einer Studie des Imperial College London,
wonach die Lockdowns in Europa angeblich
\href{https://www.bbc.com/news/health-52968523}{``3 Millionen Leben
gerettet''} hätten. In Wirklichkeit verglich das Imperial College London
einfach die
\href{https://www.telegraph.co.uk/technology/2020/05/16/neil-fergusons-imperial-model-could-devastating-software-mistake/}{unrealistischen
Vorhersagen} des eigenen Modells mit der Realität. Besonders deutlich
wird das am Beispiel Schwedens, das auch ohne Lockdown nur einen
Bruchteil der Todesfälle hatte, die vom Imperial College Modell
prognostiziert wurden (siehe Grafik).

\includegraphics{https://swprs.files.wordpress.com/2020/06/coronavirus-deaths-in-sweden-actual-deaths-vs-ferguson-projections-may-8-update.png?w=600\&h=470}

\hypertarget{zur-rolle-der-medien}{%
\subparagraph{\texorpdfstring{\textbf{Zur Rolle der
Medien}}{Zur Rolle der Medien}}\label{zur-rolle-der-medien}}

Die meisten traditionellen Medien, die fast alle in
\href{https://swprs.org/the-american-empire-and-its-media/}{geopolitische
Netzwerke} eingebunden sind, verbreiteten während der Corona-Zeit
überwiegend
\href{https://www.infosperber.ch/Artikel/Gesundheit/13-irrefuhrende-und-falsche-Behauptungen-zur-Corona-Epidemie}{Angstpropaganda},
wie man dies sonst im Zusammenhang mit
\href{https://swprs.org/the-syria-deception/}{Angriffskriegen} oder
angeblichen
\href{https://www.motherjones.com/politics/2013/01/terror-factory-fbi-trevor-aaronson-book/}{Terroranschlägen}
beobachtet.

Dabei wurde das Risiko für die Allgemeinbevölkerung stark
\href{https://www.infosperber.ch/index.cfm?go=Artikel/Gesundheit/13-irrefuhrende-und-falsche-Behauptungen-zur-Corona-Epidemie/\&g=ad}{überzeichnet},
die Regierungspolitik
\href{https://www.zora.uzh.ch/id/eprint/186723/1/jarren_corona.pdf}{kaum
hinterfragt}, die Situation in Krankenhäusern
\href{https://www.globalresearch.ca/truth-behind-refrigerated-morgue-truck-stories/5711475}{dramatisiert},
manipulative Bilder
\href{https://nypost.com/2020/04/01/cbs-admits-to-using-footage-from-italy-in-report-about-nyc/}{verwendet},
Kampagnen
\href{https://www.youtube.com/watch?v=oQWRCECbN-Y}{inszeniert}, und
Protestierende systematisch als ``Idioten'' diffamiert.

Einige konservative Medien kritisierten zwar die wirtschafts­schädlichen
Lockdown-Maßnahmen. Die wirkliche Frage wird aber sein, ob sie auch die
nun geplanten Überwachungs­maßnahmen wie die gesamt­gesell­schaftliche
Kontaktverfolgung kritisieren werden (siehe unten).

Die meisten \href{https://swprs.org/medien-navigator/}{unabhängigen
Medien} realisierten früher oder später, dass das Risiko des Coronavirus
übertrieben und politisch instrumentalisiert wurde. Nur einige wenige
unabhängige Medien realisierten das nicht, da ihnen vielleicht das
medizinische Hintergrundwissen fehlte.

Einige Analysten verglichen Covid-19 zudem
\href{https://digwithin.net/2020/06/03/coronavirus-scare/}{mit einer
Psychologischen Operation}, bei der die Angst vor dem Virus genutzt
wird, um politische und gesellschaftliche Veränderungen herbeizuführen.

US-Plattformen wie Google, Youtube, Facebook und Twitter betrieben bei
Corona-Themen eine
\href{https://nypost.com/2020/05/16/youtube-censors-epidemiologist-knut-wittkowski-for-opposing-lockdown/}{umfangreiche
Zensur}, indem kritische Standpunkte selbst von Ärzten gelöscht oder
ihre Verbreitung eingeschränkt wurde, wie dies bei geopolitischen Themen
seit langem üblich ist.

Moderne Mediennutzer haben jedoch die Möglichkeit, manipulationsfreie
Suchmaschinen wie z.B. DuckDuckGo und unabhängige Videoplattformen wie
z.B. Bitchute zu nutzen, sowie auf propagandistischen Medienseiten
generell einen Werbe- und Trackingblocker einzusetzen.

\includegraphics{https://swprs.files.wordpress.com/2020/06/sikora-interview-youtube.jpg?w=600\&h=338}

\hypertarget{politische-entwicklungen}{%
\subparagraph{\texorpdfstring{\textbf{Politische
Entwicklungen}}{Politische Entwicklungen}}\label{politische-entwicklungen}}

Zahlreiche Beobachter haben bereits darauf aufmerksam gemacht, dass die
überwiegend politisch herbeigeführte ``Corona-Krise'' für weitreichende
gesellschaftliche und ökonomische Veränderungen instrumentalisiert wird
oder werden könnte.

NSA-Whistleblower \textbf{Edward Snowden} warnte bereits im März, dass
Regierungen die vorübergehende Corona-Krise zum Anlass oder Vorwand für
den permanenten Ausbau der
\href{https://www.youtube.com/watch?v=-pcQFTzck_c}{gesellschaftlichen
Überwachung und Kontrolle} nehmen und damit eine ``Architektur der
Unterdrückung'' errichten.

Zu den derzeit diskutierten oder bereits eingeführten Maßnahmen zählen
insbesondere:

\begin{enumerate}
\def\labelenumi{\arabic{enumi}.}
\tightlist
\item
  Die Einführung von Applikationen zur gesamtgesellschaftlichen
  \textbf{Kontaktverfolgung}
\item
  Der Aufbau von \textbf{Einheiten zur Durchsetzung} der Verfolgung und
  Isolierung von Bürgern
\item
  Die Einführung von digitalen \textbf{biometrischen Ausweisen}, über
  die die Teilnahme an gesellschaftlichen und beruflichen Aktivitäten
  kontrolliert und reguliert werden kann.
\item
  Die erweiterte Kontrolle von \textbf{Reiseverkehr und Zahlungsverkehr}
  (Bargeldabschaffung).
\item
  Die Schaffung von gesetzlichen Grundlagen für einen Zugriff auf die
  \textbf{biologischen Systeme} der Bürger durch Regierungen oder
  Konzerne (durch sog. ``Pflichtimpfungen'').
\end{enumerate}

Über 600 Wissenschaftler haben vor einer ``beispiellosen Überwachung der
Gesellschaft'' durch problematische \textbf{Apps zur
Kontakt­­verfolgung}
\href{https://www.esat.kuleuven.be/cosic/sites/contact-tracing-joint-statement/}{gewarnt}.
In einigen Ländern wird diese Kontakt­ver­folgung bereits direkt vom
Geheimdienst
\href{https://www.jewishpress.com/news/the-courts/state-to-high-court-even-more-shin-bet-involvement-in-fighting-the-coronavirus/2020/04/14/}{durchgeführt}.
Weltweit kam es bereits zur Überwachung der Zivilbevölkerung
\href{https://off-guardian.org/2020/04/25/50-headlines-darker-more-of-the-new-normal/}{durch
Drohnen} und zu teilweise massiver Polizeigewalt.

Im Mai fügten \textbf{Apple und Google} eine Schnittstelle zur
Kontaktverfolgung in die Betriebs­systeme von
\href{https://www.bloomberg.com/news/articles/2020-04-10/apple-google-bring-covid-19-contact-tracing-to-3-billion-people}{drei
Milliarden Mobiltelefonen} ein, die nun von nationalen Behörden genutzt
werden kann.

Dies obschon eine WHO-Studie zu pandemischer Influenza 2019 zum Ergebnis
kam, dass eine Kontaktverfolgung aus medizinischer Sicht nicht sinnvoll
und
\href{https://apps.who.int/iris/bitstream/handle/10665/329438/9789241516839-eng.pdf\#page=9}{``unter
keinen Umständen zu empfehlen''} ist. Tatsächlich ist der
epidemiologische Nutzen solcher Apps höchst zweifelhaft.

Oftmals wird argumentiert, die Applikationen zur Kontaktverfolgung
würden ``freiwillig'' bleiben und seien ``datenschutzkonform''. Doch
beides ist in der Praxis nicht wirklich zutreffend.

In mehreren Ländern ist die Nutzung der Applikationen bereits für
gewisse Aktivitäten obligatorisch. So verlangen in \textbf{Indien}
verschiedene Arbeitgeber, Verwaltungen, Vermieter und
Transport­unter­nehmen
\href{https://www.technologyreview.com/2020/05/07/1001360/india-aarogya-setu-covid-app-mandatory/}{die
Tracing-App}, die bereits 100 Millionen Inder installiert haben. In
\textbf{Argentinien} müssen alle, die sich im ``öffentlichen Raum''
aufhalten,
\href{https://www.heise.de/tp/features/CuidAR-Argentinien-ueberwacht-mit-einer-App-4720143.html}{die
Kontaktverfolgung aktivieren}.

Auch einige \textbf{deutsche Politiker} sprachen sich bereits für eine~
\href{https://www.faz.net/aktuell/politik/wer-die-corona-app-hat-soll-zuerst-wieder-ins-restaurant-duerfen-16759932.html}{Bevorzugung
von App-Nutzern} bei Reisen oder Restaurantbesuchen aus. Der
\textbf{israelische Premier} Netanyahu sprach seinerseits von der
\href{https://norberthaering.de/die-regenten-der-welt/netanyahu-sensoren/}{Verwendung
von ``Sensoren''} zur Abstandskontrolle bei Kindern.

In \textbf{Singapur} haben sich die ``freiwillige'' App weniger Personen
als erwartet installiert, weshalb die Regierung diese nun für gewisse
öffentliche Räume oder Dienste
\href{https://www.letemps.ch/economie/singapour-tracage-app-degenere-surveillance-masse}{ebenfalls
obligatorisch} machen möchte. In einigen Parks wird die
Abstandskontrolle zudem
\href{https://www.youtube.com/watch?v=2DJmIjKtVkA}{von DARPA-Robotern
überwacht}.

In \textbf{Australien} werden Menschen, die die App zur
Kontaktverfolgung nicht nutzen möchten, von Medien
\href{https://www.news.com.au/world/coronavirus/australia/people-who-refuse-to-download-the-covidsafe-virus-tracing-app-are-the-new-antivaxxers/news-story/541c36fe5cdb56eb1a098b0b9a0dddcc}{als
Idioten und Gefährder} beschimpft und damit sozial unter Druck gesetzt.

Singapur: Eine DARPA-Roboterhund kontrolliert das Social Distancing
(\href{https://www.youtube.com/watch?v=2DJmIjKtVkA}{CNA})

Auch der \textbf{Datenschutz} der angeblich ``dezentralen''
Kontaktverfolgung ist problematisch. Der niederländische IT-Professor
Jaap-Henk Hoepmann
\href{http://blog.xot.nl/2020/04/11/stop-the-apple-and-google-contact-tracing-platform-or-be-ready-to-ditch-your-smartphone/}{erklärte
schon im April,} dass auch vermeintlich dezentrale Lösungen sehr leicht
für eine Überwachung genutzt werden können.

Der Präsident der deutschen Gesellschaft für Informatik vermutete
aufgrund der Geschwindkeit der Einführung zudem, dass die Funktionen in
den Geräten
\href{https://www.heise.de/news/Informatiker-Die-Corona-App-ist-wie-ein-trojanisches-Pferd-4764560.html}{``längst
schon drin waren''} und nur noch ein wenig ``Finetuning'' erforderlich
war. Die Apps sieht er als ein \textbf{``trojanisches Pferd''}.

Parallel zur Einführung der Applikationen haben zahlreiche Länder damit
begonnen, \textbf{Spezialeinheiten zur Kontaktverfolgung und Isolierung}
der Bürger aufzubauen.

In den \textbf{USA} wurden dazu
\href{https://heavy.com/news/2020/05/hr-6666-trace-bill-bobby-rush/}{milliardenschwere
Gesetzesvorhaben} zum Aufbau eines nationalen ``Corona Testing and
Tracing Corps'' mit bis zu
\href{https://www.washingtonpost.com/opinions/2020/05/29/corona-corps-could-fight-virus-youth-unemployment/}{180,000
Mitgliedern} eingebracht. Die Bundesstaaten
\href{https://www.jhsph.edu/news/news-releases/2020/johns-hopkins-and-bloomberg-philanthropies-with-new-york-state-launch-online-course-to-train-army-of-contact-tracers-to-slow-spread-of-COVID-19.html}{New
York} und
\href{https://timesofsandiego.com/tech/2020/05/06/california-training-a-20000-person-army-of-coronavirus-tracers/}{Kalifornien}
sind bereits dabei, ``Kontaktverfolgungs-Armeen'' mit je bis zu 20,000
Mitgliedern aufzubauen. Im Bundesstaat
\href{https://lynnwoodtimes.com/2020/05/12/governor-inslee-lays-out-statewide-contact-tracing-plan-for-covid-19/}{Washington}
soll dabei die Nationalgarde mitwirken und wer sich nicht ``freiwillig''
isoliert, kann dazu gezwungen werden.

\textbf{Italien} kündigte ebenfalls den Aufbau eines Korps
\href{https://www.thelocal.it/20200525/italy-seeks-60000-volunteers-to-enforce-coronavirus-rules}{mit
60,000 Freiwilligen} an, die
\href{https://www.nzz.ch/zuerich/coronavirus-in-zuerich-contact-tracing-ist-gut-angelaufen-ld.1556846}{Schweiz}
und andere Länder haben mit dem Aufbau solcher Einheiten bereits
begonnen. In \textbf{Deutschland} kam es bereits zu Massentestungen in
Wohnhäusern unter Androhung von
\href{https://www.youtube.com/watch?v=6ZQFpnskP8g}{polizeilichem Zwang}.

Die \textbf{Software} für die gesamtgesellschaftliche Kontaktverfolgung
in den USA, in
\href{https://www.telegraph.co.uk/technology/2020/05/16/inside-story-cia-backed-palantir-embedded-nhs-socialite-running/}{Großbritannien}
und womöglich in weiteren europäischen Ländern wird von der
\href{https://techcrunch.com/2020/04/01/palantir-coronavirus-cdc-nhs-gotham-foundry/}{CIA-nahen
Technologiefirma Palantir} des US-Milliardärs Peter Thiel
bereitgestellt. In Israel wird zur Kontaktverfolgung Software der
Spionagefirma NSO
\href{https://www.techdirt.com/articles/20200402/14261944226/controversial-spyware-vendor-nso-group-is-helping-israeli-government-spy-own-citizens.shtml}{verwendet},
mit der auch Menschenrechtler überwacht werden.

Eine \textbf{Whistleblowerin}, die an einem Ausbildungsprogramm für
Kontaktverfolger in den USA
\href{https://www.youtube.com/watch?v=qFUyZWw7qoc}{teilgenommen hatte},
beschrieb dieses als ``totalitär'' und eine ``Gefahr für die
Gesellschaft''.

Alle diese Maßnahmen erfolgen, obschon der epidemiologische Nutzen
unklar ist und sich die WHO explizit \emph{gegen} eine Kontaktverfolgung
\href{https://apps.who.int/iris/bitstream/handle/10665/329438/9789241516839-eng.pdf\#page=9}{aussprach},
die sonst vor allem bei schweren sexuellen Krankheiten oder
Lebensmittelvergiftungen zum Einsatz kommt, die sich im Unterschied zu
häufigen Atemwegserkrankungen leicht nachverfolgen lassen.

Zusätzlich zu Applikationen und Spezialeinheiten zur Kontaktverfolgung
gibt es weiterhin konkrete
\href{https://www.msn.com/en-us/news/world/the-uk-government-is-in-talks-with-facial-recognition-firms-to-develop-covid-19-immunity-passports/ar-BB12J6It}{Projekte
oder Planungen} für \textbf{``Immunitätsausweise''}, mit denen sich
beispielsweise die Reise- und Arbeitstätigkeit der Bevölkerung
regulieren ließe. Tatsächlich plante die EU die Einführung eines solchen
europäischen Impfpasses
\href{https://off-guardian.org/2020/05/22/report-eu-planning-vaccination-passport-since-2018/}{bereits
seit 2018}.

Solche ``Impfausweise'' sind wiederum verbunden mit einem weltweiten
\textbf{``Impfprogramm''}, an dem derzeit ebenfalls gearbeitet wird. So
forderte US-Milliardär und Impfstoff-Investor Bill Gates beispielsweise
eine Corona-Impfung für
\href{https://www.businessinsider.com/bill-gates-14-billion-doses-coronavirus-vaccine-may-be-needed-2020-5}{``sieben
Milliarden Menschen''}. AstraZeneca bereitet die Produktion von
\href{https://www.sciencealert.com/2-billion-doses-of-oxford-s-potential-coronavirus-vaccine-could-soon-become-available}{zwei
Milliarden Dosen} des weiterhin ungetesteten Oxford-Impfstoffes vor.

Aus strategischer Sicht ermöglicht ein solches globales Impfprogramm
längerfristig den Zugriff auf die biologischen Systeme der Menschen,
darunter insbesondere das Immun- und Nervensystem sowie das genetische
und reproduktive System.

Im \textbf{ökonomischen Bereich} zeichnet sich derweil ein
Digitalisierungs- und Zentrali­sierungs­schub zugunsten einiger weniger
Technologiekonzerne ab, wie dies die amerikanische \emph{National
Security Commission on Artificial Intelligence (NSCAI)} unter Leitung
des ehemaligen Google-CEO Eric Schmidt bereits 2019
\href{http://unlimitedhangout.com/2020/05/reports/techno-tyranny-how-the-us-national-security-state-is-using-coronavirus-to-fulfill-an-orwellian-vision/}{in
einem Strategiepapier forderte}, um mit China mithalten zu können.\\

Das World Economic Forum (WEF) Davos, das zusammen mit der Gates
Stiftung und der Johns Hopkins Universität bereits an der bekannten
\href{https://www.centerforhealthsecurity.org/event201/scenario.html}{Coronavirus-Pandemie-Übung
``Event 201''} vom Oktober 2019 beteiligt war, rief diesbezüglich einen
globalen \href{https://www.youtube.com/watch?v=u5pxhSnDr4U}{``Great
Reset''} aus, um die ökonomischen und gesellschaftlichen Strukturen für
das 21. Jahrhundert vorzubereiten.

In einem \href{https://veritasliberabitvos.info/appeal/}{Offenen Brief}
warnten derweil mehrere Kardinäle und Bischöfe der \textbf{katholischen
Kirche}, dass unter dem Vorwand des Coronavirus eine weltweite Panik
ausgelöst worden sei, um ``inakzeptable Formen der globalen Überwachung
und Kontrolle der Bevölkerung'' einzuführen.

Die Idee, dass eine Pandemie für den Ausbau globaler Überwachungs- und
Kontroll­instrumente genutzt werden kann, ist nicht neu. Bereits 2010
beschrieb die amerikanische \textbf{Rockefeller Foundation} in einem
\href{https://swprs.files.wordpress.com/2020/04/rockefeller-foundation-scenarios-2010.pdf}{Arbeitspapier
zu technologischen und gesellschaftlichen Zukunfts­entwicklungen} ein
``Gleichschritt-Szenario'' mit einer Grippe-Pandemie aus China, in dem
die heutigen Entwicklungen überraschend präzise antizipiert wurden (ab
Seite 18).

Es gibt aber auch Reaktionen der Bevölkerung: So haben zum Beispiel
\href{https://www.youtube.com/watch?v=BZo3BYD2qpI}{Spanien},
\href{https://www.youtube.com/watch?v=P2Tcs8RzK1k}{Italien} und
\href{https://www.youtube.com/watch?v=Iqm456JIaBQ}{Deutschland}
Demonstrationen für Grundrechte mit zehntausenden Menschen erlebt.

\textbf{Siehe auch}:
\href{https://www.wired.com/story/inside-the-nsas-secret-tool-for-mapping-your-social-network/}{Inside
the NSA's Secret Tool for Mapping Your Social Network} (Wired)

\hypertarget{kreative-beitruxe4ge}{%
\subparagraph{**Kreative Beiträge}\label{kreative-beitruxe4ge}}

**

\begin{itemize}
\tightlist
\item
  Video: \href{https://archive.org/details/theylivecvversion}{They Live
  -- Coronavirus Edition} (Trigger Happy Media)
\item
  Video: \href{https://www.youtube.com/watch?v=wfGAktuU93s}{Out of Touch
  -- Run for your life} (Kevin James)
\item
  Video (EN): \href{https://www.youtube.com/watch?v=QcUAG6t5aN8}{What
  It's Like to Believe Everything the Media Tells You} (JP)
\item
  Video (DE):
  \href{http://www.nsfwyoutube.com/watch?v=81_xOIgoXkQ}{Coronavirus --
  The Masterplan?} (Björn Templ)
\item
  Video (DE/IT):
  \href{https://www.youtube.com/watch?v=qaXDC4AQR9s}{Neapel: Die
  Freiheit hat keinen Preis} (MCC)
\item
  Video (Schweizerdeutsch):
  \href{https://www.youtube.com/watch?v=L7KftkC9ctU}{Billy, Billy,
  Billy} (P.G. feat. G-Are)
\end{itemize}

\includegraphics{https://swprs.files.wordpress.com/2020/06/landing-ai-social-distancing-detector.jpg?w=736\&h=335}

\hypertarget{b-luxe4nder-und-regionen-1}{%
\subparagraph{\texorpdfstring{\textbf{B. Länder und
Regionen}}{B. Länder und Regionen}}\label{b-luxe4nder-und-regionen-1}}

\hypertarget{skandinavien}{%
\subparagraph{\texorpdfstring{\textbf{Skandinavien}}{Skandinavien}}\label{skandinavien}}

\begin{itemize}
\tightlist
\item
  \textbf{Dänemark}: In Dänemark wurde durch geleakte E-Mails bekannt,
  dass sich die Gesundheitsbehörde im März anders als politisch
  dargestellt
  \href{https://www.thelocal.dk/20200529/leaked-emails-show-how-denmarks-pm-steam-rollered-her-own-health-agency}{\emph{gegen}
  einen Lockdown aussprach} und schrieb: ``Die dänische
  Gesundheitsbehörde ist weiterhin der Ansicht, dass Covid-19 nicht als
  allgemein gefährliche Krankheit bezeichnet werden kann, da es weder
  einen normalerweise schwerwiegenden Verlauf noch eine hohe
  Sterblichkeitsrate aufweist.'' Außerdem wurde eine tatsächlich tiefere
  Reproduktionszahl aus politischen Gründen nicht veröffentlicht, um den
  Lockdown zu begründen. Das dänische Parlament hat nun eine
  \textbf{Untersuchung der Regierungspolitik} durch eine
  Expertenkommission beschlossen.\\
\item
  \textbf{Norwegen}: Die norwegische Premierministerin
  \href{https://www.telegraph.co.uk/news/2020/05/30/coronavirus-norway-wonders-should-have-like-sweden/}{räumte
  öffentlich ein}, dass sie im März in Panik verfallen sei und die
  meisten der beschlossenen Lockdown-Maßnahmen vermutlich nicht
  erforderlich gewesen wären. Auch in Norwegen wurde bekannt, dass die
  Reproduktionszahl bereits vor dem Lockdown gegen den stabilen Wert von
  1 fiel. Im Falle einer ``zweiten Welle'' müsse eine wesentlich
  sanftere Strategie ohne Lockdown gewählt werden.
\item
  \textbf{Schweden}: Schweden kam trotz internationalem Druck gut ohne
  Lockdown durch die Corona-Zeit (siehe oben): Die Gesamtsterblichkeit
  bewegte sich im Rahmen
  \href{https://www.reuters.com/article/us-health-coronavirus-sweden-toll/coronavirus-pushes-swedish-deaths-to-highest-since-1993-in-april-idUSKBN22U1S4}{früherer
  starker Grippewellen}. 75\% der Todesfälle fielen in Pflegeheimen und
  Pflegewohnungen vor, wozu die Regierung eine
  \href{https://www.bloomberg.com/news/articles/2020-05-08/sweden-starts-criminal-probe-into-care-home-after-virus-deaths}{Untersuchung}
  einleitete. Internationale Medien versuchten dies als ``Scheitern der
  schwedischen Strategie'' darzustellen, was indes nicht richtig ist, da
  Pflegeheime einen gezielten Schutz benötigen und von einem allgemeinen
  gesellschaftlichen Lockdown gerade nicht profitieren.
\item
  \textbf{Schulen}: Sowohl Finnland als auch Dänemark stellten nach der
  Wiedereröffnung ihrer Schulen
  \href{https://www.telegraph.co.uk/news/2020/06/04/no-rise-finlands-coronavirus-infection-rate-since-schools-reopened/}{keine
  Zunahme} der Corona-Fälle fest. Schweden behielt seine Grundschulen
  offen.
\end{itemize}

\hypertarget{schweiz-1}{%
\subparagraph{\texorpdfstring{\textbf{Schweiz}}{Schweiz}}\label{schweiz-1}}

\includegraphics{https://swprs.files.wordpress.com/2020/06/schweiz-todesfaelle-2010-2020_woche_22.png?w=736\&h=347}

Medizinische Aspekte:

\begin{itemize}
\tightlist
\item
  Die Schweiz befindet sich nach ca. fünf Wochen relativer
  Übersterblichkeit seit Mitte Mai bereits in einer
  \href{https://swprs.files.wordpress.com/2020/06/schweiz-todesfaelle-2010-2020_woche_22.pdf}{Untersterblichkeit}.
  Die kumulierte Sterblichkeit seit Anfang Jahr liegt im Bereich einer
  \textbf{üblichen Grippesaison} und weit unter der starken Grippesaison
  von 2015 (siehe Grafik oben). Rund 50\% der Todesfälle ereigneten sich
  in
  \href{https://www.nzz.ch/zuerich/coronavirus-zuerich-aendert-nun-das-testregime-in-heimenauch-viele-aeltere-covid-19-infizierte-entwickeln-keine-symptome-zuerich-aendert-nun-das-testregime-in-heimen-ld.1552089}{Pflegeheimen},
  die von einem Lockdown nicht profitierten. Der Altersmedian der
  Todesfälle liegt bei ca. 84 Jahren und damit sogar etwas über der
  durchschnittlichen Schweizer Lebenserwartung.
\item
  Im \textbf{Kanton Zürich} kam es zu insgesamt 130 testpositiven
  Todesfällen. Mehr als die Hälfte davon (78) erfolgte in Pflegeheimen.
  Wiederum etwas mehr als die Hälfte dieser Todesfälle (40) erfolgte
  \href{https://www.republik.ch/2020/05/28/toedlicher-zufall}{in zwei
  Pflegeheimen}, die Covid19-Patienten aus Krankenhäusern aufnehmen
  mussten und diese teilweise ungenügend isolieren konnten.
\item
  Die ETH Zürich hat ihre Studie, wonach der Rückgang der Ausbreitung
  von Covid19 bereits vor dem \textbf{Lockdown} einsetzte, inzwischen
  \href{https://www.luzernerzeitung.ch/schweiz/die-schweiz-haette-die-kurve-auch-mit-weniger-einschraenkungen-gekriegt-war-der-lockdown-uebertrieben-ld.1221111}{mehrmals
  umformuliert}, um dem Bundesrat nicht mehr offen zu widersprechen.
  Doch das Ergebnis
  \href{https://infekt.ch/2020/04/sind-wir-tatsaechlich-im-blindflug/}{bleibt
  dasselbe}: Der Lockdown war medizinisch unnötig und gesellschaftlich
  kontraproduktiv. Die Prognosen der Behörden und Hochschulen beruhten
  auf teilweise völlig
  \href{https://www.cash.ch/news/politik/coronavirus-schweizer-wissenschaftler-warnen-vor-knappheit-bei-spitalbetten-1511778}{unrealistischen
  Annahmen}.
\item
  Forscher der ETH Lausanne haben eine Studie vorgestellt, in der sie
  vor dem Risiko einer \textbf{``zweiten Welle''} warnen, die die
  Schweiz überfordern könnte. Der Hauptautor der Studie, Professor
  Jacques Fellay, ist zugleich Mitglied der Corona-Taskforce des
  Bundesrates und Befürworter eines Corona-Impfstoffes. Eine unabhängige
  Analyse dieser Studie ergab indes, dass sie
  auf~\href{https://www.heise.de/tp/features/Fellay-Studie-Zweite-Corona-Welle-4726303.html}{sehr
  unrealistischen Annahmen} beruht und beispielsweise ein konstantes
  Erkrankungs- und Sterberisiko für alle Altersgruppen annimmt.
\item
  Medizinunternehmer Stephan Rietiker von \textbf{Inside-Corona} kommt
  mit Blick auf das Corona-Managements des Bundesrates zu
  \href{https://www.insidecorona.ch/2020/05/21/coronakrise-professionelles-krisenmanagement-sieht-anders-aus/}{einem
  vernichtenden Fazit}. Der Bundesrat habe die Erkenntnisse der eigenen
  Pandemiekommission weitgehend ignoriert und groteske Fehlentscheide
  mit folgenreichem Schaden für Bevölkerung und Wirtschaft getroffen. Es
  sei eine untaugliche ``Eindämmungsstrategie'' verfolgt worden im
  naiven Glauben, die Zeit bis zu einem Impfstoff ``überbrücken'' zu
  können. Das geplante generelle ``Contact Tracing'' werde Unmengen an
  Geld verschlingen und am Ende ``kläglich scheitern''. Es sei höchste
  Zeit, das Notrecht aufzuheben und die Entscheider zur Verantwortung zu
  ziehen.
\end{itemize}

Medien:

\begin{itemize}
\tightlist
\item
  Die \textbf{Aargauer Zeitung} berichtete
  \href{https://www.aargauerzeitung.ch/aargau/kanton-aargau/erstes-kind-am-coronavirus-gestorben-aargauer-saeugling-zur-behandlung-heimgeflogen-137988950}{vom
  angeblich ersten Kind}, das ``am Coronavirus gestorben'' sei.
  Tatsächlich starb der aus Mazedonien notfallmäßig eingeflogene
  Säugling an einer
  \href{https://www.srf.ch/news/regional/aargau-solothurn/verwirrung-um-todesursache-trauriges-ereignis-aargauer-baby-stirbt-nach-corona-infektion}{Hirnhautentzündung},
  die nicht durch Coronaviren verursacht wird.
\item
  Die \textbf{NZZ} verbreitete im Mai in einer Reportage noch einmal die
  \href{https://www.nzz.ch/gesellschaft/wie-corona-das-tessin-an-den-rand-einer-katastrophe-brachte-und-die-willensnation-auf-die-probe-stellte-ld.1556749}{Falschmeldung},
  Corona habe den Kanton Tessin ``an den Rand einer Katastrophe''
  gebracht. Tatsächlich gab es auch im Kanton Tessin außerhalb der
  Pflegeheime
  \href{https://www.blick.ch/news/schweiz/tessin/tessiner-spitaldirektor-widerspricht-dem-bundesamt-fuer-gesundheit-wir-haben-genuegend-intensivbetten-id15808076.html}{keine
  ernsthaften Probleme}.
\item
  Der \textbf{SRF TV Chefredakteur}
  \href{https://www.srf.ch/news/schweiz/berichterstattung-zu-corona-srf-chefredaktor-nimmt-stellung-zu-fakenews-vorwuerfen}{wehrte
  sich gegen Vorwürfe}, SRF verbreite zu Corona Propaganda. Kurz darauf
  machte ein rollstuhlfahrender Teilnehmer einer Corona-Demonstration
  \href{https://www.youtube.com/watch?v=9BA83YQT7AE}{darauf aufmerksam},
  dass er vom SRF für Propaganda missbraucht wurde.
\item
  Die \textbf{BZ Basel} berichtete von angeblich
  \href{https://www.bzbasel.ch/basel/basel-stadt/basler-primarschueler-an-corona-erkrankt-muessen-schulen-jetzt-ueber-die-buecher-137988954}{``an
  Corona erkrankten Primarschülern''}. In Wirklich­keit wurden die
  Schüler nur positiv getestet, von einer ``Erkrankung'' ist nichts
  bekannt.
\item
  \textbf{Infosperber}:
  \href{https://www.infosperber.ch/Artikel/Gesundheit/13-irrefuhrende-und-falsche-Behauptungen-zur-Corona-Epidemie}{13
  irreführende und falsche Behauptungen zur Corona-Epidemie}
\end{itemize}

Politische Aspekte:

\begin{itemize}
\tightlist
\item
  In der Schweiz kam es seit April zu wöchentlichen \textbf{Corona- und
  Grundrechte-Demonstrationen} in Bern, Zürich und anderen Städten.
  Dabei kam es zu teilweise erheblicher Polizeigewalt, siehe
  \href{https://swprs.org/corona-repression-in-der-schweiz/}{»Corona-Repression
  in der Schweiz«}. Von staatstragenden Medien wurden die Teilnehmer
  dieser Demonstrationen zumeist als Wirrköpfe oder Extremisten
  dargestellt.
\item
  Der Bundesrat plant, die bestehenden \textbf{Notverordnungen}
  \href{https://www.nau.ch/politik/bundeshaus/bundesrat-will-notverordnungen-zu-bundesrecht-machen-65700223}{in
  dringliches Bundesrecht umzuwandeln}, wogegen eine
  \href{https://www.zeitpunkt.ch/petition-gegen-umwandlung-der-notverordnungen-dringliches-bundesrecht-lanciert}{Online-Petition
  gestartet} wurde.
\item
  Der Bundesrat hat Ende Mai eine gesetzliche Grundlage für die
  \textbf{``Corona-Tracing-App''} verabschiedet, die nun vom Parlament
  \href{https://www.tagesanzeiger.ch/einsatz-der-schweizer-corona-warn-app-rueckt-naeher-690816478823}{beraten
  und bewilligt wird}. Apple und Google haben ihre mobilen
  Betriebssysteme mit einer entsprechenden Schnittstelle ausgerüstet.
  Entgegen der ursprünglichen Darstellung werden ``anonyme'' Daten doch
  im
  Ausland\href{https://www.nau.ch/politik/bundeshaus/deshalb-speichert-corona-app-daten-bei-amazon-deutschland-65717602}{auf
  Amazon-Servern} gespeichert. Die Nutzung der App solle ``freiwillig''
  bleiben. \textbf{Update}: 26 Nationalräte
  \href{https://www.parlament.ch/poly/Abstimmung/51/out/vote_51_20610.pdf}{stimmten
  gegen} die Einführung der Proximity-Tracing-App.
\item
  Der Kanton Zürich und einige andere Kantone haben bereits
  \textbf{Zentren zur gesamt­gesell­schaftlichen Kontaktverfolgung}
  \href{https://www.nzz.ch/zuerich/coronavirus-in-zuerich-contact-tracing-ist-gut-angelaufen-ld.1556846}{eingerichtet},
  bestehend aus Mitarbeitern der Polizei und Gesund­heits­direktion, die
  potentiell ``infizierte'' Personen kontaktieren und sie in eine
  Quarantäne einweisen sollen. Die Erfahrungen vom Frühjahr hätten
  gezeigt, dass das Zentrum künftig stark ausgebaut oder ausgelagert
  werden müsse.
\item
  Eine Umfrage der Hochschule Winterthur ZHAW ergab, dass
  \href{https://www.zhaw.ch/de/medien/medienmitteilungen/detailansicht-medienmitteilung/event-news/viele-schweizer-fuerchten-ueberwachung-durch-contact-tracing-app/}{40\%
  der Schweizer} eine \textbf{stärkere Überwachung} durch
  ``Corona-Apps'' befürchten. Dennoch plane eine Mehrheit, die App zur
  Kontaktverfolgung zu nutzen.
\item
  In der Schweiz stieg die \textbf{Arbeitslosigkeit} durch den Lockdown
  um 36\%
  \href{https://www.nzz.ch/wirtschaft/im-herbst-koennte-es-zu-einer-entlassungswelle-in-der-schweiz-kommen-ld.1560235}{auf
  3.4\%}. Die Kurzarbeit stieg auf knapp 40\% aller Arbeitskräfte, den
  \href{https://industriemagazin.at/a/kurzarbeit-in-europa-nur-schweiz-hat-hoeheren-anteil-als-oesterreich}{höchsten
  Wert} in Europa.
\item
  Im Rahmen von Covid19 kam nicht die seit langem bestehende Schweizer
  Pandemie-Kommission
  \href{https://www.srf.ch/news/schweiz/bag-verzichtete-auf-beratung-was-macht-die-pandemie-kommission-in-der-krise}{zum
  Einsatz}, sondern eine eigens gegründete \textbf{``Covid-19
  Taskforce''}, deren Mitglieder teilweise
  \href{https://ncs-tf.ch/de/organisation}{Interessenskonflikte} im
  Bereich der Pharmazie aufweisen.
\item
  \textbf{Video}:
  \href{https://www.youtube.com/watch?v=RyZGkdeQ6CY}{``Gehört der
  Bundesrat ins Gefängnis?''} Der Schweizer Journalist Reto Brennwald
  interviewte den Unternehmer
  \href{https://www.youtube.com/user/timturpis/videos}{Daniel Stricker},
  der Mitte März für einige Wochen aus der Schweiz nach Schweden
  flüchtete und die Corona-Politik des Schweizer Bundesrates stark
  kritisiert.
\item
  US-Milliardär und Impfstoff-Investor Bill Gates hat der Schweizer
  Heilmittelbehörde \textbf{Swissmedic} im Februar 2020 eine
  \href{https://www.christoph-pfluger.ch/2020/05/23/der-groesste-impfstoff-investor-schenkt-der-swissmedic-900000-dollar/}{Spende
  von 900,000 Dollar} für ``Projekte in Afrika'' zukommen lassen.
\item
  Die Schweiz wird die von Bill Gates geförderte \textbf{globale
  Impfstoffallianz GAVI} mit 30 Millionen Franken für die Entwicklung
  eines Corona-Impfstoffes
  \href{https://www.aargauerzeitung.ch/panorama/schweiz-unterstuetzt-impfallianz-gavi-mit-30-millionen-138082583}{unterstützen}.
\item
  Der \textbf{Corona-kritische Aargauer Arzt}, der im April von einer
  Sondereinheit der Polizei brutal festgenommen und vorübergehend in die
  Psychiatrie eingeliefert wurde, forderte eine
  \href{https://www.aargauerzeitung.ch/aargau/kanton-aargau/polizeieinsatz-wegen-wettinger-arzt-regierung-beschliesst-externe-administrativuntersuchung-137913013}{externe
  Administrativuntersuchung} zu seinem Fall, die von der Regierung nun
  eingeleitet wurde. Bereits zuvor wurde bekannt, dass der Arzt entgegen
  der Behauptung der Polizei weder Angehörige noch Behörden bedrohte
  noch im Besitz einer geladenen Waffe war.
\end{itemize}

\hypertarget{deutschland-und-uxf6sterreich}{%
\subparagraph{\texorpdfstring{\textbf{Deutschland und
Österreich}}{Deutschland und Österreich}}\label{deutschland-und-uxf6sterreich}}

\begin{itemize}
\tightlist
\item
  In Deutschland und Österreich kam es zu \textbf{keiner signifikanten
  Übersterblichkeit} seit März. Tatsächlich zeigte sich in Deutschland
  bevölkerungskorrigiert sogar
  \href{https://swprs.files.wordpress.com/2020/06/breyer-deutschland-untersterblichkeit.pdf}{eine
  leichte Untersterblichkeit} seit Jahresbeginn, wie ein Professor der
  Uni Konstanz nachwies.
\item
  In Österreich wurde
  \href{https://www.falter.at/zeitung/20200512/was-passiert-wenn-es-eng-wird}{durch
  geleakte Protokolle} wie bereits in Dänemark und England bekannt, dass
  der \textbf{Lockdown politisch motiviert} war und sich die
  Gesundheitsexperten mehrheitlich dagegen aussprachen: Diese empfahlen
  Kanzler Kurz bereits im März, von der Botschaft eines ``ganz
  gefährlichen Virus'' abzukommen, da das Virus bereits weit verbreitet
  und für den Großteil der Bevölkerung nicht gefährlich sei.
\item
  In Deutschland wurden aufgrund des Corona-Lockdowns bis Ende Mai
  \href{https://www.welt.de/wirtschaft/article208557665/Wegen-Corona-In-Deutschland-wurden-908-000-OPs-aufgeschoben.html}{fast
  eine Million} \textbf{Operationen} abgesagt. Deutsche Onkologen
  \href{https://deutsch.medscape.com/artikelansicht/4908889}{warnen vor
  Verzögerungen} bei Diagnostik und Therapie von Krebserkrankungen
  aufgrund des Lockdowns. Krebs sei eine ``weitaus größere Gefahr als
  COVID-19'', betonten die medizinischen Fachverbände.
\item
  Deutsche \textbf{Kinderärzte} berichten aufgrund des Lockdowns von
  \href{https://www.tagesspiegel.de/politik/knochenbrueche-oder-schuetteltraumata-mediziner-berichten-von-massiver-gewalt-gegen-kinder/25833740.html}{massiver
  Gewalt gegen Kinder}. Sie sähen Verletzungen wie sonst nur nach
  Autounfällen, darunter Knochenbrüche oder Schütteltraumata. Die Zahl
  der Anrufe bei der Kinderschutzhotline habe stark zugenommen.
\item
  Der Berliner Rechtsmediziner Professor Michael Tsokos
  \href{https://www.bz-berlin.de/berlin/michael-tsokos-berlins-bekanntester-gerichtsmediziner-spricht-von-corona-suiziden-durch-panikmache}{berichtete
  von} \textbf{``Corona-Suiziden''} von Menschen, die sich vor einer
  Infektion fürchten oder glauben, infiziert zu sein. Professor Tsokos
  kritisierte die Panikmache und Verbreitung von ``Horrorszenarien''
  durch Medien und einige Virologen. Er befürchtet, dass auf die
  ``infektiologische Pandemie'' nun eine ``psycho-soziale Pandemie''
  folgen wird. Tatsächlich sind die Suizide und Suizidversuche in Berlin
  im ersten Quartal bereits
  \href{https://www.tichyseinblick.de/daili-es-sentials/suizide-in-berlin-steigen-im-ersten-quartal-drastisch/}{``drastisch
  angestiegen''}.
\item
  Wirtschaftsexperten rechnen in Deutschland bis 2021
  \href{https://www.focus.de/finanzen/news/konjunktur/insgesamt-fast-30-000-insolvenzen-in-deutschland-experten-geben-erste-schaetzung-ab-corona-treibt-10-000-deutsche-firmen-in-die-pleite_id_12003269.html}{mit
  ca. 10,000 zusätzlichen} \textbf{Insolvenzen von Firmen} aufgrund der
  politischen Corona-Maßnahmen.
\item
  In deutschen Städten kam es seit April zu einigen der größten
  \textbf{Corona- und Grundrechte-Demonstrationen} Europas mit bis zu
  25,000 Teilnehmern. Dabei kam es allerdings auch zu teilweise
  gravierender Polizeigewalt, beispielsweise bei der Verhaftung des
  Vegankochs \href{https://www.youtube.com/watch?v=20HIEEQtyZ8}{Attila
  Hildmann} oder der 68-jährigen ehemaligen DDR-Bürgerrechtlerin
  \href{https://www.youtube.com/watch?v=fCbgVFRCnTc}{Angelika Barbe}.
  Außerdem kam es in Deutschland zu
  \href{https://www.swr.de/swraktuell/baden-wuerttemberg/spreng-anschlag-100.html}{Brandanschlägen}
  und teilweise lebensbedrohlichen
  \href{https://www.swr.de/swraktuell/baden-wuerttemberg/stuttgart/corona-demo-in-stuttgart-ermittlungen-wegen-versuchter-toetung-100.html}{Überfällen}
  auf Demonstrierende und
  \href{https://kenfm.de/anschlagsversuch-auf-ken-jebsen-tagesdosis-9-6-2020/}{Medienleute}
  durch zumeist schwarz vermummte Gruppierungen, die sich vor
  staatlicher Verfolgung offenbar nicht zu fürchten scheinen.
\item
  \textbf{Werner Winterstein}, Enkel eines von Nationalsozialisten
  ermordeten österreichisch-jüdischen Justizministers, nahm an einer
  Grundrechte-Demonstration in Wien teil
  \href{https://vimeo.com/418039066}{und erklärte}, er sei ``erschüttert
  über die stille Machtergreifung durch Elemente, die von ‚neuer
  Normalität` am Rande des demokratischen Modells sprechen.'' Die
  Corona-bedingte Einteilung von Bürgern in verschiedene Kategorien und
  die Schaffung eines neuen Denunziantentums seien gefährliche
  Entwicklungen. Er beobachte einen Mangel an Zivilcourage und eine
  Unterwerfung unter die Obrigkeit. Die geplanten Corona-Apps gehen ``in
  Richtung totaler Überwachungsstaat'' und seien abzulehnen.
\end{itemize}

\includegraphics{https://swprs.files.wordpress.com/2020/06/sterbefallzahlen-de-25-05.png?w=736\&h=414}

\hypertarget{vereinigte-staaten}{%
\subparagraph{**Vereinigte Staaten}\label{vereinigte-staaten}}

**

\begin{itemize}
\tightlist
\item
  In den USA gab es bis Ende Mai ca. 100,000 testpositive Todesfälle.
  Die \textbf{Gesamt­sterblichkeit} seit Anfang Jahr lag indes im
  Bereich der starken Grippesaison von 2017/2018 (siehe Grafik unten).
  Mindestens
  \href{https://www.forbes.com/sites/theapothecary/2020/05/26/nursing-homes-assisted-living-facilities-0-6-of-the-u-s-population-43-of-u-s-covid-19-deaths/}{42\%
  der Todesfälle} erfolgten in Pflegeheimen, die 0.6\% der
  US-Bevölkerung ausmachen und nicht von einem allgemeinen Lockdown
  profitierten.
\item
  In den Bundesstaaten, die \textbf{keinen Lockdown} einführten oder ihn
  früh wieder aufhoben, kam es laut einer Studie von JP Morgan zu
  \href{https://www.dailymail.co.uk/news/article-8347635/Lockdowns-failed-alter-course-pandemic-JP-Morgan-study-claims.html}{keiner
  erhöhten Sterblichkeit}.
\item
  Eine Auswertung von vier US-Professoren kommt zum Ergebnis, dass der
  Lockdown in den USA etwa
  \href{https://thehill.com/opinion/healthcare/499394-the-covid-19-shutdown-will-cost-americans-millions-of-years-of-life}{doppelt
  soviele Lebensjahre} kosten wird wie Covid-19.
\item
  Über 600 Ärzte warnten US-Präsident \textbf{Donald Trump} in einem
  \href{https://www.washingtonexaminer.com/news/mass-casualty-incident-over-600-doctors-sign-letter-warning-trump-of-dangers-of-continued-lockdowns}{Offenen
  Brief} vor den Gefahren eines verlängerten Lockdowns. Der Lockdown sei
  selbst eine medizinische Großkatastrophe gewesen (mass casualty
  incident). Außerhalb von New York City sei keine einzige US-Stadt
  durch Corona besonders belastet gewesen.
\item
  In einem weiteren
  \href{https://aapsonline.org/physician-letter-reopen-america/}{Offenen
  Brief} an US-Vizepräsident \textbf{Mike Pence} fordern amerikanische
  Ärzte die rasche Öffnung der Gesellschaft. Die Gefährlichkeit von
  Covid-19 habe sich als viel geringer herausgestellt als ursprünglich
  angenommen. Die Risikogruppen seien bekannt und könnten gezielt
  geschützt werden. Eine sichere Immunisierung der Gesellschaft sei auch
  ohne Impfung möglich. Die Medien hätten die Bevölkerung unnötig
  terrorisiert und damit einen starken Anstieg von Verzweiflung und
  Suizidabsichten ausgelöst.
\item
  Die meisten der für insgesamt 660 Millionen Dollar gebauten
  \textbf{Feldhospitale} der US-Armee haben im Mai wieder
  \href{https://www.npr.org/2020/05/07/851712311/u-s-field-hospitals-stand-down-most-without-treating-any-covid-19-patients}{geschlossen},
  ohne einen einzigen Patienten behandelt zu haben.
\item
  Ein kalifornischer Trauma-Arzt erklärt
  \href{https://www.dailymail.co.uk/news/article-8347011/Doctors-California-say-people-killed-four-weeks-YEAR.html}{in
  einem Beitrag}, dass es im Mai mehr \textbf{Suizidversuche} gegeben
  habe als normalerweise in einem ganzen Jahr, und dass die Suizide die
  Corona-Todesfälle in Kalifornien bereits bei weitem übertreffen.
\item
  Im stark betroffenen Bundesstaat \textbf{New York} wurde eine
  \href{https://nypost.com/2020/05/12/calls-for-independent-probe-of-gov-cuomos-nursing-home-policies/}{unabhängige
  Untersuchung gefordert} zur Anordnung des Gouvernours, dass
  \textbf{Pflegeheime} Corona-Patienten aufnehmen müssen. In New Yorker
  Pflegeheimen kam es zu über 5000 Todesfällen. Auch in den stark
  betroffenen Staaten New Jersey und Pennsylvania gab es solche
  Anordnungen.
\item
  Der größte Krankenhausverbund New Yorks hat eine
  \href{https://nypost.com/2020/05/29/northwell-health-probing-use-of-ventilators-for-covid-patients/}{Untersuchung
  zum Einsatz} invasiver \textbf{Beatmungsgeräte} angekündigt. Im April
  wurde bekannt, dass US-Kliniken
  \href{https://eu.usatoday.com/story/news/factcheck/2020/04/24/fact-check-medicare-hospitals-paid-more-covid-19-patients-coronavirus/3000638001/}{hohe
  Prämien erhalten}, wenn sie Covid-19-Patienten aufnehmen und diese an
  Beatmungsmaschinen anschließen, obschon Fachleute längst vor den
  Gefahren einer invasiven Beatmung
  \href{https://off-guardian.org/2020/05/06/covid19-are-ventilators-killing-people/}{warnten}.
  Eine New Yorker Krankenschwester sprach dabei sogar von einem
  \href{https://nypost.com/2020/05/29/northwell-health-probing-use-of-ventilators-for-covid-patients/}{``Massenmord''}.
\item
  Im Bundesstaat Washington
  \href{https://www.freedomfoundation.com/washington/washington-health-officials-gunshot-victims-counted-as-covid-19-deaths/}{bestätigte}
  die Gesundheitsbehörde, dass selbst \textbf{Mordopfer} als
  ``Coronatote'' gezählt wurden, wenn sie positiv auf das Coronavirus
  testeten. Auch die New York Times listete auf ihrer Titelseite zu
  ``Corona-Opfern''
  \href{https://www.washingtontimes.com/news/2020/may/24/new-york-times-lists-homicide-victim-coronavirus-d/}{ein
  Mordopfer auf}. Selbst der Ende Mai bei einer Festnahme verstorbene
  \textbf{George Floyd}
  \href{https://www.nydailynews.com/coronavirus/ny-coronavirus-george-floyd-20200604-pvhrtjn4mna5blzow3gfx5ag3y-story.html}{testete
  positiv} auf Corona.
\item
  Bis Ende Mai gerieten in den USA
  \href{https://www.businessinsider.com/us-weekly-jobless-claims-unemployment-filings-coronavirus-labor-market-layoffs-2020-5}{über
  40 Millionen Menschen} in die \textbf{Arbeitslosigkeit}. Schätzungen
  gehen davon aus, dass circa 42\% der verlorenen Arbeitsplätze
  \href{https://www.forbes.com/sites/kenrapoza/2020/05/15/some-42-of-jobs-lost-in-pandemic-are-gone-for-good/}{nicht
  mehr eröffnet} werden und die schwerste Rezession seit dem Zweiten
  Weltkrieg eintritt (siehe unten).
\item
  \textbf{US-Milliardäre} sahen durch ``Corona'' dennoch einen
  Vermögenszuwachs von
  \href{https://www.cnbc.com/2020/05/21/american-billionaires-got-434-billion-richer-during-the-pandemic.html}{\$434
  Milliarden}.
\end{itemize}

\includegraphics{https://swprs.files.wordpress.com/2020/06/us-cumulative-deaths.png?w=736\&h=438}

\includegraphics{https://swprs.files.wordpress.com/2020/06/us-2020-recession.jpg?w=736\&h=482}

\hypertarget{grouxdfbritannien-1}{%
\subparagraph{\texorpdfstring{\textbf{Großbritannien}}{Großbritannien}}\label{grouxdfbritannien-1}}

\begin{itemize}
\tightlist
\item
  In England und Wales kam es im März und April zu einer
  \href{https://www.ons.gov.uk/peoplepopulationandcommunity/birthsdeathsandmarriages/deaths/articles/analysisofdeathregistrationsnotinvolvingcoronaviruscovid19englandandwales28december2019to1may2020/technicalannex}{Übersterblichkeit}
  von ca. 46,000 Personen. Dies entspricht in etwa
  \href{https://off-guardian.org/2020/05/25/were-all-in-the-big-numbers-now/}{den
  starken Grippewellen} von 1999 und 2000 (siehe Grafik unten).
  Allerdings stellte die Gesundheitsbehörde ONS fest, dass knapp 30\%
  dieser Übersterblichkeit
  \href{https://www.ons.gov.uk/peoplepopulationandcommunity/birthsdeathsandmarriages/deaths/articles/analysisofdeathregistrationsnotinvolvingcoronaviruscovid19englandandwales28december2019to1may2020/technicalannex}{nicht
  auf das Coronavirus zurückzuführen} ist.
\item
  Bis Anfang Mai starben beispielsweise
  \href{https://www.theguardian.com/society/2020/may/08/more-people-dying-at-home-during-covid-19-pandemic-uk-analysis}{8000
  Menschen mehr} bei sich zuhause als üblich, und 80\% dieser Todesfälle
  hatten laut Totenschein nichts mit dem Coronavirus zu tun.
\item
  Allein in den britischen \textbf{Pflegeheimen} kam es bis Anfang Mai
  zu einer Übersterblichkeit von 30,000 Todesfällen, von denen jedoch
  nur bei 10,000 Covid-19 als Ursache auf dem Totenschein vermerkt
  wurde, wie Cambridge-Professor David Spiegelhalter
  \href{https://www.bmj.com/content/369/bmj.m1931}{erklärte}.~ So
  starben in England und Wales bereits im April rund
  \href{https://www.theguardian.com/world/2020/jun/05/covid-19-causing-10000-dementia-deaths-beyond-infections-research-says}{10,000
  zusätzliche Demenzpatienten} ohne Corona-Infektion aufgrund der
  wochenlangen Isolation.
\item
  Ähnlich wie in Norditalien und New York kam es auch in England zur
  \href{https://drmalcolmkendrick.org/2020/05/11/how-to-make-a-crisis-far-far-worse/}{fatalen
  Entscheidung}, Corona-Patienten von den Krankenhäusern \textbf{in die
  Pflegeheime} zu verlegen, sowie zu einer starken Ausbreitung des
  Coronavirus innerhalb des Gesundheitssystems selbst.
\item
  Die meisten zusätzlich errichteten \textbf{Feldkrankenhäuser} blieben
  indes
  \href{https://www.telegraph.co.uk/news/0/do-many-nhs-nightingale-hospitals-remain-empty/}{weitgehend
  leer}.
\item
  Das einflussreiche Computermodell des Epidemiologen \textbf{Neil
  Ferguson}, das hunderttausende Tote prognostizierte, stellte sich bei
  einer unabhängigen Analyse durch Software- und Modellierungsexperten
  als
  \href{https://www.telegraph.co.uk/technology/2020/05/16/neil-fergusons-imperial-model-could-devastating-software-mistake/}{stark
  fehlerhaft und unrealistisch} heraus. Ferguson musste als
  Regierungsberater
  \href{https://www.telegraph.co.uk/news/2020/05/05/exclusive-government-scientist-neil-ferguson-resigns-breaking/}{zurücktreten},
  nachdem er selbst den Lockdown brach, um seine verheiratete Geliebte
  zu empfangen. Inzwischen
  \href{https://www.bbc.co.uk/news/health-52968523}{behauptet seine
  Universität}, der Lockdown habe in Europa ``drei Millionen Leben
  gerettet''.
\item
  \textbf{Analyse}:
  \href{https://www.ukcolumn.org/article/who-controls-british-government-response-covid19-part-one}{Who
  controls the British Government response to Covid--19?} (Part 1) and
  \href{https://www.ukcolumn.org/article/covid\%E2\%80\%9319-big-pharma-players-behind-uk-government-lockdown}{COVID--19:
  The Big Pharma players behind UK Government lockdown} (Part 2)
\item
  In Großbritannien zeichnet sich die
  \href{https://twitter.com/FinancialTimes/status/1258499372251328515}{größte
  wirtschaftliche Rezession} seit dem ``Großen Frost'' von 1709 (in der
  Kleinen Eiszeit) ab.
\item
  \textbf{Positive Anekdote}: Der älteste Schneider Großbritanniens
  \href{https://www.yorkpress.co.uk/news/18449261.oldest-working-tailor-uk-elwyn-96-beats-coronavirus/}{überstand
  das Corona-Virus} mit 96 Jahren, nachdem er zehn Tage lang keine
  Nahrung mehr zu sich nahm.
\item
  \textbf{Weitere Statistiken} auf
  \href{http://inproportion2.talkigy.com/}{Covid-19 In Proportion}.
\end{itemize}

\includegraphics{https://swprs.files.wordpress.com/2020/06/uk-flu-comparison.png?w=736\&h=437}

\hypertarget{suxfcdamerika-und-afrika}{%
\subparagraph{**Südamerika und Afrika}\label{suxfcdamerika-und-afrika}}

**

Nachdem die Corona-Ausbreitung in Europa und den USA abklang,
fokusierten viele Medien auf die Situation in Südamerika, insbesondere
auch in \textbf{Brasilien}. Tatsache ist jedoch, dass Brasilien mit
seinen 210 Millionen Einwohner deutlich
\href{https://www.statista.com/statistics/1104709/coronavirus-deaths-worldwide-per-million-inhabitants/}{besser
dasteht} als die meisten westeuropäischen Länder.

In anderen lateinamerikanischen Ländern wie etwa \textbf{Ecuador}
verbreitet sich neben dem Coronavirus
\href{https://www.bbc.com/mundo/noticias-america-latina-52383340}{zusätzlich
noch das Denguefieber} mit ähnlichen Symptomen, was zu einer doppelten
Belastung des Gesundheitssystems führen kann. Dennoch wurde z.B. in
\textbf{Peru} festgestellt, dass ähnlich wie in anderen Ländern ca. 80\%
der bestätigten Coronafälle
\href{https://exitosanoticias.pe/v1/covid-19-minsa-el-80-de-casos-confirmados-en-el-peru-son-asintomaticos/}{symptomlos
bleiben}.

Einige Medien berichteten von angeblich ``rund um die Uhr'' laufenden
Krematorien in \textbf{Mexico City}. Ein in Mexiko lebender Youtuber
\href{https://www.youtube.com/watch?v=_vQhwEZpDPE}{besuchte daraufhin}
die Stadt und die dortigen Krankenhäuser, Bestattungs­­unternehmen und
Krematorien, die alle sehr wenig Betrieb hatten.

Generell gab es in Südamerika und auch in Afrika bisher eine deutlich
\href{https://www.msn.com/en-gb/news/world/coronavirus-why-africa-seems-to-have-few-cases/ar-BB10MNJd}{geringere
Corona-Sterblichkeit} als in Europa und den USA, was an der jüngeren
Bevölkerung und an klimatischen Faktoren liegen könnte. Hingegen rechnet
die Weltbank mit bis zu
\href{https://www.deccanherald.com/business/economy-business/world-bank-says-covid-19-to-push-60-million-into-poverty-announces-usd-160-billion-assistance-to-100-countries-839661.html}{60
Millionen Armutsopfern} aufgrund der globalen politischen
Corona-Maßnahmen.

\hypertarget{c-zur-rolle-von-bill-gates}{%
\subparagraph{\texorpdfstring{\textbf{C. Zur Rolle von Bill
Gates}}{C. Zur Rolle von Bill Gates}}\label{c-zur-rolle-von-bill-gates}}

US-Multimilliardär und Microsoft-Gründer Bill Gates ist der wichtigste
private Sponsor der WHO und der Impfstoff-Industrie und steht deshalb
aktuell besonders im Fokus. In den folgenden Abbildungen wird sein
pharmazeutisches und mediales Netzwerk grafisch dargestellt.

\textbf{Siehe auch}:

\begin{itemize}
\tightlist
\item
  SWR (2017):
  \href{https://www.swr.de/swr2/wissen/who-am-bettelstab-was-gesund-ist-bestimmt-bill-gates-100.html}{Die
  WHO am Bettelstab: Was gesund ist, bestimmt Bill Gates}
\item
  Politico (2017):
  \href{https://www.politico.eu/article/bill-gates-who-most-powerful-doctor/}{Meet
  the world's most powerful doctor: Bill Gates}
\item
  Stern (2007):
  \href{https://www.stern.de/wirtschaft/news/investitionen-die-dunkle-seite-der-gates-stiftung-3330702.html}{Die
  dunkle Seite der Gates-Stiftung}
\end{itemize}

\href{https://swprs.files.wordpress.com/2020/05/gates-funding-e1590256423190.jpg}{}

\includegraphics{https://swprs.files.wordpress.com/2020/05/gates-funding-e1590256423190.jpg?w=417\&h=388}

Gates Funding Worldwide

\href{https://swprs.files.wordpress.com/2020/05/gates-foundation.jpg}{}

\includegraphics{https://swprs.files.wordpress.com/2020/05/gates-foundation.jpg?w=311\&h=388}

Gates Funding Germany

Fördermittel von Bill Gates in USA/UK und Deutschland

~

\hypertarget{weiter-zu-den-uxe4lteren-beitruxe4gen-}{%
\paragraph{\texorpdfstring{\href{https://swprs.org/fakten-zu-covid-19-archiv-maerz-2020/}{Weiter
zu den älteren Beiträgen
→}}{Weiter zu den älteren Beiträgen →}}\label{weiter-zu-den-uxe4lteren-beitruxe4gen-}}

\begin{center}\rule{0.5\linewidth}{\linethickness}\end{center}

Beitrag teilen auf:
\href{https://twitter.com/intent/tweet?url=https://swprs.org/covid-19-hinweis-ii/}{Twitter}
/
\href{https://www.facebook.com/share.php?u=https://swprs.org/covid-19-hinweis-ii/}{Facebook}

\hypertarget{swiss-policy-research}{%
\subsubsection{Swiss Policy Research}\label{swiss-policy-research}}

\begin{itemize}
\tightlist
\item
  \href{https://swprs.org/kontakt/}{Kontakt}
\item
  \href{https://swprs.org/uebersicht/}{Übersicht}
\item
  \href{https://swprs.org/donationen/}{Donationen}
\item
  \href{https://swprs.org/disclaimer/}{Disclaimer}
\end{itemize}

\hypertarget{english}{%
\subsubsection{English}\label{english}}

\begin{itemize}
\tightlist
\item
  \href{https://swprs.org/contact/}{About Us / Contact}
\item
  \href{https://swprs.org/media-navigator/}{The Media Navigator}
\item
  \href{https://swprs.org/the-american-empire-and-its-media/}{The CFR
  and the Media}
\item
  \href{https://swprs.org/donations/}{Donations}
\end{itemize}

\hypertarget{follow-by-email}{%
\subsubsection{Follow by email}\label{follow-by-email}}

Follow

\href{https://wordpress.com/?ref=footer_custom_com}{WordPress.com}.

\protect\hyperlink{}{Up ↑}

Post to

\protect\hyperlink{}{Cancel}

\includegraphics{https://pixel.wp.com/b.gif?v=noscript}
