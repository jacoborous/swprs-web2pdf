\protect\hyperlink{content}{Skip to content}

\href{https://swprs.org/}{}

\protect\hyperlink{search-container}{Search}

Search for:

\href{https://swprs.org/}{\includegraphics{https://swprs.files.wordpress.com/2020/05/swiss-policy-research-logo-300.png}}

\href{https://swprs.org/}{Swiss Policy Research}

Geopolitics and Media

Menu

\begin{itemize}
\tightlist
\item
  \href{https://swprs.org}{Start}
\item
  \href{https://swprs.org/srf-propaganda-analyse/}{Studien}

  \begin{itemize}
  \tightlist
  \item
    \href{https://swprs.org/srf-propaganda-analyse/}{SRF / ZDF}
  \item
    \href{https://swprs.org/die-nzz-studie/}{NZZ-Studie}
  \item
    \href{https://swprs.org/der-propaganda-multiplikator/}{Agenturen}
  \item
    \href{https://swprs.org/die-propaganda-matrix/}{Medienmatrix}
  \end{itemize}
\item
  \href{https://swprs.org/medien-navigator/}{Analysen}

  \begin{itemize}
  \tightlist
  \item
    \href{https://swprs.org/medien-navigator/}{Navigator}
  \item
    \href{https://swprs.org/der-propaganda-schluessel/}{Techniken}
  \item
    \href{https://swprs.org/propaganda-in-der-wikipedia/}{Wikipedia}
  \item
    \href{https://swprs.org/logik-imperialer-kriege/}{Kriege}
  \end{itemize}
\item
  \href{https://swprs.org/netzwerk-medien-schweiz/}{Netzwerke}

  \begin{itemize}
  \tightlist
  \item
    \href{https://swprs.org/netzwerk-medien-schweiz/}{Schweiz}
  \item
    \href{https://swprs.org/netzwerk-medien-deutschland/}{Deutschland}
  \item
    \href{https://swprs.org/medien-in-oesterreich/}{Österreich}
  \item
    \href{https://swprs.org/das-american-empire-und-seine-medien/}{USA}
  \end{itemize}
\item
  \href{https://swprs.org/bericht-eines-journalisten/}{Fokus I}

  \begin{itemize}
  \tightlist
  \item
    \href{https://swprs.org/bericht-eines-journalisten/}{Journalistenbericht}
  \item
    \href{https://swprs.org/russische-propaganda/}{Russische Propaganda}
  \item
    \href{https://swprs.org/die-israel-lobby-fakten-und-mythen/}{Die
    »Israel-Lobby«}
  \item
    \href{https://swprs.org/geopolitik-und-paedokriminalitaet/}{Pädokriminalität}
  \end{itemize}
\item
  \href{https://swprs.org/migration-und-medien/}{Fokus II}

  \begin{itemize}
  \tightlist
  \item
    \href{https://swprs.org/covid-19-hinweis-ii/}{Coronavirus}
  \item
    \href{https://swprs.org/die-integrity-initiative/}{Integrity
    Initiative}
  \item
    \href{https://swprs.org/migration-und-medien/}{Migration \& Medien}
  \item
    \href{https://swprs.org/der-fall-magnitsky/}{Magnitsky Act}
  \end{itemize}
\item
  \href{https://swprs.org/kontakt/}{Projekt}

  \begin{itemize}
  \tightlist
  \item
    \href{https://swprs.org/kontakt/}{Kontakt}
  \item
    \href{https://swprs.org/uebersicht/}{Seitenübersicht}
  \item
    \href{https://swprs.org/medienspiegel/}{Medienspiegel}
  \item
    \href{https://swprs.org/donationen/}{Donationen}
  \end{itemize}
\item
  \href{https://swprs.org/contact/}{English}
\end{itemize}

\protect\hyperlink{}{Open Search}

\hypertarget{der-atlantic-council}{%
\section{Der Atlantic Council}\label{der-atlantic-council}}

\includegraphics{https://swprs.files.wordpress.com/2018/11/atlantic-council.jpg?w=400\&h=200}

Der Atlantic Council ist
\href{https://www.rubikon.news/artikel/facebook-als-waffe}{bekannt} für
sein Engagement gegen NATO-kritische »Desinformation«, seine Kooperation
mit Facebook, die zur Löschung zahlreicher Seiten führte, sowie seine
Einwirkungen auf die deutsche und europäische Außenpolitik. Doch wer ist
der Atlantic Council?

Der 1961 gegründete
\href{https://en.wikipedia.org/wiki/Atlantic_Council}{Atlantic Council
of the United States}, wie er mit vollem Namen heißt, ist das
US-Mitglied der
\href{https://en.wikipedia.org/wiki/Atlantic_Treaty_Association}{Atlantic
Treaty Association (ATA)}, die wiederum als politische und
publizistische Unter­stützungs­gruppe der North Atlantic Treaty
Organization -- das heißt der NATO -- fungiert.

Das deutsche Pendant zum amerikanischen Atlantic Council ist die
\href{https://de.wikipedia.org/wiki/Deutsche_Atlantische_Gesellschaft}{Deutsche
Atlantische Gesellschaft} mit rund 4000 Mitgliedern. Auch
\href{https://de.wikipedia.org/wiki/Partnerschaft_f\%C3\%BCr_den_Frieden}{NATO-Partner}
wie Schweden, Georgien oder die Ukraine betreiben ATA-Ableger, nicht
jedoch die Schweiz und Österreich.

Der Atlantic Council wird -- wie die meisten relevanten Denkfabriken der
US-Außenpolitik -- von einem Vertreter des Council on Foreign Relations
(CFR) geleitet und ist deshalb in unserer
\href{https://swprs.org/das-american-empire-und-seine-medien/}{Übersichtsgrafik}
zum CFR aufgeführt -- ebenso wie Brookings, RAND, CSIS oder das Aspen
Institute.

Der Atlantic Council ist folglich ein integraler Bestandteil der
transatlantischen
\href{https://swprs.org/die-propaganda-matrix/}{Informationsmatrix}. Zu
seinen Mitgliedern zählt unter anderem der Gründer des
Recherchenetzwerks
\href{https://de.wikipedia.org/wiki/Bellingcat}{Bellingcat}, auf das
sich transatlantische Medien bei geopolitischen Konflikten des Öfteren
berufen.

Auch der
\href{https://en.wikipedia.org/wiki/Dmitri_Alperovitch}{Gründer} und
Technologiechef der IT-Sicherheitsfirma
\href{https://en.wikipedia.org/wiki/CrowdStrike}{CrowdStrike} ist ein
\emph{Senior Fellow} des Atlan­tic Council. Auf CrowdStrike geht die --
weiterhin
\href{https://www.thenation.com/article/a-new-report-raises-big-questions-about-last-years-dnc-hack/}{unbelegte}
-- Behauptung zurück, die Demo­kra­tische Partei (DNC) sei 2016 von
russischen Hackern angegriffen worden.

Da im Atlantic Council kaum deutschsprachige Journalisten mitwirken, ist
er in unseren
\href{https://swprs.org/netzwerk-medien-deutschland/}{Infografiken} zu
den transatlantischen Mediennetzwerken nicht aufgeführt; er wird dort
jedoch auf der obersten Ebene durch den CFR und die NATO repräsentiert.

\includegraphics{https://swprs.files.wordpress.com/2018/11/atlanic-council-freedom-awards-2018.jpg?w=736}

\emph{Der Atlantic Council Freedom Award 2018 ging u.a. an das syrische
»Twitter-Mädchen«
\href{http://blauerbote.com/2017/07/12/bana-alabed-das-syrische-twittermaedchen/}{Bana
Alabed} und die ehemalige US-Außenministerin und CFR-Direktorin
Madeleine Albright.
(\href{http://www.atlanticcouncil.org/events/freedom-awards/2018-honorees}{AC
2018})}

\hypertarget{siehe-auch}{%
\paragraph{Siehe auch}\label{siehe-auch}}

\begin{itemize}
\tightlist
\item
  \href{https://swprs.org/die-integrity-initiative/}{Die Integrity
  Initiative}
\item
  \href{https://swprs.org/die-propaganda-matrix/}{Die Propaganda-Matrix}
\item
  \href{https://swprs.org/der-propaganda-multiplikator/}{Der
  Propaganda-Multiplikator}
\end{itemize}

\begin{center}\rule{0.5\linewidth}{\linethickness}\end{center}

Publiziert: November 2018

\hypertarget{swiss-policy-research}{%
\subsubsection{Swiss Policy Research}\label{swiss-policy-research}}

\begin{itemize}
\tightlist
\item
  \href{https://swprs.org/kontakt/}{Kontakt}
\item
  \href{https://swprs.org/uebersicht/}{Übersicht}
\item
  \href{https://swprs.org/donationen/}{Donationen}
\item
  \href{https://swprs.org/disclaimer/}{Disclaimer}
\end{itemize}

\hypertarget{english}{%
\subsubsection{English}\label{english}}

\begin{itemize}
\tightlist
\item
  \href{https://swprs.org/contact/}{About Us / Contact}
\item
  \href{https://swprs.org/media-navigator/}{The Media Navigator}
\item
  \href{https://swprs.org/the-american-empire-and-its-media/}{The CFR
  and the Media}
\item
  \href{https://swprs.org/donations/}{Donations}
\end{itemize}

\hypertarget{follow-by-email}{%
\subsubsection{Follow by email}\label{follow-by-email}}

Follow

\href{https://wordpress.com/?ref=footer_custom_com}{WordPress.com}.

\protect\hyperlink{}{Up ↑}

Post to

\protect\hyperlink{}{Cancel}

\includegraphics{https://pixel.wp.com/b.gif?v=noscript}
