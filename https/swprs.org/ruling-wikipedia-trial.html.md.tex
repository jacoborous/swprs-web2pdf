\protect\hyperlink{content}{Skip to content}

\href{https://swprs.org/}{}

\protect\hyperlink{search-container}{Search}

Search for:

\href{https://swprs.org/}{\includegraphics{https://swprs.files.wordpress.com/2020/05/swiss-policy-research-logo-300.png}}

\href{https://swprs.org/}{Swiss Policy Research}

Geopolitics and Media

Menu

\begin{itemize}
\tightlist
\item
  \href{https://swprs.org}{Start}
\item
  \href{https://swprs.org/srf-propaganda-analyse/}{Studien}

  \begin{itemize}
  \tightlist
  \item
    \href{https://swprs.org/srf-propaganda-analyse/}{SRF / ZDF}
  \item
    \href{https://swprs.org/die-nzz-studie/}{NZZ-Studie}
  \item
    \href{https://swprs.org/der-propaganda-multiplikator/}{Agenturen}
  \item
    \href{https://swprs.org/die-propaganda-matrix/}{Medienmatrix}
  \end{itemize}
\item
  \href{https://swprs.org/medien-navigator/}{Analysen}

  \begin{itemize}
  \tightlist
  \item
    \href{https://swprs.org/medien-navigator/}{Navigator}
  \item
    \href{https://swprs.org/der-propaganda-schluessel/}{Techniken}
  \item
    \href{https://swprs.org/propaganda-in-der-wikipedia/}{Wikipedia}
  \item
    \href{https://swprs.org/logik-imperialer-kriege/}{Kriege}
  \end{itemize}
\item
  \href{https://swprs.org/netzwerk-medien-schweiz/}{Netzwerke}

  \begin{itemize}
  \tightlist
  \item
    \href{https://swprs.org/netzwerk-medien-schweiz/}{Schweiz}
  \item
    \href{https://swprs.org/netzwerk-medien-deutschland/}{Deutschland}
  \item
    \href{https://swprs.org/medien-in-oesterreich/}{Österreich}
  \item
    \href{https://swprs.org/das-american-empire-und-seine-medien/}{USA}
  \end{itemize}
\item
  \href{https://swprs.org/bericht-eines-journalisten/}{Fokus I}

  \begin{itemize}
  \tightlist
  \item
    \href{https://swprs.org/bericht-eines-journalisten/}{Journalistenbericht}
  \item
    \href{https://swprs.org/russische-propaganda/}{Russische Propaganda}
  \item
    \href{https://swprs.org/die-israel-lobby-fakten-und-mythen/}{Die
    »Israel-Lobby«}
  \item
    \href{https://swprs.org/geopolitik-und-paedokriminalitaet/}{Pädokriminalität}
  \end{itemize}
\item
  \href{https://swprs.org/migration-und-medien/}{Fokus II}

  \begin{itemize}
  \tightlist
  \item
    \href{https://swprs.org/covid-19-hinweis-ii/}{Coronavirus}
  \item
    \href{https://swprs.org/die-integrity-initiative/}{Integrity
    Initiative}
  \item
    \href{https://swprs.org/migration-und-medien/}{Migration \& Medien}
  \item
    \href{https://swprs.org/der-fall-magnitsky/}{Magnitsky Act}
  \end{itemize}
\item
  \href{https://swprs.org/kontakt/}{Projekt}

  \begin{itemize}
  \tightlist
  \item
    \href{https://swprs.org/kontakt/}{Kontakt}
  \item
    \href{https://swprs.org/uebersicht/}{Seitenübersicht}
  \item
    \href{https://swprs.org/medienspiegel/}{Medienspiegel}
  \item
    \href{https://swprs.org/donationen/}{Donationen}
  \end{itemize}
\item
  \href{https://swprs.org/contact/}{English}
\end{itemize}

\protect\hyperlink{}{Open Search}

\hypertarget{ruling-in-german-wikipedia-trial}{%
\section{Ruling in German
Wikipedia~trial}\label{ruling-in-german-wikipedia-trial}}

\textbf{Published}: February 21, 2019; \textbf{Updated}: February 2020;
\textbf{Languages}: \href{https://swprs.org/der-wikipedia-prozess/}{DE},
EN*\\
*

\textbf{German court: Judgment in one of the most important modern media
trials.}

In 2018, researchers of the Vienna-based
\href{https://gruppe42.com/}{Group42} reported on one of the most
influential and aggressive German Wikipedia editors and for the first
time revealed his real name. In response, the editor obtained a
preliminary injunction with the threat of a penalty of up to €250,000.

The Hamburg Regional Court has now decided in a
\href{https://kenfm.de/tagesdosis-26-2-2019-wikipedia-manipulationen-feliks-darf-nach-gerichtsurteil-wieder-mit-klarnamen-genannt-werden/}{landmark
ruling} that the naming was lawful due to the overriding public
interest.

The Wikipedia author in question, using the pseudonym ``Feliks'', is a
former functionary of the pro-Israel wing of the German Left Party and a
former \href{https://en.wikipedia.org/wiki/Sar-El}{foreign member} of
the Israeli army with special insignia of the US army and other armed
forces.

The author edited a total of several thousand Wikipedia articles and
denounced numerous people, including in particular politicians,
publicists and researchers, who had expressed critical views on US or
Israeli positions.

The self-chosen code name ``Feliks'' refers to the founder and first
director of the Soviet secret service Cheka/GPU, Feliks Dzerzhinsky,
under whose leadership up to one hundred thousand political opponents
were executed.

The Hamburg ruling is likely to set a precedent and may have a
significant impact. According to initial statements, several affected
parties are currently considering legal action against ``Feliks'' and
other denunciative or manipulative Wikipedia authors.

The significance of the Hamburg Wikipedia verdict could even surpass
that of the
\href{http://www.spiegel.de/kultur/tv/verfassungsgericht-klage-gegen-zdf-staatsvertrag-a-960571.html}{ZDF
verdict} of the German Federal Constitutional Court in 2014, especially
in view of the general relevance of the German Wikipedia (approx. one
billion page views by 100 million devices
\href{https://stats.wikimedia.org/v2/\#/de.wikipedia.org}{per month}).

Already in August 2018, the Berlin Regional Court
\href{https://www.heise.de/newsticker/meldung/Urteil-gegen-Wikipedia-Keine-rufschaedigende-Kritik-ohne-Recherche-4209610.html}{revoked}
Wikipedia's so-called ``layman's privilege''. This means that Wikipedia
authors are no longer allowed to use reputation-damaging claims from
media articles or other sources without additional verification.

The Wikimedia Foundation as well as traditional media have so far not
made any statements about the systematic political manipulation of
Wikipedia and the new Hamburg judgment.

\textbf{Addendum:} In February 2020, the ruling was
\href{https://kanzleikompa.de/2020/02/18/olg-hamburg-deanonymisierung-von-autoren-politischer-beitraege-zulaessig/}{confirmed}
by the Hamburg Higher Regional Court.

\textbf{See also}:
\href{https://swprs.org/wikipedia-disinformation-operation/}{Wikipedia:
A Disinformation Operation?} (SPR, March 2020)

\begin{center}\rule{0.5\linewidth}{\linethickness}\end{center}

Published: February 2019; updated: February 2020

\hypertarget{swiss-policy-research}{%
\subsubsection{Swiss Policy Research}\label{swiss-policy-research}}

\begin{itemize}
\tightlist
\item
  \href{https://swprs.org/kontakt/}{Kontakt}
\item
  \href{https://swprs.org/uebersicht/}{Übersicht}
\item
  \href{https://swprs.org/donationen/}{Donationen}
\item
  \href{https://swprs.org/disclaimer/}{Disclaimer}
\end{itemize}

\hypertarget{english}{%
\subsubsection{English}\label{english}}

\begin{itemize}
\tightlist
\item
  \href{https://swprs.org/contact/}{About Us / Contact}
\item
  \href{https://swprs.org/media-navigator/}{The Media Navigator}
\item
  \href{https://swprs.org/the-american-empire-and-its-media/}{The CFR
  and the Media}
\item
  \href{https://swprs.org/donations/}{Donations}
\end{itemize}

\hypertarget{follow-by-email}{%
\subsubsection{Follow by email}\label{follow-by-email}}

Follow

\href{https://wordpress.com/?ref=footer_custom_com}{WordPress.com}.

\protect\hyperlink{}{Up ↑}

\includegraphics{https://pixel.wp.com/b.gif?v=noscript}
