\protect\hyperlink{content}{Skip to content}

\href{https://swprs.org/}{}

\protect\hyperlink{search-container}{Search}

Search for:

\href{https://swprs.org/}{\includegraphics{https://swprs.files.wordpress.com/2020/05/swiss-policy-research-logo-300.png}}

\href{https://swprs.org/}{Swiss Policy Research}

Geopolitics and Media

Menu

\begin{itemize}
\tightlist
\item
  \href{https://swprs.org}{Start}
\item
  \href{https://swprs.org/srf-propaganda-analyse/}{Studien}

  \begin{itemize}
  \tightlist
  \item
    \href{https://swprs.org/srf-propaganda-analyse/}{SRF / ZDF}
  \item
    \href{https://swprs.org/die-nzz-studie/}{NZZ-Studie}
  \item
    \href{https://swprs.org/der-propaganda-multiplikator/}{Agenturen}
  \item
    \href{https://swprs.org/die-propaganda-matrix/}{Medienmatrix}
  \end{itemize}
\item
  \href{https://swprs.org/medien-navigator/}{Analysen}

  \begin{itemize}
  \tightlist
  \item
    \href{https://swprs.org/medien-navigator/}{Navigator}
  \item
    \href{https://swprs.org/der-propaganda-schluessel/}{Techniken}
  \item
    \href{https://swprs.org/propaganda-in-der-wikipedia/}{Wikipedia}
  \item
    \href{https://swprs.org/logik-imperialer-kriege/}{Kriege}
  \end{itemize}
\item
  \href{https://swprs.org/netzwerk-medien-schweiz/}{Netzwerke}

  \begin{itemize}
  \tightlist
  \item
    \href{https://swprs.org/netzwerk-medien-schweiz/}{Schweiz}
  \item
    \href{https://swprs.org/netzwerk-medien-deutschland/}{Deutschland}
  \item
    \href{https://swprs.org/medien-in-oesterreich/}{Österreich}
  \item
    \href{https://swprs.org/das-american-empire-und-seine-medien/}{USA}
  \end{itemize}
\item
  \href{https://swprs.org/bericht-eines-journalisten/}{Fokus I}

  \begin{itemize}
  \tightlist
  \item
    \href{https://swprs.org/bericht-eines-journalisten/}{Journalistenbericht}
  \item
    \href{https://swprs.org/russische-propaganda/}{Russische Propaganda}
  \item
    \href{https://swprs.org/die-israel-lobby-fakten-und-mythen/}{Die
    »Israel-Lobby«}
  \item
    \href{https://swprs.org/geopolitik-und-paedokriminalitaet/}{Pädokriminalität}
  \end{itemize}
\item
  \href{https://swprs.org/migration-und-medien/}{Fokus II}

  \begin{itemize}
  \tightlist
  \item
    \href{https://swprs.org/covid-19-hinweis-ii/}{Coronavirus}
  \item
    \href{https://swprs.org/die-integrity-initiative/}{Integrity
    Initiative}
  \item
    \href{https://swprs.org/migration-und-medien/}{Migration \& Medien}
  \item
    \href{https://swprs.org/der-fall-magnitsky/}{Magnitsky Act}
  \end{itemize}
\item
  \href{https://swprs.org/kontakt/}{Projekt}

  \begin{itemize}
  \tightlist
  \item
    \href{https://swprs.org/kontakt/}{Kontakt}
  \item
    \href{https://swprs.org/uebersicht/}{Seitenübersicht}
  \item
    \href{https://swprs.org/medienspiegel/}{Medienspiegel}
  \item
    \href{https://swprs.org/donationen/}{Donationen}
  \end{itemize}
\item
  \href{https://swprs.org/contact/}{English}
\end{itemize}

\protect\hyperlink{}{Open Search}

\hypertarget{trump-die-medien-und-die-geopolitik}{%
\section{Trump, die Medien, und
die~Geopolitik}\label{trump-die-medien-und-die-geopolitik}}

\textbf{Publiziert}: August 2017; \textbf{Aktualisiert}: April 2018\\
\textbf{Teilen auf}:
\href{https://twitter.com/intent/tweet?url=https://swprs.org/trump-medien-geopolitik/}{Twitter}
/
\href{https://www.facebook.com/share.php?u=https://swprs.org/trump-medien-geopolitik/}{Facebook}

Die folgende Analyse befasst sich mit der Frage, wie die bislang
auffallend negative Bericht­erstattung der traditionellen westlichen
Medien über die Trump-Präsidentschaft zu erklären ist. Dabei zeigt sich,
dass keine der üblichen Erklärungen -- die angebliche Inkompetenz
Trumps, eine angebliche »Links­lastig­keit« der Medien, Einschaltquoten
oder Partikularinteressen einflussreicher Lobbys -- stichhaltig ist.

Vielmehr dürfte die negative Berichterstattung auf geostrategische
Aspekte und die (bedrohte) Rolle des \emph{Council on Foreign Relations}
als oberstes geopolitisches Gremium der Vereinigten Staaten
zurück­zu­führen sein. Die Bericht­erstattung westlicher Medien weist
denn auch deutliche Parallelen zur koordinierten Medien­aktivität im
Rahmen früherer *Regime-Change-*Operationen in Drittstaaten auf.

Zur einführenden Lektüre empfohlen:
\href{https://swprs.org/das-american-empire-und-seine-medien/}{Das
American Empire und seine Medien}

\begin{center}\rule{0.5\linewidth}{\linethickness}\end{center}

\hypertarget{ausgangslage-und-erkluxe4rungsversuche}{%
\paragraph{Ausgangslage und
Erklärungsversuche}\label{ausgangslage-und-erkluxe4rungsversuche}}

Die Ausgangslage ist eindeutig: Gemäß einer
\href{http://meedia.de/2017/05/23/harvard-studie-keiner-berichtete-negativer-ueber-donald-trump-als-die-ard/}{Harvard-Studie}
berichteten die traditionellen westlichen Medien bislang überwiegend
negativ über die Trump-Präsidentschaft: So fielen insgesamt 80\%, bei
der \emph{New York Times} 87\%, bei \emph{CNN} 93\%, und bei der
\emph{ARD} sogar 98\% der wertenden Beiträge negativ aus.

Zur Erklärung dieser einzigartig negativen Berichterstattung werden im
Allgemeinen vier mögliche Varianten diskutiert, von denen jedoch keine
stichhaltig ist, wie die folgende Analyse zeigt:

\begin{enumerate}
\def\labelenumi{\arabic{enumi}.}
\tightlist
\item
  \textbf{Trump sei ein unsympathischer und unfähiger Politiker, über
  den die Medien kritisch berichten müssen:} Diese These scheitert schon
  daran, dass rund 50\% der US-Wähler dies offenbar nicht so gesehen
  haben. Doch selbst wenn die Einschätzung zutrifft: Die USA hatten auch
  in der Vergangenheit Präsidenten mit teils fraglichen Qualifikationen,
  über die ebenso wohlwollend berichtet wurde wie über US-Verbündete,
  die nicht eben Sympathieträger sind. Hinzu kommt, dass dieselben
  Medien über denselben Trump in der Vergangenheit zumeist
  \href{https://www.youtube.com/watch?v=SEPs17_AkTI}{positiv} berichtet
  haben.
\item
  \textbf{Die Medien in den USA und in Europa seien eben »linkslastig«
  und würden den konservativen Trump deshalb ablehnen:} Diese Erklärung
  steht im Widerspruch zur positiven Berichterstattung über frühere
  republikanische Präsidenten und über republikanische Mitbewerber
  Trumps. Zudem haben gemäß der
  \href{http://meedia.de/2017/05/23/harvard-studie-keiner-berichtete-negativer-ueber-donald-trump-als-die-ard/}{Harvard-Studie}
  selbst konservative Medien wie \emph{Fox News} entgegen einer
  weitverbreiteten Annahme tendenziell kritisch (52\%) über Trump
  berichtet.
\item
  \textbf{Verantwortlich seien Partikularinteressen einflussreicher
  Lobbys, etwa der Rüstungs-, Öl- oder Finanzindustrie oder der
  »Israel-Lobby«:} Auch dieser Erklärungsversuch kann nicht überzeugen,
  denn keine dieser durchaus potenten Einflussgruppen hat Grund zur
  Klage über Trump: Trump setzte sich stets für eine Aufrüstung des
  US-Militärs und der NATO ein und schloss historische Waffengeschäfte
  mit Verbündeten wie Saudi-Arabien ab. Zudem machte er den CEO des
  Ölgiganten \emph{Exxon Mobil} zu seinem Außenminister und engagierte
  sich für die Förderung fossiler Energieträger. Überdies holte er
  zahlreiche *Wall-Street-*Banker und Finanzmilliardäre in sein
  Kabinett, und versprach mehr Unterstützung für Israel sowie eine
  mögliche Anerkennung Jerusalems als Hauptstadt.
\item
  \textbf{Ausschlaggebend seien die durch Skandale erzielbaren
  Zuschauerquoten und Leserzahlen:} Tatsächlich sorgte der
  polarisierende Trump schon immer für hohe Einschaltquoten. Dies gilt
  indes für beinahe jede Art der Berichterstattung über ihn, keineswegs
  nur für eine negative. Zudem verfolgt die beobachtete
  Berichterstattung zweifellos politische und nicht nur
  medienökonomische Ziele.
\end{enumerate}

Offensichtlich vermag keine dieser Varianten die überwiegend negative
Berichterstattung schlüssig zu erklären. Der tatsächliche Grund dürfte
denn auch tiefer liegen -- und geopolitischer Natur sein: Trump kam mit
seiner national orientierten, »isolationistischen« Politik den globalen
Ambitionen des amerikanischen \emph{Council on Foreign Relations (CFR)}
in die Quere.

Wie in einem
\href{https://swprs.org/das-american-empire-und-seine-medien/}{früheren
Beitrag} aufgezeigt, prägen der parteiübergreifende \emph{Council on
Foreign Relations} und seine inzwischen knapp 5000 Mitglieder in
Spitzenpositionen von Politik, Wirtschaft, Wissenschaft und Medien seit
Jahrzehnten die Außenpolitik der Vereinigten Staaten. Dabei haben die
Strategen des Councils nie ein Geheimnis daraus gemacht, dass das Ziel
darin
\href{https://www.amazon.de/Die-einzige-Weltmacht-Strategie-Vorherrschaft/dp/3596143586}{besteht},
ein globales, geoökonomisches Imperium unter amerikanischer Führung zu
etablieren (die sogenannte
\href{https://swprs.files.wordpress.com/2017/09/domhoff-cfr-2014.pdf}{\emph{Grand
Area Strategy}}).

\href{https://swprs.files.wordpress.com/2018/02/cfr-imperial-council-hdv.png}{\includegraphics{https://swprs.files.wordpress.com/2018/02/cfr-imperial-council-hdv.png?w=736}}\emph{CFR-Mitglieder
in den Schlüsselpositionen des American Empire von 1945 bis 2017.\\
\href{https://swprs.files.wordpress.com/2018/02/cfr-imperial-council-hdv.png}{Vergrößern}}
🔎

\hypertarget{das-trauma-von-1920}{%
\paragraph{Das »Trauma von 1920«}\label{das-trauma-von-1920}}

Tatsächlich wurde der CFR überhaupt erst aufgrund des sogenannten
\href{http://www.spiegel.de/spiegel/print/d-41389590.html}{»Traumas von
1920«} gegründet: Nach dem Ersten Weltkrieg hätten die USA erstmals die
globale Führungsrolle übernehmen können ~-- doch der Senat entschied
sich gegen den Beitritt zum Völkerbund und die kriegsmüde Bevölkerung
wählte mit Warren Harding einen Präsidenten, der eine
\href{https://en.wikipedia.org/wiki/Return_to_normalcy}{»Rückkehr zur
Normalität«} versprach und sich zuerst um die Angelegenheiten und
Probleme Amerikas und der Amerikaner kümmern wollte.

Mit seiner *»America First«-*Politik -- die bislang unter anderem in der
Aufkündigung der transatlantischen und transpazifischen
Freihandelsverträge und des Pariser Klimaabkommens, der Blockade in
Migrationsfragen, der Verständigungspolitik gegenüber Herausforderer
Russland und einem Kurswechsel in Nahost resultierte -- reaktivierte
Trump dieses hundertjährige geostrategische Trauma und stellte
gleichzeitig die geopolitische Führungsrolle des Councils und seiner
Mitglieder in Frage.

Tatsächlich dürfte Trump der erste US-Präsident seit dem Zweiten
Weltkrieg sein, der nicht CFR-Mitglied oder wenigstens CFR-konform ist
(Kennedy verließ den geopolitischen CFR-Konsens erst im Laufe seiner
Präsidentschaft). Möglich wurde dies durch die unerwartete Niederlage
von Favoritin Clinton, deren Ehemann und Tochter
\href{https://swprs.files.wordpress.com/2017/07/cfr-members-2016.pdf}{Council-Mitglieder}
sind und die als Außenministerin selbst diverse
\href{https://www.youtube.com/watch?v=Kfpgl6NqF0I}{Ansprachen} vor dem
Council hielt
(\href{https://www.wikileaks.org/clinton-emails/emailid/7006}{»Fortschrittsberichte«}
gemäß einer *Wikileaks-*Email).

Es ist verständlich, dass der Council auf dieses Debakel reagieren
musste. Dabei ist zu bedenken, dass Eigentümer, Führungskräfte und
Top-Journalisten nahezu aller etablierten US-Medien gleichzeitig
CFR-Mitglieder sind. Auch die Schlüsselpersonen der etablierten
europäischen Medien sind -- aus historischen und sicherheitspolitischen
Gründen -- via Bilderberg-Gruppe, Trilateraler Kommission,
Atlantik-Brücke und weiterer CFR-Ableger in das internationale Netzwerk
des Councils eingebunden und sorgen für eine entsprechend CFR-konforme
Berichterstattung und Kommentierung (siehe Grafiken).

\href{https://swprs.files.wordpress.com/2017/08/cfr-media-network-hdv-spr.png}{\includegraphics{https://swprs.files.wordpress.com/2017/08/cfr-media-network-hdv-spr.png?w=736\&h=526}\emph{Vergrößern}
🔎}

\href{https://swprs.files.wordpress.com/2017/08/netzwerk-medien-deutschland-spr-mt.png}{}

\includegraphics{https://swprs.files.wordpress.com/2017/08/netzwerk-medien-deutschland-spr-mt.png?w=399\&h=255}

Medien in Deutschland: Das Transatlantik-Netzwerk

\href{https://swprs.files.wordpress.com/2019/10/medien-netzwerk-schweiz-hdz.png}{}

\includegraphics{https://swprs.files.wordpress.com/2019/10/medien-netzwerk-schweiz-hdz.png?w=329\&h=255}

Schweizer Medien: Das Transatlantik-Netzwerk

Insofern kann es nicht überraschen, dass dieses historisch einzigartige,
transatlantische Publizistik-Netz­werk -- das bereits unzählige
\emph{\href{https://www.amazon.com/Killing-Hope-Military-Interventions-Updated/dp/1783601779}{Regime
Changes}} und Militärinterventionen in Drittstaaten erfolgreich
angestimmt hat -- einmal mehr aktiviert wurde, um den »Usurpator«
(Thronräuber) Trump abzuwehren beziehungsweise -- nach dessen Wahlsieg
-- doch noch zu bekehren -- oder notfalls zu stürzen.

\hypertarget{ein-grouxdfer-moment}{%
\paragraph{»Ein großer Moment«}\label{ein-grouxdfer-moment}}

Damit erklärt sich zugleich, warum es während der ersten einhundert Tage
von Trumps Präsidentschaft trotz aller negativen Schlagzeilen zwei
Ereignisse gab, über die CFR-konforme Medien beidseits des Atlantiks
beinahe einstimmig positiv berichteten: Die Ernennung von H.R. McMaster
zum Nationalen Sicherheitsberater am 20. Februar 2017, sowie der
(illegale) *Cruise-Missile-*Angriff auf Syrien am 7. April 2017. Einige
der damaligen Schlagzeilen lauteten wie folgt:

\begin{itemize}
\tightlist
\item
  \textbf{Zur Ernennung von McMaster:} „Trumps brillante Wahl von
  McMaster``
  (\href{http://edition.cnn.com/2017/02/20/opinions/trumps-brilliant-choice-of-mcmaster-bergen/index.html}{CNN});
  „gemäßigt und moralisch integer``
  (\href{http://www.sueddeutsche.de/politik/usa-trump-ernennt-hr-mcmaster-zum-nationalen-sicherheitsberater-1.3388283}{Süddeutsche});
  „ein General, der allen passt``
  (\href{http://www.zeit.de/politik/ausland/2017-02/h-r-mcmaster-donald-trump-nationaler-sicherheitsberater}{Die
  Zeit}); „ein führender Intellektueller innerhalb des Militärs``
  (\href{https://www.nytimes.com/2017/02/20/us/politics/mcmaster-national-security-adviser-trump.html}{New
  York Times}); „Die USA und die Welt sind sicherer durch diese
  Entscheidung``
  (\href{https://www.theatlantic.com/politics/archive/2017/02/trump-gets-an-upgrade-at-national-security-adviser/517278/}{The
  Atlantic}); „ein dekorierter und hoch angesehener Absolvent der
  Militärakademie West Point``
  (\href{https://web.archive.org/web/20170220203844/https://www.tagesschau.de/ausland/trump-sicherheitsberater-mcmaster-101.html}{ARD});
  „Trump erntet Lob``
  (\href{http://www.spiegel.de/politik/ausland/donald-trump-und-sein-neuer-sicherheitsberater-h-r-mcmaster-es-gibt-lob-a-1135527.html}{Der
  Spiegel}); „Eine ausgezeichnete Wahl``
  (\href{https://www.nytimes.com/2017/02/20/us/politics/mcmaster-national-security-adviser-trump.html}{John
  McCain})
\item
  \textbf{Zum Angriff auf Syrien:} „Die europäische Presse lobt Donald
  Trump, einige feiern ihn sogar``
  (\href{http://www.t-online.de/nachrichten/ausland/krisen/id_80852290/syrien-angriff-donald-trump-hat-ausnahmsweise-richtig-gehandelt-.html}{DPA});
  „Trump hat ausnahmsweise richtig gehandelt``
  (\href{http://diepresse.com/home/meinung/kommentare/leitartikel/5197748/Leitartikel_Donald-Trump-handelte-ausnahmsweise-einmal-richtig}{Presse});
  „Eine notwendige Strafe für Asad``
  (\href{https://www.nzz.ch/international/militaerschlag-der-usa-in-syrien-eine-notwendige-strafe-fuer-asad-ld.612272}{NZZ});
  „Die Profis übernehmen das Kommando``
  (\href{http://www.handelsblatt.com/politik/international/trump-regierung-die-profis-uebernehmen-das-kommando/19632534.html}{Handelsblatt});
  „Die überraschende Wandlung des US-Präsidenten``
  (\href{https://www.welt.de/politik/ausland/article163496640/Die-ueberraschende-Wandlung-des-US-Praesidenten.html}{Die
  Welt}); „Syrien-Luftschlag krönt erfolgreiche Woche für Trump``
  (\href{http://nypost.com/2017/04/08/airstrike-on-syria-caps-a-successful-week-for-trump/}{New
  York Post}); „Ein Syrer bedankt sich bei Trump``
  (\href{http://www.cnn.com/videos/world/2017/04/07/syrian-survivor-thanks-trump-nr.cnn}{CNN})
\end{itemize}

Weshalb diese beiden bemerkenswerten Ausnahmen? Mit der nachträglichen
Ernennung McMasters holte Trump -- nachdem er Vorgänger Michael Flynn
auf Druck der Medien entlassen musste -- erstmals ein
\href{https://swprs.files.wordpress.com/2017/07/cfr-members-2016.pdf}{CFR-Mitglied}
in eine Schlüsselposition seines Kabinetts. Der Council -- der seit dem
Zweiten Weltkrieg nahezu alle Außen-, Verteidigungs- und Finanzminister,
Nationalen Sicherheitsberater und CIA-Direktoren stellte~ (siehe Grafik
oben) -- konnte damit einen ersten wichtigen Etappensieg erringen.

Und der eigenmächtige Raketenangriff auf Syrien -- basierend auf einem
\href{https://consortiumnews.com/2017/04/05/another-dangerous-rush-to-judgment-in-syria/}{dubiosen}
»Giftgasangriff« -- war ein entschieden imperialer Zug, mit dem Trump
erstmals der langjährigen CFR-Strategie gegenüber Syrien und Russland
folgte. CNN-Topjournalist und Trump-Kritiker Fareed Zakaria
\href{http://www.washingtontimes.com/news/2017/apr/7/fareed-zakaria-donald-trump-became-president-last-/}{meinte
damals sogar}, dass Trump (erst) »in dieser Nacht zum Präsidenten der
Vereinigten Staaten« wurde:

„Ich denke, das war ein großer Moment. Trump realisierte, dass der
US-Präsident handeln und internationale Normen durchsetzen muss. Zum
ersten Mal sprach er von internationalen Normen und Regeln und über die
Rolle Amerikas, Gerechtigkeit in der Welt durchzusetzen. Es ist diese
Art Rhetorik, die wir von amerikanischen Präsidenten seit Truman {[}d.h.
seit dem 2. WK{]} erwarten, die aber Trump bewusst nie benutzt hat,
weder in seiner Wahlkampagne noch in seiner Inaugurationsrede. Das war
also eine interessante Wandlung und eine Art Erziehung von Donald
Trump.``

Zakaria musste es wissen, denn er ist nicht nur CNN-Journalist --
sondern auch
\href{https://www.cfr.org/board-directors}{Vorstandsmitglied} des
\emph{Council on Foreign Relations} (sowie Mitglied der Trilateralen
Kommission und mehrfacher Teilnehmer der Bilderberg-Konferenz).
Allerdings hielt diese »Wandlung und Erziehung« von Trump vorerst nur
kurz an, weshalb auch die CFR-affinen Medien alsbald zu ihrer Kritik an
Trump zurückkehrten.

\hypertarget{mord-im-weiuxdfen-haus-zum-beispiel}{%
\paragraph{»Mord im Weißen Haus zum
Beispiel«}\label{mord-im-weiuxdfen-haus-zum-beispiel}}

Die Rolle des CFR erklärt schließlich auch die ungewöhnlich offensive
Berichterstattung der europäischen Medien, die ja sonst eher
\href{https://swprs.org/die-nzz-studie/}{US-konform} ausfällt. Denn die
europäischen Regierungen und Medien richten sich durchaus nicht nach dem
jeweiligen US-Präsidenten -- der ja ohnehin nur für ein paar Jahre im
Amt ist -- sondern nach dem seit Jahrzehnten das weltumspannende
\emph{American Empire} dirigierenden Council. Dieser entscheidende
Unterschied wurde jedoch erst mit Trump bedeutsam und für die
Öffentlichkeit sichtbar, da Präsident und Council nun erstmals nicht
mehr auf derselben Linie lagen.

Wäre Trump von Anbeginn ein CFR-konformer Kandidat gewesen, so hätten
ihn die exakt selben Journalisten, die ihn nun kritisieren, vermutlich
als »visionären Unternehmer«, »pragmatischen Verhandlungspartner« und
»standhaften Führer der freien Welt« gelobt -- charakterliche Schwächen
hin oder her. Allerdings wäre Trump in diesem Fall wohl gar nicht erst
zum US-Präsidenten gewählt worden.

Nun aber muss der Council mit seiner geballten Medienmacht versuchen,
Präsident Trump auf CFR-Kurs zu bringen. Gelingt dies nicht, bliebe nur
noch die Amtsenthebung mittels eines echten oder inszenierten Skandals.
Oder aber es tritt jenes Szenario ein, das Josef Joffe, der Herausgeber
der \emph{ZEIT} und ehemaliges Mitglied von Atlantik-Brücke,~
Bilderberg-Gruppe und Trilateraler Kommission, im \emph{Presseclub} der
ARD bereits antizipiert hat:
\emph{\href{https://swprs.org/video-joffe-trump/}{»Mord im Weißen Haus
zum Beispiel«}.}

Trump versucht seinerseits, über neue und soziale Medien das
Medienimperium des Councils zu umgehen und zu untergraben -- wobei sich
beide Seiten gegenseitig vorwerfen, \emph{»Fake-News«} zu verbreiten.
Der Council reagierte hierauf mit diversen
\href{http://www.npr.org/2017/05/10/527720078/nato-takes-aim-at-disinformation-campaigns}{Kampagnen}
zur Abwehr von (angeblich russischer, also geopolitischer)
»Desinformation« sowie mit
\href{https://www.heise.de/tp/features/Facebook-Fake-News-und-die-Privatisierung-der-Zensur-3599878.html}{Restriktionen}
für soziale Medien und sogar
\href{https://www.wsws.org/en/articles/2017/08/04/goog-a04.html}{Suchmaschinen}
-- wovon längst nicht nur Trump-Anhänger
\href{https://www.wsws.org/en/articles/2017/08/02/pers-a02.html}{betroffen}
sind. All dies ist freilich nicht erstaunlich, sind doch die
\href{https://swprs.org/das-american-empire-und-seine-medien/}{Führungskräfte}
von Google, Youtube, Facebook \& Co. selbst CFR-Mitglieder.

In den kommenden Jahren wird sich zeigen, ob durch diesen
geostrategischen Machtkampf letztlich das Ende der medialen
Einheitsmeinung, oder eher das Ende der Meinungsfreiheit eingeläutet
wird.

***

\textbf{Postskriptum 1:} Am 18. August 2017 verließ Trumps
nationalkonservativer Strategiechef, Stephen Bannon, die US-Regierung.
Die \emph{New York Times} schrieb einen Tag zuvor in einem
\href{https://www.nytimes.com/2017/08/17/us/politics/steve-bannon-nationalism-trump.html}{Artikel}
über Bannon:

``Von Afghanistan und Nordkorea bis Syrien und Venezuela argumentierte
Herr Bannon gegen militärische Drohungen oder die Entsendung
amerikanischer Truppen in ausländische Konflikte. () Ban­nons Erzfeind
im Weißen Haus war {[}der Nationale Sicherheitsberater und
CFR-Vertreter{]} General McMaster, der Anführer dessen, was Bannon das
»globalistische Imperialprojekt« (globalist empire project) nannte --
ein parteiübergreifender außenpolitischer Konsens, der das aktive
amerikanische Engagement auf der ganzen Welt betont. Herr Bannon lehnt
diese Philosophie rundweg ab.''

Der Rücktritt Bannons wurde von CFR-konformen Medien beidseits des
Atlantiks und jedweder politischen Ausrichtung entsprechend einhellig
\href{http://www.spiegel.de/politik/ausland/stephen-bannon-pressestimmen-zum-ruecktritt-a-1163595.html}{begrüßt}.
Drei Tage später gaben Trump und McMaster die Ausweitung des
Afghanistan-Krieges
\href{http://www.huffingtonpost.com/entry/hr-mccaster-donald-trump-afghanistan_us_599c6105e4b06a788a2c2026}{bekannt}.

\textbf{Postskriptum 2}: Am 9. April 2018 wurde H.R. McMaster als
Nationaler Sicherheitsberater durch
\href{https://de.wikipedia.org/wiki/John_R._Bolton}{John Bolton}
abgelöst. Bolton ist ebenfalls CFR-Mitglied und vertritt insbesondere
gegenüber dem Iran eine deutlich aggressivere Politik.

\hypertarget{referenzen}{%
\paragraph{Referenzen}\label{referenzen}}

\begin{enumerate}
\def\labelenumi{\arabic{enumi}.}
\tightlist
\item
  \textbf{Council on Foreign Relations:}

  \begin{itemize}
  \tightlist
  \item
    \href{https://www.cfr.org/board-directors}{Vorstandsmitglieder}
  \item
    Mitgliederverzeichnisse,
    \href{https://swprs.files.wordpress.com/2017/07/council-on-foreign-relations-membership-rosters-1922-2013.pdf}{1922
    bis 2013} und
    \href{https://swprs.files.wordpress.com/2017/07/cfr-members-2016.pdf}{2016}
  \item
    CFR-Mitglieder in der US-Regierung,
    \href{https://swprs.files.wordpress.com/2017/07/cfr-administration-members-1900-2014.pdf}{1900
    bis 2014}
  \end{itemize}
\item
  \href{https://shorensteincenter.org/news-coverage-donald-trumps-first-100-days/}{News
  Coverage of Donald Trump's First 100 Days}; \emph{Harvard Kennedy
  School, Shorenstein Center on Media, Politics and Public Policy;} Mai
  2017
\item
  Laurence H. Shoup (2015):
  \href{https://monthlyreview.org/product/wall-streets-think-tank/}{Wall
  Street's Think Tank: The Council on Foreign Relations and the Empire
  of Neoliberal Geopolitics, 1976-2014}, Monthly Review Press
  (\href{https://swprs.files.wordpress.com/2019/08/cfr-wallstreet-think-tank-shoup-2015.pdf}{PDF})
\end{enumerate}

\begin{center}\rule{0.5\linewidth}{\linethickness}\end{center}

\hypertarget{swiss-policy-research}{%
\subsubsection{Swiss Policy Research}\label{swiss-policy-research}}

\begin{itemize}
\tightlist
\item
  \href{https://swprs.org/kontakt/}{Kontakt}
\item
  \href{https://swprs.org/uebersicht/}{Übersicht}
\item
  \href{https://swprs.org/donationen/}{Donationen}
\item
  \href{https://swprs.org/disclaimer/}{Disclaimer}
\end{itemize}

\hypertarget{english}{%
\subsubsection{English}\label{english}}

\begin{itemize}
\tightlist
\item
  \href{https://swprs.org/contact/}{About Us / Contact}
\item
  \href{https://swprs.org/media-navigator/}{The Media Navigator}
\item
  \href{https://swprs.org/the-american-empire-and-its-media/}{The CFR
  and the Media}
\item
  \href{https://swprs.org/donations/}{Donations}
\end{itemize}

\hypertarget{follow-by-email}{%
\subsubsection{Follow by email}\label{follow-by-email}}

Follow

\href{https://wordpress.com/?ref=footer_custom_com}{WordPress.com}.

\protect\hyperlink{}{Up ↑}

Post to

\protect\hyperlink{}{Cancel}

\includegraphics{https://pixel.wp.com/b.gif?v=noscript}
