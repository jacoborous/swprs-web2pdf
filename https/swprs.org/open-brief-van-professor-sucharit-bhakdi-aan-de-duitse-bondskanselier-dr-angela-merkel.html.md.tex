\protect\hyperlink{content}{Skip to content}

\href{https://swprs.org/}{}

\protect\hyperlink{search-container}{Search}

Search for:

\href{https://swprs.org/}{\includegraphics{https://swprs.files.wordpress.com/2020/05/swiss-policy-research-logo-300.png}}

\href{https://swprs.org/}{Swiss Policy Research}

Geopolitics and Media

Menu

\begin{itemize}
\tightlist
\item
  \href{https://swprs.org}{Start}
\item
  \href{https://swprs.org/srf-propaganda-analyse/}{Studien}

  \begin{itemize}
  \tightlist
  \item
    \href{https://swprs.org/srf-propaganda-analyse/}{SRF / ZDF}
  \item
    \href{https://swprs.org/die-nzz-studie/}{NZZ-Studie}
  \item
    \href{https://swprs.org/der-propaganda-multiplikator/}{Agenturen}
  \item
    \href{https://swprs.org/die-propaganda-matrix/}{Medienmatrix}
  \end{itemize}
\item
  \href{https://swprs.org/medien-navigator/}{Analysen}

  \begin{itemize}
  \tightlist
  \item
    \href{https://swprs.org/medien-navigator/}{Navigator}
  \item
    \href{https://swprs.org/der-propaganda-schluessel/}{Techniken}
  \item
    \href{https://swprs.org/propaganda-in-der-wikipedia/}{Wikipedia}
  \item
    \href{https://swprs.org/logik-imperialer-kriege/}{Kriege}
  \end{itemize}
\item
  \href{https://swprs.org/netzwerk-medien-schweiz/}{Netzwerke}

  \begin{itemize}
  \tightlist
  \item
    \href{https://swprs.org/netzwerk-medien-schweiz/}{Schweiz}
  \item
    \href{https://swprs.org/netzwerk-medien-deutschland/}{Deutschland}
  \item
    \href{https://swprs.org/medien-in-oesterreich/}{Österreich}
  \item
    \href{https://swprs.org/das-american-empire-und-seine-medien/}{USA}
  \end{itemize}
\item
  \href{https://swprs.org/bericht-eines-journalisten/}{Fokus I}

  \begin{itemize}
  \tightlist
  \item
    \href{https://swprs.org/bericht-eines-journalisten/}{Journalistenbericht}
  \item
    \href{https://swprs.org/russische-propaganda/}{Russische Propaganda}
  \item
    \href{https://swprs.org/die-israel-lobby-fakten-und-mythen/}{Die
    »Israel-Lobby«}
  \item
    \href{https://swprs.org/geopolitik-und-paedokriminalitaet/}{Pädokriminalität}
  \end{itemize}
\item
  \href{https://swprs.org/migration-und-medien/}{Fokus II}

  \begin{itemize}
  \tightlist
  \item
    \href{https://swprs.org/covid-19-hinweis-ii/}{Coronavirus}
  \item
    \href{https://swprs.org/die-integrity-initiative/}{Integrity
    Initiative}
  \item
    \href{https://swprs.org/migration-und-medien/}{Migration \& Medien}
  \item
    \href{https://swprs.org/der-fall-magnitsky/}{Magnitsky Act}
  \end{itemize}
\item
  \href{https://swprs.org/kontakt/}{Projekt}

  \begin{itemize}
  \tightlist
  \item
    \href{https://swprs.org/kontakt/}{Kontakt}
  \item
    \href{https://swprs.org/uebersicht/}{Seitenübersicht}
  \item
    \href{https://swprs.org/medienspiegel/}{Medienspiegel}
  \item
    \href{https://swprs.org/donationen/}{Donationen}
  \end{itemize}
\item
  \href{https://swprs.org/contact/}{English}
\end{itemize}

\protect\hyperlink{}{Open Search}

\hypertarget{open-brief-van-professor-sucharit-bhakdi-aan-de-duitse-bondskanselier-angela-merkel}{%
\section{Open brief van professor Sucharit Bhakdi aan de Duitse
bondskanselier
Angela~Merkel}\label{open-brief-van-professor-sucharit-bhakdi-aan-de-duitse-bondskanselier-angela-merkel}}

\includegraphics{https://swprs.files.wordpress.com/2020/03/bakhdi-letter-header.png?w=736\&h=297}

\textbf{Talen}:
\href{https://swprs.org/offener-brief-von-professor-sucharit-bhakdi-an-bundeskanzlerin-dr-angela-merkel/}{DE},
\href{https://swprs.org/open-letter-from-professor-sucharit-bhakdi-to-german-chancellor-dr-angela-merkel/}{EN},
\href{https://swprs.org/professor-sucharit-bhakdi-avalik-kiri-angela-merkelile/}{EE},
\href{http://piensachile.com/2020/03/carta-abierta-a-angela-merkel/}{ES},
\href{https://swprs.org/covid-19-lettre-ouverte-du-professeur-sucharit-bhakdi-a-la-chanceliere-allemande-dre-angela-merkel/}{FR},
\href{https://swprs.org/professor-bhakdi-open-letter-greek/}{GR},
\href{https://yanivhamo.com/open-letter-from-professor-sucharit-bhakdi-to-german-chancellor-dr-angela-merkel-hebrew/}{HE},
\href{https://swprs.org/lettera-aperta-del-professor-sucharit-bhakdi-al-cancelliere-tedesco-dr-angela-merkel/}{IT},
\href{https://swprs.org/open-brief-van-professor-sucharit-bhakdi-aan-de-duitse-bondskanselier-dr-angela-merkel/}{NL},
\href{https://swprs.org/carta-aberta-do-professor-sucharit-bhakdi-a-chanceler-alema-dra-angela-merkel/}{PT},
\href{https://swprs.org/\%d0\%be\%d1\%82\%d0\%ba\%d1\%80\%d1\%8b\%d1\%82\%d0\%be\%d0\%b5-\%d0\%bf\%d0\%b8\%d1\%81\%d1\%8c\%d0\%bc\%d0\%be-\%d0\%bf\%d1\%80\%d0\%be\%d1\%84\%d0\%b5\%d1\%81\%d1\%81\%d0\%be\%d1\%80\%d0\%b0-\%d1\%81\%d1\%83\%d1\%87\%d0\%b0\%d1\%80\%d0\%b8\%d1\%82\%d0\%b0/}{RU},
\href{https://alatyr.sk/open-letter-from-professor_sk.htm}{SK},
\href{https://swprs.org/prof-dr-sucharit-bhakdiden-basbakan-dr-angela-merkele-acik-mektup/}{TR}\\
\textbf{Deel deze brief op}:
\href{https://twitter.com/intent/tweet?url=https://swprs.org/open-brief-van-professor-sucharit-bhakdi-aan-de-duitse-bondskanselier-dr-angela-merkel/}{Twitter}
/
\href{https://www.facebook.com/share.php?u=https://swprs.org/open-brief-van-professor-sucharit-bhakdi-aan-de-duitse-bondskanselier-dr-angela-merkel/}{Facebook}

Een open brief van Dr. Sucharit Bhakdi, emeritus hoogleraar medische
microbiologie aan de Johannes Gutenberg Universiteit Mainz, aan de
Duitse bondskanselier Dr. Angela Merkel. Professor Bhakdi roept op tot
een dringende herbeoordeling van het antwoord op Covid-19 en stelt de
kanselier vijf cruciale vragen. De brief dateert van 26 maart. Dit is
een niet-officiële vertaling; zie de originele brief in het Duits
\href{https://swprs.org/offener-brief-von-professor-sucharit-bhakdi-an-bundeskanzlerin-dr-angela-merkel/}{als
PDF}.

\hypertarget{open-brief}{%
\subsubsection{\texorpdfstring{\textbf{Open
brief}}{Open brief}}\label{open-brief}}

Geachte Bondskanselier,

Als emeritus van de Johannes-Gutenberg-Universiteit in Mainz en vele
jaren leidinggevende aan het aldaar gevestigde Instituut voor Medische
Microbiologie en Hygiëne, voel ik mij verplicht om kritische vragen te
stellen bij de vergaande beperkingen op het openbare leven die we
onszelf momenteel opleggen om de verspreiding van het COVID-19-virus te
beperken.

Het is uitdrukkelijk niet mijn bedoeling om de gevaren van de
virusziekte te bagatelliseren of om een politieke boodschap te
verspreiden. Ik vind echter dat het mijn plicht is een wetenschappelijke
bijdrage te leveren om de huidige gegevenssituatie correct in te delen,
om de feiten die we tot nu toe kennen in perspectief te plaatsen -- en
bovendien ook vragen te stellen die in de verhitte discussie verloren
dreigen te gaan.

De reden voor mijn bezorgdheid ligt vooral in de werkelijk niet te
voorziene sociaal-economische gevolgen van de drastische
beperkingsmaatregelen, die momenteel in grote delen van Europa worden
toegepast en die ook in Duitsland al op grote schaal worden doorgevoerd.

Het is mijn wens om kritisch -- en met de nodige vooruitziendheid -- de
voor- en nadelen van een beperking van het openbare leven en de daaruit
voortvloeiende langetermijneffecten bespreken.

Daartoe word ik geconfronteerd met vijf vragen, die nog niet afdoende
zijn beantwoord, maar die onontbeerlijk zijn voor een evenwichtige
analyse.

Ik vraag u hierbij om snel te reageren en tegelijkertijd doe ik een
beroep op de Bondsregering om strategieën te ontwikkelen die
risicogroepen effectief beschermen, zonder het openbare leven over de
hele linie te beperken en de kiem te leggen voor een nog scherpere
polarisatie van de samenleving dan nu al het geval is.

Hoogachtend,

\textbf{Prof. em. Dr. med. Sucharit Bhakdi}\\

\hypertarget{1-statistiek}{%
\subparagraph{\texorpdfstring{\textbf{1.
Statistiek}}{1. Statistiek}}\label{1-statistiek}}

In de infectiologie -- opgericht door Robert Koch zelf -- wordt een
traditioneel onderscheid gemaakt tussen infectie en ziekte. Een ziekte
vereist een klinische manifestatie. {[}1{]} Daarom moeten alleen
patiënten met symptomen zoals koorts of hoest als nieuwe gevallen in de
statistieken worden opgenomen.

Met andere woorden, een nieuwe infectie -- zoals gemeten door de
COVID-19 test -- betekent niet noodzakelijkerwijs dat we te maken hebben
met een pas ziek geworden patiënt die een ziekenhuisbed nodig heeft. Op
dit moment wordt echter aangenomen dat vijf procent van alle
geïnfecteerde mensen ernstig ziek wordt en beademing nodig heeft.
Prognoses op basis van deze schatting suggereren dat het zorgstelsel
overbelast zou kunnen raken.

\textbf{Mijn vraag}: Hebben de prognoses onderscheid gemaakt tussen
symptoomvrije geïnfecteerde patiënten en daadwerkelijke, zieke patiënten
-- d.w.z. mensen die symptomen ontwikkelen?~

\hypertarget{2-gevaarlijkheid}{%
\subparagraph{\texorpdfstring{\textbf{2.
Gevaarlijkheid}}{2. Gevaarlijkheid}}\label{2-gevaarlijkheid}}

Een aantal coronavirussen is al lange tijd in omloop -- grotendeels
ongemerkt door de media. {[}2{]} Als zou blijken dat het COVID-19 virus
geen significant hoger risicopotentieel heeft dan de reeds circulerende
coronavirussen, dan zouden alle tegenmaatregelen uiteraard overbodig
worden.

Het internationaal erkende ``International Journal of Antimicrobial
Agents'' zal binnenkort een document publiceren dat precies op deze
vraag ingaat. Voorlopige resultaten van de studie zijn al beschikbaar en
leiden tot de conclusie dat het nieuwe virus NIET als gevaarlijker kan
worden aangemerkt ten opzichte van traditionele coronavirussen. Dit
wordt door de auteurs uitgedrukt in de titel van hun werk ``SARS-CoV-2:
Fear versus Data''. {[}3{]}

\textbf{Mijn vraag}: Hoe verhoudt het huidige gebruik van de intensive
care units met patiënten die gediagnosticeerd zijn met COVID-19 zich tot
andere coronavirusinfecties, en in welke mate zullen deze gegevens
worden meegenomen in de toekomstige besluitvorming van de Bondsregering?
Bovendien: Is in de planning tot nu toe rekening gehouden met
bovenstaande studie? Ook hier moet natuurlijk gelden: Diagnose betekent
dat het virus ook een toonaangevende rol speelt in de ziektetoestand van
de patiënt, en niet eerdere ziekten een grotere rol spelen.

\hypertarget{3-verspreiding}{%
\subparagraph{\texorpdfstring{\textbf{3.
Verspreiding}}{3. Verspreiding}}\label{3-verspreiding}}

Volgens een rapport in de Süddeutsche Zeitung weet zelfs het veel
geciteerde Robert Koch Instituut niet precies hoeveel er op COVID-19
wordt getest. Het is echter een feit, dat men met het groeiend aantal
tests de laatste tijd in Duitsland een snelle stijging van het aantal
gevallen kon zien. {[}4{]}

Het vermoeden is dan ook, dat het virus zich al ongemerkt heeft
verspreid onder de gezonde bevolking. Dit zou twee gevolgen hebben: ten
eerste zou het betekenen dat het officiële sterftecijfer -- op 26 maart
2020 waren er bijvoorbeeld 206 sterfgevallen op ongeveer 37.300
infecties, ofwel 0,55 procent {[}5{]} -- te hoog is gesteld; en ten
tweede is het nauwelijks mogelijk om de verspreiding van de ziekte onder
de gezonde bevolking te voorkomen.

\textbf{Mijn vraag}: Zijn er al steekproefsgewijze testen uitgevoerd op
de gezonde algemene bevolking om de werkelijke verspreiding van het
virus te valideren of is dit gepland in de nabije toekomst?

\hypertarget{-4-sterfte}{%
\subparagraph{\texorpdfstring{\textbf{~4.
Sterfte}}{~4. Sterfte}}\label{-4-sterfte}}

De angst voor een stijging van het sterftecijfer in Duitsland (nu 0,55
procent) wordt momenteel door de media intensief tot aandachtspunt
gemaakt. Veel mensen maken zich zorgen dat het net als in Italië (10
procent) en Spanje (7 procent) omhoog zou kunnen schieten als er niet op
tijd actie wordt ondernomen. Tegelijkertijd wordt wereldwijd de fout
gemaakt om virusgerelateerde sterfgevallen te melden zodra wordt
vastgesteld dat het virus bij de dood aanwezig was -- ongeacht andere
factoren. Dit is in strijd met een basisprincipe van de infectiologie:~
Pas als is vastgesteld dat een agens een belangrijke rol heeft gespeeld
bij de ziekte of het overlijden, kan de diagnose worden gesteld.

De werkgemeenschap van Wetenschappelijke Medische Beroeps verenigingen
stelt in haar richtlijnen uitdrukkelijk: ``Naast de doodsoorzaak moet
een causale keten worden vermeld, met de overeenkomstige onderliggende
ziekte op de overlijdensakte op de derde plaats. Af en toe moeten er ook
vier oorzakelijke ketens worden vermeld.'' {[}6{]} Op dit moment is er
geen officieel antwoord op de vraag of er, in ieder geval achteraf, meer
kritische analyses van medische dossiers zijn uitgevoerd om vast te
stellen hoeveel sterfgevallen daadwerkelijk te wijten zijn aan het
virus.

\textbf{Mijn vraag:} Heeft Duitsland simpelweg de trend van de algemene
COVID-19 verdenking gevolgd? En: is zij van plan deze categorisering
onkritisch voort te zetten, zoals in andere landen? Hoe moet dan een
onderscheid worden gemaakt tussen echte corona-geïnduceerde
sterfgevallen en toevallige aanwezigheid van het virus op het moment van
overlijden?~

\hypertarget{5-om-te-vergelijken}{%
\subparagraph{\texorpdfstring{\textbf{5. Om te
vergelijken}}{5. Om te vergelijken}}\label{5-om-te-vergelijken}}

De afschuwelijke situatie in Italië wordt herhaaldelijk als
referentiescenario gebruikt. De werkelijke rol van het virus in dat land
is echter om vele redenen volstrekt onduidelijk -- niet alleen omdat de
punten 3 en 4 ook hier van toepassing zijn, maar ook omdat er
uitzonderlijke externe factoren bestaan die deze regio's bijzonder
kwetsbaar maken.

Het gaat onder meer om de verhoogde luchtvervuiling in Noord-Italië.
Volgens schattingen van de WHO leidde deze situatie ook zonder het
virus, in 2006 alleen al in de 13 grootste Italiaanse steden tot meer
dan 8.000 extra sterfgevallen per jaar. {[}7{]} De situatie is sindsdien
niet noemenswaardig veranderd. {[}8{]} Ten slotte is ook aangetoond dat
luchtvervuiling het risico op virale longziekten bij zeer jonge en
oudere mensen sterk verhoogt. {[}9{]}

Bovendien woont 27,4 procent van de bijzonder kwetsbare bevolking in dit
land met jongeren, in Spanje zelfs 33,5 procent. In Duitsland is dat
slechts zeven procent {[}10{]}.

Bovendien is Duitsland volgens Prof. Dr. Reinhard Busse, hoofd van de
afdeling Management in de gezondheidszorg aan de TU Berlijn, aanzienlijk
beter uitgerust dan Italië op het gebied van de intensive care -- met
een factor van ongeveer 2,5 {[}11{]}.

\textbf{Mijn vraag:} Welke inspanningen worden er gedaan om de bevolking
bewust te maken van deze elementaire verschillen en om de mensen te doen
inzien dat scenario's zoals die in Italië of Spanje hier niet
realistisch zijn?~

\hypertarget{referenties}{%
\subparagraph{\texorpdfstring{\textbf{Referenties:}}{Referenties:}}\label{referenties}}

{[}1{]} Fachwörterbuch Infektionsschutz und Infektionsepidemiologie.
\href{https://www.rki.de/DE/Content/Service/Publikationen/Fachwoerterbuch_Infektionsschutz.html}{Fachwörter
-- Definitionen -- Interpretationen}. Robert Koch-Institut, Berlin 2015.
(abgerufen am 26.3.2020)

{[}2{]} Killerby et al., Human Coronavirus Circulation in the United
States 2014--2017. J Clin Virol. 2018, 101, 52-56

{[}3{]} Roussel et al. SARS-CoV-2: Fear Versus Data. Int. J. Antimicrob.
Agents 2020, 105947

{[}4{]} Charisius, H.
\href{https://www.sueddeutsche.de/gesundheit/covid-19-coronavirus-testverfahren-1.4855487}{Covid-19:
Wie gut testet Deutschland?} Süddeutsche Zeitung. (abgerufen am
27.3.2020)

{[}5{]} Johns Hopkins University,
\href{https://coronavirus.jhu.edu/map.html}{Coronavirus Resource
Center}. 2020. (abgerufen am 26.3.2020)

{[}6{]} S1-Leitlinie 054-001,
\href{https://www.awmf.org/uploads/tx_szleitlinien/054-002l_S1_Regeln-zur-Durchfuehrung-der-aerztlichen-Leichenschau_2018-02_01.pdf}{Regeln
zur Durchführung der ärztlichen Leichenschau}. AWMF Online (abgerufen am
26.3.2020)

{[}7{]} Martuzzi et al. Health Impact of PM10 and Ozone in 13 Italian
Cities. World Health Organization Regional Office for Europe. WHOLIS
number E88700 2006

{[}8{]} European Environment Agency,
\href{https://www.eea.europa.eu/themes/air/country-fact-sheets/2019-country-fact-sheets}{Air
Pollution Country Fact Sheets 2019}, (abgerufen am 26.3.2020)

{[}9{]} Croft et al. The Association between Respiratory Infection and
Air Pollution in the Setting of Air Quality Policy and Economic Change.
Ann. Am. Thorac. Soc. 2019, 16, 321--330.

{[}10{]} United Nations, Department of Economic and Social Affairs,
Population Division. Living Arrange­ments of Older Persons: A Report on
an Expanded International Dataset (ST/ESA/SER.A/407). 2017

{[}11{]} Deutsches Ärzteblatt,
\href{https://www.aerzteblatt.de/nachrichten/111029/Ueberlastung-deutscher-Krankenhaeuser-durch-COVID-19-laut-Experten-unwahrscheinlich}{Überlastung
deutscher Krankenhäuser durch COVID-19 laut Experten unwahrscheinlich},
(abgerufen am 26.3.2020)

\begin{center}\rule{0.5\linewidth}{\linethickness}\end{center}

Back to main
article:~\href{https://swprs.org/a-swiss-doctor-on-covid-19/}{Facts
about Covid-19}

\hypertarget{swiss-policy-research}{%
\subsubsection{Swiss Policy Research}\label{swiss-policy-research}}

\begin{itemize}
\tightlist
\item
  \href{https://swprs.org/kontakt/}{Kontakt}
\item
  \href{https://swprs.org/uebersicht/}{Übersicht}
\item
  \href{https://swprs.org/donationen/}{Donationen}
\item
  \href{https://swprs.org/disclaimer/}{Disclaimer}
\end{itemize}

\hypertarget{english}{%
\subsubsection{English}\label{english}}

\begin{itemize}
\tightlist
\item
  \href{https://swprs.org/contact/}{About Us / Contact}
\item
  \href{https://swprs.org/media-navigator/}{The Media Navigator}
\item
  \href{https://swprs.org/the-american-empire-and-its-media/}{The CFR
  and the Media}
\item
  \href{https://swprs.org/donations/}{Donations}
\end{itemize}

\hypertarget{follow-by-email}{%
\subsubsection{Follow by email}\label{follow-by-email}}

Follow

\href{https://wordpress.com/?ref=footer_custom_com}{WordPress.com}.

\protect\hyperlink{}{Up ↑}

Post to

\protect\hyperlink{}{Cancel}

\includegraphics{https://pixel.wp.com/b.gif?v=noscript}
