\protect\hyperlink{content}{Skip to content}

\href{https://swprs.org/}{}

\protect\hyperlink{search-container}{Search}

Search for:

\href{https://swprs.org/}{\includegraphics{https://swprs.files.wordpress.com/2020/05/swiss-policy-research-logo-300.png}}

\href{https://swprs.org/}{Swiss Policy Research}

Geopolitics and Media

Menu

\begin{itemize}
\tightlist
\item
  \href{https://swprs.org}{Start}
\item
  \href{https://swprs.org/srf-propaganda-analyse/}{Studien}

  \begin{itemize}
  \tightlist
  \item
    \href{https://swprs.org/srf-propaganda-analyse/}{SRF / ZDF}
  \item
    \href{https://swprs.org/die-nzz-studie/}{NZZ-Studie}
  \item
    \href{https://swprs.org/der-propaganda-multiplikator/}{Agenturen}
  \item
    \href{https://swprs.org/die-propaganda-matrix/}{Medienmatrix}
  \end{itemize}
\item
  \href{https://swprs.org/medien-navigator/}{Analysen}

  \begin{itemize}
  \tightlist
  \item
    \href{https://swprs.org/medien-navigator/}{Navigator}
  \item
    \href{https://swprs.org/der-propaganda-schluessel/}{Techniken}
  \item
    \href{https://swprs.org/propaganda-in-der-wikipedia/}{Wikipedia}
  \item
    \href{https://swprs.org/logik-imperialer-kriege/}{Kriege}
  \end{itemize}
\item
  \href{https://swprs.org/netzwerk-medien-schweiz/}{Netzwerke}

  \begin{itemize}
  \tightlist
  \item
    \href{https://swprs.org/netzwerk-medien-schweiz/}{Schweiz}
  \item
    \href{https://swprs.org/netzwerk-medien-deutschland/}{Deutschland}
  \item
    \href{https://swprs.org/medien-in-oesterreich/}{Österreich}
  \item
    \href{https://swprs.org/das-american-empire-und-seine-medien/}{USA}
  \end{itemize}
\item
  \href{https://swprs.org/bericht-eines-journalisten/}{Fokus I}

  \begin{itemize}
  \tightlist
  \item
    \href{https://swprs.org/bericht-eines-journalisten/}{Journalistenbericht}
  \item
    \href{https://swprs.org/russische-propaganda/}{Russische Propaganda}
  \item
    \href{https://swprs.org/die-israel-lobby-fakten-und-mythen/}{Die
    »Israel-Lobby«}
  \item
    \href{https://swprs.org/geopolitik-und-paedokriminalitaet/}{Pädokriminalität}
  \end{itemize}
\item
  \href{https://swprs.org/migration-und-medien/}{Fokus II}

  \begin{itemize}
  \tightlist
  \item
    \href{https://swprs.org/covid-19-hinweis-ii/}{Coronavirus}
  \item
    \href{https://swprs.org/die-integrity-initiative/}{Integrity
    Initiative}
  \item
    \href{https://swprs.org/migration-und-medien/}{Migration \& Medien}
  \item
    \href{https://swprs.org/der-fall-magnitsky/}{Magnitsky Act}
  \end{itemize}
\item
  \href{https://swprs.org/kontakt/}{Projekt}

  \begin{itemize}
  \tightlist
  \item
    \href{https://swprs.org/kontakt/}{Kontakt}
  \item
    \href{https://swprs.org/uebersicht/}{Seitenübersicht}
  \item
    \href{https://swprs.org/medienspiegel/}{Medienspiegel}
  \item
    \href{https://swprs.org/donationen/}{Donationen}
  \end{itemize}
\item
  \href{https://swprs.org/contact/}{English}
\end{itemize}

\protect\hyperlink{}{Open Search}

\hypertarget{covid19-letalituxe4t-wie-man-es-nicht-macht}{%
\section{COVID19-Letalität: Wie man es
nicht~macht}\label{covid19-letalituxe4t-wie-man-es-nicht-macht}}

\includegraphics{https://swprs.files.wordpress.com/2020/06/sweden-imperial-comparison-june-5.png?w=736}

\textbf{Publiziert}: 23. Juni, 2020; \textbf{Sprachen}:
\href{https://swprs.org/covid-19-letalitat-wie-man-es-nicht-macht/}{DE},
\href{https://swprs.org/covid19-lethality-how-not-to-do-it/}{EN}\\
\textbf{Teilen auf}:
\href{https://twitter.com/intent/tweet?url=https://swprs.org/covid-19-letalitat-wie-man-es-nicht-macht/}{Twitter}
/
\href{https://www.facebook.com/share.php?u=https://swprs.org/covid-19-letalitat-wie-man-es-nicht-macht/}{Facebook}

Ein deutscher Medizinprofessor argumentiert, die Sterblichkeitsrate von
Covid-19 (IFR) sei
\href{https://www.heise.de/tp/features/Wie-gefaehrlich-ist-Covid-19-im-Vergleich-zur-saisonalen-Grippe-4790313.html}{„fünf-
bis zehnmal höher``} als die IFR der saisonalen Influenza. Doch er
begeht zwei klassische Fehler.

Zunächst vergleicht er die IFR der saisonalen Influenza (bis zu 0,1\%)
mit frühen \textbf{Modell­vor­hersagen} der Covid19-IFR. Diese frühen
Modellvorhersagen erwiesen sich jedoch als
\href{https://swprs.org/studies-on-covid-19-lethality/}{völlig
unrealistisch} (siehe Grafik oben). Zuletzt reduzierte sogar die
US-Gesund­heits­behörde CDC ihren (immer noch vorsichtigen)
Covid-IFR-Wert
\href{https://www.cdc.gov/coronavirus/2019-ncov/hcp/planning-scenarios.html}{auf
nur noch 0,26\%} (``best estimate'').

Der Professor schreibt sodann, dass Antikörperstudien in Brasilien und
Spanien eine IFR von 1\% oder mehr ergaben. Doch das stimmt nicht: Der
Professor verwechselt die sogenannte \textbf{„rohe IFR``} (Todesfälle
geteilt durch Infektionen in einer Studiengruppe) mit der
\textbf{bevölkerungs­basierten IFR}, die an das Alters- und das
Risikoprofil einer Bevölkerung angepasst ist.

Diese Unterscheidung ist zentral, da Covid-19 hauptsächlich ältere
Hochrisikogruppen (tödlich) betrifft: Tatsächlich ereigneten sich 40 bis
80\% der „Covid-bedingten`` Todesfälle
\href{https://swprs.org/studies-on-covid-19-lethality/\#care-homes}{in
Pflegeheimen}. Wenn jedoch zehn von einhundert Pflegepatienten sterben,
bedeutet das nicht, dass 10\% der gesamten Bevölkerung sterben werden.

Beispielsweise hatte das berühmte Kreuzfahrtschiff \textbf{Diamond
Princess} mit seinen meist älteren Passagieren eine ``rohe IFR'' von
etwa 1,5\%. Basierend auf diesem Wert berechnete Stanford-Professor John
Ioannidis bereits im März eine bevölkerungs­basierte Covid-IFR
\href{https://www.statnews.com/2020/03/17/a-fiasco-in-the-making-as-the-coronavirus-pandemic-takes-hold-we-are-making-decisions-without-reliable-data/}{von
etwa 0,13\%} für die gesamte US-Gesellschaft.

Der deutsche Professor erwähnt zwar die Streeck-Antikörperstudie zum
Hotspot in Gangelt, die eine bevölkerungsbasierte IFR von 0,36\% ergab.
Er erwähnt jedoch nicht, dass dies
\href{https://swprs.org/studies-on-covid-19-lethality/}{eine Obergrenze
war}: Der bereinigte IFR betrug 0,27\% und das mittlere Todesalter
betrug 81 Jahre.

Die meisten Antikörperstudien zeigen eine bevölkerungsbasierte IFR
\href{https://swprs.org/studies-on-covid-19-lethality/}{zwischen 0,1\%
und 0,3\%}, was mit einer schweren Influenza vergleichbar ist. Für
Menschen unter 50 Jahren liegt die Covid-IFR sogar \emph{eher tiefer}
als bei der Influenza. Einige Hotspots zeigten lokal höhere IFRs von bis
zu 0,7\%, aber diese Orte waren zumeist von einem Zusammenbruch des
Pflegesystems betroffen.

In jüngerer Zeit haben immunologische Untersuchungen zudem gezeigt, dass
serologische (d.h. Blut-) Antikörperstudien
\href{https://swprs.org/coronavirus-antibody-tests-show-only-one-fifth-of-infections/}{höchstens
20\% der Infektionen nachweisen}, da die meisten Menschen das
Coronavirus mit ihrem mukosalen (d.h. Schleimhaut-) oder zellulären
Immunsystem neutralisieren, ohne überhaupt (permanente) Antikörper im
Blut entwickeln zu müssen.

Das bedeutet, dass der reale Covid19-IFR \textbf{deutlich unter 0,1\%}
und damit in den Bereich der saisonalen Influenza fallen kann. Das
bedeutet auch, dass ``Covid19-Immunitätspässe'' und obligatorische
Impfstoffe nicht funktionieren können. Und es erklärt, warum selbst
Hotspots wie New York City und Stockholm Antikörper bei nicht mehr als
20\% der Bevölkerung fanden.

Trotz der relativ geringen Letalität (IFR) kann die \textbf{Mortalität}
(Gesamtzahl der Todesfälle) lokal und vorübergehend stark erhöht sein,
wenn sich das Coronavirus sehr rasch ausbreitet und dabei
Hoch­risiko­gruppen in Pflegeheimen und Krankenhäusern infiziert, wie
das in vielen Hotspots wie z.B. Norditalien und Ostfrankreich
tatsächlich der Fall war.

Dennoch blieb die \textbf{kumulierte Gesamtmortalität} seit Jahresbeginn
selbst in stark betroffenen Ländern wie den USA und UK oder in Schweden
(ohne Lockdown)
\href{https://swprs.org/studies-on-covid-19-lethality/\#overall-mortality}{im
Bereich einer starken Influenza-Saison}. Länder wie Deutschland und die
Schweiz erlebten eine milde „Influenza-Saison``.

Die folgende Grafik zeigt, dass die globale Covid-19-Mortalität mit der
berüchtigten
\href{https://www.forbes.com/2010/02/05/world-health-organization-swine-flu-pandemic-opinions-contributors-michael-fumento.html}{„falschen
Pandemie``} der Schweinegrippe von 2009/10 vergleichbar ist und eine
ganze Größen­ordnung unter den Grippepandemien von 1957 (asiatische
Grippe) und 1968 (Hongkong-Grippe) liegt. Diese beiden waren ernst, aber
das gesellschaftliche Leben konnte dennoch
\href{https://nypost.com/2020/05/16/why-life-went-on-as-normal-during-the-killer-pandemic-of-1969/}{normal
weitergehen}.

Dabei sind diese Sterblichkeitsraten zwar an das Bevölkerungswachstum
angepasst, aber noch nicht an die \textbf{Alterung der Bevölkerung:}
Denn die Altersgruppe, die heute am stärksten von Covid-19 betroffen ist
(80+), gab es in den 1950er Jahren noch kaum. Eine Anpassung an die
Alterung würde Covid-19 daher im Vergleich sogar \emph{noch weniger}
dramatisch erscheinen lassen.

Im Gegensatz dazu warnt die UNO, dass aufgrund der \textbf{politischen
Reaktion} auf Covid-19 die Hälfte der Arbeiter der Welt oder rund
\textbf{1,6 Milliarden Menschen} nun unmittelbar davon bedroht sind,
\href{https://www.theguardian.com/world/2020/apr/29/half-of-worlds-workers-at-immediate-risk-of-losing-livelihood-due-to-coronavirus}{ihren
Lebensunterhalt zu verlieren}. Allein in den USA wurden bereits
\href{https://www.businessinsider.com/us-weekly-jobless-claims-unemployment-insurance-filings-economy-coronavirus-recession-2020-6}{46
Millionen Menschen} arbeitslos.

\textbf{Siehe auch}:
\href{https://swprs.org/covid-19-hinweis-ii/}{Fakten zu Covid-19}

\includegraphics{https://swprs.files.wordpress.com/2020/06/covid-19-comparison-e1592927192181.png?w=736\&h=600}

\begin{center}\rule{0.5\linewidth}{\linethickness}\end{center}

\textbf{Teilen auf}:
\href{https://twitter.com/intent/tweet?url=https://swprs.org/covid-19-letalitat-wie-man-es-nicht-macht/}{Twitter}
/
\href{https://www.facebook.com/share.php?u=https://swprs.org/covid-19-letalitat-wie-man-es-nicht-macht/}{Facebook}

\hypertarget{swiss-policy-research}{%
\subsubsection{Swiss Policy Research}\label{swiss-policy-research}}

\begin{itemize}
\tightlist
\item
  \href{https://swprs.org/kontakt/}{Kontakt}
\item
  \href{https://swprs.org/uebersicht/}{Übersicht}
\item
  \href{https://swprs.org/donationen/}{Donationen}
\item
  \href{https://swprs.org/disclaimer/}{Disclaimer}
\end{itemize}

\hypertarget{english}{%
\subsubsection{English}\label{english}}

\begin{itemize}
\tightlist
\item
  \href{https://swprs.org/contact/}{About Us / Contact}
\item
  \href{https://swprs.org/media-navigator/}{The Media Navigator}
\item
  \href{https://swprs.org/the-american-empire-and-its-media/}{The CFR
  and the Media}
\item
  \href{https://swprs.org/donations/}{Donations}
\end{itemize}

\hypertarget{follow-by-email}{%
\subsubsection{Follow by email}\label{follow-by-email}}

Follow

\href{https://wordpress.com/?ref=footer_custom_com}{WordPress.com}.

\protect\hyperlink{}{Up ↑}

Post to

\protect\hyperlink{}{Cancel}

\includegraphics{https://pixel.wp.com/b.gif?v=noscript}
