\protect\hyperlink{content}{Skip to content}

\href{https://swprs.org/}{}

\protect\hyperlink{search-container}{Search}

Search for:

\href{https://swprs.org/}{\includegraphics{https://swprs.files.wordpress.com/2020/05/swiss-policy-research-logo-300.png}}

\href{https://swprs.org/}{Swiss Policy Research}

Geopolitics and Media

Menu

\begin{itemize}
\tightlist
\item
  \href{https://swprs.org}{Start}
\item
  \href{https://swprs.org/srf-propaganda-analyse/}{Studien}

  \begin{itemize}
  \tightlist
  \item
    \href{https://swprs.org/srf-propaganda-analyse/}{SRF / ZDF}
  \item
    \href{https://swprs.org/die-nzz-studie/}{NZZ-Studie}
  \item
    \href{https://swprs.org/der-propaganda-multiplikator/}{Agenturen}
  \item
    \href{https://swprs.org/die-propaganda-matrix/}{Medienmatrix}
  \end{itemize}
\item
  \href{https://swprs.org/medien-navigator/}{Analysen}

  \begin{itemize}
  \tightlist
  \item
    \href{https://swprs.org/medien-navigator/}{Navigator}
  \item
    \href{https://swprs.org/der-propaganda-schluessel/}{Techniken}
  \item
    \href{https://swprs.org/propaganda-in-der-wikipedia/}{Wikipedia}
  \item
    \href{https://swprs.org/logik-imperialer-kriege/}{Kriege}
  \end{itemize}
\item
  \href{https://swprs.org/netzwerk-medien-schweiz/}{Netzwerke}

  \begin{itemize}
  \tightlist
  \item
    \href{https://swprs.org/netzwerk-medien-schweiz/}{Schweiz}
  \item
    \href{https://swprs.org/netzwerk-medien-deutschland/}{Deutschland}
  \item
    \href{https://swprs.org/medien-in-oesterreich/}{Österreich}
  \item
    \href{https://swprs.org/das-american-empire-und-seine-medien/}{USA}
  \end{itemize}
\item
  \href{https://swprs.org/bericht-eines-journalisten/}{Fokus I}

  \begin{itemize}
  \tightlist
  \item
    \href{https://swprs.org/bericht-eines-journalisten/}{Journalistenbericht}
  \item
    \href{https://swprs.org/russische-propaganda/}{Russische Propaganda}
  \item
    \href{https://swprs.org/die-israel-lobby-fakten-und-mythen/}{Die
    »Israel-Lobby«}
  \item
    \href{https://swprs.org/geopolitik-und-paedokriminalitaet/}{Pädokriminalität}
  \end{itemize}
\item
  \href{https://swprs.org/migration-und-medien/}{Fokus II}

  \begin{itemize}
  \tightlist
  \item
    \href{https://swprs.org/covid-19-hinweis-ii/}{Coronavirus}
  \item
    \href{https://swprs.org/die-integrity-initiative/}{Integrity
    Initiative}
  \item
    \href{https://swprs.org/migration-und-medien/}{Migration \& Medien}
  \item
    \href{https://swprs.org/der-fall-magnitsky/}{Magnitsky Act}
  \end{itemize}
\item
  \href{https://swprs.org/kontakt/}{Projekt}

  \begin{itemize}
  \tightlist
  \item
    \href{https://swprs.org/kontakt/}{Kontakt}
  \item
    \href{https://swprs.org/uebersicht/}{Seitenübersicht}
  \item
    \href{https://swprs.org/medienspiegel/}{Medienspiegel}
  \item
    \href{https://swprs.org/donationen/}{Donationen}
  \end{itemize}
\item
  \href{https://swprs.org/contact/}{English}
\end{itemize}

\protect\hyperlink{}{Open Search}

\hypertarget{le-multiplicateur-de-propagande}{%
\section{Le multiplicateur
de~propagande}\label{le-multiplicateur-de-propagande}}

\textbf{Langues} :
\href{https://swprs.org/der-propaganda-multiplikator/}{DE},
\href{https://swprs.org/the-propaganda-multiplier/}{EN},
\href{https://www.bibliotecapleyades.net/sociopolitica2/sociopol_mediacontrol225.htm}{ES},
\href{https://www.bibliotecapleyades.net/sociopolitica2/sociopol_mediacontrol228.htm}{IT},
\href{https://swprs.files.wordpress.com/2019/12/propaganda-multiplier-dutch.pdf}{NL},
\href{https://midtifleisen.wordpress.com/2018/01/04/en-titt-pa-nyhetsbyraenes-rolle/}{NO},
\href{https://wolnemedia.net/powielacze-propagandy/}{PL},
\href{https://revistaopera.com.br/2019/04/23/a-propagacao-hegemonica-como-as-agencias-globais-e-a-midia-ocidental-cobrem-a-geopolitica-parte-1/}{PT},
\href{https://csa.pnzgu.ru/infopswars/ipw1}{RU}

C'est l'un des aspects les plus importants de notre système médiatique,
et il est pourtant très peu connu : la plupart des informations
internationales publiées dans les médias occidentaux ne sont fournies
que par trois agences de presse mondiales basées à New York, Londres et
Paris.

Le rôle clé joué par ces agences signifie que les médias occidentaux
traitent souvent des mêmes sujets, et les décrivent de la même manière.
De plus, les gouvernements, les officines de l'armée ou du renseignement
utilisent ces agences de presse mondiales comme des multiplicateurs pour
diffuser leurs messages dans le monde entier.

Une étude de la couverture de la guerre de Syrie par 9 journaux
européens de référence illustre clairement ce problème : 78 \% des
articles étaient basés sur des rapports d'agence, et 0 \% sur du travail
d'investigation. Qui plus est, 82 \% des tribunes et des interviewes
étaient en faveur d'une intervention des États-Unis et de l'OTAN, tandis
que la désinformation était attribuée exclusivement à la partie adverse.

Partagez cette étude sur :
\href{https://twitter.com/intent/tweet?url=https://swprs.org/le-multiplicateur-de-propagande/}{Twitter}
/
\href{https://www.facebook.com/share.php?u=https://swprs.org/le-multiplicateur-de-propagande/}{Facebook}

\includegraphics{https://swprs.files.wordpress.com/2019/02/propaganda-multiplier.png?w=512\&h=588}

\begin{center}\rule{0.5\linewidth}{\linethickness}\end{center}

\hypertarget{le-multiplicateur-de-propagande-}{%
\subsection{Le multiplicateur de propagande
:}\label{le-multiplicateur-de-propagande-}}

\hypertarget{comment-les-agences-de-presse-mondiales-et-les}{%
\subsection{Comment les agences de presse mondiales et
les}\label{comment-les-agences-de-presse-mondiales-et-les}}

médias occidentaux parlent de géopolitique

Une étude de \href{https://swprs.org/contact/}{Swiss Propaganda
Research}*\\
*

Traduite par Vincent Lenormant

2016 / 2019

« Il faut toujours se poser la question : pourquoi est-ce que je
reçois\\
cette information précise, de cette façon précise, à ce moment précis
?\\
Au final, ce sont toujours des questions de pouvoir. »
(\href{http://www.nzz.ch/wer-lustvoll-schreibt-der-schreibt-auch-gut-1.11329756}{*})\\
Konrad Hummler, banquier et patron de presse suisse

Contenu

\begin{enumerate}
\def\labelenumi{\arabic{enumi}.}
\tightlist
\item
  \protect\hyperlink{k1}{Le multiplicateur de propagande}
\item
  \protect\hyperlink{k2}{Étude de cas: la guerre de Syrie}
\item
  \protect\hyperlink{k3}{Notes et références}
\end{enumerate}

\hypertarget{introduction---quelque-chose-duxe9trange-}{%
\paragraph{Introduction : « quelque chose d'étrange
»}\label{introduction---quelque-chose-duxe9trange-}}

« Comment les journaux savent-ils ce qu'ils savent ? » La réponse à
cette question risque de surprendre leurs lecteurs : « la principale
source, ce sont les agences de presse. D'une certaine manière, ces
agences, qui opèrent de façon quasi-anonyme, sont la clé des événements
mondiaux. Alors quels sont leurs noms, comment marchent-elles et qui les
finance ? Pour juger de la qualité de l'information à l'Est comme à
l'Ouest, il faut connaître les réponses à ces questions. »~ (Höhne 1977,
p. 11)

Comme le remarque un chercheur sur les médias suisses, « les agences de
presse sont le principal fournisseur des médias de masse. Aucun organe
de presse quotidienne ne peut s'en sortir sans eux. Les agences
influencent donc notre image du monde ; ce que nous savons, c'est ce
qu'ils ont sélectionné. » (Blum 1995, p. 9)

Vu leur importance, il est étonnant que ces agences soient à peine
connues du public : « la plupart des gens ignorent leur
existence\ldots{} alors qu'elles jouent un rôle prépondérant dans le
marché de l'information. Mais malgré ça, on ne leur a jamais prêté trop
d'attention. » (Schulten-Jaspers 2013, p. 13)

Même le patron d'une agence de presse l'a reconnu : « Il y a quelque
chose d'étrange avec les agences de presse. Elles sont peu connues du
public. Contrairement aux journaux, leur activité n'est pas sous le feu
des projecteurs, et pourtant on les trouve à la source de chaque
article. » (Segbers 2007, p. 9)

\hypertarget{-le-centre-nuxe9vralgique-du-systuxe8me-muxe9diatique-}{%
\paragraph{« Le centre névralgique du système médiatique
»}\label{-le-centre-nuxe9vralgique-du-systuxe8me-muxe9diatique-}}

Quels sont donc les noms de ces agences qui sont «à la source de chaque
article»? Il n'y en a que trois:

\begin{enumerate}
\def\labelenumi{\arabic{enumi}.}
\tightlist
\item
  \textbf{Associated Press} (États-Unis), 4000 employés à travers le
  monde. L'AP
  \href{https://en.wikipedia.org/wiki/Associated_Press}{appartient} à
  des médias étasuniens et son siège est à New York. AP News est utilisé
  par 12000 organes de presse internationaux, ce qui lui permet
  d'atteindre plus de la moitié de la population mondiale tous les
  jours.
\item
  \textbf{Agence France Presse} (France), quasi gouvernementale, basée à
  Paris, 4000 employés. L'AFP
  \href{https://en.wikipedia.org/wiki/Agence_France-Presse}{envoie} plus
  de 3000 dépêches et photos tous les jours aux médias du monde entier.
\item
  \textbf{Reuters} (Grande-Bretagne), société privée, 3000 employés.
  Reuters a été racheté en 2008 par le patron de presse canadien
  Thomson, une des 25 personnes les plus riches du monde, pour
  \href{https://en.wikipedia.org/wiki/Reuters}{devenir} Thomson Reuters,
  dont le siège est à New York.
\end{enumerate}

Plusieurs pays disposent en outre de leur propre agence de presse, comme
la DPA allemande, l'APA autrichienne, et la SDA suisse. Mais pour les
informations internationales, les agences nationales s'en tiennent en
général à ces trois agences mondiales et se contentent de copier et
traduire leurs articles.

\includegraphics{https://swprs.files.wordpress.com/2017/01/logos_agenturen.png?w=600\&h=259}

Les trois agences de presse mondiales Reuters, AFP et AP, et les trois
agences nationales des pays germanophones d'Autriche (APA), d'Allemagne
(DPA) et de Suisse (SDA).

Wolfgang Vyslozil, ancien directeur de l'APA, a décrit leur rôle en ces
termes : « Les agences de presse sont rarement visibles par le public.
Elles sont le type de média le plus influent et en même temps le moins
connu. Ce sont des institutions-clés d'une importance capitale pour tous
les organes de presse. Ce sont les centres névralgiques qui connectent
toutes les parties du système. » (Segbers 2007, p.10)

\hypertarget{petite-abruxe9viation-gros-effet}{%
\paragraph{Petite abréviation, gros
effet}\label{petite-abruxe9viation-gros-effet}}

Cependant, il y a une explication simple au fait que ces agences, malgré
leur importance, soient si peu connues du grand public. Pour citer un
professeur de journalisme suisse : « La radio et la télévision ne citent
pas leurs sources en général, et seuls les spécialistes savent
déchiffrer les références dans les magazines. » (Blum 1995, P. 9)

Les motifs de cette discrétion sont malgré tout assez clairs : les
journaux ne sont pas particulièrement enclins à laisser leurs lecteurs
se rendre compte qu'ils n'ont fourni aucun travail d'investigation pour
la plupart de leurs articles.

Voici des exemples de la façon dont les sources sont mentionnées dans
les grands médias européens. À côté des abréviations des agences on
trouve les initiales des journalistes qui ont édité la dépêche.

\includegraphics{https://swprs.files.wordpress.com/2019/02/agenturen-quellen.png?w=650}

Les agences de presse comme sources d'articles de journaux.

Il arrive que les journaux utilisent des articles sans citer leur
source. Une étude de l'Institut de Recherche Suisse pour la Sphère
Publique et la Société, menée à l'université de Zurich en 2011, est
arrivée à la conclusion suivante (FOEG 2011):

« Les articles d'agence sont exploités intégralement sans citer leur
origine, ou bien ils sont partiellement réécrits pour avoir l'air de
contributions éditoriales. De plus, ils sont souvent assaisonnés sans
grand effort : par exemple, on y ajoute un graphique ou une image et
c'est présenté comme un article de fond. »

Les agences jouent un rôle proéminent, pas seulement dans la presse mais
aussi dans l'audiovisuel public et privé. C'est ce que
\href{http://www.heise.de/tp/artikel/47/47821/3.html}{confirme} Volker
Braeutigam, qui a travaillé pour la chaîne publique ARD pendant dix ans
et voit d'un œil critique la domination de ces agences :

« Un des problèmes fondamentaux, c'est que la rédaction d'ARD ne base
ses informations que sur trois sources : DPA/AP, Reuters et l'AFP ;
l'une est germano-étasunienne, l'autre est britannique et la troisième
française. Le journaliste qui travaille sur un sujet n'a qu'à
sélectionner les passages qu'il juge essentiels, les réarranger et les
coller ensemble avec quelques fioritures. »

La Radio Télévision Suisse également se base sur ces agences. À des
spectateurs qui leurs demandaient pourquoi ils n'avaient pas parlé d'une
marche pour la paix en Ukraine, les journalistes ont
\href{http://www.srf.ch/sendungen/hallosrf/warum-berichtet-srf-nicht-ueber-den-friedensmarsch-in-der-ukraine}{répondu}
: « A ce jour, nous n'avons reçu des agences indépendantes Reuters, AP
et AFP aucune information et aucun matériel vidéo concernant cette
marche.»

En fait, non seulement le texte, mais aussi les images et les vidéos que
nous voyons dans les médias tous les jours, viennent principalement de
ces agences. Ce que le public non initié voit comme le travail des
journalistes de son pays, n'est que la copie de dépêches provenant de
New York, Londres et Paris.

Certains médias vont même plus loin, et, par manque de ressources,
sous-traitent toutes les affaires internationales à une agence. Il est
par ailleurs bien connu que de nombreux sites web ne publient que des
dépêches d'agence (voir Paterson 2007, Johnston 2011, MacGregor 2013).

Au final, cette dépendance vis-a-vis des agences mondiales crée une
similarité frappante dans le traitement de l'actualité internationale :
de Vienne à Washington, les médias traitent les mêmes sujets, avec les
mêmes mots -- un phénomène qu'on aurait tendance à associer avec les «
médias sous contrôle » des états totalitaires.

Voici des exemples tirés de publications allemandes et internationales.
On peut voir que malgré l'objectivité qu'elles revendiquent, un léger
biais géopolitique apparaît parfois.

\includegraphics{https://swprs.files.wordpress.com/2019/02/agency-headlines.png?w=650}

« Poutine menace », « l'Iran provoque », « l'OTAN s'inquiète », « le
bastion du régime » : Similitudes de contenu et de formulation dues aux
rapports des agences de presse mondiales.*\\
*

\hypertarget{le-ruxf4le-des-correspondants}{%
\paragraph{Le rôle des
correspondants}\label{le-ruxf4le-des-correspondants}}

La plupart de nos médias n'ont pas de correspondant à l'étranger, ils
n'ont donc pas d'autre choix que de se fier entièrement aux agences
mondiales. Mais qu'en est-il des grands quotidiens et des télévisions
qui disposent de correspondants ? Dans les pays germanophones, il s'agit
de journaux comme NZZ, FAZ, le Süddeutsche Zeitung, Welt et les
diffuseurs publics.

Tout d'abord, il faut garder à l'esprit l'échelle de grandeur : si les
agences mondiales disposent de plusieurs milliers d'employés à travers
le monde, même le journal suisse NZZ, connu pour ses reportages
internationaux, ne maintient que 35 correspondants à l'étranger, y
compris ses représentants commerciaux. Dans des pays aussi vastes que la
Chine ou l'Inde, il n'y a qu'un seul correspondant ; l'ensemble de
l'Amérique du Sud est couvert par deux journalistes seulement, tandis
qu'en Afrique il n'y a pas un seul permanent.

Ajoutons à ça que dans les zones de guerre, les correspondants se
risquent rarement à aller sur le terrain. Pour la guerre de Syrie par
exemple, de nombreux journalistes ont fait des reportages depuis
Istanbul, Beyrouth, le Caire ou même Chypre, et la plupart ne parlaient
pas arabe.

Alors comment les correspondants trouvent-ils leurs informations ?
Encore une fois, grâce aux agences. Le correspondant hollandais au Moyen
Orient Joris Luyendijk a fait un description impressionnante du travail
des correspondants et de leur dépendance aux agences dans
\href{https://www.amazon.com/People-Like-Us-Misrepresenting-Middle/dp/1593762569}{son
livre} « Des gens comme nous : la mauvaise représentation du
Moyen-Orient »:

« J'imaginais les correspondants comme des historiens du présent. Quand
un événement important se produisait, ils y allaient, cherchaient à
comprendre ce qui se passait, et faisaient leur reportage. Mais moi je
n'allais pas chercher à comprendre ce qui se passait ; ça avait été fait
depuis longtemps. Je suivais le mouvement pour faire un reportage sur
place.

La rédaction appelait quand quelque chose se passait, ils m'envoyaient
les dépêches d'agence, et je les reformulais pour la radio ou pour la
presse. Pour ma hiérarchie, il était plus important de savoir qu'ils
pouvaient me joindre sur le terrain que de savoir que je comprenais ce
qui se passait. Les agences fournissaient assez d'informations pour
pouvoir écrire ou parler de n'importe quelle crise ou sommet mondial.

C'est pour ça qu'on voit souvent les mêmes images et les mêmes histoires
dans tous les médias. Dans les bureaux de Londres, Paris, Berlin ou
Washington, tout le monde pensait que les mauvais sujets faisaient les
gros titres et que nous suivions trop servilement les agences.

On s'imagine que les correspondants font l'information, mais en réalité
la presse fonctionne comme une chaîne de montage dans une boulangerie
industrielle. Les correspondants sont au bout de la chaîne, et ils font
comme s'ils avaient cuit le pain eux-mêmes alors qu'ils l'ont juste
emballé.

Un ami m'a demandé comment je réussissais à répondre à toutes les
questions du présentateur sans hésitation. Quand je lui ai répondu qu'on
connaissait toutes les questions à l'avance, il était outré. Il venait
de réaliser que tout ce qu'il avait vu et entendu depuis des dizaines
d'années n'était que du théâtre. » (Luyendjik 2009, p. 20-22, 76, 189)

Autrement dit, les correspondants n'ont en général pas les moyens de
mener des recherches indépendantes, et doivent juste répéter et
renforcer les sujets prescrits par les agences de presse : c'est l' «
effet mainstream ».

Ajoutons à ça que pour des raisons budgétaires, de nombreux organes de
presse partagent leurs correspondants : au sein d'un groupe de médias,
les mêmes reportages internationaux sont utilisés par différents titres,
ce qui ne contribue aucunement à la diversité des points de vue.

\hypertarget{-ce-dont-lagence-ne-parle-pas-na-pas-lieu-}{%
\paragraph{« Ce dont l'agence ne parle pas n'a pas lieu
»}\label{-ce-dont-lagence-ne-parle-pas-na-pas-lieu-}}

Le rôle central des agences de presse explique également pourquoi, dans
les conflits internationaux, les médias utilisent les mêmes sources.
Dans la guerre de Syrie par exemple, l' « Observatoire Syrien des Droits
de l'Homme », une
\href{https://en.wikipedia.org/wiki/Syrian_Observatory_for_Human_Rights}{organisation}
douteuse constituée d'un seul homme basé à Londres, est une source
récurrente. Les médias ne se sont pas beaucoup intéressé à cet «
Observatoire », son directeur étant difficile à joindre, même pour les
journalistes.

C'est pourtant cet « Observatoire » qui livrait ses articles aux agences
mondiales, qui les diffusaient ensuite à des milliers de rédactions, qui
à leur tour en « informaient » des millions de lecteurs et de
spectateurs dans le monde entier. Mais pas grand monde n'a cherché à
savoir pourquoi les agences se référaient à cet étrange « Observatoire
», ni qui le finançait.

L'ancien rédacteur en chef de l'agence de presse allemande DPA, Manfred
Steffens, écrit dans son livre « Le Business de l'Info » :

« Une information ne devient pas correcte juste parce qu'on peut la
sourcer. En fait il est plutôt étrange de croire une info parce qu'une
source est citée. Derrière la protection qu'offre une « source », des
gens peuvent répandre des choses très aventureuses, tout en ayant
eux-mêmes des doutes sur leur véracité ; la responsabilité, au moins
morale, pourra toujours être attribuée à la source. » (Steffens 1969, p.
106)

La dépendance vis à vis des agences mondiales est également la
principale raison de la superficialité et de l'incohérence de la
couverture des conflits géopolitiques, d'où sont absents l'historique
des relations et les arrière-plans. Comme le dit Seffens : « Les agences
de presse se basent sur les événements présents et sont par nature
anhistoriques. Ils sont réticents à contextualiser les événements
au-delà du strict nécessaire. » (Steffens 1969, p. 32)

Pour finir, la domination des agences mondiales explique pourquoi
certains problèmes géopolitiques ou certains événements qui ne cadrent
pas avec le récit officiel des USA/OTAN ne figurent pas du tout dans les
médias : si les agences n'en parlent pas, la plupart des médias
occidentaux n'en auront pas conscience. « Ce dont l'agence ne parle pas
n'a pas lieu » (Wilke 2000, p. 1)

\hypertarget{-ajouter-des-informations-douteuses-}{%
\paragraph{« Ajouter des informations douteuses
»}\label{-ajouter-des-informations-douteuses-}}

Si certains sujets n'apparaissent pas du tout dans les médias, d'autres
y sont prééminents, même s'ils ne devraient pas l'être : « Souvent, les
grands médias ne s'intéressent pas à la réalité, mais à une réalité
construite ou mise en scène. Plusieurs études ont montré que les grands
médias sont avant tout déterminés par les activités de relations
publiques, et que les attitudes passives et réceptives surpassaient le
travail d'investigation. » (Blum 1995, p. 16)

En pratique, à cause des performances journalistiques plutôt faibles des
médias et de leur dépendance aux agences, il est facile pour qui en a
les moyens de diffuser à une audience mondiale et dans un format
respectable, propagande et désinformation. Steffens, journaliste à la
DPA, a averti de ce danger :

« Plus l'agence de presse ou le journal est respecté, plus le sens
critique s'émousse. Si quelqu'un veut diffuser une info douteuse dans la
presse mondiale, il lui suffit de la faire diffuser par une agence de
réputation correcte pour être sûr de la voir reprise peu après par les
autres. Parfois une fausse info passe d'agence en agence et devient
d'autant plus crédible. » (Steffens 1969, p. 234)

Parmi ceux qui « injectent » des informations douteuses on trouve les
armées et les ministères de la défense. Par exemple, en 2009 le
directeur de l'agence étasunienne AP, Tom Curley,
\href{https://harpers.org/blog/2009/02/pentagon-targeted-and-mistreated-journalists-ap-head-charges/}{a
déclaré} que que le Pentagone employait plus de 27000 spécialistes de la
communication pour travailler les médias et faire circuler des
manipulations ciblées, avec un budget de presque 5 milliards par an. Il
a ajouté que des généraux de haut rang l'avaient menacé de le ruiner si
jamais les journalistes étaient trop critiques vis-à-vis de l'armée.

Malgré, ou à cause de telles menaces, les médias publient régulièrement
des articles douteux dont les sources sont des « informateurs »
anonymes, dans les « milieux de la défense US ».

Ulrich Tilgner, correspondant au Moyen-Orient historique des télévisions
allemandes et suisses, a averti en 2003, peu après la guerre d'Iraq, de
manipulations par l'armée et du rôle joué par les médias :

« Avec l'aide des médias, l'armée contrôle ce que le public perçoit, et
ils s'en servent pour leurs opérations. Ils parviennent à faire bouger
les prévisions, à propager des scenarios fictifs. Dans ce nouveau genre
de guerre, les stratèges en communication de l'administration
étasunienne ont le même rôle que des pilotes de bombardier. Les services
de communication du Pentagone et des services secrets sont devenus des
soldats de la guerre de l'information.

Pour mener à bien ses tromperies, l'armée US utilise précisément le
manque de transparence des médias. La façon dont elle propage les
informations, qui sont ensuite distribuées par les journaux et les
diffuseurs, fait qu'il est impossible pour le lecteur ou le spectateur
d'en connaître l'origine. Ainsi, le public ne peut reconnaître la vraie
intention de l'armée. » (Tilgner 2003, p. 132)

Et ce qui est vrai pour l'armée l'est aussi pour les services de
renseignement. Au cours d'un
\href{https://swprs.org/video-the-cia-and-the-media/}{reportage
remarquable} de Channel 4, on peut voir des anciens représentants de la
CIA et un correspondant de Reuters parler candidement de la
dissémination systématique de la propagande et de la désinformation dans
la couverture des conflits géopolitiques :

L'ancien agent de la CIA et lanceur d'alerte John Stockwell a décrit son
travail lors de la guerre en Angola en ces termes : « Le thème de base
c'était de faire en sorte que ça ait l'air d'une agression ennemie.
Alors on a écrit toutes sortes d'articles qui allaient dans ce sens. Mon
unité comptait un tiers de propagandistes, dont le métier était
d'inventer des histoires et de trouver le moyen de les diffuser dans la
presse. (\ldots{}) Les journalistes de la plupart des médias occidentaux
ne sont pas sceptiques tant que ça va dans le sens des préjugés et des
généralités. (\ldots{}) Alors on écrivait un autre article, et ça durait
comme ça des semaines. (\ldots{}) Mais tout était faux. »

Fred Bridgland considère ainsi son passé de correspondant pour Reuters :
« On basait nos dépêches sur les communications officielles. Il a fallu
des années avant que j'apprenne qu'un expert en désinformation de la CIA
était à l'ambassade et écrivait ces communiqués qui n'avaient aucun lien
avec la réalité. (\ldots{}) Pour le dire de façon très crue, vous pouvez
publier n'importe quelle connerie et ça sera repris par les journaux. »

Quant à l'ancien analyste de la CIA David MacMichael , voici comment il
parle de son travail pendant la guerre des Contras au Nicaragua : « Ils
disaient qu' au Nicaragua nos services de renseignements étaient si
performants qu'on était au courant dès que quelqu'un allait aux
toilettes. Mais pour moi les articles qu'on donnait à la presse venaient
tout droit des toilettes. »
(\href{https://swprs.org/video-the-cia-and-the-media/}{Hird 1985})

Bien sûr, les services de renseignement disposent aussi d'un grand
nombre de
\href{http://www.carlbernstein.com/magazine_cia_and_media.php}{contacts
directs} dans les médias, à qui ils peuvent divulguer des informations
si nécessaire. Mais sans le rôle central que jouent les agences de
presse mondiales, la synchronisation mondiale de la propagande et de la
désinformation ne pourrait être si efficace.

À travers ce « multiplicateur de propagande », les infos douteuses
venant d'experts en communication travaillants pour les gouvernements ou
les services de renseignements peuvent atteindre le grand public sans
vérification ni filtre. Les journalistes se réfèrent aux agences et les
agences se réfèrent à leurs sources. Même si elles laissent la place au
doute (et se protègent) avec des termes comme « allégation », «
apparemment » et autres, la rumeur se répand très vite au monde entier
et l'effet se produit.

\includegraphics{https://swprs.files.wordpress.com/2019/02/propaganda-multiplier.png?w=550}

Le multiplicateur de propagande : Les gouvernements, les militaires et
les services de renseignements utilisent les agences de presse mondiales
pour diffuser leurs messages à un public global.*\\
*

\hypertarget{comme-le-rapporte-le-new-york-times}{%
\paragraph{Comme le rapporte le New York
Times\ldots{}}\label{comme-le-rapporte-le-new-york-times}}

Outre les agences de presse mondiales, d'autres sources sont utilisées
par les organes de presse du monde entier pour traiter les conflits
géopolitiques : les grands médias anglais et étasuniens.

Des enseignes comme le New York Times ou la BBC ont plus de 100
correspondants à l'étranger, et d'autres employés externes. Cependant,
comme le remarque M. Luyendijk, correspondant au Moyen-Orient :

« Notre rédaction, moi inclus, s'abreuvait d'articles sélectionnés par
des médias de qualité comme CNN, la BBC et le New York Times. On faisait
ça parce qu'on supposait que leurs correspondants connaissaient le monde
arabe et avaient autorité pour en parler ; mais au final beaucoup
d'entre eux ne parlaient pas arabe, ou pas assez pour avoir une
conversation ou suivre les informations locales. Les cadors de CNN, de
la BBC, de l'Independent, du Guardian, du New Yorker, du NYT devaient
bien souvent compter sur des assistants et des traducteurs. » (Luyendijk
p. 47)

De plus, les sources de ces médias sont souvent invérifiables : «
milieux de l'armée », « fonctionnaires anonymes du gouvernement », «
fonctionnaires du renseignement » ou autres. On peut donc s'en servir
pour disséminer la propagande. Dans tous les cas, l'orientation générale
dans le sens des grandes publications anglo-saxonnes mène à la
convergence des opinions sur les sujets liés à la géopolitique.

Voici quelques exemples de ces références aux grand médias anglo-saxons,
relevés dans le plus grand quotidien de Suisse, Tages-Anzeiger, au sujet
de la Syrie. Ces articles datent tous d'octobre 2015, lorsque la Russie
est intervenue directement pour la première fois (les sources US/UK sont
en bleu) :

\includegraphics{https://swprs.files.wordpress.com/2017/01/us-medien.png?w=620\&h=616}

Références fréquentes aux grands médias britanniques et étasuniens au
sujet de la guerre de Syrie par le quotidien suisse Tages-Anzeiger en
octobre 2015.*\\
*

\hypertarget{le-ruxe9cit-duxe9siruxe9}{%
\paragraph{Le récit désiré}\label{le-ruxe9cit-duxe9siruxe9}}

Mais pourquoi est-ce que dans nos médias, les journalistes n'essayent
pas tout simplement de faire des recherches et de travailler
indépendamment des agences mondiales et des médias anglo-saxons ? M.
Luyendijk, correspondant au Moyen-Orient, décrit ainsi son expérience :

« Vous allez peut-être me dire que j'aurais pu chercher d'autres sources
à qui me fier. J'ai essayé, mais à chaque fois que j'ai voulu écrire un
article sans me baser sur les agences de presse, les grands médias
anglo-saxons ou les discours dominants, ça n'a pas marché. (\ldots{})
Évidemment, en tant que correspondant, j'aurais pu raconter des
histoires très différentes à propos d'une même situation. Mais les
médias ne pouvaient en présenter qu'une, et souvent, c'était exactement
l'histoire qui confirmait l'opinion générale. » (Luyendijk p.54ff)

Le chercheur Noam Chomsky a décrit cet effet dans
\href{https://chomsky.info/199710__/}{son essai} « What makes the
mainstream media mainstream » : « si vous vous éloignez de la ligne
officielle, si vous produisez des reportages dissidents, voilà ce qui va
se passer. (\ldots{}) Il y a différentes manières de vous faire rentrer
dans le rang. Si vous ne suivez pas les règles, vous ne garderez pas
votre travail longtemps. Ce système marche très bien, et il reflète les
structures de pouvoir. » (Chomsky 1997)

Cependant, certains grands journalistes continuent de croire que
personne ne peut leur dire ce qu'ils doivent écrire. Comment est-ce
possible ? Chomsky \href{https://chomsky.info/199710__/}{clarifie} cette
apparente contradiction :

« Le truc c'est qu'ils ne seraient pas là s'ils n'avaient pas déjà
démontré que personne n'a besoin de leur dire ce qu'ils doivent écrire,
puisqu'ils vont dire ce qu'il faut. S'ils avaient commencé à Métro et
qu'ils s'étaient intéressés au mauvais genre d'articles, ils ne se
seraient jamais retrouvés dans la position où ils peuvent dire ce qu'ils
veulent. On peut en dire autant des carrières universitaires dans les
disciplines plus idéologiques. Ils sont passés à travers le système de
socialisation. »

En fin de compte, ce « système de socialisation » conduit à un
journalisme qui ne fait plus de recherches indépendantes et de
reportages critiques sur les conflits géopolitiques (et d'autres
sujets), mais qui cherche à consolider le récit souhaité par des
éditoriaux et des commentaires appropriés.

\hypertarget{conclusion--la--premiuxe8re-loi-du-journalisme-}{%
\paragraph{Conclusion : la « première loi du journalisme
»}\label{conclusion--la--premiuxe8re-loi-du-journalisme-}}

L'ancien journaliste de Associated Press Herbert Altschull l'a appelée
la première loi du journalisme : « dans tous les systèmes de presse, les
médias sont les instruments de ceux qui exercent le pouvoir politique et
économique. Les journaux, les périodiques, la radio et la télévision
n'agissent pas de manière indépendante, alors qu'ils en ont la
possibilité. » (Altschull 1984/1995, p. 298)

Dans ce sens, il est logique que les médias traditionnels, qui sont
financés en majeure partie par la publicité ou l'état, représentent les
intérêts géopolitiques de l'alliance transatlantique, étant donné que
les annonceurs comme les états dépendent de l'architecture économique et
sécuritaire transatlantique que dirigent les États-Unis.

De plus, les personnages-clés des médias dominants font souvent partie
des réseaux des élites transatlantiques, dans l'esprit du « système de
socialisation » de Chomsky. Les institutions les plus importantes à cet
égard incluent le Council on Foreign Relations, le groupe Bilderberg, et
la Commission Trilatérale, où l'on trouve beaucoup de grands
journalistes (voir une
\href{https://swprs.org/the-american-empire-and-its-media/}{étude en
profondeur} de ces groupes).

La plupart des grands médias peuvent donc être vus comme des « médias de
l'establishment ». Ça s'explique par le fait que dans le passé, la
liberté de la presse était assez théorique, car tout le monde ne pouvait
pas obtenir une autorisation de diffuser, une fréquence d'émission, une
infrastructure financière et technique ou autre.

La première loi d'Altschull n'a été brisée que par Internet, dans une
certaine mesure. Ainsi, au cours des dernières années, un journalisme de
qualité, financé par les lecteurs,
\href{https://swprs.org/media-navigator/}{a vu le jour}, et il dépasse
souvent les médias traditionnels en terme d'analyse critique. Quelques
unes de ces publications « alternatives » ont déjà un large public, ce
qui montre qu'un grand lectorat ne doit pas forcément rimer avec une
moindre qualité.

Néanmoins, jusqu'à maintenant les médias traditionnels réussissent à
attirer une grosse majorité de visiteurs sur leurs sites également. Et
c'est encore dû aux agences de presse, dont les dépêches instantanées
forment l'épine dorsale de la plupart des sites d'information.

Le « pouvoir politique et économique », comme le dit la loi d'Altschull,
va-t'il garder le contrôle de l'information, ou bien les « informations
incontrôlées » vont-elles changer la structure politique et économique ?
Les années qui viennent vont nous le dire.

\href{https://swprs.org/contact/}{Lire d'autres articles de SPR →}\\
Partagez cette étude sur :
\href{https://twitter.com/intent/tweet?url=https://swprs.org/le-multiplicateur-de-propagande/}{Twitter}
/
\href{https://www.facebook.com/share.php?u=https://swprs.org/le-multiplicateur-de-propagande/}{Facebook}

\begin{center}\rule{0.5\linewidth}{\linethickness}\end{center}

\hypertarget{uxe9tude-de-cas--le-traitement-de-la-guerre-de-syrie}{%
\subsubsection{Étude de cas : le traitement de la guerre de
Syrie}\label{uxe9tude-de-cas--le-traitement-de-la-guerre-de-syrie}}

Nous avons étudié la couverture de la guerre de Syrie par neuf grands
quotidiens allemands, autrichiens et suisses afin d'examiner la
pluralité des points de vue et le degré de dépendance aux agences de
presse. Les journaux sélectionnés sont :

\begin{itemize}
\tightlist
\item
  \textbf{pour l'Allemagne} : Die Welt, Süddeutsche Zeitung (SZ), et
  Frankfurter Allgemeine Zeitung (FAZ)
\item
  \textbf{pour la Suisse} : Neue Zürcher Zeitung (NZZ), Tagesanzeiger
  (TA), et Basler Zeitung (BaZ)
\item
  \textbf{pour l'Autriche} : Standard, Kurier, et Die Presse
\end{itemize}

La période d'étude va du 1 er au 15 octobre 2015, les deux premières
semaines de l'intervention russe en Syrie. Toute les publications
imprimées ou mises en ligne ont été prises en compte. Les éditions du
dimanche n'ont pas été prises en compte car tous les journaux examinés
n'en ont pas publié. Au total, 381 articles ont rempli ces critères.

Dans un premier temps, les articles ont été classés en fonction de leurs
propriétés dans les groupes suivants :

\begin{enumerate}
\def\labelenumi{\arabic{enumi}.}
\tightlist
\item
  \textbf{Agence} (agencies) : article d'agence (avec les codes de
  l'agence)
\item
  \textbf{Mélange} (mixed) : article simple basé en partie ou en
  totalité sur des articles d'agence
\item
  \textbf{Article de fond} (reports) : reportage ou analyse éditoriale
  détaillé
\item
  \textbf{Tribune / commentaire} (op-eds) : tribunes et commentaires
  d'invités
\item
  \textbf{Interview} : interview avec des experts, hommes politiques
  etc.
\item
  \textbf{Investigation} : travail d'investigation qui révèle de
  nouvelles informations
\end{enumerate}

La \textbf{figure 1} ci-dessous montre la composition des articles pour
les neuf journaux étudiés. On peut voir que 55 \% des articles venaient
des agences et qu'aucun article n'était le résultat d'une investigation.

\includegraphics{https://swprs.files.wordpress.com/2019/02/figure1.png?w=650}

Figure 1 : type d'articles (total; n = 381)*\\
*

Les articles d'agence dans leur forme originale, qu'il s'agisse de
simples dépêches ou de reportages détaillés, se trouvaient
principalement sur les sites web des journaux : la pression de
l'actualité y est plus forte que pour les éditions papier, et il n'y a
pas de limite de taille. Les autres types de publication étaient
présents dans les éditions en ligne et papier ; et seules quelques
interviews exclusives et quelques reportages n'étaient présents que dans
les éditions papier. Chaque article n'a été collecté qu'une fois pour
l'étude.

La \textbf{figure 2} ci-dessous montre la même classification journal
par journal. Pendant la période d'étude (2 semaines), la plupart des
journaux ont publié entre 40 et 50 articles au sujet du conflit syrien
(en ligne et papier). Dans le journal allemand \emph{Die Welt} il y en a
eu plus (58), et dans le \emph{Basler Zeitung} et le \emph{Kurier}
autrichien, moins (29 et 33).

Selon le journal, la part des articles d'agence est de presque 50 \%
(\emph{Welt}, \emph{Süddeutsche}, \emph{NZZ}, \emph{Basler Zeitung}),
presque 60 \% (\emph{FAZ}, \emph{Tagesanzeiger}), entre 60 \% et 70 \%
(\emph{Presse}, \emph{Standard}, \emph{Kurier}). Si on y ajoute les
articles basés sur des articles d'agence, la proportion dans chaque
journal varie entre 70 \% et 80 \%. Ces proportions sont cohérentes avec
les précédentes études sur les médias (Blum 1995, Johnston 2011,
MacGregor 2013, Paterson 2007).

Pour les articles de fond, les journaux suisses étaient en tête (cinq ou
six articles), suivis par le \emph{Welt}, \emph{Süddeutsche} et
\emph{Standard} (quatre chacun), puis les autres (de un à trois). Les
articles détaillés et les analyses étaient particulièrement consacrées
aux événements en cours au Moyen-Orient, ainsi qu'aux motifs et intérêts
des acteurs comme la Russie, la Turquie ou l'État Islamique.

La plupart des tribunes se trouvaient dans les journaux allemands (sept
tribunes chacun), suivis par \emph{Standard} (cinq), \emph{NZZ} et
\emph{Tagesanzeiger} (quatre chacun). \emph{Basler Zeitung} n'a pas
publié de tribune pendant cette période, mais deux interviewes. D'autres
interviewes ont été publiées par \emph{Standard} (trois), \emph{Kurier}
et \emph{Presse} (une chacun). Par contre, nous n'avons trouvé de
travail d'investigation dans aucun de ces journaux.

Dans le cas des trois journaux allemands, nous avons noté un mélange
d'information et d'opinion assez problématique du point de vue
journalistique : les articles contenaient de fortes expressions
d'opinion sans être présentés comme des tribunes. L'étude présente était
cependant basée sur la façon dont le journal présentait ses articles.

\includegraphics{https://swprs.files.wordpress.com/2019/03/figure2c.png?w=650}

Figure 2 : Types d'articles par journal*\\
*

La \textbf{figure 3} ci-dessous montre la répartition des articles
d'agence (par abréviation des noms d'agence) pour chaque agence de
presse, au total et par pays. Les 211 articles d'agences étaient
accompagnés de 277 codes d'agence (un article de presse peut provenir de
plusieurs articles d'agence). Au total, 24 \% des articles d'agence
venaient de l'AFP ; 20 \% chacun pour la DPA, l'APA et Reuters, 9 \%
pour la SDA, 6 \% pour l'AP, et 11 \% étaient inconnus (pas de
référence, ou seulement la mention « agence »).

En Allemagne, la DPA, l'AFP et Reuters ont chacune un tiers des
actualités. En Suisse, la SDA et l'AFP sont en tête, et en Autriche, ce
sont l'APA et Reuters.

En réalité, les parts des agences mondiales AFP, AP et Reuters sont
certainement plus grandes, puisque la SDA suisse comme l'APA
autrichienne obtiennent la plupart de leurs reportages internationaux
des agences mondiales, tandis que la DPA allemande coopère étroitement
avec l'AP.

Il faut aussi noter que pour des raisons historiques, les agences
mondiales sont présentes de manière disparate dans chaque région du
globe. Pour des événements en Asie, en Ukraine ou en Afrique, la part de
chaque agence ne sera pas la même que dans le cas du Moyen-Orient.

\includegraphics{https://swprs.files.wordpress.com/2019/02/figure3.png?w=650}

Figure 3 : Part des agences de presse, au total (n=277) et par pays

Au cours de l'étape suivante, des prises de positions ont permis
d'évaluer l'orientation des tribunes (28), des commentaires d'invités
(10) et des interviewes (7) sur un total de 45 articles. Comme le montre
la \textbf{figure 4}, 82 \% des contributions étaient favorables aux
USA/OTAN, 16 \% étaient neutres ou équilibrés, et 2 \% étaient critiques
vis-a-vis des USA/OTAN (un seul article).

Cet article était une tribune libre dans le journal autrichien
\emph{Standard} du 2 octobre, intitulé « La stratégie du changement de
régime a échoué. La distinction entre les ``bons'' et les ``mauvais''
groupes terroristes en Syrie rend la politique occidentale peu digne de
confiance. »

\includegraphics{https://swprs.files.wordpress.com/2019/02/figure4.png?w=650}

Figure 4 : Orientation des opinions rédactionnelles, des commentaires
des invités et des personnes interrogées (total ; n=45)

La \textbf{figure 5} ci-dessous montre l'orientation des contributions,
des commentaires et des interviewes, répartis par journal. On peut voir
que \emph{Welt}, \emph{Süddeutsche}, \emph{NZZ}, \emph{Tagesanzeiger} et
le journal autrichien \emph{Kurier} présentent exclusivement des
opinions favorables aux USA/OTAN. C'est aussi vrai pour \emph{FAZ} à
l'exception d'une contribution neutre ou équilibrée. Le \emph{Standard}
a présenté quatre opinions favorables aux USA/OTAN, trois neutres ou
équilibrées, ainsi que la tribune critique sus-mentionnée.

Presse a été le seul journal à ne présenter que des opinions neutres ou
équilibrées. Le \emph{Basler Zeitung} a publié une opinion favorable et
une équilibrée. Peu après la période d'étude, le 16 octobre, le
\emph{Basler Zeitung} a publié une interview du président du parlement
russe. Cela aurait bien sûr été compté comme une contribution critique.

\includegraphics{https://swprs.files.wordpress.com/2019/02/figure5.png?w=650}

Figure 5 : Orientation générale des articles d'opinion et des personnes
interrogées par journal

En allant plus loin dans l'analyse, nous avons recherché le mot «
propagande » et ses dérivés pour déterminer si elle était attribuée
plutôt aux USA/OTAN ou plutôt à la Russie (sans considérer l'État
Islamique). 20 cas ont été identifiés. La \textbf{figure 6} montre les
résultats : dans 85 \% des cas, la propagande a été attribuée à la
Russie, dans 15 \% des cas l'identification était neutre et dans 0 \%
des cas elle a été attribuée aux USA/OTAN.

Notons que la moitié des cas (neuf) se trouvaient dans le \emph{NZZ}
suisse, qui parlait fréquemment de propagande russe (« la propagande du
Kremlin », « la machine de propagande de Moscou », « des articles de
propagande », « l'appareil de propagande russe »), suivi par le
\emph{FAZ} allemand (trois), \emph{Welt} et \emph{Süddeutsche Zeitung}
(deux chacun) et le journal autrichien \emph{Kurier} (un). Les autres
journaux ne mentionnaient pas ce terme, ou alors dans un contexte neutre
(ou en parlant de l'État Islamique).

\includegraphics{https://swprs.files.wordpress.com/2019/02/figure6.png?w=650}

Figure 6 : Attribution de la propagande aux parties au conflit (total :
20)

\hypertarget{conclusion}{%
\paragraph{Conclusion}\label{conclusion}}

Au cours de cette étude de cas, nous avons examiné la diversité et les
performances de neuf grands journaux européens dans leur couverture
géopolitique de la guerre de Syrie.

Les résultats confirment la haute dépendance aux agences de presse (de
63 à 90 \%, sans compter les commentaires et les interviewes) et le
manque d'investigation, ainsi que le parti pris dans les opinions et les
commentaires en faveur des USA/OTAN (82 \% positif, 2 \% négatif), dont
les récits n'ont pas été suspectés de propagande par les journaux.

\emph{À propos des auteurs : Swiss Propaganda Research (SPR) est un
groupe de recherche indépendant} \emph{qui enquête sur le propagande
géopolitique en Suisse et dans les médias internationaux. Vous}
\emph{pouvez nous contacter \href{https://swprs.org/contact/}{ici}.}

\emph{Traduction française par Vincent Lenormant.}

\begin{center}\rule{0.5\linewidth}{\linethickness}\end{center}

\hypertarget{ruxe9fuxe9rences-}{%
\subsubsection{Références :}\label{ruxe9fuxe9rences-}}

Altschull, Herbert J. (1984/1995): Agents of power. The media and public
policy. \emph{Longman,} New York.

Becker, Jörg (2015): Medien im Krieg -- Krieg in den Medien.
\emph{Springer Verlag für Sozialwissenschaften,} Wiesbaden.

Blum, Roger et al. (Hrsg.) (1995): Die AktualiTäter.
Nachrichtenagenturen in der Schweiz. \emph{Verlag Paul Haupt,} Bern.

Chomsky, Noam (1997): What Makes Mainstream Media Mainstream. \emph{Z
Magazine,} MA. (\href{https://chomsky.info/199710__/}{PDF})

Forschungsinstitut für Öffentlichkeit und Gesellschaft der Universität
Zürich (FOEG) (2011): Jahrbuch Qualität der Medien, Ausgabe 2011.
\emph{Schwabe,} Basel.

Gritsch, Kurt (2010): Inszenierung eines gerechten Krieges?
Intellektuelle, Medien und der ``Kosovo-Krieg'' 1999. \emph{Georg Olms
Verlag,} Hildesheim.

Hird, Christopher (1985): Standard Techniques. \emph{Diverse Reports,
Channel 4 TV.} 30. Oktober 1985.
(\href{https://swprs.org/video-the-cia-and-the-media/}{Link})

Höhne, Hansjoachim (1977): Report über Nachrichtenagenturen. Band 1: Die
Situation auf den Nachrichtenmärkten der Welt. Band 2: Die Geschichte
der Nachricht und ihrer Verbreiter. \emph{Nomos Verlagsgesellschaft,}
Baden-Baden.

Johnston, Jane \& Forde, Susan (2011): The Silent Partner: News Agencies
and 21st Century News. \emph{International Journal of Communication 5
(2011),} p. 195--214.
(\href{https://ijoc.org/index.php/ijoc/article/view/928}{PDF})

Krüger, Uwe (2013): Meinungsmacht. Der Einfluss von Eliten auf
Leitmedien und Alpha-Journalisten -- eine kritische Netzwerkanalyse.
\emph{Herbert von Halem Verlag,} Köln.

Luyendijk, Joris (2015): Von Bildern und Lügen in Zeiten des Krieges:
Aus dem Leben eines Kriegsberichterstatters -- Aktualisierte Neuausgabe.
\emph{Tropen,} Stuttgart.

MacGregor, Phil (2013): International News Agencies. Global eyes that
never blink. In: Fowler-Watt/Allan (ed.): Journalism: New Challenges.
\emph{Centre for Journalism \& Communication Research,} Bournemouth
University.
(\href{https://microsites.bournemouth.ac.uk/cjcr/files/2013/10/JNC-2013-Chapter-3-MacGregor.pdf}{PDF})

Mükke, Lutz (2014): Korrespondenten im Kalten Krieg. Zwischen Propaganda
und Selbstbehauptung. \emph{Herbert von Halem Verlag,} Köln.

Paterson, Chris (2007): International news on the internet. \emph{The
International Journal of Communication Ethics.} Vol 4, No 1/2 2007.
(\href{http://www.communicationethics.net/journal/v4n1-2/v4n1-2_12.pdf}{PDF})

Queval, Jean (1945): Première page, Cinquième colonne. \emph{Arthème
Fayard,} Paris.

Schulten-Jaspers, Yasmin (2013): Zukunft der Nachrichtenagenturen.
Situation, Entwicklung, Prognosen. \emph{Nomos,} Baden-Baden.

Segbers, Michael (2007): Die Ware Nachricht. Wie Nachrichtenagenturen
ticken. \emph{UVK,} Konstanz.

Steffens, Manfred {[}Ziegler, Stefan{]} (1969): Das Geschäft mit der
Nachricht. Agenturen, Redaktionen, Journalisten. \emph{Hoffmann und
Campe}, Hamburg.

Tilgner, Ulrich (2003): Der inszenierte Krieg -- Täuschung und Wahrheit
beim Sturz Saddam Husseins. \emph{Rowohlt}, Reinbek.

Wilke, Jürgen (Hrsg.) (2000): Von der Agentur zur Redaktion.
\emph{Böhlau}, Köln.

\begin{center}\rule{0.5\linewidth}{\linethickness}\end{center}

Partagez cette étude sur :
\href{https://twitter.com/intent/tweet?url=https://swprs.org/le-multiplicateur-de-propagande/}{Twitter}
/
\href{https://www.facebook.com/share.php?u=https://swprs.org/le-multiplicateur-de-propagande/}{Facebook}\\
Publiée (FR) : Décembre 2019

\hypertarget{swiss-policy-research}{%
\subsubsection{Swiss Policy Research}\label{swiss-policy-research}}

\begin{itemize}
\tightlist
\item
  \href{https://swprs.org/kontakt/}{Kontakt}
\item
  \href{https://swprs.org/uebersicht/}{Übersicht}
\item
  \href{https://swprs.org/donationen/}{Donationen}
\item
  \href{https://swprs.org/disclaimer/}{Disclaimer}
\end{itemize}

\hypertarget{english}{%
\subsubsection{English}\label{english}}

\begin{itemize}
\tightlist
\item
  \href{https://swprs.org/contact/}{About Us / Contact}
\item
  \href{https://swprs.org/media-navigator/}{The Media Navigator}
\item
  \href{https://swprs.org/the-american-empire-and-its-media/}{The CFR
  and the Media}
\item
  \href{https://swprs.org/donations/}{Donations}
\end{itemize}

\hypertarget{follow-by-email}{%
\subsubsection{Follow by email}\label{follow-by-email}}

Follow

\href{https://wordpress.com/?ref=footer_custom_com}{WordPress.com}.

\protect\hyperlink{}{Up ↑}

Post to

\protect\hyperlink{}{Cancel}

\includegraphics{https://pixel.wp.com/b.gif?v=noscript}
