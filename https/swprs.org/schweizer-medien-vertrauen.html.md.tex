\protect\hyperlink{content}{Skip to content}

\href{https://swprs.org/}{}

\protect\hyperlink{search-container}{Search}

Search for:

\href{https://swprs.org/}{\includegraphics{https://swprs.files.wordpress.com/2020/05/swiss-policy-research-logo-300.png}}

\href{https://swprs.org/}{Swiss Policy Research}

Geopolitics and Media

Menu

\begin{itemize}
\tightlist
\item
  \href{https://swprs.org}{Start}
\item
  \href{https://swprs.org/srf-propaganda-analyse/}{Studien}

  \begin{itemize}
  \tightlist
  \item
    \href{https://swprs.org/srf-propaganda-analyse/}{SRF / ZDF}
  \item
    \href{https://swprs.org/die-nzz-studie/}{NZZ-Studie}
  \item
    \href{https://swprs.org/der-propaganda-multiplikator/}{Agenturen}
  \item
    \href{https://swprs.org/die-propaganda-matrix/}{Medienmatrix}
  \end{itemize}
\item
  \href{https://swprs.org/medien-navigator/}{Analysen}

  \begin{itemize}
  \tightlist
  \item
    \href{https://swprs.org/medien-navigator/}{Navigator}
  \item
    \href{https://swprs.org/der-propaganda-schluessel/}{Techniken}
  \item
    \href{https://swprs.org/propaganda-in-der-wikipedia/}{Wikipedia}
  \item
    \href{https://swprs.org/logik-imperialer-kriege/}{Kriege}
  \end{itemize}
\item
  \href{https://swprs.org/netzwerk-medien-schweiz/}{Netzwerke}

  \begin{itemize}
  \tightlist
  \item
    \href{https://swprs.org/netzwerk-medien-schweiz/}{Schweiz}
  \item
    \href{https://swprs.org/netzwerk-medien-deutschland/}{Deutschland}
  \item
    \href{https://swprs.org/medien-in-oesterreich/}{Österreich}
  \item
    \href{https://swprs.org/das-american-empire-und-seine-medien/}{USA}
  \end{itemize}
\item
  \href{https://swprs.org/bericht-eines-journalisten/}{Fokus I}

  \begin{itemize}
  \tightlist
  \item
    \href{https://swprs.org/bericht-eines-journalisten/}{Journalistenbericht}
  \item
    \href{https://swprs.org/russische-propaganda/}{Russische Propaganda}
  \item
    \href{https://swprs.org/die-israel-lobby-fakten-und-mythen/}{Die
    »Israel-Lobby«}
  \item
    \href{https://swprs.org/geopolitik-und-paedokriminalitaet/}{Pädokriminalität}
  \end{itemize}
\item
  \href{https://swprs.org/migration-und-medien/}{Fokus II}

  \begin{itemize}
  \tightlist
  \item
    \href{https://swprs.org/covid-19-hinweis-ii/}{Coronavirus}
  \item
    \href{https://swprs.org/die-integrity-initiative/}{Integrity
    Initiative}
  \item
    \href{https://swprs.org/migration-und-medien/}{Migration \& Medien}
  \item
    \href{https://swprs.org/der-fall-magnitsky/}{Magnitsky Act}
  \end{itemize}
\item
  \href{https://swprs.org/kontakt/}{Projekt}

  \begin{itemize}
  \tightlist
  \item
    \href{https://swprs.org/kontakt/}{Kontakt}
  \item
    \href{https://swprs.org/uebersicht/}{Seitenübersicht}
  \item
    \href{https://swprs.org/medienspiegel/}{Medienspiegel}
  \item
    \href{https://swprs.org/donationen/}{Donationen}
  \end{itemize}
\item
  \href{https://swprs.org/contact/}{English}
\end{itemize}

\protect\hyperlink{}{Open Search}

\hypertarget{die-vertrauensfrage}{%
\section{Die Vertrauensfrage}\label{die-vertrauensfrage}}

\includegraphics{https://swprs.files.wordpress.com/2017/03/foeg-jahrbuch_logo.png?w=400\&h=161}

Das Forschungsinstitut für Öffentlichkeit und Gesellschaft (FÖG) der
Universität Zürich publiziert alljährlich das »Jahrbuch Qualität der
Medien«. 2016 vermeldete das Institut, das Vertrauen in die Schweizer
Medien sei
\href{http://www.foeg.uzh.ch/dam/jcr:7234c6d3-1f09-4d36-b6ab-f14e659d046e/Medienmitteilung_JB_2016_dt.pdf}{»weiterhin
hoch«} -- so das Ergebnis eines Ländervergleichs in Zusammenarbeit mit
dem britischen \emph{Reuters Institute.}

Doch wie hoch ist das Vertrauen in die Schweizer Medien nun wirklich?
Dazu findet man in der Mitteilung des Instituts keine Angaben. Und auch
die
\href{http://www.tagesanzeiger.ch/schweiz/standard/Diese-Menschen-sind-anfaellig-fuer-Populisten/story/23804017}{Zeitungsberichte}
zur Studie erwähnen diese wichtige Kennzahl
\href{http://www.nzz.ch/schweiz/analyse-zum-medienvertrauen-oeffentliche-medien-staerken-auch-die-privaten-ld.128965}{nicht}.
Aus gutem Grund -- denn die Resultate sind erschütternd.

Demnach
\href{http://media.digitalnewsreport.org/wp-content/uploads/2018/11/Digital-News-Report-2016.pdf\#page=60}{halten}~nur
noch 50\% der Schweizer Bevölkerung die Nachrichten für glaubwürdig. Das
Vertrauen in die Medienunternehmen und in die Journalisten liegt mit
39\% bzw. 35\% sogar noch tiefer. Mit anderen Worten: Rund zwei Drittel
der Bevölkerung vertraut den eigenen Journalisten nicht mehr.*\\
*

Dennoch glaubt das Forschungsinstitut~-- das u.a. vom Bundesamt für
Kommunikation finanziert wird -- die Nutzung traditioneller und
insbesondere öffentlicher Medien würde das Vertrauen ins Mediensystem
\href{http://www.foeg.uzh.ch/dam/jcr:7234c6d3-1f09-4d36-b6ab-f14e659d046e/Medienmitteilung_JB_2016_dt.pdf}{»fördern«}.
Die Daten zeigen jedoch nur, dass regelmäßige Konsumenten dieser Medien
weniger kritisch sind -- und ihre Anzahl immer geringer wird.*\\
*

\textbf{Update:} 2017 sank das Medienvertrauen in der Schweiz
\href{http://www.digitalnewsreport.org/survey/2017/switzerland-2017/}{auf
46\%}. Die Werte bzgl. Journalisten und Unternehmen wurden nicht mehr
erhoben. Gemäß FÖG war das Vertrauen
\href{http://www.foeg.uzh.ch/dam/jcr:0d0e5a10-27be-4e97-b264-b2cf7de96bbd/Broschur_Jahrbuch_foeg_deutsch_2017_ohne_Sperrvermerk.pdf}{»weiterhin
hoch«}. 2018 stieg das Medienvertrauen wieder auf 50\%, 2019 sank es
wieder auf 46\% (siehe Abbildung).

\href{https://swprs.files.wordpress.com/2019/11/news-trust-reuters-2019.png}{\includegraphics{https://swprs.files.wordpress.com/2019/11/news-trust-reuters-2019.png?w=736}}

\textbf{Medienvertrauen 2019}: Deutschland 47\%, Schweiz 46\%,
Österreich 39\%, Frankreich 24\%.\\
Quelle:
\href{https://reutersinstitute.politics.ox.ac.uk/sites/default/files/2019-06/DNR_2019_FINAL_0.pdf}{Reuters
Digital News Report 2019}, Seite 21.

\href{https://swprs.files.wordpress.com/2016/11/medien-schweiz-vertrauen-vergleich.png}{\includegraphics{https://swprs.files.wordpress.com/2016/11/medien-schweiz-vertrauen-vergleich.png?w=594\&h=406}}

Vertrauen in \textbf{Medienorganisationen} (gelb) und
\textbf{Journalisten} (blau) im internationalen Vergleich. Die Schweiz
(SUI) liegt auf dem 17. von 26 Plätzen. Quelle:
\href{http://media.digitalnewsreport.org/wp-content/uploads/2018/11/Digital-News-Report-2016.pdf\#page=26}{Reuters
Institute (2016, S. 25)}*\\
*

\begin{center}\rule{0.5\linewidth}{\linethickness}\end{center}

Publiziert: November 2016; aktualisiert: Januar 2020

\hypertarget{swiss-policy-research}{%
\subsubsection{Swiss Policy Research}\label{swiss-policy-research}}

\begin{itemize}
\tightlist
\item
  \href{https://swprs.org/kontakt/}{Kontakt}
\item
  \href{https://swprs.org/uebersicht/}{Übersicht}
\item
  \href{https://swprs.org/donationen/}{Donationen}
\item
  \href{https://swprs.org/disclaimer/}{Disclaimer}
\end{itemize}

\hypertarget{english}{%
\subsubsection{English}\label{english}}

\begin{itemize}
\tightlist
\item
  \href{https://swprs.org/contact/}{About Us / Contact}
\item
  \href{https://swprs.org/media-navigator/}{The Media Navigator}
\item
  \href{https://swprs.org/the-american-empire-and-its-media/}{The CFR
  and the Media}
\item
  \href{https://swprs.org/donations/}{Donations}
\end{itemize}

\hypertarget{follow-by-email}{%
\subsubsection{Follow by email}\label{follow-by-email}}

Follow

\href{https://wordpress.com/?ref=footer_custom_com}{WordPress.com}.

\protect\hyperlink{}{Up ↑}

Post to

\protect\hyperlink{}{Cancel}

\includegraphics{https://pixel.wp.com/b.gif?v=noscript}
