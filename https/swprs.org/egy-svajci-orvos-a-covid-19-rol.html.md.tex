\protect\hyperlink{content}{Skip to content}

\href{https://swprs.org/}{}

\protect\hyperlink{search-container}{Search}

Search for:

\href{https://swprs.org/}{\includegraphics{https://swprs.files.wordpress.com/2020/05/swiss-policy-research-logo-300.png}}

\href{https://swprs.org/}{Swiss Policy Research}

Geopolitics and Media

Menu

\begin{itemize}
\tightlist
\item
  \href{https://swprs.org}{Start}
\item
  \href{https://swprs.org/srf-propaganda-analyse/}{Studien}

  \begin{itemize}
  \tightlist
  \item
    \href{https://swprs.org/srf-propaganda-analyse/}{SRF / ZDF}
  \item
    \href{https://swprs.org/die-nzz-studie/}{NZZ-Studie}
  \item
    \href{https://swprs.org/der-propaganda-multiplikator/}{Agenturen}
  \item
    \href{https://swprs.org/die-propaganda-matrix/}{Medienmatrix}
  \end{itemize}
\item
  \href{https://swprs.org/medien-navigator/}{Analysen}

  \begin{itemize}
  \tightlist
  \item
    \href{https://swprs.org/medien-navigator/}{Navigator}
  \item
    \href{https://swprs.org/der-propaganda-schluessel/}{Techniken}
  \item
    \href{https://swprs.org/propaganda-in-der-wikipedia/}{Wikipedia}
  \item
    \href{https://swprs.org/logik-imperialer-kriege/}{Kriege}
  \end{itemize}
\item
  \href{https://swprs.org/netzwerk-medien-schweiz/}{Netzwerke}

  \begin{itemize}
  \tightlist
  \item
    \href{https://swprs.org/netzwerk-medien-schweiz/}{Schweiz}
  \item
    \href{https://swprs.org/netzwerk-medien-deutschland/}{Deutschland}
  \item
    \href{https://swprs.org/medien-in-oesterreich/}{Österreich}
  \item
    \href{https://swprs.org/das-american-empire-und-seine-medien/}{USA}
  \end{itemize}
\item
  \href{https://swprs.org/bericht-eines-journalisten/}{Fokus I}

  \begin{itemize}
  \tightlist
  \item
    \href{https://swprs.org/bericht-eines-journalisten/}{Journalistenbericht}
  \item
    \href{https://swprs.org/russische-propaganda/}{Russische Propaganda}
  \item
    \href{https://swprs.org/die-israel-lobby-fakten-und-mythen/}{Die
    »Israel-Lobby«}
  \item
    \href{https://swprs.org/geopolitik-und-paedokriminalitaet/}{Pädokriminalität}
  \end{itemize}
\item
  \href{https://swprs.org/migration-und-medien/}{Fokus II}

  \begin{itemize}
  \tightlist
  \item
    \href{https://swprs.org/covid-19-hinweis-ii/}{Coronavirus}
  \item
    \href{https://swprs.org/die-integrity-initiative/}{Integrity
    Initiative}
  \item
    \href{https://swprs.org/migration-und-medien/}{Migration \& Medien}
  \item
    \href{https://swprs.org/der-fall-magnitsky/}{Magnitsky Act}
  \end{itemize}
\item
  \href{https://swprs.org/kontakt/}{Projekt}

  \begin{itemize}
  \tightlist
  \item
    \href{https://swprs.org/kontakt/}{Kontakt}
  \item
    \href{https://swprs.org/uebersicht/}{Seitenübersicht}
  \item
    \href{https://swprs.org/medienspiegel/}{Medienspiegel}
  \item
    \href{https://swprs.org/donationen/}{Donationen}
  \end{itemize}
\item
  \href{https://swprs.org/contact/}{English}
\end{itemize}

\protect\hyperlink{}{Open Search}

\hypertarget{tuxe9nyek-a-covid-19-rux151l}{%
\section{Tények a Covid-19-ről}\label{tuxe9nyek-a-covid-19-rux151l}}

\textbf{Frissítés}: 2020. május 7; \textbf{k}\textbf{özzététel}: 2020.
március 14.\\
\textbf{Nyelvek}: \href{https://swprs.org/fakta-o-covid-19/}{CZ},
\href{https://swprs.org/covid-19-hinweis-ii/}{DE},
\href{https://swprs.org/a-swiss-doctor-on-covid-19/}{EN},
\href{https://swprs.org/hechos-sobre-covid-19/}{ES},
\href{https://swprs.org/faktoja-covid-19sta/}{FI},
\href{https://swprs.org/coronavirus-un-medecin-suisse-parle/}{FR},
\href{https://swprs.org/facts-about-covid19-greek/}{GR},
\href{https://swprs.org/covid-19-cinjenice/}{HBS},
\href{https://yanivhamo.com/facts-about-covid-19-hebrew/}{HE},
\href{https://swprs.org/egy-svajci-orvos-a-covid-19-rol/}{HU},
\href{https://swprs.org/un-medico-svizzero-su-covid-19/}{IT},
\href{https://swprs.org/covid19-facts-japanese/}{JP},
\href{https://swprs.org/covid19-korean/}{KO},
\href{https://www.globalinfo.nl/Achtergrond/een-kritische-kijk-op-het-coronabeleid-transparantie-in-tijden-van-crisis}{NL},
\href{https://midtifleisen.wordpress.com/2020/04/15/fakta-om-covid-19/}{NO},
\href{https://swprs.org/szwajcarski-lekarz-o-covid-19/}{PL},
\href{https://swprs.org/fatos-sobre-covid-19/}{PT},
\href{https://swprs.org/informatii-despre-covid-19/}{RO},
\href{https://swprs.org/\%d0\%bd\%d0\%b0-\%d0\%ba\%d0\%be\%d0\%b2\%d0\%b8\%d0\%b4-19/}{RU},
\href{https://swprs.org/fakta-om-covid-19/}{SE},
\href{http://www.ninamvseeno.org/pregled-clanka.aspx?naslov=pomembne-informacije-o-novem-koronavirusu-covid-19\&id=148}{SI},
\href{https://alatyr.sk/covid-19_swiss_propaganda_research.htm}{SK},
\href{https://swprs.org/isvicreli-bir-doktordan-kovid-19-uezerine/}{TR}\\
\textbf{Ossza meg ezt}:
\href{https://twitter.com/intent/tweet?url=https://swprs.org/egy-svajci-orvos-a-covid-19-rol/}{Twitter}
/
\href{https://www.facebook.com/share.php?u=https://swprs.org/egy-svajci-orvos-a-covid-19-rol/}{Facebook}

A Covid-19-ről szóló, teljes hivatkozású tények, amelyeket a terület
szakértői szolgáltattak, hogy segítsék olvasóinkat a kockázatok reális
értékelésében.

\textbf{„\ldots{} az egyetlen módja a pestis elleni harcnak a
becsületesség.`` Albert Camus: A pestis (1947)}

\hypertarget{uxe1ttekintuxe9s},
  amely egy súlyosabb influenzáénak felel meg, és amely a WHO eredeti
  feltételezésének mintegy egyhuszada.
\item
  Az iskolás- és aktív keresőkorú lakosság körében a halálozás kockázata
  még a világ bármelyik gócpontjában is nagyjából akkora, mint amikor
  valaki
  \href{https://www.medrxiv.org/content/10.1101/2020.04.05.20054361v1}{naponta
  autóval jár munkába}. Ezt a kockázatot kezdetben túlbecsülték, mivel
  az enyhe tüneteket mutató és a tünetmentes személyeket nem vették
  számításba.
\item
  A pozitív teszteredményt mutató személyek akár 80 százaléka
  \href{https://www.bmj.com/content/369/bmj.m1375}{tünetmentes} maradt.
  Még a 70--79 éves korcsoportba tartozó személyek körében is mintegy
  \href{https://www.niid.go.jp/niid/en/2019-ncov-e/9407-covid-dp-fe-01.html}{60\%}
  a tünetmentesek aránya. Az összes személy több mint 97 százaléka is
  \href{https://swprs.org/studies-on-covid-19-lethality/\#hospitalizations}{legfeljebb
  enyhe} tüneteket mutatott.
\item
  Az eddigi koronavírusokkal (azaz meghűlés-vírusokkal) való
  érintkezésnek köszönhetően az összes személy közel egyharmada már
  rendelkezik a Covid19-cel szembeni
  \href{https://www.medrxiv.org/content/10.1101/2020.04.17.20061440v1}{háttér-immunitással}.
\item
  Az elhunytak medián életkora a legtöbb országban (ide értve
  \href{https://www.epicentro.iss.it/coronavirus/sars-cov-2-decessi-italia}{Olaszországot}
  is) 80 év fölött van, és az elhunytak mindössze
  \href{https://www.bloomberg.com/news/articles/2020-03-18/99-of-those-who-died-from-virus-had-other-illness-italy-says}{kb.
  1 százalékának} nem volt komoly alapbetegsége. A halálozási profil
  ezzel lényegében megfelel a
  \href{https://www.vienna.at/analyse-zeigt-covid-19-opferkurve-entspricht-normaler-mortalitaet/6581246}{normál
  halandóságnak}.
\item
  Az összes haláleset 50--70 százaléka a legtöbb országban
  \href{https://ltccovid.org/2020/04/12/mortality-associated-with-covid-19-outbreaks-in-care-homes-early-international-evidence/}{ápolóotthonokban}
  következik be, amelyek nem profitálnak az általános zárlatból.
  Ráadásul többnyire
  \href{https://www.hsj.co.uk/commissioning/thousands-of-extra-deaths-outside-hospital-not-attributed-to-covid-19/7027459.article}{nem
  világos}, hogy ezek az emberek valóban a Covid19-be haltak-e bele,
  vagy a nagy stresszbe, félelembe és
  \href{http://pflegeethik-initiative.de/2020/04/15/corona-krise-falsche-prioritaeten-gesetzt-und-ethische-prinzipien-verletzt/}{magányba}.
\item
  A normál halandóságot meghaladó összes haláleset közel 50 százalékát
  nem a Covid19
  \href{https://www.thetimes.co.uk/edition/news/coronavirus-record-weekly-death-toll-as-fearful-patients-avoid-hospitals-bm73s2tw3}{okozta},
  hanem a
  \href{https://www.telegraph.co.uk/global-health/science-and-disease/two-new-waves-deaths-break-nhs-new-analysis-warns/}{zárlat
  következményei, a pánik és a félelem}. Így például a szívinfarktusok
  és szélütések kezelése majd 60 százalékkal csökkent, mivel a páciensek
  nem mertek elmenni a klinikákra.
\item
  Gyakran még az úgynevezett „Covid19-halálesetekben``
  \href{https://spectator.us/understand-report-figures-covid-deaths/}{sem
  világos}, hogy az elhunyt a koronavírus miatt, vagy koronavírussal a
  szervezetében halt-e meg (azaz valamilyen
  \href{https://www.youtube.com/watch?v=V0lIWZpiRU0}{alapbetegségbe}
  halt-e bele), vagy mint „gyanús esetet`` nem is tesztelték. Ezek a
  különbségek azonban a hivatalos számokból
  \href{https://swprs.org/rki-relativiert-corona-todesfaelle/}{gyakran
  nem derülnek ki}.
\item
  Sok médiajelentésről, amelyek szerint fiatal és egészséges személyek
  is meghaltak Covid19-ben, kiderült, hogy hamis. E fiatal emberek közül
  sokan vagy
  \href{https://www.dailymail.co.uk/news/article-8193487/Coroner-refuses-rule-COVID-19-cause-death-six-week-old-Connecticut-baby.html}{nem}
  Covid19-ben haltak meg, vagy pedig már eleve súlyos
  \href{https://sports.yahoo.com/spanish-football-coach-francisco-garcia-163153573.html}{alapbetegségben}
  (pl. nem diagnosztizált leukémiában) szenvedtek, vagy
  \href{https://www.tagesanzeiger.ch/bund-muss-in-seiner-todesfallstatistik-fehler-korrigieren-584308129723}{9
  helyett 109 évesek} voltak.
\item
  A normál
  \href{https://www.cdc.gov/mmwr/volumes/68/wr/mm6826a5.htm}{össz-halandóság}
  az Egyesült Államokban kb. 8000 fő, Németországban kb. 2600 fő,
  Olaszországban kb. 1800 fő, Svájcban kb. 200 fő naponta. Az
  \href{https://www.statnews.com/2018/09/26/cdc-us-flu-deaths-winter/}{influenza
  halálos áldozatainak} száma egy-egy télen az Egyesült Államokban akár
  80 000 fő, Németországban és Olaszországban
  \href{https://www.sciencedirect.com/science/article/pii/S1201971219303285}{akár
  25 000 fő}, Svájcban akár 2500 fő. A Covid19 több országban
  \href{https://www.euromomo.eu/graphs-and-maps/}{nem érte el} ezeket az
  értékeket.
\item
  A jelentősen megemelkedett regionális halálozási arányokat további
  kockázati tényezők befolyásolhatták: súlyos
  \href{https://www.theguardian.com/environment/2020/apr/20/air-pollution-may-be-key-contributor-to-covid-19-deaths-study?utm_medium}{levegő-szennyezettség},
  \href{https://www.ansa.it/english/news/science_tecnology/2019/11/19/italy-top-in-eu-in-antibiotic-resistance_369e0123-0107-445e-8c17-f11932c9d27c.html}{mikróba-terhelés},
  valamint az
  \href{https://swprs.org/covid-19-a-report-from-italy/}{idős- és
  betegápolás} fertőzések, tömeges pánik vagy a zárlati intézkedések
  következtében történő összeomlása. Az elhunytakra vonatkozó különleges
  előírások részben további fennakadásokat okoztak a
  \href{https://www.ecdc.europa.eu/sites/default/files/documents/COVID-19-safe-handling-of-bodies-or-persons-dying-from-COVID19.pdf}{temetkezésben}.
\item
  Néhány országban, például Olaszországban, Spanyolországban, valamint
  részben Nagy-Britanniában és az Egyesült Államokban korábban az
  influenzahullámok is a
  \href{https://off-guardian.org/2020/04/02/coronavirus-fact-check-1-flu-doesnt-overwhelm-our-hospitals/}{kórházak
  túlterhelését} eredményezték. Most ráadásul az orvosok és ápolók
  \href{https://www.reuters.com/article/us-health-coronavirus-spain-morgue-idUSKBN21B1PP}{15
  százalékának} vesztegzárba kell vonulnia, akkor is, ha tünetmentesek.
\item
  A „korona-esetekről`` gyakran mutogatott exponenciális görbék
  \href{https://multipolar-magazin.de/artikel/coronavirus-regierung-ignoriert-daten}{félrevezetőek},
  mivel a tesztek száma is hatványozottan nő. A pozitív teszteredmények
  aránya az összes teszthez képest (ún. pozitív-arány) a legtöbb
  országban állandó,
  \href{https://swprs.org/rate-of-positive-covid19-tests/}{5\% és 25\%
  közötti} tartományban van vagy csak lassan emelkedik. A fertőzés a
  legtöbb országban már
  \href{https://www.dailymail.co.uk/news/article-8235979/UKs-coronavirus-crisis-peaked-lockdown-Expert-argues-draconian-measures-unnecessary.html}{a
  zárlat bevezetése előtt} elérte tetőpontját.
\item
  A kijárási és érintkezési tilalmat be nem vezető országokban, mint
  például
  \href{https://www.japantimes.co.jp/news/2020/03/20/national/coronavirus-explosion-expected-japan/}{Japánban},
  \href{https://www.businessinsider.com/south-korea-coronavirus-testing-death-rate-2020-3?op=1}{Dél-Koreában}
  és
  \href{https://www.washingtontimes.com/news/2020/apr/15/sweden-coronavirus-rates-easing-despite-loose-rule/}{Svédországban}
  nem lett rosszabb a helyzet, mint a többi országban. Sőt, Svédországot
  a WHO legutóbb már
  \href{https://www.nau.ch/news/schweiz/coronavirus-who-nennt-schweden-ein-vorbild-65701044}{követendő
  példaként} dicsérte, amely most már a széles körű immunitásból
  profitál.
\item
  Megalapozatlan volt a félelem attól, hogy nem lesz elég
  lélegeztetőgép.
  \href{https://off-guardian.org/2020/05/06/covid19-are-ventilators-killing-people/}{Tüdőgyógyász
  szakorvosok} szerint a Covid19-páciensek invazív lélegeztetése
  (intubáció), amely részben a vírustól való
  \href{https://www.dailymail.co.uk/news/article-8262351/Nurse-New-York-claims-city-killing-COVID-19-patients-putting-ventilators.html}{félelemből}
  történt, ráadásul gyakran
  \href{https://off-guardian.org/2020/05/06/covid19-are-ventilators-killing-people/}{kontraproduktív}
  és további tüdőkárosodást okoz.
\item
  Az eredeti vélekedésekkel szemben
  \href{https://www.who.int/news-room/commentaries/detail/modes-of-transmission-of-virus-causing-covid-19-implications-for-ipc-precaution-recommendations}{több
  tanulmány} is kimutatta, hogy sem a vírus aeroszolok (azaz a levegőben
  lebegő részecskék) útján történő terjedésére, sem az érintkezés (pl.
  kilincs, okostelefon, fodrász) útján történő fertőzésre
  \href{https://www.telegraph.co.uk/news/2020/04/02/no-proof-coronavirus-can-spread-shopping-says-leading-german/}{nincs
  bizonyíték}.
\item
  Annak
  \href{https://www.researchgate.net/publication/340570735_Masks_Don't_Work_A_review_of_science_relevant_to_COVID-19_social_policy}{sincs
  tudományos alapja}, hogy egészséges vagy tünetmentes személyek esetén
  a légzésvédelmi maszk hatásos lenne. Szakértők attól óvnak
  \href{https://www.aerztezeitung.de/Politik/Montgomery-haelt-Maskenpflicht-fuer-falsch-408844.html}{inkább},
  hogy az ilyen maszkok akadályozzák a légzést és
  „\href{https://de.sputniknews.com/interviews/20200425326953541-corona-gefahr-virologe/}{víruscsúzlikká}``
  válnak.
\item
  Európában és az
  \href{https://www.usatoday.com/story/news/health/2020/04/02/coronavirus-pandemic-jobs-us-health-care-workers-furloughed-laid-off/5102320002/}{Egyesült
  Államokban} sok klinika kihasználtsága nagyon
  \href{https://www.spiegel.de/wirtschaft/unternehmen/trotz-corona-pandemie-warum-kliniken-jetzt-kurzarbeit-anmelden-a-3dc61bc9-fb12-4298-8022-bb4c2be39d7d}{alacsony}
  maradt, sőt, részben át kellett térniük a
  \href{https://www.20min.ch/schweiz/news/story/Spitaeler-28949526}{rövidített
  munkaidőre}. Sok műtétet és kezelést
  \href{https://www.zeit.de/2020/18/kliniken-coronavirus-intensivbetten-patienten-behandlung-notaufnahme}{lemondtak},
  köztük „nem halaszthatatlan`` szerv-átültetéseket és rákvizsgálatokat.
\item
  Több médiumot
  \href{https://nypost.com/2020/04/01/cbs-admits-to-using-footage-from-italy-in-report-about-nyc/}{kaptak
  rajta} azon, hogy megpróbálták dramatizálni a klinikákon tapasztalható
  helyzetet, esetenként akár manipulatív képekkel is. A sok médium
  \href{https://onlinelibrary.wiley.com/doi/full/10.1111/eci.13222}{komolytalan
  beszámolójának} az lett az általános hatása, hogy óriásira nőtt a
  félelem a lakosság körében.
\item
  A nemzetközileg használt vírusteszt-készletek könnyen
  \href{https://www.ncbi.nlm.nih.gov/pubmed/32219885}{hibáznak} és
  adhatnak hamis pozitív és hamis negatív eredményeket. A hivatalos
  teszteszközt az idő szorítása miatt ráadásul
  \href{https://www.youtube.com/watch?v=p_AyuhbnPOI}{klinikailag nem is
  validálták}, más koronavírusokra is reagálhat.
\item
  Sok, nemzetközileg elismert virológus, immunológus és járványtani
  \href{https://www.rubikon.news/artikel/120-expertenstimmen-zu-corona}{szakértő}
  \href{https://off-guardian.org/2020/03/28/10-more-experts-criticising-the-coronavirus-panic/}{kontraproduktívnak}
  \href{https://off-guardian.org/2020/03/24/12-experts-questioning-the-coronavirus-panic/}{tartja}
  a bevezetett intézkedéseket, és a lakosság gyors és
  \href{https://off-guardian.org/2020/04/17/8-more-experts-questioning-the-coronavirus-panic/}{természetes
  immunizációját}, valamint a veszélyeztetett csoportok védelmét
  javasolja. Az iskolák bezárására
  \href{https://www.thelancet.com/journals/lanchi/article/PIIS2352-4642(20)30095-X/fulltext}{egyetlen
  korábbi időpontban sem} volt orvosi ok.
\item
  A koronavírusok ellen szorgalmazott oltóanyagokat több szakértő
  \href{https://www.youtube.com/watch?v=vrL9QKGQrWk}{szükségtelennek},
  sőt, akár veszélyesnek
  \href{https://www.nature.com/articles/d41586-020-00751-9}{tartja}. A
  2009-ben az
  \href{https://www.forbes.com/2010/02/05/world-health-organization-swine-flu-pandemic-opinions-contributors-michael-fumento.html}{úgynevezett
  sertésinfulenza} elleni oltóanyag részben súlyos
  \href{https://www.ibtimes.co.uk/brain-damaged-uk-victims-swine-flu-vaccine-get-60-million-compensation-1438572}{neurológiai
  károsodásokat} okozott és milliós összegű perekre vezetett.
\item
  Az intézkedések miatt munkanélkülivé vált, lelki problémáktól és
  családon belüli erőszaktól szenvedő emberek száma világszerte
  \href{https://www.reuters.com/article/us-health-coronavirus-usa-layoffs/us-weekly-jobless-claims-seen-at-record-high-again-idUSKBN21K0FX}{magasba
  szökött}. Több szakértő arra számít, hogy az intézkedések több életet
  \href{https://www.nytimes.com/2020/03/20/opinion/coronavirus-pandemic-social-distancing.html}{követelnek
  majd}, mint maga a vírus. Az ENSZ szerint világszerte
  \href{https://de.euronews.com/2020/04/22/un-warnen-welt-droht-wegen-corona-eine-hunger-pandemie-von-biblischen-ausma-en}{emberek
  milliói} süllyednek teljes szegénységbe és éhínségbe.
\item
  Edward Snowden, az NSA szivárogtatója attól
  \href{https://www.youtube.com/watch?v=-pcQFTzck_c}{óvott}, hogy a
  korona-válságot világméretű megfigyelő-eszközök tartós kiépítésére
  fogják felhasználni. Pablo Goldschmidt elismert virológus „globális
  médiaterrorról`` és „totalitárius intézkedésekről``
  \href{https://www.rubikon.news/artikel/der-corona-totalitarismus}{beszélt}.
  John Oxford brit infektológus „média-járványról``
  \href{https://novuscomms.com/2020/03/31/a-view-from-the-hvivo-open-orphan-orph-laboratory-professor-john-oxford/}{beszélt}.
\item
  Több mint 500 tudós
  \href{https://www.esat.kuleuven.be/cosic/sites/contact-tracing-joint-statement/}{figyelmeztetett}
  a társadalom problematikus kontaktkövető alkalmazásokkal történő
  „példátlan megfigyelésére``. Ezeket az érintkezéseket néhány országban
  a titkosszolgálat már
  \href{https://www.jewishpress.com/news/the-courts/state-to-high-court-even-more-shin-bet-involvement-in-fighting-the-coronavirus/2020/04/14/}{követi}.
  A polgári lakosság
  \href{https://off-guardian.org/2020/04/25/50-headlines-darker-more-of-the-new-normal/}{drónokkal}
  történő megfigyelése és részben súlyos rendőri erőszak világszerte
  előfordult.
\end{enumerate}

\textbf{Lásd még}:

\begin{itemize}
\tightlist
\item
  \href{https://swprs.org/open-letter-from-professor-sucharit-bhakdi-to-german-chancellor-dr-angela-merkel/}{Bhakdi
  professzor nyílt levele}\\
\item
  \href{https://swprs.org/korona-media-propaganda/}{Korona, média,
  propaganda}
\item
  \href{https://www.euromomo.eu/}{Európai halálozási megfigyelés}
\end{itemize}

\begin{center}\rule{0.5\linewidth}{\linethickness}\end{center}

\hypertarget{tovuxe1bbi-frissuxedtuxe9sek-angol-vagy-nuxe9met-nyelven}{%
\paragraph{\texorpdfstring{További frissítések
\href{https://swprs.org/a-swiss-doctor-on-covid-19/}{angol} vagy
\href{https://swprs.org/covid-19-hinweis-ii/}{német}
nyelven.}{További frissítések angol vagy német nyelven.}}\label{tovuxe1bbi-frissuxedtuxe9sek-angol-vagy-nuxe9met-nyelven}}

\hypertarget{2020-uxe1prilis-7}{%
\paragraph{2020. április 7.}\label{2020-uxe1prilis-7}}

\begin{itemize}
\tightlist
\item
  A német Robert Koch Intézet
  \href{https://multipolar-magazin.de/artikel/coronavirus-regierung-ignoriert-daten}{különleges
  jelentésének legfrissebb számai} azt mutatják, hogy az úgynevezett
  pozitívarány (azaz a pozitív teszteredmények számának aránya a tesztek
  számához képest), jóval lassabban emelkedik, mint a médiában mutatott
  exponenciális görbék, és március végén 10\% körül volt, amely érték
  alapvetően jellemző a koronavírusokra. A Multipolar magazin szerint
  ezért „szó sem lehet a vírus veszélyesen gyors terjedéséről``.
\item
  A hamburgi igazságügyi orvostani intézet professzora, Klaus Püschel
  \href{https://www.pressreader.com/germany/hamburger-morgenpost/20200403/281487868456736}{ezt
  mondta a Covid19-ről}: „Ez a vírus teljesen eltúlzott módon
  befolyásolja az életünket. Ez nem áll arányban azzal a veszéllyel,
  amelyet a vírus jelent. És az a csillagászati mértékű gazdasági kár,
  amely most keletkezik, nem áll arányban a vírus jelentette veszéllyel.
  Meg vagyok győződve arról, hogy a koronavírus-halandóság még csak
  kiugró értékként sem fog megjelenni az éves halálozási adatokban.``
  Hamburgban például eddig „egyetlen olyan ember sem halt meg
  koronavírusban, aki nem szenvedett volna valamilyen alapbetegségben:
  az összes elhunyt, akit eddig megvizsgáltunk, rákban, idült
  tüdőgyulladásban szenvedett, erős dohányos vagy súlyosan elhízott
  volt, cukorbetegségben szenvedett vagy szív- és érrendszeri
  problémákkal küzdött.`` A vírus ebben úgyszólván az utolsó csepp volt,
  amelytől betelt a pohár. „A Covid19 csak kivételes esetben halálos
  betegség, a legtöbb esetben azonban alapvetően ártalmatlan lefolyású
  vírusos fertőzés.``
\item
  \href{https://www.abendblatt.de/hamburg/article228828787/rechtsmedizin-pueschel-hamburg-corona-virus-infektion-covid-19-coronavirus-krise-patienten-krankenhaeuser-kliniken-infektionsrate-krankheit-pandemie-test-lungenkrankheit-sars-cov-epidemie-sars-cov-2.html}{Dr.
  Püschel így folytatta}: „Nem kevés esetben azt is megállapítottuk,
  hogy az aktuális koronavírus-fertőzésnek egyáltalán semmi köze a
  halálos kimenetelhez, mivel más a halálozás oka, például agyvérzés
  vagy szívinfarktus.`` A korona „nem különösebben veszélyes vírusos
  megbetegedés``, mondja a törvényszéki orvos. Konkrét vizsgálati
  eredményeken nyugvó statisztikákat sürget. „Az összes feltételezés
  egyes halálesetekről, amelyeket nem ellenőriztek szakszerűen, csak a
  félelmet táplálja``.
\item
  Hamburg szabad Hanza-város a berlini Robert Koch Intézet
  iránymutatásával ellentétesen nemrég elkezdett különbséget tenni a
  „koronavírus okozta`` halálesetek és a „koronavírussal a
  szervezetükben`` elhunytak között, aminek a Covid19-halálesetek
  számának
  \href{https://www.t-online.de/nachrichten/deutschland/id_87636856/coronavirus-hamburg-will-nur-echte-covid-19-tote-zaehlen.html}{csökkenése}
  lett az eredménye.
\item
  Hendrik Streeck német virológus jelenleg a Covid19-kórokozó
  terjedését, terjedési módját meghatározó tanulmányt végez.
  \href{https://www.zeit.de/wissen/gesundheit/2020-04/hendrik-streeck-covid-19-heinsberg-symptome-infektionsschutz-massnahmen-studie/komplettansicht}{Egy
  interjúban így fogalmazott}: „Heinsberg járásban alaposabban
  megvizsgáltam a 40 elhunytból 31-et -- és nem voltam nagyon
  meglepődve, hogy ezek az emberek meghaltak. Az egyik elhunyt több mint
  100 éves volt, az ő esetében egy sima nátha is halált okozhatott
  volna``. Kilincsen és hasonló tárgyakon keresztül történő, azaz
  érintkezés útján terjedő fertőzést eddig az eredeti feltételezések
  ellenére nem tudott kimutatni.
\item
  Svájcban érkeznek az első jelentések olyan kórházakról, amelyekben a
  nagyon alacsony kihasználtság miatt rövidített munkaidőt kell
  bevezetni: „Az összes osztályon alig van a személyzetnek tennivalója,
  első lépésben megszüntettük a meghosszabbított munkaidőt. Most pedig
  rövidített munkaidőt vezetünk be. A pénzügyi következmények
  jelentősek.`` Emlékeztetőül: a Zürichi Egyetem irreális feltevésekre
  alapozó tanulmánya április 2-ra
  \href{https://www.toponline.ch/news/coronavirus/detail/news/studie-bestaetigt-engpass-bei-spitalbetten-steht-kurz-bevor-00131333/}{jelezte
  előre} az első ellátási zavarokat, szűk keresztmetszeteket svájci
  klinikákon. Eddig ilyesmire sehol sem került sor.
\item
  Svájcban 2017-ben volt egy komoly influenza-hullám. Akkor a 65 év
  fölötti lakosság körében az év első hat hetében közel
  \href{https://www.srf.ch/news/schweiz/todesursachen-statistik-woran-die-meisten-schweizerinnen-und-schweizer-sterben}{1500
  plusz halálesetet} regisztráltak. Normális esetben Svájcban évente
  mintegy \href{https://www.nzz.ch/lungenentzuendung-1.4550285}{1300
  ember} veszti életét tüdőgyulladás következtében, 95 százalékuk 65
  évnél idősebb. Összehasonlításul: Svájcban jelenleg összesen
  \href{https://www.corona-data.ch/}{762 elhunytról} számoltak be,
  akinek Covid19 volt a szervezetében (nem feltétlenül abba halt bele).
\item
  Egy német környezeti labor ügyvezetője úgy véli, az észak-olaszországi
  Lombardia lakosai a tartósan nagy legionella-terhelés miatt
  \href{https://m.apotheke-adhoc.de/nachrichten/detail/coronavirus/erhoehen-legionellen-die-todesrate-einer-corona-infektion/}{különösen
  ki vannak téve olyan vírusos fertőzéseknek}, mint a Covid19: „Ha a
  tüdő egy vírusos fertőzés miatt le van gyengülve, mint a mostani
  helyzetben, a baktériumoknak könnyű a dolguk, negatívan hatnak a
  betegség lefolyására és szövődményeket okozhatnak.`` Lombardiában a
  legionellával szennyezett párologtatók már a múltban is okoztak
  regionális tömeges tüdőgyulladásokat.
\item
  A Kínából érkező adatok alapján az egész világon olyan orvosi
  protokollokat határoztak meg, amelyek hamar invazív mesterséges
  lélegeztetést (intubációt) irányoznak elő a pozitív teszteredményt
  mutató és az intenzív osztályon kezelt betegek számára. A protokollok
  egyrészt abból indulnak ki, hogy a kíméletesebb, nem invazív, maszkos
  lélegeztetés túl gyenge, másrészt főleg attól tartanak, hogy „a
  veszélyes vírus`` ilyen esetben képes a levegőben terjedni. Német
  orvosok már márciusban
  \href{https://www.doccheck.com/de/detail/articles/26271-covid-19-beatmung-und-dann}{felhívták
  a figyelmet arra}, hogy az intubáció további tüdőkárosodást okozhat és
  összességében rosszak a sikerességi kilátásai. Azóta az Egyesült
  Államokból is jelentkeztek orvosok, akik leírják, hogy az intubáció
  „\href{https://www.youtube.com/watch?v=k9GYTc53r2o}{többet árt, mint
  használ}`` a betegeknek. A betegek gyakran nem akut
  tüdőelégtelenségben szenvednek, hanem inkább egyfajta magassági
  betegségben, amelyen a fokozott nyomással járó mesterséges légzés csak
  tovább ront. Ezzel szemben
  \href{https://www.upi.com/Top_News/World-News/2020/02/14/Oxygen-therapy-working-for-coronavirus-patient-Seoul-says/6651581696794/}{dél-koreai
  orvosok már februárban jelezték}, hogy a kritikus állapotban lévő
  Covid19-betegek jól reagálnak a lélegeztetőkészülék nélküli
  oxigénterápiára. A fent nevezett amerikai orvos arra figyelmeztet,
  hogy sürgősen át kell gondolni a lélegeztetőkészülékek használatát,
  nehogy még több kért okozzanak.
\item
  Az Egyesült Államok hivatalos Covid19-modellje a kórházba kerülési
  arányt eddig 8-szorosan, az intenzív osztályra kerülő betegek számát
  6,4-szeresen, a szükséges lélegeztetőkészülékek számát 40,5-szörösen
  \href{https://twitter.com/NikolovScience/status/1246823479820693505}{becsülte
  túl.}
\item
  Az \href{https://vimeo.com/403175258}{ARD Monitor anyaga} a 2009. évi
  „sertésinfluenza`` eltúlzott sajtóbeli bemutatásáról meglepő
  párhuzamokat mutat a mai helyzettel. Az ARD anyagának összefoglalója
  így szólt: „A járványtól a való félelem a tulajdonképpeni járvány.``
\end{itemize}

\hypertarget{tovuxe1bbi-huxedrek}{%
\subparagraph{\texorpdfstring{\textbf{További
hírek}}{További hírek}}\label{tovuxe1bbi-huxedrek}}

\begin{itemize}
\tightlist
\item
  A Covid19-pánik egyik legkorábbi és nemzetközileg legismertebb
  kritikusának \href{http://wodarg.com/}{Dr. Wolfgang Wodargnak a
  honlapját} a mai napon a Jimdo, német szolgáltató néhány órára
  \href{https://twitter.com/wodarg}{törölte} és csak heves tiltakozások
  hatására tette újra elérhetővé. Nem tudni, hogy az időleges törlés
  általános panaszok miatt vagy politikai utasításra történt-e.
\item
  Nemrég Dr. Sucharit Bhakdi professor emeritus egyetemi e-mail címét is
  deaktiválták, de tiltakozások hatására visszaállították; Dr. Bhakdi
  volt az, aki
  \href{https://swprs.org/offener-brief-von-professor-sucharit-bhakdi-an-bundeskanzlerin-dr-angela-merkel/}{nyílt
  levelet írt Angela Merkel szövetségi kancellárnak}.
\item
  A dán parlament április 2-án
  \href{https://newsvoice.se/2020/04/danmark-forbjuder-corona-policy/}{új
  törvényt fogadott el}, amely megtiltja a Covid19-ről szóló, a kormány
  előirányzatainak nem megfelelő információk közzétételét, és lehetővé
  teszi internetes oldalak törlését, valamint a szerzők megbüntetését és
  letartóztatását. Erre néhány kommentátor azonnal visszavonult.
\item
  Harald Wiesendanger német tudományos és orvosi újságíró egy cikkében
  azt írja, hogy
  \href{https://www.nachrichten-fabrik.de/news/harald-wiesendanger-ueber-die-massenmedien-waehrend-der-corona-krise-ich-schaeme-mich---meines-berufsstands-152103}{szakmája
  teljes kudarcot vallott a jelenlegi válságban}: „Ahogy ez a szakma,
  amelynek az lenne a dolga, hogy független, kritikus, elfogulatlan
  negyedik hatalmi ágként ellenőrizze a hatalmasokat, ugyanúgy
  villámgyorsan és szinte egyöntetűen bedől ugyanannak a kollektív
  hisztériának, mint a közönsége, és odaveti magát az udvari
  tudósításnak, kormánypropagandának, a tudomány mint szent tehén
  szakértői istenítésének: nekem ez felfoghatatlan, undorít, elegem van
  belőle, szégyellem magam miattuk, és teljes mértékben elhatárolódom
  ettől a méltatlan teljesítménytől.``
\item
  Jelenleg
  \href{https://www.sciencealert.com/one-third-of-the-world-s-population-are-now-restricted-in-where-they-can-go}{az
  emberiség kb. egyharmada} él zárlatban, többen, mint ahány ember élt a
  Földön a második világháború idején.
\item
  Az Egyesült Államokban a munkanélküli segély iránti kérelmek száma
  \href{https://www.reuters.com/article/us-health-coronavirus-usa-layoffs/us-weekly-jobless-claims-seen-at-record-high-again-idUSKBN21K0FX}{több
  mint hat millióra emelkedett} (lásd grafikon), történelmi, egyedülálló
  érték az 1929. évi nagy válság óta.
\item
  Száznál több ember- és polgárjogi szervezet
  \href{https://www.dailymail.co.uk/news/article-8181381/World-sleepwalking-surveillance-state-rights-groups-warn.html}{óv
  attól}, hogy az emberiség a „korona-válság`` miatt mint egy alvajáró
  belesétáljon a megfigyelőállam rendszerébe. A Twitteren a \#covid19
  mellett megjelent a \#covid1984 hashtag is.
\item
  Henry Kissinger amerikai geostratéga az írja a Wall Street Journalban,
  hogy
  „\href{https://www.wsj.com/articles/the-coronavirus-pandemic-will-forever-alter-the-world-order-11585953005}{a
  koronavírus-világjárvány örökre megváltoztatja a világrendet}``. Az
  Egyesült Államoknak meg kell védenie polgárait, ugyanakkor „meg kell
  terveznie egy új korszakot``.
\end{itemize}

\hypertarget{2020-uxe1prilis-5}{%
\paragraph{2020. április 5.}\label{2020-uxe1prilis-5}}

\begin{itemize}
\tightlist
\item
  A nemzetközileg elismert New York-i epidemiológus professzor, Knut
  Wittkowski egy \href{https://www.youtube.com/watch?v=lGC5sGdz4kg}{40
  perces interjúban} kifejti, hogy a Covid19-cel összefüggésben
  bevezetett intézkedések összességében kontraproduktívak. A „social
  distancing``, az iskolák bezárása, a „lock down``, a maszk, a tömeges
  tesztelés és az oltások helyett az életnek minél inkább zavartalanul
  kellene tovább mennie, és minél hamarabb ki kellene építeni a lakosság
  immunitását. Az eddigi ismeretek szerint a Covid19 nem veszélyesebb a
  korábbi influenza-járványoknál.
\item
  A British Medical Journal (BMJ)
  \href{https://www.bmj.com/content/369/bmj.m1375}{arról számol be},
  hogy a legfrissebb kínai adatok szerint az új fertőzöttek (pozitív
  teszteredményt mutató személyek) 78 százaléka nem mutat semmilyen
  tünetet. Ez további, arra utaló jel, hogy a vírus viszonylag
  veszélytelen a lakosságra.
\item
  A Bécsi Orvostudományi Egyetem általános és családi orvosi osztályának
  vezetője és a bizonyítékalapú orvostudományi hálózat elnöke, Dr.
  Andreas Sönnichsen
  \href{https://www.diepresse.com/5794224/was-machen-wir-da-auf-den-intensivstationen-eigentlich}{„őrültségnek``
  tartja az eddig bevezetett intézkedéseket}. Az egész állam megbénul,
  csak azért, hogy „azokat a keveseket, akiket érinthetne, védjük``. A
  német SWR tévécsatornának adott interjúban
  \href{https://www.swr.de/swraktuell/radio/im-gespraech/corona-krise-wir-kommen-nicht-an-der-ausbreitung-vorbei-100.html}{elmondja},
  hogy a vírus terjedését amúgy sem lehet megakadályozni.
\item
  A svéd kormány
  \href{https://www.telegraph.co.uk/news/2020/04/03/coronavirus-swedish-experiment-could-prove-britain-wrong/}{a
  világon elsőként bejelentette}, hogy a jövőben hivatalosan különbséget
  tesznek a koronavírus „következtében`` és a koronavírussal „a
  szervezetükben`` elhunytak között. Ez a halálesetek számának további
  csökkenését eredményezheti. Közben egyre nő a Svédországra nehezedő
  nemzetközi nyomás, hogy adja fel liberális stratégiáját. Úgy tűnik,
  némely kormány attól tart, hogy saját intézkedéseik Svédország sikere
  (mint korábban Japáné) tükrében aránytalannak és kontraproduktívnak
  tűnnek.
\item
  A német orvosi lap, az Ärzteblatt már 2018-ban arról számolt be, hogy
  Észak-Olaszországban
  „\href{https://www.aerzteblatt.de/nachrichten/97750/Vielzahl-an-Lungenentzuendungen-beunruhigen-Behoerden-in-Norditalien}{sok
  a tüdőgyulladás}``, ami nyugtalanítja a hatóságokat. Akkoriban a többi
  között a szennyezett ivóvízzel magyarázták a jelenséget.
\item
  A német gyógyszerészeti lap, a Pharmazeutische Zeitung
  \href{https://www.pharmazeutische-zeitung.de/atemstillstand-koennte-auch-zentrale-ursache-haben-116664/}{arra
  hívja fel a figyelmet}, hogy a jelenlegi helyzetben gyakran előfordul,
  hogy „a páciensek súlyosan megbetegszenek, akár meg is halnak, anélkül
  hogy előzőleg respirációs (légzési) tüneteket mutattak volna``.
  Neurológusok úgy vélekednek erről, hogy a koronavírusok idegsejteket
  is károsíthatnak. További lehetséges magyarázat, hogy ezek a gyakran
  ápolásra szoruló betegek a nagyon nagy stresszbe halnak bele.
\item
  A lélegeztetett betegek száma Svájcban
  \href{https://www.bag.admin.ch/dam/bag/de/dokumente/mt/k-und-i/aktuelle-ausbrueche-pandemien/2019-nCoV/covid-19-lagebericht.pdf.download.pdf/COVID-19_Epidemiologische_Lage_Schweiz.pdf}{a
  legfrissebb számok szerint} összesen 280-ról 260-ra csökkent. Csak a
  pozitív teszteredményt mutató és kórházba került betegek 43
  százalékának van tüdőgyulladása. Az ő esetükben sem eleve világos,
  hogy a koronavírus vagy más vírus okozta-e a tüdőgyulladást. A pozitív
  teszteredményt mutató elhunytak medián-életkora 83 év, a legidősebb
  elhunyt 101 éves volt.
\item
  Az „\href{http://inproportion2.talkigy.com/}{In Proportion}``
  (arányaiban) elnevezésű brit projekt követi a Covid19-cel a
  szervezetükben elhunytak és az influenza okozta halálesetek arányát a
  teljes halandósághoz képest, amely Nagy-Britanniában továbbra is a
  normál tartományban vagy az alatt van, és jelenleg csökken.
\item
  Az Egyesült Államokban Indiana államban a zárlati intézkedések és azok
  gazdasági következményeinek hatására a lelki problémákkal küzdő és
  öngyilkossági gondolatokkal foglalkozó emberek hívásait fogadó
  telefonos segélyszolgálathoz naponta beérkező hívások száma 1000-ről
  több mint 2000 százalékkal, 25 000-re
  \href{https://twitter.com/JesseKellyDC/status/1246449878219145216}{emelkedett}.
\item
  A Rxisk orvosi szakmai portál
  \href{https://rxisk.org/medications-compromising-covid-infections/}{arra
  hívja fel a figyelmet}, hogy a koronavírus fertőzési kockázatát a
  különböző gyógyszerek akár 200 százalékkal is növelhetik. Az is
  ismeretes, hogy az influenzavírus elleni oltás tendenciájában
  \href{https://www.sciencedirect.com/science/article/pii/S0264410X19313647?via=ihub}{növelheti}
  a koronavírusban történő megbetegedést.
\end{itemize}

\hypertarget{tovuxe1bbi-huxedrek-1}{%
\subparagraph{\texorpdfstring{\textbf{További
hírek}}{További hírek}}\label{tovuxe1bbi-huxedrek-1}}

\begin{itemize}
\tightlist
\item
  A brit Daily Mail újságírója, Peter Hitchens
  \href{https://www.firstthings.com/web-exclusives/2020/04/we-love-big-brother}{„We
  love Big Brother``} című cikkében leírja, hogyan hagyták korábban
  kritikus emberek is mindenféle orvosi bizonyíték nélkül megfertőzni
  magukat a félelemmel. A fenyegetett alapjogokkal összefüggésben azt
  mondja egy interjúban, hogy a kritika most
  \href{https://www.spiked-online.com/podcast-episode/in-this-lockdown-dissent-is-a-moral-duty/}{erkölcsi
  kötelesség}.
\item
  Németországban több ügyvédi iroda kereseteket készít elő a meghozott
  intézkedésekkel és rendeletekkel szemben. Egy orvosi jogra szakosodott
  ügyvédnő azt írja
  \href{http://beatebahner.de/lib.medien/aktualisierte\%20Pressemitteilung.pdf}{sajtóközleményében}:
  „A szövetségi és tartományi kormány intézkedései súlyosan
  alkotmányellenesek és eddig soha nem látott mértékben sértik a
  németországi polgárok számos alapjogát. Ez a 16 szövetségi tartomány
  összes korona-rendeletére érvényes. Ezeket az intézkedéseket nem
  igazolja a fertőzés elleni védekezésről szóló törvény sem, amelyet
  csak néjány nappal ezelőtt, sebbel-lobbal módosítottak. {[}\ldots{}{]}
  A rendelkezésre álló számok és statisztikák azt mutatják, hogy a
  korona-fertőzés a lakosság több mint 95 százalékában tünetmentesen
  lefolyik (vagy vélhetőleg már le is folyt), így nem jelent súlyos
  veszélyt a lakosságra nézve.``
\item
  A német szövetségi kormány egy kiszivárogtatott,
  \href{https://fragdenstaat.de/dokumente/4123-wie-wir-covid-19-unter-kontrolle-bekommen/}{bizalmas
  stratégiai irata} megmutatja, hogy a német kormány a médiával és
  néhány tudóssal karöltve nyilvánvalóan „sokk-stratégiát`` követ,
  amellyel félelmet akarnak kelteni az emberekben a „legrosszabb
  forgatókönyvtől``. A lakosságot -- amelynek legnagyobb részére a vírus
  ártalmatlan -- „kínokkal teli, fulladásos halálra`` kell
  figyelmeztetni; a játszótéren játszó gyerekek szüleik „kínokkal teli
  halálát`` okozhatják, áll az anyagban.
\item
  Sucharit Bhakdi professzor Angela Merkel szövetségi kancellárnak írott
  \href{https://swprs.org/offener-brief-von-professor-sucharit-bhakdi-an-bundeskanzlerin-dr-angela-merkel/}{nyílt
  levele} már németül, angolul, franciául, spanyolul, oroszul, törökül,
  hollandul és észtül is elérhető; további fordítások jönnek.
\item
  Edward Snowden NSA-szivárogtató egy
  \href{https://www.youtube.com/watch?v=-pcQFTzck_c}{interjúban
  (német/angol nyelven)} azt mondja, hogy a Covid19 veszélyes, de
  időleges, míg az alapjogok eltiprása halálos és folyamatos.
\end{itemize}

\hypertarget{2020-uxe1prilis-3}{%
\paragraph{2020. április 3.}\label{2020-uxe1prilis-3}}

\textbf{Ausztria}: A
\href{https://www.heute.at/s/osterreich-bei-corona-todesstatistik-sehr-liberal-48665863}{média
tudósításai szerint} a „korona-haláleseteket`` úgy tűnik, Ausztriában is
„nagyon liberálisan`` kezelik: „Akkor is korona-halottnak számít valaki,
ha megfertőződött vírussal, de valami másba halt bele?`` „Igen``, mondja
Rudi Anschober és Bernhard Benka, az egészségügyi minisztérium
korona-különítményének két tagja. „A mostani szabály világos:
koronavírusba halt bele, vagy koronavírussal a szervezetében halt meg``,
mondja Benka. „Az összes ilyen eset beleszámít a statisztikába. Nem
tesznek különbséget aszerint, hogy mibe halt bele ténylegesen a beteg.
Pongyolán fogalmazva: ha egy 90 éves ember combnyaktörésben meghal és a
halála előtti utolsó órákban megfertőződik koronavírussal, ő is
korona-halottnak számít, hogy csak egy példát mondjak.``

\textbf{Németország}: A német Robert Koch Intézet nem tanácsolja a
pozitív teszteredményt mutató elhunytak boncolását, mivel állítólag
\href{https://www.youtube.com/watch?v=gSn_YaOYYcY}{túl nagy} az
aeroszolos cseppfertőzés kockázata. Így viszont sok esetben már nem
határozható meg a halál valódi oka.

Egy patológus szakorvos ezt így
\href{https://www.youtube.com/watch?v=gSn_YaOYYcY}{kommentálja} (a levél
szövege a videó alatt): „Rossz az, aki rosszra gondol! A patológusok
számára eddig is természetes volt, hogy megfelelő óvintézkedések mellett
fertőző betegségek -- HIV/AIDS, hepatitis, tuberkulózis, Prion-betegség
stb. -- esetén is elvégezzék a boncolást. Azért az furcsa, hogy egy
olyan járványban, amely az egész Földön több ezer beteget elragad az
életből és országok gazdaságát állítja le szinte teljesen, csak nagyon
kevés boncolási lelet áll rendelkezésre (Kínából hat betegről). A
boncolási eredményeket mind járványrendészeti, mind orvosi szempontból
különösen nagy, nyilvános érdeklődésnek kellene öveznie. És pont az
ellenkezőjét tapasztaljuk. Félnek attól, hogy megtudják a pozitív
teszteredményt mutató elhunytak valódi halálozási okát? Lehet, hogy
akkor úgy elolvadna a korona-halottak száma, mint hó a tavaszi
napsütésben?``

\textbf{Olaszország}: Lombardiai ápolóotthonokban orosz ápolók
„\href{https://de.sputniknews.com/panorama/20200402326767475-fachpersonal-todesfaelle-lombardei-zeitung/}{furcsa
haláleseteket}`` tapasztaltak: Gromóban újsághírek szerint több olyan
esetet regisztráltak, amelyek az állítólagos korona-fertőzöttek
egyszerűen elaludtak és többé nem ébredtek fel. Az elhunytak addig nem
mutatták a betegség semmilyen komolyan veendő tünetét. {[}\ldots{}{]}
Ahogy az idősek otthonának igazgatója később a RIA Novosztyinak
pontosította, nem világos, hogy az elhunytak ténylegesen fertőzöttek
voltak-e, mert az otthonban senkit nem teszteltek erre. {[}\ldots{}{]}
Azokban az otthonokban, amelyekben oroszországi orvosok és ápolók
dolgoznak, fertőtlenítik a folyosókat, a szobákat és az étkezőket.``

Hasonló esetekről már Németországból is
\href{https://web.archive.org/web/20200330082928/https://www.sueddeutsche.de/panorama/coronavirus-news-deutschland-wolfsburg-laschet-1.4828033}{beszámoltak}:
a betegség tüneteit nem mutató ápoltak hirtelen meghalnak a jelenlegi
rendkívüli helyzetben, és „korona-halálesetnek`` számítanak. Újra
felvetődik a súlyos következményekkel járó kérdés: ki hal bele a
vírusba, és ki hal bele a részben szélsőséges intézkedésekbe?

\textbf{Ápolók}: A Süddeutsche Zeitung
\href{https://www.sueddeutsche.de/politik/coronavirus-pflegekraefte-ausland-1.4866124}{szerint}:
„A járvány egész Európában veszélyezteti az otthonukban élő idősek
ellátását, mert az ápolók már nem mehetnek el hozzájuk -- vagy pedig már
menekülésszerűen elhagyták az adott országot, és hazamentek.``

\textbf{További hírek}: A Stanford orvos-professzora, Dr. Jay
Bhattacharya \href{https://www.youtube.com/watch?v=-UO3Wd5urg0}{félórás
interjút} adott, amelyben megkérdőjelezi a Covid19-ről alkotott
közfelfogást. Az eddigi intézkedéseket nagyon bizonytalan és részben
kérdéses adatok alapján vezették be.

\hypertarget{2020-uxe1prilis-2-i}{%
\paragraph{2020. április 2. (I.)}\label{2020-uxe1prilis-2-i}}

\hypertarget{usa}{%
\subparagraph{\texorpdfstring{\textbf{USA}}{USA}}\label{usa}}

Felix Scholkmann biofizikus
\href{https://swprs.org/rate-of-positive-covid19-tests/}{felismerte azt
a tényt}, hogy az USA-ban (mint a világ többi részén) nem a „fertőzött''
szám növekszik exponenciálisan, hanem a tesztek száma. A ``fertőzött''
szám a vizsgálatok számához viszonyítva lényegében állandó marad (10 és
20\% között), ami elvben a folyamatban lévő vírusjárvány ellen szól.

\includegraphics{https://swprs.files.wordpress.com/2020/04/ud-data-2-fs.png?w=736}

\hypertarget{nuxe9metorszuxe1g}{%
\subparagraph{\texorpdfstring{\textbf{Németország}}{Németország}}\label{nuxe9metorszuxe1g}}

A német Robert Koch Intézet
\href{https://influenza.rki.de/Wochenberichte/2019_2020/2020-13.pdf}{legfrissebb
influenza-jelentése} szerint az akut légúti megbetegedések száma
„országszerte jelentősen csökkent``. Az értékek „az összes korcsoportban
jelentős mértékben csökkentek``.

Az akut légúti megbetegedéssel kórházban kezelt esetek száma március
20-ig (12. hét) egyértelműen csökkent. A 80 év fölötti fölötti
korcsoportban pedig az esetszám az előző héthez képest közel
megfeleződött.

A 73 vizsgált kórházban az összes légúti megbetegedéses eset 7
százalékában diagnosztizáltak Covid19-et. Ez az arány a 35--59 éves
korcsoportban 16\%, a 60--79 éves korcsoportban 13\% volt.

Ezek a számok megfelelnek a többi országból érkező számoknak, valamint a
koronavírusok tipikus elterjedtségének (5\%--15\%).

\href{https://swprs.files.wordpress.com/2020/04/rki-ili-kw13.png}{}

\includegraphics{https://swprs.files.wordpress.com/2020/04/rki-ili-kw13.png?w=279\&h=171}

Chřipková onemocnění (RKI, 13.kal. týden)

\href{https://swprs.files.wordpress.com/2020/04/rki-sari-kw12.png}{}

\includegraphics{https://swprs.files.wordpress.com/2020/04/rki-sari-kw12.png?w=449\&h=171}

Akutní onemocnění dýchacích cest v nemocnicích

Influenza-szerű megbetegedések összesen és akut légúti megbetegedések
kórházakban\\
(Robert Koch Intézet, 2020. 12. és 13. hét)

A
\href{https://www.zeit.de/wissen/2020-04/krankenhaeuser-kapazitaeten-coronavirus-patienten-deutschland/seite-2}{Die
Zeit egyik cikke} a németországi intenzív osztályokon kezelt betegek
kérdésével foglalkozik:

„Az új fertőzöttek számának exponenciális növekedését mostanában
politikusok, szakemberek és sok polgár is naponta, aggodalommal figyeli.
De ha meg akarjuk becsülni, hogy Németországot milyen súlyosan érinti és
fogja érinteni a korona-válság, nem ez a döntő fontosságú mutatószám.
Ezt ugyanis mindenekelőtt a hetek óta egyre nagyobb számban végzett
tesztek száma torzítja.

Az egészségügy terhelésének mérésére ezzel szemben főként azok száma
fontos, akik olyan súlyosan megbetegedtek, hogy lélegeztetni kell őket.
Amíg számukra van elég lélegeztetőhely, nagyon sokukat meg lehet
menteni. Csak ha kevés lesz az ilyen ágy, akkor fenyeget az
olaszországihoz hasonló helyzet.

A DIVI-jegyzék most azt mutatja, hogy a német intenzív osztályokon eddig
nem feszült a helyzet. »Még komfortos tartományban vagyunk«, mondja
Grabenreich. »A súlyos betegek száma távolról sem emelkedik olyan
meredeken, mint a fertőzötteké, de még ha így lenne is, akkor is nagyon
sok, jól felszerelt intenzív osztályos ágy áll rendelkezésünkre«.``

\hypertarget{svuxe1jc}{%
\subparagraph{\texorpdfstring{\textbf{Svájc}}{Svájc}}\label{svuxe1jc}}

A svájci Szövetségi Egészségügyi Hivatal (BAG)
\href{https://www.bag.admin.ch/bag/de/home/krankheiten/ausbrueche-epidemien-pandemien/aktuelle-ausbrueche-epidemien/novel-cov/situation-schweiz-und-international.html}{jelentése}
szerint eddig kb. 139 330 Covid19-tesztet végeztek, ezek 15 százaléka
lett pozitív (pdf). Ez a szám is megfelel a többi országban ismert, a
koronavírusra jellemző értéknek, és úgy tűnik, ez az érték eddig
Svájcban sem nő.

Csupán a tesztek száma nő hatványozottan, ezt közli a média oly gyakran,
de nem a „fertőzötteké``, a megbetegedetteké, és főleg nem az
elhunytaké.

Március 31-én közzétették a
\href{https://www.bfs.admin.ch/bfs/de/home/statistiken/gesundheit/gesundheitszustand/sterblichkeit-todesursachen.html}{friss
heti halálozási statisztikát}, amely Svájcra a (március 22-ig tartó) 12.
hétre jelzett első ízben előre megemelkedett halálozási arányt a 65 év
fölötti korcsoportban (lásd lenti grafikon). Ez konkrétan azt jelenti,
hogy a halálozások száma a teljes lakosság körében mintegy heti 200
fővel fog növekedni.

Ez a növekmény állítólag „a jelenlegi világjárvány kifejeződése``. Itt a
következő problémáról van szó: március 22-ig összesen 106 pozitív
teszteredményt mutató haláleset volt. Egy heti 200 fős növekmény azt
jelentené, hogy a plusz halandóság nagy részét, nem a vírus, hanem az
„ellenintézkedések`` okozzák.

Egy másik magyarázat lehet, hogy a következő hét
(\href{https://de.wikipedia.org/wiki/COVID-19-Pandemie_in_der_Schweiz\#Todesf\%C3\%A4lle}{13.
hét}) mintegy 200 pozitív teszteredményt mutató halálesetét már
beleszámították ebbe. Ez azt jelentené, hogy a pozitív teszteredményt
mutató összes halálesetet plusz halálozásnak veszik. Azonban tekintettel
az elhunytak életkor és betegségprofil szerinti összetételére, valamint
a
\href{https://swprs.org/rki-relativiert-corona-todesfaelle/}{nemzetközi
tapasztalatokra}, ez igencsak kétséges feltevés.

A jelentésben aztán valóban meg is jegyzik: „ezek ez első becslések még
nagyon bizonytalanok, úgyhogy pontos adatok nem közölhetők.``

Amennyiben az derülne ki, hogy a pozitív teszteredményt mutató
halálesetek nagy része (medián életkor: 83 év) nem plusz halálesetek,
akkor a teljes halandóság vagy nem nő, vagy főleg a drasztikus
intézkedések következtében nő, ahogy attól némely szakértő
\href{https://swprs.org/offener-brief-von-professor-sucharit-bhakdi-an-bundeskanzlerin-dr-angela-merkel/}{tart}.

\includegraphics{https://swprs.files.wordpress.com/2020/04/bfs-mortaliaet-22-03.png?w=600\&h=400}

A svájci Tages-Anzeiger
\href{https://interaktiv.tagesanzeiger.ch/2020/uebersterblichkeit-wegen-coronavirus/}{összehasonlította}
a jelenlegi teljes halandóságot a korábbi évekkel (lásd lenti grafikon).
Ez megmutatja, hogy a jelenlegi halálozási arány, még ha tényleg
emelkedik is, továbbra is az elmúlt évek súlyosabb influenzát hozó téi
időszakai alatt van.

\includegraphics{https://swprs.files.wordpress.com/2020/04/mortalitc3a4t-schweiz.png?w=720\&h=339}

\hypertarget{tovuxe1bbi-huxedrek-2}{%
\subparagraph{\texorpdfstring{\textbf{További
hírek}}{További hírek}}\label{tovuxe1bbi-huxedrek-2}}

\begin{itemize}
\tightlist
\item
  A Nagy-Britanniába szánt vírusteszt-készleteket
  \href{https://www.telegraph.co.uk/news/2020/03/30/uks-attempt-ramp-coronavirus-testing-hindered-key-components/}{vissza
  kellett hívni}, mert már eleve koronavírus-alkotóelemeket
  tartalmaztak.
\item
  Amint most kiderült, a brit Imperial College több százezer halottat
  előrejelző, de soha egyetlen szakmai folyóiratban közzé nem tett vagy
  felülvizsgálatnak alá nem vetett tanulmánya
  \href{https://judithcurry.com/2020/04/01/imperial-college-uk-covid-19-numbers-dont-seem-to-add-up/}{súlyosan
  irreális feltevésekre épült}.
\item
  A BBC felteszi a kérdést:
  „\href{https://www.bbc.com/news/health-51979654}{A koronavírus okozza
  a haláleseteket}?`` És válaszol is: „Lehet a vírus fő ok, egy további
  tényező, vagy egyszerűen csak az is jelen van az elhunyt
  szervezetében.`` Így tudósítottak egy 18 éves férfiről mint
  „legfiatalabb koronavírus-áldozatról``, miután a halála előtti napon
  pozitív lett a tesztje. A kórház viszont később közölte, hogy a
  fiatalember súlyos alapbetegségbe halt bele.
\item
  Az európai egészségügyi hatóság, az ECDC
  \href{https://www.ecdc.europa.eu/sites/default/files/documents/COVID-19-safe-handling-of-bodies-or-persons-dying-from-COVID19.pdf}{nagyon
  szigorú előirányzatokat tett közzé} a pozitív teszteredményt mutató
  vagy „vélhetőleg pozitív teszteredményt mutató`` elhunytak
  holttestéről. Az eddig nagyon alacsony halálozási arányok tükrében az
  ilyen előirányzatok orvosi szempontból igencsak kérdésesek;
  szignifikáns mértékben növelik az egészségügy és a temetkezés
  terhelését, ugyanakkor nagyon hatásosan tálalhatók a médiában.
\item
  A Bayerischer Rundfunk
  \href{https://www.br.de/nachrichten/wissen/bhakdis-brief-an-die-kanzlerin-was-ist-dran-an-seinen-fragen,RutYDhd}{kritikus
  kommentárt} tett közzé Sucharit Bhakdi professzor Merkel kancellárnak
  írott nyílt leveléről.
\item
  Az ARTE 2009-ben készített dokumentumfilmje, a
  „\href{https://vimeo.com/403175258}{Profiteure der Angst}`` („Akik a
  félelmen nyerészkednek``) bemutatja, hogyan minősített fel a főleg
  privát pénzekből finanszírozott WHO egy ártalmatlan influenza-járványt
  (az ún. „sertés-influenzát``) világjárvánnyá, és hogyan adtak el ennek
  következtében több milliárd dollárért részben veszélyes oltóanyagokat
  a kormányoknak. Az akkori főbb szereplők közül néhányan a mai
  helyzetnek is
  \href{https://www.nature.com/articles/news.2009.424}{prominens
  szereplői}.
\item
  A brit Legfelső Bíróság korábbi bírája, Jonathan Sumption
  \href{https://www.spectator.co.uk/article/former-supreme-court-justice-this-is-what-a-police-state-is-like-}{egy
  a BBC-nek adott interjúban} ezt mondta a brit intézkedésekről: „Így
  néz ki egy rendőrállam.``
\end{itemize}

\hypertarget{2020-uxe1prilis-2-ii}{%
\paragraph{2020. április 2. (II.)}\label{2020-uxe1prilis-2-ii}}

\begin{itemize}
\tightlist
\item
  Most már
  \href{https://pflege-prisma.de/2020/03/31/sterbezahlen-in-pflegeheimen/}{ápolóotthonok
  képviselői} is panaszkodnak a korlátozó intézkedésekről és a média
  helytelen beszámolási gyakorlatáról a Covid19-ről.
\item
  A Velencéhez közel fekvő, észak-olaszországi Treviso számai azt
  mutatják, hogy a városi kórházakban március végéig mért teljes
  halandóság a pozitív teszteredményt mutató 108 elhunyt ellenére
  \href{https://swprs.files.wordpress.com/2020/04/reppublica-treviso.jpg}{nagyjából
  megegyezik} az előző évi értékekkel. Ez újabb, arra utaló jel, hogy a
  néhány helyen átmenetileg megnövekedett halandóság inkább harmadik
  tényezőkkel -- pánik, rendszer összeomlása -- függnek össze, mint a
  koronavírussal.
\item
  John Oxford, a londoni Queen Mary University professzora, világszerte
  vezető virológus és influenza-specialista
  \href{https://novuscomms.com/2020/03/31/a-view-from-the-hvivo-open-orphan-orph-laboratory-professor-john-oxford/}{így
  vélekedik a Covid19-ről}: „Személyesen azt mondanám, hogy a legjobb
  tanács az, hogy kevesebb időt töltsenek a szenzációs és nem nagyon jó
  tévéhírek nézésével. Ezt a Covid-járványt egy rossz téli
  influenza-járványnak tartom. Tavaly ebben az összefüggésben 8000
  halálesetünk volt a veszélyeztetett csoportokban, azaz több mint 65
  százalékban a szívbetegségben és hasonlókban szenvedő emberek körében.
  Nem hiszem, hogy az idei Covid meg fogja haladni ezt a számot.
  Média-járványban szenvedünk!``
\end{itemize}

\hypertarget{2020-uxe1prilis-1}{%
\paragraph{2020. április 1.}\label{2020-uxe1prilis-1}}

\hypertarget{az-olaszorszuxe1gi-helyzetrux151l}{%
\subparagraph{\texorpdfstring{\textbf{Az olaszországi
helyzetről}}{Az olaszországi helyzetről}}\label{az-olaszorszuxe1gi-helyzetrux151l}}

Olasz orvosok arról számolnak be, hogy már az előző év végén
\href{https://www.scmp.com/news/china/society/article/3076334/coronavirus-strange-pneumonia-seen-lombardy-november-leading}{megfigyeltek}
súlyos tüdőgyulladásos eseteket Észak-Olaszországban. Genetikai
elemzések azonban azt mutatják, hogy a „Covid19-vírus`` idén januárban
bukkant fel először Olaszországban. „A novemberben és decemberben
Olaszországban diagnosztizált súlyos tüdőgyulladásokat tehát más
kórokozóra kell visszavezetni``
\href{https://www.nzz.ch/wissenschaft/coronavirus-der-stammbaum-verraet-woher-es-kommt-ld.1548271}{írja
az NZZ}. Ez újra felveti a kérdést, hogy milyen tényleges szerepet
játszik a Covid19-vírus, és milyen szerepet játszanak más tényezők az
olaszországi helyzetben.

Március 30-án felhívtuk a figyelmet „a korona-válság alatt`` elhunyt
olasz orvosok listájára; a listán szereplő orvosok közül sokan valójában
már régóta nyugdíjasok voltak, életkoruk akár 90 év is volt, és nem volt
közvetlen közük a válsághoz. A mai napra a listáról
\href{https://portale.fnomceo.it/elenco-dei-medici-caduti-nel-corso-dellepidemia-di-covid-19/}{eltüntették
az összes születési évet} (vesd össze az legutóbbi
\href{https://web.archive.org/web/20200328152430/https://portale.fnomceo.it/elenco-dei-medici-caduti-nel-corso-dellepidemia-di-covid-19/}{archivált
változattal}). Furcsa folyamat.

Ezen kívül egy olaszországi megfigyelő olyan dolgokat közölt velünk,
amelyek az olaszországi drámai helyzettel összefüggésben olyan további
szempontokat vetnek fel, amelyek jóval túlmutatnak egy víruson:

„Az elmúlt hetekben a legtöbb kelet-európai ápoló, akik a hét 7 napján
24 órás szolgálatban dolgoztak Olaszországban az ápolásra szoruló
személyek ellátásában, menekülésszerűen elhagyta az országot. Nem utolsó
sorban a pánikkeltés és a »szükségállapoti kormány« által elrendelt
kijárási tilalom és határzár miatt. Ezért ápolásra szoruló időseket és
fogyatékosokat, akiknek nincsenek rokonaik, az őket ellátók mindenfajta
segítségnyújtás nélkül magukra hagyták.

Ezek közül az elhagyott emberek közül sokan néhány nap elteltével
bekerültek az évek óta folyamatosan túlterhelt kórházakba, a többi
között kiszáradás miatt. Ekkor azonban már a kórházakból is hiányos volt
a személyzet, amelynek tagjainak be kellett zárkózniuk lakásaikba,
gyerekeikre kellett vigyázniuk, mert az iskolák és óvodák be voltak
zárva. Ez vezetett aztán a fogyatékosokat és időseket ápoló intézmények
teljes összeomlásához és kaotikus viszonyokhoz, főleg azokban a
térségekben, ahol még súlyosabb »intézkedéseket« hoztak.

Az ápolás terén kialakult rendkívüli állapot, amely a pánik miatt
alakult ki, átmenetileg sok halálos áldozatot követelt az ápolásra
szorulók körében, és egyre többet a fiatalabb kórházi betegek körében
is. Ezeket a halálos áldozatokat a felelősök és a média arra használta,
hogy még inkább pánikot keltsen az emberekben, amikor például »újabb 475
halálos áldozatról« és arról számolnak be zenei aláfestéssel,
felhalmozott koporsókat és katonai teherautókat mutató képekkel, hogy
»katonák szállítják el a kórházból a halottakat«.

Pedig mindez csak a következménye volt annak, hogy a temetkezési
vállalkozók a »gyilkos vírustól« való félelmükben megtagadták
szolgáltatásaikat. Ezen kívül egyrészt egyszerre nagyon sok haláleset
történt, másrészt a kormány olyan törvényt hozott, amely szerint a
koronavírust szervezetükben hordozó holttesteket el kell hamvasztani.
Olaszországban mostanáig csak kevés hamvasztás volt. Az országban csak
kevés és kis krematórium van, amelyek hamar elérik kapacitásuk határát.
Az elhunytakat ezért különböző templomokban kellett felravatalozni.

Ezek a fejlemények az összes országban alapvetően ugyanúgy zajlottak le.
Az egészségügyi rendszer minősége azonban jelentősen befolyásolja a
hatásokat. Németországban, Ausztriában vagy Svájcban ezért kevesebb
probléma van, mint Olaszországban, Spanyolországban vagy az Egyesült
Államokban. Azonban ahogy a hivatalos számokból látjuk, a halálozási
arányban nincs említésre méltó emelkedés. Csak egy kis kiugró »púp«,
amely ebből a tragédiából ered.``

\hypertarget{egyesuxfclt-uxe1llamokbeli-nuxe9metorszuxe1gi-uxe9s-svuxe1jci-klinikuxe1k}{%
\subparagraph{\texorpdfstring{\textbf{Egyesült államokbeli, németországi
és svájci
klinikák}}{Egyesült államokbeli, németországi és svájci klinikák}}\label{egyesuxfclt-uxe1llamokbeli-nuxe9metorszuxe1gi-uxe9s-svuxe1jci-klinikuxe1k}}

\begin{itemize}
\tightlist
\item
  A CBS amerikai tévéadót
  \href{https://www.theblaze.com/news/cbs-news-footage-italy-hospital-nyc}{rajtakapták},
  hogy egy az aktuális New York-i helyzetről szóló anyagban egy
  olaszországi intenzív osztályon készült felvételeket használt fel,
  anélkül hogy ezt feltüntette volna.
\item
  A német intenzív osztályok jegyzéke a médiában megjelent hírekkel
  szemben szintén
  \href{https://www.divi.de/register/intensivregister}{nem mutat
  megemelkedett kihasználtságot}. Polgár-újságírók berlini klinikákon
  \href{https://www.youtube.com/watch?v=WiJszJmGdxY}{teljesen
  elhagyatott Covid-19-felvételi központokat} látogattak meg. Egy
  müncheni klinika munkatársa azt mondja, hogy „hetek óta várnak a nagy
  hullámra``, de „a betegszámban nincs emelkedés``. A politikusok
  állításai nem egyeznek a saját tapasztalataival, a „gyilkos vírus
  mítoszát nem lehet megerősíteni``.
\item
  Eddig a svájci klinikákon sem tapasztalható a kihasználtság
  emelkedése. A luzerni kantonális kórház egyik látogatója arról számol
  be, hogy „kisebb a forgalom, mint normális időben``. Teljes emeleteket
  zártak le a Covid19 számára, de a személyzet „még mindig csak várja a
  betegeket``. A berni, bázeli, zugi és zürichi kórházak is „tök
  üresek``. Az intenzív osztályok ágyai még (az Olaszországgal
  szomszédos) Ticino kantonban
  \href{https://www.nzz.ch/schweiz/tessin-verlegt-erste-corona-patienten-in-deutschschweizer-spitaeler-ld.1549417}{sincsenek
  teljesen kihasználva}, ennek ellenére az német nyelvű országrész
  üresen álló osztályaira helyeznek át betegeket. Tisztán orvosi
  szempontból ennek alig van bármi értelme.
\end{itemize}

\hypertarget{tovuxe1bbi-orvosi-jelentuxe9sek}{%
\subparagraph{\texorpdfstring{\textbf{További orvosi
jelentések}}{További orvosi jelentések}}\label{tovuxe1bbi-orvosi-jelentuxe9sek}}

\begin{itemize}
\tightlist
\item
  A Hamburg-Eppendorfi Egyetemi Klinika infektológusa és igazgatója, Dr.
  Ansgar Lohse
  \href{https://www.mopo.de/hamburg/uke-infektiologe-fordert-es-muessen-sich-mehr-menschen-mit-corona-infizieren-36483636?originalReferrer=\&originalReferrer=}{a
  kijárási tilalmak és érintkezés-tilalmak mihamarabbi megszüntetését
  követeli}. Több embernek kellene megfertőződnie koronavírussal. Az
  óvodákat és iskolákat a lehető leghamarabb újra meg kell nyitni, hogy
  a gyerekek és szüleik is immúnissá válhassanak a megfertőződés
  következtében. A szigorú intézkedések fenntartása gazdasági válságot
  okozna, amely szintén emberéleteket követel, mondja az orvos.
\item
  Spanyolországban
  \href{https://www.heise.de/tp/features/Das-ist-keine-Krise-sondern-eine-Katastrophe-4694104.html}{a
  pozitív teszteredményt mutató személyek 15 százaléka} orvos és
  betegápoló. Ők ugyan többnyire tünetmentesek maradnak, viszont
  vesztegzárba kell vonulniuk, aminek következtében a spanyol
  egészségügyi rendszer egyre inkább összeomlik.
\item
  Dr. John Lee patológus professor emeritus a brit The Spectatorban a
  „korona-halálesetek``
  \href{https://www.spectator.co.uk/article/how-to-understand-and-report-figures-for-covid-19-deaths-}{súlyosan
  félrevezető definíciójával és kommunikálásával} foglalkozik.
\item
  A
  \href{https://swprs.files.wordpress.com/2020/04/die-lage-in-norwegen.pdf}{legfrissebb
  norvégiai adatok} egy környezet-toxikológus értékelése szerint
  szintént azt mutatják, hogy a pozitív teszteredményt mutató személyek
  aránya -- szemben azzal, ahogyan az egy járvány esetén várható lenne
  -- nem nő, hanem a koronavírusra jellemző 2\%--10\% közötti normál
  tartományon belül ingadozik. A pozitív teszteredményt mutató elhunytak
  átlagéletkora 84 év, a halálozás okát nem közlik, a szokásos mértéket
  meghaladó halandóság nem áll fenn.
\item
  Svédországra, amely eddig radikális intézkedések nélkül is megvan, és
  nem jelent megemelkedett halálozási arányt (hasonlóan, mint Ázsiában
  Japán vagy Dél-Korea), a nemzetközi média furcsa módon
  \href{https://www.theguardian.com/world/2020/mar/30/catastrophe-sweden-coronavirus-stoicism-lockdown-europe}{nyomást
  gyakorol}, hogy változtasson stratégiáján.
\item
  New York államból érkező adatok azt mutatják, hogy a pozitív
  teszteredményt mutató személyek kórházba kerülési aránya az eredeti
  feltételezések
  \href{https://www.nytimes.com/2020/03/27/nyregion/new-rochelle-coronavirus.html}{kevesebb
  mint egyhuszada}.
\item
  Egy a
  \href{https://www.doccheck.com/de/detail/articles/26271-covid-19-beatmung-und-dann}{DocCheck
  szakmai portálon megjelent poszt} a pozitív teszteredményt mutató
  páciensek lélegeztetésével foglalkozik. Az ő esetükben hivatalosan nem
  javallják az egyszerű, oxigénmaszkos lélegeztetést, a többi között
  azért, hogy a koronavírus ne terjedjen az aeroszolokkal. A pozitív
  teszteredményt mutató, intenzív osztályon lévő betegeket ezért gyakran
  közvetlenül intubálják (műanyag lélegeztetőcsövet vezetnek a
  légcsövükbe). Az intubációnak azonban rosszak a sikerességi kilátásai,
  gyakran további károsodást okoz a tüdőben (ún. ventilátor-indukált
  tüdőkárosodás). Ahogy a gyógyszerezésben, itt is felmerül a kérdés,
  hogy a betegek kíméletesebb kezelésének orvosi szempontból nem volna-e
  több értelme.
\end{itemize}

\hypertarget{jelentuxe9sek-politikai-fejlemuxe9nyekrux151l}{%
\subparagraph{\texorpdfstring{\textbf{Jelentések politikai
fejleményekről}}{Jelentések politikai fejleményekről}}\label{jelentuxe9sek-politikai-fejlemuxe9nyekrux151l}}

\begin{itemize}
\tightlist
\item
  Egy német tartományi miniszter arra
  \href{https://de.nachrichten.yahoo.com/strobl-b\%C3\%BCrger-verst\%C3\%B6\%C3\%9Fe-gegen-corona-regeln-polizei-melden-095746341.html}{szólította
  fel} a lakosságot, hogy „legyenek éberek, és a korona-járvány
  korlátozására hozott szabályok megszegését jelentsék a rendőrségnek``.
  A polgárok
  „\href{https://www.br.de/nachrichten/bayern/buerger-melden-eifrig-verstoesse-gegen-corona-regeln,RuGXp1h}{buzgón
  jelentik}`` a „tiltott kiscsoportos gyülekezést, a játszótereken vagy
  zsúrokon játszó gyerekeket`` és a kirándulókat.
\item
  Német alkotmányjogászok az
  „\href{https://www.focus.de/politik/deutschland/corona-regelungen-der-regierung-medizin-darf-nicht-gefaehrlicher-sein-als-die-krankheit_id_11827625.html}{alapjogokba
  történő súlyos beavatkozások}`` miatt vészharangot kongatnak. Hans
  Michael Heinig alkotmányjogász arra figyelmeztet, hogy „a demokratikus
  jogállam nagyon rövid idő alatt fasisztoid--hisztérikus
  higiénia-állammá`` alakulhat át. Christoph Möllers, a berlini Humboldt
  Egyetem professzora kifejti, hogy a fertőzés elleni védelemről szóló
  törvény „nem szolgálhat a polgárok szabadságjogainak ilyen súlyos
  korlátozásának alapjául``. Hans Jürgen Papier, a német szövetségi
  alkotmánybíróság egykori elnöke szerint a rendkívüli helyzetben hozott
  intézkedések „nem igazolják a szabadságjogok autoriter és megfigyelő
  állam javára történő hatályon kívül helyezését``.
\item
  Több országban online petíciókat indítottak a kijárási tilalmak és az
  alapjogokba történő más beavatkozások mihamarabbi befejezésére.
  Ugyanakkor az is látható, hogy például nő a kritikus videóanyagok
  törlése. Berlinben
  \href{https://www.heise.de/tp/features/Wenn-Demonstranten-zu-Gefaehrdern-erklaert-werden-4692869.html}{a
  rendőrség feloszlatott} egy az alapjogokról szóló, bejelentett
  rendezvényt, amelyen a német Alaptörvényt osztogatták.
\end{itemize}

\hypertarget{2020-muxe1rcius-31-i}{%
\paragraph{2020. március 31. (I.)}\label{2020-muxe1rcius-31-i}}

Dr. Richard Capek és más kutatók
\href{https://coronadaten.wordpress.com/}{már rámutattak}, a pozitív
teszteredményt mutató személyek számának és az elvégzett tesztek
számának aránya az összes vizsgált országban állandó marad, ami a vírus
exponenciális terjedése („járvány``) ellen szól, és mindössze a tesztek
számának exponenciális emelkedésére utal.

A pozitív teszteredményt mutató személyek aránya országtól függően 5\%
és 15\% között van, ami megfelel a koronavírusok szokásos
elterjedtségének. Érdekes, hogy a hatóságok ezeket a konstans értékeket
nem kommunikálják aktívan
(\href{https://multipolar-magazin.de/artikel/coronavirus-irrefuhrung-fallzahlen}{sőt,
akár el is tussolják}), ehelyett mindenféle kontextus nélkül
exponenciális, de irreleváns és félrevezető görbéket mutogatnak.

Ez természetesen nem felel meg a professzionális orvostudományi
szabványoknak, ahogyan azt a német Robert Koch Intézet hagyományos
\href{https://influenza.rki.de/Saisonberichte/2017.pdf}{influenza-jelentésére}
vetett pillantás is bizonyítja (grafikon a jelentés 130. oldalán, lásd
alább). Itt az igazolt esetek száma mellett (jobbra) a tesztek számát
(balra), és a pozitív teszteredmények arányát is (kék görbe a bal oldali
grafikonon) is kimutatják.

Ebből láthatóvá válik, hogy a pozitív teszteredmények aránya az
influenza-szezonban 0\%--10\% közötti értékről akár 80 százalékra is
felugrik, majd néhány hét elteltével visszasüllyed a normál értékre.
Ezzel összevetve a pozitív Covid19-tesztek aránya a normál tartományba
esik (lásd lent).

\includegraphics{https://swprs.files.wordpress.com/2020/03/rki-influenza-report-2017.png?w=650\&h=530}

Pozitív Covid19-tesztek konstans aránya az USA példáján keresztül (Dr.
Richard Capek). Ez analóg módon érvényes az összes többi országra,
amelyekben rendelkezésre állnak adatok a tesztek számáról.

\includegraphics{https://swprs.files.wordpress.com/2020/03/infizierte-pro-test2603.jpg?w=600\&h=325}

\hypertarget{2020-muxe1rcius-31-ii}{%
\paragraph{2020. március 31. (II.)}\label{2020-muxe1rcius-31-ii}}

\begin{itemize}
\tightlist
\item
  Az
  \href{https://off-guardian.org/2020/03/30/covid19-yet-to-impact-europes-overall-mortality/}{európai
  monitoring-adatok grafikus értékelése} látványosan megmutatja, hogy a
  teljes európai halandóság március 25-ig a bevezetett intézkedésektől
  függetlenül a normál tartományban vagy az alatt van, sőt több helyen
  az előző évi értékek alatt. A halandóság egyedül Olaszországban (a 65
  év fölötti korcsoportban) emelkedett legutóbb (vélhetőleg több okból
  eredően), de még így is elmarad a korábbi influenzás téli
  időszakoktól.
\item
  A német Robert Koch Intézet elnöke egy újabb sajtótájékoztatón
  megerősítette, hogy a meglévő alapbetegségek és a halál valódi oka
  \href{https://swprs.org/rki-relativiert-corona-todesfaelle/}{nem
  játszik szerepet} az úgynevezett „korona-halálesetek``
  meghatározásában (lásd videó). Egy ilyen definíció orvosi szempontból
  egyértelműen félrevezető. Az a nyilvánvaló és általánosan ismert
  hatása, hogy rátör a félelem a politikára és a társadalomra.
\item
  Olaszországban időközben
  \href{https://www.tagesspiegel.de/politik/die-verlangsamung-ist-da-in-italien-zeichnet-sich-die-wende-in-der-coronakrise-ab/25698124.html}{csillapodni
  látszik} a helyzet. Ami eddig látható, az az, hogy a halálozási arány
  időleges emelkedése (a 65+ korcsoportban) konkrét helyi hatásokkal
  magyarázható, amelyek a tömeges pánik és a betegellátás összeomlásának
  gyakori velejárói. Egy észak-olaszországi politikus azt kérdezi: „hogy
  van az, hogy Bresciából Németországba szállítanak Covid-betegeket,
  miközben a közeli Veneto és Verona tartományokban az intenzív
  osztályos ágyak kétharmada üres?``
\item
  A Stanford orvos-professzora, John C. Ioannidis a European Journal of
  Clinical Investigationban közölt
  \href{https://onlinelibrary.wiley.com/doi/abs/10.1111/eci.13222}{cikkében}
  „az eltúlzott információk és nem bizonyítékon alapuló intézkedések
  okozta károkat`` kritizálja.
\item
  Egy március elején a Chinese Journal of Epidemiology-ban közölt és a
  Covid-teszt megbízhatatlanságát (kb. 50\% hamis pozitív eredmény a
  tünetmentes személyek körében) kimutató kínai tanulmányt időközben
  visszavontak. A tanulmány fő szerzője, egy egyetem orvosi karának
  dékánja, nem akarta megnevezni a visszavonás okát, és egy
  „\href{https://www.npr.org/sections/health-shots/2020/03/26/822084429/in-defense-of-coronavirus-testing-strategy-administration-cited-retracted-study}{kényes
  ügyről}`` beszélt, ami az NPR újságírója szerint politikai nyomásra
  utalhat. Az ún. PCR-vírustesztek nagy hibaaránya e tanulmánytól
  függetlenül már régóta ismert: 2006-ban egy kanadai ápolóotthonban
  tömeges SARS-koronavírus-fertőzést mutattak ki, amelyről később
  \href{https://www.ncbi.nlm.nih.gov/pmc/articles/PMC2095096/}{kiderült},
  hogy sima meghűléses koronavírusról van szó.
\item
  A német Risk Management Network (RiskNET) szerzői egy
  Covid19-\href{https://www.risknet.de/themen/risknews/covid-19-und-der-blindflug/}{elemzésben}
  „vakrepülésről``, valamint „hiányos adatkompetenciáról és
  adatetikáról`` beszélnek. Az egyre több teszt és intézkedés helyett
  reprezentatív szúrópróbára lenne szükség. Kritikusan meg kell
  vizsgálni a bevezetett intézkedések „ésszerűségét és
  értelemszerűségét``.
\item
  A nemzetközileg elismert argentin--francia virológussal, Pablo
  Goldschmidttel spanyolul készített interjút
  \href{https://www.rubikon.news/artikel/der-corona-totalitarismus}{lefordították
  németre}. Goldschmidt szerint a bevezetett intézkedések orvosi
  szempontból kontraproduktívak, és megjegyzi, hogy mostanában „Hannah
  Arendtet kell olvasni``, hogy megértsük „a totalitarizmus történelmi
  eredetét``.
\end{itemize}

\hypertarget{2020-muxe1rcius-30-i}{%
\paragraph{2020. március 30. (I.)}\label{2020-muxe1rcius-30-i}}

\begin{itemize}
\tightlist
\item
  Németországban néhány klinika már nem fogad több beteget. De nem
  azért, mert túl sok lenne a páciens vagy túl kevés ágy állna
  rendelkezésre, hanem mert
  \href{https://www.sueddeutsche.de/panorama/coronavirus-news-deutschland-wolfsburg-laschet-1.4828033}{az
  ápolók pozitív teszteredményt mutattak} -- pedig a legtöbb esetben
  tünetek nem is jelentkeztek. Ebből újfent kiderül, hogyan és mitől
  bénul meg az egészségügyi rendszer.
\item
  Egy előrehaladott demenciával élő embereket gondozó idős- és
  ápolóotthonban 15 pozitív teszteredményt mutató ember
  \href{https://www.sueddeutsche.de/panorama/coronavirus-news-deutschland-wolfsburg-laschet-1.4828033}{meghalt}:
  „Meglepően sokan haltak meg, anélkül hogy korona-tüneteket mutattak
  volna``. Egy német szakorvos ezt írja az ügyről: „Az én orvosi
  véleményem szerint néhány dolog szól amellett, hogy ezek kötül az
  emberek közül néhányan valószínűleg az intézkedések következményeibe
  haltak bele. A demens emberek súlyos stresszbe kerülnek, ha döntő
  változás áll be a mindennapjaikban: elszigeteltség, a testkontaktus
  hiánya, adott esetben »álruhába öltözött« ápolók.``
\item
  Svájcban a berni Inselspital
  \href{https://twitter.com/sneatio/status/1244157986832101376}{egy
  gyógyszerész szerint} a Covid19-től való félelemből
  kényszerszabadságra küldte az alkalmazottakat, terápiákat állított le
  és műtéteket halasztott el.
\item
  Gérard Krause professzor, a német járványügyi kutatóintézet, a
  Hermholtz Zentrum járványtani osztályvezetője a német ZDF csatornán
  attól óv, a korona-ellenes intézkedések
  „\href{https://www.zdf.de/nachrichten/politik/coronavirus-epidemiologe-folgen-helmholtz-100.html}{több
  halálesetet okozhatnak, mint maga a vírus}``.
\item
  Több médium arról számol be, hogy Olaszországban már több mint 40 halt
  meg „a korona-válság alatt``. A
  \href{https://portale.fnomceo.it/elenco-dei-medici-caduti-nel-corso-dellepidemia-di-covid-19/?ref=drnweb.repubblica.scroll-1}{megfelelő
  listára} vetett pillantás azonban azt mutatja, hogy az elhunyt orvosok
  túlnyomó többsége korábban különböző szakterületeken dolgozó, de már
  régóta nyugdíjban lévő orvos, köztük 90 éves pszichiáterek és
  gyermekorvosok, akiknek halálát jórészt természetes okok magyarázzák.
\item
  Egy
  \href{https://www.government.is/news/article/?newsid=c65cf658-6eb6-11ea-9462-005056bc4d74}{átfogó
  izlandi vizsgálat} megerősíti, hogy az összes pozitív teszteredményt
  mutató személy 50 százaléka „semmiféle tünetet`` nem mutat. A Covid19
  halálozási aránya az izlandi adatok szerint is ezrelékes nagyságrendű,
  azaz megegyezik az influenzáéval vagy az alatt van. A pozitív
  teszteredményt mutató két elhunytból az egyik ráadásul „szokatlan
  tüneteket produkáló turista`` volt``.
  (\href{https://www.covid.is/data}{izlandi adatok})
\item
  A brit Daily Mail újságírója, Peter Hitchens
  \href{https://hitchensblog.mailonsunday.co.uk/2020/03/theres-powerful-evidence-this-great-panic-is-foolish-yet-our-freedom-is-still-broken-and-our-economy.html}{arról
  ír}, hogy „világosan bebizonyosodott, hogy ez a nagy pánik ostobaság.
  Viszont szabadságunk még mindig korlátozva van, gazdaságunk megtört``.
  Hitchens arra hívja fel a figyelmet, hogy Nagy-Britannia egyes részein
  rendőrségi drónok
  \href{https://www.youtube.com/watch?v=fHNxDzLsPeg}{figyelik meg és
  jelentik} a „nem halaszthatatlan okból`` a természetben sétáló
  embereket. Az emberek egy részét a rendőrségi drónokról
  \href{https://www.youtube.com/watch?v=D4GEZjUTkqc}{hangszórón
  szólítják fel}, hogy menjenek haza, „hogy életet mentsenek``.
  (Megjegyzés: ennyire messze még George Orwell sem ment el
  gondolataiban.)
\item
  Az olasz titkosszolgálat társadalmi zavargásoktól és felkelésektől
  \href{https://www.focus.de/panorama/welt/sorge-vor-sozialen-unruhen-supermaerkte-gepluendert-apotheken-ueberfallen-italiens-geheimdienst-warnt-vor-aufstaenden_id_11826664.html}{tart}.
  Már volt szupermarket-fosztogatás, gyógyszertár elleni támadás.
\item
  Sucharit Bhakdi professzor egy
  \href{https://www.youtube.com/watch?v=LsExPrHCHbw\&feature=emb_title}{videót
  tett közzé} (német/angol nyelven), amelyben a Dr. Angela Merkel
  szövetségi kancellárnak írt
  \href{https://swprs.org/offener-brief-von-professor-sucharit-bhakdi-an-bundeskanzlerin-dr-angela-merkel/}{nyílt
  levelét} ismerteti.
\end{itemize}

\hypertarget{2020-muxe1rcius-30-ii}{%
\paragraph{2020. március 30. (II.)}\label{2020-muxe1rcius-30-ii}}

A Covid19-cel összefüggésben több országban egyre több jel utal arra,
hogy „a kezelés rosszabb, mint a betegség``.

Egy egyrészt az úgynevezett
\href{http://semmelweis.hu/korhazhigiene/altalanos/nozokomialis-fertozesek/}{nozokomiális
fertőzések} kockázatát jelenti, vagyis olyan fertőzéseket, amelyeket az
amúgy nem súlyos beteg a kórházban kap el. Európában évente 2,5 millió
nozokomiális fertőzés történik, amelyekhez 50 000 haláleset köthető. Még
a németországi intenzív állomásokon is a betegek mintegy 15 százaléka
fertőződik meg nozokomiális úton, a többi között mesterséges
lélegeztetés esetén tüdőgyulladással. Ezen felül külön problémát jelent
a kórházakban az antibiotikumnak ellenálló, multirezisztens baktériumok
elszaporodása.

További szempontot vetnek fel a bizonyára jó szándékkal, de részben
nagyon agresszívan alkalmazott kezelési módszerek, amelyeket egyre
gyakrabban alkalmaznak Covid19-betegeken. Ide számít a többi között
szteroidok, antibiotikumok és antivirális gyógyszerek (vagy ezek
kombinációjának) adása. Már SARS-1-betegek kezelésekor is
megmutatkozott, hogy egy ilyen kezelés
\href{https://www.sciencedaily.com/releases/2020/02/200206110703.htm}{gyakran
rosszabb és halálosabb} eredményre vezet, mint az ilyen kezelés nem
alkalmazása.

\hypertarget{2020-muxe1rcius-29}{%
\paragraph{2020. március 29.}\label{2020-muxe1rcius-29}}

\begin{itemize}
\tightlist
\item
  Dr. Sucharit Bhakdi (orvosi mikrobiológus professor emeritus, Mainzi
  Egyetem) 2020. március 26-án, csütörtökön
  \href{https://swprs.org/offener-brief-von-professor-sucharit-bhakdi-an-bundeskanzlerin-dr-angela-merkel/}{nyílt
  levelet írt Dr. Angela Merkel német szövetségi kancellárnak}, amelyben
  a Covid19-re adott reakció újraértékelését sürgeti és öt eldöntendő
  kérdést intéz a kancellárhoz.
  (\href{https://swprs.org/open-letter-from-professor-sucharit-bhakdi-to-german-chancellor-dr-angela-merkel/}{angol
  változat})
\item
  A
  \href{https://multipolar-magazin.de/artikel/coronavirus-irrefuhrung-fallzahlen}{Robert
  Koch Intézet legfrissebb adatai} szerint a pozitív teszteredményt
  mutató személyek száma a tesztek számának emelkedésével arányosan nő,
  azaz százalékosan nagyjából változatlan marad. Ez annak lehet a jele,
  hogy az esetszám emelkedésének oka a tesztek számának emelkedése, és
  nem a járvány terjedése.
\item
  Maria Rita Gismondo milánói mikrobiológus
  \href{https://www.secoloditalia.it/2020/03/coronavirus-la-gismondo-ammonisce-duramente-basta-snocciolare-numeri-sui-positivi-sono-dati-falsati/}{arra
  szólítja fel az olasz kormányt}, hogy a továbbiakban ne kommunikálja
  naponta a „korona-pozitív`` személyek számát, mivel ezek a számok
  „hamisak`` és szükségtelenül váltanak ki pánikot a lakosságban. A
  pozitív teszteredményt mutató személyek száma nagy mértékben függ a
  tesztek fajtájától és számától, és semmit nem mond e személyek
  egészségi állapotáról.
\item
  A Stanford Egyetem járványtani és orvos-professzora Dr. John Ioannidis
  \href{https://www.youtube.com/watch?v=d6MZy-2fcBw}{egyórás interjút}
  adott a Covid19-intézkedések alapjául szolgáló adatok hiányáról.
\item
  Pablo Goldschmidt Franciaországban élő, argentin virológus „teljesen
  eltúlzottnak`` tartja a Covid19-re adott politikai reakciót, és
  „\href{https://www.infobae.com/coronavirus/2020/03/28/para-un-prestigioso-cientifico-argentino-el-coronavirus-no-merece-que-el-planeta-este-en-un-estado-de-parate-total/}{totalitárius
  intézkedésektől}`` óv. Franciaországban egyes helyeken már drónokkal
  figyelik az emberek mozgását.
\item
  Az 1934-ben született olasz publicista, Fulvio Grimaldi azt mondja,
  hogy az Olaszországban jelenleg bevezetett állami intézkedések
  „\href{https://www.youtube.com/watch?v=O3BuNp01vpc}{rosszabbak, mint a
  fasizmus idején}``. A parlament és a társadalom teljesen meg van
  fosztva minden hatalmától.
\end{itemize}

\hypertarget{2020-muxe1rcius-28}{%
\paragraph{2020. március 28.}\label{2020-muxe1rcius-28}}

\begin{itemize}
\tightlist
\item
  Az
  \href{https://news.yahoo.com/oxford-study-suggests-millions-people-221100162.html}{Oxfordi
  Egyetem friss tanulmánya} arra az eredményre jut, hogy a Covid19
  vélhetőleg már 2020. január óta jelen volt Nagy-Britanniában, és
  időközben a lakosság fele már megfertőződött és úgy vált immúnissá,
  hogy a legtöbben csak nagyon enyhe vagy semmilyen tünetet sem
  tapasztaltak. Ez azt jelenti, hogy ezer emberből csak egy kerül
  kórházba, ami viszonylag alacsony érték.
  (\href{https://www.medrxiv.org/content/10.1101/2020.03.24.20042291v1}{tanulmány})
\item
  Brit médiumok
  \href{https://www.bbc.com/news/uk-england-beds-bucks-herts-52041709}{beszámoltak}
  egy 21 éves nőről, aki nem szenvedett más, meglévő betegségben, és
  Covid19-ben meghalt. Azóta
  \href{https://archive.is/20200329015127/https://www.theguardian.com/world/2020/mar/27/chloe-middleton-death-21-year-old-not-recorded-nhs-covid-19-related}{viszont
  kiderült}, hogy nem is volt pozitív a Covid19-tesztje, és halálának
  más volt az oka, valószínűleg öngyilkosság. A Covid19-pletyka azért
  kelt lábra, „mert egy picit köhögött``.
\item
  Otfried Jarren német médiatudós kritikája szerint sok médium
  \href{https://www.deutschlandfunk.de/covid-19-scharfe-kritik-an-ard-und-zdf-wegen.2849.de.html?drn:news_id=1114517}{kritika
  nélküli újságírást} folytat, ami fenyegetést jelent és a végrehajtó
  hatalmat hozza helyzetbe. Differenciálás és a szakértők közötti valódi
  vita alig szerepel a médiában.
\end{itemize}

\hypertarget{2020-muxe1rcius-27-i}{%
\paragraph{2020. március 27. (I.)}\label{2020-muxe1rcius-27-i}}

\textbf{Olaszország}: Az olasz egészségügyi minisztérium
\href{http://www.salute.gov.it/portale/caldo/SISMG_sintesi_ULTIMO.pdf}{legfrissebb
adatai} szerint (március 14.) a halandóság a 65 év fölötti összes
korcsoportban jelentős mértékben nőtt, azt követően, hogy előtte az
enyhe télnek köszönhetően még elmaradt az átlagtól. A március 14-ig mért
teljes halandóság még ugyan a 2016/2017. évi influenza-szezonban mért
érték alatt van, de lehet, hogy időközben már afölé emelkedett. A
szokásos mértéken felüli halandóság nagy része Észak-Olaszországban
jelentkezik. Mindazonáltal még nem világos, hogy ez milyen arányban
tulajdonítható a Covid19-nek, és milyen arányban az olyan tényezőknek,
mint a pánik, a rendszer összeomlása vagy maga a zárlat.

\includegraphics{https://swprs.files.wordpress.com/2020/03/italia-mortalita-marzo-14.png?w=600\&h=343}

\textbf{Franciaország}: Franciaországban a
\href{https://www.santepubliquefrance.fr/maladies-et-traumatismes/maladies-et-infections-respiratoires/infection-a-coronavirus/documents/bulletin-national/covid-19-point-epidemiologique-du-24-mars-2020}{legfrissebb
adatok} szerint az országos halandóság az enyhe influenza-szezon után
továbbra is a normál tartományban van. Egyes megyékben, főleg
Északkelet-Franciaországban azonban a 65 év fölötti korcsoportokban a
Covid19-cel összefüggésben már jelentős mértékben nőtt (lásd ábra).

\includegraphics{https://swprs.files.wordpress.com/2020/03/france-mortality.png?w=650\&h=400}

Franciaország ezen kívül
\href{https://www.santepubliquefrance.fr/maladies-et-traumatismes/maladies-et-infections-respiratoires/infection-a-coronavirus/documents/bulletin-national/covid-19-point-epidemiologique-du-24-mars-2020}{részletes
adatokat} is közöl az intenzív osztályon kezelt betegek és az elhunytak
kor szerinti összetételéről és korábbi, meglévő betegségeiről (lásd
lenti ábra):

\begin{itemize}
\tightlist
\item
  Az elhunytak átlagos életkora 81,2 év.
\item
  Az elhunytak 78 százaléka 75 évnél, 93 százaléka 65 évnél idősebb
  volt.
\item
  Az elhunytak 2,4 százaléka volt 65 év alatti és nem szenvedett
  (ismert) meglévő betegségben.
\item
  Az intenzív osztályon kezelt betegek átlagos életkora 65 év.
\item
  Az intenzív osztályon kezelt betegek 26 százaléka 75 év fölötti, 67
  százaléka meglévő betegségben szenved.
\item
  Az intenzív osztályon kezelt betegek 17 százaléka 65 év alatti és nem
  szenved meglévő betegségben.
\end{itemize}

A francia hatóságok ezt azzal egészítik ki, hogy a (Covid19-) járvány
halálozáson belüli arányát még meg kell határozni``.

\includegraphics{https://swprs.files.wordpress.com/2020/03/france-age-distribution-march-24.png?w=736}

\textbf{USA}: Stephen McIntyre kutató
\href{https://twitter.com/ClimateAudit/status/1243019315462516736}{kiértékelte}
az Egyesült Államok-beli hivatalos haláleseti és tüdőgyulladás-adatokat.
Ezek száma általában heti 3000 és 5500 között, azaz jóval a Covid19
aktuális számai fölött van. Az Egyesült Államokban a természetes
halandóság heti 50 000--60 000 fő. (Megjegyzés: a legfrissebb, 2020.
márciusi számok még nincsenek teljesen feldolgozva, ezért törik meg a
görbe lefelé.)

\includegraphics{https://swprs.files.wordpress.com/2020/03/us-pneumonia-deaths.png?w=400\&h=360}

\textbf{Nagy-Britannia}:

\begin{itemize}
\tightlist
\item
  Neil Ferguson (Imperial College London)
  \href{https://www.newscientist.com/article/2238578-uk-has-enough-intensive-care-units-for-coronavirus-expert-predicts/}{immár
  azt feltételezi}, hogy Nagy-Britannia elegendő
  intenzívállomás-kapacitással rendelkezik a Covid19-páciensek
  kezelésére.
\item
  John Lee patológus professor emeritus azzal
  \href{https://www.spectator.co.uk/article/The-evidence-on-Covid-19-is-not-as-clear-as-we-think}{érvel},
  hogy a Covid19-esetek kimutatásában a normál influenzás és meghűléses
  esetek kimutatásához képest tapasztalható sajátosságoknak az a
  következménye, hogy túlbecsülik a Covid19 jelentette kockázatot.
\end{itemize}

\textbf{További jelentések:}

\begin{itemize}
\tightlist
\item
  A Stanford Egyetem kutatóinak
  \href{https://medium.com/@nigam/higher-co-infection-rates-in-covid19-b24965088333}{vizsgálata}
  rámutatott, hogy a Covid19-pozitív betegek 20--25 százaléka más
  influenza- vagy meghűlés-teszten is pozitív eredményt mutatott.
\item
  Az Egyesült Államokban a munkanélküliségi biztosítási kérelmek száma
  \href{https://www.businessinsider.com/us-weekly-jobless-claims-record-coronavirus-unemployment-insurance-labor-recession-2020-3}{több
  3 millióra}, rekordmagasságba ugrott. Ebben az összefüggésben az
  \href{https://twitter.com/KoenSwinkels/status/1243066532390977544}{öngyilkosságok
  számának jelentős emelkedésére} is számítani lehet.
\item
  Az első pozitív tesztet mutató németországi beteg időközben
  meggyógyult. A 33 éves férfi állítása szerint a betegség
  „\href{https://www.br.de/nachrichten/bayern/coronavirus-patient-nummer-1-wie-ich-die-quarantaene-erlebte,Rrm4Ul8}{nem
  volt olyan rossz, mint az influenza}``.
\item
  Spanyol
  \href{https://elpais.com/sociedad/2020-03-25/los-test-rapidos-de-coronavirus-comprados-en-china-no-funcionan.html}{sajtójelentések
  szerint} a Covid19 antitest-gyorsteszt érzékenysége mindössze 30\%,
  pedig legalább 80 százalékosnak kellene lennie.
\item
  Egy 2003-as
  \href{https://ehjournal.biomedcentral.com/articles/10.1186/1476-069X-2-15}{kínai
  vizsgálat} arra a megállapításra jutott, hogy a SARS okozta halál
  valószínűsége 84 százalékkal nagyobb a mérsékelt légszennyezésnek
  kitett emberek körében a tiszta levegőjű területen élőkhöz képest. Az
  erősen szennyezett levegőjű területen élők esetében pedig 200
  százalékkal nagyobb ez a kockázat.
\item
  A Német Bizonyítékalapú Orvoslási Hálózat (EbM)
  \href{https://www.ebm-netzwerk.de/de/veroeffentlichungen/covid-19}{kritizálja
  a Covid19 megjelenítését a médiában}: „a média beszámolói egyáltalán
  nem veszik figyelembe a kockázatok bizonyítékalapú kommunikációjának
  általunk megkövetelt szempontjait. (\ldots{}) A nyers adatok közlése
  más halálozási okokkal összefüggésbe helyezés nélkül a kockázat
  túlbecsülésére vezet.``
\end{itemize}

\hypertarget{2020-muxe1rcius-27-ii}{%
\paragraph{2020. március 27. (II.)}\label{2020-muxe1rcius-27-ii}}

\begin{itemize}
\tightlist
\item
  Dr. Richard Capek német kutató
  \href{https://coronadaten.wordpress.com/}{egy mennyiségi elemzésben
  úgy érvel}, hogy a „korona-járvány`` valójában „a tesztek járványa``.
  Capek rámutat, hogy a tesztek száma exponenciálisan növekedett, a
  pozitív eredményt mutató személyeké viszont stabil maradt, a
  halandóság pedig csökkent, ami a vírus exponenciális terjedése ellen
  szól.
\item
  A Würzburgi Egyetem virológus professzora, Dr. Carsten Schneller
  \href{https://www.youtube.com/watch?v=w-uub0urNfw}{egy podcastban
  magyarázza el}, hogy a Covid19 igenis hasonlít az influenzához, sőt
  eddig kevesebb halálesetet is okozott. Schneller professzor úgy véli,
  hogy a médiában gyakran mutatott exponenciális görbék inkább a tesztek
  növekvő számával függenek össze, és nem a vírus szokatlanul gyors
  terjedésével. Németország számára kevésbé Olaszország a követendő
  példa, hanem Japán vagy Dél-Korea. Ez a két ország több millió kínai
  turista és csak minimális társadalmi korlátozások ellenére eddig nem
  került Covid19-válságba. Ennek egyik oka a maszk viselése lehet: a
  maszk a fertőzéstől ugyan nem véd, a vírus beteg emberek általi
  terjesztését azonban korlátozza.
\item
  A legfrissebb bergamói számok azt mutatják, hogy a abban a térségben a
  halandóság a jellemző havi 200--300 főről 2020. márciusban mintegy 900
  főre, azaz majdnem négyszeresére emelkedett. Az még nem világos, hogy
  ez milyen arányban tulajdonítható a Covid19-nek, és milyen arányban
  más, helyi tényezőknek vagy egyéb más okoknak (lásd fent).
\item
  A Stanford két orvosprofesszora, Dr. Eran Bendavid és Dr. Jay
  Bhattacharya egy
  \href{https://www.wsj.com/articles/is-the-coronavirus-as-deadly-as-they-say-11585088464}{cikkben}
  (paywall) azt írják, hogy a Covid19 halálossága nagyságrendekkel túl
  van becsülve, és hogy vélhetőleg magában Olaszországban is csak
  0,01--0,06\%, azaz elmarad az influenza súlyosságától. A túlbecsülés
  oka a már megfertőzött (és tünetmentes) személyek súlyosan alábecsült
  száma. Példaként az olasz Vo települést említik, ahol a teljes
  lakosságot tesztelték, és
  \href{https://www.repubblica.it/salute/medicina-e-ricerca/2020/03/16/news/coronavirus_studio_il_50-75_dei_casi_a_vo_sono_asintomatici_e_molto_contagiosi-251474302/}{a
  pozitív eredményt mutató személyek 50--75 százaléka tünetmentes} volt.
\item
  A német kórházi társaság elnöke, Dr. Gerald Gaß a
  \href{https://www.handelsblatt.com/politik/deutschland/coronakrise-deutsche-krankenhausgesellschaft-wir-sind-besser-vorbereitet-als-italien/25651268.html}{Handelsblattnak
  adott interjújában} elmondta, hogy „az olaszországi extrém helyzet oka
  mindenekelőtt a nagyon szűkös intenzívosztály-kapacitás``.
\item
  A Covid19-helyzet egyik
  \href{https://www.youtube.com/watch?v=p_AyuhbnPOI}{korai kritikusát},
  Dr. Wolfgang Wodargot átmenetileg
  \href{https://www.transparency.de/aktuelles/detail/article/in-eigener-sache-vorstand-beschliesst-ruhen-der-mitgliedschaft-von-wolfgang-wodarg-1/}{kizárták}
  a Transparency International németországi szervezetének elnökségéből,
  amelyben az egészségügyi munkacsoport vezetője volt. Megfogalmazott
  kritikája miatt a média hevesen
  \href{https://www.youtube.com/watch?v=xcirqmhBCvk}{támadta}.
\item
  Edward Snowden NSA-szivárogtató arra
  \href{https://www.futurezone.de/digital-life/article228779795/Gefaehrliche-weltweite-Entwicklung-Edward-Snowden-warnt-vor-Ueberwachung.html}{figyelmeztet},
  hogy a kormányok a megfigyelőállam kiépítésére és az alapjogok
  korlátozására használják ki a jelenlegi helyzetet. A most bevezetett
  ellenőrző intézkedéseket a válság elmúltával nem fogják megszüntetni.
\end{itemize}

\includegraphics{https://swprs.files.wordpress.com/2020/03/anzahl-infizierte-und-tests-2603.jpg?w=600\&h=339}

\hypertarget{2020-muxe1rcius-26-i}{%
\subparagraph{\texorpdfstring{\textbf{2020. március 26.
(I.)}}{2020. március 26. (I.)}}\label{2020-muxe1rcius-26-i}}

\begin{itemize}
\tightlist
\item
  \textbf{USA}: Az Egyesült Államokból származó
  \href{https://healthweather.us/}{legfrissebb, március 25-i adatok}
  szerint az egész országban csökken az influenza-szerű megbetegedések
  száma, sőt ezek gyakorisága jóval a sokéves átlag alatt van. Ennek
  okai közül a kormányzati intézkedések kizárhatók, hiszen még egy hete
  sincsenek hatályban.
\end{itemize}

\href{https://swprs.org/covid-19-hinweis-ii/us-influenza-trend/}{}

\includegraphics{https://swprs.files.wordpress.com/2020/03/us-influenza-trend.png?w=404\&h=242}

US Influenza Trend (March 25, 2020)

\href{https://swprs.org/covid-19-hinweis-ii/us-illness-levels/}{}

\includegraphics{https://swprs.files.wordpress.com/2020/03/us-illness-levels.png?w=324\&h=242}

US Influenza Trend (March 25, 2020)

USA: Csökkenő számú influenza-szerű megbetegedések (2020. március 25.,
KINSA)

\begin{itemize}
\tightlist
\item
  \textbf{Németország}: A Robert Koch Intézet
  \href{https://influenza.rki.de/Wochenberichte/2019_2020/2020-12.pdf}{legfrissebb,
  március 24.-i influenza-jelentése} „az akut légúti megbetegedések
  országszerte csökkenő aktivitását`` dokumentálja: az influenza-szerű
  megbetegedések száma és az e miatti kórházi tartózkodások száma az
  előző évek átlaga alatt van, és jelenleg is tovább csökken. Az
  RKI-jelentés így folytatja: „Az orvoslátogatások növekvő számát
  (\ldots{}) jelenleg sem a lakosság körében terjedő influenza-vírus,
  sem a SARS-CoV-2 nem magyarázza.``
\end{itemize}

\href{https://swprs.org/covid-19-hinweis-ii/rki-atemwegserkrankungen-20-2-2020/}{}

\includegraphics{https://swprs.files.wordpress.com/2020/03/rki-atemwegserkrankungen-20-2-2020.png?w=327\&h=202}

Deutschland: Atemwegserkrankungen 2019/2020 ggü. Vorjahren

\href{https://swprs.org/covid-19-hinweis-ii/rki-kliniken-belegung/}{}

\includegraphics{https://swprs.files.wordpress.com/2020/03/rki-kliniken-belegung.png?w=401\&h=202}

Deutschland: Krankenhausaufenthalte durch Atemwegserkrankungen nach
Altersgruppen

Németország: Csökkenő számú influenza-szerű megbetegedések (2020.
március 20., RKI)

\begin{itemize}
\tightlist
\item
  \textbf{Olaszország}: Guilio Tarro elismert olasz virológus úgy
  \href{https://www.cybermednews.eu/index.php/it/health/70871-interview-to-the-virologist-giulio-tarro-the-death-rate-of-covid-19-is-less-than-1-as-confirmed-by-the-national-institute-of-allergy-and-infectious-diseases}{érvel},
  hogy a Covid19 halálozási aránya Olaszországban is 1\% alatt van, és
  ebből a szempontból az influenzához hasonlít. A magasabb értékek abból
  csakis adódnak, hogy nem tesznek különbséget a koronavírus
  következtében és a koronavírussal a szervezetükben elhunytak között,
  és hogy a (tünetmentes) fertőzöttek számát súlyosan alulbecsülték.
\item
  \textbf{Egyesült Királyság}: Az akár 500 000 halálesetet előrejelző
  brit Imperial College tanulmányának szerzői ismét lefelé módosították
  előrejelzéseiket. Miután már
  \href{https://www.bbc.com/news/health-51979654}{elismerték}, a pozitív
  teszteredményt mutató elhunytak nagy része a normális halandóság
  számlájára írandó, most kijelentik, hogy a betegségek száma
  \href{https://www.thetimes.co.uk/article/nhs-now-likely-to-cope-with-coronavirus-says-key-scientist-rn5m6nggk}{két--három
  hét múlva fog tetőzni}.
\item
  \textbf{Egyesült Királyság}: A Guardian
  \href{https://www.theguardian.com/society/2019/feb/20/britons-urged-to-get-flu-vaccine-as-critical-cases-rise-above-2000}{2019.
  februárban} arról számolt be, hogy Nagy-Britanniában már a
  tulajdonképpen enyhe 2018/2019. évi influenza-szezonban is több mint
  2180-an kerültek influenza miatt intenzív osztályra.
\item
  \textbf{Svájc}: A Covid19 miatti, átlagos felüli halandóság Svájcban
  továbbra is nulla. A médiában
  \href{https://www.nau.ch/ort/basel/drei-weitere-covid-19-todesfalle-in-basel-stadt-65684099}{bemutatott}
  legutóbbi „halálos áldozat`` egy 100 éves nő. A svájci kormány ennek
  ellenére tovább szigorítja a korlátozó intézkedéseket.
\end{itemize}

\hypertarget{2020-muxe1rcius-26-ii}{%
\paragraph{2020. március 26. (II.)}\label{2020-muxe1rcius-26-ii}}

\begin{itemize}
\tightlist
\item
  \textbf{Svédország}: A Covid19 ügyében eddig Svédország követi a
  legliberálisabb stratégiát, amely
  \href{https://www.zeit.de/politik/ausland/2020-03/coronavirus-schweden-stockholm-oeffentliches-leben/komplettansicht}{két
  elven} nyugszik: a veszélyeztetett csoportok védelmén és az influenzás
  tüneteket mutató személyek otthon maradásán. „Ha ezt a két szabályt
  betartjuk, nincs szükség további intézkedésekre, amelyek hatása amúgy
  is csak marginális``, mondja Anders Tegnell főepidemiológus. A
  társadalmi és gazdasági élet normál módon megy tovább. Eddig a
  kórházakat sem rohamozták meg.
\item
  Dr. Jessica Hamed német büntető- és alkotmányjogász
  \href{https://www.fr.de/politik/coronakrise-deutschland-sind-kontaktsperren-ausgangsbeschraenkungen-rechtswidrig-13611821.html}{érvelése
  szerint} az olyan intézkedések, mint az általános kijárási tilalom és
  a társas érintkezés tiltása súlyos és aránytalan beavatkozást
  jelentenek az alapvető szabadságjogokba, és ezáltal vélhetőleg
  „összességében jogellenesek``.
\item
  A \href{https://www.euromomo.eu/index.html}{teljes európai
  halandóságot figyelő, legfrissebb, március 26-i monitoring-jelentés}
  az összes országban és az összes korcsoportban továbbra is normál vagy
  átlag alatti értékeket mutat, mindössze egy
  \href{https://www.euromomo.eu/outputs/zscore_country65.html}{kivétellel}:
  Olaszországban a 65 év fölöttiek körében jeleznek előre magasabb
  halandóságot (ún. delay-adjusted z-score), de még ez is a 2016/2017.
  és 2017/2018. évi influenza-járvány értékei alatt van.
\end{itemize}

\hypertarget{2020-muxe1rcius-25}{%
\paragraph{2020. március 25.}\label{2020-muxe1rcius-25}}

\begin{itemize}
\tightlist
\item
  Stefan Hockertz német immunológus és toxikológus egy interjúban
  \href{https://www.youtube.com/watch?v=7wfb-B0BWmo}{kifejti}, hogy a
  Covid-19 nem veszélyesebb, mint az influenza, csak sokkal pontosabban
  megfigyelik. A vírusnál veszélyesebb a média által kiváltott
  \href{https://corona.rs2.de/blog/interview/das-virus-macht-uns-nicht-krank/}{félelem
  és pánik}, valamint sok kormány „autoriter reakciója``. Hockertz
  professzor emellett hangsúlyozza, hogy az állítólagos
  „korona-halottak`` közül sokan valójában más betegségekbe haltak bele,
  és ezen kívül mutattak pozitív koronavírus-teszteredményt. Hockertz
  úgy véli, a beszámolókban említett adatoknál akár tízszer többen is
  megfertőződtek már Covid-19-cel, és alig vettek észre valamit belőle.
\item
  Pablo Goldschmidt argentin virológus és biokémikus
  \href{https://www.clarin.com/buena-vida/coronavirus-panico-injustificado-dice-virologo-argentino-francia_0_yVcmJ4RM.html}{kifejti},
  hogy a Covid19 nem veszélyesebb egy súlyosabb meghűlésnél vagy az
  influenzánál. Sőt, az is lehetséges, hogy a Covid19-kórokozó már
  korábbi években is itt volt, csak még nem fedezték fel, mert nem
  keresték. Dr. Goldschmidt „globális terrorról`` beszél, amelyet a
  média és a politika hozott létre. Minden évben világszerte hárommillió
  újszülött és csak az Egyesült Államokban 50 000 felnőtt hal meg
  tüdőgyulladásban.
\item
  Martin Exner professzor, a Bonni Egyetem Higiéniai Intézetének
  vezetője a
  \href{https://www.youtube.com/watch?v=9mI9trSm3PY}{phoenix-nek adott
  interjúban} elmagyarázza, hogy jelenleg miért vannak annak ellenére
  nyomás alatt az egészségügyi dolgozók, hogy Németországban eddig alig
  emelkedett a betegek száma: egyrészt a pozitív teszteredményt mutató
  orvosoknak és ápolóknak vesztegzárba kell vonulniuk és gyakran alig
  pótolhatók, másrészt a szomszédos országokból származó ápolók, akik az
  ellátás jelentős részét végzik, a lezárt határok miatt nem utazhatnak
  be Németországba.
\item
  A korábbi német kulturális államminiszter és etika-professzor, Julian
  Nida-Rümelin arra
  \href{https://www.zdf.de/nachrichten/zdf-morgenmagazin/julian-nida-ruemelin-zur-corona-krise-100.html}{hívja
  fel a figyelmet}, hogy a Covid19 az egészséges lakosság számára nem
  jelent kockázatot, ezért az olyan szélsőséges intézkedéseknek, mint a
  kijárási tilalom, nincs jogalapja.
\item
  A Stanford Egyetem professzora, John Ioannidis a Diamond Princess
  óceánjáró adatai alapján
  \href{https://www.statnews.com/2020/03/17/a-fiasco-in-the-making-as-the-coronavirus-pandemic-takes-hold-we-are-making-decisions-without-reliable-data/}{rámutat},
  hogy a Covid19 életkorral korrigált halálozási aránya 0,025--0,625\%,
  azaz akkora, mint a súlyos meghűlésé vagy az influenzáé. Egy
  \href{https://www.niid.go.jp/niid/en/2019-ncov-e/9407-covid-dp-fe-01.html}{japán
  tanulmány} pedig azt közli, hogy az összes pozitív teszteredményt
  mutató utas 48 százaléka a magas átlagéletkor ellenére teljesen
  tünetmentes maradt; még a 80--89 év közöttiek 48 százaléka is
  tünetmentes maradt, sőt, a 70--79 évesek 60 százaléka nem mutatott
  semmilyen tünetet. Úgyhogy ismét felvetődik a kérdés, hogy a korábbi,
  meglévő betegségek nem döntőbb kockázati tényezők-e, mint maga a
  vírus. Olaszország példája azt mutatja, hogy
  \href{https://www.bloomberg.com/news/articles/2020-03-18/99-of-those-who-died-from-virus-had-other-illness-italy-says}{a
  pozitív teszteredményt mutató elhunytak 99 százaléka} egy vagy több
  meglévő betegségben szenvedett, és a ő esetükben is csupán a halotti
  bizonyítványok 12 százalékán szerepel a halál okaként a Covid19.\\
\end{itemize}

\hypertarget{2020-muxe1rcius-24}{%
\paragraph{2020. március 24.}\label{2020-muxe1rcius-24}}

\begin{itemize}
\tightlist
\item
  Az Egyesült Királyság eltávolította a Covid19-et a magas szintű
  fertőző betegségek (HCID) hivatalos listájáról, kijelentve, hogy a
  halálozási
  arány\href{https://www.gov.uk/guidance/high-consequence-infectious-diseases-hcid\#status-of-covid-19}{„összességében
  alacsony''}.
\item
  A Német Nemzeti Egészségügyi Intézet (RKI)
  \href{https://swprs.org/rki-relativiert-corona-todesfaelle/}{igazgatója
  elismerte}, hogy az összes teszt-pozitív halálesetet „koronavírusos
  halálozásnak'' számítják, függetlenül a halál tényleges okától. Az
  elhunyt átlagéletkora 82 év, a legtöbb komoly előfeltételekkel. Mint a
  legtöbb más országban, a Covid19 miatti túlzott halálozás valószínűleg
  nulla közelében van Németországban.
\item
  A Covid19 betegek számára fenntartott svájci intenzív osztályok ágyai
  továbbra is
  \href{https://www.aargauerzeitung.ch/aargau/kanton-aargau/erst-3-von-100-aargauer-betten-der-intensivstationen-sind-belegt-so-ruesten-sich-die-spitaeler-auf-die-epidemie-137332716}{„többnyire
  üresek''.}
\item
  A német professzor, Karin Moelling, a Zürichi Egyetem korábbi orvosi
  virológiai tanszékén egy
  \href{https://www.radioeins.de/programm/sendungen/die_profis/archivierte_sendungen/beitraege/corona-virus-kein-killervirus.html}{interjúban
  kijelentette}, hogy a Covid19 „nem gyilkos vírus'', és hogy a
  „pániknak véget kell érnie''.
\end{itemize}

\hypertarget{2020-muxe1rcius-23-i}{%
\paragraph{2020. március 23. (I.)}\label{2020-muxe1rcius-23-i}}

\begin{itemize}
\tightlist
\item
  Egy friss francia tanulmány
  (\href{https://www.sciencedirect.com/science/article/abs/pii/S0924857920300972}{SARS-CoV-2:
  Félelem kontra adatok}) arra az eredményre jut, hogy „a SARS-CoV-2
  okozta probléma vélhetőleg túl van becsülve``, mivel „a SARS-CoV-2
  halálozási aránya nem különbözik lényeges mértékben egy franciaországi
  kórházban vizsgált szokásos koronavírusokétól
  (meghűlés-vírusokétól).``
\item
  Egy
  \href{https://www.ijidonline.com/article/S1201-9712(19)30328-5/fulltext}{2019.
  augusztusban készült olasz tanulmány} azt állapította meg, hogy az
  elmúlt években Olaszországban évente 7000--25 000 ember halt meg
  influenzában. Ez az érték Olaszország idősebb lakossága miatt
  magasabb, mint a többi európai országban, és sokkal magasabb, mint
  minden, amit eddig a Covid-19-cel összefüggésbe hoztak.
\item
  A WHO
  \href{https://www.who.int/news-room/q-a-detail/q-a-similarities-and-differences-covid-19-and-influenza}{friss
  tájékoztatójában} azt írja, hogy a Covid-19 az eddigi ismeretek
  szerint (kb. 50 százalékkal) lassabban terjed, mint az influenza, és
  hogy a tünetek jelentkezése előtti terjedése jóval kisebb mértékű,
  mint az influenzáé.
\item
  Egy olasz főorvos már 2019. novemberben
  „\href{https://www.scmp.com/news/china/society/article/3076334/coronavirus-strange-pneumonia-seen-lombardy-november-leading}{furcsa
  tüdőgyulladásokról}`` számolt be Lombardiában. Ez újra felveti a
  kérdést, hogy ezért az új vírus a felelős-e (amely hivatalosan csak
  2020. februárban bukkant fel Olaszországban), vagy más tényezők, mint
  például Észak-Olaszország
  \href{https://www.thelocal.it/20170131/our-lungs-are-breaking-smog-levels-way-above-safe-limits-in-northern-italy}{súlyosan
  szennyezett levegője}.
\item
  A nagy tekintélyű Cochrane Collaboration alapítója, Peter Gøtzsche dán
  kutató arról ír, hogy a korona-járvány
  „\href{https://www.deadlymedicines.dk/corona-an-epidemic-of-mass-panic/}{a
  pánik járványa}``, amelynek „egyik első áldozata a logika`` volt.
\end{itemize}

\hypertarget{2020-muxe1rcius-23-ii}{%
\paragraph{2020. március 23. (II.)}\label{2020-muxe1rcius-23-ii}}

\begin{itemize}
\tightlist
\item
  A korábbi izraeli egészségügyi miniszter, Yoram Lass professzor
  szerint az új koronavírus kevésbé veszélyes, mint az influenza, és a
  kijárási tilalom
  \href{https://en.globes.co.il/en/article-lockdown-lunacy-1001322696}{több
  emberéletet követel, mint a vírus}. „A számok nem indokolják a
  pánikot``, mondja Lass. Ismeretes, hogy „Olaszországban a légúti
  megbetegedések okozta halandóság rendkívüli, több mint háromszor
  akkora, mint Európa többi részén.``
\item
  A fertőző betegségek egyik svájci specialistája, Pietro Vernazza
  szerint a meghozott intézkedések
  \href{https://www.tagblatt.ch/leben/ostschweizer-infektiologe-pietro-vernazza-die-zahlen-zu-den-jungen-corona-virus-erkrankten-sind-irrefuehrend-ld.1206440}{tudományosan
  nem indokoltak}, újra kell gondolni őket. Vernazza szerint a tömeges
  tesztelésnek semmi értelme, mivel a lakosság 90 százaléka tünetmentes
  marad, a kijárási tilalom és az iskolák bezárása pedig egyenesen
  „kontraproduktív``. Vernazza azt javasolja, hogy csak a
  veszélyeztetett csoportokat védjék, a korlátozásokat pedig vonják
  vissza.
\item
  A Nemzetközi Orvostársaság elnöke, Frank Ulrich Montgomery szintén
  azon a véleményen van, hogy az olaszországi kijárási tilalom
  „\href{https://www.general-anzeiger-bonn.de/news/politik/deutschland/interview-mit-weltaerztepraesident-montgomery-ueber-corona-pandemie-ist-chaos_aid-49609561}{észszerűtlen
  és kontraproduktív}``.
\item
  Svájc: a szokásos arányokat meghaladó halandóság a médiaizgalom
  ellenére továbbra is nulla, nulla közeli: legutóbbi két, pozitív
  teszteredményt mutató
  „\href{https://www.bluewin.ch/de/newsregional/zuerich/1068-bestatigte-corona-falle-und-funf-todesfalle-im-kanton-zurich-371873.html}{halálos
  áldozat}`` egy 96 éves, palliatív kezelésben lévő férfi, és egy 97
  éves, több betegségben is szenvedő férfi volt.
\item
  Az ISS legfrissebb statisztikai jelentése immár
  \href{https://www.epicentro.iss.it/coronavirus/bollettino/Report-COVID-2019_20_marzo_eng.pdf}{angolul
  is elérhető}.
\end{itemize}

\hypertarget{2020-muxe1rcius-22-i}{%
\paragraph{2020. március 22. (I.)}\label{2020-muxe1rcius-22-i}}

\textbf{Az olaszországi helyzetről:} a legtöbb médium inkorrekt módon
közli, hogy Olaszországban naponta akár 800 ember is meghal a
koronavírus miatt. Az olasz polgári védelem elnöke azt hangsúlyozza,
hogy valójában olyan halálesetekről van szó, amelyekben az elhunytak
szervezetében benne van a koronavírus, és nem arról, hogy abba haltak
volna bele (\href{https://youtu.be/0M4kbPDHGR0?t=210}{sajtótájékoztató},
03:30-tól). Más szóval ezek a személyek meghaltak, és pozitív
teszteredményt is mutattak.

Ahogy mindkét professzor, Ioannidis és Bhakdi is
\href{https://www.statnews.com/2020/03/17/a-fiasco-in-the-making-as-the-coronavirus-pandemic-takes-hold-we-are-making-decisions-without-reliable-data/}{rámutat},
azokban az országokban, mint Dél-Korea és Japán, amelyek nem hoztak
zárlati intézkedéseket, a Covid-19-cel összefüggő, átlagon felüli
halandóság szinte nulla, míg a Diamond Princess óceánjáró hajón mért
halálozási arány ezrelék nagyságrendű, vagyis megegyezik a szezonális
influenza vagy egy erősebb meghűlés halálozási arányával, vagy valamivel
az alatt van.

A pozitív teszteredményt mutató aktuális olaszországi halálozási számok
még mindig a normál országos napi halandóság -- mintegy 1800 haláleset
naponta -- 50 százaléka alatt vannak. Így lehetséges, sőt talán
valószínű, hogy a normál napi halálozások jó részét most egyszerűen
„Covid19-halálesetként`` rögzítik (mivel a teszt eredménye pozitív
volt). Ez az a dolog, amelyet az olasz polgári védelem elnöke
hangsúlyozott.

Időközben azonban kiderült, hogy Észak-Olaszország bizonyos térségeiben,
azokban, amelyekben a
\href{https://en.wikipedia.org/wiki/2020_Italy_coronavirus_lockdown}{legkeményebb
zárlati intézkedéseket} hozták, jelentős mértékben megnövekedett a napi
halálozások száma. Az is ismert, hogy Lombardiában a pozitív
teszteredményt mutató elhunytak 90 százaléka nem az intenzív osztályon
halt meg, hanem
\href{https://www.tgcom24.mediaset.it/cronaca/coronavirus-in-lombardia-9-morti-su-10-mai-giunti-in-terapia-intensiva_16362350-202002a.shtml}{túlnyomórészt
otthon}. És ezeknek az elhunytaknak a több mint 99 százaléka más,
meglévő, súlyos betegségben szenvedett (pl. szívproblémák, légzőszervi
problémák, rák).

Sucharit Bhakdi professzor
\href{https://www.youtube.com/watch?v=JBB9bA-gXL4}{szerint} a zárlati
intézkedések „groteszkek, értelmetlenek, önpusztítók``, „kollektív
öngyilkosság``. Ezért felvetődik az a rendkívül nyugtalanító kérdés,
hogy az ezeknek az idősebb, elszigetelt, súlyosan stresszelt, több
betegségben is szenvedő embereknek a körében tapasztalható nagyobb
halandóságot mennyiben okozhatták a több hete bevezetett és még mindig
érvényben lévő zárlati korlátozások.

Akkor ez egy újabb olyan eset lenne, amikor a kezelés rosszabb, mint a
betegség. (Lásd frissítés lent: csak a halálesetek 12 százalékában adják
meg a koronavírust okként a halotti bizonyítványon).

\includegraphics{https://swprs.files.wordpress.com/2020/03/borrelli2.jpg?w=550\&h=309}

\hypertarget{2020-muxe1rcius-22-ii}{%
\paragraph{2020. március 22. (II.)}\label{2020-muxe1rcius-22-ii}}

\begin{itemize}
\tightlist
\item
  Svájcban eddig 56 pozitív teszteredményt mutató haláleset volt. Az
  elhunyt -- kora és/vagy meglévő betegségei miatt --
  \href{https://www.nzz.ch/schweiz/coronavirus-in-der-schweiz-die-neusten-entwicklungen-ld.1542664\#subtitle-wie-viele-infizierte-und-todesf-lle-gibt-es-second}{minden
  esetben} „kockázati csoportba tartozó beteg`` volt. A halál pontos
  okáról (azaz hogy a vírus miatt, vagy vírussal a szervezetben)
  továbbra sincsenek adatok.
\item
  A svájci kormány azt állította, hogy Svájc déli részén (közvetlenül
  Olaszország szomszédságában) „drámai``, de a helyi orvosok ennek
  \href{https://www.nzz.ch/schweiz/punkto-intensivbetten-sind-wir-im-tessin-besser-ausgeruestet-als-der-rest-der-schweiz-ld.1547728}{ellentmondtak}:
  szerintük minden normális mederben halad.
\item
  \href{https://www.blick.ch/news/schweiz/nicht-nur-beatmungsgeraete-werden-knapp-im-kampf-gegen-corona-es-droht-ein-engpass-beim-sauerstoff-id15808185.html}{Sajtójelentések}
  szerint ellátási nehézség fenyeget oxigénpalackból. Ennek oka azonban
  nem a megnövekedett igény, hanem a palackok felhalmozása, rejtegetése
  a szűkösségtől való félelemből.
\item
  Sok országban már most
  \href{https://www.washingtonpost.com/health/covid-19-hits-doctors-nurses-emts-threatening-health-system/2020/03/17/f21147e8-67aa-11ea-b313-df458622c2cc_story.html}{egyre
  nagyobb hiány} van orvosból és ápolóból. Ennek az a fő oka, hogy a
  pozitív teszteredményt mutató egészségügyi dolgozóknak vesztegzárba
  kell vonulniuk, pedig a legtöbb esetben tünetmentesek, vagy csak enyhe
  tüneteket mutatnak.
\end{itemize}

\hypertarget{2020-muxe1rcius-22-iii}{%
\paragraph{2020. március 22. (III.)}\label{2020-muxe1rcius-22-iii}}

\begin{itemize}
\tightlist
\item
  Az Imperial College London előrejelzése Nagy-Britanniára 250 000--500
  000 Covid-19 „miatt`` bekövetkező haláleset. A tanulmány szerzői
  azonban \href{https://www.bbc.com/news/health-51979654}{elismerték},
  hogy e halálesetek közül sok nem pluszban következik be, hanem a
  normál éves halandóság része, amelynek mértéke évente 600 000 fő.
\item
  A Yale University Prevention Research Center alapító igazgatója, Dr.
  David Katz a következő kérdést teszi fel a
  \href{https://www.nytimes.com/2020/03/20/opinion/coronavirus-pandemic-social-distancing.html}{New
  York Times}-ban: „A koronavírus elleni harcunk rosszabb mint a
  betegség? Vannak célzottabb eszközök a járvány legyőzésére.``
\item
  Walter Ricciardi olasz professzor szerint \textbf{„a halotti
  bizonyítványok mindössze 12 százalékán
  \href{https://web.archive.org/web/20200324214448/https://www.telegraph.co.uk/global-health/science-and-disease/have-many-coronavirus-patients-died-italy/}{szerepel}
  a koronavírus mint a halál oka``}, miközben a nyilvánosságban „a
  kórházakban bekövetkező és koronavírussal összefüggő összes
  halálesetet a koronavírus okozta halálozásként veszik számba``. A
  médiában elhangzó olaszországi halálozási számokat eszerint legalább
  nyolccal el kell osztani, ha meg akarjuk kapni a valóban a koronavírus
  okozta halálesetek számát. Ez pedig legfeljebb pár tucat halálozás
  naponta -- ezt kell összevetni a normál halandóság napi 1800
  halálesetével és az évente influenzában meghaló akár 20 000 emberrel.
\end{itemize}

\hypertarget{2020-muxe1rcius-21-i}{%
\paragraph{2020. március 21. (I.)}\label{2020-muxe1rcius-21-i}}

\begin{itemize}
\tightlist
\item
  Spanyolország eddig csak 3 olyan halálesetet jelentett be (kb. 1000
  fertőzöttből), amelyben az elhunyt
  \href{https://www.20minutos.es/noticia/4193883/0/media-edad-coronavirus-espana/}{65
  év alatti} személy volt. Az ő esetükben korábbi betegség és a halál
  tényleges oka eddig nem ismert.
\item
  Olaszország március 20-án országszerte egy nap alatt 627 pozitív
  teszteredményt mutató halálesetet
  \href{https://www.msn.com/en-au/news/coronavirus/italy-coronavirus-deaths-surge-by-627-in-a-day-lifting-total-death-toll-to-4032/ar-BB11tDnS}{jelentett}.
  A normális halálozási adat Olaszországban kb. 1800 fő naponta.
  Olaszország február 21. óta összesen kb. 4000 pozitív teszteredményt
  mutató halálesetet jelentett. Ugyanebben az időszakban a természetes
  halandóság a teljes olasz lakosság körében kb. 50 000 fő. Még nem
  világos, hogy mennyivel nőtt a teljes halandóság, vagy pedig
  egyszerűen csak pozitív teszteredményekről van-e szó. A 2019/2020. évi
  influenzaszezon egyébként Olaszországban és Európában is nagyon enyhe
  volt, ami sok, amúgy veszélyeztetett ember épségét megkímélte.
\item
  \href{https://www.tgcom24.mediaset.it/cronaca/coronavirus-in-lombardia-9-morti-su-10-mai-giunti-in-terapia-intensiva_16362350-202002a.shtml}{Olasz
  sajtójelentések} szerint Lombardiában eddig a pozitív teszteredményt
  mutató elhunytak mintegy 90 százaléka nem az intenzív osztályon halt
  meg, hanem túlnyomórészt otthon vagy valamelyik kórházi osztályon. A
  halál oka és a vesztegzár-intézkedések lehetséges szerepe még nincs
  tisztázva. A 2168 pozitív teszteredményt mutató elhunyt közül csupán
  260 halt meg intenzív osztályon.
\item
  A Bloomberg hírügynökség arról
  \href{https://www.bloomberg.com/news/articles/2020-03-18/99-of-those-who-died-from-virus-had-other-illness-italy-says}{számol
  be}, hogy az olaszországi halálesetek 99 százalékában az elhunyt más
  betegségben szenvedett.
\end{itemize}

\includegraphics{https://swprs.files.wordpress.com/2020/03/covid-iss-stat-bloomberg.png?w=550\&h=301}

\hypertarget{2020-muxe1rcius-21-ii}{%
\paragraph{2020. március 21. (II.)}\label{2020-muxe1rcius-21-ii}}

\begin{itemize}
\tightlist
\item
  A Japan Times felteszi a kérdést:
  \href{https://www.japantimes.co.jp/news/2020/03/20/national/coronavirus-explosion-expected-japan/}{Japán
  a coronavírus robbanásszerű terjedésére számított. Hol marad?} Bár
  Japán az elsők között volt, ahol pozitív teszteredmények születtek és
  nem hoztak zárlati intézkedéseket („lockdown``), eddig az egyik
  legkevésbé érintett ország. Nem nő a tüdőgyulladások száma, nem nő a
  kórházak kihasználtsága.
\item
  Olasz kutatók azzal
  \href{https://www.heise.de/tp/features/Feinstaubpartikel-als-Viren-Vehikel-4687454.html}{érvelnek},
  hogy az észak-olaszországi rendkívüli -- egész Európában a
  legsúlyosabb -- légszennyezettség lehet a tüdőgyulladások számának
  jelenlegi regionális emelkedésének egyik oka, hasonlóan ahhoz, ahogy
  az a kínai Vuhanban történt (lásd fent).
\item
  Az orvosi mikrobiológia egyik leggyakrabban idézett szakértője,
  Sucharit Bhakdi professzor egy
  \href{https://www.youtube.com/watch?v=JBB9bA-gXL4}{friss interjúban}
  azt magyarázza, hogy „hibás`` és „veszélyesen félrevezető`` az új
  koronavírust a halálozások fő felelősévé nyilvánítani, mivel a meglévő
  betegségek és a kínai és észak-olaszországi városokban jelen lévő
  légszennyezettség fontosabb szerepet játszik. Bhakdi professzor
  szerint a most bevezetendő vagy bevezetett intézkedések „groteszkek,
  értelmetlenek, önpusztítók``, „kollektív öngyilkosság``, amely
  lerövidíti az idősek várható élettartamát, és a társadalomnak ezeket
  nem lenne szabad elfogadnia.
\end{itemize}

\hypertarget{2020-muxe1rcius-20}{%
\paragraph{2020. március 20.}\label{2020-muxe1rcius-20}}

\begin{itemize}
\tightlist
\item
  A \href{https://www.euromomo.eu/index.html}{legfrissebb európai
  monitoring-jelentés} szerint az összesített halandóság az összes
  országban (ide értve Olaszországot is) és az összes korcsoportban
  eddig a normális tartományban vagy az alatt van.
\item
  A
  \href{https://de.wikipedia.org/wiki/COVID-19-Pandemie_in_Deutschland\#Todesf\%C3\%A4lle_in_den_Medien}{legfrissebb
  németországi számok} szerint a pozitív teszteredményt mutató elhunytak
  medián-életkora kb. 83 év, legtöbbjük idült betegségben szenvedett.
\item
  Egy a Stanford professzora, John Ioannidis által ismertetett,
  \href{https://www.ncbi.nlm.nih.gov/pmc/articles/PMC2095096/}{2006-ban
  készült kanadai tanulmány} egy idősek otthona példáján keresztül
  mutatja be, hogy a kockázati csoportokban a szokásos koronavírusok
  (meghűlésvírusok) is okozhatnak 6 százalékos halálozási arányt, és
  hogy a tesztkészletek eleinte hamis eredményeket,
  SARS-koronavírus-fertőzést mutattak ki.
\end{itemize}

\hypertarget{2020-muxe1rcius-19-i}{%
\paragraph{2020. március 19. (I.)}\label{2020-muxe1rcius-19-i}}

Az olasz ISS
\href{https://www.epicentro.iss.it/coronavirus/bollettino/Report-COVID-2019_17_marzo-v2.pdf}{új
jelentést} tett közzé a pozitív teszteredményt mutató elhunytakról:

\begin{itemize}
\tightlist
\item
  A medián-életkor 80,5 év (férfiak: 79,5 év, nők: 83,7 év).
\item
  Az elhunytak 10 százaléka 90 év fölötti, 90 százalékuk 70 év fölötti
  volt.
\item
  Az elhunytak legfeljebb 0,8 százaléka nem szenvedett semmilyen idült
  betegségben.
\item
  Az elhunytak kb. 75 százaléka kettő vagy több, kb. 50 százaléka három
  vagy több krónikus betegségben, a többi között szívbetegségben,
  cukorbetegségben, rákban szenvedett.
\item
  Öt elhunyt volt 31--39 éves, mindannyian súlyos betegséggel küzdöttek.
\item
  Az ISS továbbra sem közöl adatot arról, mibe haltak bele a vizsgált
  betegek, hanem általánosságban „Covid19-pozitív elhunytak``-ról
  beszél.
\end{itemize}

\hypertarget{2020-muxe1rcius-19-ii}{%
\paragraph{2020. március 19. (II.)}\label{2020-muxe1rcius-19-ii}}

\begin{itemize}
\tightlist
\item
  A
  \href{https://milano.corriere.it/notizie/cronaca/18_gennaio_10/milano-terapie-intensive-collasso-l-influenza-gia-48-malati-gravi-molte-operazioni-rinviate-c9dc43a6-f5d1-11e7-9b06-fe054c3be5b2.shtml}{Corriere
  della Sera} egyik tudósításában arról ír, hogy az olaszországi
  intenzív osztályok már a 2017/2018. évi súlyos influenzajárvány idején
  is összeomlottak, műtéteket kellett elhalasztani, ápolókat kellett
  szabadságról visszahívni.
\item
  Hendrik Streeck német virológus
  \href{https://www.faz.net/aktuell/gesellschaft/gesundheit/coronavirus/virologe-hendrik-streeck-ueber-corona-neue-symptome-entdeckt-16681450.html?printPagedArticle=true\#pageIndex_2}{egy
  interjúban úgy vélekedik}, hogy a Covid19 nem fogja növelni a
  németországi össz-halálozási számot, amely normális esetben mintegy
  2500 főt jelent naponta. Streeck egy 78 éves férfi példáját említi,
  aki idült betegségben szenvedett, szívelégtelenségben meghalt, majd
  utólag elvégzett Covid19-tesztje pozitív lett, ezért bekerült a
  Covid19-halálesetek statisztikájába.
\item
  A Stanford professzora, John P. A. Ioannidis szerint a most meghozott
  intézkedéseknek
  \href{https://www.statnews.com/2020/03/17/a-fiasco-in-the-making-as-the-coronavirus-pandemic-takes-hold-we-are-making-decisions-without-reliable-data/}{nincs
  elegendő orvosiadat-alapja}. Az új koronavírus valószínűleg még az
  idősebbekre sem veszélyesebb némely többi koronavírusnál.
\end{itemize}

\hypertarget{2020-muxe1rcius-18}{%
\paragraph{2020. március 18.}\label{2020-muxe1rcius-18}}

\begin{itemize}
\tightlist
\item
  Egy új (előzetesen publikált) járványtani tanulmány arra az eredményre
  jut, hogy a Covid19 halálozási aránya magában a kínai Vuhanban is
  mindössze 0,04--0,12\% volt, amely alacsonyabb mint a szezonális
  influenza halálozási aránya (kb. 0,1\%). A kutatók úgy vélik, a
  Covid19 nyilvánvalóan jelentősen túlbecsült halálozási arányának az az
  oka, hogy Vuhanban eredetileg csak az esetek kis részét rögzítették,
  mivel a betegség sok embernél tünetmentes vagy enyhe lefolyású volt.
\item
  Kínai kutatók azzal
  \href{https://www.eurasiareview.com/01022020-polluted-air-could-be-an-important-cause-of-wuhan-pneumonia-oped/}{érvelnek},
  hogy a Vuhanban kialakult rendkívüli téli szmog lehetett a
  kitörésszerűen jelentkező tüdőgyulladások egyik oka. A rossz
  levegőminőség miatt 2019 nyarán már voltak
  \href{https://www.cnn.com/2019/07/10/asia/china-wuhan-pollution-problems-intl-hnk/index.html}{nyilvános
  tiltakozások} Vuhanban.
\item
  Új műhold-felvételeken látható, hogy Európán belül
  Észak-Olaszországban a
  \href{https://twitter.com/esa/status/1238480433047916545}{legszennyezettebb
  a levegő}, és hogy ez a szennyezettség hogyan csökkent jelentős
  mértékben a vesztegzár hatására.
\item
  A Covid19-tesztkészlet egyik gyártója azt mondja, hogy ez a
  tesztkészlet
  \href{https://www.creative-diagnostics.com/sars-cov-2-coronavirus-multiplex-rt-qpcr-kit-277854-457.htm}{csak
  kutatási célokra} használható, diagnosztikai célokra nem, mivel
  klinikai validációja még nem történt meg.
\end{itemize}

\includegraphics{https://swprs.files.wordpress.com/2020/03/covid-testkit.png?w=550\&h=149}

\hypertarget{2020-muxe1rcius-17-i}{%
\paragraph{2020. március 17. (I.)}\label{2020-muxe1rcius-17-i}}

\begin{itemize}
\tightlist
\item
  Némely svájci sürgősségi osztály csak a
  \href{https://insideparadeplatz.ch/2020/03/16/notfall-stationen-bereits-seit-tagen-am-anschlag/}{magukon
  tesztet elvégeztetni akaró} személyek nagy száma miatt már most túl
  van terhelve. Ez az aktuális helyzet pszichológiai és logisztikai
  szempontjaira mutat rá.
\item
  A halálozási profil virológiai szempontból továbbra is rejtélyes,
  mivel az influenzavírustól eltérően ez a járvány megkíméli a
  gyermekeket, idős férfiakat pedig mintegy kétszer olyan gyakran érint,
  mint idős nőket. Másfelől ez a profil megfelel a
  \href{http://www.gbe-bund.de/gbe10/abrechnung.prc_abr_test_logon?p_uid=gast\&p_aid=0\&p_knoten=FID\&p_sprache=D\&p_suchstring=820}{természetes
  halandóságnak}, amely gyermekeknél közel nulla, 75 éves férfiak
  esetében pedig közel kétszer akkora, mint az azonos életkorú nők
  körében.
\item
  A pozitív teszteredményt mutató, fiatal elhunytak esetében továbbra is
  túlnyomórészt vagy szinte kizárólag nagyon súlyos betegségben szenvedő
  személyekről van szó. Ilyen az a 21 éves spanyol labdarúgás-edző,
  akinek a tesztje pozitív volt és
  \href{https://www.msn.com/de-ch/news/other/spanischer-nachwuchs-trainer-stirbt-an-corona/ar-BB11gT64}{meghalt}.
  Az orvosok addig nem diagnosztizált leukémiát állapítottak meg nála,
  amelynek egyik tipikus szövődménye súlyos tüdőgyulladás.
\item
  A betegség veszélyességének megítélése szempontjából ezért nem a
  pozitív teszteredményt mutató személyeknek és elhunytaknak a médiában
  gyakran említett száma a döntő, hanem a ténylegesen és váratlanul
  tüdőgyulladásban megbetegedettek vagy elhunytak száma (az ún. szokásos
  mennyiséget meghaladó halálozási arány). Ez az érték a legtöbb
  országban eddig \href{https://www.euromomo.eu/index.html}{nagyon
  alacsony}.
\end{itemize}

\hypertarget{2020-muxe1rcius-17-ii}{%
\paragraph{2020. március 17. (II.)}\label{2020-muxe1rcius-17-ii}}

\begin{itemize}
\tightlist
\item
  A Firenzei Egyetem olasz immunológia-professzora, Sergio Romagnani egy
  3000 főt vizsgáló tanulmányban arra az eredményre jut, hogy a pozitív
  teszteredményt mutató személyek 50--75 százaléka az összes
  korcsoportban
  \href{https://www.repubblica.it/salute/medicina-e-ricerca/2020/03/16/news/coronavirus_studio_il_50-75_dei_casi_a_vo_sono_asintomatici_e_molto_contagiosi-251474302/}{teljesen
  tünetmentes marad} -- ez jóval több, mint az eddigi feltételezések.
\item
  Az észak-olaszországi intenzív osztályok kihasználtsága a téli
  hónapokban már eleve
  \href{https://jamanetwork.com/journals/jama/fullarticle/2763188}{85--90\%}.
  Az ott fekvő betegek közül néhányan vagy akár sokan is mutathatnak
  pozitív teszteredményt. Az ezen felül jelentkező, nem várt
  tüdőgyulladásokról egyelőre nincsenek hivatalos adatok.
\item
  A spanyolországi Málaga városi kórházi orvosa
  \href{https://twitter.com/NeurologaenSAS/status/1239498772570308609}{azt
  írja a Twitteren}, hogy jelenleg inkább a pánik és a rendszer
  összeomlása miatt halnak meg az emberek, mint a vírus miatt. A
  kórházat megrohamozzák a meghűléssel, influenzával, valószínűleg
  Covid19-cel küzdő emberek, és a kórházi folyamatok összeomlanak.
\end{itemize}

\hypertarget{2020-muxe1rcius-14}{%
\paragraph{2020. március 14.}\label{2020-muxe1rcius-14}}

Az olasz Nemzeti Egészségügyi Intézet, az ISS
\href{https://www.epicentro.iss.it/coronavirus/sars-cov-2-decessi-italia}{adatai}
szerint az Olaszországban pozitív teszteredményt mutató elhunytak
átlagos életkora kb. 81 év. Az elhunyt személyek 10\% -a 90 évesnél
idősebb. Az elhunyt férfiak 90\% -a 70 évesnél idősebb.

Az elhunytak 80 százaléka kettő vagy több krónikus betegségben
szenvedett. Az elhunytak 50 százaléka három vagy több krónikus
betegségben szenvedett. Ilyen krónikus betegségek a többi között a szív-
és érrendszeri problémák, cukorbetegség, légzőszervi problémák és a rák.

Az elhunytak kevesebb mint 1 százaléka volt egészséges, azaz nem
szenvedett idült betegségben. Az elhunytak mindössze 30 százaléka nő.

Az olasz egészségügyi intézet ezen kívül különbséget tesz a koronavírus
következtében és a koronavírussal a szervezetükben elhunytak között. Sok
esetben még nem világos, hogy az elhunytak a vírusba, meglévő idült
betegségükbe vagy e kettő kombinációjába haltak-e bele.

A két 40 év alatti olasz elhunyt közül (mindketten 39 évesek voltak) az
egyikük rákbeteg volt, a másikuk további szövődményekkel is küzdő
cukorbeteg. A halál pontos oka még az ő esetükben sem világos (vagyis
hogy a vírus vagy a meglévő betegség volt-e az).

A klinikák túlterheltsége a betegek általános rohamából, valamint a
különleges vagy intenzív kezelést igénylő betegek megemelkedett számából
ered. Ezekben az esetekben főként a légzés működésének stabilizálásáról,
súlyos esetekben antivirális terápiákról van szó.

(\textbf{Frissítés}: a Nemzeti Egészségügyi Intézet közzétett egy
\href{https://www.epicentro.iss.it/coronavirus/bollettino/Report-COVID-2019_17_marzo-v2.pdf}{statisztikai
jelentést} a pozitív teszteredményt mutató betegekről és az
elhunytakról, amely jelentés alátámasztja a fenti adatokat.)

\textbf{A következő szempontokat is figyelembe kell venni:}

Észak-Olaszország Európa egyik legidősebb lakosságú és
\href{https://twitter.com/esa/status/1238480433047916545}{legrosszabb
levegőminőségű} térsége, ami már a múltban is fokozott számú légúti
megbetegedést és erre visszavezethető halálozást okozott. Ez további
kockázati tényezőnek tekintendő.

Dél-Koreában például az olaszországinál jóval enyhébben alakult a
járvány, amely már túl van a tetőpontján. Dél-Koreában eddig csupán
mintegy 70 pozitív teszteredményt mutató haláleset történt.
Olaszországhoz hasonlóan főleg a kockázati csoportba tartozó betegekről
van szó.

Az eddig kb. 12 pozitív teszteredményt mutató svájci halálesetben
szintént meglévő betegségben szenvedő, kockázati csoportba tartozó
betegekről van szó, akiknek medián-életkora 80 év, és esetükben a halál
pontos oka -- a vírus, vagy a meglévő betegség -- még nem ismert.

Ezenkívül a tanulmányok
\href{https://www.ncbi.nlm.nih.gov/pmc/articles/PMC2095096/}{kimutatták},
hogy a nemzetközileg alkalmazott vírusvizsgálati készletek bizonyos
esetekben hamis pozitív eredményt adhatnak, vagyis ezek a személyek nem
az új koronavírustól betegedtek meg, hanem valamelyik korábbi
koronavírustól, amely az évente jelentkező (és az ez évi) meghűlési és
influenzahullám részei.

A betegség veszélyességének megítélése szempontjából ezért nem a pozitív
teszteredményt mutató személyek és elhunytak gyakran említett száma a
döntő, hanem a ténylegesen és váratlanul tüdőgyulladásban
megbetegedettek vagy elhunytak száma (az ún. szokásos mennyiséget
meghaladó halálozási arány).

Az iskolás- és keresőkorú egészséges lakosság szempontjából az összes
eddigi ismeret szerint enyhe vagy mérsékelt lefolyású Covid-19-járványra
lehet számítani. Fokozottan védeni kell az időseket és a idült
betegségben szenvedőket. Az orvosi kapacitásokat optimálisan elő kell
készíteni.

\includegraphics{https://swprs.files.wordpress.com/2020/03/italy-smog.png?w=550\&h=309}

\begin{center}\rule{0.5\linewidth}{\linethickness}\end{center}

Share this on:
\href{https://twitter.com/intent/tweet?url=https://swprs.org/egy-svajci-orvos-a-covid-19-rol/}{Twitter}
/
\href{https://www.facebook.com/share.php?u=https://swprs.org/egy-svajci-orvos-a-covid-19-rol/}{Facebook}

\hypertarget{swiss-policy-research}{%
\subsubsection{Swiss Policy Research}\label{swiss-policy-research}}

\begin{itemize}
\tightlist
\item
  \href{https://swprs.org/kontakt/}{Kontakt}
\item
  \href{https://swprs.org/uebersicht/}{Übersicht}
\item
  \href{https://swprs.org/donationen/}{Donationen}
\item
  \href{https://swprs.org/disclaimer/}{Disclaimer}
\end{itemize}

\hypertarget{english}{%
\subsubsection{English}\label{english}}

\begin{itemize}
\tightlist
\item
  \href{https://swprs.org/contact/}{About Us / Contact}
\item
  \href{https://swprs.org/media-navigator/}{The Media Navigator}
\item
  \href{https://swprs.org/the-american-empire-and-its-media/}{The CFR
  and the Media}
\item
  \href{https://swprs.org/donations/}{Donations}
\end{itemize}

\hypertarget{follow-by-email}{%
\subsubsection{Follow by email}\label{follow-by-email}}

Follow

\href{https://wordpress.com/?ref=footer_custom_com}{WordPress.com}.

\protect\hyperlink{}{Up ↑}

Post to

\protect\hyperlink{}{Cancel}

\includegraphics{https://pixel.wp.com/b.gif?v=noscript}
