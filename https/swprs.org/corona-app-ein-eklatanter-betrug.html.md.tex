\protect\hyperlink{content}{Skip to content}

\href{https://swprs.org/}{}

\protect\hyperlink{search-container}{Search}

Search for:

\href{https://swprs.org/}{\includegraphics{https://swprs.files.wordpress.com/2020/05/swiss-policy-research-logo-300.png}}

\href{https://swprs.org/}{Swiss Policy Research}

Geopolitics and Media

Menu

\begin{itemize}
\tightlist
\item
  \href{https://swprs.org}{Start}
\item
  \href{https://swprs.org/srf-propaganda-analyse/}{Studien}

  \begin{itemize}
  \tightlist
  \item
    \href{https://swprs.org/srf-propaganda-analyse/}{SRF / ZDF}
  \item
    \href{https://swprs.org/die-nzz-studie/}{NZZ-Studie}
  \item
    \href{https://swprs.org/der-propaganda-multiplikator/}{Agenturen}
  \item
    \href{https://swprs.org/die-propaganda-matrix/}{Medienmatrix}
  \end{itemize}
\item
  \href{https://swprs.org/medien-navigator/}{Analysen}

  \begin{itemize}
  \tightlist
  \item
    \href{https://swprs.org/medien-navigator/}{Navigator}
  \item
    \href{https://swprs.org/der-propaganda-schluessel/}{Techniken}
  \item
    \href{https://swprs.org/propaganda-in-der-wikipedia/}{Wikipedia}
  \item
    \href{https://swprs.org/logik-imperialer-kriege/}{Kriege}
  \end{itemize}
\item
  \href{https://swprs.org/netzwerk-medien-schweiz/}{Netzwerke}

  \begin{itemize}
  \tightlist
  \item
    \href{https://swprs.org/netzwerk-medien-schweiz/}{Schweiz}
  \item
    \href{https://swprs.org/netzwerk-medien-deutschland/}{Deutschland}
  \item
    \href{https://swprs.org/medien-in-oesterreich/}{Österreich}
  \item
    \href{https://swprs.org/das-american-empire-und-seine-medien/}{USA}
  \end{itemize}
\item
  \href{https://swprs.org/bericht-eines-journalisten/}{Fokus I}

  \begin{itemize}
  \tightlist
  \item
    \href{https://swprs.org/bericht-eines-journalisten/}{Journalistenbericht}
  \item
    \href{https://swprs.org/russische-propaganda/}{Russische Propaganda}
  \item
    \href{https://swprs.org/die-israel-lobby-fakten-und-mythen/}{Die
    »Israel-Lobby«}
  \item
    \href{https://swprs.org/geopolitik-und-paedokriminalitaet/}{Pädokriminalität}
  \end{itemize}
\item
  \href{https://swprs.org/migration-und-medien/}{Fokus II}

  \begin{itemize}
  \tightlist
  \item
    \href{https://swprs.org/covid-19-hinweis-ii/}{Coronavirus}
  \item
    \href{https://swprs.org/die-integrity-initiative/}{Integrity
    Initiative}
  \item
    \href{https://swprs.org/migration-und-medien/}{Migration \& Medien}
  \item
    \href{https://swprs.org/der-fall-magnitsky/}{Magnitsky Act}
  \end{itemize}
\item
  \href{https://swprs.org/kontakt/}{Projekt}

  \begin{itemize}
  \tightlist
  \item
    \href{https://swprs.org/kontakt/}{Kontakt}
  \item
    \href{https://swprs.org/uebersicht/}{Seitenübersicht}
  \item
    \href{https://swprs.org/medienspiegel/}{Medienspiegel}
  \item
    \href{https://swprs.org/donationen/}{Donationen}
  \end{itemize}
\item
  \href{https://swprs.org/contact/}{English}
\end{itemize}

\protect\hyperlink{}{Open Search}

\hypertarget{corona-app-ein-eklatanter-betrug}{%
\section{Corona-App: ``Ein
eklatanter~Betrug''}\label{corona-app-ein-eklatanter-betrug}}

\includegraphics{https://swprs.files.wordpress.com/2020/06/covid-google-apple.jpg?w=450\&h=253}

\emph{``Der Quellcode bleibt bei Microsoft, das Protokoll wird von Apple
und Google implementiert und kontrolliert, der Server wird von Amazon
gehostet. Die aktuelle Informationspolitik leidet unter unklaren oder
falschen Angaben.'' (Aus der Analyse von Professor Vaudenay)}

\textbf{Publiziert}: 29. Juni 2020\\
\textbf{Teilen auf}:
\href{https://twitter.com/intent/tweet?url=https://swprs.org/corona-app-ein-eklatanter-betrug/}{Twitter}
/
\href{https://www.facebook.com/share.php?u=https://swprs.org/corona-app-ein-eklatanter-betrug/}{Facebook}

Während Politik und Medien weiterhin nicht offen kommunizieren,
veröffentlicht SPR Auszüge aus der
\href{https://lasec.epfl.ch/people/vaudenay/swisscovid.html}{vernichtenden
Analyse} von EPFL-Informatik-Professor Serge Vaudenay zur Intransparenz
und den Sicherheitsrisiken der ``dezentralen'' Kontaktverfolgung. Die
Analyse ist von weltweiter Bedeutung, da das Schweizer Protokoll durch
Google und Apple zum globalen Standard wurde.

Übersetzung der Auszüge durch SPR. Der englische Originalbericht
\href{https://lasec.epfl.ch/people/vaudenay/swisscovid.html}{findet sich
hier}.

\begin{center}\rule{0.5\linewidth}{\linethickness}\end{center}

\textbf{Vorbemerkung SPR}: Die WHO kam in einer Studie von 2019 zu
Grippepandemien zum Ergebnis, dass Kontaktverfolgung
\href{https://apps.who.int/iris/bitstream/handle/10665/329438/9789241516839-eng.pdf\#page=9}{``unten
keinen Umständen zu empfehlen''} ist, da epidemiologisch nicht sinnvoll.
NSA-Whistleblower Edward Snowden warnte in diesem Zusammenhang, dass
Corona als Vorwand benutzt wird, um die Massenüberwachung der
Gesellschaft
\href{https://www.youtube.com/watch?v=-pcQFTzck_c}{umfassend
auszubauen}.

Die in der Vaudenay-Analyse erwähnte, nicht-einsehbare Schnittstelle zur
Kontaktverfolgung wurde von Google und Apple inzwischen in
\href{https://www.bloomberg.com/news/articles/2020-04-10/apple-google-bring-covid-19-contact-tracing-to-3-billion-people}{drei
Milliarden Mobiltelefone} integriert. Vaudenay schreibt dazu: ``Wir
können nun sehen, dass das dezentralisierte DP3T-System zu einem
undurchsichtigen System wurde, das bei Google-Apple-Services
zentralisiert ist.''

\textbf{Siehe auch}:
\href{https://swprs.org/covid-19-hinweis-ii/}{Fakten zu Covid-19 →}

\begin{center}\rule{0.5\linewidth}{\linethickness}\end{center}

\hypertarget{analyse-von-swisscovid}{%
\paragraph{Analyse von SwissCovid}\label{analyse-von-swisscovid}}

Prof. Serge Vaudenay, Martin Vuagnoux\\
\href{https://lasec.epfl.ch/people/vaudenay/swisscovid.html}{5. / 28.
Juni 2020}

\hypertarget{1-aus-der-einleitung}{%
\paragraph{1) Aus der Einleitung}\label{1-aus-der-einleitung}}

``Die Website des Nationalen Cyber-Security Centers NCSC listet viele
Sicherheits­­berichte auf, die SwissCovid recht positiv bewerten. Aber
sie listet unseren Bericht nicht auf. Stattdessen enthält sie eine
``detaillierte Analyse'' des NCSC über unseren Bericht. Wir sind mit
dieser Analyse nicht einverstanden. Da es ziemlich klar zu sein scheint,
dass die Kommunikation nicht transparent ist, legen wir hier unsere
Beobachtungen für die Öffentlichkeit dar.''

\hypertarget{2-zusammenfassung}{%
\paragraph{2) Zusammenfassung}\label{2-zusammenfassung}}

``Zusammengefasst lauten unsere Beobachtungen wie folgt:

\begin{enumerate}
\def\labelenumi{\arabic{enumi}.}
\tightlist
\item
  Obwohl der Quellcode der App zur Verfügung steht, können wir sie nicht
  kompilieren, ausführen und zum Laufen bringen, ohne eine Vereinbarung
  mit Apple oder Google zu unterzeichnen. Wir finden die App daher nicht
  kompatibel mit dem Begriff ``Open Source''.
\item
  Ein großer Teil des Kontakt­verfolgungs­protokolls~ wird von
  Apple-Google in einem Teil des Systems namens GAEN implementiert.
  Dieser Teil hat keinen verfügbaren Quellcode, obwohl das Gesetz die
  Offenlegung des Quellcodes aller Komponenten des Systems verlangt.
\item
  Einige Server werden von Amazon als Teil eines externen Dienstes
  gehostet.
\item
  Die verfügbaren Informationen für potenzielle Benutzer sind unklar,
  unvollständig oder falsch.
\item
  Benutzer können bei der Benutzung von SwissCovid durch
  Überwachungssysteme Dritter aufgespürt oder identifiziert werden.
\item
  Diagnostizierte Benutzer, die einen Bericht absenden, haben ein
  Risiko, von einer Drittpartei identifiziert zu werden.
\item
  Dritte könnten auf einem Zieltelefon oder auf einer grossen Gruppe von
  Zieltelefonen falsche Warnungen vor einer möglichen Infizierung
  eingeben. Dies würde dazu führen, dass Menschen in Quarantäne gehen
  müssten, ohne wirklich gefährdet zu sein.
\end{enumerate}

Um das Problem zu umgehen, dass GAEN (die Google-Apple-Schnittstelle)
keinen verfügbaren Quellcode hat, obwohl das Gesetz für alle Komponenten
einen Quellcode vorschreibt, erliess der Bundesrat eine Verordnung, die
alle Komponenten aufzählt, aber in der GAEN \textbf{nicht} enthalten
ist.

Um einen solchen Ausschluss zu rechtfertigen, argumentieren die
Promotoren von SwissCovid, dass GAEN Teil des Betriebssystems des
Telefons oder Teil der Bluetooth-Schnittstelle des Telefons sei und dass
es nicht üblich sei, die Offenlegung des Quellcodes solcher Teile zu
verlangen.

Wir bestreiten, dass GAEN ein solcher Teil des Telefons ist, zumindest
auf Android-Telefonen. GAEN ist Teil der Google Play Services, die
unabhängig vom Betriebssystem und von den Kommunikations­schnittstellen
sind. ()

Darüber hinaus ist der grösste Teil des früheren DP3T-Protokolls (zur
``dezentralen'' Kontakt­verfolgung), das in der ursprünglichen Version
implementiert war, in der aktuellen Version der Anwendung verschwunden,
da ein äquivalentes Protokoll jetzt in GAEN integriert ist.

Wir kommen zu dem Schluss, dass es keine fundierte technische
Rechtfertigung für den Ausschluss von GAEN aus den Komponenten des
Systems gibt. Wir sind der festen Überzeugung, dass die Verordnung ein
juristischer Trick ist, um das Gesetz zu umgehen, was die Folge einer
Meinungs­ver­schiedenheit zwischen SwissCovid und Apple-Google ist.

Wir fordern Verfassungsexperten auf, eine Beurteilung der Gültigkeit der
Verordnung vorzunehmen.''

\hypertarget{3-zur-intransparenten-kontrolle-des-tracings-durch-google-und-apple}{%
\paragraph{3) Zur intransparenten Kontrolle des Tracings durch Google
und
Apple}\label{3-zur-intransparenten-kontrolle-des-tracings-durch-google-und-apple}}

Alle Hervorherbungen durch SPR.

\begin{itemize}
\tightlist
\item
  ``Wir stellen fest, dass SwissCovid weit davon entfernt ist, Open
  Source zu sein. \textbf{Der Quellcode bleibt bei Microsoft, das
  Protokoll wird von Apple und Google implementiert und kontrolliert.
  Der Server wird von Amazon gehostet.} Die aktuelle Informationspolitik
  leidet unter unklaren oder falschen Angaben.''
\item
  ``Fast alles, was sensibel ist, wird von der GAEN-API {[}der
  Schnittstelle von Google und Apple{]} behandelt, \textbf{die keinen
  Quellcode zur Verfügung stellt und die wir niemals kompilieren oder
  analysieren können.}``
\item
  ``\textbf{Deshalb ist SwissCovid weit davon entfernt, Open Source zu
  sein.} Im besten Fall ist der Quellcode der grafischen
  Benutzeroberfläche verfügbar, aber er kann weder die laufende
  Anwendung reproduzieren noch modifiziert werden.''
\item
  ``Es gibt einige seltsame Aspekte in der Beziehung zu Google-Apple.
  Das DP3T-Projekt {[}für ``dezentrales'' Contact-Tracing{]} bittet
  seinen Partner Google-Apple in einer Mitteilung, die Schnittstelle
  wenigstens für externe Audits zu öffnen und ihre Implementierung zu
  aktualisieren. \textbf{Dies lässt uns vermuten, dass DP3T die
  Kontrolle über SwissCovid verloren hat.}``
\item
  ``Die aktuelle Beziehung mit Google-Apple bringt SwissCovid in eine
  seltsame Situation. () \textbf{Die Benutzer müssen zustimmen, ihre
  persönlichen Informationen an Google-Apple weiter­zu­geben, während
  SwissCovid diese nicht verwenden darf}. Ebenso ist es der
  SwissCovid-Applikation verboten, den Standort zu verwenden. Die
  Google-Play-Services verwenden jedoch Zugriff auf Geräte, Fotos,
  Standort, Lesezeichen, Kalender, Speicherplatz, Telefon, Mikrofon,
  Geräte-ID, Kamera, Kontakte, Wi-Fi, Gerätestatus und -verlauf,
  Identität, SMS und viele andere Privilegien. Da iOS geschlossen ist,
  könnten wir nichts sagen, aber es wird wohl dasselbe sein.''
\item
  ``Es gab eine Kontroverse über die Einführung zentralisierter oder
  dezentralisierter Systeme. Wir können nun sehen, \textbf{dass das
  dezentralisierte DP3T-System zu einem undurchsichtigen System wurde,
  das bei Google-Apple-Services zentralisiert ist.}``
\item
  ``Unabhängig von SwissCovid wird die gleiche Bluetooth-Technologie
  bereits von Apple und Google zur Ortung von Bluetooth-Geräten
  verwendet. \textbf{Die Nichtverwendung von GPS bedeutet nicht, dass es
  unmöglich ist, ein Telefon zu orten.}``
\item
  ``Da der größte Teil des Systems von der Google-Apple-Schnittstelle
  implementiert wird, bleibt von DP3T (dem ``dezentralen'' Contact
  Tracing) nicht viel übrig.''
\end{itemize}

\hypertarget{4-zu-den-sicherheitsrisiken}{%
\paragraph{4) Zu den
Sicherheitsrisiken}\label{4-zu-den-sicherheitsrisiken}}

Auflistung der einzelnen Sicherheitsrisiken im Originalbericht.

\begin{itemize}
\tightlist
\item
  ``\textbf{Wir haben gezeigt, dass SwissCovid kritische Bedrohungen der
  Sicherheit und der Privatsphäre schafft.} Egal ob sie reduziert werden
  oder nicht, so sind wir der Meinung, dass sie auf jeden Fall
  kommuniziert werden müssen.''
\item
  ``Noch wichtiger ist, dass die verfügbaren Informationen unzureichend
  sind, dass es \textbf{Falschinformationen über Anonymität und Open
  Source gibt}, dass es keine öffentlichen Sicherheitstests zu geben
  scheint, und dass die Entwickler von SwissCovid \textbf{an
  Entscheidungen von Google-Apple gebunden sind}.''
\item
  ``Uns ist bekannt, \textbf{dass bereits mehrere Angriffe empirisch
  getestet und gemeldet worden sind}. Unser Hauptpunkt ist, \textbf{dass
  freiwillige Nutzer sich dieser Angriffe bewusst sein sollten}. Sie
  mögen für die meisten von ihnen als geringfügig, für einige jedoch als
  von entscheidender Bedeutung angesehen werden. \textbf{Bislang
  schweigt die dem Benutzer zur Verfügung gestellte Dokumentation
  dazu.}``
\end{itemize}

\hypertarget{5-zur-umgehung-des-tracing-gesetzes-durch-den-bundesrat}{%
\paragraph{5) Zur Umgehung des Tracing-Gesetzes durch den
Bundesrat}\label{5-zur-umgehung-des-tracing-gesetzes-durch-den-bundesrat}}

\begin{itemize}
\tightlist
\item
  ``Das Gesetz vom 19. Juni 2020 besagt, dass alle Komponenten des
  SwissCovid-Systems über einen öffentlich zugänglichen Quellcode
  verfügen müssen und überlässt es dem Bundesrat, sich mit den
  Einzelheiten des Einsatzes zu befassen. Die Verordnung des Bundesrates
  vom 24. Juni 2020 definiert die Komponenten so, dass sie
  \emph{ausschliesst}, was von Google-Apple bereitgestellt wird und die
  DP3T-Funktionalitäten implementiert. \textbf{Die Umsetzung von DP3T
  hat somit das Gesetz umgangen.}``
\item
  ``Wir glauben, dass die Verordnung bereits in Vorbereitung war, als
  Ständerat und Nationalrat über die Notwendigkeit eines öffentlich
  zugänglichen Quellcodes diskutierten und \emph{unsere Analyse zensiert
  wurde}. \textbf{Die Bürgerinnen und Bürger und das Parlament sind
  getäuscht worden}. Mag es aus guten Gründen sein (z.B. um die zweite
  Welle zu verhindern), \textbf{es ist ein eklatanter Betrug}. Unserer
  Meinung nach hat sich das Gesetz, das geschaffen wurde, um die
  Menschen davor zu schützen, ein undurchsichtiges System benutzen zu
  müssen, \textbf{5 Tage nach seiner Verabschiedung als unzureichend
  erwiesen.}``
\item
  ``Auf Android ist GAEN Teil der Google Play Services, die die
  Google-spezifischen Dienste regulieren. () \textbf{Dies beweist, dass
  GAEN weder Teil der Kommunikationstreiber noch Teil des
  Betriebssystems ist, im Gegensatz zu der üblichen Ausrede für die
  Nicht-Offenlegung von GAEN, die immer wieder von der Presse verbreitet
  wird.}``
\item
  ``Wir sind überzeugt, dass die rechtliche Definition von
  ``Komponenten'' in einer Verordnung \textbf{ein Trick ist, um das
  Gesetz über die Verfügbarkeit von Quellcode zu umgehen.''}
\item
  \textbf{``Wir fordern unabhängige Rechtsexperten auf, sich zu dieser
  Kontroverse zu äussern,} um zu bestimmen, ob GAEN als Bestandteil von
  SwissCovid betrachtet werden soll und deshalb dem Gesetz unterstehen
  soll, das einen verfügbaren Quellcode verlangt.
\end{itemize}

\hypertarget{zum-vollstuxe4ndigen-vaudenay-bericht-auf-englisch-}{%
\subparagraph{\texorpdfstring{\href{https://lasec.epfl.ch/people/vaudenay/swisscovid.html}{Zum
vollständigen Vaudenay-Bericht auf Englisch
→}}{Zum vollständigen Vaudenay-Bericht auf Englisch →}}\label{zum-vollstuxe4ndigen-vaudenay-bericht-auf-englisch-}}

\begin{center}\rule{0.5\linewidth}{\linethickness}\end{center}

\textbf{Teilen auf}:
\href{https://twitter.com/intent/tweet?url=https://swprs.org/corona-app-ein-eklatanter-betrug/}{Twitter}
/
\href{https://www.facebook.com/share.php?u=https://swprs.org/corona-app-ein-eklatanter-betrug/}{Facebook}

\textbf{Zum Hauptartikel}:
\href{https://swprs.org/covid-19-hinweis-ii/}{Fakten zu Covid-19}

\hypertarget{swiss-policy-research}{%
\subsubsection{Swiss Policy Research}\label{swiss-policy-research}}

\begin{itemize}
\tightlist
\item
  \href{https://swprs.org/kontakt/}{Kontakt}
\item
  \href{https://swprs.org/uebersicht/}{Übersicht}
\item
  \href{https://swprs.org/donationen/}{Donationen}
\item
  \href{https://swprs.org/disclaimer/}{Disclaimer}
\end{itemize}

\hypertarget{english}{%
\subsubsection{English}\label{english}}

\begin{itemize}
\tightlist
\item
  \href{https://swprs.org/contact/}{About Us / Contact}
\item
  \href{https://swprs.org/media-navigator/}{The Media Navigator}
\item
  \href{https://swprs.org/the-american-empire-and-its-media/}{The CFR
  and the Media}
\item
  \href{https://swprs.org/donations/}{Donations}
\end{itemize}

\hypertarget{follow-by-email}{%
\subsubsection{Follow by email}\label{follow-by-email}}

Follow

\href{https://wordpress.com/?ref=footer_custom_com}{WordPress.com}.

\protect\hyperlink{}{Up ↑}

Post to

\protect\hyperlink{}{Cancel}

\includegraphics{https://pixel.wp.com/b.gif?v=noscript}
