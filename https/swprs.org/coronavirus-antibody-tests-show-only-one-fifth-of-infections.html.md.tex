\protect\hyperlink{content}{Skip to content}

\href{https://swprs.org/}{}

\protect\hyperlink{search-container}{Search}

Search for:

\href{https://swprs.org/}{\includegraphics{https://swprs.files.wordpress.com/2020/05/swiss-policy-research-logo-300.png}}

\href{https://swprs.org/}{Swiss Policy Research}

Geopolitics and Media

Menu

\begin{itemize}
\tightlist
\item
  \href{https://swprs.org}{Start}
\item
  \href{https://swprs.org/srf-propaganda-analyse/}{Studien}

  \begin{itemize}
  \tightlist
  \item
    \href{https://swprs.org/srf-propaganda-analyse/}{SRF / ZDF}
  \item
    \href{https://swprs.org/die-nzz-studie/}{NZZ-Studie}
  \item
    \href{https://swprs.org/der-propaganda-multiplikator/}{Agenturen}
  \item
    \href{https://swprs.org/die-propaganda-matrix/}{Medienmatrix}
  \end{itemize}
\item
  \href{https://swprs.org/medien-navigator/}{Analysen}

  \begin{itemize}
  \tightlist
  \item
    \href{https://swprs.org/medien-navigator/}{Navigator}
  \item
    \href{https://swprs.org/der-propaganda-schluessel/}{Techniken}
  \item
    \href{https://swprs.org/propaganda-in-der-wikipedia/}{Wikipedia}
  \item
    \href{https://swprs.org/logik-imperialer-kriege/}{Kriege}
  \end{itemize}
\item
  \href{https://swprs.org/netzwerk-medien-schweiz/}{Netzwerke}

  \begin{itemize}
  \tightlist
  \item
    \href{https://swprs.org/netzwerk-medien-schweiz/}{Schweiz}
  \item
    \href{https://swprs.org/netzwerk-medien-deutschland/}{Deutschland}
  \item
    \href{https://swprs.org/medien-in-oesterreich/}{Österreich}
  \item
    \href{https://swprs.org/das-american-empire-und-seine-medien/}{USA}
  \end{itemize}
\item
  \href{https://swprs.org/bericht-eines-journalisten/}{Fokus I}

  \begin{itemize}
  \tightlist
  \item
    \href{https://swprs.org/bericht-eines-journalisten/}{Journalistenbericht}
  \item
    \href{https://swprs.org/russische-propaganda/}{Russische Propaganda}
  \item
    \href{https://swprs.org/die-israel-lobby-fakten-und-mythen/}{Die
    »Israel-Lobby«}
  \item
    \href{https://swprs.org/geopolitik-und-paedokriminalitaet/}{Pädokriminalität}
  \end{itemize}
\item
  \href{https://swprs.org/migration-und-medien/}{Fokus II}

  \begin{itemize}
  \tightlist
  \item
    \href{https://swprs.org/covid-19-hinweis-ii/}{Coronavirus}
  \item
    \href{https://swprs.org/die-integrity-initiative/}{Integrity
    Initiative}
  \item
    \href{https://swprs.org/migration-und-medien/}{Migration \& Medien}
  \item
    \href{https://swprs.org/der-fall-magnitsky/}{Magnitsky Act}
  \end{itemize}
\item
  \href{https://swprs.org/kontakt/}{Projekt}

  \begin{itemize}
  \tightlist
  \item
    \href{https://swprs.org/kontakt/}{Kontakt}
  \item
    \href{https://swprs.org/uebersicht/}{Seitenübersicht}
  \item
    \href{https://swprs.org/medienspiegel/}{Medienspiegel}
  \item
    \href{https://swprs.org/donationen/}{Donationen}
  \end{itemize}
\item
  \href{https://swprs.org/contact/}{English}
\end{itemize}

\protect\hyperlink{}{Open Search}

\hypertarget{coronavirus-likely-five-times-more-common-and-less-deadly-than-assumed}{%
\section{Coronavirus likely five times more common and less deadly
than~assumed}\label{coronavirus-likely-five-times-more-common-and-less-deadly-than-assumed}}

**A new immunological study shows that many more people may have had
contact with the coronavirus than previously thought -- which means the
virus is likely much less deadly.\\
**

Alexandra Broehm, Sonntags-Zeitung, June 2, 2020
(\href{https://swprs.files.wordpress.com/2020/06/tagesanzeiger-antibody-study-june-2020.pdf}{German
original})\\
\textbf{Share on}:
\href{https://twitter.com/intent/tweet?url=https://swprs.org/coronavirus-antibody-tests-show-only-one-fifth-of-infections/}{Twitter}
/
\href{https://www.facebook.com/share.php?u=https://swprs.org/coronavirus-antibody-tests-show-only-one-fifth-of-infections/}{Facebook};
\textbf{Main article}:
\href{https://swprs.org/a-swiss-doctor-on-covid-19/}{Facts about
Covid-19}

Anyone who gets infected with the corona virus eventually forms
antibodies, and these antibodies can be detected with a blood test. That
was our current state of knowledge. But
\href{https://www.biorxiv.org/content/10.1101/2020.05.21.108308v1}{new
research} is now calling these certainties into question: immunologists
at the University of Zurich have discovered that people with a severe
course of disease have detectable antibodies in their blood, whereas
mild cases hardly ever do. But more than 80 percent of Covid-19 cases
are mild. What does this new finding mean for broad-based antibody tests
in the population?

For the study, which has now been published as a preprint, the team led
by Onur Boyman, professor at the Department of Immunology at the
University Hospital in Zurich, examined two different groups. The first
group consisted of patients with mild or severe disease progression; the
participants in the second group were healthcare professionals that had
been exposed to corona virus. In both groups, the researchers searched
for antibodies not only in the blood, as the usual antibody tests do,
but also in the eyes, nose and mouth.

\begin{quote}
The immunologists were able to prove for the first time for Covid-19
that infected people also have antibodies in the mucous membranes.
\end{quote}

Our immune system defends itself against infection with various weapons.
In doing so, it forms different antibodies (immunoglobulins). What
before the pandemic were common abbreviations only in expert circles
could be read again and again in the discussions about antibody tests in
recent weeks: IgM, IgA or IgG are the names of the immunoglobulins with
which the immune system fights against invaders. They have different
abilities, occur at different stages of the infection and above all at
different places in the body.

\hypertarget{only-a-fraction-falls-seriously-ill}{%
\subparagraph{\texorpdfstring{\textbf{Only a fraction falls seriously
ill}}{Only a fraction falls seriously ill}}\label{only-a-fraction-falls-seriously-ill}}

The IgM are the first line of defence. They are the largest antibodies
and can therefore block more viruses at once, but because of their size
they cannot penetrate the tissue easily. They also disappear again the
fastest. IgA and IgG are smaller and bind more strongly, IgA is mainly
found in the mucous membranes, IgG are the most agile and can reach
everywhere. In the new study, IgA appeared in mild cases about eight
days after the onset of symptoms, and could be detected in the blood in
a small number of cases, but only temporarily.

However, the infected persons with mild cases usually had no IgG at all
in their blood -- actually those antibodies that are detectable for the
longest time. On the other hand, the scientists found IgA in the nasal
mucous membranes of mildly affected patients and, above all, of health
care workers, even when the patients did not show any symptoms. The
immunologists were thus able to prove for the first time for Covid-19
that infected persons also have antibodies in the mucous membranes -- as
is known from other diseases. However, the usual antibody tests look in
the blood. There, the scientists found clearly detectable amounts of IgG
only in the severely ill, which also occurred at an early stage.
\textbf{``The current antibody tests do not cover all cases by a long
way,'' says study director Boyman.} \textbf{Therefore, it can be assumed
that five times more people than are detected in broad-based antibody
tests have already had contact with the new coronavirus.} This is
because less than one fifth of all infected people fall seriously ill
and consequently have clearly detectable antibodies in their blood. If
one takes the example of Geneva, where around five percent of the
population had a positive antibody test in April, the actual figures
there could be 25 percent of the population.

\begin{quote}
It is known from other viral diseases that severe disease progression
also leads to stronger immune responses.
\end{quote}

\textbf{According to Boyman, the number of infections might even exceed
five times the known figures}. This has to do with how our immune system
works. Besides antibodies, there is also the cellular immune response,
the so-called T-lymphocytes. The antibodies fight the virus as soon as
it is in the body and before it can enter cells. Once inside the cell,
it is the T-lymphocytes, specialised defence cells, which eliminate
infected body cells. They also have a memory, remember illnesses that
have already occurred and activate correspondingly specialised cells. If
a person has only these T-lymphocytes as remnants of an infection, no
antibodies are detectable. Nevertheless, at least partial immunity could
exist.

\hypertarget{how-long-does-the-protection-last}{%
\subparagraph{\texorpdfstring{\textbf{How long does the protection
last?}}{How long does the protection last?}}\label{how-long-does-the-protection-last}}

``This is a fascinating study,'' says Francois Spertini, Professor of
Immunology at the University Hospital of Lausanne CHUV. ``It helps us to
explain why so few people have antibodies in their blood and why these
levels are probably misleading.'' ``This is an interesting study,'' says
Daniel Pinschewer, who heads the Department of Experimental Virology at
the University of Basel's Department of Biomedicine. ``I find it
plausible that with today's blood antibody tests we cannot detect all
surviving infections.''

But what do these findings -- and this is the price question -- mean for
a possible protection against further infection with the sars-CoV-2 in
the future? Are only people who are seriously ill protected? Or do they
simply have even stronger protection than mild cases? And how long does
a possible protection actually last?

\begin{quote}
The new corona virus is dangerous because our immune system has never
had to deal with it before.
\end{quote}

There are no conclusive answers to these questions yet, but there are
nevertheless certain indications. It is known from other viral diseases
that severe disease progression also produces stronger immune responses.
However, there is not yet much research on this aspect, whether a
stronger current immune response means longer-term protection. ``But
antibodies in the mucous membranes, not only in the blood, can also
offer protection in principle,'' says Boyman. The younger the
participants in the current study were, the more likely the researchers
were to find antibodies in their mucous membranes.

However, researchers from King's College, London, wrote this week in a
commentary for the ``Physiological Society'': ``Viruses that affect the
respiratory tract are not known to cause people to develop a long
immunity after they have been through illness. Experts are also warning
at the moment that the results of the study should not lead to a neglect
of protective measures. The study was not particularly large, with 165
participants. And there are still too many unanswered questions as to
who might be immune and how, and whether people who only had a mild
disease might easily become infected again in the future. And even a
mild case could in turn infect someone who is then very seriously ill.

\hypertarget{protection-through-frequent-infections}{%
\subparagraph{\texorpdfstring{\textbf{Protection through frequent
infections}}{Protection through frequent infections}}\label{protection-through-frequent-infections}}

Nevertheless, one factor will be important, which was often formulated
as a warning at the beginning of the pandemic: the new corona virus is
dangerous because our immune system has never dealt with it before and
its defence against this new threat has yet to learn. Now, if more
people than previously thought had at least had contact with the virus,
there is hope that a second infection would not be too severe for those
affected.

The new study also provides some clues to another mystery that is
affecting many people during this pandemic. ``Because children often
suffer from upper respiratory tract infections, they may have protective
IgA antibodies in their mucous membranes,'' the authors write. These
antibodies could be caused by a fitter immune defence in the mucous
membranes and therefore offer at least partial protection. ``This could
explain, among other things, why children rarely develop severe
diseases.''

\hypertarget{see-also-studies-on-covid-19-lethality}{%
\subparagraph{\texorpdfstring{\textbf{See also}:
\href{https://swprs.org/studies-on-covid-19-lethality/}{Studies on
Covid-19
lethality}}{See also: Studies on Covid-19 lethality}}\label{see-also-studies-on-covid-19-lethality}}

\begin{center}\rule{0.5\linewidth}{\linethickness}\end{center}

\textbf{Share on}:
\href{https://twitter.com/intent/tweet?url=https://swprs.org/coronavirus-antibody-tests-show-only-one-fifth-of-infections/}{Twitter}
/
\href{https://www.facebook.com/share.php?u=https://swprs.org/coronavirus-antibody-tests-show-only-one-fifth-of-infections/}{Facebook}\\
\textbf{Main article}:
\href{https://swprs.org/a-swiss-doctor-on-covid-19/}{Facts about
Covid-19}

\hypertarget{swiss-policy-research}{%
\subsubsection{Swiss Policy Research}\label{swiss-policy-research}}

\begin{itemize}
\tightlist
\item
  \href{https://swprs.org/kontakt/}{Kontakt}
\item
  \href{https://swprs.org/uebersicht/}{Übersicht}
\item
  \href{https://swprs.org/donationen/}{Donationen}
\item
  \href{https://swprs.org/disclaimer/}{Disclaimer}
\end{itemize}

\hypertarget{english}{%
\subsubsection{English}\label{english}}

\begin{itemize}
\tightlist
\item
  \href{https://swprs.org/contact/}{About Us / Contact}
\item
  \href{https://swprs.org/media-navigator/}{The Media Navigator}
\item
  \href{https://swprs.org/the-american-empire-and-its-media/}{The CFR
  and the Media}
\item
  \href{https://swprs.org/donations/}{Donations}
\end{itemize}

\hypertarget{follow-by-email}{%
\subsubsection{Follow by email}\label{follow-by-email}}

Follow

\href{https://wordpress.com/?ref=footer_custom_com}{WordPress.com}.

\protect\hyperlink{}{Up ↑}

\includegraphics{https://pixel.wp.com/b.gif?v=noscript}
